%!TEX root = ../RfCPN.tex


\modHeadchapter{\IncutBoring の幾何}
ここでは主に、\textbf{\IncutBoring}に関する測定・加工に必要な\expandafterindex{きかてきせいしつ(\yomiIncutBoring)@幾何的性質(\nameIncutBoring)}幾何的性質を考える。



%%%%%%%%%%%%%%%%%%%%%%%%%%%%%%%%%%%%%%%%%%%%%%%%%%%%%%%%%%
%% section 39.01 %%%%%%%%%%%%%%%%%%%%%%%%%%%%%%%%%%%%%%%%%
%%%%%%%%%%%%%%%%%%%%%%%%%%%%%%%%%%%%%%%%%%%%%%%%%%%%%%%%%%
\modHeadsection{\IncutBoringMilling の基準}
\IncutBoring はトップ側のみにある。
\IncutBoring の基準点は、トップ端の内側中心を基準とする。
このとき、\IncutBoringCenter は、トップ端の内側中心とする
%% footnote %%%%%%%%%%%%%%%%%%%%%
\footnote{すなわち、\IncutBoringLength による$X$補正はしない。}。
%%%%%%%%%%%%%%%%%%%%%%%%%%%%%%%%%
