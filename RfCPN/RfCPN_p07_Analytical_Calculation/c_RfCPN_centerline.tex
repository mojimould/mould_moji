%!TEX root = ../RfCPN.tex


\modHeadchapter[loColumn]{\CenterlineEndFaceDif の幾何}
トップ・ボトムの両方に\Outcut がある場合を考える。
通常、それぞれの\OutcutCenter は個別に決められはせず、片方の中心の位置を基準として、もう片方の中心が定められる。
これらの中心の位置の差(\textbf{\CenterlineEndFaceDif}%
%% footnote %%%%%%%%%%%%%%%%%%%%%
\footnote{通常、\CenterlineEndFaceDif(centerline)というのはその名の通り中心線を表すことが多い。
しかし、ここでは\TopOutcutCenter と\BottomOutcutCenter との位置の差を表す用語として「\CenterlineEndFaceDif」と呼んでいる。})
%%%%%%%%%%%%%%%%%%%%%%%%%%%%%%%%%
$T_x$, $T_y$ ($T_x \geq 0$)を機内で測定する際は、C面が工具側に向くようにテーブルを$\pm$90$^\circ$回転($B$軸回転)し、\index{タッチセンサープローブ}タッチセンサープローブを用いてそれぞれの\Outcut 部の$Z$座標および$Y$座標を見ることで測定する。

ここでは、この\expandafterindex{\yomiCenterlineEndFaceDif そくてい@\nameCenterlineEndFaceDif 測定}\nameCenterlineEndFaceDif の測定に必要な位置等について定量的に求める。
なお、\TableCenter Pを原点として考えることにする。
またC面が工具側に向くように$B$軸を(\verb|G91|にて)$\pm$90$^\circ$回転した状態であるとする
%% footnote %%%%%%%%%%%%%%%%%%%%%
\footnote{{\ttfamily G90}(\index{ぜったいざひょう@絶対座標}絶対座標)の場合、\index{テーブル}テーブルを傾けて\AlocationLength を調整した場合は\AlocationAngle$-\theta$を忘れないよう注意。}。
%%%%%%%%%%%%%%%%%%%%%%%%%%%%%%%%%



%%%%%%%%%%%%%%%%%%%%%%%%%%%%%%%%%%%%%%%%%%%%%%%%%%%%%%%%%%
%% section 27.1 %%%%%%%%%%%%%%%%%%%%%%%%%%%%%%%%%%%%%%%%%%
%%%%%%%%%%%%%%%%%%%%%%%%%%%%%%%%%%%%%%%%%%%%%%%%%%%%%%%%%%
\modHeadsection{\BottomOutcutCenter が基準の場合}
通常、\TopOutcutCenter は、\BottomOutcutCenter よりA面側($-Z$側)にある。
このとき、\KeywayPos$\kappa_p$および\BottomOutcutLength$h_\mathrm B$ ($h_\mathrm B > 0$)を用いると、ボトム側($-X$側)およびトップ側($+X$側)の\index{Cがわがいさくめん@C側外削面}C側外削面の中心
%% footnote %%%%%%%%%%%%%%%%%%%%%
\footnote{トップ側には\Keyway があるので、\TopOutcutLength は\KeywayPos$\kappa_p$とみなしている。}
%%%%%%%%%%%%%%%%%%%%%%%%%%%%%%%%%
は、それぞれ
%% footnote %%%%%%%%%%%%%%%%%%%%%
\footnote{通常、\CenterlineEndFaceDifBD は$T_y = 0$である。}
%%%%%%%%%%%%%%%%%%%%%%%%%%%%%%%%%
\begin{align}
  \label{eq:centerlineB}
  \text{ボトム側:}\quad
  \left[
    \begin{array}{c}
      \displaystyle -f_\mathrm B'+\frac{h_\mathrm B}2\\[5pt]
      \mathcal G_{\mathrm By}\\[3pt]
      \displaystyle \mathcal G_{\mathrm Bx}+\frac{\mathfrak W_\mathrm B}2
    \end{array}
    \right]~, \qquad
  \text{トップ側:}\quad
  \left[
    \begin{array}{c}
      \displaystyle f_\mathrm T'-\frac{\kappa_p}2\\[5pt]
      \mathcal G_{\mathrm By}-T_y\\[3pt]
      \displaystyle \mathcal G_{\mathrm Bx}-T_x+\frac{\mathfrak W_\mathrm T}2
    \end{array}
  \right].
\end{align}



%%%%%%%%%%%%%%%%%%%%%%%%%%%%%%%%%%%%%%%%%%%%%%%%%%%%%%%%%%
%% section 27.2 %%%%%%%%%%%%%%%%%%%%%%%%%%%%%%%%%%%%%%%%%%
%%%%%%%%%%%%%%%%%%%%%%%%%%%%%%%%%%%%%%%%%%%%%%%%%%%%%%%%%%
\modHeadsection{\TopOutcutCenter が基準の場合}
通常、\BottomOutcutCenter は、\TopOutcutCenter よりC面側($+Z$側)にある。
このとき、トップ側($+X$側)およびボトム側($-X$側)の\Outcut 部C面の中心は、それぞれ
\begin{align}
  \label{eq:centerlineT}
  \text{トップ側:}~~
  \left[
    \begin{array}{c}
      \displaystyle f_\mathrm T'-\frac{\kappa_p}2\\[5pt]
      \mathcal G_{\mathrm Ty}\\[3pt]
      \displaystyle \mathcal G_{\mathrm Tx}+\frac{\mathfrak W_\mathrm B}2
    \end{array}
    \right]~, \qquad
  \text{ボトム側:}~~
  \left[
    \begin{array}{c}
      \displaystyle -f_\mathrm B'+\frac{h_\mathrm B}2\\[5pt]
      \mathcal G_{\mathrm Ty}+T_y\\[3pt]
      \displaystyle \mathcal G_{\mathrm Tx}+T_x+\frac{\mathfrak W_\mathrm B}2
    \end{array}
  \right].
\end{align}

\clearpage
~\vfill
%%%%%%%%%%%%%%%%%%%%%%%%%%%%%%%%%%%%%%%%%%%%%%%%%%%%%%%%%%
%% Column %%%%%%%%%%%%%%%%%%%%%%%%%%%%%%%%%%%%%%%%%%%%%%%%
%%%%%%%%%%%%%%%%%%%%%%%%%%%%%%%%%%%%%%%%%%%%%%%%%%%%%%%%%%
\begin{\Columnname}{C側面の測定}
通常、\OutcutMilling をせずに\CenterlineEndFaceDifAC の測定を行うことはない。
ただ、\index{NCプログラム}NCプログラムの\index{しうんてん@試運転}試運転などで動きをみるといった可能性はありうるので、\OutcutMilling を行っていない状態で\index{そくてい(\yomiCenterlineEndFaceDif)@測定(\nameCenterlineEndFaceDif)}測定する場合についても述べておく。
なお、ここでは\index{テーブル}テーブルを回転して\AlocationLength の調整を行った場合、かつ\BottomOutcut(\BottomOutcutAsideThickness)が基準の場合を考える。
このとき、測定するC側外面の位置は、\pageeqref{eq:tableTi}, \pageeqref{eq:tableBRi}より、
\begin{align*}
  \text{トップ側:}~~
  & \left[
    \begin{array}{c}
      \displaystyle f_\mathrm T'-\frac{\kappa_p}2\\[5pt]
      G_{\mathrm By}-T_y\\[3pt]
      \displaystyle
      -G_{\mathrm Tx}
      -\sqrt{R_\mathrm c^2-f_\mathrm T^2}
      +\sqrt{R_\mathrm c^2-\left(f_\mathrm T-\frac\kappa2\right)^2}
    \end{array}
    \right],\\
  \text{ボトム側:}~~
  & \left[
    \begin{array}{c}
      \displaystyle -f_\mathrm B'+\frac{h_\mathrm B}2\\[5pt]
      G_{\mathrm By}\\[3pt]
      \displaystyle
      G_{\mathrm Bx}
      -\sqrt{R_\mathrm c^2-f_\mathrm B^2}
      +\sqrt{R_\mathrm c^2-\left(f_\mathrm B-\frac{h_\mathrm B}2\right)^2}
    \end{array}
    \right].
\end{align*}
なお、これらの$Z$座標の差は以下で与えられる。
\begin{align*}
  \sqrt{R_\mathrm i^2-\left(f_\mathrm T-\frac{\kappa_p}2\right)^2}
  -\sqrt{R_\mathrm i^2-\left(f_\mathrm B-\frac{h_\mathrm B}2\right)^2}~.
\end{align*}
\end{\Columnname}
%%%%%%%%%%%%%%%%%%%%%%%%%%%%%%%%%%%%%%%%%%%%%%%%%%%%%%%%%%
%%%%%%%%%%%%%%%%%%%%%%%%%%%%%%%%%%%%%%%%%%%%%%%%%%%%%%%%%%
%%%%%%%%%%%%%%%%%%%%%%%%%%%%%%%%%%%%%%%%%%%%%%%%%%%%%%%%%%
