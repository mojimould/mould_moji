%!TEX root = ../RfCPN.tex


\modHeadchapter{\EndFaceCChamfer の幾何}
ここでは主に、\index{テーパエンドミル}テーパエンドミルを用いた、\textbf{\EndFaceOutCChamfer}および\textbf{\EndFaceInCChamfer}に関する測定・加工に必要な\expandafterindex{きかてきせいしつ(\yomiEndFaceCChamfer)@幾何的性質(\nameEndFaceCChamfer)}幾何的性質を考える。



%%%%%%%%%%%%%%%%%%%%%%%%%%%%%%%%%%%%%%%%%%%%%%%%%%%%%%%%%%
%% section 23.1 %%%%%%%%%%%%%%%%%%%%%%%%%%%%%%%%%%%%%%%%%%
%%%%%%%%%%%%%%%%%%%%%%%%%%%%%%%%%%%%%%%%%%%%%%%%%%%%%%%%%%
\modHeadsection{\EndFaceCChamfer の寸法}
\TopEndFaceOutCChamferLength および\TopEndFaceInCChamferLength をそれぞれ$c_\mathrm{To}$, $c_\mathrm{Ti}$とする。
ここで$c_\mathrm{To}$や$c_\mathrm{Ti}$は、端面に垂直な方向の距離とする。
このとき、片角$\xi_\mathrm e$\,($>0$)の\index{テーパエンドミル}テーパエンドミルに対して、$XY$方向の\EndFaceCChamferLength は、
\begin{align*}
  c_\mathrm{To}\tan\xi_\mathrm e\qquad\text{および}\qquad c_\mathrm{Ti}\tan\xi_\mathrm e
\end{align*}
で与えられる。
\BottomEndFaceOutCChamferLength および\TopEndFaceInCChamferLength$c_\mathrm{Bo}$, $c_\mathrm{Bi}$についても同様である。



%%%%%%%%%%%%%%%%%%%%%%%%%%%%%%%%%%%%%%%%%%%%%%%%%%%%%%%%%%
%% section 23.2 %%%%%%%%%%%%%%%%%%%%%%%%%%%%%%%%%%%%%%%%%%
%%%%%%%%%%%%%%%%%%%%%%%%%%%%%%%%%%%%%%%%%%%%%%%%%%%%%%%%%%
\modHeadsection{テーパエンドミルの参照直径}
\index{テーパエンドミル}テーパエンドミルは、その名の通り\index{テーパ(テーパエンドミル)}テーパの付いた工具であり、先端が平坦になっているものも多い。
しかし、先端部分を\index{こうぐちょう@工具長}工具長として設定すると、その部分が段差となり\index{テーパかこう@テーパ加工}テーパ加工を適切に行うことができない。
そのため、先端部分から一定の距離$d_\mathrm e$だけずらした箇所を\index{こうぐちょう@工具長}工具長として設定し、またその箇所の直径(\index{さんしょうちょっけい(テーパエンドミル)@参照直径(テーパエンドミル)}\textbf{参照直径})$D_\mathrm r$を\index{こうぐちょっけい(テーパエンドミル)@工具直径(テーパエンドミル)}工具直径として補正を行うことが推奨される。

ここで、先端径(直径)
%% footnote %%%%%%%%%%%%%%%%%%%%%
\footnote{先端が平坦でなく尖っている場合は$D_\mathrm e = 0$とする。}
%%%%%%%%%%%%%%%%%%%%%%%%%%%%%%%%%
およびテーパ角(片角)をそれぞれ$D_\mathrm e$, $\xi_\mathrm e$とすると、参照直径$D_\mathrm r$は
\begin{align*}
  D_\mathrm r = D_\mathrm e+2d_\mathrm e\tan\xi_\mathrm e
\end{align*}
で与えられる。
通常、\index{こうぐけいほせい@工具径補正}工具径補正は工具の半径を用いて行うので、工具径を
\begin{align*}
  \frac{D_\mathrm r}2 = \frac{D_\mathrm e}2+d_\mathrm e\tan\xi_\mathrm e
\end{align*}
として設定すればよい。
あるいは、先端径を基準としてそこから補正を行う形にする場合は、その差
\begin{align*}
  \frac{D_\mathrm r}2-\frac{D_\mathrm e}2 = d_\mathrm e\tan\xi_\mathrm e
\end{align*}
だけ補正すればよい。
%%%%%%%%%%%%%%%%%%%%%%%%%%%%%%%%%%%%%%%%%%%%%%%%%%%%%%%%%%
%% hosoku %%%%%%%%%%%%%%%%%%%%%%%%%%%%%%%%%%%%%%%%%%%%%%%%
%%%%%%%%%%%%%%%%%%%%%%%%%%%%%%%%%%%%%%%%%%%%%%%%%%%%%%%%%%
\begin{hosoku}
なお、$\xi_\mathrm e = \nicefrac\pi{12}$\,($15^\circ$), $\nicefrac\pi6$\,($30^\circ$), $\nicefrac\pi4$\,($45^\circ$)のとき、それぞれ
\begin{align*}
  \tan\frac\pi{12} = 2-\sqrt3\ , \quad
  \tan\frac\pi6 = \frac1{\sqrt3}\ , \quad
  \tan\frac\pi4 = 1\ .
\end{align*}
\end{hosoku}
%%%%%%%%%%%%%%%%%%%%%%%%%%%%%%%%%%%%%%%%%%%%%%%%%%%%%%%%%%
%%%%%%%%%%%%%%%%%%%%%%%%%%%%%%%%%%%%%%%%%%%%%%%%%%%%%%%%%%
%%%%%%%%%%%%%%%%%%%%%%%%%%%%%%%%%%%%%%%%%%%%%%%%%%%%%%%%%%



\clearpage
%%%%%%%%%%%%%%%%%%%%%%%%%%%%%%%%%%%%%%%%%%%%%%%%%%%%%%%%%%
%% section 23.2 %%%%%%%%%%%%%%%%%%%%%%%%%%%%%%%%%%%%%%%%%%
%%%%%%%%%%%%%%%%%%%%%%%%%%%%%%%%%%%%%%%%%%%%%%%%%%%%%%%%%%
\modHeadsection{中心座標\texorpdfstring{$X$}{X}の移動}
\Outcut があり、かつ\EndFaceOutCChamfer の場合であれば、Cの大きさに依らず加工径の中心座標($XY$)は変わらない。
一方、外削のない場合は、中心の$X$座標はCの大きさに応じて湾曲中心線に沿って移動する。
このとき\EndFace と面取先端部との中心座標($X$)の差
%% footnote %%%%%%%%%%%%%%%%%%%%%
\footnote{どちらの場合も端面が工具側にある場合を考えている。}
%%%%%%%%%%%%%%%%%%%%%%%%%%%%%%%%%
は、
\begin{align*}
  \text{トップ側:}&~~
  \sqrt{R_\mathrm c^2-\left(f_\mathrm T-c_\mathrm{To}\right)^2}-\sqrt{R_\mathrm c^2-f_\mathrm T^2}\ ,\\
  \text{ボトム側:}&~~
  \sqrt{R_\mathrm c^2-f_\mathrm B^2}-\sqrt{R_\mathrm c^2-\left(f_\mathrm B-c_\mathrm{Bo}\right)^2}\ .
\end{align*}
これは\EndFaceInCChamfer でも同様であり、
\begin{align*}
  \text{トップ側:}&~~
  \sqrt{R_\mathrm c^2-\left(f_\mathrm T-c_\mathrm{Ti}\right)^2}-\sqrt{R_\mathrm c^2-f_\mathrm T^2}\ ,\\
  \text{ボトム側:}&~~
  \sqrt{R_\mathrm c^2-f_\mathrm B^2}-\sqrt{R_\mathrm c^2-\left(f_\mathrm B-c_\mathrm{Bi}\right)^2}\ .
\end{align*}



%\clearpage
%%%%%%%%%%%%%%%%%%%%%%%%%%%%%%%%%%%%%%%%%%%%%%%%%%%%%%%%%%
%% section 25.2 %%%%%%%%%%%%%%%%%%%%%%%%%%%%%%%%%%%%%%%%%%
%%%%%%%%%%%%%%%%%%%%%%%%%%%%%%%%%%%%%%%%%%%%%%%%%%%%%%%%%%
\modHeadsection{面取長の補正\TBW}
\EndFaceChamferMilling では、$Z$方向の\index{めんとりちょう@面取長}面取長が揃う形になるように加工を行う。
このとき、単純に\expandafterindex{ないけいちゅうしん(\yomiEndFace)@内径中心(\nameEndFace)}内径中心を\expandafterindex{きじゅん(\yomiEndFaceChamferMilling)@基準(\nameEndFaceChamferMilling)}基準にすると、外面あるいは内面には\index{わんきょく@湾曲}湾曲があるためA面側およびC面側の面取長は揃わない
%% footnote %%%%%%%%%%%%%%%%%%%%%
\footnote{$XY$方向の面取幅が揃う形になる。}。
%%%%%%%%%%%%%%%%%%%%%%%%%%%%%%%%%
そのため、各面の面取長が揃う形になるように$X$方向に補正を入れる必要がある。













