%!TEX root = ../RfCPN.tex


\modHeadchapter{\EndFaceBoring の幾何}
ここでは主に、\textbf{\EndFaceBoring}に関する測定・加工に必要な\expandafterindex{きかてきせいしつ(\yomiEndFaceBoring)@幾何的性質(\nameEndFaceBoring)}幾何的性質を考える。



%%%%%%%%%%%%%%%%%%%%%%%%%%%%%%%%%%%%%%%%%%%%%%%%%%%%%%%%%%
%% section 06.1 %%%%%%%%%%%%%%%%%%%%%%%%%%%%%%%%%%%%%%%%%%
%%%%%%%%%%%%%%%%%%%%%%%%%%%%%%%%%%%%%%%%%%%%%%%%%%%%%%%%%%
\modHeadsection{\EndFaceBoring の位置}
\EndFaceBoring はトップD面側のみにある。
\EndFaceBoring のA面側からの距離, \EndFaceBoringWidth, \EndFaceBoringDepth, \EndFaceBoringCornerR をそれぞれ$p_\beta$, $w_\beta$, $d_\beta$, $R_\beta$とする。
このとき、\TopEndFace の外側中心T$'_\mathrm c$を原点として、A面側の\EndFaceBoringCornerR の中心の位置を($\beta_x$, $\beta_y$)とすると、
\begin{align*}
  \beta_x &= -\frac{\mathcal W_x}2+p_\beta+\sqrt{d_\beta(2R_\beta-d_\beta)}\ ,\\
  \beta_y &= \frac{\mathcal W_y}2-d_\beta+R_\beta\ .
\end{align*}
ここで、$\mathcal W_x$, $\mathcal W_y$は実測した\TopEndFace の外径を示す。
%%%%%%%%%%%%%%%%%%%%%%%%%%%%%%%%%%%%%%%%%%%%%%%%%%%%%%%%%%
%% hosoku %%%%%%%%%%%%%%%%%%%%%%%%%%%%%%%%%%%%%%%%%%%%%%%%
%%%%%%%%%%%%%%%%%%%%%%%%%%%%%%%%%%%%%%%%%%%%%%%%%%%%%%%%%%
\begin{hosoku}
\EndFaceBoring のA面側の位置から$\beta_x$の位置までの$x$方向の距離は、
\begin{align*}
  R_\beta(1-\cos\theta) = d_\beta\quad
  \longrightarrow\quad
  \cos\theta = 1-\frac{d_\beta}{R_\beta}
\end{align*}
に対して$R_\beta\sin\theta$である。
すなわち、
\begin{align*}
  R_\beta\sin\theta
  = R_\beta\sqrt{\frac{d_\beta}{R_\beta}\left(2-\frac{d_\beta}{R_\beta}\right)}
  = \sqrt{d_\beta(2R_\beta-d_\beta)}\ .
\end{align*}
\end{hosoku}
%%%%%%%%%%%%%%%%%%%%%%%%%%%%%%%%%%%%%%%%%%%%%%%%%%%%%%%%%%
%%%%%%%%%%%%%%%%%%%%%%%%%%%%%%%%%%%%%%%%%%%%%%%%%%%%%%%%%%
%%%%%%%%%%%%%%%%%%%%%%%%%%%%%%%%%%%%%%%%%%%%%%%%%%%%%%%%%%
同様に、C面側の\EndFaceBoringCornerR の中心の位置($\beta_x'$, $\beta_y'$)は、
\begin{align*}
  \beta_x &= -\frac{\mathcal W_x}2+p_\beta+w_\beta-\sqrt{d_\beta(2R_\beta-d_\beta)}\ ,\\
  \beta_y &= \frac{\mathcal W_y}2-d_\beta+R_\beta\ .
\end{align*}
