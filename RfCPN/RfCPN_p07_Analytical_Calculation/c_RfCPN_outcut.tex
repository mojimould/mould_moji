%!TEX root = ../RfCPN.tex


\modHeadchapter{\Outcut の幾何}
ここでは主に、\textbf{\Outcut}に関する測定・加工に必要な\expandafterindex{きかてきせいしつ(\yomiOutcut)@幾何的性質(\nameOutcut)}幾何的性質を考える。

\index{ワーク}ワークに\textbf{\Outcut}があるときは、たいていの場合、\textbf{\OutcutAsideThickness})を\expandafterindex{きじゅん(\yomiOutcutCenter)@基準(\nameOutcutCenter)}基準として考えることが多い。
\expandafterindex{\yomiEndFaceID@\nameEndFaceID}トップ・ボトム端における\textbf{\nameInnerDiameter}をそれぞれ$w_\mathrm T$, $w_\mathrm B$, \textbf{\OutcutWidth}をそれぞれ$\mathfrak W_\mathrm T$, $\mathfrak W_\mathrm B$, \OutcutAsideThickness をそれぞれ$\tau_\mathrm T$, $\tau_\mathrm B$, \EndFaceInCChamfer の垂直方向の大きさをそれぞれ$c_\mathrm{Ti}$, $c_\mathrm{Bi}$とする。
%%%%%%%%%%%%%%%%%%%%%%%%%%%%%%%%%%%%%%%%%%%%%%%%%%%%%%%%%%
%% hosoku %%%%%%%%%%%%%%%%%%%%%%%%%%%%%%%%%%%%%%%%%%%%%%%%
%%%%%%%%%%%%%%%%%%%%%%%%%%%%%%%%%%%%%%%%%%%%%%%%%%%%%%%%%%
\begin{hosoku}
\EndFaceInCChamfer がある場合、\OutcutAsideThickness$\tau_\mathrm T$, $\tau_\mathrm B$は、\EndFace から\EndFaceInCChamferLength$c_\mathrm{Ti}$または$c_\mathrm{Bi}$の位置における\expandafterindex{すんぽう(\yomiThickness)@寸法(\nameThickness)}寸法が与えられる。
\EndFaceInRChamferRadius の場合も同様。
なお、\EndFaceChamfer がない場合(\index{いとめんとり@糸面取})は、その\EndFaceChamferLength を0(すなわち\EndFace 部)とみなす。
\end{hosoku}
%%%%%%%%%%%%%%%%%%%%%%%%%%%%%%%%%%%%%%%%%%%%%%%%%%%%%%%%%%
%%%%%%%%%%%%%%%%%%%%%%%%%%%%%%%%%%%%%%%%%%%%%%%%%%%%%%%%%%
%%%%%%%%%%%%%%%%%%%%%%%%%%%%%%%%%%%%%%%%%%%%%%%%%%%%%%%%%%
また、\index{ワークのないめん@ワークの内面}ワークの内面の\textbf{\PlatingThk}を$\mu$とし、\textbf{\CenterlineEndFaceDif}(\TopOutcutCenter$\mathfrak T_\mathrm c$と\BottomOutcutCenter$\mathfrak B_\mathrm c$の差)の$X$, $Y$成分をそれぞれ$T_x$, $T_y$とする。
ただし、$T_x \geq 0$として、\TopOutcutCenter$\mathfrak T_\mathrm c$は\BottomOutcutCenter$\mathfrak B_\mathrm c$よりA面方向にあるものとする。
%%%%%%%%%%%%%%%%%%%%%%%%%%%%%%%%%%%%%%%%%%%%%%%%%%%%%%%%%%
%% hosoku %%%%%%%%%%%%%%%%%%%%%%%%%%%%%%%%%%%%%%%%%%%%%%%%
%%%%%%%%%%%%%%%%%%%%%%%%%%%%%%%%%%%%%%%%%%%%%%%%%%%%%%%%%%
\begin{hosoku}
\InnerDiameter$w_\mathrm T$, $w_\mathrm B$は、\CurvatureCenter O(またはO$'$)に向かった方向にあることに注意。
\IDCenter がそれぞれの端に位置している。
\end{hosoku}
%%%%%%%%%%%%%%%%%%%%%%%%%%%%%%%%%%%%%%%%%%%%%%%%%%%%%%%%%%
%%%%%%%%%%%%%%%%%%%%%%%%%%%%%%%%%%%%%%%%%%%%%%%%%%%%%%%%%%
%%%%%%%%%%%%%%%%%%%%%%%%%%%%%%%%%%%%%%%%%%%%%%%%%%%%%%%%%%



%%%%%%%%%%%%%%%%%%%%%%%%%%%%%%%%%%%%%%%%%%%%%%%%%%%%%%%%%%
%% section 3.1 %%%%%%%%%%%%%%%%%%%%%%%%%%%%%%%%%%%%%%%%%%%
%%%%%%%%%%%%%%%%%%%%%%%%%%%%%%%%%%%%%%%%%%%%%%%%%%%%%%%%%%
\modHeadsection{\BottomOutcutCenter(ボトム基準)}
\BottomOutcutAsideThickness を基準とする場合、ボトム端から$c_\mathrm{Bi}$の位置における\ODCenter から、\BottomEndACID$w_\mathrm B$の半分を引き、さらに\BottomOutcutAsideThickness$\tau_\mathrm B$と\PlatingThk$\mu$との差を引いたものが(おおよその)\index{Aがわがいさくめん@A側外削面}A側外削面の位置$\mathfrak B_\mathrm o'$に相当する
%% footnote %%%%%%%%%%%%%%%%%%%%%
\footnote{ボトム側が工具側にある場合は、A面は$X$の負方向にあることに注意。}。
%%%%%%%%%%%%%%%%%%%%%%%%%%%%%%%%%


%%%%%%%%%%%%%%%%%%%%%%%%%%%%%%%%%%%%%%%%%%%%%%%%%%%%%%%%%%
%% subsubsection 3.1.1 %%%%%%%%%%%%%%%%%%%%%%%%%%%%%%%%%%%
%%%%%%%%%%%%%%%%%%%%%%%%%%%%%%%%%%%%%%%%%%%%%%%%%%%%%%%%%%
\subsection[\Spacer を用いた場合の\texorpdfstring{$\mathfrak B_\mathrm c'$}{Bc'}]
           {\Spacer を用いた場合の$\boldsymbol{\mathfrak B_\mathrm c'}$}
厚さ$\delta_\mathrm s$の\Spacer を用いた場合、\TableCenter Pを\index{げんてんP@原点P}原点とした
%% footnote %%%%%%%%%%%%%%%%%%%%%
\footnote{\index{マシニングセンタ}マシニングセンタによって\index{きかいげんてん@機械原点}機械原点(の$X$座標)が\TableCenter Pと同じだったり異なったりする場合がある。}\relax
%%%%%%%%%%%%%%%%%%%%%%%%%%%%%%%%%
\BottomOutcutCenter$\mathfrak B_\mathrm c'$の(おおよその)$X$座標は、\pageeqref{eq:spacerBc}より、
\begin{align*}
  \Delta-\frac{\sqrt{R_\mathrm o^2-f_\mathrm B^2}+\sqrt{R_\mathrm i^2-f_\mathrm B^2}}2-\frac{\delta_\mathrm s}2
  +\sqrt{R_\mathrm i'^2-\frac{\delta_\mathrm s^2+(2\bar l)^2}4}\frac{2\bar l}{\sqrt{\delta_\mathrm s^2+(2\bar l)^2}}
  -\frac{w_\mathrm B}2-\tau_\mathrm B+\frac{\mathfrak W_\mathrm B}2\ .
\end{align*}
%%%%%%%%%%%%%%%%%%%%%%%%%%%%%%%%%%%%%%%%%%%%%%%%%%%%%%%%%%
%% hosoku %%%%%%%%%%%%%%%%%%%%%%%%%%%%%%%%%%%%%%%%%%%%%%%%
%%%%%%%%%%%%%%%%%%%%%%%%%%%%%%%%%%%%%%%%%%%%%%%%%%%%%%%%%%
\begin{hosoku}
正確には、\index{ボトムたんのないめんちゅうしん@ボトム端の内面中心}\nameBottomEndFace における(\InnerDiameter ではなく)A・C面側の内面中心b$_\mathrm c'$を見る必要がある。
しかし実際の作業においては、これは\index{タッチセンサー}タッチセンサーの\index{そくていかいしてん@測定開始点}測定開始点として用いるものであるため、おおよその値($\pm10$mm以内程度)で十分である。
そのため、ここでは単純に中心b$_\mathrm c'$の代わりに\BottomODCenter B$_\mathrm c'$とし、また\BottomEndACID$w_\mathrm B$を用いている。
さらにいうと、\BottomODCenter B$_\mathrm c'$は\BottomCurvatureCenter B$_{\mathrm R_\mathrm c}'$で代用してもまず問題はない。
\end{hosoku}
%%%%%%%%%%%%%%%%%%%%%%%%%%%%%%%%%%%%%%%%%%%%%%%%%%%%%%%%%%
%%%%%%%%%%%%%%%%%%%%%%%%%%%%%%%%%%%%%%%%%%%%%%%%%%%%%%%%%%
%%%%%%%%%%%%%%%%%%%%%%%%%%%%%%%%%%%%%%%%%%%%%%%%%%%%%%%%%%
これは飽くまで(\index{ずめん@図面}図面の数字をもとにした)計算値であり、\index{タッチセンサープローブ}タッチセンサープローブでの\index{そくていかいしてん@測定開始点}測定開始点として用いる。
そして、\index{ボトムがわのAがわうちたんてん@ボトム側のA側内面端点}ボトム端におけるA側内面b$_\mathrm o'$の位置を直接測定し、その位置を\expandafterindex{きじゅん(\yomiOutcutCenter)@基準(\nameOutcutCenter)}基準として(\index{ワークざひょうけい@ワーク座標系}ワーク座標系の)\index{げんてん(ワークざひょうけい)@原点(ワーク座標系)}原点$\mathfrak B_\mathrm c'$を定める。
このとき、A側内面b$_\mathrm o'$の$X$座標が以下になるように、原点$\mathfrak B_\mathrm c'$を定める。
\begin{align*}
  -\left(\frac{\mathfrak W_\mathrm B}2-\tau_\mathrm B+\mu\right).
\end{align*}

%%%%%%%%%%%%%%%%%%%%%%%%%%%%%%%%%%%%%%%%%%%%%%%%%%%%%%%%%%
%% subsubsection 3.1.2 %%%%%%%%%%%%%%%%%%%%%%%%%%%%%%%%%%%
%%%%%%%%%%%%%%%%%%%%%%%%%%%%%%%%%%%%%%%%%%%%%%%%%%%%%%%%%%
\subsection[\Table を傾けた場合の\texorpdfstring{$\mathfrak B_\mathrm c'$}{Bc'}]
           {\Table を傾けた場合の$\boldsymbol{\mathfrak B_\mathrm c'}$}
\Table を$-\theta$傾けた場合、\TableCenter Pを\index{げんてんP@原点P}原点とした\BottomOutcutCenter$\mathfrak B_\mathrm c'$の(おおよその)$X$座標は、\pageeqref{eq:tableBc}より、
\begin{align}
  \label{eq:gaisakucenterBt}
  \Delta'\cos\theta-\frac{\sqrt{R_\mathrm o^2-f_\mathrm B^2}+\sqrt{R_\mathrm i^2-f_\mathrm B^2}}2
  -\frac{w_\mathrm B}2-\tau_\mathrm B+\frac{\mathfrak W_\mathrm B}2\ .
\end{align}
このとき、測定した\index{Aがわないめん@A側内面}A側内面b$_\mathrm o'$の$X$座標が以下になるように、\index{げんてん(ワークざひょうけい)@原点(ワーク座標系)}原点$\mathfrak B_\mathrm c'$を定める。
\begin{align}
  \label{eq:gaisakucenterBr}
  -\left(\frac{\mathfrak W_\mathrm B}2-\tau_\mathrm B+\mu\right).
\end{align}


%%%%%%%%%%%%%%%%%%%%%%%%%%%%%%%%%%%%%%%%%%%%%%%%%%%%%%%%%%
%% subsection 3.1.3 %%%%%%%%%%%%%%%%%%%%%%%%%%%%%%%%%%%%%%
%%%%%%%%%%%%%%%%%%%%%%%%%%%%%%%%%%%%%%%%%%%%%%%%%%%%%%%%%%
\subsection{\TopOutcutCenter(ボトム基準)}
\expandafterindex{\yomiTopOutcut@\nameTopOutcut}トップ側にも\nameOutcut がある場合、\BottomOutcutCenter から\CenterlineEndFaceDif を指定する形で\TopOutcutCenter の位置を決めるのが通常である。
このとき、\TableCenter Pを\index{げんてんP@原点P}原点とした\TopOutcutCenter$\mathfrak T_\mathrm c'$の$X$座標は、測定で定めた$\mathfrak B_\mathrm c'$の$X$座標$\mathcal G_{\mathrm Bx}$の符号を反転し
%% footnote %%%%%%%%%%%%%%%%%%%%%
\footnote{トップ側が工具側にある場合は、A面は$X$の正方向にある。
ボトム側と比べて\Table を$B$軸($Y$軸まわり)に$\nicefrac\pi2$回転する必要があるため、$X$座標の符号が反転する形になる。}、
%%%%%%%%%%%%%%%%%%%%%%%%%%%%%%%%%
\CenterlineEndFaceDifAC$T_x$の分を加味すればよい。
したがって、
\begin{align}
  \label{eq:BbasedTx}
  -\mathcal G_{Bx}+T_x
\end{align}
で与えられる
%% footnote %%%%%%%%%%%%%%%%%%%%%
\footnote{$Y$座標については、$B$軸の回転に影響しないので、$\mathcal G_{\mathrm By}+T_y$となる。
なお、実際の作業においては、$T_y = 0$であることが通常である。}。
%%%%%%%%%%%%%%%%%%%%%%%%%%%%%%%%%
ただし実際の作業では、\TableCenter からのずれも考慮する必要がある
%% footnote %%%%%%%%%%%%%%%%%%%%%
\footnote{\TableCenter と\JigCenter は通常一致しているものとして考えるが、実際にはわずかにずれている。
特に$X$方向のずれは、$B$軸回転を伴う場合に効いてくる。}。
%%%%%%%%%%%%%%%%%%%%%%%%%%%%%%%%%



\clearpage
%%%%%%%%%%%%%%%%%%%%%%%%%%%%%%%%%%%%%%%%%%%%%%%%%%%%%%%%%%
%% section 3.2 %%%%%%%%%%%%%%%%%%%%%%%%%%%%%%%%%%%%%%%%%%%
%%%%%%%%%%%%%%%%%%%%%%%%%%%%%%%%%%%%%%%%%%%%%%%%%%%%%%%%%%
\modHeadsection{\TopOutcutCenter}


%%%%%%%%%%%%%%%%%%%%%%%%%%%%%%%%%%%%%%%%%%%%%%%%%%%%%%%%%%
%% subsection 3.2.1 %%%%%%%%%%%%%%%%%%%%%%%%%%%%%%%%%%%%%%
%%%%%%%%%%%%%%%%%%%%%%%%%%%%%%%%%%%%%%%%%%%%%%%%%%%%%%%%%%
\subsection[\Spacer を用いた場合の\texorpdfstring{$\mathfrak T_\mathrm c'$}{Tc'}]
           {\Spacer を用いた場合の$\boldsymbol{\mathfrak T_\mathrm c'}$}
\index{トップがわのAがわがいさくめん@トップ側のA側外削面}トップ側A側外削面が基準となる場合も考慮しておく。
この場合も考えかたはボトム基準のそれと同様である。
\TableCenter Pを\index{げんてんP@原点P}原点とした場合の、\TopOutcutCenter$\mathfrak T_\mathrm c'$のおおよその$X$座標は、\pageeqref{eq:spacerTc}より、
\begin{align*}
  -\Delta+\frac{\sqrt{R_\mathrm o^2-f_\mathrm T^2}+\sqrt{R_\mathrm i^2-f_\mathrm T^2}}2+\frac{\delta_\mathrm s}2
  -\sqrt{R_\mathrm i'^2-\frac{\delta_\mathrm s^2+(2\bar l)^2}4}\frac{2\bar l}{\sqrt{\delta_\mathrm s^2+(2\bar l)^2}}
  +\frac{w_\mathrm T}2+\tau_\mathrm T-\frac{\mathfrak W_\mathrm T}2\ .
\end{align*}
これを\index{タッチセンサープローブ}タッチセンサープローブでの\index{そくていかいしてん@測定開始点}測定開始点とし、計測した\index{げんてん(ワークざひょうけい)@原点(ワーク座標系)}原点の$X$座標(実測値)を$\mathcal G_{Tx}$とすると、\index{トップがわのAがわうちたんてん@トップ側のA側内端点}トップ側のA側内端点と$\mathcal G_{Tx}$との差の$X$座標は、
\begin{align*}
  \frac{\mathfrak W_\mathrm T}2-\tau_\mathrm T+\mu~.
\end{align*}


%%%%%%%%%%%%%%%%%%%%%%%%%%%%%%%%%%%%%%%%%%%%%%%%%%%%%%%%%%
%% subsection 3.2.2 %%%%%%%%%%%%%%%%%%%%%%%%%%%%%%%%%%%%%%
%%%%%%%%%%%%%%%%%%%%%%%%%%%%%%%%%%%%%%%%%%%%%%%%%%%%%%%%%%
\subsection[\Table を傾けた場合の\texorpdfstring{$\mathfrak T_\mathrm c'$}{Tc'}]
           {\Table を傾けた場合の$\boldsymbol{\mathfrak T_\mathrm c'}$}
\Table を$-\theta$傾けた場合、\TableCenter Pを\index{げんてんP@原点P}原点とした\TopOutcutCenter$\mathfrak T_\mathrm c'$の(おおよその)$X$座標は、\pageeqref{eq:tableTc}より、
\begin{align}
  \label{eq:gaisakucenterTt}
  \frac{\sqrt{R_\mathrm o^2-f_\mathrm T^2}+\sqrt{R_\mathrm i^2-f_\mathrm T^2}}2-\Delta'\cos\theta
  +\frac{w_\mathrm T}2+\tau_\mathrm T-\frac{\mathfrak W_\mathrm T}2\ .
\end{align}
測定して定めた\index{げんてん(ワークざひょうけい)@原点(ワーク座標系)}原点$\mathfrak T_\mathrm c'$と、トップ端A側内面t$_\mathrm o'$との差の$X$座標は、
\begin{align}
  \label{eq:gaisakucenterTr}
  \frac{\mathfrak W_\mathrm T}2-\tau_\mathrm T+\mu~.
\end{align}


%%%%%%%%%%%%%%%%%%%%%%%%%%%%%%%%%%%%%%%%%%%%%%%%%%%%%%%%%%
%% subsection 3.2.3 %%%%%%%%%%%%%%%%%%%%%%%%%%%%%%%%%%%%%%
%%%%%%%%%%%%%%%%%%%%%%%%%%%%%%%%%%%%%%%%%%%%%%%%%%%%%%%%%%
\subsection{\BottomOutcutCenter(トップ基準)}
ボトム側にも\Outcut がある場合、\TopOutcutCenter から\CenterlineEndFaceDifAC を指定する形で\BottomOutcutCenter の位置を決める。
このとき、\TableCenter Pを\index{げんてんP@原点P}原点とした\BottomOutcutCenter$\mathfrak B_\mathrm c'$の$X$座標は、測定で定めた$\mathfrak T_\mathrm c'$の$X$座標$\mathcal G_{\mathrm Tx}$の符号を反転し、\CenterlineEndFaceDifAC$T_x$の分を加味すればよい。
したがって、
\begin{align}
  \label{eq:TbasedTx}
  -\mathcal G_{Tx}+T_x
\end{align}
で与えられる。



\clearpage
%%%%%%%%%%%%%%%%%%%%%%%%%%%%%%%%%%%%%%%%%%%%%%%%%%%%%%%%%%
%% section 3.3 %%%%%%%%%%%%%%%%%%%%%%%%%%%%%%%%%%%%%%%%%%%
%%%%%%%%%%%%%%%%%%%%%%%%%%%%%%%%%%%%%%%%%%%%%%%%%%%%%%%%%%
\modHeadsection{\TopOutcutLength}
\TopOutcutLength に関しては、基本的には\AlocationLength$f_\mathrm T$から\TopOutcutLength$h_\mathrm T$を引いた位置に$Z$座標を合わせればよい。
すなわち、\TableCenter Pを\index{げんてんP@原点P}原点として$Z$座標を
\begin{align*}
  f_\mathrm T - h_\mathrm T
\end{align*}
とすればよい。
ただし、\TopOutcutLength$h_\mathrm T$が\KeywayPos$\kappa_p$と\KeywayWidth$\kappa_w$の和に一致する場合は、\TopOutcutLength を$\kappa_p+1$mmとして切削する
%% footnote %%%%%%%%%%%%%%%%%%%%%
\footnote{$f_\mathrm T-(\kappa_p+\kappa_w) < Z < f_\mathrm T-\kappa_p$であれば問題ない。
1mmとしているのは慣例によるものである。}。
%%%%%%%%%%%%%%%%%%%%%%%%%%%%%%%%%
すなわち、
\begin{align*}
  h_\mathrm T = \kappa_p+\kappa_w \quad \longrightarrow \quad f_\mathrm T-\kappa_p-1[\mathrm{mm}]
\end{align*}



\clearpage
%%%%%%%%%%%%%%%%%%%%%%%%%%%%%%%%%%%%%%%%%%%%%%%%%%%%%%%%%%
%% section 35.04 %%%%%%%%%%%%%%%%%%%%%%%%%%%%%%%%%%%%%%%%%
%%%%%%%%%%%%%%%%%%%%%%%%%%%%%%%%%%%%%%%%%%%%%%%%%%%%%%%%%%
\modHeadsection{\CurvedOutcut の幾何}
ここでは\Outcut が\CurvedOutcut である場合を考える。


%%%%%%%%%%%%%%%%%%%%%%%%%%%%%%%%%%%%%%%%%%%%%%%%%%%%%%%%%%
%% subsection 35.04.01 %%%%%%%%%%%%%%%%%%%%%%%%%%%%%%%%%%%
%%%%%%%%%%%%%%%%%%%%%%%%%%%%%%%%%%%%%%%%%%%%%%%%%%%%%%%%%%
\subsection{\BottomCurvedOutcut の幾何}
ここでは\TableCenter Pを\index{げんてんP@原点P}原点とし、\BottomEndFace が工具側に向いているとする。

%%%%%%%%%%%%%%%%%%%%%%%%%%%%%%%%%%%%%%%%%%%%%%%%%%%%%%%%%%
%% subsubsection 35.04.01.1 %%%%%%%%%%%%%%%%%%%%%%%%%%%%%%
%%%%%%%%%%%%%%%%%%%%%%%%%%%%%%%%%%%%%%%%%%%%%%%%%%%%%%%%%%
\subsubsection{\BottomCurvedOutcut の傾き角}
\BottomEndFace におけるA面側・C面側・中心の$X$座標をそれぞれ$x_\mathrm{Bo}$, $x_\mathrm{Bi}$, $x_\mathrm{Bm}$とする。
このとき\BottomEndFace のそれぞれの位置
%% footnote %%%%%%%%%%%%%%%%%%%%%
\footnote{$Z$の正方向を実軸、$X$の正方向を虚軸として考えている。}は、
%%%%%%%%%%%%%%%%%%%%%%%%%%%%%%%%%
\begin{alignat*}{2}
  \text{中心:}&\quad
  \sqrt{f_\mathrm B^{'2}+x_\mathrm{Bm}^2}e^{i\theta_\mathrm{Bm}}~, \quad &
  \tan\theta_\mathrm{Bm} &= \frac{x_\mathrm{Bm}}{f_\mathrm B'}\ ,\\
  \text{C面側端点:}&\quad
  \sqrt{f_\mathrm B^{'2}+x_\mathrm{Bi}^2}e^{i\theta_\mathrm{Bi}}~, \quad &
  \tan\theta_\mathrm{Bi} &= \frac{x_\mathrm{Bi}}{f_\mathrm B'}\ ,\\
  \text{A面側端点:}&\quad
  \sqrt{f_\mathrm B^{'2}+x_\mathrm{Bo}^2}e^{i\theta_\mathrm{Bo}}~, \quad &
  \tan\theta_\mathrm{Bo} &= \frac{x_\mathrm{Bo}}{f_\mathrm B'}\ .
\end{alignat*}
\BottomOutcutLength を$h_\mathrm B$とすると、\expandafterindex{かたむき(\yomiCurvedOutcut)@傾き(\nameCurvedOutcut)}\nameCurvedOutcut 部分の傾きはボトム端から距離$\nicefrac{h_\mathrm B}2$の断面に平行になる形にとる。
そのときの\expandafterindex{かたむきかく(\yomiCurvedOutcut)@傾き角(\nameCurvedOutcut)}傾き角$\theta_b$は、
\begin{align}
  \label{eq:BCurvedOutcutAngle}
  \sin\theta_b = \frac{f_\mathrm B'-\nicefrac{h_\mathrm B}2}{R_\mathrm c}\ .
\end{align}

%%%%%%%%%%%%%%%%%%%%%%%%%%%%%%%%%%%%%%%%%%%%%%%%%%%%%%%%%%
%% subsubsection 35.04.01.2 %%%%%%%%%%%%%%%%%%%%%%%%%%%%%%
%%%%%%%%%%%%%%%%%%%%%%%%%%%%%%%%%%%%%%%%%%%%%%%%%%%%%%%%%%
\subsubsection{\BottomCurvedOutcut の端点の位置}
傾き後の\expandafterindex{\yomiBottomEndFace(\yomiCurvedOutcut)@\nameBottomEndFace(\nameCurvedOutcut)}\nameBottomEndFace の中心の位置は、
\begin{align*}
  \sqrt{f_\mathrm B^{'2}+x_\mathrm{Bm}^2}e^{i(\theta_\mathrm{Bm}-\theta_b)}
\end{align*}
となるので、これの虚部が$X$座標、実部が$Z$座標となる
%% footnote %%%%%%%%%%%%%%%%%%%%%
\footnote{ここでは\index{ふくそへいめん@複素平面}複素平面を考えているが、通常の\index{ちょっこうざひょうけい@直交座標系}直行座標系で単純に\index{かいてんぎょうれつ@回転行列}回転行列をかけているのと同義である。
\begin{align*}
  \left[
    \begin{array}{c}
      x_\mathrm{Bm}\\
      f_\mathrm B'
    \end{array}
  \right]
  ~~\longrightarrow~~
  \left[
    \begin{array}{cc}
      \cos\theta_b & -\sin\theta_b\\
      \sin\theta_b & \cos\theta_b
    \end{array}
  \right]\!\!
  \left[
    \begin{array}{c}
      x_\mathrm{Bm}\\
      f_\mathrm B'
    \end{array}
  \right]
  = \left[
    \begin{array}{c}
      x_\mathrm{Bm}\cos\theta_b-f_\mathrm B'\sin\theta_b\\
      x_\mathrm{Bm}\sin\theta_b+f_\mathrm B'\cos\theta_b
    \end{array}
  \right].
\end{align*}%
}。
%%%%%%%%%%%%%%%%%%%%%%%%%%%%%%%%%
\begin{align*}
  \sqrt{f_\mathrm B^{'2}+x_\mathrm{Bm}^2}\sin(\theta_\mathrm{Bm}-\theta_b)
  &= \sqrt{f_\mathrm B^{'2}+x_\mathrm{Bm}^2}(\sin\theta_\mathrm{Bm}\cos\theta_b-\cos\theta_\mathrm{Bm}\sin\theta_b)
   = x_\mathrm{Bm}\cos\theta_b-f_\mathrm B'\sin\theta_b~,\\
  \sqrt{f_\mathrm B^{'2}+x_\mathrm{Bm}^2}\cos(\theta_\mathrm{Bm}-\theta_b)
  &= \sqrt{f_\mathrm B^{'2}+x_\mathrm{Bm}^2}(\cos\theta_\mathrm{Bm}\cos\theta_b+\sin\theta_\mathrm{Bm}\sin\theta_b)
   = f_\mathrm B'\cos\theta_b+x_\mathrm{Bm}\sin\theta_b~.
\end{align*}
\BottomEndFace のA面側・C面側の位置についても同様である。
まとめると、
\begin{subequations}
\begin{align}
  \notag
  \text{中点:}&\quad
    \left[
      \begin{array}{c}
        x'_{Bm}\\
        z'_{Bm}
      \end{array}
    \right]
    = \left[
      \begin{array}{c}
        x_\mathrm{Bm}\cos\theta_b-f_\mathrm B'\sin\theta_b\\[6pt]
        f_\mathrm B'\cos\theta_b+x_\mathrm{Bm}\sin\theta_b
      \end{array}
    \right],\\
  \label{eq:bottomcurvedoutcutCsideendpoint}
  \text{C面側端点:}&\quad
    \left[
      \begin{array}{c}
        x'_{BC}\\
        z'_{BC}
      \end{array}
    \right]
    = \left[
      \begin{array}{c}
        x_\mathrm{Bi}\cos\theta_b-f_\mathrm B'\sin\theta_b\\[3pt]
        f_\mathrm B'\cos\theta_b+x_\mathrm{Bi}\sin\theta_b
      \end{array}
    \right],\\
  \label{eq:bottomcurvedoutcutAsideendpoint}
  \text{A面側端点:}&\quad
    \left[
      \begin{array}{c}
        x'_{BA}\\
        z'_{BA}
      \end{array}
    \right]
    = \left[
      \begin{array}{c}
        x_\mathrm{Bo}\cos\theta_b-f_\mathrm B'\sin\theta_b\\[3pt]
        f_\mathrm B'\cos\theta_b+x_\mathrm{Bo}\sin\theta_b
      \end{array}
    \right].
\end{align}
\end{subequations}

%%%%%%%%%%%%%%%%%%%%%%%%%%%%%%%%%%%%%%%%%%%%%%%%%%%%%%%%%%
%% subsubsection 35.04.01.3 %%%%%%%%%%%%%%%%%%%%%%%%%%%%%%
%%%%%%%%%%%%%%%%%%%%%%%%%%%%%%%%%%%%%%%%%%%%%%%%%%%%%%%%%%
\subsubsection{\BottomCurvedOutcut のコーナー端点の位置}
\BottomCurvedOutcut のコーナーがCまたはRの形状をしているものとする。
このとき、\EndFaceChamferLength を$R_\mathrm E$とすると、特にB側およびD側のコーナーの端点の位置は、
\begin{alignat*}{2}
  \text{BD\,$+X$側:}&\quad
  \sqrt{f_\mathrm B^{'2}+(x_\mathrm{Bi}-R_\mathrm E)^2}e^{i\theta_\mathrm{BB+}}~, \quad &
  \tan\theta_\mathrm{BB+} &= \frac{x_\mathrm{Bi}-R_\mathrm E}{f_\mathrm B'}\ ,\\
  \text{BD\,$-X$側:}&\quad
  \sqrt{f_\mathrm B^{'2}+(x_\mathrm{Bo}+R_\mathrm E)^2}e^{i\theta_\mathrm{BB-}}~, \quad &
  \tan\theta_\mathrm{BB-} &= \frac{x_\mathrm{Bo}+R_\mathrm E}{f_\mathrm B'}\ .
\end{alignat*}
したがって、傾き後の$XZ$座標は、
\begin{align*}
  \text{BD\,$+X$側:}&\quad
    \left[
      \begin{array}{c}
        x'_{BB+}\\
        z'_{BB+}
      \end{array}
    \right]
    = \left[
      \begin{array}{c}
        (x_\mathrm{Bi}-R_\mathrm E)\cos\theta_b-f_\mathrm B'\sin\theta_b\\[6pt]
        f_\mathrm B'\cos\theta_b+(x_\mathrm{Bi}-R_\mathrm E)\sin\theta_b
      \end{array}
    \right],\\
  \text{BD\,$-X$側:}&\quad
    \left[
      \begin{array}{c}
        x'_{BB-}\\
        z'_{BB-}
      \end{array}
    \right]
    = \left[
      \begin{array}{c}
        (x_\mathrm{Bo}+R_\mathrm E)\cos\theta_b-f_\mathrm B'\sin\theta_b\\[6pt]
        f_\mathrm B'\cos\theta_b+(x_\mathrm{Bo}+R_\mathrm E)\sin\theta_b
      \end{array}
    \right].
\end{align*}


\clearpage
%%%%%%%%%%%%%%%%%%%%%%%%%%%%%%%%%%%%%%%%%%%%%%%%%%%%%%%%%%
%% subsection 35.04.2 %%%%%%%%%%%%%%%%%%%%%%%%%%%%%%%%%%%%
%%%%%%%%%%%%%%%%%%%%%%%%%%%%%%%%%%%%%%%%%%%%%%%%%%%%%%%%%%
\subsection{\TopCurvedOutcut の幾何}
同様に、\TableCenter Pを\index{げんてんP@原点P}原点とし、\TopEndFace が工具側に向いているとする。

%%%%%%%%%%%%%%%%%%%%%%%%%%%%%%%%%%%%%%%%%%%%%%%%%%%%%%%%%%
%% subsubsection 35.04.02.1 %%%%%%%%%%%%%%%%%%%%%%%%%%%%%%
%%%%%%%%%%%%%%%%%%%%%%%%%%%%%%%%%%%%%%%%%%%%%%%%%%%%%%%%%%
\subsubsection{\TopCurvedOutcut の傾き角}
\TopEndFace におけるA面側・C面側・中心の$X$座標をそれぞれ$x_\mathrm{To}$, $x_\mathrm{Ti}$, $x_\mathrm{Tm}$とする。
このとき\TopEndFace のそれぞれの位置は
\begin{alignat*}{2}
  \text{中心:}&\quad
  \sqrt{f_\mathrm T^{'2}+x_\mathrm{Tm}^2}e^{i\theta_\mathrm{Tm}}~, \quad &
  \tan\theta_\mathrm{Tm} &= \frac{x_\mathrm{Tm}}{f_\mathrm T'}\ ,\\
  \text{C面側端点:}&\quad
  \sqrt{f_\mathrm T^{'2}+x_\mathrm{Ti}^2}e^{i\theta_\mathrm{Ti}}~, \quad &
  \tan\theta_\mathrm{Ti} &= \frac{x_\mathrm{Ti}}{f_\mathrm T'}\ ,\\
  \text{A面側端点:}&\quad
  \sqrt{f_\mathrm T^{'2}+x_\mathrm{To}^2}e^{i\theta_\mathrm{To}}~, \quad &
  \tan\theta_\mathrm{To} &= \frac{x_\mathrm{To}}{f_\mathrm T'}\ .
\end{alignat*}
\TopOutcutLength を$h_\mathrm T$とすると、\expandafterindex{かたむき(\yomiCurvedOutcut)@傾き(\nameCurvedOutcut)}\nameCurvedOutcut 部分の傾きはトップ端から距離$\nicefrac{h_\mathrm T}2$の断面に平行になる形にとる。
そのときの\expandafterindex{かたむきかく(\yomiCurvedOutcut)@傾き角(\nameCurvedOutcut)}傾き角$\theta_t$は、
\begin{align}
  \label{eq:TCurvedOutcutAngle}
  \sin\theta_t = \frac{f_\mathrm T'-\nicefrac{h_\mathrm T}2}{R_\mathrm c}\ .
\end{align}

%%%%%%%%%%%%%%%%%%%%%%%%%%%%%%%%%%%%%%%%%%%%%%%%%%%%%%%%%%
%% subsubsection 35.04.01.2 %%%%%%%%%%%%%%%%%%%%%%%%%%%%%%
%%%%%%%%%%%%%%%%%%%%%%%%%%%%%%%%%%%%%%%%%%%%%%%%%%%%%%%%%%
\subsubsection{\TopCurvedOutcut の端点の位置}
傾き後の\expandafterindex{\yomiTopEndFace(\yomiCurvedOutcut)@\nameTopEndFace(\nameCurvedOutcut)}\nameTopEndFace の中心の位置は、
\begin{align*}
  \sqrt{f_\mathrm T^{'2}+x_\mathrm{Tm}^2}e^{i(\theta_\mathrm{Tm}+\theta_t)}
\end{align*}
となるので、これの虚部が$X$座標、実部が$Z$座標となる。
\begin{align*}
  \sqrt{f_\mathrm T^{'2}+x_\mathrm{Tm}^2}\sin(\theta_\mathrm{Tm}+\theta_t)
  &= x_\mathrm{Tm}\cos\theta_t+f_\mathrm T'\sin\theta_t~,\\
  \sqrt{f_\mathrm T^{'2}+x_\mathrm{Tm}^2}\cos(\theta_\mathrm{Tm}-\theta_t)
  &= f_\mathrm T'\cos\theta_T-x_\mathrm{Tm}\sin\theta_t~.
\end{align*}
\TopEndFace のA面側・C面側の位置についても同様である。
まとめると、
\begin{subequations}
\begin{align}
  \notag
  \text{中点:}&\quad
    \left[
      \begin{array}{c}
        x'_m\\
        z'_t
      \end{array}
    \right]
    = \left[
      \begin{array}{c}
        x_\mathrm{Tm}\cos\theta_t+f_\mathrm T'\sin\theta_t\\[3pt]
        f_\mathrm T'\cos\theta_t-x_\mathrm{Tm}\sin\theta_t
      \end{array}
    \right],\\
  \label{eq:topcurvedoutcutCsideendpoint}
  \text{C面側端点:}&\quad
    \left[
      \begin{array}{c}
        x'_C\\
        z'_t
      \end{array}
    \right]
    = \left[
      \begin{array}{c}
        x_\mathrm{Ti}\cos\theta_t+f_\mathrm T'\sin\theta_t\\[3pt]
        f_\mathrm T'\cos\theta_t-x_\mathrm{Ti}\sin\theta_t
      \end{array}
    \right],\\
  \label{eq:topcurvedoutcutAsideendpoint}
  \text{A面側端点:}&\quad
    \left[
      \begin{array}{c}
        x'_A\\
        z'_t
      \end{array}
    \right]
    = \left[
      \begin{array}{c}
        x_\mathrm{To}\cos\theta_t+f_\mathrm T'\sin\theta_t\\[3pt]
        f_\mathrm T'\cos\theta_t-x_\mathrm{To}\sin\theta_t
      \end{array}
    \right].
\end{align}
\end{subequations}

\clearpage
%%%%%%%%%%%%%%%%%%%%%%%%%%%%%%%%%%%%%%%%%%%%%%%%%%%%%%%%%%
%% subsubsection 35.04.01.4 %%%%%%%%%%%%%%%%%%%%%%%%%%%%%%
%%%%%%%%%%%%%%%%%%%%%%%%%%%%%%%%%%%%%%%%%%%%%%%%%%%%%%%%%%
\subsubsection{\TopCurvedOutcut のコーナー端点の位置\TBW}
(to be written...)
