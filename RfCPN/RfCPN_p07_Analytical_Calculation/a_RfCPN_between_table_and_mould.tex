%!TEX root = ../RfCPN.tex


\modHeadchapter{\index{ワーク}ワークと\index{テーブル}テーブルとの位置}
\index{CAD}CADによる描画において、\TableCenter が原点(\index{ワールドげんてん@ワールド原点}ワールド原点)に置かれているとする。
ここで\index{ワーク}ワークを描画する際、\index{モールドちゅうしん@モールド中心}モールドの中心
%% footnote %%%%%%%%%%%%%%%%%%%%%
\footnote{$R_\mathrm c$に相当する点。}\relax
%%%%%%%%%%%%%%%%%%%%%%%%%%%%%%%%%
を\index{CAD}CAD上の\index{ワールドげんてん@ワールド原点}ワールド原点にして描くほうが都合のいいことがある。
このとき、\index{ワーク}ワークと\index{うけいた@受板}受板が接するように移動する必要がある。
鉛直方向(トップ-ボトム方向)においては$f_d$だけ動かせばよいが、水平方向の移動距離はあまり自明とはいいがたい。
\index{Cがわがいめん@C側外面}C側外面と\index{うけいためん@受板面}受板面との寸法を単純に測ると、(水平方向でなく)最短距離が測定されてしまう。
工夫により水平方向の距離を出すことも可能ではあるが、ここではその距離を定量的に求めておく。



%%%%%%%%%%%%%%%%%%%%%%%%%%%%%%%%%%%%%%%%%%%%%%%%%%%%%%%%%%
%% section D.1 %%%%%%%%%%%%%%%%%%%%%%%%%%%%%%%%%%%%%%%%%%%
%%%%%%%%%%%%%%%%%%%%%%%%%%%%%%%%%%%%%%%%%%%%%%%%%%%%%%%%%%
\modHeadsection{\index{スペーサ}スペーサ取付前}
(\index{スペーサ}スペーサを取付る前の)\index{ワーク}ワークの中心が\TableCenter Pに置かれている場合を考える。
ボトム側の受板に接する\index{ワーク}ワークの点と、\TableCenter Pとは、実軸方向に
\begin{align*}
  R_\mathrm c-R_\mathrm i\cos\alpha_{\mathrm U_\mathrm B}
\end{align*}
だけ差がある。
したがって、\index{ワーク}ワークと受板が接する点の位置は実軸方向に、
\begin{align*}
  \Delta+\sqrt{R_\mathrm i'^2-\bar l^2}-R_\mathrm c+R_\mathrm i\cos\alpha_{\mathrm U_\mathrm B}\ .
\end{align*}
そのため\pageeqref{eq:afterUBcontact} ($\delta_s = 0$)より、\index{ワーク}ワークと(ボトム側の)受板は
\begin{align*}
  \Delta+\sqrt{R_\mathrm i'^2-\bar l^2}-R_\mathrm c
\end{align*}
だけ実軸方向に離れていることがわかる。
\pageeqref{eq:tableCenter}より、これは\TableCenter Pと\index{ワーク}ワークの\CenterCurvature$R_\mathrm c$との差であることがわかる。



\clearpage
%%%%%%%%%%%%%%%%%%%%%%%%%%%%%%%%%%%%%%%%%%%%%%%%%%%%%%%%%%
%% section D.2 %%%%%%%%%%%%%%%%%%%%%%%%%%%%%%%%%%%%%%%%%%%
%%%%%%%%%%%%%%%%%%%%%%%%%%%%%%%%%%%%%%%%%%%%%%%%%%%%%%%%%%
\modHeadsection{\index{スペーサ}スペーサ取付後}
\index{\yomiSpacerThickness@\nameSpacerThickness}厚さ$\delta_s\,(>0)$のスペーサを取付けた場合、\index{ワーク}ワークの受板と接する点と\TableCenter Pとは、実軸方向に
\begin{align*}
  R_\mathrm c-R_\mathrm i\cos\alpha'_{\mathrm U_\mathrm B}
\end{align*}
だけ差があるので、その実軸方向の位置は、
\begin{align*}
  \Delta+\sqrt{R_\mathrm i'^2-\bar l^2}-R_\mathrm c+R_\mathrm i\cos\alpha'_{\mathrm U_\mathrm B}\ .
\end{align*}
そのため\pageeqref{eq:afterUBcontact}より、\index{ワーク}ワークと(ボトム側の)受板は
\begin{align*}
  &  \Delta+\sqrt{R_\mathrm i'^2-\bar l^2}-R_\mathrm c+R_\mathrm i\cos\alpha'_{\mathrm U_\mathrm B}
     -\left(R_\mathrm i'\cos\alpha_{\mathrm U_\mathrm B}+\rho\cos\alpha'_{\mathrm U_\mathrm B}\right)\\
  &= \Delta-R_\mathrm c+R_\mathrm i'\cos\alpha'_{\mathrm U_\mathrm B}\\
  &= \Delta-R_\mathrm c
     -\frac{\delta_s}2+\sqrt{R_\mathrm i'^2-\frac{\delta_s^2+(2\bar l)^2}4}\frac{2\bar l}{\sqrt{\delta_s^2+(2\bar l)^2}}
\end{align*}
だけ実軸方向に離れていることがわかる。
\pageeqref{eq:tableCenter}および\pageeqref{eq:spacerMoveHdistance}より、これは\index{スペーサ}スペーサ取付け後の\index{モールドちゅうしん@モールド中心}モールド中心と\CenterCurvature$R_\mathrm c$との差であることがわかる。
