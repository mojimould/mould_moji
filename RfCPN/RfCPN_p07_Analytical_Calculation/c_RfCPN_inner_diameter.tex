%!TEX root = ../RfCPN.tex


\modHeadchapter[lot]{内径の幾何}



%%%%%%%%%%%%%%%%%%%%%%%%%%%%%%%%%%%%%%%%%%%%%%%%%%%%%%%%%%
%% section 07.1 %%%%%%%%%%%%%%%%%%%%%%%%%%%%%%%%%%%%%%%%%%
%%%%%%%%%%%%%%%%%%%%%%%%%%%%%%%%%%%%%%%%%%%%%%%%%%%%%%%%%%
\modHeadsection{\eTaper の算定\TBW}
使用する鋼材の種類として、\index{C(たんそ)@C(炭素)}C, \index{Si(ケイそ)@Si(ケイ素)}Si, \index{Mn(マンガン)@Mn(マンガン)}Mn, \index{P(リン)@P(リン)}P, \index{S(いおう)@S(硫黄)}Sが含まれている場合を考える。
\index{JISきかく(こうしゅ)@JIS規格(鋼種)}JIS規格に基づいた鋼種を用いるものとすれば、その規格によってそれぞれの\expandafterindex{かがくそせい(\yomieTaper)@化学組成(\nameeTaper)}化学組成の\index{がんゆうりょう(かがくそせい)@含有量(化学組成)}含有量も決定される。
それぞれの化学組成の含有量($\mathrm{wt}\%$)を$X_\mathrm C$, $X_\mathrm{Si}$, $X_\mathrm{Mn}$, $X_\mathrm P$, $X_\mathrm S$とし、またその\expandafterindex{えいきょうけいすう(\yomieTaper)@影響係数(\nameeTaper)}影響係数を$k_\mathrm C$, $k_\mathrm{Si}$, $k_\mathrm{Mn}$, $k_\mathrm P$, $k_\mathrm S$とする。
また、その鋼材の\index{えきそうせんおんど@液相線温度}液相線温度を$T_\mathrm l$[$^\circ\mathrm C$], \index{こそうせんおんど@固相線温度}固相線温度を$T_\mathrm s$[$^\circ\mathrm C$]とする。
一般にこれらの温度は、ある\expandafterindex{きじゅんおんど(\yomieTaper)@基準温度(\nameeTaper)}基準となる温度$T_0$に対して、
\begin{align*}
  T = T_0-\sum_i k_iX_i
\end{align*}
として与えられる。
今の場合だと、
\begin{align*}
  T_l
  &= 1536-78X_\mathrm C-7.6X_\mathrm{Si}-4.9X_\mathrm{Mn}-34.4X_\mathrm P-38X_\mathrm S~,\\
  T_s
  &= 1536-415.5X_\mathrm C-12.3X_\mathrm{Si}-6.8X_\mathrm{Mn}-124.5X_\mathrm P-183.9X_\mathrm S
\end{align*}
となることが知られている\cite{article:1986KO}。\\

\captionof{table}{化学組成の含有量}
\begin{longtblr}[
  theme=commontblr,
  entry=none,
  label=none,
  presep=0pt,
]{%
  hlines,
  vlines,
  vline{2}={1pt},
  colspec={lX[c]X[c]X[c]X[c]X[c]},
}
  鋼材 & C & Si & Mn & P & S
  \\
  \index{JISコード(がんゆうりょう)@JISコード(含有量)}JISコード(含有量wt\%) & 0.1 & 0.5 & 0.6 & 0.7 & 0.8
\end{longtblr}



\clearpage
%%%%%%%%%%%%%%%%%%%%%%%%%%%%%%%%%%%%%%%%%%%%%%%%%%%%%%%%%%
%% section 09.2 %%%%%%%%%%%%%%%%%%%%%%%%%%%%%%%%%%%%%%%%%%
%%%%%%%%%%%%%%%%%%%%%%%%%%%%%%%%%%%%%%%%%%%%%%%%%%%%%%%%%%
\modHeadsection{内側湾曲の近似曲線\TBW}
内面は\index{テーパ(ないめん)@テーパ(内面)}テーパがついた形になっている。
テーパは\IDTaperTable 等の数値に基づき、\index{CAD}CADにて\index{アイソパラメトリックきょくせん@アイソパラメトリック曲線}アイソパラメトリック曲線として記述され、それを\index{CAM}CAMに設定し、\index{しんがね@芯金}芯金が作成されている。
