%!TEX root = ../RfCPN.tex


\modHeadchapter{その他の幾何}



%%%%%%%%%%%%%%%%%%%%%%%%%%%%%%%%%%%%%%%%%%%%%%%%%%%%%%%%%%
%% section 29.1 %%%%%%%%%%%%%%%%%%%%%%%%%%%%%%%%%%%%%%%%%%
%%%%%%%%%%%%%%%%%%%%%%%%%%%%%%%%%%%%%%%%%%%%%%%%%%%%%%%%%%
\modHeadsection{回転中心とのずれの考慮}
\TableCenter 座標($X$, $Z$)に対して、\JigCenter の位置が($\delta x$, $\delta z$)だけずれている場合を考える。
このとき、任意の点($X$, $Z$)は\Table を$-\theta$だけ回転すると、
\begin{align*}
  \left[
    \begin{array}{c}
      X'\\
      Z'
    \end{array}
  \right]
  = \left[
    \begin{array}{cc}
      \cos\theta & \sin\theta\\
      -\sin\theta & \cos\theta
    \end{array}
  \right]\!\!
  \left[
    \begin{array}{c}
      X+\delta x\\
      Z+\delta z
    \end{array}
  \right]
  = \left[
    \begin{array}{c}
      (X+\delta x)\cos\theta+(Z+\delta z)\sin\theta\\
      -(X+\delta x)\sin\theta+(Z+\delta z)\cos\theta
    \end{array}
  \right].
\end{align*}
に移動する。
また、ずれによる差分は、
\begin{align*}
  \left[
    \begin{array}{c}
      \delta x\cos\theta+\delta z\sin\theta\\
      -\delta x\sin\theta+\delta z\cos\theta
    \end{array}
  \right].
\end{align*}
特に、$\theta = \pi$のとき、
\begin{align*}
  \left[
    \begin{array}{c}
      -\delta x\\
      \delta z
    \end{array}
  \right].
\end{align*}


%%%%%%%%%%%%%%%%%%%%%%%%%%%%%%%%%%%%%%%%%%%%%%%%%%%%%%%%%%
%% subsection 27.3.1 %%%%%%%%%%%%%%%%%%%%%%%%%%%%%%%%%%%%%
%%%%%%%%%%%%%%%%%%%%%%%%%%%%%%%%%%%%%%%%%%%%%%%%%%%%%%%%%%
\subsection{\WorkTotalLength の変化}
\EndFacecutMilling では\TableCenter の位置から\ReAlocationLength の長さの位置を加工する。
そのため回転によるずれを考慮すると、加工後の\ReAlocationLength はそれぞれ、
\begin{align*}
  f_\mathrm T' &\quad\longrightarrow\quad f_\mathrm T'-\delta z\ ,\\
  f_\mathrm B' &\quad\longrightarrow\quad f_\mathrm B'-\delta z\ .
\end{align*}
したがって、\WorkTotalLength は
\begin{align*}
  f_\mathrm T+f_\mathrm B \quad\longrightarrow\quad f_\mathrm T+f_\mathrm B-2\delta z\ .
\end{align*}


\clearpage
%%%%%%%%%%%%%%%%%%%%%%%%%%%%%%%%%%%%%%%%%%%%%%%%%%%%%%%%%%
%% subsection 27.3.2 %%%%%%%%%%%%%%%%%%%%%%%%%%%%%%%%%%%%%
%%%%%%%%%%%%%%%%%%%%%%%%%%%%%%%%%%%%%%%%%%%%%%%%%%%%%%%%%%
\subsection{\CenterlineEndFaceDif の変化}

%%%%%%%%%%%%%%%%%%%%%%%%%%%%%%%%%%%%%%%%%%%%%%%%%%%%%%%%%%
%% subsubsection 27.3.2.1 %%%%%%%%%%%%%%%%%%%%%%%%%%%%%%%%
%%%%%%%%%%%%%%%%%%%%%%%%%%%%%%%%%%%%%%%%%%%%%%%%%%%%%%%%%%
\subsubsection{ボトム側が基準の場合}
回転によるずれを考慮すると、\BottomOutcutCenter$\mathcal G_\mathrm B$が基準の場合、(\TableCenter Pを原点とした)\TopOutcutCenter の$X$座標$\mathcal G_{\mathrm Tx}$は、
\begin{align*}
  \mathcal G_{\mathrm Tx} = -\mathcal G_{\mathrm Bx}+T_x-\delta x\ .
\end{align*}
よって、ボトム側($-X$側)およびトップ側($+X$側)の\index{Cがわがいさくめん@C側外削面}C側外削面の中心(\Keyway を除く)は、\pageeqref{eq:centerlineB}より、
\begin{align*}
  \text{ボトム側:}\quad
  \left[
    \begin{array}{c}
      \displaystyle -f_\mathrm B'+\frac{h_\mathrm B}2+\delta z\\[5pt]
      \mathcal G_{\mathrm By}\\[3pt]
      \displaystyle \mathcal G_{\mathrm Bx}+\frac{\mathfrak W_\mathrm B}2
    \end{array}
    \right]~, \qquad
  \text{トップ側:}\quad
  \left[
    \begin{array}{c}
      \displaystyle f_\mathrm T'-\frac{\kappa_p}2-\delta z\\[5pt]
      \mathcal G_{\mathrm By}-T_y\\[3pt]
      \displaystyle \mathcal G_{\mathrm Bx}-T_x+\frac{\mathfrak W_\mathrm T}2+\delta x
    \end{array}
  \right].
\end{align*}

%\clearpage
%%%%%%%%%%%%%%%%%%%%%%%%%%%%%%%%%%%%%%%%%%%%%%%%%%%%%%%%%%
%% subsubsection 27.3.2.2 %%%%%%%%%%%%%%%%%%%%%%%%%%%%%%%%
%%%%%%%%%%%%%%%%%%%%%%%%%%%%%%%%%%%%%%%%%%%%%%%%%%%%%%%%%%
\subsubsection{トップ側が基準の場合}
回転によるずれを考慮すると、\TopOutcutCenter$\mathcal G_\mathrm T$が基準の場合、(\TableCenter Pを原点とした)\BottomOutcutCenter の$X$座標$\mathcal G_{\mathrm Bx}$は、
\begin{align*}
  \mathcal G_{\mathrm Bx} = -\mathcal G_{\mathrm Tx}+T_x-\delta x\ .
\end{align*}
よって、トップ側($+X$側)およびボトム側($-X$側)の\index{Cがわがいさくめん@C側外削面}C側外削面の中心(\Keyway を除く)は、\pageeqref{eq:centerlineT}より、
\begin{align*}
  \text{トップ側:}~~
  \left[
    \begin{array}{c}
      \displaystyle f_\mathrm T'-\frac{\kappa_p}2-\delta z\\[5pt]
      \mathcal G_{\mathrm Ty}\\[3pt]
      \displaystyle \mathcal G_{\mathrm Tx}+\frac{\mathfrak W_\mathrm B}2
    \end{array}
    \right]~, \qquad
  \text{ボトム側:}~~
  \left[
    \begin{array}{c}
      \displaystyle -f_\mathrm B'+\frac{h_\mathrm B}2+\delta z\\[5pt]
      \mathcal G_{\mathrm Ty}+T_y\\[3pt]
      \displaystyle \mathcal G_{\mathrm Tx}+T_x+\frac{\mathfrak W_\mathrm B}2
    \end{array}
  \right].
\end{align*}



\clearpage
%%%%%%%%%%%%%%%%%%%%%%%%%%%%%%%%%%%%%%%%%%%%%%%%%%%%%%%%%%
%% section 31.2 %%%%%%%%%%%%%%%%%%%%%%%%%%%%%%%%%%%%%%%%%%
%%%%%%%%%%%%%%%%%%%%%%%%%%%%%%%%%%%%%%%%%%%%%%%%%%%%%%%%%%
\modHeadsection{ワーク\FixtureBolt\TBW}
(to be written...)



\clearpage
%%%%%%%%%%%%%%%%%%%%%%%%%%%%%%%%%%%%%%%%%%%%%%%%%%%%%%%%%%
%% section 31.2 %%%%%%%%%%%%%%%%%%%%%%%%%%%%%%%%%%%%%%%%%%
%%%%%%%%%%%%%%%%%%%%%%%%%%%%%%%%%%%%%%%%%%%%%%%%%%%%%%%%%%
\modHeadsection{\index{ワークのこゆうしんどう@ワークの固有振動}ワークの固有振動}
ここでは\index{ワークのこゆうしんどう@ワークの固有振動}ワークの固有振動を考える。
簡単なモデルとして、\index{ワーク}ワークは\Jig で十分に固定されているものとし、固定部分からの\index{かたもちはり@片持ち梁}片持ち梁として\index{ワーク}ワークを捉えることにする。

\index{はりのながさ@梁の長さ}梁の長さを$L$, \index{ヤングりつ@ヤング率}ヤング率を$E$, \index{だんめん2じモーメント@断面2次モーメント}断面2次モーメントを$I$とすると、\index{かたもちはりのごうせい@片持ち梁の剛性}片持ち梁の剛性$k$は
\begin{align*}
  k = \frac{3EI}{L^3}
\end{align*}
で与えられることが知られている。
このとき、この梁の\index{こゆうしんどうすう@固有振動数}固有振動数$f$は、次で与えられる。
\begin{align*}
  f = \frac1{2\pi}\sqrt{\frac mk}
    = \frac1{2\pi}\sqrt{\frac{mL^3}{3EI}}\ .
\end{align*}
$m$は梁の先端に作用する質量を示す。

さらに話を簡単化し、\index{ワークのだんめん@ワークの断面}ワークの断面が中空長方形(角パイプ状)として考える。
外側の長方形の高さおよび幅をそれぞれ$H$, $W$, 内側の長方形の高さおよび幅をそれぞれ$h$, $w$とすると、その断面積$A$は、
\begin{align*}
  A = HW-hw
\end{align*}
で与えられ、\index{だんめん2じモーメント@断面2次モーメント}断面2次モーメント$I$は
\begin{align*}
  I = \frac{H^3W-h^3b}{12}\ .
\end{align*}
また、\index{だんめんけいすう@断面係数}断面係数$Z$は、
\begin{align*}
  \frac{H^3W-h^3b}{6H}\ .
\end{align*}
これより、この場合の\index{こゆうしんどうすう@固有振動数}固有振動数は、
\begin{align*}
  \frac1\pi\sqrt{\frac{mL^3}{E\left(H^3W-h^3b\right)}}\ .
\end{align*}
