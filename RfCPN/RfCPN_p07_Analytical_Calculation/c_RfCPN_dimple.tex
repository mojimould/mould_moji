%!TEX root = ../RfCPN.tex


\modHeadchapter[loColumn]{\Dimple の幾何}
ここでは主に\textbf{\Dimple}に関する測定・加工に必要な\expandafterindex{きかてきせいしつ(\yomiDimple)@幾何的性質(\nameDimple)}幾何的性質を考える。

なお、\DimpleMilling は\MMC で行うことはできず、\DMC のみで行う。
また\DMC では、\AlocationLength の調整について\index{スペーサ}スペーサを用いた方法は行わず、\index{テーブル}テーブルの回転を用いた方法のみで行う方針である。
したがって、\index{スペーサ}スペーサを用いた方法の場合は考慮する必要がない。
そのため以降では、(\Dimple に関する測定・加工については)\index{テーブル}テーブルを$-\theta$だけ回転した場合についてのみを考えることにする。



%%%%%%%%%%%%%%%%%%%%%%%%%%%%%%%%%%%%%%%%%%%%%%%%%%%%%%%%%%
%% section 26.1 %%%%%%%%%%%%%%%%%%%%%%%%%%%%%%%%%%%%%%%%%%
%%%%%%%%%%%%%%%%%%%%%%%%%%%%%%%%%%%%%%%%%%%%%%%%%%%%%%%%%%
\modHeadsection{\Dimple の表記法}
初めに、\expandafterindex{ひょうき(\yomiDimple)@表記(\nameDimple)}\nameDimple に関する表記を簡単にまとめておく。
なお\Dimple はトップ側にあるため、トップ側が工具側に向いているものとして話を進める。

%%%%%%%%%%%%%%%%%%%%%%%%%%%%%%%%%%%%%%%%%%%%%%%%%%%%%%%%%%
%% tcolorbox %%%%%%%%%%%%%%%%%%%%%%%%%%%%%%%%%%%%%%%%%%%%%
%%%%%%%%%%%%%%%%%%%%%%%%%%%%%%%%%%%%%%%%%%%%%%%%%%%%%%%%%%
\begin{tcolorbox}[title={\expandafterindex{ひょうき(\yomiDimple)@表記(\nameDimple)}\nameDimple に関する表記法}, fonttitle=\gtfamily\bfseries, breakable, enhanced jigsaw]
\begin{enumerate}[label=\sarrow]
\item
\subparagraph*{列の数えかた}
\Dimple が$m$列あるとき、トップ側から順に1列目, 2列目, …,$m$列目のように数える。

\item
\subparagraph*{列内の個数の数えかた}
各々の\expandafterindex{れつ(\yomiDimple)@列(\nameDimple)}列の\DimpleNum は、AC面側については工具側からみて下から順に、BD面については工具側からみて右から順に1つ目,2つ目,…のように数える。

\item
\subparagraph*{\Dimple の寸法}
\expandafterindex{トップたんと\yomiDimpleFirstRow とのきょり@トップ端と\nameDimpleFirstRow との距離}トップ端から1列目までの距離を$q$, 鉛直・水平方向の\expandafterindex{ピッチ(\yomiDimple)@ピッチ(\nameDimple)}ピッチをそれぞれ$p_z$, $p_x$とし、\expandafterindex{iれつめのながさ(\yomiDimple)@$i$列目の長さ(\nameDimple)}$i$列目の長さをそれぞれ$d_i$とする。

特に、\expandafterindex{きすうれつめのながさ(\yomiDimple)@奇数列目の長さ(\nameDimple)}奇数列目の長さが全て同じ場合はその長さを$d_\mathrm o$, \expandafterindex{ぐうすうれつめのながさ(\yomiDimple)@偶数列目の長さ(\nameDimple)}偶数列目の長さが全て同じ場合はその長さを$d_\mathrm e$とも表記する。
(\pageautoref{fn:generallyDimpleN}および\pageautoref{hosoku:generallyDimpleN}参照)

\item
\subparagraph*{\IDTaperTable の寸法}
\IDTaperTable における\expandafterindex{トップたんからのきょり(\yomiIDTaperTable)@トップ端からの距離(\nameIDTaperTable)}トップ端からの距離を$\lambda_i$ ($i = 0$, $1$, $2$, $\cdots$), それに対するAC・BD側\expandafterindex{ないけい(\yomiIDTaperTable)@内径(\nameIDTaperTable)}内径をそれぞれ$w_{\mathrm Ai}$, $w_{\mathrm Bi}$とする。
(\pageautoref{hosoku:example4taper}参照)

\item
\subparagraph*{\InnerDiameter の(近似)寸法}
トップ端から$\lambda$の位置の\ACID を$w_{\mathrm A\lambda}$と表す。
このとき$w_{\mathrm A\lambda}$は、$\lambda_j \leq \lambda < \lambda_{j+1}$に対する$w_{\mathrm Aj}$, $w_{\mathrm Aj+1}$の\expandafterindex{かじゅうさんじゅつへいきん(\yomiInnerDiameter)@加重算術平均(\nameInnerDiameter)}加重算術平均(\expandafterindex{ウェイトさんじゅつへいきん(\yomiInnerDiameter)@ウェイト算術平均(\nameInnerDiameter)}ウェイト算術平均・\expandafterindex{おもみつきさんじゅつへいきん(\yomiInnerDiameter)@重み付き算術平均(\nameInnerDiameter)}重み付き算術平均)
\begin{align*}
  w_{\mathrm A\lambda}
  = \frac{(\lambda-\lambda_j)w_{\mathrm Aj+1}+(\lambda_{j+1}-\lambda)w_{\mathrm Aj}}{\lambda_{j+1}-\lambda_j}
  \qquad
  \Big(\lambda_j \leq \lambda < \lambda_{j+1}\Big)
\end{align*}
とみなすことにする。($w_{\mathrm B\lambda}$についても同様)

\item
\subparagraph*{\PlatingThk を含めた\InnerDiameter の(近似)寸法}
\PlatingThk$\mu$を考慮した\ACID・\BDID$w'_{\mathrm A\lambda}$, $w'_{\mathrm B\lambda}$をそれぞれ以下のように表す。
\begin{align*}
  w'_{\mathrm A\lambda} \equiv w_{\mathrm A\lambda}+2\mu~, \quad
  w'_{\mathrm B\lambda} \equiv w_{\mathrm B\lambda}+2\mu~.
\end{align*}
\end{enumerate}
\end{tcolorbox}\noindent
%%%%%%%%%%%%%%%%%%%%%%%%%%%%%%%%%%%%%%%%%%%%%%%%%%%%%%%%%%
%%%%%%%%%%%%%%%%%%%%%%%%%%%%%%%%%%%%%%%%%%%%%%%%%%%%%%%%%%
%%%%%%%%%%%%%%%%%%%%%%%%%%%%%%%%%%%%%%%%%%%%%%%%%%%%%%%%%%
このとき$m$列目の\DimpleNum$n_m$は、$n_m = \nicefrac{d_m}{p_x}+1$となる
%% footnote %%%%%%%%%%%%%%%%%%%%%
\phantomsection
\footnote{\label{fn:generallyDimpleN}%
たいていの場合、\expandafterindex{きすうれつのこすう(\yomiDimple)@奇数列の個数(\nameDimple)}奇数列の個数は全て同じ数$n_\mathrm o$であり、\expandafterindex{ぐうすうれつのこすう(\yomiDimple)@偶数列の個数(\nameDimple)}偶数列の個数も全て同じ$n_\mathrm e$である。
また$|n_\mathrm o-n_\mathrm d| = 1$である。}。
%%%%%%%%%%%%%%%%%%%%%%%%%%%%%%%%%
%%%%%%%%%%%%%%%%%%%%%%%%%%%%%%%%%%%%%%%%%%%%%%%%%%%%%%%%%%
%% hosoku %%%%%%%%%%%%%%%%%%%%%%%%%%%%%%%%%%%%%%%%%%%%%%%%
%%%%%%%%%%%%%%%%%%%%%%%%%%%%%%%%%%%%%%%%%%%%%%%%%%%%%%%%%%
\begin{hosoku}[label=hosoku:example4taper]
たとえば\IDTaperTable の値が25mm\expandafterindex{ピッチ(\yomiIDTaperTable)@ピッチ(\nameIDTaperTable)}ピッチの場合、$\lambda_0=0$, $\lambda_1=25$, $\lambda_2=50$, $\cdots$とし、それぞれのACおよびBD側\expandafterindex{ないけい(\yomiIDTaperTable)@内径(\nameIDTaperTable)}内径を$w_{\mathrm A0}$, $w_{\mathrm A1}$, $w_{\mathrm A2}$, $\cdots$および$w_{\mathrm B0}$, $w_{\mathrm B1}$, $w_{\mathrm B2}$, $\cdots$とする、という意味である。
ここでは離散値である$\lambda_i$を、連続値$\lambda$に(近似的に)置きかえている。
実際、たとえば$\lambda = \lambda_j$のとき$w_{\mathrm Aj} = w_{\mathrm A\lambda}$となることがわかる。
\end{hosoku}\relax
%%%%%%%%%%%%%%%%%%%%%%%%%%%%%%%%%%%%%%%%%%%%%%%%%%%%%%%%%%
%%%%%%%%%%%%%%%%%%%%%%%%%%%%%%%%%%%%%%%%%%%%%%%%%%%%%%%%%%
%%%%%%%%%%%%%%%%%%%%%%%%%%%%%%%%%%%%%%%%%%%%%%%%%%%%%%%%%%
%%%%%%%%%%%%%%%%%%%%%%%%%%%%%%%%%%%%%%%%%%%%%%%%%%%%%%%%%%
%% hosoku %%%%%%%%%%%%%%%%%%%%%%%%%%%%%%%%%%%%%%%%%%%%%%%%
%%%%%%%%%%%%%%%%%%%%%%%%%%%%%%%%%%%%%%%%%%%%%%%%%%%%%%%%%%
\begin{hosoku}
\IDTaperTable の\expandafterindex{ピッチ(\yomiIDTaperTable)@ピッチ(\nameIDTaperTable)}ピッチ$\lambda_{i+1}-\lambda_i$は常に一定の場合が多い。
$\lambda_{i+1}-\lambda_i$が$i$について常に一定であれば、$\lambda_j \leq z < \lambda_{j+1}$となる$j$は、
\begin{align*}
  j = z \bDiv (\lambda_{i+1}-\lambda_i) = \left\lfloor\frac z{\lambda_{i+1}-\lambda_i}\right\rfloor
\end{align*}
のように表すことができる。
\end{hosoku}
%%%%%%%%%%%%%%%%%%%%%%%%%%%%%%%%%%%%%%%%%%%%%%%%%%%%%%%%%%
%%%%%%%%%%%%%%%%%%%%%%%%%%%%%%%%%%%%%%%%%%%%%%%%%%%%%%%%%%
%%%%%%%%%%%%%%%%%%%%%%%%%%%%%%%%%%%%%%%%%%%%%%%%%%%%%%%%%%
\vfill
%%%%%%%%%%%%%%%%%%%%%%%%%%%%%%%%%%%%%%%%%%%%%%%%%%%%%%%%%%
%% Column %%%%%%%%%%%%%%%%%%%%%%%%%%%%%%%%%%%%%%%%%%%%%%%%
%%%%%%%%%%%%%%%%%%%%%%%%%%%%%%%%%%%%%%%%%%%%%%%%%%%%%%%%%%
\begin{\Columnname}{商$\boldsymbol{\bDiv}$と余り$\boldsymbol{\bmod}$とガウス括弧$\boldsymbol{\lfloor\,\rfloor}$}
\renewcommand\theequation{c\thechapter.\arabic{equation}}
\setcounter{equation}{0}
\paragraph*{$\boldsymbol\bDiv$と$\boldsymbol\bmod$}
割り算の余りを表す記号としては\index{mod(あまり)@$\bmod$(余り)}$\bmod$が広く使われる。
商を表す記号は一般的な数学のテキスト等ではあまり用いられないが、プログラミング言語等では\index{div(しょう)@$\bDiv$(商)}$\bDiv$を用いられることがある。
これに倣って、ここでは商には$\bDiv$, 余りには$\bmod$を用いている。

 一般に、実数$a$, $b$ ($b\neq0$)に対して$a = bq+r$ ($0 \leq r < |b|$)を満たす整数$q$を\index{しょう(div)@商($\bDiv$)}商、$r$を\index{あまり(mod)@余り($\bmod$)}余りと呼び、このとき$a \bDiv b = q$および$a \bmod b = r$のように表される。
なお、ここでは簡単のため、$q \geq 0$として考えることにする。
\tcbline*
\paragraph*{ガウス括弧}
\index{ガウスかっこ@ガウス括弧}$\lfloor x\rfloor$は、$x \in R$ に対して$x$を超えない最大の整数。
簡単にいうと、($x > 0$の場合は)小数点以下を切り捨てた整数部分を表す。
\index{ガウスきごう@ガウス記号}ガウス記号, \index{ゆかかんすう@床関数}床関数(floor function)などとも呼ばれる。
\end{\Columnname}



\clearpage
%%%%%%%%%%%%%%%%%%%%%%%%%%%%%%%%%%%%%%%%%%%%%%%%%%%%%%%%%%
%% section 6.2 %%%%%%%%%%%%%%%%%%%%%%%%%%%%%%%%%%%%%%%%%%%
%%%%%%%%%%%%%%%%%%%%%%%%%%%%%%%%%%%%%%%%%%%%%%%%%%%%%%%%%%
\modHeadsection{\DimpleMilling の基本方針}
\DimpleMilling における留意事項の1つに、ワークの内面(特にトップ端)と工具が接触してしまう\index{アンダーカット}アンダーカットというものがある
%% footnote %%%%%%%%%%%%%%%%%%%%%
\footnote{\Dimple の測定・加工ではとりわけアンダーカットが生じやすい、という意味である。
その他の測定・加工についても当然アンダーカットは十分に生じうる。}。
%%%%%%%%%%%%%%%%%%%%%%%%%%%%%%%%%
特にA面側は工具へ向かう方向に\index{わんきょく@湾曲}湾曲があるため、アンダーカットが生じやすい。
そこで、アンダーカットを避けつつ加工ができるようにするため、\index{ワーク}をいくらか(湾曲と反対側に)傾けて加工を行う。
その\expandafterindex{かたむきかく(\yomiDimple)@傾き角(\nameDimple)}傾き角$\phi$ ($0 \leq \phi < \nicefrac\pi2$)について、ここでは次の2点を基準に考えることにする。
\begin{tcolorbox}[title=A面の\Dimple, fonttitle=\gtfamily\bfseries]
\begin{enumerate}[label=\sarrow]
\item A側内面のトップ端点
\item \AFaceDimpleFirstRow(トップ端から$q$)の位置
\end{enumerate}
\end{tcolorbox}\noindent
この2点を通る直線と鉛直方向との角度を、傾き角$-\phi$とする
%% footnote %%%%%%%%%%%%%%%%%%%%%
\footnote{\AlocationLength の調整に用いたテーブルの\index{かたむきかく(ふりわけちょうせい)@傾き角(振分調整)}傾き角$\theta$と混同しないように注意。}。
%%%%%%%%%%%%%%%%%%%%%%%%%%%%%%%%%
なお、\TopEndACID は$w'_{\mathrm A0}$で代用してもよいものとする。
このとき$\phi > 0$となる(C面側に傾く)場合は$\phi$だけ傾けて加工を行う。
一方、$\phi \leq 0$となる(A面側に傾く)場合は、そもそもアンダーカットが生じないので、傾けずにそのまま加工を行うものとする。
%%%%%%%%%%%%%%%%%%%%%%%%%%%%%%%%%%%%%%%%%%%%%%%%%%%%%%%%%%
%% hosoku %%%%%%%%%%%%%%%%%%%%%%%%%%%%%%%%%%%%%%%%%%%%%%%%
%%%%%%%%%%%%%%%%%%%%%%%%%%%%%%%%%%%%%%%%%%%%%%%%%%%%%%%%%%
\begin{hosoku}
ここでは\DimpleMilling 用の工具として、\index{Tスロットカッター}Tスロットカッターを考えている。
しかし、当然ながら\index{こうぐけい@工具径}工具径は有限であるため、いくら適切に傾けたところで限界はある。
ここではその限界として、A側内面のトップ端の$X$座標と、それと最も$X$座標が近い\Dimple との($X$方向の)距離を算出する。
そしてそれを工具径と比べることで、どこまでの範囲を加工するかを決定する。
加工できない部分に\Dimple がある場合は、別の工具(\index{アングルヘッド}アングルヘッド)等を使用して加工を行う。
\end{hosoku}
%%%%%%%%%%%%%%%%%%%%%%%%%%%%%%%%%%%%%%%%%%%%%%%%%%%%%%%%%%
%%%%%%%%%%%%%%%%%%%%%%%%%%%%%%%%%%%%%%%%%%%%%%%%%%%%%%%%%%
%%%%%%%%%%%%%%%%%%%%%%%%%%%%%%%%%%%%%%%%%%%%%%%%%%%%%%%%%%
%%%%%%%%%%%%%%%%%%%%%%%%%%%%%%%%%%%%%%%%%%%%%%%%%%%%%%%%%%
%% Column %%%%%%%%%%%%%%%%%%%%%%%%%%%%%%%%%%%%%%%%%%%%%%%%
%%%%%%%%%%%%%%%%%%%%%%%%%%%%%%%%%%%%%%%%%%%%%%%%%%%%%%%%%%
\begin{\Columnname}{曲率と傾き}
内面A側・C側の\index{わんきょく(ないめん)@湾曲(内面)}湾曲をそれぞれ$\mathcal R_\mathrm o$, $\mathcal R_\mathrm i$とすると、\index{きょくりつ(ないめん)@曲率(内面)}曲率はそれぞれ$\mathcal R_\mathrm o^{-1} < R_\mathrm c^{-1} < \mathcal R_\mathrm i^{-1}$である。
そのため、(トップ側の)A側の$\mathcal R_\mathrm o$を基準にするとより緩やかに、C側の$\mathcal R_\mathrm i$を基準にするとよりきつく傾くことになる。
また、トップ端から($Z$方向に)遠い点を基準にするとより緩やかに、近い点を基準にするとよりきつく傾くことになる。
\end{\Columnname}
%%%%%%%%%%%%%%%%%%%%%%%%%%%%%%%%%%%%%%%%%%%%%%%%%%%%%%%%%%
%%%%%%%%%%%%%%%%%%%%%%%%%%%%%%%%%%%%%%%%%%%%%%%%%%%%%%%%%%
%%%%%%%%%%%%%%%%%%%%%%%%%%%%%%%%%%%%%%%%%%%%%%%%%%%%%%%%%%

以下ではこの傾き角$\phi$と、回転後の\Dimple や内面の位置を定量的に与えることを試みる。



\clearpage
%%%%%%%%%%%%%%%%%%%%%%%%%%%%%%%%%%%%%%%%%%%%%%%%%%%%%%%%%6
%% section 6.3 %%%%%%%%%%%%%%%%%%%%%%%%%%%%%%%%%%%%%%%%%%%
%%%%%%%%%%%%%%%%%%%%%%%%%%%%%%%%%%%%%%%%%%%%%%%%%%%%%%%%%%
\modHeadsection{\Dimple の位置と傾き角(傾き前)}
\pageeqref{eq:tableTRc}より、テーブルを$-\theta$傾けて\AlocationLength の調整を行った場合、\TableCenter Pを原点とした\CenterCurvatureLine のトップ端における$X$座標は、
\begin{align*}
  R_\mathrm c\cos\alpha_\mathrm c-\Delta'\cos\theta = \sqrt{R_\mathrm c^2-f_\mathrm T^2}-\Delta'\cos\theta
\end{align*}
で与えられる。
これは\index{タッチセンサープローブ}タッチセンサープローブによる\expandafterindex{そくていかいしてん(\yomiDimple)@測定開始点(\nameDimple)}測定の開始点として用いることができる。
一方で、それ以外の作業では、\TopIDCenter の座標$g_\mathrm T$を直接測定するので、それを用いることにする
%% footnote %%%%%%%%%%%%%%%%%%%%%
\footnote{これは\\CenterCurvatureLine 上にない点であるが、\index{こうさ@公差}公差の範囲内であるものとして、ここではこれで代用する。}。
%%%%%%%%%%%%%%%%%%%%%%%%%%%%%%%%%
よって、\TableCenter Pを原点とした場合における、\DimpleFirstRow 中央の(だいたいの)位置
%% footnote %%%%%%%%%%%%%%%%%%%%%
\footnote{$w_{\mathrm Aq}$, $w_{\mathrm Bq}$は\CurvatureCenter\index{O(\yomiCurvatureCenter)@O(\nameCurvatureCenter)}O(0, 0)方向への長さであるため正確ではないことに注意。}\relax
%%%%%%%%%%%%%%%%%%%%%%%%%%%%%%%%%
は、次で与えられる。
\begin{align*}
\begin{array}{rl}
  \text{A面($+X$方向):}
  & \displaystyle
    \left(
      g_{\mathrm Tx}+\mathcal L_0+\frac{w'_{\mathrm Aq}}2~,~
      g_{\mathrm Ty}~,~
      f_\mathrm T'-q
    \right),\\[12pt]
  \text{C面($-X$方向):}
  & \displaystyle
    \left(
      g_{\mathrm Tx}+\mathcal L_0-\frac{w'_{\mathrm Aq}}2~,~
      g_{\mathrm Ty}~,~
      f_\mathrm T'-q
    \right),\\[12pt]
  \text{B面($+Y$方向):}
  & \displaystyle
    \left(
      g_{\mathrm Tx}+\mathcal L_0~,~
      g_{\mathrm Ty}+\frac{w'_{\mathrm Bq}}2~,~
      f_\mathrm T'-q
    \right),\\[12pt]
  \text{D面($-Y$方向):}
  & \displaystyle
    \left(
      g_{\mathrm Tx}+\mathcal L_0~,~
      g_{\mathrm Ty}-\frac{w'_{\mathrm Bq}}2~,~
      f_\mathrm T'-q
    \right).
\end{array}
\end{align*}
ここで、\DimpleIRow の\CurvatureCenter と\TopCurvatureCenter との$X$座標の差を、
%% label{eq:}
\begin{align}
  \label{eq:dimpleCenterDistance}
  \mathcal L_i
  \equiv \sqrt{R_\mathrm c^2-\left\{f_\mathrm T-q-(i-1)p_z\right\}^2}-\sqrt{R_\mathrm c^2-f_\mathrm T^2}
\end{align}
と表した。
なお、$i$列目の\CurvatureCenter と$j$列目の\CurvatureCenter との$X$座標の差を
\begin{align*}
  \mathcal L_{i,j}
  \equiv \mathcal L_i-\mathcal L_j
  = \sqrt{R_\mathrm c^2-\left(f_\mathrm T-q-(i-1)p_z\right)^2}
    -\sqrt{R_\mathrm c^2-\left\{f_\mathrm T-q-(j-1)p_z\right\}^2}
\end{align*}
と表すことにする。


%%%%%%%%%%%%%%%%%%%%%%%%%%%%%%%%%%%%%%%%%%%%%%%%%%%%%%%%%%
%% subsection 5.3.1 %%%%%%%%%%%%%%%%%%%%%%%%%%%%%%%%%%%%%%
%%%%%%%%%%%%%%%%%%%%%%%%%%%%%%%%%%%%%%%%%%%%%%%%%%%%%%%%%%
\subsection{\Dimple の\texorpdfstring{$X$}{X}座標(傾き前)}
\TableCenter Pを原点としたとき、傾き前の\DimpleIRowJ の$X$座標は、
%% label{eq:dPosXBefore}
\begin{align}
  \notag
  \text{A面:}\quad
  \mathcal D_{xi,\mathrm A}
  &= g_{\mathrm Tx}+\mathcal L_i+\frac{w'_{\mathrm Aq+(i-1)p_z}}2\\
  \label{eq:dPosXBefore}
  \text{C面:}\quad
  \mathcal D_{xi,\mathrm C}
  &= g_{\mathrm Tx}+\mathcal L_i-\frac{w'_{\mathrm Aq+(i-1)p_z}}2\\
  \notag
  \text{B, D面:}\quad
  \mathcal D_{xij,\mathrm B}
  &= g_{\mathrm Tx}+\mathcal L_i+\frac{d_i}2-(j-1)p_x
\end{align}
なお、A・C面については$j$に依らないことがわかる。
そのため、たとえば$\mathcal D_{xij,\mathrm A}$ではなく、$\mathcal D_{xi,\mathrm A}$のように表記している。


\clearpage
%%%%%%%%%%%%%%%%%%%%%%%%%%%%%%%%%%%%%%%%%%%%%%%%%%%%%%%%%%
%% subsection 5.3.2 %%%%%%%%%%%%%%%%%%%%%%%%%%%%%%%%%%%%%%
%%%%%%%%%%%%%%%%%%%%%%%%%%%%%%%%%%%%%%%%%%%%%%%%%%%%%%%%%%
\subsection{\Dimple の\texorpdfstring{$Y$}{Y}座標(傾き前)}
\TableCenter Pを原点としたとき、傾き前の\DimpleIRowJ の$Y$座標は、
%% label{eq:dPosYBefore}
\begin{alignat}{3}
  \notag
  \text{A, C面:}\quad
  && \mathcal D_{yij,\mathrm A} &= g_{\mathrm Ty}-\frac{d_i}2+(j-1)p_x\\
  \label{eq:dPosYBefore}
  \text{B面:}\quad
  && \mathcal D_{yi,\mathrm B} &= g_{\mathrm Ty}+\frac{w'_{\mathrm Bq+(i-1)p_z}}2\\
  \notag
  \text{D面:}\quad
  && \mathcal D_{yi,\mathrm D} &= g_{\mathrm Ty}-\frac{w'_{\mathrm Bq+(i-1)p_z}}2
\end{alignat}
B・D面については$j$に依らないことがわかる。


%%%%%%%%%%%%%%%%%%%%%%%%%%%%%%%%%%%%%%%%%%%%%%%%%%%%%%%%%%
%% subsection 5.3.3 %%%%%%%%%%%%%%%%%%%%%%%%%%%%%%%%%%%%%%
%%%%%%%%%%%%%%%%%%%%%%%%%%%%%%%%%%%%%%%%%%%%%%%%%%%%%%%%%%
\subsection{\Dimple の\texorpdfstring{$Z$}{Z}座標(傾き前)}
\TableCenter Pを原点としたとき、傾き前の\DimpleIRowJ の$Z$座標は、
%% label{eq:dPosZBefore}
\begin{align}
  \label{eq:dPosZBefore}
  \text{A, B, C, D面:}\quad
  \mathcal D_{zi} = f_\mathrm T'-q-(i-1)p_z
\end{align}
$Z$座標についてはどの面も$j$に依らないことがわかる。


%%%%%%%%%%%%%%%%%%%%%%%%%%%%%%%%%%%%%%%%%%%%%%%%%%%%%%%%%%
%% subsection 5.3.4 %%%%%%%%%%%%%%%%%%%%%%%%%%%%%%%%%%%%%%
%%%%%%%%%%%%%%%%%%%%%%%%%%%%%%%%%%%%%%%%%%%%%%%%%%%%%%%%%%
\subsection{傾き角}
\index{トップがわのAがわうちたんてん@トップ側のA側内端点}A側内面トップ端と、\expandafterindex{Aがわないめん(\yomiDimpleFirstRow)@A側内面(\nameDimpleFirstRow)}A側内面のトップ端から$q$の位置との$x$方向の差は、
\begin{align*}
  \sqrt{\left(R_\mathrm c+\frac{w'_{\mathrm Aq}}2\right)^2-(f_\mathrm T-q)^2}
  -\sqrt{\left(R_\mathrm c+\frac{w'_{\mathrm A0}}2\right)^2-f_\mathrm T^2}
\end{align*}
で与えられる。
このとき、これが負になる場合は傾ける必要はなく、正となる場合のみ傾ける。
したがってその\index{かたむきかく(\yomiDimple)@傾き角(\nameDimple)}傾き角$\phi$は、
%% label{eq:dKatamuki}
\begin{subequations}
\label{eq:dKatamuki}
\begin{alignat}{2}
  \text{正の場合:}&&\quad
  \tan\phi
  &= \frac{\displaystyle
           \sqrt{\left(R_\mathrm c+\frac{w'_{\mathrm Aq}}2\right)^2-(f_\mathrm T-q)^2}
           -\sqrt{\left(R_\mathrm c+\frac{w'_{\mathrm A0}}2\right)^2-f_\mathrm T^2}}q\\[8pt]
  \text{負の場合:}&&
  \phi
  &= 0
\end{alignat}
\end{subequations}
で与えられる。
%%%%%%%%%%%%%%%%%%%%%%%%%%%%%%%%%%%%%%%%%%%%%%%%%%%%%%%%%%
%% Column %%%%%%%%%%%%%%%%%%%%%%%%%%%%%%%%%%%%%%%%%%%%%%%%
%%%%%%%%%%%%%%%%%%%%%%%%%%%%%%%%%%%%%%%%%%%%%%%%%%%%%%%%%%
\begin{\Columnname}{傾き角が負となるモールド}
$\phi < 0$となるのは、
\begin{align*}
  & \left(R_\mathrm c+\frac{w'_{\mathrm Aq}}2\right)^2-(f_\mathrm T-q)^2
    < \left(R_\mathrm c+\frac{w'_{\mathrm A0}}2\right)^2-f_\mathrm T^2\\
  \longrightarrow~~
  & \frac{w_{\mathrm A0}-w_{\mathrm Aq}}2
    \left(2R_\mathrm c+\frac{w_{\mathrm A0}'+w_{\mathrm Aq}'}2\right)
    > q(2f_\mathrm T-q)
\end{align*}
の場合である。
したがって、以下のような場合に生じる傾向があることがわかる。
\begin{enumerate}
\item \index{きょくりつ@曲率}曲率が小さい(湾曲$R_\mathrm c$が大きい)
\item \IDTaper がきつい($w_{\mathrm A0}-w_{\mathrm Aq}$が大きい)
\item \InnerDiameter・\PlatingThk が大きい($w_\mathrm A'$が大きい)
\end{enumerate}
たとえば、曲率0 ($R = +\infty$)の\index{モールド}モールド、つまり(\index{がいけい(ワーク)@外形(ワーク)}外形が)まっすぐの\index{わんきょくのないモールド@湾曲のないモールド}湾曲のないモールドなどが、これに該当する。
\end{\Columnname}
%%%%%%%%%%%%%%%%%%%%%%%%%%%%%%%%%%%%%%%%%%%%%%%%%%%%%%%%%%
%%%%%%%%%%%%%%%%%%%%%%%%%%%%%%%%%%%%%%%%%%%%%%%%%%%%%%%%%%
%%%%%%%%%%%%%%%%%%%%%%%%%%%%%%%%%%%%%%%%%%%%%%%%%%%%%%%%%%
%%%%%%%%%%%%%%%%%%%%%%%%%%%%%%%%%%%%%%%%%%%%%%%%%%%%%%%%%%
%% Column %%%%%%%%%%%%%%%%%%%%%%%%%%%%%%%%%%%%%%%%%%%%%%%%
%%%%%%%%%%%%%%%%%%%%%%%%%%%%%%%%%%%%%%%%%%%%%%%%%%%%%%%%%%
\begin{\Columnname}{\CfaceDimple の傾き角}
\CfaceDimple については傾斜が外側に向いているため、傾けなくとも\index{アンダーカット}アンダーカットの心配はない。
しかし、傾けたまま加工をすると形状が歪になってしまうため、\Dimple の形状をより円に近い形にするためには傾いていないほうが望ましい。
また一方で、面によって傾ける傾けないを分けると、\index{NCプログラム}NCプログラムが複雑になる(\index{じょうけんぶんき@条件分岐}条件分岐が増える)要因にもなる。
そのためここでは、どの面の\Dimple に対しても同じ角度$\phi$を用いて加工を行うことにする。
\tcbline*
なお、\CfaceDimple の形状をできるだけよいものにするには、\index{Cがわないめんテーパ@C側内面テーパ}C側の内面テーパに基づいた角度を用いるほうが望ましい。
そのため、\CfaceDimple に対する\expandafterindex{かたむきかく(\yomiCfaceDimple)@傾き角(\nameCfaceDimple)}傾き角$\phi_\mathrm C$についても(1つの例として)与えておく。
具体的には、以下の2点を基準として角度$\phi_\mathrm C$を取ることとする。
\begin{enumerate}
\item[a)] \CFaceDimpleFirstRow(トップ端から$q$)の位置
\item[b)] \CFaceDimpleLastRow(トップ端から$q+(m-1)p_z$)の位置
\end{enumerate}
C側内面のトップ端から$q$の位置と、C側内面のトップ端から$q+(m-1)p_z$の位置との$x$方向の差は、
\begin{align*}
  \sqrt{\left(R_\mathrm c-\frac{w'_{\mathrm Aq+(m-1)p_z}}2\right)^2
        -\left\{f_\mathrm T-q-(m-1)p_z\right\}^2}
  -\sqrt{\left(R_\mathrm c-\frac{w'_{\mathrm Aq}}2\right)^2-(f_\mathrm T-q)^2}
\end{align*}
これより、\CfaceDimple に対する傾き角$\phi_\mathrm C$ ($\phi_\mathrm C > 0$)は、
\begin{align*}
  \tan\phi_\mathrm C
  = \frac{\displaystyle
          \sqrt{\left(R_\mathrm c-\frac{w'_{\mathrm Aq+(m-1)p_z}}2\right)^2
                -\left\{f_\mathrm T-q-(m-1)p_z\right\}^2}
          -\sqrt{\left(R_\mathrm c-\frac{w'_{\mathrm Aq}}2\right)^2-(f_\mathrm T-q)^2}}
         {(m-1)p_z}
\end{align*}
で与えられる。
なお、前述の通り$w_{\mathrm Aq+(m-1)p_z}$は$\lambda_j \leq q+(m-1)p_z < \lambda_{j+1}$に対する$w_{\mathrm Aj}$, $w_{\mathrm Aj+1}$の\expandafterindex{かじゅうさんじゅつへいきん(\yomiInnerDiameter)@加重算術平均(\nameInnerDiameter)}加重算術平均
\begin{align*}
  w_{\mathrm Aq+(m-1)p_z}
  = \frac{\{q+(m-1)p_z-\lambda_j\}w_{\mathrm Aj+1}+\{\lambda_{j+1}-q-(m-1)p_z\}w_{\mathrm Aj}}
         {\lambda_{j+1}-\lambda_j}
\end{align*}
であり、\InnerDiameter として代用している。($w_{\mathrm Bq+(m-1)p_z}$についても同様)
\end{\Columnname}
%%%%%%%%%%%%%%%%%%%%%%%%%%%%%%%%%%%%%%%%%%%%%%%%%%%%%%%%%%
%%%%%%%%%%%%%%%%%%%%%%%%%%%%%%%%%%%%%%%%%%%%%%%%%%%%%%%%%%
%%%%%%%%%%%%%%%%%%%%%%%%%%%%%%%%%%%%%%%%%%%%%%%%%%%%%%%%%%


\clearpage
%%%%%%%%%%%%%%%%%%%%%%%%%%%%%%%%%%%%%%%%%%%%%%%%%%%%%%%%%%
%% subsection 26.3.5 %%%%%%%%%%%%%%%%%%%%%%%%%%%%%%%%%%%%%
%%%%%%%%%%%%%%%%%%%%%%%%%%%%%%%%%%%%%%%%%%%%%%%%%%%%%%%%%%
\subsection{\BfaceDimple, \DfaceDimple の位置(傾き前)}
\BfaceDimple, \DfaceDimple において、その$X$座標がA側内面に最も近いものは、$m-1$列目または$m$列目の\expandafterindex{\yomiDimple1ばんめ@\nameDimple1番目}1番目の\nameDimple である。
これらの$X$座標は\pageeqref{eq:dPosXBefore}よりそれぞれ、
\begin{align*}
  m-1\text{列目:}&\quad
  g_{\mathrm Tx}+\mathcal L_{m-1}+\frac{d_{m-1}}2\\
  m\text{列目:}&\quad
  g_{\mathrm Tx}+\mathcal L_m+\frac{d_m}2
\end{align*}
%%%%%%%%%%%%%%%%%%%%%%%%%%%%%%%%%%%%%%%%%%%%%%%%%%%%%%%%%%
%% hosoku %%%%%%%%%%%%%%%%%%%%%%%%%%%%%%%%%%%%%%%%%%%%%%%%
%%%%%%%%%%%%%%%%%%%%%%%%%%%%%%%%%%%%%%%%%%%%%%%%%%%%%%%%%%
\begin{hosoku}
$d_{m-1} > d_m$のときは$m-1$列目, $d_m > d_{m-1}$のときは$m$列目をみればよい。
\end{hosoku}
%%%%%%%%%%%%%%%%%%%%%%%%%%%%%%%%%%%%%%%%%%%%%%%%%%%%%%%%%%
%%%%%%%%%%%%%%%%%%%%%%%%%%%%%%%%%%%%%%%%%%%%%%%%%%%%%%%%%%
%%%%%%%%%%%%%%%%%%%%%%%%%%%%%%%%%%%%%%%%%%%%%%%%%%%%%%%%%%
A側内面のトップ端からの($X$方向の)距離は、\TopEndACID として$w'_{\mathrm A0}$を代用すると、それぞれ
\begin{align*}
  m-1\text{列目:}&\quad
  \frac{w'_{\mathrm A0}}2-\mathcal L_{m-1}-\frac{d_{m-1}}2\\
  m\text{列目:}&\quad
  \frac{w'_{\mathrm A0}}2-\mathcal L_m-\frac{d_m}2
\end{align*}
これらのいずれか小さいほうが\index{こうぐけい@工具径}工具径(半径)よりも小さければ、\index{ワーク}ワークを傾けて加工をする必要があると判断できる
%% footnote %%%%%%%%%%%%%%%%%%%%%
\footnote{もちろん、いくらか余裕代をとる必要がある。}。
%%%%%%%%%%%%%%%%%%%%%%%%%%%%%%%%%
\vfill
%%%%%%%%%%%%%%%%%%%%%%%%%%%%%%%%%%%%%%%%%%%%%%%%%%%%%%%%%%
%% Column %%%%%%%%%%%%%%%%%%%%%%%%%%%%%%%%%%%%%%%%%%%%%%%%
%%%%%%%%%%%%%%%%%%%%%%%%%%%%%%%%%%%%%%%%%%%%%%%%%%%%%%%%%%
\begin{\Columnname}{\BfaceDimpleMilling, \DfaceDimpleMilling での考慮点}
\paragraph*{工具径とシャンク径}
\index{アンダーカット}アンダーカットが生じるのは主に\index{トップがわのAがわうちたんてん@トップ側のA側内端点}(A側内面の)トップ端なので、実際には\index{こうぐけい@工具径}工具径(工具の切削する部分)ではなく\index{シャンクけい(Tスロットカッター)@シャンク径(Tスロットカッター)}シャンク径等(工具のトップ端に相当する箇所)でよい。
そのため工具径よりシャンク径のほうが小さい場合は、より広い範囲の\BfaceDimpleMilling, \DfaceDimpleMilling を傾けずに切削することが可能となる。
\tcbline*
\paragraph*{\expandafterindex{けずりしろ(\yomiEndFacecut)@削り代(\nameEndFacecut)}\nameEndFacecut の削り代}
\Dimple の測定・加工は、\TopEndFacecutMilling の前に行う。
そのため測定・加工の際は、\expandafterindex{ぜんけずりしろ(\yomiTopEndFacecutMilling)@全削り代(\nameTopEndFacecutMilling)}\nameEndFacecut の全削り代の分だけ大きい(長い)ことに注意する必要がある。
削り代の分だけ\index{わんきょく@湾曲}湾曲も加味する必要があり、特に\index{Aがわないめん(トップがわ)@A側内面(トップ側)}A側内面と工具とのアンダーカットに留意しなければならない。
\tcbline*
\paragraph*{その他のずれ}
\index{ワークのけいじょう@ワークの形状}ワークの形状は当然ながら\index{ずめん(モールド)@図面(モールド)}図面のものとは一致はしない。
特に\index{わんきょく@湾曲}湾曲や\index{にくあつ@肉厚}肉厚などの図面とのずれは、アンダーカットに大きく寄与するのでこれも注意する必要がある。
\end{\Columnname}
%%%%%%%%%%%%%%%%%%%%%%%%%%%%%%%%%%%%%%%%%%%%%%%%%%%%%%%%%%
%%%%%%%%%%%%%%%%%%%%%%%%%%%%%%%%%%%%%%%%%%%%%%%%%%%%%%%%%%
%%%%%%%%%%%%%%%%%%%%%%%%%%%%%%%%%%%%%%%%%%%%%%%%%%%%%%%%%%



\clearpage
%%%%%%%%%%%%%%%%%%%%%%%%%%%%%%%%%%%%%%%%%%%%%%%%%%%%%%%%%%
%% section 7.4 %%%%%%%%%%%%%%%%%%%%%%%%%%%%%%%%%%%%%%%%%%%
%%%%%%%%%%%%%%%%%%%%%%%%%%%%%%%%%%%%%%%%%%%%%%%%%%%%%%%%%%
\modHeadsection{傾き後の\Dimple}
機内での回転は\TableCenter Pを\index{げんてんP@原点P}原点として行われる。
また\DimpleMilling は\TopIDCenter を基準にして切削を行う。
傾ける前の\TopIDCenter $g_\mathrm T$の座標は実測により(Pを中心とした$XYZ$直交座標でいうところの)[$g_{\mathrm Tx}$, $g_{\mathrm Ty}$, $f_\mathrm T'$]で与えられる
%% footnote %%%%%%%%%%%%%%%%%%%%%
\footnote{ここではこれを\TableCenter Pを原点とした座標値として取り扱っている。
しかし、測定では機械座標系の値として$g_\mathrm T$が与えられる。
たとえば\Dimple の場合、$g_\mathrm T$は通常(今の場合は\TableCenter Pより負側に\CurvatureCenter があることが多いので)負の値として得られることに注意。
(ここでの測定では$XY$成分のみであり、$Z$については測定しないことにも注意。)}。
%%%%%%%%%%%%%%%%%%%%%%%%%%%%%%%%%
このとき、テーブルを角度$-\phi$だけ傾けた後のトップ端内面中心の座標$g'_\mathrm T$は
%% footnote %%%%%%%%%%%%%%%%%%%%%
\footnote{これらをワーク座標原点としてもよいし、ワーク座標原点$g_\mathrm T$はそのままで各面ごとに傾けてもよい。
ここでは後者の方法で加工を行うものとする。}、
%%%%%%%%%%%%%%%%%%%%%%%%%%%%%%%%%
%% label{eq:afterPhiTCenterFromO}
\begin{align}
  \label{eq:afterPhiTCenterFromO}
  \left[
  \begin{array}{c}
    g_{\mathrm Tx}'\\
    g_{\mathrm Ty}'\\
    g_{\mathrm Tz}'
  \end{array}
  \right]
  =\left[
   \begin{array}{c}
     g_{\mathrm Tx}\cos\phi+f_\mathrm T'\sin\phi\\
     g_{\mathrm Ty}\\
     -g_{\mathrm Tx}\sin\phi+f_\mathrm T'\cos\phi
   \end{array}
   \right].
   \end{align}
同様に、$i$列目における(傾ける前の)\CurvatureCenter の位置は、[$g_{\mathrm Tx}+\mathcal L_i$, $g_{\mathrm Ty}$, $f_\mathrm T'-q-(i-1)p_z$]で与えられる
%% footnote %%%%%%%%%%%%%%%%%%%%%
\footnote{ここでは\TopCurvatureCenter を、\TopIDCenter と同一視している。}
%%%%%%%%%%%%%%%%%%%%%%%%%%%%%%%%%
ので、テーブルを角度$-\phi$だけ傾けた後の$i$列目における\CurvatureCenter の位置は、
\begin{align*}
  \left[
  \begin{array}{c}
    (g_{\mathrm Tx}+\mathcal L_i)\cos\phi+\{f_\mathrm T'-q-(i-1)p_z\}\sin\phi\\
    g_{\mathrm Ty}\\
    -(g_{\mathrm Tx}+\mathcal L_i)\sin\phi+\{f_\mathrm T'-q-(i-1)p_z\}\cos\phi
  \end{array}
  \right].
\end{align*}
したがって、傾けた後の\TopCurvatureCenter と$i$列目に対する\CurvatureCenter との差分は、
%% label{eq:afterPhidimpleCenterDistance}
\begin{align}
  \label{eq:afterPhidimpleCenterDistance}
  \left[
  \begin{array}{c}
    \mathcal L_i\cos\phi-\{q+(i-1)p_z\}\sin\phi\\
    0\\
    -\mathcal L_i\sin\phi-\{q+(i-1)p_z\}\cos\phi
  \end{array}
  \right].
\end{align}
%%%%%%%%%%%%%%%%%%%%%%%%%%%%%%%%%%%%%%%%%%%%%%%%%%%%%%%%%%
%% hosoku %%%%%%%%%%%%%%%%%%%%%%%%%%%%%%%%%%%%%%%%%%%%%%%%
%%%%%%%%%%%%%%%%%%%%%%%%%%%%%%%%%%%%%%%%%%%%%%%%%%%%%%%%%%
\begin{hosoku}
傾けた後の$i$列目に対する\CurvatureCenter と$j$列目に対する\CurvatureCenter との差分は、
\begin{align*}
  \left[
  \begin{array}{c}
    \mathcal L_{j,i}\cos\phi-(j-i)p_z\sin\phi\\
    0\\
    -\mathcal L_{j,i}\sin\phi-(j-i)p_z\cos\phi
  \end{array}
  \right].
\end{align*}
特に、$j = i+1$の場合は、
\begin{align*}
  \left[
  \begin{array}{c}
    \mathcal L_{i+1,i}\cos\phi-p_z\sin\phi\\
    0\\
    -\mathcal L_{i+1,i}\sin\phi-p_z\cos\phi
  \end{array}
  \right].
\end{align*}
\end{hosoku}
%%%%%%%%%%%%%%%%%%%%%%%%%%%%%%%%%%%%%%%%%%%%%%%%%%%%%%%%%%
%%%%%%%%%%%%%%%%%%%%%%%%%%%%%%%%%%%%%%%%%%%%%%%%%%%%%%%%%%
%%%%%%%%%%%%%%%%%%%%%%%%%%%%%%%%%%%%%%%%%%%%%%%%%%%%%%%%%%

\clearpage
%%%%%%%%%%%%%%%%%%%%%%%%%%%%%%%%%%%%%%%%%%%%%%%%%%%%%%%%%%
%% Column %%%%%%%%%%%%%%%%%%%%%%%%%%%%%%%%%%%%%%%%%%%%%%%%
%%%%%%%%%%%%%%%%%%%%%%%%%%%%%%%%%%%%%%%%%%%%%%%%%%%%%%%%%%
\begin{\Columnname}{タッチセンサープローブ径の考慮:$XY$と$Z$方向の非対称性}
マシニングセンタ内の測定では\index{タッチセンサープローブ}タッチセンサープローブを用いる。
そのため、\index{タッチセンサープローブせんたんきゅう@タッチセンサープローブ先端球}タッチセンサープローブ先端球の径の大きさに対して考慮・補正しなければならない。
タッチセンサープローブの位置の基準については、以下のようにとるのが通常である。
\begin{enumerate}
\item $X$方向:基準はタッチセンサープローブ先端球の($X$方向の)中心
\item $Y$方向:基準はタッチセンサープローブ先端球の($Y$方向の)中心
\item $Z$方向:基準はタッチセンサープローブ先端球の($Z$方向の)先端
\end{enumerate}
したがって、$XY$方向と$Z$方向とでは\index{きじゅんてん@基準点(タッチセンサープローブ)}基準点が異なり非対称となっている。
今の場合、基準が非対称な$X$と$Z$が混合する移動(回転)であるが、あくまでもタッチセンサープローブの先端(上記の基準点)が回転後の位置にある、ということである。
そのため補正については(傾きに関係なく)$Z$方向に対してのみ径の半分だけ補正すればよい。
\end{\Columnname}
%%%%%%%%%%%%%%%%%%%%%%%%%%%%%%%%%%%%%%%%%%%%%%%%%%%%%%%%%%
%%%%%%%%%%%%%%%%%%%%%%%%%%%%%%%%%%%%%%%%%%%%%%%%%%%%%%%%%%
%%%%%%%%%%%%%%%%%%%%%%%%%%%%%%%%%%%%%%%%%%%%%%%%%%%%%%%%%%


%%%%%%%%%%%%%%%%%%%%%%%%%%%%%%%%%%%%%%%%%%%%%%%%%%%%%%%%%%
%% subsection 5.4.1 %%%%%%%%%%%%%%%%%%%%%%%%%%%%%%%%%%%%%%
%%%%%%%%%%%%%%%%%%%%%%%%%%%%%%%%%%%%%%%%%%%%%%%%%%%%%%%%%%
\subsection{傾き後の\AfaceDimple, \CfaceDimple}
\expandafterindex{かたむきかく(\yomiDimple)@傾き角(\nameDimple)}傾ける角度$\phi$は\pageeqref{eq:dKatamuki}で与えられる。
このとき、傾けた後の\AFaceDimpleIRowJ および\CFaceDimpleIRowJ の位置は、\pageeqref{eq:dPosXBefore}, \eqref{eq:dPosYBefore}, \pageeqref{eq:dPosZBefore}より、
\begin{alignat*}{3}
  \text{A面:}&~~&
  \left[
  \begin{array}{c}
    \mathcal D_{xij,\mathrm A}'\\
    \mathcal D_{yij,\mathrm A}'\\
    \mathcal D_{zij,\mathrm A}'
  \end{array}
  \right]
 &= \left[
    \begin{array}{c}
      \mathcal D_{xi,\mathrm A}\cos\phi+\mathcal D_{zi}\sin\phi\\
      \mathcal D_{yij,\mathrm A}\\
      -\mathcal D_{xi,\mathrm A}\sin\phi+\mathcal D_{zi}\cos\phi
    \end{array}
    \right],\\[2pt]
  \text{C面:}&~~&
  \left[
  \begin{array}{c}
    \mathcal D_{xij,\mathrm C}'\\
    \mathcal D_{yij,\mathrm C}'\\
    \mathcal D_{zij,\mathrm C}'
  \end{array}
  \right]
 &= \left[
    \begin{array}{c}
      \mathcal D_{xi,\mathrm C}\cos\phi+\mathcal D_{zi}\sin\phi\\
      \mathcal D_{yij,\mathrm A}\\
      -\mathcal D_{xi,\mathrm C}\sin\phi+\mathcal D_{zi}\cos\phi
    \end{array}
    \right].
\end{alignat*}
特に、各列の中央(各列の\expandafterindex{\yomiCurvatureCenter(\yomiDimple)@\nameCurvatureCenter(\nameDimple)}\nameCurvatureCenter)$[g_{\mathrm Tx}+\mathcal L_i, g_{\mathrm Ty}, f_\mathrm T'-q-(i-1)p_z]$を原点としてみた場合の位置は、
\begin{align*}
  \left[
  \begin{array}{c}
    \displaystyle \pm\frac{w_{Aq+(i-1)p_z}'}2\cos\phi\\[6pt]
    \displaystyle -\frac{d_i}2+(j-1)p_x\\[6pt]
    \displaystyle \mp\frac{w_{Aq+(i-1)p_z}'}2\sin\phi
  \end{array}
  \right]\qquad
  %%%%%%%%
  \left(
  \text{複号}
  \left\{
  \begin{array}{rl}
    \!\text{上}\!\!\!\!& \text{: A面}\\
    \!\text{下}\!\!\!\!& \text{: C面}\\
  \end{array}
  \right.
  \right).
\end{align*}





\paragraph*{$j$方向の差分}
$Y$方向の隣同士の差分、すなわち$i$を固定したときの$j$番目と$j+1$番目の位置の差分は、
\begin{align*}
  \left[
  \begin{array}{c}
    0\\
    \mathcal D_{yi(j+1),\mathrm A}-\mathcal D_{yij,\mathrm A}\\
    0
  \end{array}
  \right]
  = \left[
    \begin{array}{c}
      0\\
      p_x\\
      0
    \end{array}
    \right]\ .
\end{align*}


\clearpage
\paragraph*{$i$方向の差分}
$Z$方向の隣同士の差分、すなわち$j$を固定したときの$i$番目と$i+1$番目の位置の差分については、
\begin{align*}
 &\left[
  \begin{array}{c}
    (\mathcal D_{x(i+1),\mathrm A}-\mathcal D_{xi,\mathrm A})\cos\phi
    +(\mathcal D_{z(i+1)}-\mathcal D_{zi})\sin\phi\\
    (\mathcal D_{y(i+1)j,\mathrm A}-\mathcal D_{yij,\mathrm A})\\
    (\mathcal D_{xi,\mathrm A}-\mathcal D_{x(i+1),\mathrm A})\sin\phi
    +(\mathcal D_{z(i+1)}-\mathcal D_{zi})\cos\phi
  \end{array}
  \right]\\
 &= \left[
    \begin{array}{c}
      \displaystyle
      \left(\mathcal L_{i+1, i}+\frac{w'_{\mathrm Aq+ip_z}-w'_{\mathrm Aq+(i-1)p_z}}2\right)\cos\phi
      -p_z\sin\phi\\[6pt]
      \displaystyle-\frac{d_{i+1}-d_i}2\\[6pt]
      \displaystyle
      -\left(\mathcal L_{i+1, i}+\frac{w'_{\mathrm Aq+ip_z}-w'_{\mathrm Aq+(i-1)p_z}}2\right)\sin\phi
      -p_z\cos\phi
    \end{array}
    \right]\ .
\end{align*}
C面に対しては、これの各々の\InnerDiameter$w_\mathrm A'$の符号を入れ換えたものとなる。
%%%%%%%%%%%%%%%%%%%%%%%%%%%%%%%%%%%%%%%%%%%%%%%%%%%%%%%%%%
%% hosoku %%%%%%%%%%%%%%%%%%%%%%%%%%%%%%%%%%%%%%%%%%%%%%%%
%%%%%%%%%%%%%%%%%%%%%%%%%%%%%%%%%%%%%%%%%%%%%%%%%%%%%%%%%%
\begin{hosoku}
$X$成分の差分の大きさが($\mathcal L_{i+1, i}$からみて)A面(の$\cos\phi$成分)のそれと同じであることがわかる。
これは\HorizontalID を$w_{\mathrm A\lambda}$等で代用したからであり、実際の長さは異なる(振分中心を除いて対称ではなく、C側のほうが長い)ことに注意。
\end{hosoku}
%%%%%%%%%%%%%%%%%%%%%%%%%%%%%%%%%%%%%%%%%%%%%%%%%%%%%%%%%%
%%%%%%%%%%%%%%%%%%%%%%%%%%%%%%%%%%%%%%%%%%%%%%%%%%%%%%%%%%
%%%%%%%%%%%%%%%%%%%%%%%%%%%%%%%%%%%%%%%%%%%%%%%%%%%%%%%%%%
%%%%%%%%%%%%%%%%%%%%%%%%%%%%%%%%%%%%%%%%%%%%%%%%%%%%%%%%%%
%% hosoku %%%%%%%%%%%%%%%%%%%%%%%%%%%%%%%%%%%%%%%%%%%%%%%%
%%%%%%%%%%%%%%%%%%%%%%%%%%%%%%%%%%%%%%%%%%%%%%%%%%%%%%%%%%
\begin{hosoku}[label=hosoku:generallyDimpleN]
\pageautoref{fn:generallyDimpleN}でも述べたように、たいていの場合は$|d_{i+1}-d_i|=p_x$であり、また$d_{i+2} = d_i$である。
\end{hosoku}
%%%%%%%%%%%%%%%%%%%%%%%%%%%%%%%%%%%%%%%%%%%%%%%%%%%%%%%%%%
%%%%%%%%%%%%%%%%%%%%%%%%%%%%%%%%%%%%%%%%%%%%%%%%%%%%%%%%%%
%%%%%%%%%%%%%%%%%%%%%%%%%%%%%%%%%%%%%%%%%%%%%%%%%%%%%%%%%%


%%%%%%%%%%%%%%%%%%%%%%%%%%%%%%%%%%%%%%%%%%%%%%%%%%%%%%%%%%
%% subsection 5.4.2 %%%%%%%%%%%%%%%%%%%%%%%%%%%%%%%%%%%%%%
%%%%%%%%%%%%%%%%%%%%%%%%%%%%%%%%%%%%%%%%%%%%%%%%%%%%%%%%%%
\subsection{傾き後の\BfaceDimple, \DfaceDimple}
傾けた後の\BFaceDimpleIRowJ および\DFaceDimpleIRowJ の位置は、A面側のときと同様に、
\begin{alignat*}{3}
  \text{B面:}&~~&
  \left[
    \begin{array}{c}
      \mathcal D_{xij,\mathrm B}'\\
      \mathcal D_{yij,\mathrm B}'\\
      \mathcal D_{zij,\mathrm B}'
    \end{array}
  \right]
 &= \left[
    \begin{array}{c}
      \mathcal D_{xij,\mathrm B}\cos\phi+\mathcal D_{zi}\sin\phi\\
      \mathcal D_{yi,\mathrm B}\\
      -\mathcal D_{xij,\mathrm B}\sin\phi+\mathcal D_{zi}\cos\phi
    \end{array}
    \right],\\[2pt]
  \text{D面:}&~~&
  \left[
    \begin{array}{c}
      \mathcal D_{xij,\mathrm D}'\\
      \mathcal D_{yij,\mathrm D}'\\
      \mathcal D_{zij,\mathrm D}'
    \end{array}
  \right]
 &= \left[
    \begin{array}{c}
      \mathcal D_{xij,\mathrm B}\cos\phi+\mathcal D_{zi}\sin\phi\\
      \mathcal D_{yi,\mathrm D}\\
      -\mathcal D_{xij,\mathrm B}\sin\phi+\mathcal D_{zi}\cos\phi
    \end{array}
    \right].
\end{alignat*}
特に、各列の中央(各列の\CurvatureCenter)$[g_{\mathrm Tx}+\mathcal L_i, g_{\mathrm Ty}, f_\mathrm T'-q-(i-1)p_z]$を原点としてみた場合の位置は、
\begin{align*}
  \left[
  \begin{array}{c}
    \displaystyle \left\{\frac{d_i}2-(j-1)p_z\right\}\cos\phi\\
    \displaystyle \pm\frac{w_{\mathrm Bq+(i-1)p_z}'}2\\
    \displaystyle -\left\{\frac{d_i}2-(j-1)p_z\right\}\sin\phi
  \end{array}
  \right]\qquad
  %%%%%%%%
  \left(
  \text{複号}
  \left\{
  \begin{array}{rl}
    \!+\!\!\!\!& \text{: B面}\\
    \!-\!\!\!\!& \text{: D面}\\
  \end{array}
  \right.
  \right).
\end{align*}

\clearpage
\paragraph*{$j$方向の差分}\noindent
$Y$方向の隣同士の差分、すなわち$i$を固定したときの$j$番目と$j+1$番目の位置の差分は、
\begin{align*}
  \left[
  \begin{array}{c}
    \left(\mathcal D_{xi(j+1),\mathrm B}-\mathcal D_{xij,\mathrm B}\right)\cos\phi\\
    0\\
    -\left(\mathcal D_{xi(j+1),\mathrm B}-\mathcal D_{xij,\mathrm B}\right)\sin\phi
  \end{array}
  \right]
  = \left[
    \begin{array}{c}
      -p_x\cos\phi\\[6pt]
      0\\
      p_x\sin\phi
    \end{array}
    \right]\ .
\end{align*}

\paragraph*{$i$方向の差分}\noindent
B面に対する$Z$方向の隣同士の差分、すなわち$j$を固定したときの$i$番目と$i+1$番目の位置の差分については、
\begin{align*}
 &\left[
  \begin{array}{c}
    \left(\mathcal D_{x(i+1)j,\mathrm B}-\mathcal D_{xij,\mathrm B}\right)\cos\phi
    +\left(\mathcal D_{z(i+1)}-\mathcal D_{zi}\right)\sin\phi\\[3pt]
    \mathcal D_{yi+1,\mathrm B}-\mathcal D_{yi,\mathrm B}\\[3pt]
    -\left(\mathcal D_{x(i+1)j,\mathrm B}-\mathcal D_{xij,\mathrm B}\right)\sin\phi
    +\left(\mathcal D_{z(i+1)}-\mathcal D_{zi}\right)\cos\phi
  \end{array}
  \right]\\
 &= \left[
    \begin{array}{c}
      \displaystyle\left(\mathcal L_{i+1, i}+\frac{d_{i+1}-d_i}2\right)\cos\phi-p_z\sin\phi\\[10pt]
      \displaystyle\frac{w'_{\mathrm Bq+ip_z}-w'_{\mathrm Bq+(i-1)p_z}}2\\[8pt]
      \displaystyle-\left(\mathcal L_{i+1, i}+\frac{d_{i+1}-d_i}2\right)\cos\phi-p_z\cos\phi
    \end{array}
    \right]\ .
\end{align*}
D面に対しては、これの各々の\InnerDiameter$w_\mathrm B'$の符号(この場合$Y$成分の符号)を入れ換えたものとなる。
