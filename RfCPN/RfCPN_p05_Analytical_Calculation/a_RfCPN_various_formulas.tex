%!TEX root = ../RfCPN.tex


\modHeadchapter{諸公式}



%%%%%%%%%%%%%%%%%%%%%%%%%%%%%%%%%%%%%%%%%%%%%%%%%%%%%%%%%%
%% section F.1 %%%%%%%%%%%%%%%%%%%%%%%%%%%%%%%%%%%%%%%%%%%
%%%%%%%%%%%%%%%%%%%%%%%%%%%%%%%%%%%%%%%%%%%%%%%%%%%%%%%%%%
\modHeadsection{近似計算}
\begin{Formula}[label=formula:taylorexpansion]{テイラー展開(マクローリン展開)}
$f(x)$の$x = 0$に対する\index{テイラーてんかい@テイラー展開}テイラー展開(\index{マクローリンてんかい@マクローリン展開}マクローリン展開)は、以下で与えられる。
\begin{align*}
  f(x) = \sum_{n=1}^{\infty}\frac{f^{(n)}(0)}{n!}x^n\ .
\end{align*}
\tcbline*
特に、
\begin{align*}
  (1+x)^\frac12 &= 1+\frac x2-\frac{x^2}8+\frac{x^3}{16}-\frac{5x^4}{128}+o\ab(x^5)\ ,\\
  \frac{\cos x}{1+\cos^2x} &= \frac12-\frac{x^4}{16}+o\ab(x^6)\ ,\\
  \frac{\sin x\cos x}{1+\cos^2x} &= \frac x2-\frac{x^3}{12}-\frac{7x^5}{120}+o\ab(x^7)\ .
\end{align*}
\end{Formula}



%%%%%%%%%%%%%%%%%%%%%%%%%%%%%%%%%%%%%%%%%%%%%%%%%%%%%%%%%%
%% section F.02 %%%%%%%%%%%%%%%%%%%%%%%%%%%%%%%%%%%%%%%%%%
%%%%%%%%%%%%%%%%%%%%%%%%%%%%%%%%%%%%%%%%%%%%%%%%%%%%%%%%%%
\modHeadsection{2直線の関係}
\begin{Formula}[label=formula:intersectionof2lines]{2直線の交点}
2つの直線$a_1x+b_1y+c_1 = 0$, $a_2x+b_2y+c_2 = 0$の交点は、以下で与えられる。
\begin{align*}
  \ab(
  \frac{b_1c_2-b_2c_1}{a_1b_2-a_2b_1}~,~
  \frac{a_2c_1-a_1c_2}{a_1b_2-a_2b_1}
  )\ .
\end{align*}
\end{Formula}

\begin{Formula}[label=formula:anblebetween2lines]{2直線のなす角度}
2つの直線$y = m_1x$, $y = m_2x$ ($m_1m_2 \ne -1$)に対して、$m_1 = \tan\alpha$, $m_2 = \tan\beta$とみなすことができる。
したがって、2直線のなす角度$\phi$ ($0 < \phi < \nicefrac\pi2$)は、以下で与えられる。
\begin{align*}
  \tan\phi = \tan(\beta-\alpha) = \frac{m_2-m_1}{1+m_1m_2}\ .
\end{align*}
\end{Formula}



%\clearpage
%%%%%%%%%%%%%%%%%%%%%%%%%%%%%%%%%%%%%%%%%%%%%%%%%%%%%%%%%%%
%%% section F.2 %%%%%%%%%%%%%%%%%%%%%%%%%%%%%%%%%%%%%%%%%%%
%%%%%%%%%%%%%%%%%%%%%%%%%%%%%%%%%%%%%%%%%%%%%%%%%%%%%%%%%%%
%\modHeadsection{2点間の距離}
%\begin{Formula}{点と直線間の距離}
%点($p$, $q$)と直線$ax+by+c=0$との距離$d$は、以下で与えられる。
%\begin{align*}
%  d = \frac{|ap+bq+c|}{\sqrt{a^2+b^2}}.
%\end{align*}
%\end{Formula}
%\begin{Formula}{直線上の点と直線間の距離}
%点$\boldsymbol p$を通り方向ベクトルが$\boldsymbol m$の直線L上の点と、点$\boldsymbol q$を通り方向ベクトルが$\boldsymbol m'$の直線$\mathrm L'$上の点は、それぞれパラメタ$t$, $t'$を用いて、
%\begin{align*}
%  \mathrm L: \boldsymbol p+t\boldsymbol m\ , \qquad
%  \mathrm L': \boldsymbol q+t'\boldsymbol m'
%\end{align*}
%で表される。
%このとき、L上の点の中で$\mathrm L'$に最も近づく点の位置$\boldsymbol k$は、以下で与えられる
%%% footnote %%%%%%%%%%%%%%%%%%%%%
%\footnote{2点間の距離の2乗$|\boldsymbol p-\boldsymbol q+t\boldsymbol m-t'\boldsymbol m'|^2$に対し、それぞれのパラメタ$t$, $t'$に関する微分が0となる。
%それらを連立して解けば$\boldsymbol k$, $\boldsymbol k'$が求まる。}。
%%%%%%%%%%%%%%%%%%%%%%%%%%%%%%%%%%
%$\mathrm L'$上の点の中でLに最も近づく点の位置$\boldsymbol k'$についても同様である。
%\begin{align*}
%  \boldsymbol k
%  = \boldsymbol p
%    +\frac{\ab\{\boldsymbol m-(\boldsymbol m, \boldsymbol m')\boldsymbol m', \boldsymbol p-\boldsymbol q\}}
%          {1+\ab(\boldsymbol m, \boldsymbol m')^2}\boldsymbol m
%\end{align*}
%また、これらの差の大きさ$\big|\boldsymbol k-\boldsymbol k'\big|$から、2直線間の距離$d$が求まる。
%\end{Formula}
