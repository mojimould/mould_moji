%!TEX root = ../RfCPN.tex


\modHeadchapter{\EndFaceCChamfer の幾何}
ここでは主に、\index{テーパエンドミル}テーパエンドミルを用いた、\textbf{\EndFaceOutCChamfer}および\textbf{\EndFaceInCChamfer}%
%% footnote %%%%%%%%%%%%%%%%%%%%%
\footnote{一般に、\index{Cめんとり@C面取}C面取というと、角度$45^\circ$のものを指すことが多い。
しかしここでは、$45^\circ$以外のもの(たとえば$15^\circ$, $30^\circ$等)も含めて\index{Cめんとり@C面取}C面取と称している。}
%%%%%%%%%%%%%%%%%%%%%%%%%%%%%%%%%
に関する測定・加工に必要な\expandafterindex{きかてきせいしつ(\yomiEndFaceCChamfer)@幾何的性質(\nameEndFaceCChamfer)}幾何的性質を考える。



%%%%%%%%%%%%%%%%%%%%%%%%%%%%%%%%%%%%%%%%%%%%%%%%%%%%%%%%%%
%% section 23.1 %%%%%%%%%%%%%%%%%%%%%%%%%%%%%%%%%%%%%%%%%%
%%%%%%%%%%%%%%%%%%%%%%%%%%%%%%%%%%%%%%%%%%%%%%%%%%%%%%%%%%
\modHeadsection{\EndFaceCChamfer の寸法}
\TopEndFaceOutCChamferLength および\TopEndFaceInCChamferLength をそれぞれ$c_\mathrm{To}$, $c_\mathrm{Ti}$とする。
ここで$c_\mathrm{To}$や$c_\mathrm{Ti}$は、\EndFace に垂直な方向の距離とする。
このとき、片角$\xi_\mathrm e$\,($>0$)の\TaperEndMill に対して、$XY$方向の\EndFaceCChamferLength は、
\begin{align*}
  c_\mathrm{To}\tan\xi_\mathrm e\qquad\text{および}\qquad c_\mathrm{Ti}\tan\xi_\mathrm e
\end{align*}
で与えられる。
\BottomEndFaceOutCChamferLength および\TopEndFaceInCChamferLength$c_\mathrm{Bo}$, $c_\mathrm{Bi}$についても同様である。



%%%%%%%%%%%%%%%%%%%%%%%%%%%%%%%%%%%%%%%%%%%%%%%%%%%%%%%%%%
%% section 23.2 %%%%%%%%%%%%%%%%%%%%%%%%%%%%%%%%%%%%%%%%%%
%%%%%%%%%%%%%%%%%%%%%%%%%%%%%%%%%%%%%%%%%%%%%%%%%%%%%%%%%%
\modHeadsection{\nameTaperEndMill の参照直径}
\TaperEndMill は、その名の通り\expandafterindex{テーパ(\yomiTaperEndMill)@テーパ(\nameTaperEndMill)}テーパの付いた工具であり、先端が平坦になっているものも多い。
しかし、先端部分を\ToolLength として設定すると、その部分が段差となり\index{テーパかこう@テーパ加工}テーパ加工を適切に行うことができない。
そのため、先端部分から一定の距離$d_\mathrm e$だけずらした箇所を\indexTLTaperEndMill\nameToolLength として設定し、またその箇所の直径(\textbf{\TaperEndMillReferenceDiameter})$D_\mathrm r$を\expandafterindex{こうぐちょっけい(\yomiTaperEndMill)@工具直径(\nameTaperEndMill)}工具直径として補正を行うことが推奨される。

ここで、\expandafterindex{\yomiTaperEndMill せんたんけい@\nameTaperEndMill 先端径}\nameTaperEndMill 先端径(直径)
%% footnote %%%%%%%%%%%%%%%%%%%%%
\footnote{先端が平坦でなく尖っている場合は$D_\mathrm e = 0$とみなす。}
%%%%%%%%%%%%%%%%%%%%%%%%%%%%%%%%%
および\TaperEndMillAngle をそれぞれ$D_\mathrm e$, $\xi_\mathrm e$とすると、\TaperEndMillReferenceDiameter$D_\mathrm r$は
\begin{align*}
  D_\mathrm r = D_\mathrm e+2d_\mathrm e\tan\xi_\mathrm e
\end{align*}
で与えられる。
通常、\TDCorrection は工具の半径を用いて行うので、\indexTDTaperEndMill\nameToolDiameter を
\begin{align*}
  \frac{D_\mathrm r}2 = \frac{D_\mathrm e}2+d_\mathrm e\tan\xi_\mathrm e
\end{align*}
として設定すればよい。
あるいは、\expandafterindex{\yomiTaperEndMill せんたんけい@\nameTaperEndMill 先端径}\nameTaperEndMill 先端径を基準としてそこから補正を行う形にする場合は、その差
\begin{align*}
  \frac{D_\mathrm r}2-\frac{D_\mathrm e}2 = d_\mathrm e\tan\xi_\mathrm e
\end{align*}
だけ補正すればよい。
%%%%%%%%%%%%%%%%%%%%%%%%%%%%%%%%%%%%%%%%%%%%%%%%%%%%%%%%%%
%% hosoku %%%%%%%%%%%%%%%%%%%%%%%%%%%%%%%%%%%%%%%%%%%%%%%%
%%%%%%%%%%%%%%%%%%%%%%%%%%%%%%%%%%%%%%%%%%%%%%%%%%%%%%%%%%
\begin{hosoku}
なお、$\xi_\mathrm e = \nicefrac\pi{12}$\,($15^\circ$), $\nicefrac\pi6$\,($30^\circ$), $\nicefrac\pi4$\,($45^\circ$)のとき、それぞれ
\begin{align*}
  \tan\frac\pi{12} = 2-\sqrt3\ , \quad
  \tan\frac\pi6 = \frac1{\sqrt3}\ , \quad
  \tan\frac\pi4 = 1\ .
\end{align*}
\end{hosoku}
%%%%%%%%%%%%%%%%%%%%%%%%%%%%%%%%%%%%%%%%%%%%%%%%%%%%%%%%%%
%%%%%%%%%%%%%%%%%%%%%%%%%%%%%%%%%%%%%%%%%%%%%%%%%%%%%%%%%%
%%%%%%%%%%%%%%%%%%%%%%%%%%%%%%%%%%%%%%%%%%%%%%%%%%%%%%%%%%



%\clearpage
%%%%%%%%%%%%%%%%%%%%%%%%%%%%%%%%%%%%%%%%%%%%%%%%%%%%%%%%%%
%% section 23.2 %%%%%%%%%%%%%%%%%%%%%%%%%%%%%%%%%%%%%%%%%%
%%%%%%%%%%%%%%%%%%%%%%%%%%%%%%%%%%%%%%%%%%%%%%%%%%%%%%%%%%
\modHeadsection{中心座標\texorpdfstring{$X$}{X}の移動}
\Outcut があり、かつ\EndFaceOutCChamfer の場合であれば、Cの大きさに依らず加工径の中心座標($XY$)は変わらない。
一方、\Outcut のない場合は、中心の$X$座標はCの大きさに応じて\CenterCurvatureLine に沿って移動する。
このとき\EndFace と面取先端部との中心座標($X$)の差
%% footnote %%%%%%%%%%%%%%%%%%%%%
\footnote{どちらの場合も\EndFace が工具側にある場合を考えている。}
%%%%%%%%%%%%%%%%%%%%%%%%%%%%%%%%%
は、
\begin{align*}
  \text{トップ側:}&~~
  \sqrt{R_\mathrm c^2-\ab(f_\mathrm T-c_\mathrm{To})^2}-\sqrt{R_\mathrm c^2-f_\mathrm T^2}\ ,\\
  \text{ボトム側:}&~~
  \sqrt{R_\mathrm c^2-f_\mathrm B^2}-\sqrt{R_\mathrm c^2-\ab(f_\mathrm B-c_\mathrm{Bo})^2}\ .
\end{align*}
これは\EndFaceInCChamfer でも同様であり、
\begin{align*}
  \text{トップ側:}&~~
  \sqrt{R_\mathrm c^2-\ab(f_\mathrm T-c_\mathrm{Ti})^2}-\sqrt{R_\mathrm c^2-f_\mathrm T^2}\ ,\\
  \text{ボトム側:}&~~
  \sqrt{R_\mathrm c^2-f_\mathrm B^2}-\sqrt{R_\mathrm c^2-\ab(f_\mathrm B-c_\mathrm{Bi})^2}\ .
\end{align*}



%\clearpage
%%%%%%%%%%%%%%%%%%%%%%%%%%%%%%%%%%%%%%%%%%%%%%%%%%%%%%%%%%
%% section 37.04 %%%%%%%%%%%%%%%%%%%%%%%%%%%%%%%%%%%%%%%%%
%%%%%%%%%%%%%%%%%%%%%%%%%%%%%%%%%%%%%%%%%%%%%%%%%%%%%%%%%%
\modHeadsection{\EndFaceChamferLength による補正\TBW}
\EndFaceChamferMilling では、$Z$方向の\EndFaceChamferLength が揃う形になるように加工を行う。
このとき、単純に\EndFaceIDCenter を\expandafterindex{きじゅん(\yomiEndFaceChamferMilling)@基準(\nameEndFaceChamferMilling)}基準にすると、外面あるいは内面には\index{わんきょく@湾曲}湾曲があるためA面側およびC面側の\EndFaceChamferLength は揃わない
%% footnote %%%%%%%%%%%%%%%%%%%%%
\footnote{$XY$方向の\EndFaceChamferWidth が揃う形になる。}。
%%%%%%%%%%%%%%%%%%%%%%%%%%%%%%%%%
そのため、各面の\EndFaceChamferLength が揃う形になるように$X$方向に補正を入れる必要がある。



\clearpage
%%%%%%%%%%%%%%%%%%%%%%%%%%%%%%%%%%%%%%%%%%%%%%%%%%%%%%%%%%
%% section 37.05 %%%%%%%%%%%%%%%%%%%%%%%%%%%%%%%%%%%%%%%%%
%%%%%%%%%%%%%%%%%%%%%%%%%%%%%%%%%%%%%%%%%%%%%%%%%%%%%%%%%%
\modHeadsection{\EndFaceCChamferMilling の負荷}


%%%%%%%%%%%%%%%%%%%%%%%%%%%%%%%%%%%%%%%%%%%%%%%%%%%%%%%%%%
%% subsection 37.05.1 %%%%%%%%%%%%%%%%%%%%%%%%%%%%%%%%%%%%
%%%%%%%%%%%%%%%%%%%%%%%%%%%%%%%%%%%%%%%%%%%%%%%%%%%%%%%%%%
\subsection{削り代の体積}
\EndFaceChamferMilling において、\EndFaceOutChamferMilling ではC面側, \EndFaceInChamferMilling ではA面側の加工の際に、切削する体積が最も多くなる。
たとえば\BottomEndFaceInCChamfer のA面側(直線部)に着目すると、A面側内面の湾曲を$R_\mathrm i$%
%% footnote %%%%%%%%%%%%%%%%%%%%%
\footnote{簡単のため、ここではA面側内面の湾曲を$R_\mathrm i$は一定とする。},
%%%%%%%%%%%%%%%%%%%%%%%%%%%%%%%%%
\EndFaceInCChamferLength を$c_\mathrm{Bi}$, \EndFaceInCChamferAngle を$\xi_\mathrm e$とすると、単位長さあたりの全削り代(体積)は、
\begin{align*}
  \frac{c_\mathrm{Bi}^2}2\tan\xi_\mathrm e
  +\frac{k_cc_\mathrm{Bi}}2
  -\frac{R_\mathrm i^2}2(\psi_c-\sin\psi_c)\ .
\end{align*}
ここで、
\begin{align*}
  k_c = \sqrt{R_\mathrm i^2-(f_\mathrm B-c_\mathrm{Bi})^2}
        -\sqrt{R_\mathrm i^2-f_\mathrm B^2}\quad, \quad
  \sin\frac{\psi_c}2 = \frac{\sqrt{c_\mathrm{Bi}^2+k_c^2}}{2R_\mathrm i}\ .
\end{align*}
仕上げ加工の前に、径方向に$\delta_\mathrm c$\,($\leq c_\mathrm{Bi}\tan\xi_\mathrm e$)だけ残す形に加工するものとすると、仕上げ加工前の全削り代は、
\begin{align*}
  \ab(1-\frac{\delta_\mathrm e}{c_\mathrm{Bi}\tan\xi_\mathrm e+k_c})^2
\end{align*}
だけ乗じたものとなる。


%%%%%%%%%%%%%%%%%%%%%%%%%%%%%%%%%%%%%%%%%%%%%%%%%%%%%%%%%%
%% subsection 37.05.02 %%%%%%%%%%%%%%%%%%%%%%%%%%%%%%%%%%%
%%%%%%%%%%%%%%%%%%%%%%%%%%%%%%%%%%%%%%%%%%%%%%%%%%%%%%%%%%
\subsection{加工1回あたりの体積比\TBW}
仕上げ前の加工に対して、切削する体積が一定であり、かつ最後は径方向に$d_c$だけ加工するものとする。
このとき、最後から1回前の加工は、
\begin{align*}
  d_c\ab(1-\frac{d_c}{c_\mathrm{Bi}\tan\xi_\mathrm e+k_c-\delta_\mathrm e})^2
\end{align*}



\clearpage
%%%%%%%%%%%%%%%%%%%%%%%%%%%%%%%%%%%%%%%%%%%%%%%%%%%%%%%%%%
%% section 37.06 %%%%%%%%%%%%%%%%%%%%%%%%%%%%%%%%%%%%%%%%%
%%%%%%%%%%%%%%%%%%%%%%%%%%%%%%%%%%%%%%%%%%%%%%%%%%%%%%%%%%
\modHeadsection{\index{ワークのゆがみ@ワークの歪み}ワークの歪みを考慮した\EndFaceCChamferMilling}
ここまで長方形の形状の\EndFaceCChamferMilling を考えてきた。
しかし、実際には\index{ワーク}ワークに歪みがあり変形しているため、それも考慮する必要がある。
ここでは簡単のため、\index{とつしかくけい@凸四角形}凸四角形の形状をした\EndFaceCChamferMilling を考える。


%%%%%%%%%%%%%%%%%%%%%%%%%%%%%%%%%%%%%%%%%%%%%%%%%%%%%%%%%%
%% subsection 37.06.01 %%%%%%%%%%%%%%%%%%%%%%%%%%%%%%%%%%%
%%%%%%%%%%%%%%%%%%%%%%%%%%%%%%%%%%%%%%%%%%%%%%%%%%%%%%%%%%
\subsection{測定点とその座標:\BottomEndFaceOutChamfer}
まず、ボトム側について考える。
測定点として、A面側およびC面側の辺については、\BDOD の第1四分点および第3四分点の$Y$座標に対する$X$座標の測定を行う。
同様に、B面側およびD面側の辺については、\ACOD の第1四分点および第3四分点の$X$座標に対する$Y$座標の測定を行う。

%%%%%%%%%%%%%%%%%%%%%%%%%%%%%%%%%%%%%%%%%%%%%%%%%%%%%%%%%%
%% subsubsection 37.06.01.01 %%%%%%%%%%%%%%%%%%%%%%%%%%%%%
%%%%%%%%%%%%%%%%%%%%%%%%%%%%%%%%%%%%%%%%%%%%%%%%%%%%%%%%%%
\subsubsection{各々の辺の式}
A面側の辺における2測定点をそれぞれ$\mathcal A_\mathrm a$, $\mathcal A_\mathrm b$とし、その座標をそれぞれ($x_{\mathcal A_\mathrm a}$, $\nicefrac{W_y}4$), ($x_{\mathcal A_\mathrm b}$, $-\nicefrac{W_y}4$)とすると、この2点を結んだ直線は以下で与えられる。
\begin{align*}
  \frac{W_y}2x
  -\ab(x_{\mathcal A_\mathrm a}-x_{\mathcal A_\mathrm b})y
  -\frac{W_y}4\ab(x_{\mathcal A_\mathrm a}+x_{\mathcal A_\mathrm b})
  = 0\quad
  \ab(
    y = \frac12\frac{W_y}{x_{\mathcal A_\mathrm a}-x_{\mathcal A_\mathrm b}}x
        -\frac{W_y}4\frac{x_{\mathcal A_\mathrm a}+x_{\mathcal A_\mathrm b}}
                         {x_{\mathcal A_\mathrm a}-x_{\mathcal A_\mathrm b}}
  ).
\end{align*}
同様に、C面側の辺における2測定点をそれぞれ$\mathcal C_\mathrm a$, $\mathcal C_\mathrm b$とし、その座標をそれぞれ($x_{\mathcal C_\mathrm a}$, $\nicefrac{W_y}4$), ($x_{\mathcal C_\mathrm b}$, $-\nicefrac{W_y}4$)とすると、この2点を結んだ直線は以下で与えられる。
\begin{align*}
  \frac{W_y}2x
  -\ab(x_{\mathcal C_\mathrm a}-x_{\mathcal C_\mathrm b})y
  -\frac{W_y}4\ab(x_{\mathcal C_\mathrm a}+x_{\mathcal C_\mathrm b})
  = 0\quad
  \ab(
    y = \frac12\frac{W_y}{x_{\mathcal C_\mathrm a}-x_{\mathcal C_\mathrm b}}x
        -\frac{W_y}4\frac{x_{\mathcal C_\mathrm a}+x_{\mathcal C_\mathrm b}}
                         {x_{\mathcal C_\mathrm a}-x_{\mathcal C_\mathrm b}}
  ).
\end{align*}
B面側の辺における2測定点をそれぞれ$\mathcal B_\mathrm r$, $\mathcal B_\mathrm l$とし、その座標をそれぞれ($\nicefrac{W_x}4$, $y_{\mathcal B_\mathrm r}$), ($-\nicefrac{W_x}4$, $y_{\mathcal B_\mathrm l}$)とすると、この2点を結んだ直線は以下で与えられる。
\begin{align*}
  \ab(y_{\mathcal B_\mathrm r}-y_{\mathcal B_\mathrm l})x
  -\frac{W_x}2y
  +\frac{W_x}4\ab(y_{\mathcal B_\mathrm r}+y_{\mathcal B_\mathrm l})
  = 0\quad
  \ab(
    y = 2\frac{y_{\mathcal B_\mathrm r}-y_{\mathcal B_\mathrm l}}{W_x}x
        +\frac{y_{\mathcal B_\mathrm r}+y_{\mathcal B_\mathrm l}}2
  ).
\end{align*}
同様に、D面側の辺における2測定点をそれぞれ$\mathcal D_\mathrm r$, $\mathcal D_\mathrm l$とし、その座標をそれぞれ($\nicefrac{W_x}4$, $y_{\mathcal D_\mathrm r}$), ($-\nicefrac{W_x}4$, $y_{\mathcal D_\mathrm l}$)とすると、この2点を結んだ直線は以下で与えられる。
\begin{align*}
  \ab(y_{\mathcal D_\mathrm r}-y_{\mathcal D_\mathrm l})x
  -\frac{W_x}2y
  +\frac{W_x}4\ab(y_{\mathcal D_\mathrm r}+y_{\mathcal D_\mathrm l})
  = 0\quad
  \ab(
    y = 2\frac{y_{\mathcal D_\mathrm r}-y_{\mathcal D_\mathrm l}}{W_x}x
        +\frac{y_{\mathcal D_\mathrm r}+y_{\mathcal D_\mathrm l}}2
  ).
\end{align*}

\clearpage
%%%%%%%%%%%%%%%%%%%%%%%%%%%%%%%%%%%%%%%%%%%%%%%%%%%%%%%%%%
%% subsubsection 37.06.01.02 %%%%%%%%%%%%%%%%%%%%%%%%%%%%%
%%%%%%%%%%%%%%%%%%%%%%%%%%%%%%%%%%%%%%%%%%%%%%%%%%%%%%%%%%
\subsubsection{各々の辺のなす角度}
\pageautoref{formula:anblebetween2lines}より、直線$\mathcal B_\mathrm l\mathcal B_\mathrm r$と直線$\mathcal A_\mathrm a\mathcal A_\mathrm b$のなす内側の角度$\theta_{\mathcal{BA}}$ ($0 < \theta_{\mathcal{BA}} < \pi$)は、以下で与えられる。
\begin{align*}
  \tan\theta_{\mathcal{BA}}
  &= \frac12
     \frac{4\ab(x_{\mathcal A_\mathrm a}-x_{\mathcal A_\mathrm b})
            \ab(y_{\mathcal B_\mathrm l}-y_{\mathcal B_\mathrm r})
           -W_xW_y}
          {W_x\ab(x_{\mathcal A_\mathrm a}-x_{\mathcal A_\mathrm b})
           -W_y\ab(y_{\mathcal B_\mathrm l}-y_{\mathcal B_\mathrm r})}\ .
\end{align*}
ただし、$W_x\ab(x_{\mathcal A_\mathrm a}-x_{\mathcal A_\mathrm b}) = W_y\ab(y_{\mathcal B_\mathrm l}-y_{\mathcal B_\mathrm r})$の場合は$\theta_{\mathcal{BA}} = \nicefrac\pi2$である。

同様に、直線$\mathcal A_\mathrm a\mathcal A_\mathrm b$と直線$\mathcal D_\mathrm l\mathcal D_\mathrm r$のなす内側の角度$\theta_{\mathcal{AD}}$ ($0 < \theta_{\mathcal{AD}} < \pi$)は、
\begin{align*}
  \tan\theta_{\mathcal{AD}}
  &= \frac12
     \frac{W_xW_y
           -4\ab(x_{\mathcal A_\mathrm a}-x_{\mathcal A_\mathrm b})
             \ab(y_{\mathcal D_\mathrm l}-y_{\mathcal D_\mathrm r})}
          {W_x\ab(x_{\mathcal A_\mathrm a}-x_{\mathcal A_\mathrm b})
           -W_y\ab(y_{\mathcal D_\mathrm l}-y_{\mathcal D_\mathrm r})}\ .
\end{align*}
ただし、$W_x\ab(x_{\mathcal A_\mathrm a}-x_{\mathcal A_\mathrm b}) = W_y\ab(y_{\mathcal D_\mathrm l}-y_{\mathcal D_\mathrm r})$の場合は$\theta_{\mathcal{AD}} = \nicefrac\pi2$.

直線$\mathcal D_\mathrm l\mathcal D_\mathrm r$と直線$\mathcal C_\mathrm a\mathcal C_\mathrm b$のなす内側の角度$\theta_{\mathcal{DC}}$ ($0 < \theta_{\mathcal{DC}} < \pi$)は、
\begin{align*}
  \tan\theta_{\mathcal{DC}}
  &= \frac12
     \frac{4\ab(x_{\mathcal C_\mathrm a}-x_{\mathcal C_\mathrm b})
            \ab(y_{\mathcal D_\mathrm l}-y_{\mathcal D_\mathrm r})
           -W_xW_y}
          {W_x\ab(x_{\mathcal C_\mathrm a}-x_{\mathcal C_\mathrm b})
           -W_y\ab(y_{\mathcal D_\mathrm l}-y_{\mathcal D_\mathrm r})}\ .
\end{align*}
ただし、$W_x\ab(x_{\mathcal C_\mathrm a}-x_{\mathcal C_\mathrm b}) = W_y\ab(y_{\mathcal D_\mathrm l}-y_{\mathcal D_\mathrm r})$の場合は$\theta_{\mathcal{DC}} = \nicefrac\pi2$.

直線$\mathcal C_\mathrm a\mathcal C_\mathrm b$と直線$\mathcal B_\mathrm l\mathcal B_\mathrm r$のなす内側の角度$\theta_{\mathcal{CB}}$ ($0 < \theta_{\mathcal{CB}} < \pi$)は、
\begin{align*}
  \tan\theta_{\mathcal{CB}}
  &= \frac12
     \frac{W_xW_y
           -4\ab(x_{\mathcal C_\mathrm a}-x_{\mathcal C_\mathrm b})
             \ab(y_{\mathcal B_\mathrm l}-y_{\mathcal B_\mathrm r})}
          {W_x\ab(x_{\mathcal C_\mathrm a}-x_{\mathcal C_\mathrm b})
           -W_y\ab(y_{\mathcal B_\mathrm l}-y_{\mathcal B_\mathrm r})}\ .
\end{align*}
ただし、$W_x\ab(x_{\mathcal C_\mathrm a}-x_{\mathcal C_\mathrm b}) = W_y\ab(y_{\mathcal B_\mathrm l}-y_{\mathcal B_\mathrm r})$の場合は$\theta_{\mathcal{CB}} = \nicefrac\pi2$.

\clearpage
%%%%%%%%%%%%%%%%%%%%%%%%%%%%%%%%%%%%%%%%%%%%%%%%%%%%%%%%%%
%% subsubsection 37.06.01.03 %%%%%%%%%%%%%%%%%%%%%%%%%%%%%
%%%%%%%%%%%%%%%%%%%%%%%%%%%%%%%%%%%%%%%%%%%%%%%%%%%%%%%%%%
\subsubsection{凸四角形の頂点の座標}
\pageautoref{formula:intersectionof2lines}より、直線$\mathcal B_\mathrm l\mathcal B_\mathrm r$と直線$\mathcal A_\mathrm a\mathcal A_\mathrm b$からなる頂点の座標($v_{\mathcal{AB}x}$, $v_{\mathcal{AB}y}$)は、
\begin{align*}
  v_{\mathcal{AB}x}
  &= \frac12\frac{
       W_xW_y\ab(x_{\mathcal A_\mathrm a}+x_{\mathcal A_\mathrm b})
       +2W_x\ab(x_{\mathcal A_\mathrm a}-x_{\mathcal A_\mathrm b})\ab(y_{\mathcal B_\mathrm r}+y_{\mathcal B_\mathrm l})
     }{
       W_xW_y
       -4\ab(y_{\mathcal B_\mathrm r}-y_{\mathcal B_\mathrm l})\ab(x_{\mathcal A_\mathrm a}-x_{\mathcal A_\mathrm b})
     }\ ,\\
  v_{\mathcal{AB}y}
  &= \frac12\frac{
       W_xW_y\ab(y_{\mathcal B_\mathrm r}+y_{\mathcal B_\mathrm l})
       +2W_y\ab(y_{\mathcal B_\mathrm r}-y_{\mathcal B_\mathrm l})\ab(x_{\mathcal A_\mathrm a}+x_{\mathcal A_\mathrm b})
     }{
       W_xW_y
       -4\ab(y_{\mathcal B_\mathrm r}-y_{\mathcal B_\mathrm l})\ab(x_{\mathcal A_\mathrm a}-x_{\mathcal A_\mathrm b})
     }\ .
\end{align*}
\begin{align*}
  \frac{b_1c_2-b_2c_1}{a_1b_2-a_2b_1}~,~
  \frac{a_2c_1-a_1c_2}{a_1b_2-a_2b_1}
\end{align*}

%\clearpage
%%%%%%%%%%%%%%%%%%%%%%%%%%%%%%%%%%%%%%%%%%%%%%%%%%%%%%%%%%
%% subsubsection 37.06.01.04 %%%%%%%%%%%%%%%%%%%%%%%%%%%%%
%%%%%%%%%%%%%%%%%%%%%%%%%%%%%%%%%%%%%%%%%%%%%%%%%%%%%%%%%%
\subsubsection{コーナーRの円の接点\TBW}
直線$\mathcal B_\mathrm l\mathcal B_\mathrm r$と直線$\mathcal A_\mathrm a\mathcal A_\mathrm b$のなすコーナーに半径$R$の円が内接している場合を考える。

(to be written...)

%\clearpage
%%%%%%%%%%%%%%%%%%%%%%%%%%%%%%%%%%%%%%%%%%%%%%%%%%%%%%%%%%
%% subsubsection 37.06.01.04 %%%%%%%%%%%%%%%%%%%%%%%%%%%%%
%%%%%%%%%%%%%%%%%%%%%%%%%%%%%%%%%%%%%%%%%%%%%%%%%%%%%%%%%%
\subsubsection{コーナーRの円の中心\TBW}
(to be written...)


%%%%%%%%%%%%%%%%%%%%%%%%%%%%%%%%%%%%%%%%%%%%%%%%%%%%%%%%%%
%% subsection 37.06.01 %%%%%%%%%%%%%%%%%%%%%%%%%%%%%%%%%%%
%%%%%%%%%%%%%%%%%%%%%%%%%%%%%%%%%%%%%%%%%%%%%%%%%%%%%%%%%%
\subsection{測定点とその座標:\EndFaceInChamfer\TBW}
(to be written...)
