%!TEX root = ../RfCPN.tex


\modHeadchapter[loColumn]{\EndFacecut および\OuterDiameter の幾何}
ここでは主に、\textbf{\EndFacecut}および\textbf{\OuterDiameter}に関する測定・加工に必要な\expandafterindex{きかてきせいしつ(\yomiEndFacecut)@幾何的性質(\nameEndFacecut)}幾何的性質を考える。
ただし\index{ワーク}ワークは、考えている側の\EndFace が工具側に向いているものとする。



%%%%%%%%%%%%%%%%%%%%%%%%%%%%%%%%%%%%%%%%%%%%%%%%%%%%%%%%%%
%% section 2.1 %%%%%%%%%%%%%%%%%%%%%%%%%%%%%%%%%%%%%%%%%%%
%%%%%%%%%%%%%%%%%%%%%%%%%%%%%%%%%%%%%%%%%%%%%%%%%%%%%%%%%%
\modHeadsection{\TopEndFace の幾何}


%%%%%%%%%%%%%%%%%%%%%%%%%%%%%%%%%%%%%%%%%%%%%%%%%%%%%%%%%%
%% subsection 2.1.1 %%%%%%%%%%%%%%%%%%%%%%%%%%%%%%%%%%%%%%
%%%%%%%%%%%%%%%%%%%%%%%%%%%%%%%%%%%%%%%%%%%%%%%%%%%%%%%%%%
\subsection{\TopCurvatureCenter の位置}

%%%%%%%%%%%%%%%%%%%%%%%%%%%%%%%%%%%%%%%%%%%%%%%%%%%%%%%%%%
%% subsubsection 2.1.1.1 %%%%%%%%%%%%%%%%%%%%%%%%%%%%%%%%%
%%%%%%%%%%%%%%%%%%%%%%%%%%%%%%%%%%%%%%%%%%%%%%%%%%%%%%%%%%
\subsubsection{\Spacer を用いた場合のT\texorpdfstring{$_{R_\mathrm c}'$}{Rc'}}
\Spacer を取付けた後の\TopCurvatureCenter の位置T$_{R_\mathrm c}'$と、\TableCenter Pとの$X$方向の差は、
\begin{align*}
  \left(
    R_\mathrm ce^{i\alpha_\mathrm c}
    -R_\mathrm i'e^{-i\alpha'_{\mathrm U_\mathrm B}}
    +R_\mathrm i'e^{-i\alpha_{\mathrm U_\mathrm B}}
  \right)
  -\Delta_x'
  = R_\mathrm ce^{i\alpha_\mathrm c}-R_\mathrm i'e^{-i\alpha'_{\mathrm U_\mathrm B}}-\Delta_x \qquad
    \left(\sin\alpha_\mathrm c = \frac{f_\mathrm T}{R_\mathrm c}\right)
\end{align*}
の実部を見ればよい。
したがって
%% footnote %%%%%%%%%%%%%%%%%%%%%
\footnote{この場合、トップ側が工具側に向いている。}、
%%%%%%%%%%%%%%%%%%%%%%%%%%%%%%%%%
%% label{eq:spacerTRc}
\begin{align}
  \notag
  &  R_\mathrm c\cos\alpha_\mathrm c-R_\mathrm i'\cos\alpha'_{\mathrm U_\mathrm B}-\Delta_x\\
  &= -\Delta_x+\sqrt{R_\mathrm c^2-f_\mathrm T^2}+\frac{\delta_\mathrm s}2
     -\sqrt{R_\mathrm i'^2-\frac{\delta_\mathrm s^2+(2\bar l)^2}4}
      \frac{2\bar l}{\sqrt{\delta_\mathrm s^2+\left(2\bar l\right)^2}}
     \label{eq:spacerTRc}
\end{align}
で与えられる
%% footnote %%%%%%%%%%%%%%%%%%%%%
\footnote{実際の作業では、この点を(\TopCurvatureCenter T$_{R_\mathrm c}\!$でなく)\OutcutCenter T$_\mathrm c$とみなすことが多い。}。
%%%%%%%%%%%%%%%%%%%%%%%%%%%%%%%%%


%%%%%%%%%%%%%%%%%%%%%%%%%%%%%%%%%%%%%%%%%%%%%%%%%%%%%%%%%%
%% subsubsection 2.1.1.2 %%%%%%%%%%%%%%%%%%%%%%%%%%%%%%%%%
%%%%%%%%%%%%%%%%%%%%%%%%%%%%%%%%%%%%%%%%%%%%%%%%%%%%%%%%%%
\subsubsection{\Table を傾けた場合のT\texorpdfstring{$_{R_\mathrm c}'$}{Rc'}}
\Table を傾けた後の\TopCurvatureCenter の位置T$_{R_\mathrm c}'$と、\TableCenter Pとの$X$方向の差は、
\begin{align*}
  \left(R_\mathrm ce^{i\alpha_\mathrm c}-\Delta_x'e^{-i\theta}+\Delta_x'\right)-\Delta_x'
  = R_\mathrm ce^{i\alpha_\mathrm c}-\Delta_x'e^{-i\theta}
\end{align*}
の実部を見ればよい。
すなわち、
%% label{eq:tableTRc}
\begin{align}
  \label{eq:tableTRc}
  R_\mathrm c\cos\alpha_\mathrm c-\Delta_x'\cos\theta
  = \sqrt{R_\mathrm c^2-f_\mathrm T^2}-\left(\Delta_x+\sqrt{R_i'^2-\bar l^2}\right)\cos\theta~.
\end{align}


\clearpage
%%%%%%%%%%%%%%%%%%%%%%%%%%%%%%%%%%%%%%%%%%%%%%%%%%%%%%%%%%
%% subsection 2.1.2 %%%%%%%%%%%%%%%%%%%%%%%%%%%%%%%%%%%%%%
%%%%%%%%%%%%%%%%%%%%%%%%%%%%%%%%%%%%%%%%%%%%%%%%%%%%%%%%%%
\subsection{\TopEndFace における外側中心の位置}
\TopCurvatureCenter T$_{R_\mathrm c}'$と\TopODCenter T$_\mathrm c'$との差は、以下で与えられる。
%% label{eq:TRc-Tc}
\begin{align}
  \label{eq:TRc-Tc}
  \sqrt{R_\mathrm c^2-f_\mathrm T^2}
  -\frac{\sqrt{R_\mathrm o^2-f_\mathrm T^2}+\sqrt{R_\mathrm i^2-f_\mathrm T^2}}2\ .
\end{align}
よって、\TopODCenter T$_\mathrm c'$の位置は、\TopCurvatureCenter T$_{R_\mathrm c}'$から\pageeqref{eq:TRc-Tc}だけ加味すればよい。
以下では、\TopODCenter T$_\mathrm c'$の位置を直接計算し、このことが整合していることを確かめる。

%%%%%%%%%%%%%%%%%%%%%%%%%%%%%%%%%%%%%%%%%%%%%%%%%%%%%%%%%%
%% subsubsection 2.1.2.1 %%%%%%%%%%%%%%%%%%%%%%%%%%%%%%%%%
%%%%%%%%%%%%%%%%%%%%%%%%%%%%%%%%%%%%%%%%%%%%%%%%%%%%%%%%%%
\subsubsection{\Spacer を用いた場合のT\texorpdfstring{$_\mathrm c'$}{c'}}
同様にして、外面A・C側のトップ端点T$_\mathrm o'$, T$_\mathrm i'$の、\TableCenter Pを原点とした場合の$X$座標はそれぞれ、
\begin{align*}
  \text{C側端点:}&
  -\Delta_x+\sqrt{R_\mathrm i^2-f_\mathrm T^2}+\frac{\delta_\mathrm s}2
  -\sqrt{R_\mathrm i'^2-\frac{\delta_\mathrm s^2+(2\bar l)^2}4}\frac{2\bar l}{\sqrt{\delta_\mathrm s^2+(2\bar l)^2}}\ ,\\
  \text{A側端点:}&
  -\Delta_x+\sqrt{R_\mathrm o^2-f_\mathrm T^2}+\frac{\delta_\mathrm s}2
  -\sqrt{R_\mathrm i'^2-\frac{\delta_\mathrm s^2+(2\bar l)^2}4}\frac{2\bar l}{\sqrt{\delta_\mathrm s^2+(2\bar l)^2}}\ .
\end{align*}
したがって、\TopODCenter T$_\mathrm c'$の$X$座標は、
%% label{eq:spacerTc}
\begin{align}
  \label{eq:spacerTc}
  -\Delta_x+\frac{\sqrt{R_\mathrm o^2-f_\mathrm T^2}+\sqrt{R_\mathrm i^2-f_\mathrm T^2}}2
  +\frac{\delta_\mathrm s}2
  -\sqrt{R_\mathrm i'^2-\frac{\delta_\mathrm s^2+(2\bar l)^2}4}
   \frac{2\bar l}{\sqrt{\delta_\mathrm s^2+(2\bar l)^2}}\ .
\end{align}
これより、\TopCurvatureCenter T$_{R_\mathrm c}'$と\TopODCenter T$_\mathrm c'$との差は\pageeqref{eq:TRc-Tc}となることがわかる。

%%%%%%%%%%%%%%%%%%%%%%%%%%%%%%%%%%%%%%%%%%%%%%%%%%%%%%%%%%
%% subsubsection 2.1.2.2 %%%%%%%%%%%%%%%%%%%%%%%%%%%%%%%%%
%%%%%%%%%%%%%%%%%%%%%%%%%%%%%%%%%%%%%%%%%%%%%%%%%%%%%%%%%%
\subsubsection{\Table を傾けた場合のT\texorpdfstring{$_\mathrm c'$}{c'}}
同様にして、外面A・C側のトップ端点T$_\mathrm o'$, T$_\mathrm i'$の、\TableCenter Pを原点とした場合の$X$座標はそれぞれ、
\begin{align}
%% label{eq:tableTi}
  \label{eq:tableTi}
  \text{C側端点:}&~
  \sqrt{R_\mathrm i^2-f_\mathrm T^2}-\Delta_x'\cos\theta\ ,\\
  \notag
  \text{A側端点:}&~
  \sqrt{R_\mathrm o^2-f_\mathrm T^2}-\Delta_x'\cos\theta\ .
\end{align}
したがって、トップ端における(\ACOD の)中点T$_\mathrm c'$の$X$座標は、
%% label{eq:tableTc}
\begin{align}
  \label{eq:tableTc}
  \frac{\sqrt{R_\mathrm o^2-f_\mathrm T^2}+\sqrt{R_\mathrm i^2-f_\mathrm T^2}}2
  -\Delta_x'\cos\theta\ .
\end{align}
これより、\TopCurvatureCenter T$_{R_\mathrm c}'$と\TopODCenter T$_\mathrm c'$との差は\pageeqref{eq:TRc-Tc}となることがわかる。




\clearpage
%%%%%%%%%%%%%%%%%%%%%%%%%%%%%%%%%%%%%%%%%%%%%%%%%%%%%%%%%%
%% section 2.2 %%%%%%%%%%%%%%%%%%%%%%%%%%%%%%%%%%%%%%%%%%%
%%%%%%%%%%%%%%%%%%%%%%%%%%%%%%%%%%%%%%%%%%%%%%%%%%%%%%%%%%
\modHeadsection{\BottomEndFace の幾何}


%%%%%%%%%%%%%%%%%%%%%%%%%%%%%%%%%%%%%%%%%%%%%%%%%%%%%%%%%%
%% subsection 2.2.1 %%%%%%%%%%%%%%%%%%%%%%%%%%%%%%%%%%%%%%
%%%%%%%%%%%%%%%%%%%%%%%%%%%%%%%%%%%%%%%%%%%%%%%%%%%%%%%%%%
\subsection{\BottomCurvatureCenter の位置}

%%%%%%%%%%%%%%%%%%%%%%%%%%%%%%%%%%%%%%%%%%%%%%%%%%%%%%%%%%
%% subsubsection 2.2.1.1 %%%%%%%%%%%%%%%%%%%%%%%%%%%%%%%%%
%%%%%%%%%%%%%%%%%%%%%%%%%%%%%%%%%%%%%%%%%%%%%%%%%%%%%%%%%%
\subsubsection{\Spacer を用いた場合のB\texorpdfstring{$_{R_\mathrm c}'$}{Rc'}}
\Spacer 取付後の\BottomCurvatureCenter B$_{R_\mathrm c}'$と、\TableCenter Pとの$X$方向の差は、トップ側の場合と同様に考えて
%% footnote %%%%%%%%%%%%%%%%%%%%%
\footnote{この場合は、ボトム側が工具側に向いている。}、
%%%%%%%%%%%%%%%%%%%%%%%%%%%%%%%%%
\begin{align*}
%  \label{eq:spacerBRc}
  \Delta_x-\sqrt{R_\mathrm c^2-f_\mathrm B^2}-\frac{\delta_\mathrm s}2
  +\sqrt{R_\mathrm i'^2-\frac{\delta_\mathrm s^2+(2\bar l)^2}4}\frac{2\bar l}{\sqrt{\delta_\mathrm s^2+(2\bar l)^2}}\ .
\end{align*}

%%%%%%%%%%%%%%%%%%%%%%%%%%%%%%%%%%%%%%%%%%%%%%%%%%%%%%%%%%
%% subsubsection 2.2.1.2 %%%%%%%%%%%%%%%%%%%%%%%%%%%%%%%%%
%%%%%%%%%%%%%%%%%%%%%%%%%%%%%%%%%%%%%%%%%%%%%%%%%%%%%%%%%%
\subsubsection{\Table を傾けた場合のB\texorpdfstring{$_{R_\mathrm c}'$}{Rc'}}
\Table を傾けた後の\BottomCurvatureCenter の位置B$_{R_\mathrm c}'$と、\TableCenter Pとの$X$方向の差は、トップ側の場合と同様に考えて
\begin{align}
  \label{eq:tableBRc}
  \left(\Delta_x+\sqrt{R_i'^2-\bar l^2}\right)\cos\theta-\sqrt{R_\mathrm c^2-f_\mathrm B^2}~.
\end{align}


%%%%%%%%%%%%%%%%%%%%%%%%%%%%%%%%%%%%%%%%%%%%%%%%%%%%%%%%%%
%% subsection 2.2.2 %%%%%%%%%%%%%%%%%%%%%%%%%%%%%%%%%%%%%%
%%%%%%%%%%%%%%%%%%%%%%%%%%%%%%%%%%%%%%%%%%%%%%%%%%%%%%%%%%
\subsection{\BottomEndFace における外側中心の位置}
\BottomCurvatureCenter B$_{R_\mathrm c}'$と\BottomODCenter B$_\mathrm c'$との差は、以下で与えられる。
%% label{eq:BRc-Bc}
\begin{align}
  \label{eq:BRc-Bc}
  \sqrt{R_\mathrm c^2-f_\mathrm B^2}
  -\frac{\sqrt{R_\mathrm o^2-f_\mathrm B^2}+\sqrt{R_\mathrm i^2-f_\mathrm B^2}}2\ .
\end{align}
よって、\BottomODCenter B$_\mathrm c'$の位置は、\BottomCurvatureCenter B$_{R_\mathrm c}'$から\pageeqref{eq:BRc-Bc}だけ加味すればよい。
以下では、\BottomODCenter B$_\mathrm c'$の位置を直接計算し、このことが整合していることを確かめる。

%%%%%%%%%%%%%%%%%%%%%%%%%%%%%%%%%%%%%%%%%%%%%%%%%%%%%%%%%%
%% subsubsection 2.2.2.1 %%%%%%%%%%%%%%%%%%%%%%%%%%%%%%%%%
%%%%%%%%%%%%%%%%%%%%%%%%%%%%%%%%%%%%%%%%%%%%%%%%%%%%%%%%%%
\subsubsection{\Spacer を用いた場合のB\texorpdfstring{$_\mathrm c'$}{c'}}
外面A・C面側のボトム端点B$_{R_\mathrm o}'$, B$_{R_\mathrm i}'$の、\TableCenter Pを原点とした場合の$X$座標はそれぞれ、
\begin{align*}
  \text{C側端点:}&~~
  \Delta_x-\sqrt{R_\mathrm i^2-f_\mathrm B^2}-\frac{\delta_\mathrm s}2+\sqrt{R_\mathrm i'^2-\frac{\delta_\mathrm s^2+(2\bar l)^2}4}\frac{2\bar l}{\sqrt{\delta_\mathrm s^2+(2\bar l)^2}}\ ,\\
  \text{A側端点:}&~~
  \Delta_x-\sqrt{R_\mathrm o^2-f_\mathrm B^2}-\frac{\delta_\mathrm s}2+\sqrt{R_\mathrm i'^2-\frac{\delta_\mathrm s^2+(2\bar l)^2}4}\frac{2\bar l}{\sqrt{\delta_\mathrm s^2+(2\bar l^2}}\ .
\end{align*}
したがって、ボトム端における(\ACOD の)中点B$_\mathrm c'$の$X$座標は、
%% label{eq:spacerBc}
\begin{align}
  \label{eq:spacerBc}
  \Delta_x-\frac{\sqrt{R_\mathrm o^2-f_\mathrm B^2}+\sqrt{R_\mathrm i^2-f_\mathrm B^2}}2
  -\frac{\delta_\mathrm s}2+\sqrt{R_\mathrm i'^2-\frac{\delta_\mathrm s^2+(2\bar l)^2}4}\frac{2\bar l}{\sqrt{\delta_\mathrm s^2+(2\bar l)^2}}\ .
\end{align}
これより、\BottomCurvatureCenter B$_{R_\mathrm c}'$と\BottomODCenter B$_\mathrm c'$との差は\pageeqref{eq:BRc-Bc}となることがわかる。

\clearpage
%%%%%%%%%%%%%%%%%%%%%%%%%%%%%%%%%%%%%%%%%%%%%%%%%%%%%%%%%%
%% subsubsection 2.2.2.2 %%%%%%%%%%%%%%%%%%%%%%%%%%%%%%%%%
%%%%%%%%%%%%%%%%%%%%%%%%%%%%%%%%%%%%%%%%%%%%%%%%%%%%%%%%%%
\subsubsection{\Table を傾けた場合のB\texorpdfstring{$_\mathrm c'$}{c'}}
外面A・C面側のボトム端点B$_{R_\mathrm o}'$, B$_{R_\mathrm i}'$の、\TableCenter Pを原点とした場合の$X$座標はそれぞれ、
\begin{align}
  \label{eq:tableBRi}
  \text{C側端点:}&~~
  \Delta_x'\cos\theta-\sqrt{R_\mathrm i^2-f_\mathrm B^2}\ ,\\
  \notag
  \text{A側端点:}&~~
  \Delta_x'\cos\theta-\sqrt{R_\mathrm o^2-f_\mathrm B^2}\ .
\end{align}
したがって、ボトム端における(\ACOD の)中点B$_\mathrm c'$の$X$座標は、
%% label{eq:tableBc}
\begin{align}
  \label{eq:tableBc}
  \Delta_x'\cos\theta-\frac{\sqrt{R_\mathrm o^2-f_\mathrm B^2}+\sqrt{R_\mathrm i^2-f_\mathrm B^2}}2
\end{align}
これより、\BottomCurvatureCenter B$_{R_\mathrm c}'$と\BottomODCenter B$_\mathrm c'$との差は\pageeqref{eq:BRc-Bc}となることがわかる。
\vfill
%%%%%%%%%%%%%%%%%%%%%%%%%%%%%%%%%%%%%%%%%%%%%%%%%%%%%%%%%%
%% Column %%%%%%%%%%%%%%%%%%%%%%%%%%%%%%%%%%%%%%%%%%%%%%%%
%%%%%%%%%%%%%%%%%%%%%%%%%%%%%%%%%%%%%%%%%%%%%%%%%%%%%%%%%%
\begin{\Columnname}{\EndFaceHorizontalOD の近似計算}
\ACOD $W_x$に対して、\TopEndHorizontalOD$W_\mathrm T$は以下で与えられる。(\BottomEndHorizontalOD も同様)
\begin{align*}
  W_\mathrm T
  = \sqrt{\left(R+\frac{W_x}2\right)^2-f_\mathrm T^2}
    -\sqrt{\left(R-\frac{W_x}2\right)^2-f_\mathrm T^2}\ .
\end{align*}
\index{テイラーてんかい@テイラー展開}テイラー展開(\index{マクローリンてんかい@マクローリン展開}マクローリン展開)\pageautoref{formula:taylorexpansion}より、
\begin{align*}
  (1+x)^\frac12 = 1+\frac x2-\frac{x^2}8+\frac{x^3}{16}-\frac{5x^4}{128}+o\left(x^5\right)
\end{align*}
なので、
\begin{align*}
  & (1+x)^\frac12(1+y)^\frac12-(1-x)^\frac12(1-y)^\frac12\\
  &= x+y+\frac{(x+y)(x-y)^2}8-\frac{xy(x+y)\big\{5(x-y)^2+7xy\big\}}{128}+\cdots\ .
\end{align*}
したがって、
\begin{align*}
  x = \frac{\nicefrac{W_x}2+f_\mathrm T}R\ ,\quad y = \frac{\nicefrac{W_x}2-f_\mathrm T}R\quad
  \longrightarrow \quad
  x+y = \frac{W_x}R\ , \quad x-y = \frac{2f_\mathrm T}R
\end{align*}
とすると、
\begin{align*}
  W_\mathrm T
  = R\left\{(1+x)^\frac12(1+y)^\frac12-(1-x)^\frac12(1-y)^\frac12\right\}
  = W_x\left(1+\frac{f_\mathrm T^2}{2R^2}+\cdots\right)
\end{align*}
と近似できる。
また$W_\mathrm T > W_x$であり、$R\to\infty$ ($R^{-1}\to0$)のとき$W_\mathrm T = W_x$であることもわかる。
\end{\Columnname}
%%%%%%%%%%%%%%%%%%%%%%%%%%%%%%%%%%%%%%%%%%%%%%%%%%%%%%%%%%
%%%%%%%%%%%%%%%%%%%%%%%%%%%%%%%%%%%%%%%%%%%%%%%%%%%%%%%%%%
%%%%%%%%%%%%%%%%%%%%%%%%%%%%%%%%%%%%%%%%%%%%%%%%%%%%%%%%%%



\clearpage
%%%%%%%%%%%%%%%%%%%%%%%%%%%%%%%%%%%%%%%%%%%%%%%%%%%%%%%%%%
%% section 22.3 %%%%%%%%%%%%%%%%%%%%%%%%%%%%%%%%%%%%%%%%%%
%%%%%%%%%%%%%%%%%%%%%%%%%%%%%%%%%%%%%%%%%%%%%%%%%%%%%%%%%%
\modHeadsection{\EndFacecutMilling の\TDCorrection}
\EndFacecutMilling として、$X+$, $Y+$方向の角から始めて
%% footnote %%%%%%%%%%%%%%%%%%%%%
\footnote{\DMC の場合、\ToolExchangePoint に近いので、このほうが移動距離が短くなる。}、
%%%%%%%%%%%%%%%%%%%%%%%%%%%%%%%%%
(工具から見て)左回りに加工する場合を考える。
このとき\index{かこうのけいろ@加工の経路}加工の経路として、単純に\OuterDiameter の値を指定すれば加工することは可能である。
しかしその場合、\index{こうぐ@工具}工具(\index{フェイスミル}フェイスミル)のほぼ中心に近い位置で切削する形になるので、工具に大きな負荷がかかることになる。
これを避けるために、理想的には、\InnerDiameter$w_{\mathrm T, \mathrm B}$の外側に工具が沿う形で切削するのが望ましい。
つまり、\EndFaceID$w_y$\,(BD方向の$w_{\mathrm T}$または$w_{\mathrm B}$)を基準として、\ToolRadius 分だけ(進行方向に対して右側に)補正をする形にすればよい。
ここでは誤差等を考慮して、\EndFaceID から$\delta_w$だけ縮めた輪郭(の外側)に沿う形を考える。


%%%%%%%%%%%%%%%%%%%%%%%%%%%%%%%%%%%%%%%%%%%%%%%%%%%%%%%%%%
%% subsection 22.3.1 %%%%%%%%%%%%%%%%%%%%%%%%%%%%%%%%%%%%%
%%%%%%%%%%%%%%%%%%%%%%%%%%%%%%%%%%%%%%%%%%%%%%%%%%%%%%%%%%
\subsection{加工の開始可能範囲}
初めの位置は$X+$, $Y+$方向の角の右方向($X+$方向)に工具があるものとする。
工具の\index{はけい(フェイスミル)@刃径(フェイスミル)}刃径(直径)を$\phi_\mathrm D$, \index{さいだいはけい(フェイスミル)@最大刃径(フェイスミル)}最大刃径(直径)$\phi'_\mathrm D$とすると
%% footnote %%%%%%%%%%%%%%%%%%%%%
\footnote{通常、刃径は\index{DC(フェイスミルはけい)@DC(フェイスミル刃径)}DC、最大刃径は\index{DCX(フェイスミルさいだいはけい)@DCX(フェイスミル最大刃径)}DCXと表記され、それぞれ直径として与えられることが多い。}、
%%%%%%%%%%%%%%%%%%%%%%%%%%%%%%%%%
$Y$位置については、工具の中心が
\begin{align}
  \label{eq:tanmenKakouStartY}
  \frac{w_y}2-\delta_w+\frac{\phi_\mathrm D}2
\end{align}
にあればよい。
そのためここでは、まず$Y$方向に\index{ぜったいざひょう@絶対座標}絶対座標(\verb|G90|)
\begin{align*}
  \frac{w_y}2-\delta_w
\end{align*}
まで移動し、その後に\ToolRadius 分の補正量として$\nicefrac{\phi_\mathrm D}2$だけ$Y+$方向にずらす形で、左方向($X-$方向)に移動する場合を考える。


%%%%%%%%%%%%%%%%%%%%%%%%%%%%%%%%%%%%%%%%%%%%%%%%%%%%%%%%%%
%% subsection 20.3.1 %%%%%%%%%%%%%%%%%%%%%%%%%%%%%%%%%%%%%
%%%%%%%%%%%%%%%%%%%%%%%%%%%%%%%%%%%%%%%%%%%%%%%%%%%%%%%%%%
\subsection{\TDCorrection を用いる場合}
\TDCorrection{\ttfamily G42}を用いる場合を考える。
このとき、動き始めは$Y$方向の補正分も加えて斜めに移動することになる。
ここで、
\begin{enumerate}[label=\sarrow]
\item $X$, $Y$方向には同じ速さで動く
\item $Y$方向の移動がなくなるまで工具は\index{ワーク}ワークに触れない
\end{enumerate}
とすると、加工(移動)の開始位置の$X$座標は、
\begin{align*}
  \frac{W_x}2+\frac{\phi'_\mathrm D+\phi_\mathrm D}2
  = \frac{w_x}2+\tau_x+\frac{\phi'_\mathrm D+\phi_\mathrm D}2
\end{align*}
より右方向($X+$方向)であればよい。
なお、$W_x$, $\tau_x$は\ACOD および\ACThickness である
%% footnote %%%%%%%%%%%%%%%%%%%%%
\footnote{ここでは話を単純化し、\PlatingThk や\index{へんにく@偏肉}偏肉は無視している。}。
%%%%%%%%%%%%%%%%%%%%%%%%%%%%%%%%%


\clearpage
%%%%%%%%%%%%%%%%%%%%%%%%%%%%%%%%%%%%%%%%%%%%%%%%%%%%%%%%%%
%% subsection 2.30.2 %%%%%%%%%%%%%%%%%%%%%%%%%%%%%%%%%%%%%
%%%%%%%%%%%%%%%%%%%%%%%%%%%%%%%%%%%%%%%%%%%%%%%%%%%%%%%%%%
\subsection{手動で補正を行う場合}
手動で補正する場合は、予め$Y$位置を\pageeqref{eq:tanmenKakouStartY}に移動しておいて、そのまま左方向($X-$方向)に移動すればよい。
よって、加工(移動)の開始位置の$X$座標は、
\begin{align*}
  \frac{W_x+\phi_\mathrm D}2 = \frac{w_x+\phi_\mathrm D}2+\tau_x
\end{align*}
より右方向($X+$方向)にあればよい
%% footnote %%%%%%%%%%%%%%%%%%%%%
\footnote{実際の\index{NCプログラム}NCプログラムでは、安全を考慮して$\nicefrac{w_x}2+\phi'_\mathrm D$とする。
この場合、
\begin{align*}
  \phi'_\mathrm D > \frac{\phi_\mathrm D}2+\tau_x
\end{align*}
である限り、衝突は生じないことになる。
一般に、$\tau_x < \nicefrac{\phi_\mathrm D}2$であるので、これは常に満たされる。}。
%%%%%%%%%%%%%%%%%%%%%%%%%%%%%%%%%
