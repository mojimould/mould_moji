%!TEX root = ../RfCPN.tex


\modHeadchapter{\EndFaceRChamfer の幾何}
ここでは主に、\textbf{\EndFaceOutRChamfer}および\textbf{\EndFaceInRChamfer}に関する測定・加工に必要な\expandafterindex{きかてきせいしつ(\yomiEndFaceRChamfer)@幾何的性質(\nameEndFaceRChamfer)}幾何的性質を考える。

\EndFaceRChamferMilling では、\index{ボールエンドミル}ボールエンドミルの工具を用いて行われるのが一般的である。
ボールエンドミルは\DMC には搭載しない予定であり、基本的には\EndFaceRChamfer は\index{てもちけんまき@手持ち研磨機}手持ち研磨機を用いて\index{さぎょうしゃ@作業者}作業者の手作業で行う。
しかし\EndFaceChamferLength が大きくなると、削る体積は3次関数的に増加する。
そのため作業者の安全性が低下することはもちろん、作業者にかかる負担も大きくなり、さらには\index{マシニングセンタ}マシニングセンタ外での作業時間も大きく増えることになる。

そこで、\EndFaceCChamferMilling で用いられる\index{テーパエンドミル}テーパエンドミルを用いて、\EndFaceRChamfer 部分の一部を切削することを考える。



%%%%%%%%%%%%%%%%%%%%%%%%%%%%%%%%%%%%%%%%%%%%%%%%%%%%%%%%%%
%% section 25.1 %%%%%%%%%%%%%%%%%%%%%%%%%%%%%%%%%%%%%%%%%%
%%%%%%%%%%%%%%%%%%%%%%%%%%%%%%%%%%%%%%%%%%%%%%%%%%%%%%%%%%
\modHeadsection{\index{テーパエンドミル}テーパエンドミルによる\EndFaceChamfer}
\TopEndFaceOutRChamferRadius および\BottomEndFaceOutRChamferRadius を、それぞれ$r_\mathrm{To}$, $r_\mathrm{Bo}$とする。
このとき、\index{かたかく(テーパエンドミル)@片角(テーパエンドミル)}片角$\xi_\mathrm e$のテーパエンドミルに対して、
\begin{align*}
  c_\mathrm{To} &= r_\mathrm{To}\ab(1+\cot\xi_\mathrm e-\csc\xi_\mathrm e)\\
  c_\mathrm{Bo} &= r_\mathrm{Bo}\ab(1+\cot\xi_\mathrm e-\csc\xi_\mathrm e)
\end{align*}
の\EndFaceOutCChamferLength とみなして加工を行うと、\EndFaceRChamfer に接する形で加工を行うことができる。
\TopEndFaceInRChamferRadius および\BottomFaceInRChamferRadius$r_\mathrm{Ti}$, $r_\mathrm{Bi}$についても同様に、
\begin{align*}
  c_\mathrm{Ti} &= r_\mathrm{Ti}\ab(1+\cot\xi_\mathrm e-\csc\xi_\mathrm e)\\
  c_\mathrm{Bi} &= r_\mathrm{Bi}\ab(1+\cot\xi_\mathrm e-\csc\xi_\mathrm e)
\end{align*}
とみなすことができる。
特に、$\xi_\mathrm e = \nicefrac\pi{12}$\,($15^\circ$), $\nicefrac\pi6$\,($30^\circ$), $\nicefrac\pi4$\,($45^\circ$)のとき、それぞれ
\begin{align*}
  c_\mathrm{To} &= \ab(3+\sqrt3-\sqrt6-\sqrt2)r_\mathrm{To} \sim 0.8683\cdot r_\mathrm{To}\\
  c_\mathrm{To} &= \ab(\sqrt3-1)r_\mathrm{To} \sim 0.7321\cdot r_\mathrm{To}\\
  c_\mathrm{To} &= \ab(2-\sqrt2)r_\mathrm{To} \sim 0.5858\cdot r_\mathrm{To}
\end{align*}
%%%%%%%%%%%%%%%%%%%%%%%%%%%%%%%%%%%%%%%%%%%%%%%%%%%%%%%%%%
%% hosoku %%%%%%%%%%%%%%%%%%%%%%%%%%%%%%%%%%%%%%%%%%%%%%%%
%%%%%%%%%%%%%%%%%%%%%%%%%%%%%%%%%%%%%%%%%%%%%%%%%%%%%%%%%%
\begin{hosoku}
原点を中心とする半径$r_\mathrm{To}$の円の第一象限に対し、$y$軸との角度が$\xi_\mathrm e$となる接線を考えればよい。
このとき、原点と接点を結んだ直線の傾きは$\tan\xi_\mathrm e$なので、接線の傾きは$-\cot\xi_\mathrm e$となる。
またこの接線は接点($r_\mathrm{To}\cos\xi_\mathrm e$, $r_\mathrm{To}\sin\xi_\mathrm e$)あるいは(0, $r\sec\xi_\mathrm e$)等を通ることを用いると、接線$y = -x\cot\xi_\mathrm e+r_\mathrm{To}\csc\xi_\mathrm e$が得られる。
あとは$x = r_\mathrm{To}$のときの位置と、端点($r_\mathrm{To}$, $r_\mathrm{To}$)との差をみればよい。
\end{hosoku}
%%%%%%%%%%%%%%%%%%%%%%%%%%%%%%%%%%%%%%%%%%%%%%%%%%%%%%%%%%
%%%%%%%%%%%%%%%%%%%%%%%%%%%%%%%%%%%%%%%%%%%%%%%%%%%%%%%%%%
%%%%%%%%%%%%%%%%%%%%%%%%%%%%%%%%%%%%%%%%%%%%%%%%%%%%%%%%%%



\clearpage
%%%%%%%%%%%%%%%%%%%%%%%%%%%%%%%%%%%%%%%%%%%%%%%%%%%%%%%%%%
%% section 25.2 %%%%%%%%%%%%%%%%%%%%%%%%%%%%%%%%%%%%%%%%%%
%%%%%%%%%%%%%%%%%%%%%%%%%%%%%%%%%%%%%%%%%%%%%%%%%%%%%%%%%%
\modHeadsection[中心座標\texorpdfstring{$X$}{X}の移動]{中心座標$X$の移動}
\EndFaceCChamfer とみなして加工を行うため、\Outcut のある場合を除いて$X$方向への移動を考える必要がある。
移動の大きさは、通常の\EndFaceCChamfer の場合と同じである。
\begin{align*}
  \text{外面取 トップ側:}&~~
  \sqrt{R_\mathrm c^2-\ab(f_\mathrm T-c_\mathrm{To})^2}-\sqrt{R_\mathrm c^2-f_\mathrm T^2}\ ,\\
  \text{外面取 ボトム側:}&~~
  \sqrt{R_\mathrm c^2-f_\mathrm B^2}-\sqrt{R_\mathrm c^2-\ab(f_\mathrm B-c_\mathrm{Bo})^2}\ ,\\
  \text{内面取 トップ側:}&~~
  \sqrt{R_\mathrm c^2-\ab(f_\mathrm T-c_\mathrm{Ti})^2}-\sqrt{R_\mathrm c^2-f_\mathrm T^2}\ ,\\
  \text{内面取 ボトム側:}&~~
  \sqrt{R_\mathrm c^2-f_\mathrm B^2}-\sqrt{R_\mathrm c^2-\ab(f_\mathrm B-c_\mathrm{Bi})^2}\ .
\end{align*}
