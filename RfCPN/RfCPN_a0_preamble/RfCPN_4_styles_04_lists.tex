%!TEX root = ../RfCPN.tex


%%%%%%%%%%%%%%%%%%%%%%%%%%%%%%%%%%%%%%%%%%%%%%
%%%%% FOR STYLE OF TOC %%%%%%%%%%%%%%%%%%%%%%%
%%%%%%%%%%%%%%%%%%%%%%%%%%%%%%%%%%%%%%%%%%%%%%
\setcounter{tocdepth}{3}
\renewcommand\contentsname{\texorpdfstring{\hbox to 2em{目次}}{目次}}
%%%%% FOR PART %%%%%
\renewcommand{\partautorefname}{part}
\DeclareTOCStyleEntry[
  indent=0em, % エントリのインデントを調整
  dynnumwidth=true,
  pagenumberwidth=19.305pt
]{tocline}{part}
%%%%% FOR CHAPTER %%%%%
\renewcommand{\thechapter}{\ifnum\value{chapter}<10 0\fi\arabic{chapter}}
\renewcommand{\chapterautorefname}{章}  % chapter --> 章
\DeclareTOCStyleEntry[
  level=\chaptertocdepth,
  indent=0em, % エントリのインデントを調整
  dynnumwidth=true,
  pagenumberwidth=17.968pt
]{tocline}{chapter}
%%%%% FOR SECTION %%%%%
\renewcommand{\thesection}{\thechapter.\ifnum\value{section}<10 0\fi\arabic{section}}
\renewcommand{\sectionautorefname}{節\!}
\DeclareTOCStyleEntry[
  level=\sectiontocdepth,
  dynindent=true,
  dynnumwidth=true,
]{tocline}{section}
%%%%% FOR SUBSECTION %%%%%
\renewcommand{\thesubsection}{\thesection.\ifnum\value{subsection}<10 0\fi\arabic{subsection}}
\renewcommand{\subsectionautorefname}{\sectionautorefname}
\DeclareTOCStyleEntry[
  dynindent=true,
  dynnumwidth=true,
]{tocline}{subsection}
%%%%% FOR SUBSUBSECTION %%%%%
\renewcommand{\subsubsectionautorefname}{\sectionautorefname}
\DeclareTOCStyleEntry[
  dynindent=true,
  dynnumwidth=true,
]{tocline}{subsubsection}
%%%%% FOR TABLE %%%%%
\DeclareRobustCommand{\tablename}{表}
\renewcommand{\tableautorefname}{\tablename}
\renewcommand{\thetable}{\thechapter.\ifnum\value{table}<10 0\fi\arabic{table}}
\DeclareTOCStyleEntry[
  level=\sectiontocdepth,
  indent=1.74em, % エントリのインデントを調整
  dynnumwidth=true,
  entrynumberformat={表},
]{tocline}{table}
%%%%% FOR FIGURE %%%%%
\DeclareRobustCommand{\figurename}{図}
\renewcommand{\figureautorefname}{\figurename}
\renewcommand{\thefigure}{\thechapter.\ifnum\value{figure}<10 0\fi\arabic{figure}}
\DeclareTOCStyleEntry[
  level=\sectiontocdepth,
  indent=1.74em, % エントリのインデントを調整
  dynnumwidth=true,
  entrynumberformat={図},
]{tocline}{figure}
%%%%% lop %%%%%%%%%%%%%%%%%%%%
\DeclareNewTOC[
  owner=\jobname,
  name=Part,
]{lop}
%%%%% locode %%%%%%%%%%%%%%%%%%%%
\DeclareNewTOC[
  owner=\jobname,
  name=code,
  float,
  nonfloat,
  counterwithin=chapter,
  floatpos=tbp,
  floattype=2,
]{locode}
\DeclareRobustCommand{\locodename}{Prg.\,}
\renewcommand{\thelocode}{\thechapter.\ifnum\value{locode}<10 0\fi\arabic{locode}}
\DeclareTOCStyleEntry[
  level=\sectiontocdepth,
  indent=1.74em, % エントリのインデントを調整
  dynnumwidth=true,
  entrynumberformat={Prg.\,},
]{tocline}{locode}
%%%%% loColumn %%%%%%%%%%%%%%%%%%%%
\DeclareNewTOC[
  owner=\jobname,
  name=Column,
  float,
  nonfloat,
  counterwithin=chapter,
  floatpos=tbp,
  floattype=2,
]{loColumn}
\DeclareRobustCommand{\loColumnname}{\Columnname}
\renewcommand{\theloColumn}{\thechapter.\ifnum\value{loColumn}<10 0\fi\arabic{loColumn}}
\DeclareTOCStyleEntry[
  level=\sectiontocdepth,
  indent=2.35em, % エントリのインデントを調整
  dynnumwidth=true,
  entrynumberformat={Column.},
]{tocline}{loColumn}

%%%%% STYLE OF LIST FOR APPENDIX %%%%%
\AtBeginEnvironment{appendices}{%
  \clearrightpage%
  \@appendixtrue%
  \patchcmd{\part}{\newpage}{\relax}{}{}%
  \pretocmd{\part}{\addtocontents{toc}{\protect\tcbline*}}{}{}{}% add page break before parts, except part 1
}
\AtEndEnvironment{appendices}{\clearrightpage}
\AfterEndEnvironment{appendices}{%
  \patchcmd{\part}{\tocAPartSeparateline}{\relax}{}{}%
  \@appendixfalse%
}
\renewcommand{\setthesection}{\Alph{section}}
\renewcommand{\appendixautorefname}{補遺\!}

%%%%%%%%%%%%%%%%%%%%%%%%%%%%%%%%%%%%%%%%%%%%%%
%%%%% FOR STYLE OF OTHER LISTS %%%%%%%%%%%%%%%
%%%%%%%%%%%%%%%%%%%%%%%%%%%%%%%%%%%%%%%%%%%%%%
\renewcommand\listtablename{\texorpdfstring{\hbox to 3em{表目次}}{表目次}}
\renewcommand\listfigurename{\texorpdfstring{\hbox to 3em{図目次}}{図目次}}
\renewcommand\listoflopname{\texorpdfstring{\hbox to 3em{大目次}}{大目次}}
\renewcommand{\listoflocodename}{コード 目次}
\renewcommand\listofloColumnname{Column\,一覧}

%%%%% FOR STYLE OF INDEX %%%%%%%%%%%%%%%%%%%%%
\renewcommand{\indexname}{\texorpdfstring{\hbox to 2em{索引}}{索引}}
\renewcommand\indexpagestyle{plainheadback}
\newcommand\symbolindexname{記号・数字}

%%%%% FOR STYLE OF BIBLATEX %%%%%%%%%%%%%%%%%%
\DeclareLanguageMapping{japanese}{english}
%\DefineBibliographyStrings{english}{in={in}}
\renewcommand{\bibname}{\texorpdfstring{\hbox to 4em{文献一覧}}{文献一覧}}
\newcommand{\referencesname}{参考文献}
\newcommand{\citedworksname}{参照文献}
\newcommand{\Articlename}{論文}
\newcommand{\Bookname}{書籍}
\newcommand{\OnlineSourcename}{ウェブサイト}
\newcommand{\Manualname}{マニュアル}
\DeclareFieldFormat{title}{``#1"}
\DeclareFieldFormat{urldate}{(urlseen~\thefield{urlyear}/\ifnum\thefield{urlmonth}<10 0\fi\thefield{urlmonth})}
\ExecuteBibliographyOptions{
  language=auto,
  autolang=other,
  sorting=none,
  defernumbers=false,
  hyperref=true,
  block=nbpar,
  subentry=true,
  citecounter=true,
}
\appto\bibfont{\footnotesize\setstretch{1.1}}
\DeclareFieldFormat{labelnumberwidth}{\mkbibbrackets{#1}\hspace{-6pt}}
\DeclareFieldFormat{date}{%
  \thefield{year}%
  \iffieldundef{month}{}{/}%
  \iffieldundef{month}{}{\two@digits{\thefield{month}}}%
  \iffieldundef{day}{}{/}%
  \iffieldundef{day}{}{\two@digits{\thefield{day}}}%
}
\defbibheading{subsubbibintoc}{\subsection*{#1}\addsubsectiontocentry{}{#1}}

%%%%% CAPTION STYLE %%%%%%%%%%%%%%%%%%%%%%%%%%
%\counterwithin{locode}{chapter}
%\DeclareCaptionLabelFormat{table}{\tablename\protect\thechapter.\two@digits{\value{table}}}
%\DeclareCaptionLabelFormat{figure}{\figurename\protect\thechapter.\two@digits{\value{figure}}}
\captionsetup[table]{
  justification=raggedright,
  singlelinecheck=false,
  position=above,
  aboveskip=4pt,
  belowskip=5pt,
  labelfont={bf, sf},
  labelsep={space},
  font={bf, large, sf},
  hypcap=false,
  name={表},
}
\captionsetup[figure]{%
  width=.9\textwidth,
  format=hang,
  labelfont={bf, sf},
  labelsep={colon},
  aboveskip=3pt,
  name={図},
}
\captionsetup[locode]{
  justification=raggedright,
  singlelinecheck=false,
  position=above,
  aboveskip=5pt,
  belowskip=0.0pt,
  labelfont={bf, sf},
  labelsep={space},
  font={bf, large, sf},
  hypcap=false,
}
