%!TEX root = ../RfCPN.tex


%%%%% FOR STYLE OF PART %%%%%%%%%%%%%%%%%%%%%%
\renewcommand{\partautorefname}{part}
\renewcommand\partpagestyle{emptydate}

%%%%% FOR STYLE OF CHAPTER %%%%%
\renewcommand{\thechapter}{\ifnum\value{chapter}<10 0\fi\arabic{chapter}}
\renewcommand{\chapterautorefname}{章}  % chapter --> 章
\renewcommand\chapterpagestyle{\if@frontmatter plainheadfront\else plainhead\fi}
\renewcommand*{\chapterformat}{%
  \begin{tikzpicture}[every node/.style={signal, draw, text=white, signal to=nowhere},
                      baseline=-7.75pt] % +: title up, -: title down
    \node[fill=\if@appendix white\else Maroon!65\fi,
          text=\if@appendix Maroon!65\else white\fi,
          signal from=east,
          inner sep=5pt,
          text height=1.5ex,
          text depth=2pt]
      at (0, 0) {\LARGE{\sffamily\bfseries\,\thepart.\,\thechapter\,\relax}};
  \end{tikzpicture}~~%
}
%%%%% STYLE OF APPENDICES %%%%%%%%%%%%%%%%%%%%%%%%%%%%
\renewcommand{\setthesection}{\Alph{section}}
\renewcommand{\appendixautorefname}{補遺\!}

%%%%% FOR STYLE OF SECTION %%%%%%%%%%%%%%%%%%%
\renewcommand{\thesection}{\thechapter.\ifnum\value{section}<10 0\fi\arabic{section}}
\renewcommand{\sectionautorefname}{節\!}
\setcounter{secnumdepth}{4}
\renewcommand*{\sectionformat}{%
  \begin{tikzpicture}[every node/.style={signal, draw, text=white, signal to=nowhere},
                      baseline=-5.0pt]
    \node[fill=Green!65!,
          text=white,
          signal from=east,
          inner sep=5pt,
          text height=1.4ex,
          text depth=0pt]
      at (0, 0) {\Large{\sffamily\bfseries\,\thesection~\relax}};
  \end{tikzpicture}~~%
}

%%%%% FOR STYLE OF SUBSECTION %%%%%%%%%%%%%%%%
\renewcommand{\subsectionautorefname}{\sectionautorefname}
\renewcommand*{\subsectionformat}{%
  \begin{tikzpicture}[every node/.style={signal, draw, text=white, signal to=nowhere},
                      baseline=-3.5pt]
    \node[fill=blue!50!,
          text=white,
          signal from=east,
          inner sep=5pt,
          text height=1.3ex,
          text depth=0pt]
      at (0, 0) {\large{\sffamily\bfseries\,\thesubsection~\relax}};
  \end{tikzpicture}~~%
}

%%%%% FOR STYLE OF SUBSUBSECTION %%%%%%%%%%%%%
\renewcommand{\subsubsectionautorefname}{\sectionautorefname}
\renewcommand*{\subsubsectionformat}{%
  \begin{tikzpicture}[every node/.style={signal, draw, text=white, signal to=nowhere},
                      baseline=-3.75pt]
    \node[fill=Green!50!Blue!40!,
          text=white,
          signal from=east,
          inner sep=5pt,
          text height=1.25ex,
          text depth=0pt]
      at (0, 0) {\small{\sffamily\bfseries\,\thesubsubsection~\relax}};
  \end{tikzpicture}~~%
}

%%%%% FOR STYLE OF PARAGRAPH %%%%%%%%%%%%%%%%%
%for scrbook.cls
\RedeclareSectionCommand[%
  style=section,%
  level=4,%
  indent=0pt,%
  afterindent=false,
  beforeskip=3.25ex \@plus1ex \@minus.2ex,%
  afterskip=0.1ex \@plus.1ex \@minus.1ex,% -1em から変更
  tocindentfollows=subsubsection,%
  tocstyle=section,%
  tocindent=10em,%
  tocnumwidth=5em,% def: 5em
  font=\raggedsection\textgt\nobreak\sball{blue}~
]{paragraph}
%for book.cls
%\renewcommand\paragraph[1]{%
%  \@startsection{paragraph}{\paragraphnumdepth}{0pt}%
%  {3.25ex \@plus1ex \@minus.2ex}% \@plus, \@minusは伸び縮みできるスペースの長さ
%  {0.1ex\@plus.1ex \@minus.1ex}% ここが正だと改行されて、値だけ垂直スペースが入る
%  {\raggedsection\normalfont\sectfont\gtfamily\nobreak\size@paragraph\sball{blue}~}{#1}\noindent
%}

%%%%% FOR STYLE OF SUBPARAGRAPH %%%%%%%%%%%%%%
\RedeclareSectionCommand[%
  style=section,%
  level=5,%
  indent=0pt, % default: \scr@parindent,%
  afterindent=false,
  beforeskip=0.5ex \@plus1ex \@minus.2ex,% 3.25ex \@plus1ex \@minus .2ex から変更
  afterskip=0.1ex \@plus.1ex \@minus.1ex,% -1em から変更
  tocstyle=section,%
  tocindent=12em,%
  tocnumwidth=6em%
]{subparagraph}
