%!TEX root = ../RfCPN.tex


%%%%% COLOR %%%%%%%%%%%%%%%%%%%%%%%%%%%%%%%%%%
\definecolor{ai}     {rgb}{0.2039, 0.3765, 0.4314}
\definecolor{hosoku} {cmyk}{0, 0, 0, .1}
\definecolor{kon}    {rgb}{0.0000, 0.2000, 0.4000}
\definecolor{konpeki}{rgb}{0.0902, 0.5098, 0.7333}
\definecolor{moegi}  {rgb}{0.3020, 0.5961, 0.1882}
\definecolor{sssec}  {rgb}{0.7333, 0.5, 0.7333}
\definecolor{sora}   {rgb}{0.1451, 0.7216, 0.8039}
\definecolor{sumire} {rgb}{0.3882, 0.2157, 0.5922}
\newcommand{\getlinkcolor}{\@linkcolor}
%%%%% LOGO %%%%%%%%%%%%%%%%%%
\newcommand\BibLaTeX{{\textrm{B\kern-.05em{\textsc{i\kern-.025em b}}\LaTeX}}}
\newcommand\pgflogo{\textsc{pgf}}
\newcommand\tikzlogo{Ti\textit{k}Z}
%%%%% DATE %%%%%%%%%%%%%%%%%%%%%%%%%%%%%%%%%%%
\newcommand{\customtoday}{\the\year/\two@digits{\the\month}/\two@digits{\the\day}}
\newcommand{\customdate}{\customtoday\ \currenttime\ (\jadayofweek{\the\year}{\the\month}{\the\day})}
\newcommand{\customtodayap}{\ifnum\currenthour<12 \customtoday\,a.m.\else\customtoday\,p.m.\fi}
%%%%% FOR INDEX %%%%%%%%%%%%%%%%%%
\NewDocumentCommand{\expandafterindex}{m}{\index{#1}}
\pdfstringdefDisableCommands{\def\expandafterindex#1{}}
%%%%% DECLAREMATHOPERATOR %%%%%%%%%%%%%%%%%%%%
\DeclareMathOperator{\IP}{Im}
\DeclareMathOperator{\RP}{Re}
%%%%% OTHER DEFINITION %%%%%%%%%%%%%%%%%%%%%%%
\newcommand\ttNum{\ifmmode{\text{\texttt\#}}\else\texttt\#\fi}
\pdfstringdefDisableCommands{\def\leavevmode@ifvmode{}}
\pdfstringdefDisableCommands{\def\kern{}}
\newcommand\cf{{\itshape cf.\,}}
\newcommand{\TBW}{\texorpdfstring{\small{\color{red}\,*}}{}}
%%%%% OTHER TIKZ DEFINITION %%%%%%%%%%%%%%%%%%
%%%%% TIKZSET %%%%%%%%%%%%%%%%%%%
\tikzset{
  %%%%% TERMINAL STYLE %%%%%
  terminal/.style={%
    rectangle,%
    minimum size=10pt,%
    rounded corners=1.5mm,%
    thin,%
    draw=black!75,%
    top color=white,%
    font=\ttfamily,%
    inner sep=3pt,%
    inner xsep=3pt,%
    text height=1ex,%
    text depth=0pt,%
  },
  %%%%% BMATRIX STYLE %%%%%
%  every left delimiter/.style={xshift=.5em},
%  every right delimiter/.style={xshift=-.5em},
%  bmatrix/.style={matrix of math nodes, left delimiter=[, right delimiter=],},
}
\tikzfading[name=fade ball, inner color=transparent!60, outer color=transparent!30]
\newcommand\sarrow[1][blue]{%
  \tikz\draw[decorate,
             decoration={shape backgrounds, shape=dart, shape size=2.0mm},
             color=#1,
             fill=#1,
             path fading=fade ball] (0, 0);
}
\newcommand\sball[1]{\tikz\shade[ball color=#1, path fading=fade ball] (0,0) circle (.7ex);}
\newcommand\terminal[2]{\tikz[baseline=(a.base)] \node (a) [terminal, bottom color=#2] {\small #1};}
\newcommand\termblue[1]{\terminal{\color{blue}\fontsize{8pt}{0pt}\textbf{#1}}{gray!25}\hskip6pt}
%%%%% DECLAREROBUSTCOMMAND %%%%%%%%%%%%%%%%%%%%
\DeclareRobustCommand{\bDiv}{\nonscript\mskip-\medmuskip\mkern5mu\mathbin
  {\operator@font div}\penalty900
  \mkern5mu\nonscript\mskip-\medmuskip}
\DeclareRobustCommand{\pod}[1]{\allowbreak
  \if@display\mkern18mu\else\mkern8mu\fi(#1)}
\DeclareRobustCommand{\pDiv}[1]{\pod{{\operator@font div}\mkern6mu#1}}
\DeclareRobustCommand{\Div}[1]{\allowbreak\if@display\mkern18mu
  \else\mkern12mu\fi{\operator@font div}\,\,#1}
%%%%% AUTOREFNAME %%%%%%%%%%%%%%%%%%%%%%%%%%%%
%\newcommand{\subfigureautorefname}{\figureautorefname}
%\newcommand{\subtableautorefname}{\tableautorefname}
%%%%% PAGEREF %%%%%%%%%%%%%%%%%%%%%%%%%%%%%%%%%%%
\newcommand{\pageautoref}[1]{%
  \ifthenelse{\equal{\pageref{#1}}{\thepage}}%
    {\autoref{#1}}%
    {\autoref{#1}~[p.\pageref{#1}]}%
}
\newcommand{\pageeqref}[1]{%
  \ifthenelse{\equal{\pageref{#1}}{\thepage}}%
    {\eqref{#1}}%
    {\eqref{#1}~[p.\pageref{#1}]}%
}
%%%%% tPart %%%%%%%%%%%%%%%%%%%%%%%%%%%%%%%%%%%%%%%%%
\newcounter{tocwatermark}
\setcounter{tocwatermark}{1}
\def\tmppartnum{\Roman{tocwatermark}}%
\newcommand{\tPart}[4][]{%
  \addtocounter{tocwatermark}{1}
  \def\tmppartnum{\Roman{tocwatermark}}%
  \begingroup
  \clearrightpage
  \let\oldnewpage\newpage
  \let\newpage\relax
  \part{\label{part:\thepart}#2\vphantom{\ref{part:\thepart}}}
  \addtocontents{lop}{\protect\vspace*{-2.5mm}}%
  \addcontentsline{lop}{part}{\protect\numberline{\thepart}#2}%
  \ifx#1\relax\else%
    \foreach \x in {#1}{\addcontentsline{\x}{part}{\protect\numberline{\thepart}#2}}%
  \fi
  \let\newpage\oldnewpage
  \clearpage
%  \@openrightfalse
  \thispagestyle{emptydate}
  \vspace*{0.1\textheight}%
  \begin{tablePart}{Introduction}
  #3
  \end{tablePart}%
  \begin{tablePart}{Conclusion}
  #4
  \end{tablePart}%
%  \@openrightfalse
  \@mainmattertrue\pagestyle{main}
  \endgroup
}
%%%%% PART FOR APPENDIX %%%%%%%%%%%%%%%%%%%%%%%%%%%%%%%%%%%%%%%%%
\newcommand{\Appendixpart}{
  \cleardoublepage
  \part*{\partname\ \thepart\hx の補遺\label{Apart:\thepart}\vphantom{\ref{Apart:\thepart}}}
  \addcontentsline{toc}{part}{\partname\ \thepart\hx の補遺}
}

%%%%%%%%%%%%%%%%%%%%%%%%%%%%%%%%%%%%%%%%%%%%%%
%%%%% AUTO LABELING %%%%%%%%%%%%%%%%%%%%%%%%%%
%%%%%%%%%%%%%%%%%%%%%%%%%%%%%%%%%%%%%%%%%%%%%%
%%%%% FOR CHAPTER OR APPENDIX %%%%%%%%%%%%%%%%%%%%%%%%%%%%%
\newcommand{\modHeadchapter}[2][]{%
  \cleardoublepage
  \let\newpage\relax
  \ifx\@chapapp\appendixname%
    \chapter{#2\label{app:\thepart.\thechapter}\vphantom{\ref{app:\thepart.\thechapter}}}%
  \else%
    \chapter{#2\label{chap:\thepart.\thechapter}\vphantom{\ref{chap:\thepart.\thechapter}}}%
  \fi%
  \ifx#1\relax\else%
    \foreach \x in {#1}{\addcontentsline{\x}{chapter}{\protect\numberline{\thechapter}#2}}%
  \fi%
  \thispagestyle{main}%
  \indentspace%
  \let\newpage\tmpnewpage%
}
%%%%% FOR SECTION %%%%%%%%%%%%%%%%%%%%%%%%%%%%%
\newcommand{\modHeadsection}[2][]{%
  \ifx\relax#1\relax
    \section{#2\label{sec:\thepart.\thesection}\vphantom{\protect\ref{sec:\thepart.\thesection}}}%
  \else%
    \section[#1]{#2\label{sec:\thepart.\thesection}\vphantom{\protect\ref{sec:\thepart.\thesection}}}%
  \fi%
}
%%%%% MODCAPTOINOF %%%%%%%%%%%%%%%%%%%%%%%%%%%%%%
\setlength{\abovecaptionskip}{0pt}
\newcommand{\modcaptionof}[2]{%
  \captionof{#1}{%
    \csname #1name\endcsname\thechapter.%
    \ifnum\value{#1}<10 0\fi
      \arabic{#1}.#2%
  }%
}
