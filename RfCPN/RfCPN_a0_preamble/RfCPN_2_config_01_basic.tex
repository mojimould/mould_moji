%!TEX root = ../RfCPN.tex


%%%%% GRAPHICSPATH %%%%%%%%%%%%%%%%%%%%%%%%%%%%%%%
\graphicspath{{./RfCPN_a1_includefiles/RfCPN_2_pictures/}}
%%%%% FONT %%%%%%%%%%%%%%%%%%%%%%%%%%%%%%%
%\setmainjfont[BoldFont={HaranoAjiGothic}, BoldFeatures={FakeBold=2}]{HaranoAjiMincho}
%\setsansjfont{HaranoAjiGothic}
%\setmonojfont{HaranoAjiMincho}
%\setmainfont{Latin Modern Roman}
%\setsansfont{Latin Modern Sans}
\setmonofont{Consolas}
\setmathfont[BoldFont=*, BoldFeatures={FakeBold=2}]{Euler Math}
\setmathfont[range={\leq, \geq},  BoldFont=*, BoldFeatures={FakeBold=2}]{Latin Modern Math}
\setmathfont[range={up/num, bfup/num}, Scale=MatchUppercase, BoldFont=*, BoldFeatures={FakeBold=2}]{Latin Modern Math}
%%%%% NEWIF %%%%%%%%%%%%%%%%%%%%%%%%%%%%%%%%%%
\newif\if@backmatter%\@backmattertrue
\newif\if@frontmatter%\@frontmattertrueko
\newif\if@appendix%\@appendixtrue
%%%%%% DIMENSION %%%%%%%%%%%%%%%%%%%%%%%%%%%%%%
\ltjsetparameter{kanjiskip=0.0pt plus 0.4pt minus 0.5pt}
\ltjsetparameter{xkanjiskip=2.40555pt plus 1.0pt minus 1.0pt}
\newcommand{\hk}{\hspace{\glueexpr\ltjgetparameter{kanjiskip}\relax}}
\newcommand{\hx}{\hspace{\glueexpr\ltjgetparameter{xkanjiskip}\relax}}
%%%%% TMPPAGE %%%%%%%%%%%%%%%%%%%%%%%%%%%%%%%%%%
\let\tmpcleardoublepage\cleardoublepage
\let\tmpclearpage\clearpage
\let\tmpnewpage\newpage
%%%%% CLEARRIGHTPAGE %%%%%%%%%%%%%%%%%%%%%%%%%%
\newcommand{\clearrightpage}{%
  \if@twoside%
    \ifodd\numexpr\thepage\relax\cleardoublepage\else\clearpage\fi
  \else\clearpage%
  \fi
}
%%%%% DECLARENEWTOC %%%%%%%%%%%%%%%%%%%%
\DeclareNewTOC[
  owner=\jobname,
  name=Part,
]{lop}
\DeclareNewTOC[
  owner=\jobname,
  name=Code,
  float,
  nonfloat,
  counterwithin=chapter,
  floatpos=tbp,
  floattype=2,
  tocentryindent=2.35em,
  tocentrynumwidth=2.71em,
]{locode}
\DeclareNewTOC[
  owner=\jobname,
  name=Column,
  float,
  nonfloat,
  counterwithin=chapter,
  floatpos=tbp,
  floattype=2,
%  tocentryindent=2.35em,
%  tocentrynumwidth=5.10em,
]{loColumn}
%%%%% TO GET CHAPTER TITLE %%%%%%
\newcommand\Chaptername{} % initialize \Chaptername
\let\old@chapter\@chapter
\def\@chapter[#1]#2{\gdef\Chaptername{#2}\old@chapter[#1]{#2}}
%%%%% TO GET SECTION TITLE %%%%%%
\newcommand\Sectionname{} % initialize \Sectionname
\let\Sectionmark\sectionmark
\def\sectionmark#1{\def\Sectionname{#1}\Sectionmark{#1}}
%%%%% GEOMETRY %%%%%%%%%%%%%%%%%%%%%%%%%%%%%%%
\geometry{
  a4paper, % paper size
  centering,
  textwidth={6.55in},
  includehead,  % include the head of the page
%  headheight = 13.6pt,
  includefoot,  % include the foot of the page
  top=15.0truemm,
  bottom=-0.5truemm,
}
%%%%% HEYPERSETUP %%%%%%%%%%%%%%%%%%%%%%%%%%%%
\newcommand{\TheDocTitle}{管理項目のRDBS構築およびその応用}
\newcommand{\TheDocSubject}{関連ドキュメント作成の自動化}
\hypersetup{
%  pdfcreationdate=date,
%  pdfcreator={luaLaTeX with hyperref}, % creator for PDF subjct field
  pdftitle={\TheDocTitle}, % title for PDF subjct field
  pdfsubject={\TheDocSubject}, % text for PDF subjct field
  pdfauthor={Kurahashi Nobuaki},  % text for PDF Author field
  pdfkeywords={
    mold,
    mould,
    G-code,
    NC program,
    Machining center,
  },
  pdfproducer=producer,
  linktoc=all,
%  linktocpage=false,   % (if it is true) make page number, not text, be link on TOC, LOF and LOT
  pdfcenterwindow=false, % position the document window center of the screen
  pdffitwindow=true,     % resize document window to fit document size
  bookmarksnumbered=true,
  bookmarksopen=true, %bookmarks open
  pdfstartview={FitH}, % Fit, FitV, FitH, FitB
  pdfpagemode=UseThumbs, % set default mode of PDF display
  unicode=true,
  pdfencoding=unicode,   % PDFDocEncoding or Unicode
  colorlinks=true,     % color links
  linkcolor=ai,        % color of links
  urlcolor=ai,         % color of urls
  citecolor=sora,      % color of citation links
}
%%%%% DISPLAYBREAK %%%%%%%%%%%%%%%%%%%%%%%%%%%
\allowdisplaybreaks
%%%%% UNIT LENGTH %%%%%%%%%%%%%%%%%%%%%%%%%%%%
\setlength{\unitlength}{1pt}
%%%%% LINESPREAD %%%%%%%%%%%%%%%%%%%%%%%%%%%%%
\linespread{1.15}\selectfont
%%%%% PARINDENT %%%%%%%%%%%%%%%%%%%%%%%%%%%%%%
\newcommand{\indentspace}{\setlength\parindent{11pt}}
\indentspace
%%%%% EQUATION %%%%%%%%%%%%%%%%%%%%%%%%%%%%
\renewcommand{\theequation}{\thesection.\arabic{equation}}
\@addtoreset{equation}{section}
%%%%% FOR FOOTNOTE %%%%%%%%%%%%%%%%%%%%%%%%%%%
\renewcommand*{\footnoteautorefname}{脚注}
%\renewcommand{\thefootnote}{\ifnum\value{footnote}<10 0\fi\arabic{footnote}}
\interfootnotelinepenalty=10000
\counterwithout{footnote}{chapter}
\def\@makefnmark{\hbox{}\hbox{\@textsuperscript{\normalfont\@thefnmark}}\hbox{}}
\deffootnote[1em]{1em}{1em}{\textsuperscript{\thefootnotemark}}
\renewcommand\footnoterule{%
  \kern3pt
  \hrule\@width.75\columnwidth
  \kern2.6pt
}
\makesavenoteenv{Column}
\makesavenoteenv{hosoku}
\makesavenoteenv{tablePart}
\makesavenoteenv{marker}
\makesavenoteenv{Formula}
%%%%% SETLIST %%%%%%%%%%%%%%%%%%%%%%%%%%%%%%%%
\setlist[enumerate]{listparindent=\parindent, parsep=0pt, partopsep=0pt, topsep=3pt, itemsep=3pt, leftmargin=*}
\setlist[enumerate, 1]{leftmargin=\leftmargini}
%%%%% CAPTION STYLE %%%%%%%%%%%%%%%%%%%%%%%%%%
\renewcommand{\tablename}{表}
\addto\captionsjapanese{\renewcommand{\tablename}{表}}
\addto\captionsjapanese{\renewcommand{\figurename}{図}}
\renewcommand{\figurename}{図}
\renewcommand{\locodename}{Prg:\,}
\renewcommand{\thefigure}{\ifnum\value{figure}<10 0\fi\arabic{figure}}
\renewcommand{\thetable}{\ifnum\value{table}<10 0\fi\arabic{talble}}
\renewcommand{\thelocode}{\ifnum\value{locode}<10 0\fi\arabic{locode}}
\counterwithin{figure}{chapter}
\counterwithin{table}{chapter}
\DeclareCaptionLabelFormat{table}{\tablename#2}
\DeclareCaptionLabelFormat{figure}{\figurename#2}
\captionsetup[table]{
  justification=raggedright,
  singlelinecheck=false,
  position=above,
  aboveskip=4pt,
  belowskip=5pt,
  labelfont={bf, sf},
  labelsep={space},
  font={bf, large, sf},
  hypcap=false,
}
\captionsetup[figure]{%
  width=.9\textwidth,
  format=hang,
  labelfont={bf, sf},
  labelsep={colon},
  aboveskip=3pt,
}
\captionsetup[locode]{
  justification=raggedright,
  singlelinecheck=false,
  position=above,
  aboveskip=5pt,
  belowskip=0.0pt,
  labelfont={bf, sf},
  labelsep={space},
  font={bf, large, sf},
  hypcap=false,
  name={Prg:\,},
}
%%%%% SETMINTED %%%%%%%%%%%%%%%%%%%%%%%%%%
\usemintedstyle{emacs}
\renewcommand{\theFancyVerbLine}{%
  \ttfamily\color{black!75!}{%
    \small\oldstylenums{\arabic{FancyVerbLine}}
    }%
}
\setminted{
  breaklines=true,
  breakbefore=-,
  frame=single,
  linenos,
  numbers=left,
  numbersep=-3pt,
  mathescape,
  ignorelexererrors,
}
\tcbset{default minted options={fontsize=\large, baselinestretch=1.05}}
\newtcbinputlisting{\GcodeInputminted}[2][{"./RfCPN_a1_includefiles/g_code.py:GCodeLexer -x"}]{
  listing engine=minted,
  minted language=#1,
  listing file=#2,
  opacityback=0,
  opacityframe=0,
  boxsep=0pt,
  top=0pt,
  bottom=0pt,
  left=0pt,
  right=0pt,
  before skip=0pt,
  after skip=0pt,
  left skip=0pt,
  right skip=0pt,
  toprule=0pt,
  bottomrule=0pt,
  leftrule=0pt,
  rightrule=0pt,
  listing only,
  breakable,
}
