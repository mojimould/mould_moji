%!TEX root = ../RfCPN.tex


\titlehead{\hfill\small 発行日時 (Publication date)\\~\hfill\customdate}
%\subject{\Large--- \TheDocSubject{} ---}
\title{\huge\TheDocTitle}
\subtitle{\ \\(\thisdocumentversion)}
\date{}
\author{\ \\{\small Author:} \fontspec[Path=./RfCPN_a1_includefiles/RfCPN_1_fonts/]{HuiFontP29.ttf}{\TheDocAuthor}}

\publishers{%
\begin{marker}
\scriptsize\normalfont
本書は一時的に複製されたものです。\\
発行日から6ヶ月以上経過した場合、または新たなバージョンのものが存在する場合は、本書は速やかに破棄してください。
\tcbline*
This document is a temporary copy.
If more than 6 months have passed since the publication date, or if a new version exists, please dispose of this document promptly.
\end{marker}%
}

\uppertitleback{%
{\thispagestyle{plainheadfront}%
\small%
\begin{enumerate}[label={\sarrow}, leftmargin=8pt]
\item
This document was created using \nameTeX{} (\linkLaTeXe, \linkLaTeXthree), specifically utilizing tools such as \linkTeXLiveYear, \linkTLContrib, \linkLuaTeX{} (\linkLuaLaTeX, \linkLuaTeXja, \linkLua), \linkKOMAScript{} (\linkScrbook, \linkScrlayerscrpage) and \linkPGFTikZ, and many useful packages, libraries and modules.
\item
The \nameTeX{} documents were edited using \linkTeXstudio, \linkSumatraPDF{} and \linkSyncTeX.
\item
The some documents were edited using \linkExcel{} (\linkExcelVBA) and \linkPython{} (\linkOpenPyXL, \linkxlwings).
\item
For the bibliography, \linkBibTeX{} (\linkBibLaTeX{}, \linkBiber) was used.
\item
For the index, \linkUpmendex{} and \linkMendexDoc{} (\linkJpbase) were used.
\item
For displaying source code and its syntax highlighting, \linkMinted{} (\linkPygments) was used.
\item
Additional utilities such as \linkTlmgr{} and \linkTeXLiveUtility{} were also employed to enhance the functionality and manage the packages of \linkTeXLive.
\item
The document was typeset using \linkEuler, \linkAMSFonts, \linkLatinModern, \linkHaranoAji, \linkArvo, \linkRoboto, \linkSourceSansPro, \linkTrajanPro, \linkURWArial, \linkURWClassico, \linkNimbusSans, \linkSegoeUI, \linkTestSohne, \linkStereoGothic, \linkYuGothic, \linkConsolas, \linkProductSans{} and \linkHuiFont{} font families.
\item
The analytical approximations were computed using \linkWolframAlpha.
\item
Numerical calculations were performed using \linkPython.
\item
The source codes for the \nameGcode{} (\nameNCProgram) on \nameArumatik{} were written and editting using \linkVSCode{} (\linkNcgcode{} and many useful extensions).
\item
Version, release and issue control for these documents were managed using \linkGit{} and \linkGitHub{} (\linkGitHubDesktop).
\item
For internet connectivity, \linkRakutenMobile{} and sometimes \linkGooglePublecDNS{} were greatly utilized.
%\item
%The environment for these tools was managed using \linkDocker{} and \linkUbuntu.
%\item
%The database used was \linkSQLite.
\end{enumerate}
Thanks to these tools, with the all-around support of \linkChatGPT-4 (\linkCopilot), the creation of the documents and system was made possible.
This was achieved even while I had to navigate solo, finding my way in the quiet corners, despite being a novice with almost all of these tools, software and languages.
}\\%

\begin{spacing}{0.75}
{\tiny%
Among these tools, this was my first time using \linkTeXLiveYear, \linkTLContrib, \linkLuaTeX, \linkLaTeXthree, \linkScrlayerscrpage, \linkTeXstudio, \linkSumatraPDF, \linkBibTeX, \linkUpmendex, \linkMendexDoc, \nameGcode{} (\nameNCProgram), \nameArumatik, \linkNcgcode, \linkGit, \linkGitHub, \linkPython, \linkMinted, \linkPygments, \linkOpenPyXL, \linkxlwings{} and \linkCopilot.%
}%
\end{spacing}
}

\lowertitleback{%
{\tiny\relax%
\setcounter{page}{1}%
\begin{spacing}{1.2}%
\setlength{\tabcolsep}{1.0pt}%
\hfill\\
\hrulefill\\
Copyright © 2023, 2024.\\
These documents and these systems are owned by the individual writer (the author of this documnet), not any corporation.\\
All rights reserved.
\end{spacing}
}%
}
