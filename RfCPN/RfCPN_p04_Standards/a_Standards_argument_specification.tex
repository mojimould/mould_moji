%!TEX root = ../RfCPN.tex


\modHeadchapter[lot]{引数指定\label{chap:argumentSpecification}}



%%%%%%%%%%%%%%%%%%%%%%%%%%%%%%%%%%%%%%%%%%%%%%%%%%%%%%%%%%
%% section A.01 %%%%%%%%%%%%%%%%%%%%%%%%%%%%%%%%%%%%%%%%%%
%%%%%%%%%%%%%%%%%%%%%%%%%%%%%%%%%%%%%%%%%%%%%%%%%%%%%%%%%%
\modHeadsection{引数の指定}
\index{ひきすう@引数}引数の指定の仕方は2通りある。
これらは一般に\index{ひきすうしてい@引数指定}引数指定Iおよび引数指定IIと呼ばれる。


%%%%%%%%%%%%%%%%%%%%%%%%%%%%%%%%%%%%%%%%%%%%%%%%%%%%%%%%%%
%% subsection A.01.01 %%%%%%%%%%%%%%%%%%%%%%%%%%%%%%%%%%%%
%%%%%%%%%%%%%%%%%%%%%%%%%%%%%%%%%%%%%%%%%%%%%%%%%%%%%%%%%%
\subsection{\index{ひきすうしていI@引数指定I}引数指定I}
\index{ひきすうしていI@引数指定I}引数指定Iでは、\index{ひきすうアドレス@引数アドレス}引数アドレスとして{\ttfamily A}-{\ttfamily Z}のアルファベットをそれぞれ1回ずつ用いることができる。
そのため使用できる引数の数は、アルファベットの数(26個)である。
ただし、通常は{\ttfamily G}, {\ttfamily L}, {\ttfamily N}, {\ttfamily O}, {\ttfamily P}は用いることができず、実質的に21個が使用可能な数である。


%%%%%%%%%%%%%%%%%%%%%%%%%%%%%%%%%%%%%%%%%%%%%%%%%%%%%%%%%%
%% subsection A.01.02 %%%%%%%%%%%%%%%%%%%%%%%%%%%%%%%%%%%%
%%%%%%%%%%%%%%%%%%%%%%%%%%%%%%%%%%%%%%%%%%%%%%%%%%%%%%%%%%
\subsection{\index{ひきすうしていII@引数指定II}引数指定II}
\index{ひきすうしていII@引数指定II}引数指定IIでは、\index{ひきすうアドレス@引数アドレス}引数アドレスとして{\ttfamily A}, {\ttfamily B}, {\ttfamily C}を1回と{\ttfamily I}, {\ttfamily J}, {\ttfamily K}を10組まで用いることができる。
そのため使用できる引数の数は、33個である。

{\ttfamily A}, {\ttfamily B}, {\ttfamily C}にはそれぞれ\hk\ttNum01, \ttNum02, \ttNum03が割り当てられ、{\ttfamily I}, {\ttfamily J}, {\ttfamily K}は入力の順序でマクロの\hk\ttNum04から順番に\hk\ttNum33まで割り当てられる
%% footnote %%%%%%%%%%%%%%%%%%%%%
\footnote{たとえば、{\ttfamily K}の後に{\ttfamily I}を用いた場合でも、{\ttfamily K}: {\ttfamily\ttNum04}, {\ttfamily I}: {\ttfamily\ttNum07}のようになることに注意。
これは\index{ひきすうしていI@引数指定I}引数指定Iでも同様である。}。
%%%%%%%%%%%%%%%%%%%%%%%%%%%%%%%%%



\clearpage
%%%%%%%%%%%%%%%%%%%%%%%%%%%%%%%%%%%%%%%%%%%%%%%%%%%%%%%%%%
%% section A.02 %%%%%%%%%%%%%%%%%%%%%%%%%%%%%%%%%%%%%%%%%%
%%%%%%%%%%%%%%%%%%%%%%%%%%%%%%%%%%%%%%%%%%%%%%%%%%%%%%%%%%
\modHeadsection{\index{ひきすうアドレス@引数アドレス}引数アドレスとその\index{ローカルへんすう(ひきすう)@ローカル変数(引数)}ローカル変数}
原則として、引数の数が22個以上必要でない限り、\DMC では\index{ひきすうしていI@引数指定I}引数指定Iを用いるものとする。


%%%%%%%%%%%%%%%%%%%%%%%%%%%%%%%%%%%%%%%%%%%%%%%%%%%%%%%%%%
%% subsection A.02.01 %%%%%%%%%%%%%%%%%%%%%%%%%%%%%%%%%%%%
%%%%%%%%%%%%%%%%%%%%%%%%%%%%%%%%%%%%%%%%%%%%%%%%%%%%%%%%%%
\subsection{\index{ひきすうしていI@引数指定I}引数指定I 一覧}
\index{ひきすうアドレス(ひきすうしていI)@引数アドレス(引数指定I)}引数指定Iの引数アドレスとそれに対応する\index{ローカルへんすう(ひきすう)@ローカル変数(引数)}ローカル変数は以下の通りである。\\
\noindent%
\begin{minipage}[t]{0.66\textwidth}
%%%%%%%%%%%%%%%%%%%%%%%%%%%%%%%%%%%%%%%%%%%%%%%%%%%%%%%%%%
%% captionof %%%%%%%%%%%%%%%%%%%%%%%%%%%%%%%%%%%%%%%%%%%%%
%%%%%%%%%%%%%%%%%%%%%%%%%%%%%%%%%%%%%%%%%%%%%%%%%%%%%%%%%%
\begin{twocolbreaktblr}{\index{ひきすうしていI@引数指定I}引数指定I 一覧}{cc|[3pt, white!0!]|cc|[3pt, white!0!]|cc|[3pt, white!0!]|cc}
\cmidrule[r=0]{1-2}\cmidrule[lr=0]{3-4}\cmidrule[lr=0]{5-6}\cmidrule[l=0]{7-8}
記号 & 変数 & 記号 & 変数 & 記号 & 変数 & 記号 & 変数\\
\cmidrule[r=0]{1-2}\cmidrule[lr=0]{3-4}\cmidrule[lr=0]{5-6}\cmidrule[l=0]{7-8}
\ttfamily A & \ttfamily\#01 & \ttfamily H & \ttfamily\#11 & \ttfamily R & \ttfamily\#18 & \ttfamily X & \ttfamily\#24\\
\cmidrule[r=0]{1-2}\cmidrule[lr=0]{3-4}\cmidrule[lr=0]{5-6}\cmidrule[l=0]{7-8}
\ttfamily B & \ttfamily\#02 & \ttfamily I & \ttfamily\#04 & \ttfamily S & \ttfamily\#19 & \ttfamily Y & \ttfamily\#25\\
\cmidrule[r=0]{1-2}\cmidrule[lr=0]{3-4}\cmidrule[lr=0]{5-6}\cmidrule[l=0]{7-8}
\ttfamily C & \ttfamily\#03 & \ttfamily J & \ttfamily\#05 & \ttfamily T & \ttfamily\#20 & \ttfamily Z & \ttfamily\#26\\
\cmidrule[r=0]{1-2}\cmidrule[lr=0]{3-4}\cmidrule[lr=0]{5-6}\cmidrule[l=0]{7-8}
\ttfamily D & \ttfamily\#07 & \ttfamily K & \ttfamily\#06 & \ttfamily U & \ttfamily\#21\\
\cmidrule[r=0]{1-2}\cmidrule[lr=0]{3-4}\cmidrule[lr=0]{5-6}\cmidrule[l=0]{7-8}
\ttfamily E & \ttfamily\#08 & \ttfamily M & \ttfamily\#13 & \ttfamily V & \ttfamily\#22\\
\cmidrule[r=0]{1-2}\cmidrule[lr=0]{3-4}\cmidrule[lr=0]{5-6}\cmidrule[l=0]{7-8}
\ttfamily F & \ttfamily\#09 & \ttfamily Q & \ttfamily\#17 & \ttfamily W & \ttfamily\#23\\
\cmidrule[r=0]{1-2}\cmidrule[lr=0]{3-4}\cmidrule[lr=0]{5-6}\cmidrule[l=0]{7-8}
\end{twocolbreaktblr}%
\end{minipage}%
\begin{minipage}[t]{0.34\textwidth}
%%%%%%%%%%%%%%%%%%%%%%%%%%%%%%%%%%%%%%%%%%%%%%%%%%%%%%%%%%
%% captionof %%%%%%%%%%%%%%%%%%%%%%%%%%%%%%%%%%%%%%%%%%%%%
%%%%%%%%%%%%%%%%%%%%%%%%%%%%%%%%%%%%%%%%%%%%%%%%%%%%%%%%%%
\begin{twocolbreaktblr}{通常指定不可な引数}{cc}
\cmidrule{1-Z}
記号 & 変数\\
\cmidrule{1-Z}
\ttfamily G & \ttfamily\#10\\
\cmidrule{1-Z}
\ttfamily L & \ttfamily\#12\\
\cmidrule{1-Z}
\ttfamily N & \ttfamily\#14\\
\cmidrule{1-Z}
\ttfamily O & \ttfamily\#15\\
\cmidrule{1-Z}
\ttfamily P & \ttfamily\#16\\
\cmidrule{1-Z}
\end{twocolbreaktblr}%
\end{minipage}


%%%%%%%%%%%%%%%%%%%%%%%%%%%%%%%%%%%%%%%%%%%%%%%%%%%%%%%%%%
%% subsection A.02.02 %%%%%%%%%%%%%%%%%%%%%%%%%%%%%%%%%%%%
%%%%%%%%%%%%%%%%%%%%%%%%%%%%%%%%%%%%%%%%%%%%%%%%%%%%%%%%%%
\subsection{\index{ひきすうしていII@引数指定II}引数指定II 一覧}
\index{ひきすうアドレス(ひきすうしていII)@引数アドレス(引数指定II)}引数指定IIの引数アドレスとそれに対応する\index{ローカルへんすう(ひきすう)@ローカル変数(引数)}ローカル変数は以下の通りである。
なお、{\ttfamily I}, {\ttfamily J}, {\ttfamily K}の添字は便宜上記述したものであり、実際の\index{NCプログラム}NCプログラムにはすべて同じ{\ttfamily I}, {\ttfamily J}, {\ttfamily K}の記号を用いる。\\

%%%%%%%%%%%%%%%%%%%%%%%%%%%%%%%%%%%%%%%%%%%%%%%%%%%%%%%%%%
%% captionof %%%%%%%%%%%%%%%%%%%%%%%%%%%%%%%%%%%%%%%%%%%%%
%%%%%%%%%%%%%%%%%%%%%%%%%%%%%%%%%%%%%%%%%%%%%%%%%%%%%%%%%%
\begin{twocolbreaktblr}{\index{ひきすうしていII@引数指定II}引数指定II 一覧}{cc|[3pt, white!0!]|cc|[3pt, white!0!]|cc|[3pt, white!0!]|cc}
\cmidrule[r=0]{1-2}\cmidrule[lr=0]{3-4}\cmidrule[lr=0]{5-6}\cmidrule[l=0]{7-8}
記号 & 変数 & 記号 & 変数 & 記号 & 変数 & 記号 & 変数\\
\cmidrule[r=0]{1-2}\cmidrule[lr=0]{3-4}\cmidrule[lr=0]{5-6}\cmidrule[l=0]{7-8}
\ttfamily A & \ttfamily\#01 & \ttfamily I$_3$ & \ttfamily\#10 & \ttfamily I$_6$ & \ttfamily\#19 & \ttfamily I$_9$ & \ttfamily\#28\\
\cmidrule[r=0]{1-2}\cmidrule[lr=0]{3-4}\cmidrule[lr=0]{5-6}\cmidrule[l=0]{7-8}
\ttfamily B & \ttfamily\#02 & \ttfamily J$_3$ & \ttfamily\#11 & \ttfamily J$_6$ & \ttfamily\#20 & \ttfamily J$_9$ & \ttfamily\#29\\
\cmidrule[r=0]{1-2}\cmidrule[lr=0]{3-4}\cmidrule[lr=0]{5-6}\cmidrule[l=0]{7-8}
\ttfamily C & \ttfamily\#03 & \ttfamily K$_3$ & \ttfamily\#12 & \ttfamily K$_6$ & \ttfamily\#21 & \ttfamily K$_9$ & \ttfamily\#30\\
\cmidrule[r=0]{1-2}\cmidrule[lr=0]{3-4}\cmidrule[lr=0]{5-6}\cmidrule[l=0]{7-8}
\ttfamily I$_1$ & \ttfamily\#04 & \ttfamily I$_4$ & \ttfamily\#13 & \ttfamily I$_7$ & \ttfamily\#22 & \ttfamily I$_{10}$ & \ttfamily\#31\\
\cmidrule[r=0]{1-2}\cmidrule[lr=0]{3-4}\cmidrule[lr=0]{5-6}\cmidrule[l=0]{7-8}
\ttfamily J$_1$ & \ttfamily\#05 & \ttfamily J$_4$ & \ttfamily\#14 & \ttfamily J$_7$ & \ttfamily\#23 & \ttfamily J$_{10}$ & \ttfamily\#32\\
\cmidrule[r=0]{1-2}\cmidrule[lr=0]{3-4}\cmidrule[lr=0]{5-6}\cmidrule[l=0]{7-8}
\ttfamily K$_1$ & \ttfamily\#06 & \ttfamily K$_4$ & \ttfamily\#15 & \ttfamily K$_7$ & \ttfamily\#24 & \ttfamily K$_{10}$ & \ttfamily\#33\\
\cmidrule[r=0]{1-2}\cmidrule[lr=0]{3-4}\cmidrule[lr=0]{5-6}\cmidrule[l=0]{7-8}
\ttfamily I$_2$ & \ttfamily\#07 & \ttfamily I$_5$ & \ttfamily\#16 & \ttfamily I$_8$ & \ttfamily\#25\\
\cmidrule[r=0]{1-2}\cmidrule[lr=0]{3-4}\cmidrule[lr=0]{5-6}\cmidrule[l=0]{7-8}
\ttfamily J$_2$ & \ttfamily\#08 & \ttfamily J$_5$ & \ttfamily\#17 & \ttfamily J$_8$ & \ttfamily\#26\\
\cmidrule[r=0]{1-2}\cmidrule[lr=0]{3-4}\cmidrule[lr=0]{5-6}\cmidrule[l=0]{7-8}
\ttfamily K$_2$ & \ttfamily\#09 & \ttfamily K$_5$ & \ttfamily\#18 & \ttfamily K$_8$ & \ttfamily\#27\\
\cmidrule[r=0]{1-2}\cmidrule[lr=0]{3-4}\cmidrule[lr=0]{5-6}\cmidrule[l=0]{7-8}
\end{twocolbreaktblr}%
