%!TEX root = ./RfCPN.tex


\modHeadchapter{\index{NCメインプログラム}NCメインプログラムに必要な数値情報}
ここでは主に\index{NCメインプログラム}NCメインプログラムの記述に際して、\index{すうちけいさん@数値計算}数値計算の必要な部分をピックアップする。
なお、ここでは主に\DMC について述べるため、\Spacer に関するものは省略する。



%%%%%%%%%%%%%%%%%%%%%%%%%%%%%%%%%%%%%%%%%%%%%%%%%%%%%%%%%%
%% section 40.01 %%%%%%%%%%%%%%%%%%%%%%%%%%%%%%%%%%%%%%%%%
%%%%%%%%%%%%%%%%%%%%%%%%%%%%%%%%%%%%%%%%%%%%%%%%%%%%%%%%%%
\modHeadsection{\AlocationAngle の数値情報}
\TopAlocationLength が$f_\mathrm T'$となる\AlocationAngle は、\pageeqref{eq:alocationangle}より、
\begin{align*}
  \text{トップ側:}~\HLbox{\sin^{-1}\frac{f_\mathrm T'-f_\mathrm T}{\Delta_x'}}~,\quad
  \text{ボトム側:}~\HLbox{\pi-\sin^{-1}\frac{f_\mathrm T'-f_\mathrm T}{\Delta_x'}}\ .
\end{align*}
特に、\TopAlocationLength および\BottomAlocationLength が同じとなる\EqualAlocationAngle は、\pageeqref{eq:equalalocationangle}より、
\begin{align*}
  \text{トップ側:}~\HLbox{\sin^{-1}\frac{f_d}{\Delta_x'}}~,\quad
  \text{ボトム側:}~\HLbox{\pi-\sin^{-1}\frac{f_d}{\Delta_x'}}\ .
\end{align*}



\clearpage
%%%%%%%%%%%%%%%%%%%%%%%%%%%%%%%%%%%%%%%%%%%%%%%%%%%%%%%%%%
%% section F.2 %%%%%%%%%%%%%%%%%%%%%%%%%%%%%%%%%%%%%%%%%%%
%%%%%%%%%%%%%%%%%%%%%%%%%%%%%%%%%%%%%%%%%%%%%%%%%%%%%%%%%%
\modHeadsection{\index{ワークざひょうけいげんてんせってい@ワーク座標系原点設定}ワーク座標系原点設定用の数値情報}


%%%%%%%%%%%%%%%%%%%%%%%%%%%%%%%%%%%%%%%%%%%%%%%%%%%%%%%%%%
%% subsection F.02.01 %%%%%%%%%%%%%%%%%%%%%%%%%%%%%%%%%%%%
%%%%%%%%%%%%%%%%%%%%%%%%%%%%%%%%%%%%%%%%%%%%%%%%%%%%%%%%%%
\subsection{ボトム端外側中心の位置}
ボトム端における外側中心の$X$座標は、\pageeqref{eq:tableBc}より、
\begin{align*}
  \HLbox{\Delta_x'\cos\theta-\frac{\sqrt{R_\mathrm o^2-f_\mathrm B^2}+\sqrt{R_\mathrm i^2-f_\mathrm B^2}}2}\ .
\end{align*}


%%%%%%%%%%%%%%%%%%%%%%%%%%%%%%%%%%%%%%%%%%%%%%%%%%%%%%%%%%
%% subsection F.02.02 %%%%%%%%%%%%%%%%%%%%%%%%%%%%%%%%%%%%
%%%%%%%%%%%%%%%%%%%%%%%%%%%%%%%%%%%%%%%%%%%%%%%%%%%%%%%%%%
\subsection{\BottomOutcutCenter の位置}

%%%%%%%%%%%%%%%%%%%%%%%%%%%%%%%%%%%%%%%%%%%%%%%%%%%%%%%%%%
%% subsubsection F.02.02.1 %%%%%%%%%%%%%%%%%%%%%%%%%%%%%%%
%%%%%%%%%%%%%%%%%%%%%%%%%%%%%%%%%%%%%%%%%%%%%%%%%%%%%%%%%%
\subsubsection{\BottomOutcutCenter:\BottomOutcutAsideThickness 基準の場合}
\TableCenter を\index{げんてんP@原点P}原点とした\BottomOutcutCenter の(おおよその)$X$座標は、\pageeqref{eq:gaisakucenterBt}より、
\begin{align*}
  \HLbox{%
    G_{\mathrm Bx}
    -\frac{\sqrt{R_\mathrm c^2-(f_\mathrm B-c_\mathrm{Bi})^2}+\sqrt{R_\mathrm c^2-f_\mathrm B^2}}2
    -\frac{w_\mathrm B}2-\tau_\mathrm B+\frac{\mathfrak W_\mathrm B}2
  }\ .
\end{align*}

%%%%%%%%%%%%%%%%%%%%%%%%%%%%%%%%%%%%%%%%%%%%%%%%%%%%%%%%%%
%% subsubsection F.02.02.2 %%%%%%%%%%%%%%%%%%%%%%%%%%%%%%%
%%%%%%%%%%%%%%%%%%%%%%%%%%%%%%%%%%%%%%%%%%%%%%%%%%%%%%%%%%
\subsubsection{\BottomOutcutCenter:\TopOutcutAsideThickness 基準の場合}
\TopOutcutCenter$\mathcal G_{Tx}$が基準の場合、\BottomOutcutCenter の(おおよその)$X$座標は、\pageeqref{eq:gaisakucenterBt}より、
\begin{align*}
  \HLbox{-\mathcal G_{Tx}+T_x}\ .
\end{align*}


%%%%%%%%%%%%%%%%%%%%%%%%%%%%%%%%%%%%%%%%%%%%%%%%%%%%%%%%%%
%% subsection F.02.03 %%%%%%%%%%%%%%%%%%%%%%%%%%%%%%%%%%%%
%%%%%%%%%%%%%%%%%%%%%%%%%%%%%%%%%%%%%%%%%%%%%%%%%%%%%%%%%%
\subsection{ボトム端内側中心の位置}
ボトム端における内側中心の(おおよその)$X$座標は、\BottomCurvatureCenter\pageeqref{eq:tableBRc}として与えられる。
\begin{align*}
  \HLbox{\Delta_x'\cos\theta-\sqrt{R_\mathrm c^2-f_\mathrm B^2}}\ .
\end{align*}


%%%%%%%%%%%%%%%%%%%%%%%%%%%%%%%%%%%%%%%%%%%%%%%%%%%%%%%%%%
%% subsection F.02.04 %%%%%%%%%%%%%%%%%%%%%%%%%%%%%%%%%%%%
%%%%%%%%%%%%%%%%%%%%%%%%%%%%%%%%%%%%%%%%%%%%%%%%%%%%%%%%%%
\subsection{トップ端外側中心の位置}
トップ端における外側中心の$X$座標は、\pageeqref{eq:tableTc}より、
\begin{align*}
  \HLbox{\frac{\sqrt{R_\mathrm o^2-f_\mathrm T^2}+\sqrt{R_\mathrm i^2-f_\mathrm T^2}}2-\Delta_x'\cos\theta}\ .
\end{align*}


\clearpage
%%%%%%%%%%%%%%%%%%%%%%%%%%%%%%%%%%%%%%%%%%%%%%%%%%%%%%%%%%
%% subsection F.02.05 %%%%%%%%%%%%%%%%%%%%%%%%%%%%%%%%%%%%
%%%%%%%%%%%%%%%%%%%%%%%%%%%%%%%%%%%%%%%%%%%%%%%%%%%%%%%%%%
\subsection{\TopOutcutCenter の位置}

%%%%%%%%%%%%%%%%%%%%%%%%%%%%%%%%%%%%%%%%%%%%%%%%%%%%%%%%%%
%% subsubsection F.02.05.1 %%%%%%%%%%%%%%%%%%%%%%%%%%%%%%%
%%%%%%%%%%%%%%%%%%%%%%%%%%%%%%%%%%%%%%%%%%%%%%%%%%%%%%%%%%
\subsubsection{\TopOutcutCenter:\TopOutcutAsideThickness 基準の場合}
\TableCenter を\index{げんてんP@原点P}原点とした\TopOutcutCenter の(おおよその)$X$座標は、\pageeqref{eq:gaisakucenterTt}より、
\begin{align*}
  \HLbox{%
    G_{\mathrm Tx}
    +\frac{\sqrt{R_\mathrm c^2-(f_\mathrm T-c_\mathrm{Ti})^2}+\sqrt{R_\mathrm c^2-f_\mathrm T^2}}2
    +\frac{w_\mathrm T}2+\tau_\mathrm T-\frac{\mathfrak W_\mathrm T}2
  }\ .
\end{align*}

%%%%%%%%%%%%%%%%%%%%%%%%%%%%%%%%%%%%%%%%%%%%%%%%%%%%%%%%%%
%% subsubsection F.02.05.2 %%%%%%%%%%%%%%%%%%%%%%%%%%%%%%%
%%%%%%%%%%%%%%%%%%%%%%%%%%%%%%%%%%%%%%%%%%%%%%%%%%%%%%%%%%
\subsubsection{\TopOutcutCenter:\BottomOutcutAsideThickness 基準の場合}
\BottomOutcutCenter$\mathcal G_{Bx}$が基準の場合、\TopOutcutCenter の(おおよその)$X$座標は、\pageeqref{eq:BbasedTx}より、
\begin{align*}
  \HLbox{-\mathcal G_{Bx}+T_x}\ .
\end{align*}


%%%%%%%%%%%%%%%%%%%%%%%%%%%%%%%%%%%%%%%%%%%%%%%%%%%%%%%%%%
%% subsection F.02.04 %%%%%%%%%%%%%%%%%%%%%%%%%%%%%%%%%%%%
%%%%%%%%%%%%%%%%%%%%%%%%%%%%%%%%%%%%%%%%%%%%%%%%%%%%%%%%%%
\subsection{\KeywayCenter の位置}
\Outcut がなく、かつ\AsideKeywayDepth が基準の場合、\KeywayCenter の位置の$X$座標は\pageeqref{eq:mizocenterA}より、
\begin{align*}
  \HLbox{%
    \sqrt{R_\mathrm o^2-\left(f_\mathrm T-\kappa_p-\frac{\kappa_w}2\right)^2}
    -\kappa_d
    -\frac{W_{mx}}2
    -\Delta_x%
  }\ .
\end{align*}



%%%%%%%%%%%%%%%%%%%%%%%%%%%%%%%%%%%%%%%%%%%%%%%%%%%%%%%%%%
%% section H.2 %%%%%%%%%%%%%%%%%%%%%%%%%%%%%%%%%%%%%%%%%%%
%%%%%%%%%%%%%%%%%%%%%%%%%%%%%%%%%%%%%%%%%%%%%%%%%%%%%%%%%%
\modHeadsection{\Dimple の数値情報}
\DimpleAngle は、\pageeqref{eq:dKatamuki}より、
\begin{align*}
  \HLbox{%
  \max\left[
  \tan^{-1}\frac{\displaystyle
           \sqrt{\left(R_\mathrm c+\frac{w'_{\mathrm Aq}}2\right)^2-(f_\mathrm T-q)^2}
           -\sqrt{\left(R_\mathrm c+\frac{w'_{\mathrm A0}}2\right)^2-f_\mathrm T^2}}q~,~0
  \right]%
  }\ .
\end{align*}
