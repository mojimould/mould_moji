%!TEX root = ./RfCPN.tex


\addtocontents{toc}{\protect\cleardoublepage}
%%%%%%%%%%%%%%%%%%%%%%%%%%%%%%%%%%%%%%%%%%%%%%%%%%%%%%%%%
%% Part Numerical Calculation %%%%%%%%%%%%%%%%%%%%%%%%%%%
%%%%%%%%%%%%%%%%%%%%%%%%%%%%%%%%%%%%%%%%%%%%%%%%%%%%%%%%%
\addtocontents{toc}{\protect\begin{tocBox}{\tmppartnum}}%
\tPart{解析計算に基づく数値解析\label{part:NC}}{%
\paragraph*{\tpartgoal}
\index{めいさい(モールド)@明細(モールド)}明細ごとに異なる\index{すんぽう@寸法}寸法・形状を持つすべての\index{ワーク}ワークに対し、\index{NCプログラム}NCプログラムの作成に必要な数値情報および\index{じょうけんぶんきじょうほう@条件分岐情報}条件分岐情報等が自動的に得られるシステムを構築する。
\tcbline*
\paragraph*{\tpartmethod}
前段階で導出した解析的な情報を用いて、各明細における具体的な数値的な情報に自動的に変換するシステムの構築を試みる。
\tcbline*
\paragraph*{\tpartbackground}
一般に、ワークの形状や使用する工具は明細ごとに異なり、固有の寸法・形状を持つ。
NCプログラムの作成の際には、それらをすべて考慮した具体的な数値情報が必要となる。
こうした数値情報は明細ごとに膨大にあるが、現時点(\DMC 設置時点)において、こうした手続きは明細ごとに手作業で行われている。。

 したがって、こうした手続きのシステム化を行い、可能な限り自動化することが喫緊の課題である。
そうすることで、\textbf{危険を伴う作業の削減(\index{あんぜんせい@安全性}安全性の向上)}や、\textbf{品質の低下の防止}に大きく寄与できることが期待される。
また副次的効果として、作業効率・人的資源・\index{ほしゅ@保守}保守などのいずれの観点からみた\textbf{能率の低下の防止}にも大きく貢献することも自ずと期待される。
}{%
\paragraph*{\tpartconclusion}
各明細のワークにおける固有数値情報の入力により、\index{NCプログラム}NCプログラムの作成に必要な数値情報および\index{じょうけんぶんきじょうほう@条件分岐情報}条件分岐情報等が(手動計算を介することなく)自動的に得られるシステムを構築した。
\tcbline*
\paragraph*{\tpartnextstep}
加工システムの具体的な設計を行う。
}

%%%%%%%%%%%%%%%%%%%%%%%%%%%%%%%%%%%%%%%%%%%%%%%%%%%%%%%%%%
%% chapters %%%%%%%%%%%%%%%%%%%%%%%%%%%%%%%%%%%%%%%%%%%%%%
%%%%%%%%%%%%%%%%%%%%%%%%%%%%%%%%%%%%%%%%%%%%%%%%%%%%%%%%%%
%!TEX root = ./RfCPN.tex


\modHeadchapter[lot]{入力する数値情報・パラメータ}
\index{すんぽう@寸法}寸法・形状等の数値情報や\index{じょうけんぶんきじょうほう@条件分岐情報}条件分岐情報は、\index{めいさい(モールド)@明細(モールド)}明細ごとに固有である。
そのため、その\index{こゆうじょうほう(ワーク)@固有情報(ワーク)}固有情報は入力する必要がある。
ここではそうした入力する必要のある情報をまとめておく。
なお、\index{にゅうりょくするすうちじょうほう@入力する数値情報}入力する数値情報に関しては、原則として\index{ずめん@図面}図面上の\index{すんぽう@寸法}寸法をそのまま入力する形となるような方針とする
%% footnote %%%%%%%%%%%%%%%%%%%%%
\footnote{ただし、これらの値にはそれぞれの\index{こうさ@公差}公差が考慮されている。}。
%%%%%%%%%%%%%%%%%%%%%%%%%%%%%%%%%



%%%%%%%%%%%%%%%%%%%%%%%%%%%%%%%%%%%%%%%%%%%%%%%%%%%%%%%%%%
%% section 30.1 %%%%%%%%%%%%%%%%%%%%%%%%%%%%%%%%%%%%%%%%%%
%%%%%%%%%%%%%%%%%%%%%%%%%%%%%%%%%%%%%%%%%%%%%%%%%%%%%%%%%%
\modHeadsection{湾曲・\Alocation に関する入力数値}

\begin{multicollongtblr}{入力情報:湾曲・\Alocation}{X[l]c}
内容 & 型\\
\CenterCurvatureExists & boolean\\
\CenterCurvatureRadius & float\\
\TopAlocationLength & float\\
\BottomAlocationLength & float\\
\end{multicollongtblr}



%%%%%%%%%%%%%%%%%%%%%%%%%%%%%%%%%%%%%%%%%%%%%%%%%%%%%%%%%%
%% section 30.2 %%%%%%%%%%%%%%%%%%%%%%%%%%%%%%%%%%%%%%%%%%
%%%%%%%%%%%%%%%%%%%%%%%%%%%%%%%%%%%%%%%%%%%%%%%%%%%%%%%%%%
\modHeadsection{外形・内形に関する入力数値}

\begin{multicollongtblr}{入力情報:外形}{X[l]c}
内容 & 型\\
\ACOD & float\\
\BDOD & float\\
\ODCornerR & float\\
\end{multicollongtblr}

\begin{multicollongtblr}{入力情報:内形}{X[l]c}
内容 & 型\\
\PlatingThk & float\\
\IDTaperTableNum & list\\
\end{multicollongtblr}
%%%%%%%%%%%%%%%%%%%%%%%%%%%%%%%%%%%%%%%%%%%%%%%%%%%%%%%%%%
%% marker %%%%%%%%%%%%%%%%%%%%%%%%%%%%%%%%%%%%%%%%%%%%%%%%
%%%%%%%%%%%%%%%%%%%%%%%%%%%%%%%%%%%%%%%%%%%%%%%%%%%%%%%%%%
\begin{marker}
\IDTaperTable には、\InnerCornerR の\index{すんぽう(\yomiInnerCornerR)@寸法(\nameInnerCornerR)}寸法情報も記載されているものとする。
\end{marker}
%%%%%%%%%%%%%%%%%%%%%%%%%%%%%%%%%%%%%%%%%%%%%%%%%%%%%%%%%%
%%%%%%%%%%%%%%%%%%%%%%%%%%%%%%%%%%%%%%%%%%%%%%%%%%%%%%%%%%
%%%%%%%%%%%%%%%%%%%%%%%%%%%%%%%%%%%%%%%%%%%%%%%%%%%%%%%%%%



\clearpage
%%%%%%%%%%%%%%%%%%%%%%%%%%%%%%%%%%%%%%%%%%%%%%%%%%%%%%%%%%
%% section 30.3 %%%%%%%%%%%%%%%%%%%%%%%%%%%%%%%%%%%%%%%%%%
%%%%%%%%%%%%%%%%%%%%%%%%%%%%%%%%%%%%%%%%%%%%%%%%%%%%%%%%%%
\modHeadsection{\Outcut に関する入力数値}

\begin{multicollongtblr}{入力情報:\BottomOutcut}{X[l]c}
内容 & 型\\
\BottomOutcutExists & boolean\\
\index{こうぐせんたんテーパのうむ@工具先端テーパの有無}工具先端テーパの有無 & boolean\\
\BottomCurvedOutcutExists & boolean\\
\BottomOutcutAsideThickness & float\\
\BottomOutcutACwidth & float\\
\BottomOutcutBDwidth & float\\
\BottomOutcutLength & float\\
\BottomOutcutConerR & float\\
\end{multicollongtblr}

\begin{multicollongtblr}{入力情報:\TopOutcut}{X[l]c}
内容 & 型\\
\TopOutcutExists & boolean\\
\index{こうぐせんたんテーパのうむ@工具先端テーパの有無}工具先端テーパの有無 & boolean\\
\TopCurvedOutcutExists & boolean\\
\TopOutcutAsideThickness & float\\
\TopOutcutACwidth & float\\
\TopOutcutBDwidth & float\\
\TopOutcutLength & float\\
\TopOutcutCornerR & float\\
\end{multicollongtblr}

\begin{multicollongtblr}{入力情報:両外削}{X[l]c}
内容 & 型\\
\OutcutCenter 基準 & enum\\
\CenterlineEndFaceDif & float\\
\end{multicollongtblr}



\clearpage
%%%%%%%%%%%%%%%%%%%%%%%%%%%%%%%%%%%%%%%%%%%%%%%%%%%%%%%%%%
%% section 30.4 %%%%%%%%%%%%%%%%%%%%%%%%%%%%%%%%%%%%%%%%%%
%%%%%%%%%%%%%%%%%%%%%%%%%%%%%%%%%%%%%%%%%%%%%%%%%%%%%%%%%%
\modHeadsection{\Keyway に関する入力数値}

\begin{multicollongtblr}{入力情報:\Keyway}{X[l]c}
内容 & 型\\
\KeywayType & enum\\
\AsideKeywayDepth 指定 有無 & boolean\\
\KeywayACOD & float\\
\KeywayBDOD & float\\
\KeywayPos & float\\
\KeywayWidth & float\\
\AsideKeywayDepth & float\\
\KeywayCornerR & float\\
\KeywayCornerC & float\\
\end{multicollongtblr}



%\clearpage
%%%%%%%%%%%%%%%%%%%%%%%%%%%%%%%%%%%%%%%%%%%%%%%%%%%%%%%%%%
%% section 30.5 %%%%%%%%%%%%%%%%%%%%%%%%%%%%%%%%%%%%%%%%%%
%%%%%%%%%%%%%%%%%%%%%%%%%%%%%%%%%%%%%%%%%%%%%%%%%%%%%%%%%%
\modHeadsection{\Dimple に関する入力数値}

\begin{multicollongtblr}{入力情報:\Dimple}{X[l]c}
内容 & 型\\
\DimpleExists & boolean\\
トップ端と\DimpleFirstRow までの距離 & float\\
\DimpleVerticalPitch & float\\
\DimpleHorizontalPitch & float\\
\DimpleOddRowLength & float\\
\DimpleEvenRowLength & float\\
\DimpleRowNum & enum\\
\DimpleDepth & float\\
\DimpleRadius(工具小半径) & float\\
\end{multicollongtblr}



\clearpage
%%%%%%%%%%%%%%%%%%%%%%%%%%%%%%%%%%%%%%%%%%%%%%%%%%%%%%%%%%
%% section 30.6 %%%%%%%%%%%%%%%%%%%%%%%%%%%%%%%%%%%%%%%%%%
%%%%%%%%%%%%%%%%%%%%%%%%%%%%%%%%%%%%%%%%%%%%%%%%%%%%%%%%%%
\modHeadsection{\EndFaceChamfer に関する入力数値}


%%%%%%%%%%%%%%%%%%%%%%%%%%%%%%%%%%%%%%%%%%%%%%%%%%%%%%%%%%
%% subsection 30.6.1 %%%%%%%%%%%%%%%%%%%%%%%%%%%%%%%%%%%%%
%%%%%%%%%%%%%%%%%%%%%%%%%%%%%%%%%%%%%%%%%%%%%%%%%%%%%%%%%%
\subsection{\BottomEndFaceChamfer に関する入力情報}

\begin{multicollongtblr}{入力情報:\BottomEndFaceOutCChamfer}{X[l]c}
内容 & 型\\
\BottomEndFaceOutCChamferExists & boolean\\
\BottomEndFaceOutCChamferLength & float\\
\BottomEndFaceOutCChamferAngle & float\\
\end{multicollongtblr}

\begin{multicollongtblr}{入力情報:\BottomEndFaceOutRChamfer}{X[l]c}
内容 & 型\\
\BottomEndFaceOutRChamferExists & boolean\\
\BottomEndFaceOutRChamferRadius & float\\
\end{multicollongtblr}

\begin{multicollongtblr}{入力情報:\BottomEndFaceInCChamfer}{X[l]c}
内容 & 型\\
\BottomEndFaceInCChamferExists & boolean\\
\BottomEndFaceInCChamferLength & float\\
\BottomEndFaceInCChamferAngle & integer\\
\end{multicollongtblr}

\begin{multicollongtblr}{入力情報:\BottomEndFaceInRChamfer}{X[l]c}
内容 & 型\\
\BottomFaceInRChamferExsits & boolean\\
\BottomFaceInRChamferRadius & float\\
\end{multicollongtblr}



\clearpage
%%%%%%%%%%%%%%%%%%%%%%%%%%%%%%%%%%%%%%%%%%%%%%%%%%%%%%%%%%
%% subsection 30.6.2 %%%%%%%%%%%%%%%%%%%%%%%%%%%%%%%%%%%%%
%%%%%%%%%%%%%%%%%%%%%%%%%%%%%%%%%%%%%%%%%%%%%%%%%%%%%%%%%%
\subsection{\TopEndFaceChamfer に関する入力情報}

\begin{multicollongtblr}{入力情報:\TopEndFaceOutCChamfer}{X[l]c}
内容 & 型\\
\TopEndFaceOutCChamferExists & boolean\\
\TopEndFaceOutCChamferLength & float\\
\TopEndFaceOutCChamferAngle & integer\\
\end{multicollongtblr}

\begin{multicollongtblr}{入力情報:\TopEndFaceOutRChamfer}{X[l]c}
内容 & 型\\
\TopEndFaceOutRChamferExists & boolean\\
\TopEndFaceOutRChamferRadius & float\\
\end{multicollongtblr}

\begin{multicollongtblr}{入力情報:\TopEndFaceInCChamfer}{X[l]c}
内容 & 型\\
\TopEndFaceInCChamferExists & boolean\\
\TopEndFaceInCChamferLength & float\\
\TopEndFaceInCChamferAngle & integer\\
\end{multicollongtblr}

\begin{multicollongtblr}{入力情報:\TopEndFaceInRChamfer}{X[l]c}
内容 & 型\\
\TopEndFaceInRChamferExists & boolean\\
\TopEndFaceInRChamferRadius & float\\
\end{multicollongtblr}



%\clearpage
%%%%%%%%%%%%%%%%%%%%%%%%%%%%%%%%%%%%%%%%%%%%%%%%%%%%%%%%%%
%% section 30.6 %%%%%%%%%%%%%%%%%%%%%%%%%%%%%%%%%%%%%%%%%%
%%%%%%%%%%%%%%%%%%%%%%%%%%%%%%%%%%%%%%%%%%%%%%%%%%%%%%%%%%
\modHeadsection{\EndFaceBoring に関する入力情報}

\begin{multicollongtblr}{入力情報:\EndFaceBoring}{X[l]c}
内容 & 型\\
\EndFaceBoringExists & boolean\\
\EndFaceBoringWidth & float\\
\EndFaceBoringDepth & float\\
\EndFaceBoringCornerR & float\\
\EndFaceBoringLength & float\\
\end{multicollongtblr}



\clearpage
%%%%%%%%%%%%%%%%%%%%%%%%%%%%%%%%%%%%%%%%%%%%%%%%%%%%%%%%%%
%% section 30.6 %%%%%%%%%%%%%%%%%%%%%%%%%%%%%%%%%%%%%%%%%%
%%%%%%%%%%%%%%%%%%%%%%%%%%%%%%%%%%%%%%%%%%%%%%%%%%%%%%%%%%
\modHeadsection{\IncutBoring に関する入力情報}

\begin{multicollongtblr}{入力情報:\IncutBoring}{X[l]c}
内容 & 型\\
\IncutBoringExists & boolean\\
\IncutBoringACWidth & float\\
\IncutBoringBDWidth & float\\
\IncutBoringCornerR & float\\
\IncutBoringLength & float\\
\end{multicollongtblr}


\clearrightpage

%!TEX root = ./RfCPN.tex


\modHeadchapter[lot]{必要な条件分岐情報(\index{NCプログラム}NCプログラム)}
入力するパラメタの中には\index{じょうけんぶんき@条件分岐}条件分岐に用いるものが含まれる。
それらを組み合わせることで、加工にパターン分けをすることができる。
ここではそうした\index{じょうけんぶんきじょうほう@条件分岐情報}条件分岐情報と、それに直接的に影響する\index{こうてい@工程}工程またはパラメタをまとめておく。



%%%%%%%%%%%%%%%%%%%%%%%%%%%%%%%%%%%%%%%%%%%%%%%%%%%%%%%%%%
%% section 46.01 %%%%%%%%%%%%%%%%%%%%%%%%%%%%%%%%%%%%%%%%%
%%%%%%%%%%%%%%%%%%%%%%%%%%%%%%%%%%%%%%%%%%%%%%%%%%%%%%%%%%
\modHeadsection{湾曲・\Alocation に関する\index{じょうけんぶんき@条件分岐}条件分岐}

\begin{multicollongtblr}{\index{じょうけんぶんき@条件分岐}条件分岐:\PMCenterCurvatureExists}{X[l]}
直接的に影響する工程\\
ワーク座標系 原点設定\\
外側中心$X$および幅$X$ 両側測定\\
\KeywayCenter$X$ 片側測定\\
内側中心$X$および幅$X$ 両側測定\\
\OutcutCenter$X$ 片側測定\\
\Dimple 各列の中心上への移動\\
\KeywayMilling\\
\EndFaceOutChamferMilling\\
\EndFaceInChamferMilling\\
\indexReliefGrooveMeasurement\indexReliefGrooveMilling\nameReliefGroove 測定および加工\\
\end{multicollongtblr}

\begin{multicollongtblr}{\index{じょうけんぶんき@条件分岐}条件分岐:\PMBottomALBracketDimensionExists}{X[l]}
直接的に影響するパラメタ\\
\AlocationLength\\
\end{multicollongtblr}



\clearpage
%%%%%%%%%%%%%%%%%%%%%%%%%%%%%%%%%%%%%%%%%%%%%%%%%%%%%%%%%%
%% section 46.02 %%%%%%%%%%%%%%%%%%%%%%%%%%%%%%%%%%%%%%%%%
%%%%%%%%%%%%%%%%%%%%%%%%%%%%%%%%%%%%%%%%%%%%%%%%%%%%%%%%%%
\modHeadsection{\Outcut に関する\index{じょうけんぶんき@条件分岐}条件分岐}

\begin{multicollongtblr}{\index{じょうけんぶんき@条件分岐}条件分岐:\PMTopOutcutExists}{X[l]}
直接的に影響する工程\\
ワーク座標系 原点設定\\
トップ外側中心$X$ 片側測定\\
\KeywayCenter$X$ 芯出し\\
\TopOutcutMilling\\
\KeywayMilling\\
\TopEndFaceOutChamferMilling\\
\CenterlineEndFaceDifMeasurement\\
\end{multicollongtblr}

\begin{multicollongtblr}{\index{じょうけんぶんき@条件分岐}条件分岐:\PMTopOutcutEndKeywayExists}{X[l]}
直接的に影響する工程\\
\TopOutcutMilling\\
\TopCurvedOutcutMilling\\
\end{multicollongtblr}

\begin{multicollongtblr}{\index{じょうけんぶんき@条件分岐}条件分岐:\PMBottomOutcutExists}{X[l]}
直接的に影響する工程\\
ワーク座標系 原点設定\\
ボトム外側中心$X$ 片側測定\\
\BottomOutcutMilling\\
\BottomEndFaceOutChamferMilling\\
\CenterlineEndFaceDifMeasurement\\
\end{multicollongtblr}

\begin{multicollongtblr}{\index{じょうけんぶんき@条件分岐}条件分岐:\PMSquareEndMillTaperExists}{X[l]}
直接的に影響する工程\\
\OutcutMilling\\
\CurvedOutcutMilling\\
\end{multicollongtblr}

\begin{multicollongtblr}{\index{じょうけんぶんき@条件分岐}条件分岐:\PMOutcutCenterReference}{X[l]}
直接的に影響する工程\\
\CenterlineEndFaceDifMeasurement\\
\end{multicollongtblr}



\clearpage
%%%%%%%%%%%%%%%%%%%%%%%%%%%%%%%%%%%%%%%%%%%%%%%%%%%%%%%%%%
%% section 46.03 %%%%%%%%%%%%%%%%%%%%%%%%%%%%%%%%%%%%%%%%%
%%%%%%%%%%%%%%%%%%%%%%%%%%%%%%%%%%%%%%%%%%%%%%%%%%%%%%%%%%
\modHeadsection{\Keyway に関する\index{じょうけんぶんき@条件分岐}条件分岐}

\begin{multicollongtblr}{\index{じょうけんぶんき@条件分岐}条件分岐:\PMKeywayCornerType}{X[l]}
直接的に影響する工程\\
\KeywayMilling\\
\end{multicollongtblr}

\begin{multicollongtblr}{\index{じょうけんぶんき@条件分岐}条件分岐:\PMAKDToleranceExists}{X[l]}
直接的に影響する工程\\
ワーク座標系 原点設定\\
\KeywayCenter$X$ 片側測定\\
\KeywayMilling\\
\end{multicollongtblr}



%\clearpage
%%%%%%%%%%%%%%%%%%%%%%%%%%%%%%%%%%%%%%%%%%%%%%%%%%%%%%%%%%
%% section 46.04 %%%%%%%%%%%%%%%%%%%%%%%%%%%%%%%%%%%%%%%%%
%%%%%%%%%%%%%%%%%%%%%%%%%%%%%%%%%%%%%%%%%%%%%%%%%%%%%%%%%%
\modHeadsection{\Dimple に関する\index{じょうけんぶんき@条件分岐}条件分岐}

\begin{multicollongtblr}{\index{じょうけんぶんき@条件分岐}条件分岐:\PMDimpleExists}{X[l]}
直接的に影響する工程\\
\DimpleMilling\\
\end{multicollongtblr}



%\clearpage
%%%%%%%%%%%%%%%%%%%%%%%%%%%%%%%%%%%%%%%%%%%%%%%%%%%%%%%%%%
%% section 46.05 %%%%%%%%%%%%%%%%%%%%%%%%%%%%%%%%%%%%%%%%%
%%%%%%%%%%%%%%%%%%%%%%%%%%%%%%%%%%%%%%%%%%%%%%%%%%%%%%%%%%
\modHeadsection{\EndFaceChamfer に関する\index{じょうけんぶんき@条件分岐}条件分岐}

\begin{multicollongtblr}{\index{じょうけんぶんき@条件分岐}条件分岐:\PMEndFaceOutChamferType}{X[l]}
直接的に影響する工程\\
\EndFaceOutChamferMilling\\
\end{multicollongtblr}

\begin{multicollongtblr}{\index{じょうけんぶんき@条件分岐}条件分岐:\PMEndFaceInChamferType}{X[l]}
直接的に影響する工程\\
\EndFaceInChamferMilling\\
\end{multicollongtblr}



\clearpage
%%%%%%%%%%%%%%%%%%%%%%%%%%%%%%%%%%%%%%%%%%%%%%%%%%%%%%%%%%
%% section 46.06 %%%%%%%%%%%%%%%%%%%%%%%%%%%%%%%%%%%%%%%%%
%%%%%%%%%%%%%%%%%%%%%%%%%%%%%%%%%%%%%%%%%%%%%%%%%%%%%%%%%%
\modHeadsection{\index{とくしゅなかこう@特殊な加工}特殊な加工に関する\index{じょうけんぶんき@条件分岐}条件分岐}

\begin{multicollongtblr}{\index{じょうけんぶんき@条件分岐}条件分岐:\PMEndFaceBoringExists}{X[l]}
直接的に影響する工程\\
\EndFaceBoringMilling\\
\end{multicollongtblr}

\begin{multicollongtblr}{\index{じょうけんぶんき@条件分岐}条件分岐:\PMIncutBoringExists}{X[l]}
直接的に影響する工程\\
\IncutBoringMilling\\
\TopEndFaceInChamferMilling\\
\end{multicollongtblr}

\begin{multicollongtblr}{\index{じょうけんぶんき@条件分岐}条件分岐:\PMCurvedOutcutExists}{X[l]}
直接的に影響する工程\\
\CurvedOutcutMilling\\
\EndFaceOutChamferMilling\\
\end{multicollongtblr}

%!TEX root = ./RPA_for_Creating_Program_Note.tex


\modHeadchapter{サブプログラムの取扱説明\TBW}



%%%%%%%%%%%%%%%%%%%%%%%%%%%%%%%%%%%%%%%%%%%%%%%%%%%%%%%%%%
%% section 31.1 %%%%%%%%%%%%%%%%%%%%%%%%%%%%%%%%%%%%%%%%%%
%%%%%%%%%%%%%%%%%%%%%%%%%%%%%%%%%%%%%%%%%%%%%%%%%%%%%%%%%%
\modHeadsection{\MXOThickness :測定 外側両側\texorpdfstring{$X$}{X}\TBW}
(to be written ...)


%%%%%%%%%%%%%%%%%%%%%%%%%%%%%%%%%%%%%%%%%%%%%%%%%%%%%%%%%%
%% subsection 32.1.1 %%%%%%%%%%%%%%%%%%%%%%%%%%%%%%%%%%%%%
%%%%%%%%%%%%%%%%%%%%%%%%%%%%%%%%%%%%%%%%%%%%%%%%%%%%%%%%%%
\subsection{引数\TBW}
(to be written ...)



\clearpage
%%%%%%%%%%%%%%%%%%%%%%%%%%%%%%%%%%%%%%%%%%%%%%%%%%%%%%%%%%
%% section 32.2 %%%%%%%%%%%%%%%%%%%%%%%%%%%%%%%%%%%%%%%%%%
%%%%%%%%%%%%%%%%%%%%%%%%%%%%%%%%%%%%%%%%%%%%%%%%%%%%%%%%%%
\modHeadsection{\MYOThickness :測定 外側両側\texorpdfstring{$Y$}{Y}\TBW}
(to be written ...)


%%%%%%%%%%%%%%%%%%%%%%%%%%%%%%%%%%%%%%%%%%%%%%%%%%%%%%%%%%
%% subsection 32.2.1 %%%%%%%%%%%%%%%%%%%%%%%%%%%%%%%%%%%%%
%%%%%%%%%%%%%%%%%%%%%%%%%%%%%%%%%%%%%%%%%%%%%%%%%%%%%%%%%%
\subsection{引数\TBW}
(to be written ...)



\clearpage
%%%%%%%%%%%%%%%%%%%%%%%%%%%%%%%%%%%%%%%%%%%%%%%%%%%%%%%%%%
%% section 32.2 %%%%%%%%%%%%%%%%%%%%%%%%%%%%%%%%%%%%%%%%%%
%%%%%%%%%%%%%%%%%%%%%%%%%%%%%%%%%%%%%%%%%%%%%%%%%%%%%%%%%%
\modHeadsection{\MXOface :測定 溝基準面\texorpdfstring{$X$}{X}\TBW}
(to be written ...)


%%%%%%%%%%%%%%%%%%%%%%%%%%%%%%%%%%%%%%%%%%%%%%%%%%%%%%%%%%
%% subsection 32.2.1 %%%%%%%%%%%%%%%%%%%%%%%%%%%%%%%%%%%%%
%%%%%%%%%%%%%%%%%%%%%%%%%%%%%%%%%%%%%%%%%%%%%%%%%%%%%%%%%%
\subsection{引数\TBW}
(to be written ...)



\clearpage
%%%%%%%%%%%%%%%%%%%%%%%%%%%%%%%%%%%%%%%%%%%%%%%%%%%%%%%%%%
%% section 32.2 %%%%%%%%%%%%%%%%%%%%%%%%%%%%%%%%%%%%%%%%%%
%%%%%%%%%%%%%%%%%%%%%%%%%%%%%%%%%%%%%%%%%%%%%%%%%%%%%%%%%%
\modHeadsection{\MXIWidth :測定 内側両側\texorpdfstring{$X$}{X}\TBW}
(to be written ...)


%%%%%%%%%%%%%%%%%%%%%%%%%%%%%%%%%%%%%%%%%%%%%%%%%%%%%%%%%%
%% subsection 32.2.1 %%%%%%%%%%%%%%%%%%%%%%%%%%%%%%%%%%%%%
%%%%%%%%%%%%%%%%%%%%%%%%%%%%%%%%%%%%%%%%%%%%%%%%%%%%%%%%%%
\subsection{引数\TBW}
(to be written ...)



\clearpage
%%%%%%%%%%%%%%%%%%%%%%%%%%%%%%%%%%%%%%%%%%%%%%%%%%%%%%%%%%
%% section 32.2 %%%%%%%%%%%%%%%%%%%%%%%%%%%%%%%%%%%%%%%%%%
%%%%%%%%%%%%%%%%%%%%%%%%%%%%%%%%%%%%%%%%%%%%%%%%%%%%%%%%%%
\modHeadsection{\MYIWidth :測定 内側両側\texorpdfstring{$Y$}{Y}\TBW}
(to be written ...)


%%%%%%%%%%%%%%%%%%%%%%%%%%%%%%%%%%%%%%%%%%%%%%%%%%%%%%%%%%
%% subsection 32.2.1 %%%%%%%%%%%%%%%%%%%%%%%%%%%%%%%%%%%%%
%%%%%%%%%%%%%%%%%%%%%%%%%%%%%%%%%%%%%%%%%%%%%%%%%%%%%%%%%%
\subsection{引数\TBW}
(to be written ...)



\clearpage
%%%%%%%%%%%%%%%%%%%%%%%%%%%%%%%%%%%%%%%%%%%%%%%%%%%%%%%%%%
%% section 32.2 %%%%%%%%%%%%%%%%%%%%%%%%%%%%%%%%%%%%%%%%%%
%%%%%%%%%%%%%%%%%%%%%%%%%%%%%%%%%%%%%%%%%%%%%%%%%%%%%%%%%%
\modHeadsection{\MXIface :測定 外削基準面\texorpdfstring{$X$}{X}\TBW}
(to be written ...)


%%%%%%%%%%%%%%%%%%%%%%%%%%%%%%%%%%%%%%%%%%%%%%%%%%%%%%%%%%
%% subsection 32.2.1 %%%%%%%%%%%%%%%%%%%%%%%%%%%%%%%%%%%%%
%%%%%%%%%%%%%%%%%%%%%%%%%%%%%%%%%%%%%%%%%%%%%%%%%%%%%%%%%%
\subsection{引数\TBW}
(to be written ...)



\clearpage
%%%%%%%%%%%%%%%%%%%%%%%%%%%%%%%%%%%%%%%%%%%%%%%%%%%%%%%%%%
%% section 32.2 %%%%%%%%%%%%%%%%%%%%%%%%%%%%%%%%%%%%%%%%%%
%%%%%%%%%%%%%%%%%%%%%%%%%%%%%%%%%%%%%%%%%%%%%%%%%%%%%%%%%%
\modHeadsection{\MYcenterline :測定 通り芯\texorpdfstring{$Y$}{Y}\TBW}
(to be written ...)


%%%%%%%%%%%%%%%%%%%%%%%%%%%%%%%%%%%%%%%%%%%%%%%%%%%%%%%%%%
%% subsection 32.2.1 %%%%%%%%%%%%%%%%%%%%%%%%%%%%%%%%%%%%%
%%%%%%%%%%%%%%%%%%%%%%%%%%%%%%%%%%%%%%%%%%%%%%%%%%%%%%%%%%
\subsection{引数\TBW}
(to be written ...)



\clearpage
%%%%%%%%%%%%%%%%%%%%%%%%%%%%%%%%%%%%%%%%%%%%%%%%%%%%%%%%%%
%% section 32.2 %%%%%%%%%%%%%%%%%%%%%%%%%%%%%%%%%%%%%%%%%%
%%%%%%%%%%%%%%%%%%%%%%%%%%%%%%%%%%%%%%%%%%%%%%%%%%%%%%%%%%
\modHeadsection{\MXcenterline :測定 通り芯\texorpdfstring{$X$}{X}\TBW}
(to be written ...)


%%%%%%%%%%%%%%%%%%%%%%%%%%%%%%%%%%%%%%%%%%%%%%%%%%%%%%%%%%
%% subsection 32.2.1 %%%%%%%%%%%%%%%%%%%%%%%%%%%%%%%%%%%%%
%%%%%%%%%%%%%%%%%%%%%%%%%%%%%%%%%%%%%%%%%%%%%%%%%%%%%%%%%%
\subsection{引数\TBW}
(to be written ...)



\clearpage
%%%%%%%%%%%%%%%%%%%%%%%%%%%%%%%%%%%%%%%%%%%%%%%%%%%%%%%%%%
%% section 32.2 %%%%%%%%%%%%%%%%%%%%%%%%%%%%%%%%%%%%%%%%%%
%%%%%%%%%%%%%%%%%%%%%%%%%%%%%%%%%%%%%%%%%%%%%%%%%%%%%%%%%%
\modHeadsection{\KTanmenRight :加工 端面 コーナーR 右回り1周\TBW}
(to be written ...)


%%%%%%%%%%%%%%%%%%%%%%%%%%%%%%%%%%%%%%%%%%%%%%%%%%%%%%%%%%
%% subsection 32.2.1 %%%%%%%%%%%%%%%%%%%%%%%%%%%%%%%%%%%%%
%%%%%%%%%%%%%%%%%%%%%%%%%%%%%%%%%%%%%%%%%%%%%%%%%%%%%%%%%%
\subsection{引数\TBW}
(to be written ...)



\clearpage
%%%%%%%%%%%%%%%%%%%%%%%%%%%%%%%%%%%%%%%%%%%%%%%%%%%%%%%%%%
%% section 32.2 %%%%%%%%%%%%%%%%%%%%%%%%%%%%%%%%%%%%%%%%%%
%%%%%%%%%%%%%%%%%%%%%%%%%%%%%%%%%%%%%%%%%%%%%%%%%%%%%%%%%%
\modHeadsection{\KGaisakuRLeft :加工 外削 コーナーR 左回り1周\TBW}
(to be written ...)


%%%%%%%%%%%%%%%%%%%%%%%%%%%%%%%%%%%%%%%%%%%%%%%%%%%%%%%%%%
%% subsection 32.2.1 %%%%%%%%%%%%%%%%%%%%%%%%%%%%%%%%%%%%%
%%%%%%%%%%%%%%%%%%%%%%%%%%%%%%%%%%%%%%%%%%%%%%%%%%%%%%%%%%
\subsection{引数\TBW}
(to be written ...)



\clearpage
%%%%%%%%%%%%%%%%%%%%%%%%%%%%%%%%%%%%%%%%%%%%%%%%%%%%%%%%%%
%% section 32.2 %%%%%%%%%%%%%%%%%%%%%%%%%%%%%%%%%%%%%%%%%%
%%%%%%%%%%%%%%%%%%%%%%%%%%%%%%%%%%%%%%%%%%%%%%%%%%%%%%%%%%
\modHeadsection{\KMizoConerLeft :加工 溝 左回り1周\TBW}
(to be written ...)


%%%%%%%%%%%%%%%%%%%%%%%%%%%%%%%%%%%%%%%%%%%%%%%%%%%%%%%%%%
%% subsection 32.2.1 %%%%%%%%%%%%%%%%%%%%%%%%%%%%%%%%%%%%%
%%%%%%%%%%%%%%%%%%%%%%%%%%%%%%%%%%%%%%%%%%%%%%%%%%%%%%%%%%
\subsection{引数\TBW}
(to be written ...)



\clearpage
%%%%%%%%%%%%%%%%%%%%%%%%%%%%%%%%%%%%%%%%%%%%%%%%%%%%%%%%%%
%% section 32.2 %%%%%%%%%%%%%%%%%%%%%%%%%%%%%%%%%%%%%%%%%%
%%%%%%%%%%%%%%%%%%%%%%%%%%%%%%%%%%%%%%%%%%%%%%%%%%%%%%%%%%
\modHeadsection{\KSotoMentoriRLeft :加工 外面取 コーナーR 左回り1周\TBW}
(to be written ...)


%%%%%%%%%%%%%%%%%%%%%%%%%%%%%%%%%%%%%%%%%%%%%%%%%%%%%%%%%%
%% subsection 32.2.1 %%%%%%%%%%%%%%%%%%%%%%%%%%%%%%%%%%%%%
%%%%%%%%%%%%%%%%%%%%%%%%%%%%%%%%%%%%%%%%%%%%%%%%%%%%%%%%%%
\subsection{引数\TBW}
(to be written ...)



\clearpage
%%%%%%%%%%%%%%%%%%%%%%%%%%%%%%%%%%%%%%%%%%%%%%%%%%%%%%%%%%
%% section 32.2 %%%%%%%%%%%%%%%%%%%%%%%%%%%%%%%%%%%%%%%%%%
%%%%%%%%%%%%%%%%%%%%%%%%%%%%%%%%%%%%%%%%%%%%%%%%%%%%%%%%%%
\modHeadsection{\KUchiMentoriRLeft :加工 内面取 コーナーR 左回り1周\TBW}
(to be written ...)


%%%%%%%%%%%%%%%%%%%%%%%%%%%%%%%%%%%%%%%%%%%%%%%%%%%%%%%%%%
%% subsection 32.2.1 %%%%%%%%%%%%%%%%%%%%%%%%%%%%%%%%%%%%%
%%%%%%%%%%%%%%%%%%%%%%%%%%%%%%%%%%%%%%%%%%%%%%%%%%%%%%%%%%
\subsection{引数\TBW}
(to be written ...)



\clearpage
%%%%%%%%%%%%%%%%%%%%%%%%%%%%%%%%%%%%%%%%%%%%%%%%%%%%%%%%%%
%% section 32.2 %%%%%%%%%%%%%%%%%%%%%%%%%%%%%%%%%%%%%%%%%%
%%%%%%%%%%%%%%%%%%%%%%%%%%%%%%%%%%%%%%%%%%%%%%%%%%%%%%%%%%
\modHeadsection{\DLone :測定・加工 \dimple\TBW}
(to be written ...)


%%%%%%%%%%%%%%%%%%%%%%%%%%%%%%%%%%%%%%%%%%%%%%%%%%%%%%%%%%
%% subsection 32.2.1 %%%%%%%%%%%%%%%%%%%%%%%%%%%%%%%%%%%%%
%%%%%%%%%%%%%%%%%%%%%%%%%%%%%%%%%%%%%%%%%%%%%%%%%%%%%%%%%%
\subsection{引数\TBW}
(to be written ...)



\clearpage
%%%%%%%%%%%%%%%%%%%%%%%%%%%%%%%%%%%%%%%%%%%%%%%%%%%%%%%%%%
%% section 32.2 %%%%%%%%%%%%%%%%%%%%%%%%%%%%%%%%%%%%%%%%%%
%%%%%%%%%%%%%%%%%%%%%%%%%%%%%%%%%%%%%%%%%%%%%%%%%%%%%%%%%%
\modHeadsection{\OwarmingupA :暖機運転\TBW}
(to be written ...)


%%%%%%%%%%%%%%%%%%%%%%%%%%%%%%%%%%%%%%%%%%%%%%%%%%%%%%%%%%
%% subsection 32.2.1 %%%%%%%%%%%%%%%%%%%%%%%%%%%%%%%%%%%%%
%%%%%%%%%%%%%%%%%%%%%%%%%%%%%%%%%%%%%%%%%%%%%%%%%%%%%%%%%%
\subsection{引数\TBW}
(to be written ...)



\clearpage
%%%%%%%%%%%%%%%%%%%%%%%%%%%%%%%%%%%%%%%%%%%%%%%%%%%%%%%%%%
%% section 32.2 %%%%%%%%%%%%%%%%%%%%%%%%%%%%%%%%%%%%%%%%%%
%%%%%%%%%%%%%%%%%%%%%%%%%%%%%%%%%%%%%%%%%%%%%%%%%%%%%%%%%%
\modHeadsection{\OtoolLengthA :工具長測定\TBW}
(to be written ...)


%%%%%%%%%%%%%%%%%%%%%%%%%%%%%%%%%%%%%%%%%%%%%%%%%%%%%%%%%%
%% subsection 32.2.1 %%%%%%%%%%%%%%%%%%%%%%%%%%%%%%%%%%%%%
%%%%%%%%%%%%%%%%%%%%%%%%%%%%%%%%%%%%%%%%%%%%%%%%%%%%%%%%%%
\subsection{引数\TBW}
(to be written ...)

%!TEX root = ./RfCPN.tex


\modHeadchapter{各工程用\index{NCサブプログラム}NCサブプログラムに必要な数値情報}
ここでは主に各工程用の\index{NCサブプログラム}NCサブプログラムの記述に際して、\index{すうちけいさん@数値計算}数値計算の必要な部分をピックアップする。



%%%%%%%%%%%%%%%%%%%%%%%%%%%%%%%%%%%%%%%%%%%%%%%%%%%%%%%%%%
%% section 41.01 %%%%%%%%%%%%%%%%%%%%%%%%%%%%%%%%%%%%%%%%%
%%%%%%%%%%%%%%%%%%%%%%%%%%%%%%%%%%%%%%%%%%%%%%%%%%%%%%%%%%
\modHeadsection{\TBW}
(to be written...)

%%%%%%%%%%%%%%%%%%%%%%%%%%%%%%%%%%%%%%%%%%%%%%%%%%%%%%%%%
%% Appendiodes %%%%%%%%%%%%%%%%%%%%%%%%%%%%%%%%%%%%%%%%%%
%%%%%%%%%%%%%%%%%%%%%%%%%%%%%%%%%%%%%%%%%%%%%%%%%%%%%%%%%
\begin{appendices}
%\Appendixpart
\end{appendices}

\addtocontents{toc}{\protect\end{tocBox}}
