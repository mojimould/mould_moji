%!TEX root = ./RfCPN.tex


\addtocontents{toc}{\protect\cleardoublepage}
%%%%%%%%%%%%%%%%%%%%%%%%%%%%%%%%%%%%%%%%%%%%%%%%%%%%%%%%%
%% Part Numerical Calculation %%%%%%%%%%%%%%%%%%%%%%%%%%%
%%%%%%%%%%%%%%%%%%%%%%%%%%%%%%%%%%%%%%%%%%%%%%%%%%%%%%%%%
\addtocontents{toc}{\protect\begin{tocBox}{\tmppartnum}}%
\tPart{解析計算に基づく数値解析\label{part:NC}}{%
\paragraph*{\tpartgoal}
\index{めいさい(モールド)@明細(モールド)}明細ごとに異なる\index{すんぽう@寸法}寸法・形状を持つすべての\index{ワーク}ワークに対し、\index{NCプログラム}NCプログラムの作成に必要な\textbf{数値情報および\index{じょうけんぶんきじょうほう@条件分岐情報}条件分岐情報等が自動的に得られるシステムを構築}する。
\tcbline*
\paragraph*{\tpartmethod}
前段階で導出した\textbf{解析的な情報}を用いて、各明細における具体的な\textbf{数値的な情報}に自動的に変換するシステムの構築を試みる。
\tcbline*
\paragraph*{\tpartbackground}
一般に、個々の明細や\index{こうぐ@工具}工具等の寸法は異なり、明細ごとに固有の寸法・形状を持つ。
NCプログラムの作成の際には、それらをすべて考慮した\textbf{具体的な数値情報}を指定する必要がある。

 こうした数値情報は\textbf{明細ごとに膨大にある}が、現時点(\DMC 設置時点)において、こうした手続きは\textbf{明細ごとに手作業}で行われている
%% footnote %%%%%%%%%%%%%%%%%%%%%
\footnote{さらには、「\index{ずめん(モールド)@図面(モールド)}図面の作成後、それを見て\index{NCプログラム}NCプログラムを作成する」という明らかに異常な(奇妙な)事態が放置され続けている。}。
%%%%%%%%%%%%%%%%%%%%%%%%%%%%%%%%%

 したがって、こうした手続きのシステム化を行い、可能な限り\textbf{自動化}することが喫緊の課題である。
そうすることで、\textbf{危険を伴う作業の削減(\index{あんぜんせい@安全性}安全性の向上)}や、\textbf{品質の低下の防止}に大きく寄与できることが期待される。
また副次的効果として、作業効率・人的資源・安全(security)・\index{ほしゅ@保守}保守などのいずれの観点からみた\textbf{能率の低下の防止}にも大きく貢献することも自ずと期待される。
}{%
\paragraph*{\tpartconclusion}
各明細のワークにおける固有数値情報の入力により、\index{NCプログラム}NCプログラムの作成に必要な数値情報および\index{じょうけんぶんきじょうほう@条件分岐情報}条件分岐情報等が(人による手動計算を介することなく)自動的に得られるシステムを構築した。
\tcbline*
\paragraph*{\tpartnextstep}
得られた固有数値情報および条件分岐情報を用いて、具体的な\index{NCプログラム}NCプログラムを記述する。
}

%%%%%%%%%%%%%%%%%%%%%%%%%%%%%%%%%%%%%%%%%%%%%%%%%%%%%%%%%%
%% chapters %%%%%%%%%%%%%%%%%%%%%%%%%%%%%%%%%%%%%%%%%%%%%%
%%%%%%%%%%%%%%%%%%%%%%%%%%%%%%%%%%%%%%%%%%%%%%%%%%%%%%%%%%
%!TEX root = ./RfCPN.tex


\modHeadchapter[lot]{入力する数値情報・パラメータ}
\index{すんぽう@寸法}寸法・形状等の数値情報や\index{じょうけんぶんきじょうほう@条件分岐情報}条件分岐情報は、\index{めいさい(モールド)@明細(モールド)}明細ごとに固有である。
そのため、その\index{こゆうじょうほう(ワーク)@固有情報(ワーク)}固有情報は入力する必要がある。
ここではそうした入力する必要のある情報をまとめておく。
なお、\index{にゅうりょくするすうちじょうほう@入力する数値情報}入力する数値情報に関しては、原則として\index{ずめん@図面}図面上の\index{すんぽう@寸法}寸法をそのまま入力する形となるような方針とする
%% footnote %%%%%%%%%%%%%%%%%%%%%
\footnote{ただし、これらの値にはそれぞれの\index{こうさ@公差}公差が考慮されている。}。
%%%%%%%%%%%%%%%%%%%%%%%%%%%%%%%%%



%%%%%%%%%%%%%%%%%%%%%%%%%%%%%%%%%%%%%%%%%%%%%%%%%%%%%%%%%%
%% section 30.1 %%%%%%%%%%%%%%%%%%%%%%%%%%%%%%%%%%%%%%%%%%
%%%%%%%%%%%%%%%%%%%%%%%%%%%%%%%%%%%%%%%%%%%%%%%%%%%%%%%%%%
\modHeadsection{湾曲・\Alocation に関する入力数値}

\begin{multicollongtblr}{入力情報:湾曲・\Alocation}{X[l]c}
内容 & 型\\
\CenterCurvatureExists & boolean\\
\CenterCurvatureRadius & float\\
\TopAlocationLength & float\\
\BottomAlocationLength & float\\
\end{multicollongtblr}



%%%%%%%%%%%%%%%%%%%%%%%%%%%%%%%%%%%%%%%%%%%%%%%%%%%%%%%%%%
%% section 30.2 %%%%%%%%%%%%%%%%%%%%%%%%%%%%%%%%%%%%%%%%%%
%%%%%%%%%%%%%%%%%%%%%%%%%%%%%%%%%%%%%%%%%%%%%%%%%%%%%%%%%%
\modHeadsection{外形・内形に関する入力数値}

\begin{multicollongtblr}{入力情報:外形}{X[l]c}
内容 & 型\\
\ACOD & float\\
\BDOD & float\\
\ODCornerR & float\\
\end{multicollongtblr}

\begin{multicollongtblr}{入力情報:内形}{X[l]c}
内容 & 型\\
\PlatingThk & float\\
\IDTaperTableNum & list\\
\end{multicollongtblr}
%%%%%%%%%%%%%%%%%%%%%%%%%%%%%%%%%%%%%%%%%%%%%%%%%%%%%%%%%%
%% marker %%%%%%%%%%%%%%%%%%%%%%%%%%%%%%%%%%%%%%%%%%%%%%%%
%%%%%%%%%%%%%%%%%%%%%%%%%%%%%%%%%%%%%%%%%%%%%%%%%%%%%%%%%%
\begin{marker}
\IDTaperTable には、\InnerCornerR の\index{すんぽう(\yomiInnerCornerR)@寸法(\nameInnerCornerR)}寸法情報も記載されているものとする。
\end{marker}
%%%%%%%%%%%%%%%%%%%%%%%%%%%%%%%%%%%%%%%%%%%%%%%%%%%%%%%%%%
%%%%%%%%%%%%%%%%%%%%%%%%%%%%%%%%%%%%%%%%%%%%%%%%%%%%%%%%%%
%%%%%%%%%%%%%%%%%%%%%%%%%%%%%%%%%%%%%%%%%%%%%%%%%%%%%%%%%%



\clearpage
%%%%%%%%%%%%%%%%%%%%%%%%%%%%%%%%%%%%%%%%%%%%%%%%%%%%%%%%%%
%% section 30.3 %%%%%%%%%%%%%%%%%%%%%%%%%%%%%%%%%%%%%%%%%%
%%%%%%%%%%%%%%%%%%%%%%%%%%%%%%%%%%%%%%%%%%%%%%%%%%%%%%%%%%
\modHeadsection{\Outcut に関する入力数値}

\begin{multicollongtblr}{入力情報:\BottomOutcut}{X[l]c}
内容 & 型\\
\BottomOutcutExists & boolean\\
\index{こうぐせんたんテーパのうむ@工具先端テーパの有無}工具先端テーパの有無 & boolean\\
\BottomCurvedOutcutExists & boolean\\
\BottomOutcutAsideThickness & float\\
\BottomOutcutACwidth & float\\
\BottomOutcutBDwidth & float\\
\BottomOutcutLength & float\\
\BottomOutcutConerR & float\\
\end{multicollongtblr}

\begin{multicollongtblr}{入力情報:\TopOutcut}{X[l]c}
内容 & 型\\
\TopOutcutExists & boolean\\
\index{こうぐせんたんテーパのうむ@工具先端テーパの有無}工具先端テーパの有無 & boolean\\
\TopCurvedOutcutExists & boolean\\
\TopOutcutAsideThickness & float\\
\TopOutcutACwidth & float\\
\TopOutcutBDwidth & float\\
\TopOutcutLength & float\\
\TopOutcutCornerR & float\\
\end{multicollongtblr}

\begin{multicollongtblr}{入力情報:両外削}{X[l]c}
内容 & 型\\
\OutcutCenter 基準 & enum\\
\CenterlineEndFaceDif & float\\
\end{multicollongtblr}



\clearpage
%%%%%%%%%%%%%%%%%%%%%%%%%%%%%%%%%%%%%%%%%%%%%%%%%%%%%%%%%%
%% section 30.4 %%%%%%%%%%%%%%%%%%%%%%%%%%%%%%%%%%%%%%%%%%
%%%%%%%%%%%%%%%%%%%%%%%%%%%%%%%%%%%%%%%%%%%%%%%%%%%%%%%%%%
\modHeadsection{\Keyway に関する入力数値}

\begin{multicollongtblr}{入力情報:\Keyway}{X[l]c}
内容 & 型\\
\KeywayType & enum\\
\AsideKeywayDepth 指定 有無 & boolean\\
\KeywayACOD & float\\
\KeywayBDOD & float\\
\KeywayPos & float\\
\KeywayWidth & float\\
\AsideKeywayDepth & float\\
\KeywayCornerR & float\\
\KeywayCornerC & float\\
\end{multicollongtblr}



%\clearpage
%%%%%%%%%%%%%%%%%%%%%%%%%%%%%%%%%%%%%%%%%%%%%%%%%%%%%%%%%%
%% section 30.5 %%%%%%%%%%%%%%%%%%%%%%%%%%%%%%%%%%%%%%%%%%
%%%%%%%%%%%%%%%%%%%%%%%%%%%%%%%%%%%%%%%%%%%%%%%%%%%%%%%%%%
\modHeadsection{\Dimple に関する入力数値}

\begin{multicollongtblr}{入力情報:\Dimple}{X[l]c}
内容 & 型\\
\DimpleExists & boolean\\
トップ端と\DimpleFirstRow までの距離 & float\\
\DimpleVerticalPitch & float\\
\DimpleHorizontalPitch & float\\
\DimpleOddRowLength & float\\
\DimpleEvenRowLength & float\\
\DimpleRowNum & enum\\
\DimpleDepth & float\\
\DimpleRadius(工具小半径) & float\\
\end{multicollongtblr}



\clearpage
%%%%%%%%%%%%%%%%%%%%%%%%%%%%%%%%%%%%%%%%%%%%%%%%%%%%%%%%%%
%% section 30.6 %%%%%%%%%%%%%%%%%%%%%%%%%%%%%%%%%%%%%%%%%%
%%%%%%%%%%%%%%%%%%%%%%%%%%%%%%%%%%%%%%%%%%%%%%%%%%%%%%%%%%
\modHeadsection{\EndFaceChamfer に関する入力数値}


%%%%%%%%%%%%%%%%%%%%%%%%%%%%%%%%%%%%%%%%%%%%%%%%%%%%%%%%%%
%% subsection 30.6.1 %%%%%%%%%%%%%%%%%%%%%%%%%%%%%%%%%%%%%
%%%%%%%%%%%%%%%%%%%%%%%%%%%%%%%%%%%%%%%%%%%%%%%%%%%%%%%%%%
\subsection{\BottomEndFaceChamfer に関する入力情報}

\begin{multicollongtblr}{入力情報:\BottomEndFaceOutCChamfer}{X[l]c}
内容 & 型\\
\BottomEndFaceOutCChamferExists & boolean\\
\BottomEndFaceOutCChamferLength & float\\
\BottomEndFaceOutCChamferAngle & float\\
\end{multicollongtblr}

\begin{multicollongtblr}{入力情報:\BottomEndFaceOutRChamfer}{X[l]c}
内容 & 型\\
\BottomEndFaceOutRChamferExists & boolean\\
\BottomEndFaceOutRChamferRadius & float\\
\end{multicollongtblr}

\begin{multicollongtblr}{入力情報:\BottomEndFaceInCChamfer}{X[l]c}
内容 & 型\\
\BottomEndFaceInCChamferExists & boolean\\
\BottomEndFaceInCChamferLength & float\\
\BottomEndFaceInCChamferAngle & integer\\
\end{multicollongtblr}

\begin{multicollongtblr}{入力情報:\BottomEndFaceInRChamfer}{X[l]c}
内容 & 型\\
\BottomFaceInRChamferExsits & boolean\\
\BottomFaceInRChamferRadius & float\\
\end{multicollongtblr}



\clearpage
%%%%%%%%%%%%%%%%%%%%%%%%%%%%%%%%%%%%%%%%%%%%%%%%%%%%%%%%%%
%% subsection 30.6.2 %%%%%%%%%%%%%%%%%%%%%%%%%%%%%%%%%%%%%
%%%%%%%%%%%%%%%%%%%%%%%%%%%%%%%%%%%%%%%%%%%%%%%%%%%%%%%%%%
\subsection{\TopEndFaceChamfer に関する入力情報}

\begin{multicollongtblr}{入力情報:\TopEndFaceOutCChamfer}{X[l]c}
内容 & 型\\
\TopEndFaceOutCChamferExists & boolean\\
\TopEndFaceOutCChamferLength & float\\
\TopEndFaceOutCChamferAngle & integer\\
\end{multicollongtblr}

\begin{multicollongtblr}{入力情報:\TopEndFaceOutRChamfer}{X[l]c}
内容 & 型\\
\TopEndFaceOutRChamferExists & boolean\\
\TopEndFaceOutRChamferRadius & float\\
\end{multicollongtblr}

\begin{multicollongtblr}{入力情報:\TopEndFaceInCChamfer}{X[l]c}
内容 & 型\\
\TopEndFaceInCChamferExists & boolean\\
\TopEndFaceInCChamferLength & float\\
\TopEndFaceInCChamferAngle & integer\\
\end{multicollongtblr}

\begin{multicollongtblr}{入力情報:\TopEndFaceInRChamfer}{X[l]c}
内容 & 型\\
\TopEndFaceInRChamferExists & boolean\\
\TopEndFaceInRChamferRadius & float\\
\end{multicollongtblr}



%\clearpage
%%%%%%%%%%%%%%%%%%%%%%%%%%%%%%%%%%%%%%%%%%%%%%%%%%%%%%%%%%
%% section 30.6 %%%%%%%%%%%%%%%%%%%%%%%%%%%%%%%%%%%%%%%%%%
%%%%%%%%%%%%%%%%%%%%%%%%%%%%%%%%%%%%%%%%%%%%%%%%%%%%%%%%%%
\modHeadsection{\EndFaceBoring に関する入力情報}

\begin{multicollongtblr}{入力情報:\EndFaceBoring}{X[l]c}
内容 & 型\\
\EndFaceBoringExists & boolean\\
\EndFaceBoringWidth & float\\
\EndFaceBoringDepth & float\\
\EndFaceBoringCornerR & float\\
\EndFaceBoringLength & float\\
\end{multicollongtblr}



\clearpage
%%%%%%%%%%%%%%%%%%%%%%%%%%%%%%%%%%%%%%%%%%%%%%%%%%%%%%%%%%
%% section 30.6 %%%%%%%%%%%%%%%%%%%%%%%%%%%%%%%%%%%%%%%%%%
%%%%%%%%%%%%%%%%%%%%%%%%%%%%%%%%%%%%%%%%%%%%%%%%%%%%%%%%%%
\modHeadsection{\IncutBoring に関する入力情報}

\begin{multicollongtblr}{入力情報:\IncutBoring}{X[l]c}
内容 & 型\\
\IncutBoringExists & boolean\\
\IncutBoringACWidth & float\\
\IncutBoringBDWidth & float\\
\IncutBoringCornerR & float\\
\IncutBoringLength & float\\
\end{multicollongtblr}


\clearrightpage

%!TEX root = ./RfCPN.tex


\modHeadchapter[lot]{必要な条件分岐情報(\index{NCプログラム}NCプログラム)}
入力するパラメタの中には\index{じょうけんぶんき@条件分岐}条件分岐に用いるものが含まれる。
それらを組み合わせることで、加工にパターン分けをすることができる。
ここではそうした\index{じょうけんぶんきじょうほう@条件分岐情報}条件分岐情報と、それに直接的に影響する\index{こうてい@工程}工程またはパラメタをまとめておく。



%%%%%%%%%%%%%%%%%%%%%%%%%%%%%%%%%%%%%%%%%%%%%%%%%%%%%%%%%%
%% section 46.01 %%%%%%%%%%%%%%%%%%%%%%%%%%%%%%%%%%%%%%%%%
%%%%%%%%%%%%%%%%%%%%%%%%%%%%%%%%%%%%%%%%%%%%%%%%%%%%%%%%%%
\modHeadsection{湾曲・\Alocation に関する\index{じょうけんぶんき@条件分岐}条件分岐}

\begin{multicollongtblr}{\index{じょうけんぶんき@条件分岐}条件分岐:\PMCenterCurvatureExists}{X[l]}
直接的に影響する工程\\
ワーク座標系 原点設定\\
外側中心$X$および幅$X$ 両側測定\\
\KeywayCenter$X$ 片側測定\\
内側中心$X$および幅$X$ 両側測定\\
\OutcutCenter$X$ 片側測定\\
\Dimple 各列の中心上への移動\\
\KeywayMilling\\
\EndFaceOutChamferMilling\\
\EndFaceInChamferMilling\\
\indexReliefGrooveMeasurement\indexReliefGrooveMilling\nameReliefGroove 測定および加工\\
\end{multicollongtblr}

\begin{multicollongtblr}{\index{じょうけんぶんき@条件分岐}条件分岐:\PMBottomALBracketDimensionExists}{X[l]}
直接的に影響するパラメタ\\
\AlocationLength\\
\end{multicollongtblr}



\clearpage
%%%%%%%%%%%%%%%%%%%%%%%%%%%%%%%%%%%%%%%%%%%%%%%%%%%%%%%%%%
%% section 46.02 %%%%%%%%%%%%%%%%%%%%%%%%%%%%%%%%%%%%%%%%%
%%%%%%%%%%%%%%%%%%%%%%%%%%%%%%%%%%%%%%%%%%%%%%%%%%%%%%%%%%
\modHeadsection{\Outcut に関する\index{じょうけんぶんき@条件分岐}条件分岐}

\begin{multicollongtblr}{\index{じょうけんぶんき@条件分岐}条件分岐:\PMTopOutcutExists}{X[l]}
直接的に影響する工程\\
ワーク座標系 原点設定\\
トップ外側中心$X$ 片側測定\\
\KeywayCenter$X$ 芯出し\\
\TopOutcutMilling\\
\KeywayMilling\\
\TopEndFaceOutChamferMilling\\
\CenterlineEndFaceDifMeasurement\\
\end{multicollongtblr}

\begin{multicollongtblr}{\index{じょうけんぶんき@条件分岐}条件分岐:\PMTopOutcutEndKeywayExists}{X[l]}
直接的に影響する工程\\
\TopOutcutMilling\\
\TopCurvedOutcutMilling\\
\end{multicollongtblr}

\begin{multicollongtblr}{\index{じょうけんぶんき@条件分岐}条件分岐:\PMBottomOutcutExists}{X[l]}
直接的に影響する工程\\
ワーク座標系 原点設定\\
ボトム外側中心$X$ 片側測定\\
\BottomOutcutMilling\\
\BottomEndFaceOutChamferMilling\\
\CenterlineEndFaceDifMeasurement\\
\end{multicollongtblr}

\begin{multicollongtblr}{\index{じょうけんぶんき@条件分岐}条件分岐:\PMSquareEndMillTaperExists}{X[l]}
直接的に影響する工程\\
\OutcutMilling\\
\CurvedOutcutMilling\\
\end{multicollongtblr}

\begin{multicollongtblr}{\index{じょうけんぶんき@条件分岐}条件分岐:\PMOutcutCenterReference}{X[l]}
直接的に影響する工程\\
\CenterlineEndFaceDifMeasurement\\
\end{multicollongtblr}



\clearpage
%%%%%%%%%%%%%%%%%%%%%%%%%%%%%%%%%%%%%%%%%%%%%%%%%%%%%%%%%%
%% section 46.03 %%%%%%%%%%%%%%%%%%%%%%%%%%%%%%%%%%%%%%%%%
%%%%%%%%%%%%%%%%%%%%%%%%%%%%%%%%%%%%%%%%%%%%%%%%%%%%%%%%%%
\modHeadsection{\Keyway に関する\index{じょうけんぶんき@条件分岐}条件分岐}

\begin{multicollongtblr}{\index{じょうけんぶんき@条件分岐}条件分岐:\PMKeywayCornerType}{X[l]}
直接的に影響する工程\\
\KeywayMilling\\
\end{multicollongtblr}

\begin{multicollongtblr}{\index{じょうけんぶんき@条件分岐}条件分岐:\PMAKDToleranceExists}{X[l]}
直接的に影響する工程\\
ワーク座標系 原点設定\\
\KeywayCenter$X$ 片側測定\\
\KeywayMilling\\
\end{multicollongtblr}



%\clearpage
%%%%%%%%%%%%%%%%%%%%%%%%%%%%%%%%%%%%%%%%%%%%%%%%%%%%%%%%%%
%% section 46.04 %%%%%%%%%%%%%%%%%%%%%%%%%%%%%%%%%%%%%%%%%
%%%%%%%%%%%%%%%%%%%%%%%%%%%%%%%%%%%%%%%%%%%%%%%%%%%%%%%%%%
\modHeadsection{\Dimple に関する\index{じょうけんぶんき@条件分岐}条件分岐}

\begin{multicollongtblr}{\index{じょうけんぶんき@条件分岐}条件分岐:\PMDimpleExists}{X[l]}
直接的に影響する工程\\
\DimpleMilling\\
\end{multicollongtblr}



%\clearpage
%%%%%%%%%%%%%%%%%%%%%%%%%%%%%%%%%%%%%%%%%%%%%%%%%%%%%%%%%%
%% section 46.05 %%%%%%%%%%%%%%%%%%%%%%%%%%%%%%%%%%%%%%%%%
%%%%%%%%%%%%%%%%%%%%%%%%%%%%%%%%%%%%%%%%%%%%%%%%%%%%%%%%%%
\modHeadsection{\EndFaceChamfer に関する\index{じょうけんぶんき@条件分岐}条件分岐}

\begin{multicollongtblr}{\index{じょうけんぶんき@条件分岐}条件分岐:\PMEndFaceOutChamferType}{X[l]}
直接的に影響する工程\\
\EndFaceOutChamferMilling\\
\end{multicollongtblr}

\begin{multicollongtblr}{\index{じょうけんぶんき@条件分岐}条件分岐:\PMEndFaceInChamferType}{X[l]}
直接的に影響する工程\\
\EndFaceInChamferMilling\\
\end{multicollongtblr}



\clearpage
%%%%%%%%%%%%%%%%%%%%%%%%%%%%%%%%%%%%%%%%%%%%%%%%%%%%%%%%%%
%% section 46.06 %%%%%%%%%%%%%%%%%%%%%%%%%%%%%%%%%%%%%%%%%
%%%%%%%%%%%%%%%%%%%%%%%%%%%%%%%%%%%%%%%%%%%%%%%%%%%%%%%%%%
\modHeadsection{\index{とくしゅなかこう@特殊な加工}特殊な加工に関する\index{じょうけんぶんき@条件分岐}条件分岐}

\begin{multicollongtblr}{\index{じょうけんぶんき@条件分岐}条件分岐:\PMEndFaceBoringExists}{X[l]}
直接的に影響する工程\\
\EndFaceBoringMilling\\
\end{multicollongtblr}

\begin{multicollongtblr}{\index{じょうけんぶんき@条件分岐}条件分岐:\PMIncutBoringExists}{X[l]}
直接的に影響する工程\\
\IncutBoringMilling\\
\TopEndFaceInChamferMilling\\
\end{multicollongtblr}

\begin{multicollongtblr}{\index{じょうけんぶんき@条件分岐}条件分岐:\PMCurvedOutcutExists}{X[l]}
直接的に影響する工程\\
\CurvedOutcutMilling\\
\EndFaceOutChamferMilling\\
\end{multicollongtblr}


%%%%%%%%%%%%%%%%%%%%%%%%%%%%%%%%%%%%%%%%%%%%%%%%%%%%%%%%%
%% Appendiodes %%%%%%%%%%%%%%%%%%%%%%%%%%%%%%%%%%%%%%%%%%
%%%%%%%%%%%%%%%%%%%%%%%%%%%%%%%%%%%%%%%%%%%%%%%%%%%%%%%%%
\begin{appendices}
\Appendixpart
%!TEX root = ./RfCPN.tex


\modHeadchapter{計算の必要な数値計算(サブプログラム用)\TBW}
ここでは主に\index{サブプログラム}サブプログラムの記述に際して、\index{すうちけいさん@数値計算}数値計算に必要な部分をピックアップする。
なお、ここでは主に\DMC について述べるため、\index{スペーサ}スペーサに関するものは省略する。



%%%%%%%%%%%%%%%%%%%%%%%%%%%%%%%%%%%%%%%%%%%%%%%%%%%%%%%%%%
%% section H.1 %%%%%%%%%%%%%%%%%%%%%%%%%%%%%%%%%%%%%%%%%%%
%%%%%%%%%%%%%%%%%%%%%%%%%%%%%%%%%%%%%%%%%%%%%%%%%%%%%%%%%%
\modHeadsection{\Outcut の数値情報}


%%%%%%%%%%%%%%%%%%%%%%%%%%%%%%%%%%%%%%%%%%%%%%%%%%%%%%%%%%
%% subsection H.3.1 %%%%%%%%%%%%%%%%%%%%%%%%%%%%%%%%%%%%%%
%%%%%%%%%%%%%%%%%%%%%%%%%%%%%%%%%%%%%%%%%%%%%%%%%%%%%%%%%%
\subsection{\OutcutCenter:\BottomOutcutAsideThickness 基準の場合}
\index{テーブルちゅうしん@テーブル中心}テーブル中心\index{P(テーブルちゅうしん)@P(テーブル中心)}Pを\index{げんてんP@原点P}原点とした\BottomOutcutCenter$\mathfrak B_\mathrm c'$の(おおよその)$X$座標は、\pageeqref{eq:gaisakucenterBt}より、
\begin{align*}
  \HLbox{%
    \Delta_x'\cos\theta
    -\frac{\sqrt{R_\mathrm o^2-f_\mathrm B^2}+\sqrt{R_\mathrm i^2-f_\mathrm B^2}}2
    -\frac{w_\mathrm B}2
    -\tau_\mathrm B
    +\frac{\mathfrak W_\mathrm B}2
  }\ .
\end{align*}
このとき、測定したA側内面b$_\mathrm o'$の$X$座標が\pageeqref{eq:gaisakucenterBr}となるように、原点$\mathfrak B_\mathrm c'$を定める。
\begin{align*}
  \HLbox{-\left(\frac{\mathfrak W_\mathrm B}2-\tau_\mathrm B+\mu\right)}\ .
\end{align*}
トップ側にも\Outcut がある場合、測定で定めた$\mathfrak B_\mathrm c'$の$X$座標$\mathcal G_{\mathrm Bx}$および\CenterlineEndFaceDifAC$T_x$を用いて\pageeqref{eq:BbasedTx}で与えられる。
\begin{align*}
  \HLbox{-\mathcal G_{Bx}+T_x}\ .
\end{align*}


%%%%%%%%%%%%%%%%%%%%%%%%%%%%%%%%%%%%%%%%%%%%%%%%%%%%%%%%%%
%% subsection H.3.2 %%%%%%%%%%%%%%%%%%%%%%%%%%%%%%%%%%%%%%
%%%%%%%%%%%%%%%%%%%%%%%%%%%%%%%%%%%%%%%%%%%%%%%%%%%%%%%%%%
\subsection{\OutcutCenter:\TopOutcutAsideThickness 基準の場合}
\index{テーブルちゅうしん@テーブル中心}テーブル中心\index{P(テーブルちゅうしん)@P(テーブル中心)}Pを\index{げんてんP@原点P}原点とした\TopOutcutCenter$\mathfrak T_\mathrm c'$の(おおよその)$X$座標は、\pageeqref{eq:gaisakucenterTt}より、
\begin{align*}
  \HLbox{%
    \frac{\sqrt{R_\mathrm o^2-f_\mathrm T^2}+\sqrt{R_\mathrm i^2-f_\mathrm T^2}}2
    -\Delta_x'\cos\theta
    +\frac{w_\mathrm T}2
    +\tau_\mathrm T
    -\frac{\mathfrak W_\mathrm T}2
  }\ .
\end{align*}
このとき、測定したA側内面t$_\mathrm o'$の$X$座標が\pageeqref{eq:gaisakucenterTr}となるように、原点$\mathfrak T_\mathrm c'$を定める。
\begin{align*}
  \HLbox{\frac{\mathfrak W_\mathrm T}2-\tau_\mathrm T+\mu}~.
\end{align*}
ボトム側にも\Outcut がある場合、測定で定めた$\mathfrak T_\mathrm c'$の$X$座標$\mathcal G_{\mathrm Tx}$および\CenterlineEndFaceDifAC$T_x$を用いて\pageeqref{eq:TbasedTx}で与えられる。
\begin{align*}
  \HLbox{-\mathcal G_{Tx}+T_x}\ .
\end{align*}


%%%%%%%%%%%%%%%%%%%%%%%%%%%%%%%%%%%%%%%%%%%%%%%%%%%%%%%%%%
%% subsection 30.3.3 %%%%%%%%%%%%%%%%%%%%%%%%%%%%%%%%%%%%%
%%%%%%%%%%%%%%%%%%%%%%%%%%%%%%%%%%%%%%%%%%%%%%%%%%%%%%%%%%
\subsection{\OutcutLength}

%%%%%%%%%%%%%%%%%%%%%%%%%%%%%%%%%%%%%%%%%%%%%%%%%%%%%%%%%%
%% subsubsection 30.3.3.1 %%%%%%%%%%%%%%%%%%%%%%%%%%%%%%%%
%%%%%%%%%%%%%%%%%%%%%%%%%%%%%%%%%%%%%%%%%%%%%%%%%%%%%%%%%%
\subsubsection{\BottomOutcutLength}
\BottomOutcut における\index{こうぐ@工具}工具の先端の$Z$座標は、\BottomOutcutLength を$h_\mathrm B$として、
\begin{align*}
  \HLbox{f_\mathrm B'-h_\mathrm B}\ .
\end{align*}

%\clearpage
%%%%%%%%%%%%%%%%%%%%%%%%%%%%%%%%%%%%%%%%%%%%%%%%%%%%%%%%%%
%% subsubsection 30.3.3.1 %%%%%%%%%%%%%%%%%%%%%%%%%%%%%%%%
%%%%%%%%%%%%%%%%%%%%%%%%%%%%%%%%%%%%%%%%%%%%%%%%%%%%%%%%%%
\subsubsection{\TopOutcutLength}
\TopOutcut における工具の先端の$Z$座標は、\TopOutcutLength, \KeywayPos, \KeywayWidth をそれぞれ$h_\mathrm T$, $\kappa_p$, $\kappa_w$として、
\begin{alignat*}{3}
  & \HLbox{f_\mathrm T'-h_\mathrm T} & \quad & \Big(\text{if}~h_\mathrm T > \kappa_p+\kappa_w\Big)\\
  & \HLbox{f_\mathrm T'-\left(\kappa_p+1[\mathrm{mm}]\right)} & \quad  & \Big(\text{if}~h_\mathrm T = \kappa_p+\kappa_w\Big)
\end{alignat*}


%\clearpage
%%%%%%%%%%%%%%%%%%%%%%%%%%%%%%%%%%%%%%%%%%%%%%%%%%%%%%%%%%
%% subsection 30.3.4 %%%%%%%%%%%%%%%%%%%%%%%%%%%%%%%%%%%%%
%%%%%%%%%%%%%%%%%%%%%%%%%%%%%%%%%%%%%%%%%%%%%%%%%%%%%%%%%%
\subsection{\CurvedOutcut\TBW}
(to be written...)



\clearpage
%%%%%%%%%%%%%%%%%%%%%%%%%%%%%%%%%%%%%%%%%%%%%%%%%%%%%%%%%%
%% section 30.4 %%%%%%%%%%%%%%%%%%%%%%%%%%%%%%%%%%%%%%%%%%
%%%%%%%%%%%%%%%%%%%%%%%%%%%%%%%%%%%%%%%%%%%%%%%%%%%%%%%%%%
\modHeadsection{\Keyway の数値情報}


%%%%%%%%%%%%%%%%%%%%%%%%%%%%%%%%%%%%%%%%%%%%%%%%%%%%%%%%%%
%% subsection 30.4.1 %%%%%%%%%%%%%%%%%%%%%%%%%%%%%%%%%%%%%
%%%%%%%%%%%%%%%%%%%%%%%%%%%%%%%%%%%%%%%%%%%%%%%%%%%%%%%%%%
\subsection{\KeywayCenter\texorpdfstring{$Z$}{Z}}
\KeywayPos$\kappa_p$および\KeywayWidth$\kappa_w$に対し、\index{テーブルちゅうしん@テーブル中心}テーブル中心\index{P(テーブルちゅうしん)@P(テーブル中心)}Pを\index{げんてんP@原点P}原点とした\KeywayCenter M$'$の$Z$座標は、\pageeqref{eq:mizocenterZ}より
\begin{align*}
  \HLbox{f_\mathrm T'-\kappa_p-\frac{\kappa_w}2}\ .
\end{align*}


%%%%%%%%%%%%%%%%%%%%%%%%%%%%%%%%%%%%%%%%%%%%%%%%%%%%%%%%%%
%% subsection 30.4.2 %%%%%%%%%%%%%%%%%%%%%%%%%%%%%%%%%%%%%
%%%%%%%%%%%%%%%%%%%%%%%%%%%%%%%%%%%%%%%%%%%%%%%%%%%%%%%%%%
\subsection{\CurvatureCenter が基準の場合}
\index{トップたんのそとがわちゅうしん@トップ端の外側中心}トップ端の外側中心T$_\mathrm c'$と\KeywayCenter M$'$との$X$方向の差は、\pageeqref{eq:difTopMizoCenter}より、
\begin{align*}
  \HLbox{%
    \sqrt{R_\mathrm c^2-\left(f_\mathrm T-\kappa_p-\frac{\kappa_w}2\right)^2}
    -\frac{\sqrt{R_\mathrm o^2-f_\mathrm T^2}+\sqrt{R_\mathrm i^2-f_\mathrm T^2}}2%
  }\ .
\end{align*}


%%%%%%%%%%%%%%%%%%%%%%%%%%%%%%%%%%%%%%%%%%%%%%%%%%%%%%%%%%
%% subsection 30.4.3 %%%%%%%%%%%%%%%%%%%%%%%%%%%%%%%%%%%%%
%%%%%%%%%%%%%%%%%%%%%%%%%%%%%%%%%%%%%%%%%%%%%%%%%%%%%%%%%%
\subsection{\OutcutCenter が基準の場合}
\KeywayCenter は\TopOutcutCenter とする。


%%%%%%%%%%%%%%%%%%%%%%%%%%%%%%%%%%%%%%%%%%%%%%%%%%%%%%%%%%
%% subsection 30.4.4 %%%%%%%%%%%%%%%%%%%%%%%%%%%%%%%%%%%%%
%%%%%%%%%%%%%%%%%%%%%%%%%%%%%%%%%%%%%%%%%%%%%%%%%%%%%%%%%%
\subsection{\AsideKeywayDepth 指定の場合}

%%%%%%%%%%%%%%%%%%%%%%%%%%%%%%%%%%%%%%%%%%%%%%%%%%%%%%%%%%
%% subsubsection 30.4.4.1 %%%%%%%%%%%%%%%%%%%%%%%%%%%%%%%%
%%%%%%%%%%%%%%%%%%%%%%%%%%%%%%%%%%%%%%%%%%%%%%%%%%%%%%%%%%
\subsubsection{\Outcut のない場合}
\AsideKeywayDepth$\kappa_d$は、その測定値$\kappa_d'$が\index{ずめん(モールド)@図面(モールド)}図面上の値となるように与えられるものとする。このとき\pageeqref{eq:keydepthDif1}より、
\begin{align*}
  \HLbox{%
    \kappa_d
    = \frac{2\kappa_d'-\kappa_w\sin\zeta}{1+\cos^2\zeta}\cos\zeta
      +\sqrt{R_\mathrm o^2-\left(f_\mathrm T-\kappa_p-\frac{\kappa_w}2\right)^2}
      -\sqrt{R_\mathrm o^2-\left(f_\mathrm T-\kappa_p\right)^2}%
  }\ .
\end{align*}
ここで$\zeta$は\pageeqref{eq:angleZeta}より、
\begin{align*}
  \HLbox{%
    \tan\zeta
    = \frac{\sqrt{R_\mathrm o^2-\left(f_\mathrm T-\kappa_p-\kappa_w\right)^2}
            -\sqrt{R_\mathrm o^2-\left(f_\mathrm T-\kappa_p\right)^2}}
           {\kappa_w}%
  }\ .
\end{align*}
\AsideKeywayDepth$\kappa_d$に対し、\KeywayCenter の位置の$X$座標は\pageeqref{eq:mizocenterA}より、
\begin{align*}
  \HLbox{%
    \sqrt{R_\mathrm o^2-\left(f_\mathrm T-\kappa_p-\frac{\kappa_w}2\right)^2}
    -\kappa_d
    -\frac{W_{mx}}2
    -\Delta_x%
  }\ .
\end{align*}
また\expandafterindex{Aがわがいめん(\yomiKeywayCenter)@A側外面(\nameKeywayCenter)}A側外面の\index{じっそくち@実測値}実測値を$\mathcal G_m$とすると、\KeywayCenter と$G_m$との$X$座標の差は、\pageeqref{eq:mizocenterAd}より、
\begin{align*}
  \HLbox{-\frac{W_{mx}}2-\kappa_d}\ .
\end{align*}

\clearpage
%%%%%%%%%%%%%%%%%%%%%%%%%%%%%%%%%%%%%%%%%%%%%%%%%%%%%%%%%%
%% subsubsection 30.4.4.2 %%%%%%%%%%%%%%%%%%%%%%%%%%%%%%%%
%%%%%%%%%%%%%%%%%%%%%%%%%%%%%%%%%%%%%%%%%%%%%%%%%%%%%%%%%%
\subsubsection{\Outcut のある場合}
\AsideKeywayDepth$\kappa_d$に対し、\TopOutcutCenter$X$を$\mathcal G_{\mathrm Tx}$とすると、$\mathcal G_{\mathrm Tx}$と\KeywayCenter との$X$座標の差は、\pageeqref{eq:mizocenterAG}より、
\begin{align*}
  \HLbox{\frac{\mathfrak W_x}2-\kappa_d-\frac{W_{mx}}2}\ .
\end{align*}



\clearpage
%%%%%%%%%%%%%%%%%%%%%%%%%%%%%%%%%%%%%%%%%%%%%%%%%%%%%%%%%%
%% section 9.2 %%%%%%%%%%%%%%%%%%%%%%%%%%%%%%%%%%%%%%%%%%%
%%%%%%%%%%%%%%%%%%%%%%%%%%%%%%%%%%%%%%%%%%%%%%%%%%%%%%%%%%
\modHeadsection{\EndFaceOutCChamfer の数値情報\TBW}
(to be written...)



%\clearpage
%%%%%%%%%%%%%%%%%%%%%%%%%%%%%%%%%%%%%%%%%%%%%%%%%%%%%%%%%%
%% section 9.2 %%%%%%%%%%%%%%%%%%%%%%%%%%%%%%%%%%%%%%%%%%%
%%%%%%%%%%%%%%%%%%%%%%%%%%%%%%%%%%%%%%%%%%%%%%%%%%%%%%%%%%
\modHeadsection{\EndFaceInCChamfer の数値情報\TBW}
(to be written...)



%%%%%%%%%%%%%%%%%%%%%%%%%%%%%%%%%%%%%%%%%%%%%%%%%%%%%%%%%%
%% section 9.2 %%%%%%%%%%%%%%%%%%%%%%%%%%%%%%%%%%%%%%%%%%%
%%%%%%%%%%%%%%%%%%%%%%%%%%%%%%%%%%%%%%%%%%%%%%%%%%%%%%%%%%
\modHeadsection{\EndFaceOutRChamfer の数値情報\TBW}
(to be written...)



%\clearpage
%%%%%%%%%%%%%%%%%%%%%%%%%%%%%%%%%%%%%%%%%%%%%%%%%%%%%%%%%%
%% section 9.2 %%%%%%%%%%%%%%%%%%%%%%%%%%%%%%%%%%%%%%%%%%%
%%%%%%%%%%%%%%%%%%%%%%%%%%%%%%%%%%%%%%%%%%%%%%%%%%%%%%%%%%
\modHeadsection{\EndFaceInRChamfer の数値情報\TBW}
(to be written...)


%\clearpage
%%%%%%%%%%%%%%%%%%%%%%%%%%%%%%%%%%%%%%%%%%%%%%%%%%%%%%%%%%
%% section 9.2 %%%%%%%%%%%%%%%%%%%%%%%%%%%%%%%%%%%%%%%%%%%
%%%%%%%%%%%%%%%%%%%%%%%%%%%%%%%%%%%%%%%%%%%%%%%%%%%%%%%%%%
\modHeadsection{\EndFaceBoring の数値情報\TBW}
(to be written...)



\clearpage
%%%%%%%%%%%%%%%%%%%%%%%%%%%%%%%%%%%%%%%%%%%%%%%%%%%%%%%%%%
%% section H.2 %%%%%%%%%%%%%%%%%%%%%%%%%%%%%%%%%%%%%%%%%%%
%%%%%%%%%%%%%%%%%%%%%%%%%%%%%%%%%%%%%%%%%%%%%%%%%%%%%%%%%%
\modHeadsection{\Dimple の数値情報\TBW}
(to be written...)

\end{appendices}

\addtocontents{toc}{\protect\end{tocBox}}
