%!TEX root = ../RfCPN.tex


\modHeadchapter[lot]{\index{シーケンスばんごう@シーケンス番号}シーケンス番号(\index{Nコード}Nコード値)}
一般に、\index{シーケンスばんごう@シーケンス番号}シーケンス番号は(重複していなければ)自由に付けて問題ない。
しかしこれに一定のルールを与えておくことで、\index{NCプログラム}NCプログラムの
\begin{enumerate}[label=\sarrow]
\item どの部分で何が行われているか
\item どの部分で\index{エラー}エラーが起きているか
\item 途中から稼働する場合、どの\index{ブロック}ブロックから始めればよいか
\end{enumerate}
など、作業や管理をする際に効率よく制御することが可能になる。

そこで、ここでは\index{シーケンスばんごう@シーケンス番号}シーケンス番号(\index{Nコード}Nコード)についての規則を与える。



%%%%%%%%%%%%%%%%%%%%%%%%%%%%%%%%%%%%%%%%%%%%%%%%%%%%%%%%%%
%% section 14.1 %%%%%%%%%%%%%%%%%%%%%%%%%%%%%%%%%%%%%%%%%%
%%%%%%%%%%%%%%%%%%%%%%%%%%%%%%%%%%%%%%%%%%%%%%%%%%%%%%%%%%
\modHeadsection{\index{シーケンスばんごう@シーケンス番号}シーケンス番号の基本事項}
\begin{enumerate}[label=\Roman*), ref=\Roman*)]
\item \index{シーケンスばんごう@シーケンス番号}シーケンス番号は3桁または4桁とし、0埋めする
\item \index{NCプログラム}NCプログラムの始まりの\index{シーケンスばんごう@シーケンス番号}シーケンス番号は{\ttfamily N001}または{\ttfamily N0001}とする
\item 原則として、\index{シーケンスばんごう@シーケンス番号}シーケンス番号は昇順とし、特に1桁目は連番とする
\end{enumerate}


%%%%%%%%%%%%%%%%%%%%%%%%%%%%%%%%%%%%%%%%%%%%%%%%%%%%%%%%%%
%% section 14.2 %%%%%%%%%%%%%%%%%%%%%%%%%%%%%%%%%%%%%%%%%%
%%%%%%%%%%%%%%%%%%%%%%%%%%%%%%%%%%%%%%%%%%%%%%%%%%%%%%%%%%
\modHeadsection{サブプログラムのシーケンス番号}
\DMC においては、原則として\index{サブプログラム}サブプログラムは始めから実行されるものであり、途中の部分から実行されることはない。
そのため、サブプログラムについてはシーケンス番号は記述の順に(概ね\index{ブロック}ブロックごとに)連番とする。

なお、エラー検出時に関するシーケンス番号、および\index{プログラムしゅうりょう@プログラム終了}プログラム終了に関するシーケンス番号については、以降で述べる\index{メインプログラム}メインプログラムのもの(\autoref{subsec:sequenceNerror}, \pageautoref{subsec:sequenceNprgEnd})と同様とする。



\clearpage
%%%%%%%%%%%%%%%%%%%%%%%%%%%%%%%%%%%%%%%%%%%%%%%%%%%%%%%%%%
%% section 14.3 %%%%%%%%%%%%%%%%%%%%%%%%%%%%%%%%%%%%%%%%%%
%%%%%%%%%%%%%%%%%%%%%%%%%%%%%%%%%%%%%%%%%%%%%%%%%%%%%%%%%%
\modHeadsection{メインプログラムのシーケンス番号}
\DMC において、\index{メインプログラム}メインプログラム
%% footnote %%%%%%%%%%%%%%%%%%%%%
\footnote{ここでいうメインプログラムとは、下5桁が\DrawingNumber と一致するものを指す。}
%%%%%%%%%%%%%%%%%%%%%%%%%%%%%%%%%
は\index{さぎょうしゃ@作業者}作業者が実際に設定を変更したり、途中の箇所から始めたりし得る。
そのため、メインプログラムについては各作業(測定・加工)ごとにシーケンス番号を割り振ることにする。


%%%%%%%%%%%%%%%%%%%%%%%%%%%%%%%%%%%%%%%%%%%%%%%%%%%%%%%%%%
%% subsection 14.3.1 %%%%%%%%%%%%%%%%%%%%%%%%%%%%%%%%%%%%%
%%%%%%%%%%%%%%%%%%%%%%%%%%%%%%%%%%%%%%%%%%%%%%%%%%%%%%%%%%
\subsection{N100:測定(\Dimple・\ReliefGroove 以外)}
\index{タッチセンサープローブ}タッチセンサープローブを用いた測定(\Dimple ・\ReliefGroove を除く)を行う工程のシーケンス番号は100番台とする。
これには以下の\index{こうてい@工程}工程が含まれ、これらは2桁目の番号で区別される。
\begin{enumerate}
\item[100:] 芯出し測定
\item[650:] \expandafterindex{\yomiCenterlineEndFaceDif そくてい@\nameCenterlineEndFaceDif 測定}\nameCenterlineEndFaceDif 測定
\end{enumerate}


%\clearpage
%%%%%%%%%%%%%%%%%%%%%%%%%%%%%%%%%%%%%%%%%%%%%%%%%%%%%%%%%%
%% subsection 14.3.2 %%%%%%%%%%%%%%%%%%%%%%%%%%%%%%%%%%%%%
%%%%%%%%%%%%%%%%%%%%%%%%%%%%%%%%%%%%%%%%%%%%%%%%%%%%%%%%%%
\subsection{N200:測定(\Dimple・\ReliefGroove)}
\Dimple および\ReliefGroove に関する\index{タッチセンサープローブ}タッチセンサープローブを用いた測定を行う工程のシーケンス番号は200番台とする。
これには以下の工程が含まれ、これらは2桁目の番号で区別される。
\begin{enumerate}
\item[200:] \DimpleMeasurement
\item[250:] \ReliefGrooveMeasurement
\end{enumerate}


%%%%%%%%%%%%%%%%%%%%%%%%%%%%%%%%%%%%%%%%%%%%%%%%%%%%%%%%%%
%% subsection 14.3.1 %%%%%%%%%%%%%%%%%%%%%%%%%%%%%%%%%%%%%
%%%%%%%%%%%%%%%%%%%%%%%%%%%%%%%%%%%%%%%%%%%%%%%%%%%%%%%%%%
\subsection{N300:\DimpleMilling ・\ReliefGrooveMilling}
\DimpleMilling および\ReliefGrooveMilling を行う工程のシーケンス番号は300番台とする。
これには以下の工程が含まれ、これらは2桁目の番号で区別される。
\begin{enumerate}
\item[300:] \DimpleMilling
\item[350:] \ReliefGrooveMilling
\end{enumerate}


%%%%%%%%%%%%%%%%%%%%%%%%%%%%%%%%%%%%%%%%%%%%%%%%%%%%%%%%%%
%% subsection 14.3.1 %%%%%%%%%%%%%%%%%%%%%%%%%%%%%%%%%%%%%
%%%%%%%%%%%%%%%%%%%%%%%%%%%%%%%%%%%%%%%%%%%%%%%%%%%%%%%%%%
\subsection{N400:トップ側の加工}
トップ側の加工を行う工程のシーケンス番号は400番台とする。
これには以下の工程が含まれ、これらは2桁目の番号で区別される。
\begin{enumerate}
\item[400:] \TopEndFacecutMilling
\item[410:] \TopOutcutMilling
\item[420:] \KeywayMilling
\item[430:] \TopEndFaceOutCChamferMilling
\item[440:] \TopEndFaceInCChamferMilling
\item[450:] \EndFaceBoringMilling
\end{enumerate}


\clearpage
%%%%%%%%%%%%%%%%%%%%%%%%%%%%%%%%%%%%%%%%%%%%%%%%%%%%%%%%%%
%% subsection 14.2.1 %%%%%%%%%%%%%%%%%%%%%%%%%%%%%%%%%%%%%
%%%%%%%%%%%%%%%%%%%%%%%%%%%%%%%%%%%%%%%%%%%%%%%%%%%%%%%%%%
\subsection{N500:ボトム側の加工}
ボトム側の加工を行う工程の\index{シーケンスばんごう@シーケンス番号}シーケンス番号は500番台とする。
これには以下の工程が含まれ、これらは2桁目の番号で区別される。
\begin{enumerate}
\item[500:] \BottomEndFacecutMilling
\item[510:] \BottomOutcutMilling
\item[530:] \BottomEndFaceOutCChamferMilling
\item[540:] \BottomEndFaceInCChamferMilling
\end{enumerate}


%\clearpage
%%%%%%%%%%%%%%%%%%%%%%%%%%%%%%%%%%%%%%%%%%%%%%%%%%%%%%%%%%
%% subsection 14.2.1 %%%%%%%%%%%%%%%%%%%%%%%%%%%%%%%%%%%%%
%%%%%%%%%%%%%%%%%%%%%%%%%%%%%%%%%%%%%%%%%%%%%%%%%%%%%%%%%%
\subsection{N800:エラー\label{subsec:sequenceNerror}\vphantom{\ref{subsec:sequenceNerror}}\TBW}
\index{エラー}エラー検出時に\index{ジャンプ}ジャンプするシーケンス番号は800番台とする。
エラーの種類(\index{システムへんすう@システム変数}システム変数\ttNum3000の値)に応じて以下のように分類し、(概ね)\index{ブロック}プロックごとに連番とする。
\begin{enumerate}
\item[800:] \verb|#3000=121 (Argument is not assigned)|
\item[810:] \verb|#3000=...|
\item[820:] \verb|#3000=1 (Pallet Alarm)|,\\
            \verb|#3000=145 (Sensor-Low-Battery)|, \verb|#3000=146 (Sensor-Alarm)|
\end{enumerate}


%%%%%%%%%%%%%%%%%%%%%%%%%%%%%%%%%%%%%%%%%%%%%%%%%%%%%%%%%%
%% subsection 14.2.1 %%%%%%%%%%%%%%%%%%%%%%%%%%%%%%%%%%%%%
%%%%%%%%%%%%%%%%%%%%%%%%%%%%%%%%%%%%%%%%%%%%%%%%%%%%%%%%%%
\subsection{N990:NCプログラムの終了\label{subsec:sequenceNprgEnd}}
\index{こうてい(プログラムしゅうりょう)@工程(プログラム終了)}NCプログラムを終了する工程の\index{シーケンスばんごう@シーケンス番号}シーケンス番号は990番台または9990番台とする。
特に、\index{NCプログラム}NCプログラムの終了はN999またはN9999とする。



\clearpage
\noindent
改めて上記の\index{シーケンスばんごういちらん@シーケンス番号一覧}シーケンス番号を一覧にしておく。\\

%%%%%%%%%%%%%%%%%%%%%%%%%%%%%%%%%%%%%%%%%%%%%%%%%%%%%%%%%%
%% sequence numbers %%%%%%%%%%%%%%%%%%%%%%%%%%%%%%%%%%%%%%
%%%%%%%%%%%%%%%%%%%%%%%%%%%%%%%%%%%%%%%%%%%%%%%%%%%%%%%%%%
\begin{multicollongtblr}{シーケンス番号 一覧(メインプログラム)\TBW}{cX[l]}
N番号 & 内容\\
\ttfamily N001 & \index{NCプログラム}NCプログラムの始まり\\
\ttfamily N10x & タッチセンサープローブ測定(\index{しんだしそくてい@芯出し測定}芯出し測定)\\
\ttfamily N20x & タッチセンサープローブ測定(\DimpleMeasurement)\\
\ttfamily N25x & タッチセンサープローブ測定(\ReliefGrooveMeasurement)\\
\ttfamily N30x & \DimpleMilling\\
\ttfamily N35x & \ReliefGrooveMilling\\
\ttfamily N40x & \TopEndFacecutMilling\\
\ttfamily N41x & \TopOutcutMilling\\
\ttfamily N42x & \KeywayMilling\\
\ttfamily N43x & \TopEndFaceOutCChamferMilling\\
\ttfamily N44x & \TopEndFaceInCChamferMilling\\
\ttfamily N45x & \EndFaceBoringMilling\\
\ttfamily N500 & \BottomEndFacecutMilling\\
\ttfamily N51x & \BottomOutcutMilling\\
\ttfamily N53x & \BottomEndFaceOutCChamferMilling\\
\ttfamily N54x & \BottomEndFaceInCChamferMilling\\
\ttfamily N65x & タッチセンサープローブ測定(\CenterlineEndFaceDifMeasurement)\\
\ttfamily N80x & \index{エラー(ひきすう)@エラー(引数)}引数によるエラー\\
\ttfamily N81x\TBW & \\
\ttfamily N82x & \index{エラー(パレット)}\index{エラー(タッチセンサープローブ)}パレットまたはタッチセンサープローブによるエラー\\
\ttfamily N99x & \index{こうてい(プログラムしゅうりょう)@工程(プログラム終了)}プログラム終了の工程\\
\ttfamily N999 & \index{プログラムしゅうりょう@プログラム終了}プログラム終了({\ttfamily M02}または{\ttfamily M30}または{\ttfamily M99})
\end{multicollongtblr}


