%!TEX root = ../RfCPN.tex


\modHeadchapter{\index{マシニングセンタ}マシニングセンタにおける\index{すんぽう@寸法}寸法}
ここでは\index{NCプログラム}NCプログラムを記述する際や、\expandafterindex{\yomiDrawing(モールド)@\nameDrawing(モールド)}\nameDrawing・\index{3Dモデル(ワーク)}3Dモデルの描画をする際に必要となる、\index{すんぽう@寸法}寸法や\index{こうさ@公差}公差等の取り扱いについて触れる。

なお、以降で述べる水平方向とは、\EndFace のAC方向のことを指す。



%%%%%%%%%%%%%%%%%%%%%%%%%%%%%%%%%%%%%%%%%%%%%%%%%%%%%%%%%%
%% section 23.01 %%%%%%%%%%%%%%%%%%%%%%%%%%%%%%%%%%%%%%%%%
%%%%%%%%%%%%%%%%%%%%%%%%%%%%%%%%%%%%%%%%%%%%%%%%%%%%%%%%%%
\modHeadsection{\index{すんぽう@寸法}寸法における基本事項}


%%%%%%%%%%%%%%%%%%%%%%%%%%%%%%%%%%%%%%%%%%%%%%%%%%%%%%%%%%
%% subsection 23.01.01 %%%%%%%%%%%%%%%%%%%%%%%%%%%%%%%%%%%
%%%%%%%%%%%%%%%%%%%%%%%%%%%%%%%%%%%%%%%%%%%%%%%%%%%%%%%%%%
\subsection{\index{すんぽうこうさ@寸法公差}寸法公差の取扱い}
全般的に、\index{すんぽうこうさ@寸法公差}寸法公差がある場合、\index{+こうさ@$+$公差}$+$公差と\index{-こうさ@$-$公差}$-$公差の中央(算術平均)を見るものとする。
ただし、\IDTaperTable の数値については、この限りではない。
%%%%%%%%%%%%%%%%%%%%%%%%%%%%%%%%%%%%%%%%%%%%%%%%%%%%%%%%%%
%% hosoku %%%%%%%%%%%%%%%%%%%%%%%%%%%%%%%%%%%%%%%%%%%%%%%%
%%%%%%%%%%%%%%%%%%%%%%%%%%%%%%%%%%%%%%%%%%%%%%%%%%%%%%%%%%
\begin{hosoku}
たとえば、$100^{+0.5}_{\phantom -0}$であれば、100.25とみなす。
\end{hosoku}
%%%%%%%%%%%%%%%%%%%%%%%%%%%%%%%%%%%%%%%%%%%%%%%%%%%%%%%%%%
%%%%%%%%%%%%%%%%%%%%%%%%%%%%%%%%%%%%%%%%%%%%%%%%%%%%%%%%%%
%%%%%%%%%%%%%%%%%%%%%%%%%%%%%%%%%%%%%%%%%%%%%%%%%%%%%%%%%%


%%%%%%%%%%%%%%%%%%%%%%%%%%%%%%%%%%%%%%%%%%%%%%%%%%%%%%%%%%
%% subsection 23.01.02 %%%%%%%%%%%%%%%%%%%%%%%%%%%%%%%%%%%
%%%%%%%%%%%%%%%%%%%%%%%%%%%%%%%%%%%%%%%%%%%%%%%%%%%%%%%%%%
\subsection{\index{ゆうせんど(すんぽう)@優先度(寸法)}寸法の優先度}
\index{こうさのあるすんぽう@公差のある寸法}公差のある寸法と\index{こうさのないすんぽう@公差のない寸法}公差のない寸法(\index{かっこすんぽう@括弧寸法}括弧寸法含む)とが共存して記載されている場合、\index{こうさのあるすんぽう@公差のある寸法}公差のある寸法を優先する。

ただし、\index{とっきじこう(すんぽう)@特記事項(寸法)}特記事項等がある場合は、それを優先するものとする。
%%%%%%%%%%%%%%%%%%%%%%%%%%%%%%%%%%%%%%%%%%%%%%%%%%%%%%%%%%
%% hosoku %%%%%%%%%%%%%%%%%%%%%%%%%%%%%%%%%%%%%%%%%%%%%%%%
%%%%%%%%%%%%%%%%%%%%%%%%%%%%%%%%%%%%%%%%%%%%%%%%%%%%%%%%%%
\begin{hosoku}
たとえば、2つの線の\index{すんぽう@寸法}寸法がそれぞれ$12^{+0.1}_{\phantom -0}$, 4.05と記述されていて、かつその和に相当する部分の\index{すんぽう@寸法}寸法が16と記述されている場合は、16.10とみなす。
\end{hosoku}
%%%%%%%%%%%%%%%%%%%%%%%%%%%%%%%%%%%%%%%%%%%%%%%%%%%%%%%%%%
%%%%%%%%%%%%%%%%%%%%%%%%%%%%%%%%%%%%%%%%%%%%%%%%%%%%%%%%%%
%%%%%%%%%%%%%%%%%%%%%%%%%%%%%%%%%%%%%%%%%%%%%%%%%%%%%%%%%%



\clearpage
%%%%%%%%%%%%%%%%%%%%%%%%%%%%%%%%%%%%%%%%%%%%%%%%%%%%%%%%%%
%% section 23.02 %%%%%%%%%%%%%%%%%%%%%%%%%%%%%%%%%%%%%%%%%
%%%%%%%%%%%%%%%%%%%%%%%%%%%%%%%%%%%%%%%%%%%%%%%%%%%%%%%%%%
\modHeadsection{\indexWorkTotalLengthDimension\indexAlocationLengthDimension\nameWorkTotalLength および\nameAlocationLength に関する寸法}


%%%%%%%%%%%%%%%%%%%%%%%%%%%%%%%%%%%%%%%%%%%%%%%%%%%%%%%%%%
%% subsection 23.02.01 %%%%%%%%%%%%%%%%%%%%%%%%%%%%%%%%%%%
%%%%%%%%%%%%%%%%%%%%%%%%%%%%%%%%%%%%%%%%%%%%%%%%%%%%%%%%%%
\subsection{\indexWorkTotalLengthTolerance\indexAlocationLengthTolerance\nameWorkTotalLength と\nameAlocationLength の公差の関係}
\AlocationLengthTolerance については、\WorkTotalLength の\WorkTotalLengthTolerance を\TopAlocationLength と\BottomAlocationLength とで等分配する。
%%%%%%%%%%%%%%%%%%%%%%%%%%%%%%%%%%%%%%%%%%%%%%%%%%%%%%%%%%
%% hosoku %%%%%%%%%%%%%%%%%%%%%%%%%%%%%%%%%%%%%%%%%%%%%%%%
%%%%%%%%%%%%%%%%%%%%%%%%%%%%%%%%%%%%%%%%%%%%%%%%%%%%%%%%%%
\begin{hosoku}
たとえば、\WorkTotalLength が$1000^{\phantom +0}_{-1.0}$で\TopAlocationLength が200であれば、\WorkTotalLength の寸法公差分$-0.5$を等分配し、それぞれ$-0.25$, $-0.25$とする。
つまり、\TopAlocationLength は199.75, \BottomAlocationLength は799.75とする
%% footnote %%%%%%%%%%%%%%%%%%%%%
\footnote{\expandafterindex{\yomiAlocationCenter からのずれ@\nameAlocationCenter からのずれ}\nameAlocationCenter からのずれとして考えると、\AlocationLength に依らず等分配するのが自然、と捉えることができる。}。
%%%%%%%%%%%%%%%%%%%%%%%%%%%%%%%%%
\end{hosoku}
%%%%%%%%%%%%%%%%%%%%%%%%%%%%%%%%%%%%%%%%%%%%%%%%%%%%%%%%%%
%%%%%%%%%%%%%%%%%%%%%%%%%%%%%%%%%%%%%%%%%%%%%%%%%%%%%%%%%%
%%%%%%%%%%%%%%%%%%%%%%%%%%%%%%%%%%%%%%%%%%%%%%%%%%%%%%%%%%


%%%%%%%%%%%%%%%%%%%%%%%%%%%%%%%%%%%%%%%%%%%%%%%%%%%%%%%%%%
%% subsection 23.02.02 %%%%%%%%%%%%%%%%%%%%%%%%%%%%%%%%%%%
%%%%%%%%%%%%%%%%%%%%%%%%%%%%%%%%%%%%%%%%%%%%%%%%%%%%%%%%%%
\subsection{\AlocationLength が\index{かっこすんぽう@括弧寸法}括弧寸法の場合}
片方の\AlocationLength が\index{かっこすんぽう@括弧寸法}括弧寸法の場合は、\WorkTotalLength の\index{すんぽうこうさ@寸法公差}寸法公差をそのまま\index{かっこすんぽう@括弧寸法}括弧寸法に割り当てる。
%%%%%%%%%%%%%%%%%%%%%%%%%%%%%%%%%%%%%%%%%%%%%%%%%%%%%%%%%%
%% hosoku %%%%%%%%%%%%%%%%%%%%%%%%%%%%%%%%%%%%%%%%%%%%%%%%
%%%%%%%%%%%%%%%%%%%%%%%%%%%%%%%%%%%%%%%%%%%%%%%%%%%%%%%%%%
\begin{hosoku}
たとえば、\WorkTotalLength が$1000^{\phantom +0}_{-1.0}$で\TopAlocationLength が200, \BottomAlocationLength が(800)であれば、\TopAlocationLength は200, \BottomAlocationLength は799.5とする。
\end{hosoku}
%%%%%%%%%%%%%%%%%%%%%%%%%%%%%%%%%%%%%%%%%%%%%%%%%%%%%%%%%%
%%%%%%%%%%%%%%%%%%%%%%%%%%%%%%%%%%%%%%%%%%%%%%%%%%%%%%%%%%
%%%%%%%%%%%%%%%%%%%%%%%%%%%%%%%%%%%%%%%%%%%%%%%%%%%%%%%%%%


%%%%%%%%%%%%%%%%%%%%%%%%%%%%%%%%%%%%%%%%%%%%%%%%%%%%%%%%%%
%% subsection 23.02.03 %%%%%%%%%%%%%%%%%%%%%%%%%%%%%%%%%%%
%%%%%%%%%%%%%%%%%%%%%%%%%%%%%%%%%%%%%%%%%%%%%%%%%%%%%%%%%%
\subsection{\AlocationAdjustment}
\AlocationAdjustment を行う場合は、\Spacer または\expandafterindex{\yomiTable かいてん(\yomiAlocationAdjustment)@\nameTable 回転(\nameAlocationAdjustment)}\nameTable 回転のどちらかを用いて行うものとする。

%%%%%%%%%%%%%%%%%%%%%%%%%%%%%%%%%%%%%%%%%%%%%%%%%%%%%%%%%%
%% subsubsection 23.02.03.01 %%%%%%%%%%%%%%%%%%%%%%%%%%%%%
%%%%%%%%%%%%%%%%%%%%%%%%%%%%%%%%%%%%%%%%%%%%%%%%%%%%%%%%%%
\subsubsection{\indexSpacerAlocationAdjustment\nameSpacer による\nameAlocationAdjustment}
\Spacer は原則として\Jig の\TopSideReceiverPlate に設置する。
厚さ$\delta_\mathrm s$の\indexSpacerAlocationAdjustment\nameSpacer による調整を行う場合、もともとの\TopAlocationLength$f_\mathrm T$に対し、\TopReAlocationLength$f'_\mathrm T$を以下のように調整する。
\begin{align*}
  f'_\mathrm T
  = f_\mathrm T
    +\sqrt{R_\mathrm i'^2-\frac{\delta_\mathrm s^2+(2\bar l)^2}4}\frac{\delta_\mathrm s}{\sqrt{\delta_\mathrm s^2+(2\bar l)^2}}\qquad
    \left(R_\mathrm i' = R_\mathrm c-\frac{W_x}2-\rho~,~~\bar l = l-\frac\sigma2\right).
\end{align*}
$R_\textrm c$, $W_x$, $l$, $\rho$, $\sigma$はそれぞれ\CenterCurvatureRadius, \ACOD, \JigWidth の半分, \ReceiverPlateRadius, \ReceiverPlateWidth を示す。

%%%%%%%%%%%%%%%%%%%%%%%%%%%%%%%%%%%%%%%%%%%%%%%%%%%%%%%%%%
%% subsubsection 23.01.02.2 %%%%%%%%%%%%%%%%%%%%%%%%%%%%%%
%%%%%%%%%%%%%%%%%%%%%%%%%%%%%%%%%%%%%%%%%%%%%%%%%%%%%%%%%%
\subsubsection{\indexTableAlocationAdjustment\nameTable 回転による\nameAlocationAdjustment}
\index{かたむきかく(\yomiAlocationAdjustment)@傾き角(\nameAlocationAdjustment)}角度$\theta$だけ\indexTableAlocationAdjustment\nameTable 回転をして\nameAlocationAdjustment を行う場合は、もともとの\TopAlocationLength$f_\mathrm T$に対し、\TopReAlocationLength$f'_\mathrm T$を以下のように調整する。
\begin{align*}
  f_\mathrm T'
  = f_\mathrm T+\left(\Delta+\sqrt{R_\mathrm i'-\bar l^2}\right)\sin\theta\qquad
    \left(R_\mathrm i' = R_\mathrm c-\frac{W_x}2-\rho~,~~\bar l = l-\frac\sigma2\right).
\end{align*}
$R_\textrm c$, $W_x$, $l$, $\rho$, $\sigma$, $\Delta$はそれぞれ\CenterCurvatureRadius, \ACOD, \JigWidth の半分, \ReceiverPlateRadius, \ReceiverPlateWidth, \ReceiverPlateCenter と\TableCenter との水平距離を示す。



\clearpage
%%%%%%%%%%%%%%%%%%%%%%%%%%%%%%%%%%%%%%%%%%%%%%%%%%%%%%%%%%
%% section 13.3 %%%%%%%%%%%%%%%%%%%%%%%%%%%%%%%%%%%%%%%%%%
%%%%%%%%%%%%%%%%%%%%%%%%%%%%%%%%%%%%%%%%%%%%%%%%%%%%%%%%%%
\modHeadsection{\indexODDimension\nameOuterDiameter に関する寸法}
\nameCenterCurvatureRadius を$R_\mathrm c$, \TopAlocationLength を$f_\mathrm T$, \ACOD を$W_x$とすると、\TopEndFace 部の水平方向の長さ$W_\mathrm T$は以下で与えられる。(\BottomEndFace 部も同様)
\begin{align*}
  W_\mathrm T
  = \sqrt{\left(R_\mathrm c+\frac{W_x}2\right)^2-f_\mathrm T^2}
    -\sqrt{\left(R_\mathrm c-\frac{W_x}2\right)^2-f_\mathrm T^2}\ .
\end{align*}
なお、$(\nicefrac{f_\mathrm T}{R_\mathrm c})^2$が十分小さい場合は、$W_\mathrm T$は
\begin{align*}
  W_\mathrm T \simeq W_x\left(1+\frac{f_\mathrm T^2}{2R^2}\right)
\end{align*}
とみなしてもよいものとする。



%%%%%%%%%%%%%%%%%%%%%%%%%%%%%%%%%%%%%%%%%%%%%%%%%%%%%%%%%%
%% section 10.4 %%%%%%%%%%%%%%%%%%%%%%%%%%%%%%%%%%%%%%%%%%
%%%%%%%%%%%%%%%%%%%%%%%%%%%%%%%%%%%%%%%%%%%%%%%%%%%%%%%%%%
\modHeadsection{\indexIDDimension\nameInnerDiameter に関する寸法}

%%%%%%%%%%%%%%%%%%%%%%%%%%%%%%%%%%%%%%%%%%%%%%%%%%%%%%%%%%
%% subsection 04.4.1 %%%%%%%%%%%%%%%%%%%%%%%%%%%%%%%%%%%%%
%%%%%%%%%%%%%%%%%%%%%%%%%%%%%%%%%%%%%%%%%%%%%%%%%%%%%%%%%%
\subsection{\IDTaperTableTolerance}
\IDTaperTable を参照する際は、\WorkTotalLengthTolerance は考慮しないものとする。
また、トップ端からの距離の\expandafterindex{ピッチ(\yomiIDTaperTable)@ピッチ(\nameIDTaperTable)}ピッチも、同様に\index{すんぽうこうさ@寸法公差}寸法公差は考慮しないものとする。
%%%%%%%%%%%%%%%%%%%%%%%%%%%%%%%%%%%%%%%%%%%%%%%%%%%%%%%%%%
%% hosoku %%%%%%%%%%%%%%%%%%%%%%%%%%%%%%%%%%%%%%%%%%%%%%%%
%%%%%%%%%%%%%%%%%%%%%%%%%%%%%%%%%%%%%%%%%%%%%%%%%%%%%%%%%%
\begin{hosoku}
たとえば、\WorkTotalLength が$800^{+0.5}_{\phantom -0}$, \TopAlocationLength が400, \expandafterindex{ピッチ(\yomiIDTaperTable)@ピッチ(\nameIDTaperTable)}ピッチが25である場合を考える。
このとき、トップ端は\AlocationCenter から400の位置にあり、\expandafterindex{ピッチ(\yomiIDTaperTable)@ピッチ(\nameIDTaperTable)}ピッチは25であるものとし、両端についてはそれを適宜延長して調整する。
\end{hosoku}
%%%%%%%%%%%%%%%%%%%%%%%%%%%%%%%%%%%%%%%%%%%%%%%%%%%%%%%%%%
%%%%%%%%%%%%%%%%%%%%%%%%%%%%%%%%%%%%%%%%%%%%%%%%%%%%%%%%%%
%%%%%%%%%%%%%%%%%%%%%%%%%%%%%%%%%%%%%%%%%%%%%%%%%%%%%%%%%%

%%%%%%%%%%%%%%%%%%%%%%%%%%%%%%%%%%%%%%%%%%%%%%%%%%%%%%%%%%
%% subsection 04.4.1 %%%%%%%%%%%%%%%%%%%%%%%%%%%%%%%%%%%%%
%%%%%%%%%%%%%%%%%%%%%%%%%%%%%%%%%%%%%%%%%%%%%%%%%%%%%%%%%%
\subsection{\expandafterindex{\yomiIDTaperTable にない\yomiInnerDiameter@\nameIDTaperTable にない\nameInnerDiameter}\IDTaperTable にない\InnerDiameter}
\expandafterindex{トップたんからのきょり(\yomiIDTaperTable)@トップ端からの距離(\nameIDTaperTable)}\nameIDTaperTable におけるトップ端からの距離$\lambda_i$ ($i = 0, 1, 2, \cdots$), それに対する\ACID $w_{\mathrm Ai}$に対し、トップ端から$\lambda$の位置にある\ACID$w_{\mathrm A\lambda}$は、
\begin{align*}
  w_{\mathrm A\lambda}
  = \frac{(\lambda-\lambda_i)w_{\mathrm Ai+1}+(\lambda_{i+1}-\lambda)w_{\mathrm Ai}}{\lambda_{i+1}-\lambda_i}
  \qquad
  \Big(\lambda_i \leq \lambda < \lambda_{i+1}\Big)
\end{align*}
とみなしてもよいものとする。
\BDID$w_{\mathrm B\lambda}$についても同様である。

%%%%%%%%%%%%%%%%%%%%%%%%%%%%%%%%%%%%%%%%%%%%%%%%%%%%%%%%%%
%% subsection 11.4.3 %%%%%%%%%%%%%%%%%%%%%%%%%%%%%%%%%%%%%
%%%%%%%%%%%%%%%%%%%%%%%%%%%%%%%%%%%%%%%%%%%%%%%%%%%%%%%%%%
\subsection{\HorizontalID}
\CenterCurvatureLine 上のトップ端から$\lambda$の位置における水平方向の\HorizontalID は、トップ端から$\lambda$の位置における\ACID$w_{A\lambda}$とみなしてもよいものとする。

%%%%%%%%%%%%%%%%%%%%%%%%%%%%%%%%%%%%%%%%%%%%%%%%%%%%%%%%%%
%% subsection 11.4.4 %%%%%%%%%%%%%%%%%%%%%%%%%%%%%%%%%%%%%
%%%%%%%%%%%%%%%%%%%%%%%%%%%%%%%%%%%%%%%%%%%%%%%%%%%%%%%%%%
\subsection{\PlatingThk の考慮}
後工程にて内面に\Plating を施す場合は、\PlatingThk$\mu$を考慮した上で\expandafterindex{\yomiInnerDiameter のちょうせい@\nameInnerDiameter の調整}\InnerDiameter の調整を行うものとする。
このとき、\ACID を
\begin{align*}
  w_{\mathrm A\lambda}' = w_{\mathrm A\lambda}+2\mu
\end{align*}
とみなすものとする。
\BDID$w_{\mathrm B\lambda}'$についても同様である。



\clearpage
%%%%%%%%%%%%%%%%%%%%%%%%%%%%%%%%%%%%%%%%%%%%%%%%%%%%%%%%%%
%% section 13.5 %%%%%%%%%%%%%%%%%%%%%%%%%%%%%%%%%%%%%%%%%%
%%%%%%%%%%%%%%%%%%%%%%%%%%%%%%%%%%%%%%%%%%%%%%%%%%%%%%%%%%
\modHeadsection{\indexEndFacecutMillingDimension\nameEndFacecutMilling に関する寸法}


%%%%%%%%%%%%%%%%%%%%%%%%%%%%%%%%%%%%%%%%%%%%%%%%%%%%%%%%%%
%% subsection 10.5.1 %%%%%%%%%%%%%%%%%%%%%%%%%%%%%%%%%%%%%
%%%%%%%%%%%%%%%%%%%%%%%%%%%%%%%%%%%%%%%%%%%%%%%%%%%%%%%%%%
\subsection{\EndFacecutMillingReferencePoint}
\EndFacecutMillingReferencePoint は、\EndFaceIDCenter を基準として行うものとする。


%%%%%%%%%%%%%%%%%%%%%%%%%%%%%%%%%%%%%%%%%%%%%%%%%%%%%%%%%%
%% subsection 11.5.1 %%%%%%%%%%%%%%%%%%%%%%%%%%%%%%%%%%%%%
%%%%%%%%%%%%%%%%%%%%%%%%%%%%%%%%%%%%%%%%%%%%%%%%%%%%%%%%%%
\subsection{\indexTCEndFacecutMilling\nameToolCorrection:\nameEndFacecutMilling}

%%%%%%%%%%%%%%%%%%%%%%%%%%%%%%%%%%%%%%%%%%%%%%%%%%%%%%%%%%
%% subsubsection 10.7.3.1 %%%%%%%%%%%%%%%%%%%%%%%%%%%%%%%%
%%%%%%%%%%%%%%%%%%%%%%%%%%%%%%%%%%%%%%%%%%%%%%%%%%%%%%%%%%
\subsubsection{\indexTLCEndFacecutMilling\nameTLCorrection:\nameEndFacecutMilling}
\EndFacecutMilling に使用する\indexTLFaceMill\nameFaceMill の\nameToolLength は、その刃の\expandafterindex{せんたんぶ(\nameFaceMill)@先端部(\yomiFaceMill)}先端部を\ToolLength として\expandafterindex{オフセット(\yomiTLCorrection)@オフセット(\nameTLCorrection)}オフセット量の設定を行うものとする。

なお、このとき\TLCWearValue は0とする。

%%%%%%%%%%%%%%%%%%%%%%%%%%%%%%%%%%%%%%%%%%%%%%%%%%%%%%%%%%
%% subsubsection 10.7.3.1 %%%%%%%%%%%%%%%%%%%%%%%%%%%%%%%%
%%%%%%%%%%%%%%%%%%%%%%%%%%%%%%%%%%%%%%%%%%%%%%%%%%%%%%%%%%
\subsubsection{\indexTDCEndFacecutMilling\nameTDCorrection:\nameEndFacecutMilling}
\EndFacecutMilling に使用する\indexTDFaceMill\nameFaceMill の\nameToolDiameter は、その刃の径方向の先端部を\ToolDiameter として\expandafterindex{オフセット(\yomiTDCorrection)@オフセット(\nameTDCorrection)}オフセット量の設定を行うものとする。

なお、このとき\TDCWearValue は0とする。


%%%%%%%%%%%%%%%%%%%%%%%%%%%%%%%%%%%%%%%%%%%%%%%%%%%%%%%%%%
%% subsection 11.5.1 %%%%%%%%%%%%%%%%%%%%%%%%%%%%%%%%%%%%%
%%%%%%%%%%%%%%%%%%%%%%%%%%%%%%%%%%%%%%%%%%%%%%%%%%%%%%%%%%
\subsection{\EndFacecutCornerR}
\nameEndFacecutMilling の際の\expandafterindex{コーナーR(\yomiEndFacecutMilling)@コーナーR(\nameEndFacecutMilling)}コーナーRは、\expandafterindex{\yomiIDCornerR(\yomiEndFace)@\nameIDCornerR(\nameEndFace)}\nameEndFace における\nameIDCornerR とする。
このとき、\PlatingThk の考慮は問わない。


\clearpage
%%%%%%%%%%%%%%%%%%%%%%%%%%%%%%%%%%%%%%%%%%%%%%%%%%%%%%%%%%
%% section 10.06 %%%%%%%%%%%%%%%%%%%%%%%%%%%%%%%%%%%%%%%%%
%%%%%%%%%%%%%%%%%%%%%%%%%%%%%%%%%%%%%%%%%%%%%%%%%%%%%%%%%%
\modHeadsection{\indexOutcutMillingDimension\nameOutcutMilling に関する寸法}


%%%%%%%%%%%%%%%%%%%%%%%%%%%%%%%%%%%%%%%%%%%%%%%%%%%%%%%%%%
%% subsection 10.5.1 %%%%%%%%%%%%%%%%%%%%%%%%%%%%%%%%%%%%%
%%%%%%%%%%%%%%%%%%%%%%%%%%%%%%%%%%%%%%%%%%%%%%%%%%%%%%%%%%
\subsection{\OutcutMillingReferencePoint}
\OutcutMillingReferencePoint は、\OutcutCenter を基準として行うものとする。


%%%%%%%%%%%%%%%%%%%%%%%%%%%%%%%%%%%%%%%%%%%%%%%%%%%%%%%%%%
%% subsection 10.06.1 %%%%%%%%%%%%%%%%%%%%%%%%%%%%%%%%%%%%
%%%%%%%%%%%%%%%%%%%%%%%%%%%%%%%%%%%%%%%%%%%%%%%%%%%%%%%%%%
\subsection{\indexTCOutcutMilling\nameToolCorrection:\nameOutcutMilling}


%%%%%%%%%%%%%%%%%%%%%%%%%%%%%%%%%%%%%%%%%%%%%%%%%%%%%%%%%%
%% subsubsection 10.06.1.1 %%%%%%%%%%%%%%%%%%%%%%%%%%%%%%%
%%%%%%%%%%%%%%%%%%%%%%%%%%%%%%%%%%%%%%%%%%%%%%%%%%%%%%%%%%
\subsubsection{\indexTLCOutcutMilling\nameTLCorrection:\nameOutcutMilling}
\OutcutMilling に使用する\indexTLSquareEndMill\nameSquareEndMill の\nameToolLength は、その刃の\expandafterindex{せんたんぶ(\yomiSquareEndMill)@先端部(\nameSquareEndMill)}先端部を\nameToolLength として\expandafterindex{オフセット(\yomiTLCorrection)@オフセット(\nameTLCorrection)}オフセット量の設定を行うものとする。

なお、このとき\TLCWearValue は0とする。


%%%%%%%%%%%%%%%%%%%%%%%%%%%%%%%%%%%%%%%%%%%%%%%%%%%%%%%%%%
%% subsubsection 10.06.1.2 %%%%%%%%%%%%%%%%%%%%%%%%%%%%%%%
%%%%%%%%%%%%%%%%%%%%%%%%%%%%%%%%%%%%%%%%%%%%%%%%%%%%%%%%%%
\subsubsection{\indexTDCOutcutMilling\nameTDCorrection:\nameOutcutMilling}
\OutcutMilling に使用する\indexTDSquareEndMill\nameSquareEndMill の\nameToolDiameter は、その刃の径方向の先端部を\ToolDiameter として\expandafterindex{オフセット(\yomiTDCorrection)@オフセット(\nameTDCorrection)}オフセット量の設定を行うものとする。

なお、このとき\TDCWearValue は0とする。


%%%%%%%%%%%%%%%%%%%%%%%%%%%%%%%%%%%%%%%%%%%%%%%%%%%%%%%%%%
%% subsection 10.06.2 %%%%%%%%%%%%%%%%%%%%%%%%%%%%%%%%%%%%
%%%%%%%%%%%%%%%%%%%%%%%%%%%%%%%%%%%%%%%%%%%%%%%%%%%%%%%%%%
\subsection{\OutcutLengthDimension}
\OutcutLengthDimension は、\EndFace に垂直な方向の値とする。
また\TopOutcutLength については、\KeywayWidthDimension も含むものとする。
このとき、\TopOutcutLength$h_\mathrm T$が、\KeywayPos$\kappa_p$と\KeywayWidth$\kappa_w$の和に等しい場合は、
\begin{align*}
  h_\mathrm T = \kappa_p+1[\text{mm}]
\end{align*}
とみなして加工を行うものとする。


%%%%%%%%%%%%%%%%%%%%%%%%%%%%%%%%%%%%%%%%%%%%%%%%%%%%%%%%%%
%% subsection 10.06.3 %%%%%%%%%%%%%%%%%%%%%%%%%%%%%%%%%%%%
%%%%%%%%%%%%%%%%%%%%%%%%%%%%%%%%%%%%%%%%%%%%%%%%%%%%%%%%%%
\subsection{\CurvedOutcutDimension}
\Outcut が\CurvedOutcut の場合は、\CenterCurvatureLine に対して\EndFace から\OutcutLength の半分の位置における接線に対して平行になるように傾けるものとする。
このときのトップ側およびボトム側の\expandafterindex{かたむきかく(\yomiCurvedOutcut)@傾き角(\nameCurvedOutcut)}傾き角の大きさ$\theta_t$, $\theta_b$は、それぞれ以下を満たす。
\begin{align*}
  \sin\theta_t = \frac{f_\mathrm T'-\nicefrac{h_\mathrm T}2}{R_\mathrm c}
  \quad\left(0 \leq \theta_t \leq \frac\pi2\right)~~,\qquad
  \sin\theta_b = \frac{f_\mathrm B'-\nicefrac{h_\mathrm B}2}{R_\mathrm c}
  \quad\left(0 \leq \theta_b \leq \frac\pi2\right)
\end{align*}
$R_\mathrm c$, $f'_\mathrm T$, $h_T$, $h_B$はそれぞれ\CenterCurvatureRadius, \TopReAlocationLength, \TopOutcutLength, \BottomOutcutLength を示す。



\clearpage
%%%%%%%%%%%%%%%%%%%%%%%%%%%%%%%%%%%%%%%%%%%%%%%%%%%%%%%%%%
%% section 11.6 %%%%%%%%%%%%%%%%%%%%%%%%%%%%%%%%%%%%%%%%%%
%%%%%%%%%%%%%%%%%%%%%%%%%%%%%%%%%%%%%%%%%%%%%%%%%%%%%%%%%%
\modHeadsection{\indexKeywayMillingDimension\nameKeywayMilling に関する寸法}


%%%%%%%%%%%%%%%%%%%%%%%%%%%%%%%%%%%%%%%%%%%%%%%%%%%%%%%%%%
%% subsection 10.7.1 %%%%%%%%%%%%%%%%%%%%%%%%%%%%%%%%%%%%%
%%%%%%%%%%%%%%%%%%%%%%%%%%%%%%%%%%%%%%%%%%%%%%%%%%%%%%%%%%
\subsection{\KeywayMillingReferencePoint}
\KeywayMillingReferencePoint は、以下を基準として行うものとする。
\begin{enumerate}[label=\sarrow]
\item \TopOutcut がなく、かつ\AsideKeywayDepth に公差がない場合は、\KeywayWidth 中心部における\CurvatureCenter
\item \TopOutcut があり、かつ\AsideKeywayDepth に公差がない場合は、\TopOutcutCenter
\item \AsideKeywayDepth に公差がある場合は、\KeywayWidth 中心部における\AsideKeywayDepth および\KeywayDiameter 等から算出した\KeywayCenter
\end{enumerate}


%%%%%%%%%%%%%%%%%%%%%%%%%%%%%%%%%%%%%%%%%%%%%%%%%%%%%%%%%%
%% subsection 10.06.1 %%%%%%%%%%%%%%%%%%%%%%%%%%%%%%%%%%%%
%%%%%%%%%%%%%%%%%%%%%%%%%%%%%%%%%%%%%%%%%%%%%%%%%%%%%%%%%%
\subsection{\indexTCKeywayMilling\nameToolCorrection:\nameKeywayMilling}


%%%%%%%%%%%%%%%%%%%%%%%%%%%%%%%%%%%%%%%%%%%%%%%%%%%%%%%%%%
%% subsubsection 10.06.1.1 %%%%%%%%%%%%%%%%%%%%%%%%%%%%%%%
%%%%%%%%%%%%%%%%%%%%%%%%%%%%%%%%%%%%%%%%%%%%%%%%%%%%%%%%%%
\subsubsection{\indexTLCKeywayMilling\nameTLCorrection:\nameKeywayMilling}
\KeywayMilling に使用する\indexTLSideCutter\nameSideCutter の\nameToolLength は、その切削部(刃の部分)の\expandafterindex{せんたんぶ(\yomiSideCutter)@先端部(\nameSideCutter)}先端部を\ToolLength として\expandafterindex{オフセット(\yomiTLCorrection)@オフセット(\nameTLCorrection)}オフセット量の設定を行うものとする。

なお、このとき\TLCWearValue は0とする。


%%%%%%%%%%%%%%%%%%%%%%%%%%%%%%%%%%%%%%%%%%%%%%%%%%%%%%%%%%
%% subsubsection 10.06.1.2 %%%%%%%%%%%%%%%%%%%%%%%%%%%%%%%
%%%%%%%%%%%%%%%%%%%%%%%%%%%%%%%%%%%%%%%%%%%%%%%%%%%%%%%%%%
\subsubsection{\indexTDCKeywayMilling\nameTDCorrection:\nameKeywayMilling}
\KeywayMilling に使用する\indexTDSideCutter\nameSideCutter の\nameToolDiameter は、その刃の径方向の\expandafterindex{せんたんぶ(\yomiSideCutter)@先端部(\nameSideCutter)}先端部を\ToolDiameter として\expandafterindex{オフセット(\yomiTDCorrection)@オフセット(\nameTDCorrection)}オフセット量の設定を行うものとする。

なお、このとき\TDCWearValue は0とする。


%%%%%%%%%%%%%%%%%%%%%%%%%%%%%%%%%%%%%%%%%%%%%%%%%%%%%%%%%%
%% subsection 11.6.2 %%%%%%%%%%%%%%%%%%%%%%%%%%%%%%%%%%%%%
%%%%%%%%%%%%%%%%%%%%%%%%%%%%%%%%%%%%%%%%%%%%%%%%%%%%%%%%%%
\subsection{\indexKeywayPosDimension\indexKeywayWidthDimension\KeywayPos および\KeywayWidth の寸法}
トップ端から垂直方向に、\Keyway のトップ側の端までの距離を\KeywayPos とする。
また、同様の方向に、\Keyway のトップ側の端から\Keyway のボトム側の端までの距離を\KeywayWidth とする。


%%%%%%%%%%%%%%%%%%%%%%%%%%%%%%%%%%%%%%%%%%%%%%%%%%%%%%%%%%
%% subsection 11.6.2 %%%%%%%%%%%%%%%%%%%%%%%%%%%%%%%%%%%%%
%%%%%%%%%%%%%%%%%%%%%%%%%%%%%%%%%%%%%%%%%%%%%%%%%%%%%%%%%%
\subsection{\KeywayDepthDimension}
トップ側に\Outcut がなく、かつ\AsideKeywayDepth が\index{こうさ@公差}公差のある寸法$\kappa_d'$を持つ場合、\KeywayCenter における\Keyway A側面とA側外面との距離$\kappa_d$は以下のものとみなす。
\begin{gather*}
  \kappa_d
  = \frac{2\kappa_d'-\kappa_w\sin\zeta}{1+\cos^2\zeta}\cos\zeta
    +\sqrt{R_\mathrm o^2-\left(f_\mathrm T-\kappa_p-\frac{\kappa_w}2\right)^2}
    -\sqrt{R_\mathrm o^2-\left(f_\mathrm T-\kappa_p\right)^2}\\[3pt]
  \left(
  \tan\zeta
  = \frac{\sqrt{R_\mathrm o^2-\left(f_\mathrm T-\kappa_p-\kappa_w\right)^2}
          -\sqrt{R_\mathrm o^2-\left(f_\mathrm T-\kappa_p\right)^2}}
         {\kappa_w}
    ~~, \quad
    R_\mathrm o = R_\mathrm c+\frac{W_x}2
  \right).
\end{gather*}
$R_\mathrm c$, $W_x$, $f_\mathrm T$, $\kappa_p$, $\kappa_w$はそれぞれ\CenterCurvatureRadius, \OuterDiameter, \TopAlocationLength, \KeywayPos, \KeywayWidth を示す。

なお、$(\nicefrac{f_\mathrm T}{R_\mathrm c})^2$が十分小さい場合は、
\begin{align*}
  \kappa_d \simeq \kappa_d'+\frac{\kappa_w^2}{8R_\mathrm o}
\end{align*}
とみなしてもよいものとする。



\clearpage
%%%%%%%%%%%%%%%%%%%%%%%%%%%%%%%%%%%%%%%%%%%%%%%%%%%%%%%%%%
%% section 12.7 %%%%%%%%%%%%%%%%%%%%%%%%%%%%%%%%%%%%%%%%%%
%%%%%%%%%%%%%%%%%%%%%%%%%%%%%%%%%%%%%%%%%%%%%%%%%%%%%%%%%%
\modHeadsection{\indexEndFaceCChamferMillingDimension\nameEndFaceCChamferMilling に関する寸法}


%%%%%%%%%%%%%%%%%%%%%%%%%%%%%%%%%%%%%%%%%%%%%%%%%%%%%%%%%%
%% subsection 10.8.1 %%%%%%%%%%%%%%%%%%%%%%%%%%%%%%%%%%%%%
%%%%%%%%%%%%%%%%%%%%%%%%%%%%%%%%%%%%%%%%%%%%%%%%%%%%%%%%%%
\subsection{\EndFaceCChamferMillingReferencePoint}

%%%%%%%%%%%%%%%%%%%%%%%%%%%%%%%%%%%%%%%%%%%%%%%%%%%%%%%%%%
%% subsubsection 10.8.1.1 %%%%%%%%%%%%%%%%%%%%%%%%%%%%%%%%
%%%%%%%%%%%%%%%%%%%%%%%%%%%%%%%%%%%%%%%%%%%%%%%%%%%%%%%%%%
\subsubsection{\EndFaceOutCChamferMillingReferencePoint}
\EndFaceOutCChamferMillingReferencePoint は、\Outcut のある場合は\OutcutCenter を基準とし、\Outcut のない場合は\index{うちがわちゅうしん@内側中心}内側中心(\nameEndFace)を基準として行うものとする。

%%%%%%%%%%%%%%%%%%%%%%%%%%%%%%%%%%%%%%%%%%%%%%%%%%%%%%%%%%
%% subsubsection 10.8.1.1 %%%%%%%%%%%%%%%%%%%%%%%%%%%%%%%%
%%%%%%%%%%%%%%%%%%%%%%%%%%%%%%%%%%%%%%%%%%%%%%%%%%%%%%%%%%
\subsubsection{\EndFaceInCChamferMillingReferencePoint}
\EndFaceInCChamferMillingReferencePoint は、\IncutBoring のある場合は\IncutBoringCenter を基準とし、\IncutBoring のない場合は\index{うちがわちゅうしん@内側中心}内側中心を基準として行うものとする。


%%%%%%%%%%%%%%%%%%%%%%%%%%%%%%%%%%%%%%%%%%%%%%%%%%%%%%%%%%
%% subsection 11.6.2 %%%%%%%%%%%%%%%%%%%%%%%%%%%%%%%%%%%%%
%%%%%%%%%%%%%%%%%%%%%%%%%%%%%%%%%%%%%%%%%%%%%%%%%%%%%%%%%%
\subsection{\EndFaceCChamferMillingDimension}
\EndFaceOutCChamferLength$c_\mathrm o$ならびに\EndFaceInCChamferLength$c_\mathrm i$は、\EndFace に垂直な方向の距離とみなす。
このとき、\expandafterindex{かたかく(\yomiTaperEndMill)@片角(\nameTaperEndMill)}片角が$\xi_\mathrm e$の\TaperEndMill に対して、$XY$方向の\EndFaceCChamferLength は、$c_\mathrm o\tan\xi_\mathrm e$および$c_\mathrm i\tan\xi_\mathrm e$で与えられる。


%%%%%%%%%%%%%%%%%%%%%%%%%%%%%%%%%%%%%%%%%%%%%%%%%%%%%%%%%%
%% subsection 11.6.2 %%%%%%%%%%%%%%%%%%%%%%%%%%%%%%%%%%%%%
%%%%%%%%%%%%%%%%%%%%%%%%%%%%%%%%%%%%%%%%%%%%%%%%%%%%%%%%%%
\subsection{\indexTCEndFaceCChamferMilling\nameToolCorrection:\EndFaceCChamferMilling}

%%%%%%%%%%%%%%%%%%%%%%%%%%%%%%%%%%%%%%%%%%%%%%%%%%%%%%%%%%
%% subsubsection 10.7.3.1 %%%%%%%%%%%%%%%%%%%%%%%%%%%%%%%%
%%%%%%%%%%%%%%%%%%%%%%%%%%%%%%%%%%%%%%%%%%%%%%%%%%%%%%%%%%
\subsubsection{\indexTLCEndFaceCChamferMilling\nameTLCorrection:\EndFaceCChamferMilling}
\EndFaceCChamferMilling に使用する\indexTLTaperEndMill\nameTaperEndMill の\nameToolLength は、その\expandafterindex{せんたんぶ(\yomiTaperEndMill)@先端部(\nameTaperEndMill)}先端部から軸方向に適当な距離$d_\mathrm e$を差し引いた長さを\ToolLength として\expandafterindex{オフセット(\yomiTLCorrection)@オフセット(\nameTLCorrection)}オフセット量の設定を行うものとする。

したがって、工具の先端部を測定し、先端から$d_\mathrm e$を引いた値を\TLCValue とする。
なお、このとき\TLCWearValue は0とする。

%%%%%%%%%%%%%%%%%%%%%%%%%%%%%%%%%%%%%%%%%%%%%%%%%%%%%%%%%%
%% subsubsection 10.7.3.1 %%%%%%%%%%%%%%%%%%%%%%%%%%%%%%%%
%%%%%%%%%%%%%%%%%%%%%%%%%%%%%%%%%%%%%%%%%%%%%%%%%%%%%%%%%%
\subsubsection{\indexTDCEndFaceCChamferMilling\nameTDCorrection:\EndFaceCChamferMilling}
\expandafterindex{せんたんけい(\yomiTaperEndMill)@先端径(\nameTaperEndMill)}先端径(直径)および先端の\index{かたかく(\yomiTaperEndMill)@片角(\nameTaperEndMill)}片角がそれぞれ$D_\mathrm e$, $\xi_\mathrm e$の\TaperEndMill に対し、その\TDCValue を$\nicefrac{D_\mathrm e}2$として設定を行い、さらに\TDCWearValue を$d_\mathrm e\tan\xi_\mathrm e$として設定を行うものとする。
ここで$d_\mathrm e$は、\nameTLCorrection において用いたものとする。


%%%%%%%%%%%%%%%%%%%%%%%%%%%%%%%%%%%%%%%%%%%%%%%%%%%%%%%%%%
%% subsection 10.7.3 %%%%%%%%%%%%%%%%%%%%%%%%%%%%%%%%%%%%%
%%%%%%%%%%%%%%%%%%%%%%%%%%%%%%%%%%%%%%%%%%%%%%%%%%%%%%%%%%
\subsection{\EndFaceOutCChamferMilling}

%%%%%%%%%%%%%%%%%%%%%%%%%%%%%%%%%%%%%%%%%%%%%%%%%%%%%%%%%%
%% subsubsection 10.7.3.1 %%%%%%%%%%%%%%%%%%%%%%%%%%%%%%%%
%%%%%%%%%%%%%%%%%%%%%%%%%%%%%%%%%%%%%%%%%%%%%%%%%%%%%%%%%%
\subsubsection{\EndFaceCChamferLength が小さい場合}
\EndFaceOutCChamferLength が小さい場合は、\index{てもちけんまき@手持ち研磨機}手持ち研磨機を用いた手動による加工で行ってもよいものとする。

%%%%%%%%%%%%%%%%%%%%%%%%%%%%%%%%%%%%%%%%%%%%%%%%%%%%%%%%%%
%% subsubsection 10.7.3.2 %%%%%%%%%%%%%%%%%%%%%%%%%%%%%%%%
%%%%%%%%%%%%%%%%%%%%%%%%%%%%%%%%%%%%%%%%%%%%%%%%%%%%%%%%%%
\subsubsection{\EndFaceCChamferLength が大きい場合}
\index{マシニングセンタ}マシニングセンタを用いて加工する場合、\Outcut のない場合は、\expandafterindex{かこうけい(\yomiEndFaceOutCChamfer)@加工径(\nameEndFaceOutCChamfer)}加工径の中心座標$X$をトップ側・ボトム側のそれぞれに対して以下だけ補正する。
\begin{align*}
  \text{トップ側:}&~~
  \sqrt{R_\mathrm c^2-\left(f_\mathrm T-c_\mathrm{To}\right)^2}-\sqrt{R_\mathrm c^2-f_\mathrm T^2}\ ,\\
  \text{ボトム側:}&~~
  \sqrt{R_\mathrm c^2-f_\mathrm B^2}-\sqrt{R_\mathrm c^2-\left(f_\mathrm B-c_\mathrm{Bo}\right)^2}\ .
\end{align*}
ここで$c_\mathrm{To}$, $c_\mathrm{Bo}$, $R_\mathrm c$, $f_\mathrm T$, $f_\mathrm B$はそれぞれ\TopEndFaceOutCChamferLength, \BottomEndFaceOutCChamferLength, \CenterCurvatureRadius, \TopAlocationLength, \BottomAlocationLength を示す。


\clearpage
%%%%%%%%%%%%%%%%%%%%%%%%%%%%%%%%%%%%%%%%%%%%%%%%%%%%%%%%%%
%% subsection 10.7.4 %%%%%%%%%%%%%%%%%%%%%%%%%%%%%%%%%%%%%
%%%%%%%%%%%%%%%%%%%%%%%%%%%%%%%%%%%%%%%%%%%%%%%%%%%%%%%%%%
\subsection{\EndFaceInCChamferMilling}

%%%%%%%%%%%%%%%%%%%%%%%%%%%%%%%%%%%%%%%%%%%%%%%%%%%%%%%%%%
%% subsubsection 10.7.4.1 %%%%%%%%%%%%%%%%%%%%%%%%%%%%%%%%
%%%%%%%%%%%%%%%%%%%%%%%%%%%%%%%%%%%%%%%%%%%%%%%%%%%%%%%%%%
\subsubsection{\EndFaceInCChamferLength が小さい場合}
\EndFaceInCChamferLength が小さい場合は、\index{てもちけんまき@手持ち研磨機}手持ち研磨機を用いた手動による加工で行ってもよいものとする。

%%%%%%%%%%%%%%%%%%%%%%%%%%%%%%%%%%%%%%%%%%%%%%%%%%%%%%%%%%
%% subsubsection 10.7.4.2 %%%%%%%%%%%%%%%%%%%%%%%%%%%%%%%%
%%%%%%%%%%%%%%%%%%%%%%%%%%%%%%%%%%%%%%%%%%%%%%%%%%%%%%%%%%
\subsubsection{\EndFaceInCChamferLength が大きい場合}
\index{マシニングセンタ}マシニングセンタを用いて加工する場合、\expandafterindex{かこうけい(\yomiEndFaceInCChamfer)@加工径(\nameEndFaceInCChamfer)}加工径の中心座標$X$をトップ側・ボトム側のそれぞれに対して以下だけ補正する。
\begin{align*}
  \text{トップ側:}&~~
  \sqrt{R_\mathrm c^2-\left(f_\mathrm T-c_\mathrm{Ti}\right)^2}-\sqrt{R_\mathrm c^2-f_\mathrm T^2}\ ,\\
  \text{ボトム側:}&~~
  \sqrt{R_\mathrm c^2-f_\mathrm B^2}-\sqrt{R_\mathrm c^2-\left(f_\mathrm B-c_\mathrm{Bi}\right)^2}\ .
\end{align*}
ここで$c_\mathrm{Ti}$, $c_\mathrm{Bi}$, $R_\mathrm c$, $f_\mathrm T$, $f_\mathrm B$はそれぞれ\TopEndFaceInCChamferLength, \BottomEndFaceInCChamferLength, \CenterCurvatureRadius, \TopAlocationLength, \BottomAlocationLength を示す。



\clearpage
%%%%%%%%%%%%%%%%%%%%%%%%%%%%%%%%%%%%%%%%%%%%%%%%%%%%%%%%%%
%% section 10.8 %%%%%%%%%%%%%%%%%%%%%%%%%%%%%%%%%%%%%%%%%%
%%%%%%%%%%%%%%%%%%%%%%%%%%%%%%%%%%%%%%%%%%%%%%%%%%%%%%%%%%
\modHeadsection{\indexEndFaceRChamferMillingDimension\nameEndFaceRChamferMilling に関する寸法}


%%%%%%%%%%%%%%%%%%%%%%%%%%%%%%%%%%%%%%%%%%%%%%%%%%%%%%%%%%
%% subsection 10.9.1 %%%%%%%%%%%%%%%%%%%%%%%%%%%%%%%%%%%%%
%%%%%%%%%%%%%%%%%%%%%%%%%%%%%%%%%%%%%%%%%%%%%%%%%%%%%%%%%%
\subsection{\EndFaceRChamferMillingReferencePoint}

%%%%%%%%%%%%%%%%%%%%%%%%%%%%%%%%%%%%%%%%%%%%%%%%%%%%%%%%%%
%% subsubsection 10.8.1.1 %%%%%%%%%%%%%%%%%%%%%%%%%%%%%%%%
%%%%%%%%%%%%%%%%%%%%%%%%%%%%%%%%%%%%%%%%%%%%%%%%%%%%%%%%%%
\subsubsection{\EndFaceOutRChamferMillingReferencePoint}
\EndFaceOutRChamferMillingReferencePoint は、\Outcut のある場合は\OutcutCenter を基準とし、\Outcut のない場合は\EndFaceIDCenter を基準として行うものとする。

%%%%%%%%%%%%%%%%%%%%%%%%%%%%%%%%%%%%%%%%%%%%%%%%%%%%%%%%%%
%% subsubsection 10.8.1.1 %%%%%%%%%%%%%%%%%%%%%%%%%%%%%%%%
%%%%%%%%%%%%%%%%%%%%%%%%%%%%%%%%%%%%%%%%%%%%%%%%%%%%%%%%%%
\subsubsection{\EndFaceInRChamferMilling}
\EndFaceInRChamferMillingReferencePoint は、\EndFaceIDCenter を基準として行うものとする。


%%%%%%%%%%%%%%%%%%%%%%%%%%%%%%%%%%%%%%%%%%%%%%%%%%%%%%%%%%
%% subsection 10.8.1 %%%%%%%%%%%%%%%%%%%%%%%%%%%%%%%%%%%%%
%%%%%%%%%%%%%%%%%%%%%%%%%%%%%%%%%%%%%%%%%%%%%%%%%%%%%%%%%%
\subsection{\EndFaceOutRChamferMilling}

%%%%%%%%%%%%%%%%%%%%%%%%%%%%%%%%%%%%%%%%%%%%%%%%%%%%%%%%%%
%% subsubsection 10.8.1.1 %%%%%%%%%%%%%%%%%%%%%%%%%%%%%%%%
%%%%%%%%%%%%%%%%%%%%%%%%%%%%%%%%%%%%%%%%%%%%%%%%%%%%%%%%%%
\subsubsection{\EndFaceOutRChamferRadius が小さい場合}
\EndFaceOutRChamfer については、基本的には\index{マシニングセンタ}マシニングセンタでは行わず、\index{てもちけんまき@手持ち研磨機}手持ち研磨機・やすり等を用いて手動で行うものとする。

%%%%%%%%%%%%%%%%%%%%%%%%%%%%%%%%%%%%%%%%%%%%%%%%%%%%%%%%%%
%% subsubsection 10.8.1.2 %%%%%%%%%%%%%%%%%%%%%%%%%%%%%%%%
%%%%%%%%%%%%%%%%%%%%%%%%%%%%%%%%%%%%%%%%%%%%%%%%%%%%%%%%%%
\subsubsection{\EndFaceOutRChamferRadius が大きい場合}
\EndFaceOutRChamferRadius が大きい場合、\index{さぎょうしゃのふたん@作業者の負担}作業者の負担や\index{さぎょうじかん@作業時間}作業時間を考慮して、その一部を\TaperEndMill を用いて加工してもよいものとする。

このとき、\TopEndFaceOutRChamferRadius および\BottomEndFaceOutRChamferRadius$r_\mathrm{To}$, $r_\mathrm{Bo}$, および片角$\xi_\mathrm e$の\TaperEndMill に対して、
\begin{align*}
  c_\mathrm{To} &= r_\mathrm{To}\left(1+\cot\xi_\mathrm e-\csc\xi_\mathrm e\right)\ ,\\
  c_\mathrm{Bo} &= r_\mathrm{Bo}\left(1+\cot\xi_\mathrm e-\csc\xi_\mathrm e\right)
\end{align*}
の\EndFaceOutCChamferLength とみなして、\EndFaceInCChamferMilling を行うものとする。


%%%%%%%%%%%%%%%%%%%%%%%%%%%%%%%%%%%%%%%%%%%%%%%%%%%%%%%%%%
%% subsection 10.8.2 %%%%%%%%%%%%%%%%%%%%%%%%%%%%%%%%%%%%%
%%%%%%%%%%%%%%%%%%%%%%%%%%%%%%%%%%%%%%%%%%%%%%%%%%%%%%%%%%
\subsection{\EndFaceInRChamferMilling}

%%%%%%%%%%%%%%%%%%%%%%%%%%%%%%%%%%%%%%%%%%%%%%%%%%%%%%%%%%
%% subsubsection 10.8.2.1 %%%%%%%%%%%%%%%%%%%%%%%%%%%%%%%%
%%%%%%%%%%%%%%%%%%%%%%%%%%%%%%%%%%%%%%%%%%%%%%%%%%%%%%%%%%
\subsubsection{\EndFaceInRChamferRadius が小さい場合}
\TopEndFaceInRChamferMilling および\BottomEndFaceInRChamferMilling については、基本的には\index{マシニングセンタ}マシニングセンタでは行わず、\index{てもちけんまき@手持ち研磨機}手持ち研磨機を用いて手動で行うものとする。

%%%%%%%%%%%%%%%%%%%%%%%%%%%%%%%%%%%%%%%%%%%%%%%%%%%%%%%%%%
%% subsubsection 10.8.2.2 %%%%%%%%%%%%%%%%%%%%%%%%%%%%%%%%
%%%%%%%%%%%%%%%%%%%%%%%%%%%%%%%%%%%%%%%%%%%%%%%%%%%%%%%%%%
\subsubsection{\EndFaceInRChamferRadius が大きい場合}
\EndFaceInRChamferRadius が大きい場合、\index{さぎょうしゃのふたん@作業者の負担}作業者の負担や\index{さぎょうじかん@作業時間}作業時間を考慮して、その一部を\TaperEndMill を用いて加工してもよいものとする。

このとき、\TopEndFaceInRChamferRadius および\BottomFaceInRChamferRadius$r_\mathrm{Ti}$, $r_\mathrm{Bi}$, および片角$\xi_\mathrm e$の\TaperEndMill に対して、
\begin{align*}
  c_\mathrm{Ti} &= r_\mathrm{Ti}\left(1+\cot\xi_\mathrm e-\csc\xi_\mathrm e\right)\ ,\\
  c_\mathrm{Bi} &= r_\mathrm{Bi}\left(1+\cot\xi_\mathrm e-\csc\xi_\mathrm e\right)
\end{align*}
の\EndFaceCChamferLength とみなして、\EndFaceInCChamferMilling を行うものとする。



\clearpage
%%%%%%%%%%%%%%%%%%%%%%%%%%%%%%%%%%%%%%%%%%%%%%%%%%%%%%%%%%
%% section 10.9 %%%%%%%%%%%%%%%%%%%%%%%%%%%%%%%%%%%%%%%%%%
%%%%%%%%%%%%%%%%%%%%%%%%%%%%%%%%%%%%%%%%%%%%%%%%%%%%%%%%%%
\modHeadsection{\indexEndFaceBoringMillingDimension\nameEndFaceBoringMilling に関する寸法}


%%%%%%%%%%%%%%%%%%%%%%%%%%%%%%%%%%%%%%%%%%%%%%%%%%%%%%%%%%
%% subsection 10.8.2 %%%%%%%%%%%%%%%%%%%%%%%%%%%%%%%%%%%%%
%%%%%%%%%%%%%%%%%%%%%%%%%%%%%%%%%%%%%%%%%%%%%%%%%%%%%%%%%%
\subsection{\EndFaceBoringMillingReferencePoint}
\EndFaceBoringMilling の基準点は\index{そとがわちゅうしん@外側中心}外側中心(\nameTopEndFace)を基準として行うものとする。


%%%%%%%%%%%%%%%%%%%%%%%%%%%%%%%%%%%%%%%%%%%%%%%%%%%%%%%%%%
%% subsection 10.8.2 %%%%%%%%%%%%%%%%%%%%%%%%%%%%%%%%%%%%%
%%%%%%%%%%%%%%%%%%%%%%%%%%%%%%%%%%%%%%%%%%%%%%%%%%%%%%%%%%
\subsection{\EndFaceBoringCornerR の中心}
\EndFaceBoring のA側外面からの距離, \EndFaceBoringWidth, \EndFaceBoringDepth, \EndFaceBoringCornerR, \EndFaceBoringLength をそれぞれ$p_\beta$, $w_\beta$, $d_\beta$, $R_\beta$, $l_\beta$とすると、A面側およびC面側の\EndFaceBoringCornerR の中心の位置は、\TopEndFace の外側中心を原点として、それぞれ以下で与えられる。
\begin{align*}
  \text{A面側:}
  \left[
  \begin{array}{c}
  \displaystyle
  \frac{\mathcal W_x}2-p_\beta-\sqrt{d_\beta(2R_\beta-d_\beta)}\\[8pt]
  \displaystyle
  -\frac{\mathcal W_y}2+d_\beta-R_\beta
  \end{array}
  \right]~,\quad
  \text{C面側:}
  \left[
  \begin{array}{c}
  \displaystyle
  \frac{\mathcal W_x}2-p_\beta-w_\beta+\sqrt{d_\beta(2R_\beta-d_\beta)}\\[8pt]
  \displaystyle
  -\frac{\mathcal W_y}2+d_\beta-R_\beta
  \end{array}
  \right]
\end{align*}
ここで$\mathcal W_x$, $\mathcal W_y$はそれぞれ\TopEndACOD, \TopEndBDOD を示す。


%%%%%%%%%%%%%%%%%%%%%%%%%%%%%%%%%%%%%%%%%%%%%%%%%%%%%%%%%%
%% subsection 10.06.1 %%%%%%%%%%%%%%%%%%%%%%%%%%%%%%%%%%%%
%%%%%%%%%%%%%%%%%%%%%%%%%%%%%%%%%%%%%%%%%%%%%%%%%%%%%%%%%%
\subsection{\indexTCOutcutMilling\nameToolCorrection:\EndFaceBoringMilling}
\EndFaceBoringMilling に使用する\indexTCSquareEndMill\nameSquareEndMill の\ToolCorrection については、\OutcutMilling のそれと同様とする。

%\clearpage
%%%%%%%%%%%%%%%%%%%%%%%%%%%%%%%%%%%%%%%%%%%%%%%%%%%%%%%%%%
%% section 23.11 %%%%%%%%%%%%%%%%%%%%%%%%%%%%%%%%%%%%%%%%%
%%%%%%%%%%%%%%%%%%%%%%%%%%%%%%%%%%%%%%%%%%%%%%%%%%%%%%%%%%
\modHeadsection{\indexIncutBoringMillingDimension\nameIncutBoringMilling に関する寸法}


%%%%%%%%%%%%%%%%%%%%%%%%%%%%%%%%%%%%%%%%%%%%%%%%%%%%%%%%%%
%% subsection 10.8.2 %%%%%%%%%%%%%%%%%%%%%%%%%%%%%%%%%%%%%
%%%%%%%%%%%%%%%%%%%%%%%%%%%%%%%%%%%%%%%%%%%%%%%%%%%%%%%%%%
\subsection{\IncutBoringMillingReferencePoint}
\IncutBoringMillingReferencePoint は\TopEndFace における\index{うちがわちゅうしん@内側中心}内側中心(\nameTopEndFace)を基準として行うものとする。


%%%%%%%%%%%%%%%%%%%%%%%%%%%%%%%%%%%%%%%%%%%%%%%%%%%%%%%%%%
%% subsection 10.06.1 %%%%%%%%%%%%%%%%%%%%%%%%%%%%%%%%%%%%
%%%%%%%%%%%%%%%%%%%%%%%%%%%%%%%%%%%%%%%%%%%%%%%%%%%%%%%%%%
\subsection{\indexTCOutcutMilling\nameToolCorrection:\IncutBoringMilling}
\IncutBoringMilling に使用する\indexTCSquareEndMill\nameSquareEndMill の\ToolCorrection については、\OutcutMilling のそれと同様とする。


\clearpage
%%%%%%%%%%%%%%%%%%%%%%%%%%%%%%%%%%%%%%%%%%%%%%%%%%%%%%%%%%
%% section 23.12 %%%%%%%%%%%%%%%%%%%%%%%%%%%%%%%%%%%%%%%%%
%%%%%%%%%%%%%%%%%%%%%%%%%%%%%%%%%%%%%%%%%%%%%%%%%%%%%%%%%%
\modHeadsection{\indexDimpleMillingDimension\nameDimpleMilling に関する寸法}


%%%%%%%%%%%%%%%%%%%%%%%%%%%%%%%%%%%%%%%%%%%%%%%%%%%%%%%%%%
%% subsection 23.12.01 %%%%%%%%%%%%%%%%%%%%%%%%%%%%%%%%%%%
%%%%%%%%%%%%%%%%%%%%%%%%%%%%%%%%%%%%%%%%%%%%%%%%%%%%%%%%%%
\subsection{\indexTCDimpleMilling\nameToolCorrection:\nameDimpleMilling}

%%%%%%%%%%%%%%%%%%%%%%%%%%%%%%%%%%%%%%%%%%%%%%%%%%%%%%%%%%
%% subsubsection 23.12.01.01 %%%%%%%%%%%%%%%%%%%%%%%%%%%%%
%%%%%%%%%%%%%%%%%%%%%%%%%%%%%%%%%%%%%%%%%%%%%%%%%%%%%%%%%%
\subsubsection{\indexTLCDimpleMilling\nameTLCorrection:\nameDimpleMilling}
\DimpleMilling に用いる\indexTLTSlotCutter\nameTSlotCutter の\nameToolLength については、切削部(刃の部分)の中央を\ToolLength として\expandafterindex{オフセット(\yomiTLCorrection)@オフセット(\nameTLCorrection)}オフセット量の設定を行うものとする。

したがって、工具の切削部の先端、および切削部の厚みを測定し、切削部の先端から厚みの半分の値を引いた値を\nameTLCorrection とする。
なお、このとき\TLCWearValue は0とする。

%%%%%%%%%%%%%%%%%%%%%%%%%%%%%%%%%%%%%%%%%%%%%%%%%%%%%%%%%%
%% subsubsection 23.12.01.02 %%%%%%%%%%%%%%%%%%%%%%%%%%%%%
%%%%%%%%%%%%%%%%%%%%%%%%%%%%%%%%%%%%%%%%%%%%%%%%%%%%%%%%%%
\subsubsection{\indexTDCDimpleMilling\nameTDCorrection:\nameDimpleMilling}
\DimpleMilling に用いる\indexTDTSlotCutter\nameTSlotCutter の\nameToolDiameter は、その刃の径方向の先端部を\ToolDiameter として\expandafterindex{オフセット(\yomiTDCorrection)@オフセット(\nameTDCorrection)}オフセット量の設定を行うものとする。

なお、このとき\TDCWearValue は0とする。


%%%%%%%%%%%%%%%%%%%%%%%%%%%%%%%%%%%%%%%%%%%%%%%%%%%%%%%%%%
%% subsection 10.9.2 %%%%%%%%%%%%%%%%%%%%%%%%%%%%%%%%%%%%%
%%%%%%%%%%%%%%%%%%%%%%%%%%%%%%%%%%%%%%%%%%%%%%%%%%%%%%%%%%
\subsection{\DimpleMilling の基準点}
\DimpleMilling の際は、\TopEndFace における内側の中心座標の\index{じっそくち(トップたんうちがわちゅうしん)@実測値(トップ端内側中心)}実測値を基準にして行うものとする。


%%%%%%%%%%%%%%%%%%%%%%%%%%%%%%%%%%%%%%%%%%%%%%%%%%%%%%%%%%
%% subsection 10.9.3 %%%%%%%%%%%%%%%%%%%%%%%%%%%%%%%%%%%%%
%%%%%%%%%%%%%%%%%%%%%%%%%%%%%%%%%%%%%%%%%%%%%%%%%%%%%%%%%%
\subsection{\indexDimpleAngle\nameDimpleMilling の傾き角}
\Dimple の測定および加工は、\index{アンダーカット}アンダーカットが生じないように適切に\index{ワーク}ワークを$B$軸方向に傾けるものとする。
このとき\indexDimpleAngle 傾ける角度$\phi$は、\expandafterindex{\yomiDimple1れつめ@\nameDimple1列目}トップ端から1列目の\nameDimple までの距離$q$に対して、
\begin{align*}
  \tan\phi
  &= \frac{\displaystyle
           \sqrt{\left(R_\mathrm c+\frac{w'_{\mathrm Aq}}2\right)^2-(f_\mathrm T-q)^2}
           -\sqrt{\left(R_\mathrm c+\frac{w'_{\mathrm A0}}2\right)^2-f_\mathrm T^2}}q
\end{align*}
とする。
ここで$w'_{\mathrm A0}$, $w'_{\mathrm Aq}$は(\PlatingThk$\mu$を考慮した)\TopEndACID およびトップ端から距離$q$におけるAC方向の\InnerDiameter を示し、$R_\mathrm c$, $f_\mathrm T$はそれぞれ\CenterCurvatureRadius, \TopAlocationLength を示す。

ただし、$\phi < 0$となる場合は、$\phi = 0$とみなすものとする。
