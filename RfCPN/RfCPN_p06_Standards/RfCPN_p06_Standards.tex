%!TEX root = ./RfCPN.tex


\addtocontents{toc}{\protect\cleardoublepage}
\addtocontents{lot}{\protect\tcbline*}
%%%%%%%%%%%%%%%%%%%%%%%%%%%%%%%%%%%%%%%%%%%%%%%%%%%%%%%%%
%% Part Standards %%%%%%%%%%%%%%%%%%%%%%%%%%%%%%%%%%%%%%%
%%%%%%%%%%%%%%%%%%%%%%%%%%%%%%%%%%%%%%%%%%%%%%%%%%%%%%%%%
\addtocontents{toc}{\protect\begin{tocBox}{\tmppartnum}}%
\tPart[lot,loColumn]{機械の稼働に向けた諸標準の策定}{%
\paragraph*{\tpartgoal}
機械の稼働に関して一定の基準・規則を設けることで、一貫性・効率性・保守性を向上する
\tcbline*
\paragraph*{\tpartmethod}
\index{よこがたマシニングセンタ@横型マシニングセンタ}横型マシニングセンタについての(ソフトウェアの観点による)\textbf{諸標準の策定}を試みる。
\tcbline*
\paragraph*{\tpartbackground}
\index{NCプログラム}NCプログラムの記述には一般に\index{Gcode@G-code}G-codeが用いられる。
一方、\index{Gcode@G-code}G-codeの記述に関してはその規格が多岐にわたり、統一された標準が存在しない。
そのため、記述に際して社内標準を参照する必要がある。
しかし、\DMC 設置時点において、\textbf{ソフトウェア視点による社内標準が存在しない}
%% footnote %%%%%%%%%%%%%%%%%%%%%
\footnote{つまり、\index{マシニングセンタ}マシニングセンタについては長年にわたり管理業務が事実上放棄され続けている。
またその皺寄せの大部分が、\index{さぎょうしゃ@作業者}作業者(一般職)に押し付けられている。}。
%%%%%%%%%%%%%%%%%%%%%%%%%%%%%%%%%

 したがって、\index{NCプログラム}NCプログラムの記述に関する標準の策定が急務である。
これにより、\index{NCプログラム}NCプログラムの記述における一貫性と効率性が向上し、業務の効率化(外注を含む)とコスト削減が期待される。
また標準を設けることにより、\index{NCプログラム}NCプログラムの記述に関する混乱を解消し、より高品質なソフトウェア開発を実現するための重要なステップとなる。
}{%
\paragraph*{\tpartconclusion}
\DMC 関連のソフトウェア開発における、(ソフトウェアの観点による)寸法・コードの記述法・工具・保守等の標準を策定した。
\tcbline*
\paragraph*{\tpartnextstep}
具体的な\index{NCプログラム}NCプログラムを記述するために、これらの諸標準に加えて、マシニングセンタ内の幾何的情報を解析的に導出し、体系化を行う。
}

%%%%%%%%%%%%%%%%%%%%%%%%%%%%%%%%%%%%%%%%%%%%%%%%%%%%%%%%%%
%% chapters %%%%%%%%%%%%%%%%%%%%%%%%%%%%%%%%%%%%%%%%%%%%%%
%%%%%%%%%%%%%%%%%%%%%%%%%%%%%%%%%%%%%%%%%%%%%%%%%%%%%%%%%%
%!TEX root = ../RPA_for_Creating_Program_Note.tex


\modHeadchapter[lot]{\DMC の設置環境}
マシニングセンタの精度を保つためには、\index{せっちかんきょう(きかいほんたい)@設置環境(機械本体)}設置環境を整える必要がある。
ここではその目安を与える。



%%%%%%%%%%%%%%%%%%%%%%%%%%%%%%%%%%%%%%%%%%%%%%%%%%%%%%%%%%
%% section 09.1 %%%%%%%%%%%%%%%%%%%%%%%%%%%%%%%%%%%%%%%%%%
%%%%%%%%%%%%%%%%%%%%%%%%%%%%%%%%%%%%%%%%%%%%%%%%%%%%%%%%%%
\modHeadsection{設置箇所における基本事項}
\index{マシニングセンタ}マシニングセンタを設置する場所としては以下のような場所を選ぶものとする。
\begin{enumerate}
\item 日光が直接当たらない
\item 温風や冷風が直接当たらない
\item 基礎の\index{ちたいりょく@地耐力}地耐力が10t/m$^2$以上である
\end{enumerate}



%%%%%%%%%%%%%%%%%%%%%%%%%%%%%%%%%%%%%%%%%%%%%%%%%%%%%%%%%%
%% section 09.2 %%%%%%%%%%%%%%%%%%%%%%%%%%%%%%%%%%%%%%%%%%
%%%%%%%%%%%%%%%%%%%%%%%%%%%%%%%%%%%%%%%%%%%%%%%%%%%%%%%%%%
\modHeadsection{設置の条件}
\DMC についてメーカーが指定する主な\index{せっちじょうけん(きかいほんたい)@設置条件(機械本体)}設置条件は以下のとおりである。
詳細については\expandafterindex{オペレーションマニュアル(\yomiDMC)@オペレーションマニュアル(\nameDMC)}オペレーションマニュアルを参照されたし。\\

\begin{multicollongtblr}{\DMC 据付要件}{l X[l]}
項目 & 内容\\
\index{しつおん@室温}室温 & 20$\pm$1$^\circ$C\\
室温の温度変化 & 0.5$^\circ$C/day以内\\
温度勾配 & 0.2$^\circ$C/h以内\\
床面より5m間での上下間の室温差 & 0.7$^\circ$C以内\\
基礎床面温度と室温の差 & 0.5$^\circ$C以内\\
\index{しつど@湿度}湿度 & 60$\pm$5\%\\
\index{せっさくゆ@切削油}切削油の温度 & $\pm$2$^\circ$C以内\\
周囲の振動による機械への影響 & 0.1$\mu$m以内\\
機械と天井との間隔 & 1.5m以上\\
床面の\index{へいめんど(とこめん)@平面度(床面)}平面度 & $\pm$5mm(推奨値)\\
\end{multicollongtblr}


%!TEX root = ../RPA_for_Creating_Program_Note.tex

ここでは\index{G-code}G-codeを記述する際や、\index{ずめん@図面}図面・3Dモデルの描画をする際に必要となる、\index{すんぽう@寸法}寸法や\index{こうさ@公差}公差等の取り扱いについて触れる。

ただし、特記事項等がある場合は、それを優先するものとする。
以下では主にそうした特別な記述のない、いわゆる一般的な場合について記載する。

なお、以降で述べる水平方向とは、端面のAC方向のことを指す。


%%%%%%%%%%%%%%%%%%%%%%%%%%%%%%%%%%%%%%%%%%%%%%%%%%%%%%%%%%
%% section 13.1 %%%%%%%%%%%%%%%%%%%%%%%%%%%%%%%%%%%%%%%%%%
%%%%%%%%%%%%%%%%%%%%%%%%%%%%%%%%%%%%%%%%%%%%%%%%%%%%%%%%%%
\modHeadsection{基本事項}

%%%%%%%%%%%%%%%%%%%%%%%%%%%%%%%%%%%%%%%%%%%%%%%%%%%%%%%%%%
%% subsection 13.1.1 %%%%%%%%%%%%%%%%%%%%%%%%%%%%%%%%%%%%%
%%%%%%%%%%%%%%%%%%%%%%%%%%%%%%%%%%%%%%%%%%%%%%%%%%%%%%%%%%
\subsection{寸法公差の取扱い}
全般的に、\index{すんぽうこうさ@寸法公差}寸法公差がある場合、\index{+こうさ@$+$公差}$+$公差と\index{-こうさ@$-$公差}$-$公差の中央(平均)を見るものとする。
ただし、\index{ないめんテーパひょう@内面テーパ表}内面テーパ表を見る際は、この限りではない。

たとえば、$100^{+0.5}_{\phantom -0}$であれば、100.25とみなす。

%%%%%%%%%%%%%%%%%%%%%%%%%%%%%%%%%%%%%%%%%%%%%%%%%%%%%%%%%%
%% subsection 13.1.2 %%%%%%%%%%%%%%%%%%%%%%%%%%%%%%%%%%%%%
%%%%%%%%%%%%%%%%%%%%%%%%%%%%%%%%%%%%%%%%%%%%%%%%%%%%%%%%%%
\subsection{寸法の優先度}
公差のある寸法と公差のない寸法(\index{かっこすんぽう@括弧寸法}括弧寸法含む)とが共存して記載されている場合、公差のある寸法を優先する。

たとえば、2つの線の寸法がそれぞれ$12^{+0.1}_{\phantom -0}$, $4.05$と記述されていて、かつその和に相当する部分の寸法が16と記述されている場合は、16.10とみなす。


%%%%%%%%%%%%%%%%%%%%%%%%%%%%%%%%%%%%%%%%%%%%%%%%%%%%%%%%%%
%% section 13.2 %%%%%%%%%%%%%%%%%%%%%%%%%%%%%%%%%%%%%%%%%%
%%%%%%%%%%%%%%%%%%%%%%%%%%%%%%%%%%%%%%%%%%%%%%%%%%%%%%%%%%
\modHeadsection{全長・振分長}

%%%%%%%%%%%%%%%%%%%%%%%%%%%%%%%%%%%%%%%%%%%%%%%%%%%%%%%%%%
%% subsection 13.2.1 %%%%%%%%%%%%%%%%%%%%%%%%%%%%%%%%%%%%%
%%%%%%%%%%%%%%%%%%%%%%%%%%%%%%%%%%%%%%%%%%%%%%%%%%%%%%%%%%
\subsection{全長と振分長の公差の関係}
\index{ふりわけちょう@振分長}振分長の公差については、\index{ぜんちょう@全長}全長の公差を\index{トップふりわけちょう@トップ振分長}トップ振分長と\index{ボトムふりわけちょう@ボトム振分長}ボトム振分長とで等分配する。

たとえば、全長が$1000^{\phantom +0}_{-1.0}$でトップ振分長が200であれば、全長の公差分$-0.5$を等分配し、それぞれ$-0.25$, $-0.25$とする。
つまり、トップ振分長は199.75, ボトム振分長は799.75とする
%% footnote %%%%%%%%%%%%%%%%%%%%%
\footnote{\index{ふりわけちゅうしん@振分中心}振分中心からのずれとして考えると、振分長に依らず等分配するのが自然、と捉えることができる。}。
%%%%%%%%%%%%%%%%%%%%%%%%%%%%%%%%%

%%%%%%%%%%%%%%%%%%%%%%%%%%%%%%%%%%%%%%%%%%%%%%%%%%%%%%%%%%
%% subsection 13.2.2 %%%%%%%%%%%%%%%%%%%%%%%%%%%%%%%%%%%%%
%%%%%%%%%%%%%%%%%%%%%%%%%%%%%%%%%%%%%%%%%%%%%%%%%%%%%%%%%%
\subsection{振分長が括弧寸法の場合}
片方の振分長が括弧寸法の場合は、全長の公差をそのまま括弧寸法に割り当てる。

たとえば、全長が$1000^{\phantom +0}_{-1.0}$でトップ振分長が200, ボトム振分長が(800)であれば、トップ振分長は200, ボトム振分長は799.5とする。



\clearpage
%%%%%%%%%%%%%%%%%%%%%%%%%%%%%%%%%%%%%%%%%%%%%%%%%%%%%%%%%%
%% section 13.3 %%%%%%%%%%%%%%%%%%%%%%%%%%%%%%%%%%%%%%%%%%
%%%%%%%%%%%%%%%%%%%%%%%%%%%%%%%%%%%%%%%%%%%%%%%%%%%%%%%%%%
\modHeadsection{外径}
\index{ちゅうしんわんきょく@中心湾曲}中心湾曲を$R_\mathrm c$, トップ振分長を$f_\mathrm T$, 外径を$W_x$とすると、トップ端面部の水平方向の長さ$W_\mathrm T$は以下で与えられる。(ボトム端面部も同様)
\begin{align*}
  W_\mathrm T
  = \sqrt{\left(R+\frac{W_x}2\right)^{\!2}-f_\mathrm T^2}
    -\sqrt{\left(R-\frac{W_x}2\right)^{\!2}-f_\mathrm T^2}\ .
\end{align*}
ただし、$(\nicefrac{f_\mathrm T}{R_\mathrm c})^2$が公差に対して十分小さい場合は、端面部の水平方向の長さ$W_\mathrm T$は\index{がいけい@外径}外径$W_x$とみなしてもよいものとする。


%%%%%%%%%%%%%%%%%%%%%%%%%%%%%%%%%%%%%%%%%%%%%%%%%%%%%%%%%%
%% section 04.4 %%%%%%%%%%%%%%%%%%%%%%%%%%%%%%%%%%%%%%%%%%
%%%%%%%%%%%%%%%%%%%%%%%%%%%%%%%%%%%%%%%%%%%%%%%%%%%%%%%%%%
\modHeadsection{内径}

%%%%%%%%%%%%%%%%%%%%%%%%%%%%%%%%%%%%%%%%%%%%%%%%%%%%%%%%%%
%% subsection 04.4.1 %%%%%%%%%%%%%%%%%%%%%%%%%%%%%%%%%%%%%
%%%%%%%%%%%%%%%%%%%%%%%%%%%%%%%%%%%%%%%%%%%%%%%%%%%%%%%%%%
\subsection{内面テーパ表の公差}
\index{ないめんテーパひょう@内面テーパ表}内面テーパ表を参照する際は、\index{ぜんちょう@全長}全長の\index{こうさ@公差}公差は考慮しないものとする。
また、トップ端からの距離のピッチも、同様に公差は考慮しないものとする。

たとえば、全長が$800^{+0.5}_{\phantom -0}$, トップ振分長が400, ピッチが25である場合を考える。
このとき、トップ端は振分中心から400の位置にあり、ピッチは25であるものとし、両端についてはそれを適宜延長して調整する。

%%%%%%%%%%%%%%%%%%%%%%%%%%%%%%%%%%%%%%%%%%%%%%%%%%%%%%%%%%
%% subsection 04.4.1 %%%%%%%%%%%%%%%%%%%%%%%%%%%%%%%%%%%%%
%%%%%%%%%%%%%%%%%%%%%%%%%%%%%%%%%%%%%%%%%%%%%%%%%%%%%%%%%%
\subsection{内面テーパ表にない内径}
内面テーパ表におけるトップ端からの距離$\lambda_i$ ($i = 0, 1, 2, ...$), それに対するAC側内径$w_{\mathrm Ai}$に対し、トップ端から$\lambda$の位置にある内径$w_{\mathrm A\lambda}$は、
\begin{align*}
  w_{\mathrm A\lambda}
  = \frac{(\lambda-\lambda_i)w_{\mathrm Ai+1}+(\lambda_{i+1}-\lambda)w_{\mathrm Ai}}{\lambda_{i+1}-\lambda_i}
  \qquad
  \Big(\lambda_i \leqq \lambda < \lambda_{i+1}\Big)
\end{align*}
とみなしてもよいものとする。
BD側内径$w_{\mathrm B\lambda}$についても同様である。

%%%%%%%%%%%%%%%%%%%%%%%%%%%%%%%%%%%%%%%%%%%%%%%%%%%%%%%%%%
%% subsection 04.4.2 %%%%%%%%%%%%%%%%%%%%%%%%%%%%%%%%%%%%%
%%%%%%%%%%%%%%%%%%%%%%%%%%%%%%%%%%%%%%%%%%%%%%%%%%%%%%%%%%
\subsection{水平方向の内径}
トップ端から$\lambda$の位置における内径を$w_\lambda$は、中心湾曲線上のトップ端から$\lambda$の位置における水平方向の内径とみなしてよいものとする。



\clearpage
%%%%%%%%%%%%%%%%%%%%%%%%%%%%%%%%%%%%%%%%%%%%%%%%%%%%%%%%%%
%% section 13.5 %%%%%%%%%%%%%%%%%%%%%%%%%%%%%%%%%%%%%%%%%%
%%%%%%%%%%%%%%%%%%%%%%%%%%%%%%%%%%%%%%%%%%%%%%%%%%%%%%%%%%
\modHeadsection{外削\TBW}
(to be written...)


%%%%%%%%%%%%%%%%%%%%%%%%%%%%%%%%%%%%%%%%%%%%%%%%%%%%%%%%%%
%% section 13.6 %%%%%%%%%%%%%%%%%%%%%%%%%%%%%%%%%%%%%%%%%%
%%%%%%%%%%%%%%%%%%%%%%%%%%%%%%%%%%%%%%%%%%%%%%%%%%%%%%%%%%
\modHeadsection{溝}
\index{Aがわみぞふかさ@A側溝深さ}A側溝深さが\index{こうさ@公差}公差のある寸法で、かつトップ側に外削のない場合、\pageautoref{subsec:keywayDepthDif}における$\kappa_d'$を図面上の値とする。
したがって\pageeqref{eq:keydepthDif1}より、溝幅中央における溝A側面とA側外面との距離$\kappa_d$を
\begin{align*}
  \kappa_d
  &= \frac{2\kappa_d'-\kappa_w\sin\zeta}{1+\cos^2\zeta}\cos\zeta
     +\sqrt{R_\mathrm o^2-\left(f_\mathrm T-\kappa_p-\frac{\kappa_w}2\right)^{\!2}}
     -\sqrt{R_\mathrm o^2-\left(f_\mathrm T-\kappa_p\right)^2}
\end{align*}
として扱う。
表記等の詳細については\pageautoref{subsec:keywayDepthDif}を参照。



%%%%%%%%%%%%%%%%%%%%%%%%%%%%%%%%%%%%%%%%%%%%%%%%%%%%%%%%%%
%% section 13.5 %%%%%%%%%%%%%%%%%%%%%%%%%%%%%%%%%%%%%%%%%%
%%%%%%%%%%%%%%%%%%%%%%%%%%%%%%%%%%%%%%%%%%%%%%%%%%%%%%%%%%
\modHeadsection{面取\TBW}
(to be written about $d_\mathrm e$ and $D_\mathrm r$)

%%%%%%%%%%%%%%%%%%%%%%%%%%%%%%%%%%%%%%%%%%%%%%%%%%%%%%%%%%
%% subsection 04.4.2 %%%%%%%%%%%%%%%%%%%%%%%%%%%%%%%%%%%%%
%%%%%%%%%%%%%%%%%%%%%%%%%%%%%%%%%%%%%%%%%%%%%%%%%%%%%%%%%%
\subsection{外側C面取}




%!TEX root = ../RfCPN.tex


\modHeadchapter{\index{NCプログラム}NCプログラムの番号付け}
ここでは\DMC で使用する\index{プログラムばんごう@プログラム番号}プログラム番号(\index{Oコード}Oコード)についての規則を与える
%% footnote %%%%%%%%%%%%%%%%%%%%%
\footnote{機械設置時に付属の\BundledNCPrg についてはこの限りではない。}。
%%%%%%%%%%%%%%%%%%%%%%%%%%%%%%%%%



%%%%%%%%%%%%%%%%%%%%%%%%%%%%%%%%%%%%%%%%%%%%%%%%%%%%%%%%%%
%% section 12.1 %%%%%%%%%%%%%%%%%%%%%%%%%%%%%%%%%%%%%%%%%%
%%%%%%%%%%%%%%%%%%%%%%%%%%%%%%%%%%%%%%%%%%%%%%%%%%%%%%%%%%
\modHeadsection{\index{プログラムばんごう@プログラム番号}プログラム番号の基本事項}
\begin{enumerate}[label=\Roman*., ref=\Roman*]
\item \index{プログラムばんごう@プログラム番号}プログラム番号は8桁の半角英数字で表される({\ttfamily(a-zA-Z|\textbackslash d){8}})
\item 原則として、\index{プログラムばんごう@プログラム番号}プログラム番号には半角数字のみを用いる({\ttfamily\textbackslash d{8}})
\item \index{プログラムばんごう@プログラム番号}プログラム番号には右から順に1桁目, 2桁目, ..., 8桁目と数えるものとする
\item\label{item:PNbasicGE4}\index{プログラムばんごう@プログラム番号}プログラム番号は4桁までは左側0埋めを行い、5桁目以上の左側0埋めの有無は問わない
\end{enumerate}
%%%%%%%%%%%%%%%%%%%%%%%%%%%%%%%%%%%%%%%%%%%%%%%%%%%%%%%%%%
%% hosoku %%%%%%%%%%%%%%%%%%%%%%%%%%%%%%%%%%%%%%%%%%%%%%%%
%%%%%%%%%%%%%%%%%%%%%%%%%%%%%%%%%%%%%%%%%%%%%%%%%%%%%%%%%%
\begin{hosoku}
なおこの規則だと、\BundledNCPrg(\Gprgbox{O7xxx}, \Gprgbox{O8xxx}, \Gprgbox{O9xxx})と重複する恐れがある。
これについては、実際にそうした問題に直面したときにその都度に対応するものとする。
基本的には、\BundledNCPrg を(可能であれば)変更する方針とする。
\end{hosoku}
%%%%%%%%%%%%%%%%%%%%%%%%%%%%%%%%%%%%%%%%%%%%%%%%%%%%%%%%%%
%%%%%%%%%%%%%%%%%%%%%%%%%%%%%%%%%%%%%%%%%%%%%%%%%%%%%%%%%%
%%%%%%%%%%%%%%%%%%%%%%%%%%%%%%%%%%%%%%%%%%%%%%%%%%%%%%%%%%


%%%%%%%%%%%%%%%%%%%%%%%%%%%%%%%%%%%%%%%%%%%%%%%%%%%%%%%%%%
%% section 12.2 %%%%%%%%%%%%%%%%%%%%%%%%%%%%%%%%%%%%%%%%%%
%%%%%%%%%%%%%%%%%%%%%%%%%%%%%%%%%%%%%%%%%%%%%%%%%%%%%%%%%%
\modHeadsection{番号付け:\index{8けため(プログラムばんごう)@8桁目(プログラム番号)}\index{7けため(プログラムばんごう)@7桁目(プログラム番号)}8, 7桁目}
\index{7けため(プログラムばんごう)@7桁目(プログラム番号)}7桁目および\index{8けため(プログラムばんごう)@8桁目(プログラム番号)}8桁目はともに0とする。

これに伴い、\ref{item:PNbasicGE4}に基づき、以下では0埋めを省略してすべての\index{プログラムばんごう@プログラム番号}プログラム番号を6桁の数字 ({\ttfamily\textbackslash d{6}})で表す。


\clearpage
%%%%%%%%%%%%%%%%%%%%%%%%%%%%%%%%%%%%%%%%%%%%%%%%%%%%%%%%%%
%% section 12.3 %%%%%%%%%%%%%%%%%%%%%%%%%%%%%%%%%%%%%%%%%%
%%%%%%%%%%%%%%%%%%%%%%%%%%%%%%%%%%%%%%%%%%%%%%%%%%%%%%%%%%
\modHeadsection{番号付け:\index{6けため(プログラムばんごう)@6桁目(プログラム番号)}6桁目}
\index{6けため(プログラムばんごう)@6桁目(プログラム番号)}6桁目は主に\index{NCプログラム}NCプログラムの種類を表すものとし、以下のように分類する。

%%%%%%%%%%%%%%%%%%%%%%%%%%%%%%%%%%%%%%%%%%%%%%%%%%%%%%%%%%
%% subsection 9.3.1 %%%%%%%%%%%%%%%%%%%%%%%%%%%%%%%%%%%%%%
%%%%%%%%%%%%%%%%%%%%%%%%%%%%%%%%%%%%%%%%%%%%%%%%%%%%%%%%%%
\subsection{6桁目:0}
6桁目が0のものは、原則として\index{メインプログラム}メインプログラムとする。
このとき、下5桁は製品の\DrawingNumber(番号部分・右詰め, {\ttfamily\textbackslash d{5}})とする
%% footnote %%%%%%%%%%%%%%%%%%%%%
\footnote{稀に、\DrawingNumber にアルファベットが含まれるものが存在する。
その場合は、その都度に別途対応する。}。
%%%%%%%%%%%%%%%%%%%%%%%%%%%%%%%%%

%%%%%%%%%%%%%%%%%%%%%%%%%%%%%%%%%%%%%%%%%%%%%%%%%%%%%%%%%%
%% subsection 9.3.2 %%%%%%%%%%%%%%%%%%%%%%%%%%%%%%%%%%%%%%
%%%%%%%%%%%%%%%%%%%%%%%%%%%%%%%%%%%%%%%%%%%%%%%%%%%%%%%%%%
\subsection{6桁目:0, 9以外}
6桁目が0, 9以外の場合は、以下のように分類する。
\begin{enumerate}[label=\arabic*., ref=\arabic*, start=1]
\item\label{item:6Mmain} 測定(\Dimple 以外)を行うNCプログラム({\ttfamily1\textbackslash d{5}})
\item\label{item:6MD} 測定(\Dimple)を行うNCプログラム({\ttfamily2\textbackslash d{5}})
%\item\label{item:6MN} 測定(\ReliefGroove)を行うNCプログラム
\setcounter{enumi}{3}
\item\label{item:6Kmain} 加工(\Dimple 以外)を行うNCプログラム({\ttfamily4\textbackslash d{5}})
\item\label{item:6KD} 加工(\Dimple)を行うNCプログラム({\ttfamily5\textbackslash d{5}})
%\item\label{item:6KN} 加工(\ReliefGroove)を行うNCプログラム
\end{enumerate}
複数の用途での使用が想定されるものに対しては、番号の若いほうに合わせる。


%%%%%%%%%%%%%%%%%%%%%%%%%%%%%%%%%%%%%%%%%%%%%%%%%%%%%%%%%%
%% subsection 9.3.1 %%%%%%%%%%%%%%%%%%%%%%%%%%%%%%%%%%%%%%
%%%%%%%%%%%%%%%%%%%%%%%%%%%%%%%%%%%%%%%%%%%%%%%%%%%%%%%%%%
\subsection{6桁目:9}
6桁目が9のものは、製品の測定・加工に直接関係しない\index{NCプログラム}NCプログラムとする。
このとき、下5桁はその都度に別途考慮し番号付けを行う。({\ttfamily9\textbackslash d{5}})
%%%%%%%%%%%%%%%%%%%%%%%%%%%%%%%%%%%%%%%%%%%%%%%%%%%%%%%%%%
%% hosoku %%%%%%%%%%%%%%%%%%%%%%%%%%%%%%%%%%%%%%%%%%%%%%%%
%%%%%%%%%%%%%%%%%%%%%%%%%%%%%%%%%%%%%%%%%%%%%%%%%%%%%%%%%%
\begin{hosoku}
たとえば、\index{タッチセンサーでんげん@タッチセンサー電源}タッチセンサー電源のON・OFF, \index{だんきうんてん@暖機運転}暖機運転, \index{こうぐそくてい@工具測定}工具測定などのNCプログラムはこれに属するものとする。
\end{hosoku}
%%%%%%%%%%%%%%%%%%%%%%%%%%%%%%%%%%%%%%%%%%%%%%%%%%%%%%%%%%
%%%%%%%%%%%%%%%%%%%%%%%%%%%%%%%%%%%%%%%%%%%%%%%%%%%%%%%%%%
%%%%%%%%%%%%%%%%%%%%%%%%%%%%%%%%%%%%%%%%%%%%%%%%%%%%%%%%%%


\clearpage
%%%%%%%%%%%%%%%%%%%%%%%%%%%%%%%%%%%%%%%%%%%%%%%%%%%%%%%%%%
%% section 20.04 %%%%%%%%%%%%%%%%%%%%%%%%%%%%%%%%%%%%%%%%%
%%%%%%%%%%%%%%%%%%%%%%%%%%%%%%%%%%%%%%%%%%%%%%%%%%%%%%%%%%
\modHeadsection{番号付け:\index{5けため(プログラムばんごう)@5桁目(プログラム番号)}5桁目}
\index{5けため(プログラムばんごう)@5桁目(プログラム番号)}5桁目は、以下のように分類する。
なお、以降では6桁目が\ref{item:6Mmain}, \ref{item:6MD}, \ref{item:6Kmain}, \ref{item:6KD}の場合のみについて記述する
\begin{enumerate}[label=\alph*)]
\item 6桁目が\ref{item:6Mmain}(\Dimple 以外の測定)の場合、5桁目を以下にように分類する
  \begin{enumerate}[label=\arabic*., ref=\arabic*, start=1]
  \item%\label{item:5MCOBsZ}
    芯出しにおいて、$Z$一定で($X$または$Y$の)外側両面を測定する\index{NCプログラム(そとがわりょうめんそくてい)@NCプログラム(外側両面測定)}NCプログラムNCプログラム({\ttfamily 11\textbackslash d{4}})
  \item%\label{item:5MCOO}
    \index{しんだし@芯出し}芯出しにおいて、($XY$)外側片面を測定する\index{NCプログラム(そとがわかためんそくてい)@NCプログラム(外側片面測定)}NCプログラム({\ttfamily 12\textbackslash d{4}})
  \item%\label{item:5MCIB}
    芯出しにおいて、$Z$一定で($X$または$Y$の)内側両面を測定する\index{NCプログラム(うちがわりょうめんそくてい)@NCプログラム(内側両面測定)}NCプログラム({\ttfamily 13\textbackslash d{4}})
  \item%\label{item:5MCIO}
    芯出しにおいて、($XY$)内側片面を測定する\index{NCプログラム(うちがわかためんそくてい)@NCプログラム(内側片面測定)}NCプログラム({\ttfamily 14\textbackslash d{4}})
  \item%\label{item:5MCL}
    \expandafterindex{\yomiCenterlineEndFaceDifMeasurement@\nameCenterlineEndFaceDifMeasurement}\nameCenterlineEndFaceDif を測定する\expandafterindex{NCプログラム(\yomiCenterlineEndFaceDif)@NCプログラム(\nameCenterlineEndFaceDif)}NCプログラム({\ttfamily 15\textbackslash d{5}})
  \end{enumerate}
\item 6桁目が\ref{item:6MD}, \ref{item:6KD}(\Dimple の測定, 加工)の場合、5桁目を以下のように分類する
  \begin{enumerate}[label=\arabic*., ref=\arabic*]
  \item \Dimple の行と\CenterCurvatureLine 上の交点への移動を繰返す\expandafterindex{NCプログラム(\yomiDimple)@NCプログラム(\nameDimple)}NCプログラム({\ttfamily 21\textbackslash d{4}})
%  \item \Dimple において、各内面方向への($X$または$Y$方向)の移動を繰返す\expandafterindex{NCプログラム(\yomiDimple)@NCプログラム(\nameDimple)}NCプログラム({\ttfamily 22\textbackslash d{4}})
  \item \Dimple の個々の行方向の移動を繰返す\expandafterindex{NCプログラム(\yomiDimple)@NCプログラム(\nameDimple)}NCプログラム({\ttfamily 22\textbackslash d{4}})
  \item \Dimple の個々の深さ方向に測定または加工する\expandafterindex{NCプログラム(\yomiDimple)@NCプログラム(\nameDimple)}NCプログラム({\ttfamily [25]3\textbackslash d{4}})
%  \item 主にレベル4の階層で用いるNCプログラム(c)
  \end{enumerate}
%  複数の用途での使用が想定されるものに対しては、番号の若いほうに合わせる。
\item 6桁目が\ref{item:6Kmain}(\Dimple 以外の加工)の場合、5桁目を以下にように分類する
  \begin{enumerate}[label=\arabic*., ref=\arabic*, start=1]
%  \item\label{item:5Kaux} 加工の種類に依存しない加工のNCプログラム(\ttfamily{40\textbackslash d{4}})
  \item%\label{item:5KF}
    \EndFacecutMilling の位置決めを行う\expandafterindex{NCプログラム(\yomiEndFacecut)@NCプログラム(\nameEndFacecut)}NCプログラム({\ttfamily 41\textbackslash d{4}})
  \item\label{item:5KO} 主に、\OutcutMilling の位置決めを行う\expandafterindex{NCプログラム(\yomiOutcut)@NCプログラム(\nameOutcut)}NCプログラム({\ttfamily 42\textbackslash d{4}})
  \item%\label{item:5KK}
    \KeywayMilling の位置決めを行う\expandafterindex{NCプログラム(\yomiKeyway)@NCプログラム(\nameKeyway)}NCプログラム({\ttfamily 43\textbackslash d{4}})
  \item%\label{item:5KCO}
    \EndFaceOutCChamferMilling の位置決めを行う\expandafterindex{NCプログラム(\yomiEndFaceOutCChamfer)@NCプログラム(\nameEndFaceOutCChamfer)}NCプログラム({\ttfamily 44\textbackslash d{4}})
  \item%\label{item:5KCI}
    \EndFaceInCChamferMilling の位置決めを行う\expandafterindex{NCプログラム(\yomiEndFaceInCChamfer)@NCプログラム(\nameEndFaceInCChamfer)}NCプログラム({\ttfamily 45\textbackslash d{4}})
  \item%\label{item:5KZ}
    \EndFaceBoringMilling の\expandafterindex{NCプログラム(\yomiEndFaceBoring)@NCプログラム(\nameEndFaceBoring)}NCプログラム({\ttfamily 46\textbackslash d{4}})
  \item%\label{item:5KI}
    \IncutBoringMilling の\expandafterindex{NCプログラム(\yomiIncutBoring)@NCプログラム(\nameIncutBoring)}NCプログラム({\ttfamily 47\textbackslash d{4}})
  \setcounter{enumii}{8}
  \item 位置決め後、実際に加工を行う\index{NCプログラム}NCプログラム({\ttfamily 49\textbackslash d{4}})
  \end{enumerate}
\end{enumerate}


%%%%%%%%%%%%%%%%%%%%%%%%%%%%%%%%%%%%%%%%%%%%%%%%%%%%%%%%%%
%% section 12.4 %%%%%%%%%%%%%%%%%%%%%%%%%%%%%%%%%%%%%%%%%%
%%%%%%%%%%%%%%%%%%%%%%%%%%%%%%%%%%%%%%%%%%%%%%%%%%%%%%%%%%
\modHeadsection{番号付け:\index{4けため(プログラムばんごう)@4桁目(プログラム番号)}4桁目}
(6桁目が\ref{item:6Mmain}, \ref{item:6MD}, \ref{item:6Kmain}, \ref{item:6KD}の場合)\index{4けため(プログラムばんごう)@4桁目(プログラム番号)}4桁目は、以下のように分類する。
\begin{enumerate}[label=\alph*), ref=\alph*)]
\item 6桁目が\ref{item:6Kmain}\hx 以外の場合、4桁目は0とする({\ttfamily{[126][1-5]0\textbackslash d{3}}})
\item 6桁目が\ref{item:6Kmain}(\Dimple 以外の加工)の場合、4桁目を以下にように分類する
  \begin{enumerate}[label=\alph{enumi}\,-\arabic*), leftmargin=\leftmargini]
  \item 5桁目が\ref{item:5KO}(\OutcutMilling)以外の場合、4桁目は0とする({\ttfamily{4[13-59]0\textbackslash d{3}}})
  \item 5桁目が\ref{item:5KO}(\OutcutMilling)の場合、4桁目を以下のように分類する
    \begin{enumerate}[label=\arabic*., ref=\arabic*, start=0, leftmargin=*]
    \item \EndFace に垂直方向の\OutcutMilling の\expandafterindex{NCプログラム(\yomiOutcut)@NCプログラム(\nameOutcut)}NCプログラム({\ttfamily{420\textbackslash d{3}}})
    \item \CurvedOutcutMilling の\expandafterindex{NCプログラム(\yomiCurvedOutcut)@NCプログラム(\nameCurvedOutcut)}NCプログラム({\ttfamily{421\textbackslash d{3}}})
    \end{enumerate}
  \end{enumerate}
\end{enumerate}



%%%%%%%%%%%%%%%%%%%%%%%%%%%%%%%%%%%%%%%%%%%%%%%%%%%%%%%%%%
%% section 12.6 %%%%%%%%%%%%%%%%%%%%%%%%%%%%%%%%%%%%%%%%%%
%%%%%%%%%%%%%%%%%%%%%%%%%%%%%%%%%%%%%%%%%%%%%%%%%%%%%%%%%%
\modHeadsection{番号付け:\index{3けため(プログラムばんごう)@3桁目(プログラム番号)}\index{2けため(プログラムばんごう)@2桁目(プログラム番号)}3, 2桁目}
(6桁目が\ref{item:6Mmain}, \ref{item:6MD}, \ref{item:6Kmain}, \ref{item:6KD}\hx の場合の)\index{3けため(プログラムばんごう)@3桁目(プログラム番号)}3桁目および\index{2けため(プログラムばんごう)@2桁目(プログラム番号)}2桁目はともに0とする。({\ttfamily{[1245][1-69][01]0{2}\textbackslash d}})


\clearpage
%%%%%%%%%%%%%%%%%%%%%%%%%%%%%%%%%%%%%%%%%%%%%%%%%%%%%%%%%%
%% section 12.7 %%%%%%%%%%%%%%%%%%%%%%%%%%%%%%%%%%%%%%%%%%
%%%%%%%%%%%%%%%%%%%%%%%%%%%%%%%%%%%%%%%%%%%%%%%%%%%%%%%%%%
\modHeadsection{番号付け:1桁目}
(6桁目が\ref{item:6Mmain}, \ref{item:6MD}, \ref{item:6Kmain}, \ref{item:6KD}\hx の場合の)\index{1けため(プログラムばんごう)@1桁目(プログラム番号)}1桁目は、以下のように分類する。
\begin{enumerate}[label=\arabic*.]
\item 主に$X$方向に測定または加工する\index{NCプログラム(Xほうこう)@NCプログラム($X$方向)}NCプログラム({\ttfamily{[125][1-69]0{3}1}})
\item 主に$Y$方向に測定または加工する\index{NCプログラム(Yほうこう)@NCプログラム($Y$方向)}NCプログラム({\ttfamily{[125][1-69]0{3}2}})
\item 主に$Z$方向に測定または加工する\index{NCプログラム(Zほうこう)@NCプログラム($Z$方向)}NCプログラム({\ttfamily{[125][1-69]0{3}3}})
\item 工具から見て左回り・工具径補正{\ttfamily G41}で、外側を加工する\index{NCプログラム(そとがわG41)@NCプログラム(外側{\ttfamily G41})}NCプログラム({\ttfamily{[125][1-69]0{3}4}})
\item 工具から見て左回り・工具径補正{\ttfamily G42}で、外側を加工する\index{NCプログラム(そとがわG42)@NCプログラム(外側{\ttfamily G42})}NCプログラム({\ttfamily{[125][1-69]0{3}5}})
\item 工具から見て左回り・工具径補正{\ttfamily G42}で、内側を加工する\index{NCプログラム(うちがわG42)@NCプログラム(内側{\ttfamily G42})}NCプログラム({\ttfamily{[125][1-69]0{3}6}})
\item 工具から見て左回り・工具径補正{\ttfamily G41}で、内側を加工する\index{NCプログラム(うちがわG41)@NCプログラム(内側{\ttfamily G41})}NCプログラム({\ttfamily{[125][1-69]0{3}7}})
\end{enumerate}

%!TEX root = ../RPA_for_Creating_Program_Note.tex


ここでは\DMname で使用する\index{こうぐばんごう@工具番号}工具番号について記述する。



%%%%%%%%%%%%%%%%%%%%%%%%%%%%%%%%%%%%%%%%%%%%%%%%%%%%%%%%%%
%% section 05.1 %%%%%%%%%%%%%%%%%%%%%%%%%%%%%%%%%%%%%%%%%%
%%%%%%%%%%%%%%%%%%%%%%%%%%%%%%%%%%%%%%%%%%%%%%%%%%%%%%%%%%
\modHeadsection{工具番号の基本事項}
\begin{enumerate}[label=\Roman*), ref=\Roman*)]
\item \index{こうぐしゅうのうかのうほんすう@工具収録可能本数}工具収納可能本数:50本
\item \index{こうぐとうろくかのうほんすう@工具登録可能本数}工具登録可能本数:200本
\item \index{こうぐせんたくほうしき@工具選択方式}工具選択方式:番地固定近回りランダム
\item \index{さいだいこうぐすんぽう(マガジン)@最大工具寸法(マガジン)}最大工具寸法:$\phi125\times650$mm\\
両隣のポット1個ずつが空の場合:$\phi320\times650$mm, ただし、$\phi250$を超える場合は2つ隣の工具は$\phi180$mm以下であること
\item \index{さいだいこうぐしつりょう(マガジン)@最大工具質量(マガジン)}最大工具質量(ツールホルダ含む):30kg
\item \index{きょようさいだいモーメント(マガジン)@許容最大モーメント(マガジン)}許容最大モーメント(ゲージラインより):30Nm
\item \index{こうぐこうかんじかん@工具交換時間}工具交換時間: Tool-to-Tool: 2.1s, Chip-to-Chip: 5.2s
\end{enumerate}



%\clearpage
%%%%%%%%%%%%%%%%%%%%%%%%%%%%%%%%%%%%%%%%%%%%%%%%%%%%%%%%%%
%% section 14.3 %%%%%%%%%%%%%%%%%%%%%%%%%%%%%%%%%%%%%%%%%%
%%%%%%%%%%%%%%%%%%%%%%%%%%%%%%%%%%%%%%%%%%%%%%%%%%%%%%%%%%
\modHeadsection{工具番号の割当て}
\index{こうぐばんごう@工具番号}工具番号は、以下の通りに割り当てるものとする。\\
\begin{multicollongtblr}{\DMname の工具番号の振当て}{c X[l]}
番号 & 内容\\
01 & (空)\\
02-05 & \index{たんめんかこう@端面加工}端面加工用の工具\\
06-10 & \index{みぞかこう@溝加工}溝加工用の工具\\
11-15 & \index{Cめんとりかこう@C面取加工}C面取加工用の工具\\
16-20 & \index{がいさくかこう@外削加工}外削加工用の工具\\
21-30 & (不使用)\\
31-35 & \expandafterindex{\dimplekana かこう@\dimple 加工}\dimple 加工用の工具(\index{Tスロットカッター}Tスロットカッター)\\
36-40 & \dimple 加工用の工具(\index{アングルヘッド}アングルヘッド)\\
41-45 & \index{にがしみぞかこう@逃し溝加工}逃し溝加工用の工具\\
46-48 & (不使用)\\
49, 50 & 測定用の工具(\index{センサープローブ}センサープローブ)\\
\end{multicollongtblr}



\clearpage
%%%%%%%%%%%%%%%%%%%%%%%%%%%%%%%%%%%%%%%%%%%%%%%%%%%%%%%%%%
%% section 13.2 %%%%%%%%%%%%%%%%%%%%%%%%%%%%%%%%%%%%%%%%%%
%%%%%%%%%%%%%%%%%%%%%%%%%%%%%%%%%%%%%%%%%%%%%%%%%%%%%%%%%%
\modHeadsection{登録工具}
\dateTourokuKougu 時点において、\index{とうろくこうぐ@登録工具}登録されている工具は以下の通りである。\\
\begin{multicollongtblr}{\DMname の登録工具}{c X[l]}
T番号 & 使用工具\\
\ttfamily T01 & 空\\
\ttfamily T02 & $\phi100$\index{たんめんかこうようフェイスミル@端面加工用フェイスミル}端面加工用フェイスミル\\
\ttfamily T06 & $\phi100\times t7$\index{みぞかこうようサイドカッター@溝加工用サイドカッター}溝加工用サイドカッター\\
%\ttfamily T07 & $\phi100\times t5$溝加工用サイドカッター\\
\ttfamily T11 & $\phi4.0\times 15^\circ$\index{めんとりかこうようテーパエンドミル@面取加工用テーパエンドミル}面取加工用テーパエンドミル\\
\ttfamily T12 & $\phi0.8\times 30^\circ$面取加工用テーパエンドミル\\
\ttfamily T13 & $\phi2.0\times 45^\circ$面取加工用テーパエンドミル\\
\ttfamily T16 & $\phi20$\index{がいさくかこうようスクエアエンドミル@外削加工用スクエアエンドミル}外削加工用スクエアエンドミル\\
\ttfamily T31 & $\phi40\times R20$\expandafterindex{\dimplekana かこうようTスロットカッター@\dimple 加工用Tスロットカッター}\dimple 加工用Tスロットカッター\\
\ttfamily T32 & $\phi40\times R6.6$\dimple 加工用Tスロットカッター\\
%\ttfamily T36 & 10.0in\expandafterindex{\dimplekana かこうようアングルヘッド@\dimple 加工用アングルヘッド}\dimple 加工用アングルヘッド\\
%\ttfamily T41 & 15.5in\index{にがしみぞかこうようアングルヘッド@逃し溝加工用アングルヘッド}逃し溝加工用アングルヘッド\\
\ttfamily T50 & $\phi10\times200$(延長100)測定用\index{タッチセンサープローブ}タッチセンサープローブ\\
\end{multicollongtblr}

%!TEX root = ../RPA_for_Creating_Program_Note.tex



工具の移動には、主に\verb|G00|, \verb|G01|, \verb|G02|, \verb|G03|, \verb|G31|が用いられる。
一般に、
\begin{enumerate}
\item \verb|G00|は主に早送りに使用されることが多い
\item \verb|G01|は直線状に移動し、主に切削の際の送りに使用されることが多い
\item \verb|G02|, \verb|G03|は円弧状に移動し、主に切削の際の送りに使用されることが多い
\item \verb|G31|は主に測定の際のスキップ機能に伴って使用されることが多い
\end{enumerate}
\verb|G00|は\index{さいだいおくりはやさ@最大送り速さ}最大送り速さで移動し、その速さは(Fコード値では)指定することはできない。
\verb|G01|, \verb|G02|, \verb|G03|, \verb|G31|はその送りの速さをFコード値で指定することができる。

なお、\verb|G28|, \verb|G30|では\verb|G00|によって移動する。
また、\verb|G31|はすべての工具で指定することができるが、スキップ機能を有するものはタッチセンサープローブのみである。
%%%%%%%%%%%%%%%%%%%%%%%%%%%%%%%%%%%%%%%%%%%%%%%%%%%%%%%%%%
%% hosoku %%%%%%%%%%%%%%%%%%%%%%%%%%%%%%%%%%%%%%%%%%%%%%%%
%%%%%%%%%%%%%%%%%%%%%%%%%%%%%%%%%%%%%%%%%%%%%%%%%%%%%%%%%%
\begin{hosoku}
この章では工具の具体的な送り速さ値を記述している。
しかし、工具の送り速さ値の具体的な数値は、(ソフトウェアでなく)ハードウェアの標準に記載するほうが望ましい。
\end{hosoku}
%%%%%%%%%%%%%%%%%%%%%%%%%%%%%%%%%%%%%%%%%%%%%%%%%%%%%%%%%%
%%%%%%%%%%%%%%%%%%%%%%%%%%%%%%%%%%%%%%%%%%%%%%%%%%%%%%%%%%
%%%%%%%%%%%%%%%%%%%%%%%%%%%%%%%%%%%%%%%%%%%%%%%%%%%%%%%%%%



%%%%%%%%%%%%%%%%%%%%%%%%%%%%%%%%%%%%%%%%%%%%%%%%%%%%%%%%%%
%% section 07.1 %%%%%%%%%%%%%%%%%%%%%%%%%%%%%%%%%%%%%%%%%%
%%%%%%%%%%%%%%%%%%%%%%%%%%%%%%%%%%%%%%%%%%%%%%%%%%%%%%%%%%
\modHeadsection{基本事項}
\begin{enumerate}
\item $X$, $Y$, $Z$方向の早送り速さ\verb|G00|のデフォルト値:60000mm/min
\item $B$方向の早送り速さ\verb|G00|のデフォルト値:12000$^\circ$/min ($\sim 33.33$回転/min)
\item $X$, $Y$, $Z$方向の指定できる送り速さの範囲:0~60000mm/min
\item 設定可能な最小単位:0.1mm/min
\item \verb|G01|, \verb|G02|, \verb|G03|, \verb|G31|を使用する際は、その速さ値(Fコード値)を省略せずに記述する
\end{enumerate}



\clearpage
%%%%%%%%%%%%%%%%%%%%%%%%%%%%%%%%%%%%%%%%%%%%%%%%%%%%%%%%%%
%% section 07.2 %%%%%%%%%%%%%%%%%%%%%%%%%%%%%%%%%%%%%%%%%%
%%%%%%%%%%%%%%%%%%%%%%%%%%%%%%%%%%%%%%%%%%%%%%%%%%%%%%%%%%
\modHeadsection{タッチセンサープローブ}
\DMname では、全長の長い\index{タッチセンサープローブ}タッチセンサープローブを用いる。
したがって、速さを大きくして移動をすると、その加速度によってセンサーが反応してしまったり、\index{タッチセンサー}タッチセンサーそのものに大きな負担がかかる。
そのため、タッチセンサーの速さに関しては他の工具よりも低めに設定する。
\begin{enumerate}[label=\Roman*., ref=\Roman*]
\item \label{item:FDTS} 原則として、\verb|G00|を用いた移動はしない
\item \ref{item:FDTS}に伴い、\verb|G28|, \verb|G30|を(直接的に)用いた移動はしない
\item \verb|G01|を位置決め(早送り)として用いるものとし、速さはF5400以下とする
\item ワークへの\index{アプローチ}アプローチの際は、\verb|G31|を用いるものとし、速さはF1500以下とする
\item 計測の際の\index{スキップ}スキップ(\verb|G31|)の速さは、計測の仕方に応じて以下のものとする
  \begin{enumerate}
  \item \index{しんごうおくれほせい@信号遅れ補正}信号遅れ補正を考慮する必要があるような場合は、速さはF50とする
  \item 信号遅れ補正を考慮する必要がない場合は、速さはF50以上300以下とする
  \end{enumerate}
\item ワークからの\index{リトラクト}リトラクトの際は\verb|G01|を用いるものとし、速さはF5400以下とする。
\end{enumerate}
ただし$Z$方向の移動はタッチセンサーではなくテーブルが移動するため、アプローチを除いて次節(タッチセンサー以外の工具)にしたがうものとする。

なお、\verb|G31|で送る場合は、センサーの電源を入れて行うこと。
電源が入っていなければ移動せず、その点がブロックエンドとなる。



%\clearpage
%%%%%%%%%%%%%%%%%%%%%%%%%%%%%%%%%%%%%%%%%%%%%%%%%%%%%%%%%%
%% section 10.2 %%%%%%%%%%%%%%%%%%%%%%%%%%%%%%%%%%%%%%%%%%
%%%%%%%%%%%%%%%%%%%%%%%%%%%%%%%%%%%%%%%%%%%%%%%%%%%%%%%%%%
\modHeadsection{タッチセンサー以外の工具}
\DMname では、$Z$方向については\index{テーブル}テーブル(\index{パレット}パレット)が移動する。
したがって、$Z$方向の速さを大きくして移動すると、その加速度によってテーブル上の\index{ジグ}ジグが動いてしまう恐れがある。
そのため、$Z$方向の速さに関しては$X$, $Y$方向の速さより低めに設定する。
\begin{enumerate}[label=\Roman*., ref=\Roman*]
\item $X$, $Y$方向の移動は、\verb|G00|を用いてもよい
\item $Z$方向の移動は、原則として、\verb|G00|を用いた移動はしない
\item $Z$方向の送り速さは、F12000以下とする
\item ワークへの\index{アプローチ}アプローチの際は、\verb|G01|を用いるものとし、速さはF6000以下とする
\item 加工の際は、それぞれの加工や状況に応じて適切なFコード値を設定する
\item ワークからの\index{リトラクト}リトラクトの際は\verb|G01|を用いるものとし、速さはF12000以下とする。
\end{enumerate}


\clearpage
\dateKouguSpeed における速さの設定値は、以下のとおりである。\\
\addtocontents{lot}{\protect\addvspace{3pt}}{}{}
\addcontentsline{lot}{section}{\numberline{\thesection}\Sectionname}

\begin{3columnstable}{工具の速さ 一覧\TBW}{内容}{速さ値}{備考}{Sl}{Sc}{Sc}
最大送り$XY$(タッチセンサープローブ以外) & 60000 & \verb|G00|\\\hline
最大送り$Z$(タッチセンサープローブ以外)  & 12000 &\\\hline
最大送り$XYZ$(タッチセンサープローブ)   & 5400 & \\\hline
アプローチ(タッチセンサープローブ以外)   & 6000 & \\\hline
アプローチ(タッチセンサープローブ)      & 1500 & \verb|G31|\\\hline
アプローチ(片側測定)                 & 50 & \verb|G31|\\\hline
アプローチ(両側測定)                 & 200 & \verb|G31|\\\hline
端面加工(直線)                     & 900 & \\\hline
端面加工(コーナー)                  & 900 &\\\hline
外削加工(直線)                     & 400 &\\\hline
外削加工(コーナー)                  & 400 &\\\hline
溝加工(直線)                       & 500 &\\\hline
溝加工(コーナー)                    & 400 &\\\hline
外面取加工(直線)                    & 500 &\\\hline
外面取加工(コーナー)                 & 400 &\\\hline
内面取加工(直線)                    & 400 &\\\hline
内面取加工(コーナー)                 & 400 &\\\hline
座ぐり加工(直線)\TBW                &  & \\\hline
座ぐり加工(コーナー)\TBW             &  &\\\hline
\dimple 加工(表面アプローチ)         & 540 &\\\hline
\dimple 加工(加工)                 & 100 &
\end{3columnstable}



%!TEX root = ../RPA_for_Creating_Program_Note.tex



(to be written...)
%%%%%%%%%%%%%%%%%%%%%%%%%%%%%%%%%%%%%%%%%%%%%%%%%%%%%%%%%%
%% hosoku %%%%%%%%%%%%%%%%%%%%%%%%%%%%%%%%%%%%%%%%%%%%%%%%
%%%%%%%%%%%%%%%%%%%%%%%%%%%%%%%%%%%%%%%%%%%%%%%%%%%%%%%%%%
\begin{hosoku}
この章では工具の具体的な回転数を記述している。
しかし、工具の回転数の具体的な数値は、(ソフトウェアでなく)ハードウェアの標準に記載するほうが望ましい。
\end{hosoku}
%%%%%%%%%%%%%%%%%%%%%%%%%%%%%%%%%%%%%%%%%%%%%%%%%%%%%%%%%%
%%%%%%%%%%%%%%%%%%%%%%%%%%%%%%%%%%%%%%%%%%%%%%%%%%%%%%%%%%
%%%%%%%%%%%%%%%%%%%%%%%%%%%%%%%%%%%%%%%%%%%%%%%%%%%%%%%%%%


%!TEX root = ../RfCPN.tex


\modHeadchapter[lot]{\index{シーケンスばんごう@シーケンス番号}シーケンス番号(\index{Nコード}Nコード値)}
一般に、\index{シーケンスばんごう@シーケンス番号}シーケンス番号(\index{Nコードち@Nコード値}Nコード値)は重複していなければ自由に付けて問題はない。
しかしこれに一定の規則を与えておくことで、\index{NCプログラム}NCプログラムの
\begin{enumerate}[label=\sarrow]
\item どの部分で何が行われているか
\item どの部分で\index{エラー}エラーが起きているか
\item 途中から稼働する場合、どの\index{ブロック}ブロックから始めればよいか
\end{enumerate}
など、作業や管理をする際に効率よく制御できることが見込まれる。

そこで、ここでは\index{シーケンスばんごう@シーケンス番号}シーケンス番号(\index{Nコード}Nコード)についての規則を与える。



%%%%%%%%%%%%%%%%%%%%%%%%%%%%%%%%%%%%%%%%%%%%%%%%%%%%%%%%%%
%% section 14.1 %%%%%%%%%%%%%%%%%%%%%%%%%%%%%%%%%%%%%%%%%%
%%%%%%%%%%%%%%%%%%%%%%%%%%%%%%%%%%%%%%%%%%%%%%%%%%%%%%%%%%
\modHeadsection{\index{シーケンスばんごう@シーケンス番号}シーケンス番号の基本事項}
\begin{enumerate}[label=\Roman*), ref=\Roman*)]
\item \index{シーケンスばんごう@シーケンス番号}シーケンス番号は、右から順に1桁目, 2桁目, ...と数えるものとする
\item \expandafterindex{シーケンスばんごう(\yomiCreatedNCMainPrg)@シーケンス番号(\nameCreatedNCMainPrg)}\nameCreatedNCMainPrg%
%% footnote %%%%%%%%%%%%%%%%%%%%%
\footnote{ここでいう\index{メインプログラム}メインプログラムとは、下5桁が\DrawingNumber と一致するものを指す。}%
%%%%%%%%%%%%%%%%%%%%%%%%%%%%%%%%%
のシーケンス番号は4桁とし、0埋めする
\item \expandafterindex{シーケンスばんごう(\yomiCreatedNCSubPrg)@シーケンス番号(\nameCreatedNCSubPrg)}\nameCreatedNCSubPrg のシーケンス番号は4桁とし、0埋めする
\item \index{NCプログラム}NCプログラムの始まりの\index{シーケンスばんごう@シーケンス番号}シーケンス番号は{\ttfamily N0001}とする
\item 原則として、\index{シーケンスばんごう@シーケンス番号}シーケンス番号は昇順とし、特に1桁目は連番とする
\end{enumerate}


%%%%%%%%%%%%%%%%%%%%%%%%%%%%%%%%%%%%%%%%%%%%%%%%%%%%%%%%%%
%% section 14.2 %%%%%%%%%%%%%%%%%%%%%%%%%%%%%%%%%%%%%%%%%%
%%%%%%%%%%%%%%%%%%%%%%%%%%%%%%%%%%%%%%%%%%%%%%%%%%%%%%%%%%
\modHeadsection{\expandafterindex{シーケンスばんごう(\yomiCreatedNCSubPrg)@シーケンス番号(\nameCreatedNCSubPrg)}\nameCreatedNCSubPrg のシーケンス番号}
\DMC においては、原則として\CreatedNCSubPrg は始めから実行されるものであり、途中の部分から実行されることはない。
そのため、\expandafterindex{シーケンスばんごう(\yomiCreatedNCSubPrg)@シーケンス番号(\nameCreatedNCSubPrg)}\nameCreatedNCSubPrg のシーケンス番号は記述の順に(概ね\index{ブロック}ブロックごとに)連番とする。

なお、\expandafterindex{シーケンスばんごう(エラーけんしゅつ)@シーケンス番号(エラー検出)}エラー検出時に関するシーケンス番号、および\expandafterindex{シーケンスばんごう(プログラムしゅうりょう)@シーケンス番号(プログラム終了)}プログラム終了に関するシーケンス番号については、以降で述べる\CreatedNCMainPrg のもの(\autoref{subsec:sequenceNerror}, \pageautoref{subsec:sequenceNprgEnd})と同様とする。



\clearpage
%%%%%%%%%%%%%%%%%%%%%%%%%%%%%%%%%%%%%%%%%%%%%%%%%%%%%%%%%%
%% section 14.3 %%%%%%%%%%%%%%%%%%%%%%%%%%%%%%%%%%%%%%%%%%
%%%%%%%%%%%%%%%%%%%%%%%%%%%%%%%%%%%%%%%%%%%%%%%%%%%%%%%%%%
\modHeadsection{\expandafterindex{シーケンスばんごう(\yomiCreatedNCMainPrg)@シーケンス番号(\nameCreatedNCMainPrg)}\nameCreatedNCMainPrg のシーケンス番号}
\DMC において、\CreatedNCMainPrg は\index{さぎょうしゃ@作業者}作業者が実際に設定を変更したり、途中の箇所から始めたりし得る。
そのため、\CreatedNCMainPrg では各作業(測定・加工)ごとに\index{シーケンスばんごう@シーケンス番号}シーケンス番号を割り振ることにする。


%%%%%%%%%%%%%%%%%%%%%%%%%%%%%%%%%%%%%%%%%%%%%%%%%%%%%%%%%%
%% subsection 14.3.1 %%%%%%%%%%%%%%%%%%%%%%%%%%%%%%%%%%%%%
%%%%%%%%%%%%%%%%%%%%%%%%%%%%%%%%%%%%%%%%%%%%%%%%%%%%%%%%%%
\subsection{N1000:測定(\Dimple・\ReliefGroove 以外)}
\index{タッチセンサープローブ}タッチセンサープローブを用いた測定(\Dimple ・\ReliefGroove を除く)を行う工程の\index{シーケンスばんごう@シーケンス番号}シーケンス番号は1000番台とする。
これには以下の\index{こうてい@工程}工程が含まれ、これらは2桁目の番号で区別される。
\begin{enumerate}
\item[1000:] 芯出し測定
\item[6500:] \expandafterindex{\yomiCenterlineEndFaceDif そくてい@\nameCenterlineEndFaceDif 測定}\nameCenterlineEndFaceDif 測定
\end{enumerate}


%\clearpage
%%%%%%%%%%%%%%%%%%%%%%%%%%%%%%%%%%%%%%%%%%%%%%%%%%%%%%%%%%
%% subsection 28.03.02 %%%%%%%%%%%%%%%%%%%%%%%%%%%%%%%%%%%
%%%%%%%%%%%%%%%%%%%%%%%%%%%%%%%%%%%%%%%%%%%%%%%%%%%%%%%%%%
\subsection{N2000:測定(\Dimple・\ReliefGroove)}
\Dimple および\ReliefGroove に関する\index{タッチセンサープローブ}タッチセンサープローブを用いた測定を行う工程の\index{シーケンスばんごう@シーケンス番号}シーケンス番号は2000番台とする。
これには以下の工程が含まれ、これらは3桁目の番号で区別される。
\begin{enumerate}
\item[2000:] \DimpleMeasurement
\item[2500:] \ReliefGrooveMeasurement
\end{enumerate}


%%%%%%%%%%%%%%%%%%%%%%%%%%%%%%%%%%%%%%%%%%%%%%%%%%%%%%%%%%
%% subsection 28.03.03 %%%%%%%%%%%%%%%%%%%%%%%%%%%%%%%%%%%
%%%%%%%%%%%%%%%%%%%%%%%%%%%%%%%%%%%%%%%%%%%%%%%%%%%%%%%%%%
\subsection{N3000:\DimpleMilling ・\ReliefGrooveMilling}
\DimpleMilling および\ReliefGrooveMilling を行う工程の\index{シーケンスばんごう@シーケンス番号}シーケンス番号は3000番台とする。
これには以下の工程が含まれ、これらは3桁目の番号で区別される。
\begin{enumerate}
\item[3000:] \DimpleMilling
\item[3500:] \ReliefGrooveMilling
\end{enumerate}


%%%%%%%%%%%%%%%%%%%%%%%%%%%%%%%%%%%%%%%%%%%%%%%%%%%%%%%%%%
%% subsection 28.03.04 %%%%%%%%%%%%%%%%%%%%%%%%%%%%%%%%%%%
%%%%%%%%%%%%%%%%%%%%%%%%%%%%%%%%%%%%%%%%%%%%%%%%%%%%%%%%%%
\subsection{N4000:トップ側の加工}
トップ側の加工を行う工程の\index{シーケンスばんごう@シーケンス番号}シーケンス番号は4000番台とする。
これには以下の工程が含まれ、これらは3桁目の番号で区別される。
\begin{enumerate}
\item[4000:] \TopEndFacecutMilling
\item[4100:] \TopOutcutMilling または\EndFaceBoringMilling または\IncutBoringMilling
\item[4200:] \KeywayMilling
\item[4300:] \TopEndFaceOutCChamferMilling
\item[4400:] \TopEndFaceInCChamferMilling
\end{enumerate}


\clearpage
%%%%%%%%%%%%%%%%%%%%%%%%%%%%%%%%%%%%%%%%%%%%%%%%%%%%%%%%%%
%% subsection 28.03.05 %%%%%%%%%%%%%%%%%%%%%%%%%%%%%%%%%%%
%%%%%%%%%%%%%%%%%%%%%%%%%%%%%%%%%%%%%%%%%%%%%%%%%%%%%%%%%%
\subsection{N5000:ボトム側の加工}
ボトム側の加工を行う工程の\index{シーケンスばんごう@シーケンス番号}シーケンス番号は5000番台とする。
これには以下の工程が含まれ、これらは3桁目の番号で区別される。
\begin{enumerate}
\item[5000:] \BottomEndFacecutMilling
\item[5100:] \BottomOutcutMilling
\item[5300:] \BottomEndFaceOutCChamferMilling
\item[5400:] \BottomEndFaceInCChamferMilling
\end{enumerate}


%\clearpage
%%%%%%%%%%%%%%%%%%%%%%%%%%%%%%%%%%%%%%%%%%%%%%%%%%%%%%%%%%
%% subsection 28.03.06 %%%%%%%%%%%%%%%%%%%%%%%%%%%%%%%%%%%
%%%%%%%%%%%%%%%%%%%%%%%%%%%%%%%%%%%%%%%%%%%%%%%%%%%%%%%%%%
\subsection{N8000:エラー\label{subsec:sequenceNerror}\vphantom{\ref{subsec:sequenceNerror}}\TBW}
\index{エラー}エラー検出時に\index{ジャンプ}ジャンプする\index{シーケンスばんごう@シーケンス番号}シーケンス番号は8000番台とする。
\index{エラーのしゅるい@エラーの種類}エラーの種類(\index{システムへんすう@システム変数}システム変数\ttNum3000の値)に応じて以下のように分類し、(概ね)\index{ブロック}プロックごとに連番とする。
\begin{enumerate}
\item[8000:] \verb|#3000=121 (Argument is not assigned)|
\item[8100:] \verb|#3000=...|
\item[8200:] \verb|#3000=1 (Pallet Alarm)|,\\
            \verb|#3000=145 (Sensor-Low-Battery)|, \verb|#3000=146 (Sensor-Alarm)|
\end{enumerate}


%%%%%%%%%%%%%%%%%%%%%%%%%%%%%%%%%%%%%%%%%%%%%%%%%%%%%%%%%%
%% subsection 28.03.07 %%%%%%%%%%%%%%%%%%%%%%%%%%%%%%%%%%%
%%%%%%%%%%%%%%%%%%%%%%%%%%%%%%%%%%%%%%%%%%%%%%%%%%%%%%%%%%
\subsection{N9990:\index{NCプログラム}NCプログラムの終了\label{subsec:sequenceNprgEnd}}
\index{こうてい(プログラムしゅうりょう)@工程(プログラム終了)}NCプログラムを終了する工程の\index{シーケンスばんごう@シーケンス番号}シーケンス番号は9990番台とする。
特に、\index{NCプログラム}NCプログラムの終了はN9999とする。



\clearpage
\noindent
改めて上記の\index{シーケンスばんごういちらん@シーケンス番号一覧}シーケンス番号を一覧にしておく。\\

%%%%%%%%%%%%%%%%%%%%%%%%%%%%%%%%%%%%%%%%%%%%%%%%%%%%%%%%%%
%% sequence numbers %%%%%%%%%%%%%%%%%%%%%%%%%%%%%%%%%%%%%%
%%%%%%%%%%%%%%%%%%%%%%%%%%%%%%%%%%%%%%%%%%%%%%%%%%%%%%%%%%
\begin{multicollongtblr}{シーケンス番号 一覧(メインプログラム)\TBW}{cX[l]}
N番号 & 内容\\
\ttfamily N0001 & \index{NCプログラム}NCプログラムの始まり\\
\ttfamily N10xx & \index{タッチセンサープローブそくてい@タッチセンサープローブ測定}タッチセンサープローブ測定(\index{しんだしそくてい@芯出し測定}芯出し測定)\\
\ttfamily N20xx & \index{タッチセンサープローブそくてい@タッチセンサープローブ測定}タッチセンサープローブ測定(\DimpleMeasurement)\\
\ttfamily N25xx & \index{タッチセンサープローブそくてい@タッチセンサープローブ測定}タッチセンサープローブ測定(\ReliefGrooveMeasurement)\\
\ttfamily N30xx & \DimpleMilling\\
\ttfamily N35xx & \ReliefGrooveMilling\\
\ttfamily N40xx & \TopEndFacecutMilling\\
\ttfamily N41xx & \TopOutcutMilling\\
\ttfamily N42xx & \KeywayMilling\\
\ttfamily N43xx & \TopEndFaceOutCChamferMilling\\
\ttfamily N44xx & \TopEndFaceInCChamferMilling\\
\ttfamily N45xx & \EndFaceBoringMilling\\
\ttfamily N50xx & \BottomEndFacecutMilling\\
\ttfamily N51xx & \BottomOutcutMilling\\
\ttfamily N53xx & \BottomEndFaceOutCChamferMilling\\
\ttfamily N54xx & \BottomEndFaceInCChamferMilling\\
\ttfamily N65xx & \index{タッチセンサープローブそくてい@タッチセンサープローブ測定}タッチセンサープローブ測定(\CenterlineEndFaceDifMeasurement)\\
\ttfamily N80xx & \index{エラー(ひきすう)@エラー(引数)}引数によるエラー\\
\ttfamily N81xx\TBW & \\
\ttfamily N82xx & \index{エラー(パレット)}\index{エラー(タッチセンサープローブ)}パレットまたはタッチセンサープローブによるエラー\\
\ttfamily N999x & \index{こうてい(プログラムしゅうりょう)@工程(プログラム終了)}プログラム終了の工程\\
\ttfamily N9999 & \index{プログラムしゅうりょう@プログラム終了}プログラム終了({\ttfamily M02}または{\ttfamily M30}または{\ttfamily M99})
\end{multicollongtblr}

%!TEX root = ../RPA_for_Creating_Program_Note.tex



ここでは\DMname におけるアラーム発生システム変数\hx\ttNum3000について記述する。


%%%%%%%%%%%%%%%%%%%%%%%%%%%%%%%%%%%%%%%%%%%%%%%%%%%%%%%%%%
%% section 17.1 %%%%%%%%%%%%%%%%%%%%%%%%%%%%%%%%%%%%%%%%%%
%%%%%%%%%%%%%%%%%%%%%%%%%%%%%%%%%%%%%%%%%%%%%%%%%%%%%%%%%%
\modHeadsection{アラームの分類:\DMname\TBW}
\DMname で使用される\index{アラームへんすう@アラーム変数}アラーム変数\hx\ttNum3000の値は、以下のように分類する。\\

%%%%%%%%%%%%%%%%%%%%%%%%%%%%%%%%%%%%%%%%%%%%%%%%%%%%%%%%%%
%% Alerm Classification %%%%%%%%%%%%%%%%%%%%%%%%%%%%%%%%%%
%%%%%%%%%%%%%%%%%%%%%%%%%%%%%%%%%%%%%%%%%%%%%%%%%%%%%%%%%%
\begin{3columnstable}{\DMname のアラームの分類\TBW}{アラーム番号}{メッセージ}{内容}
001 & Pallet Alarm & パレット\\\hline
002 & Sensor-Low-Battery & センサー電池減\\\hline
003 & Sensor-Alarm & センサー\\\hline
100 & Work Coordinate Is Not Assigned & ワーク座標\\\hline
200 & Argument Is Not Assigned & 引数\\
\end{3columnstable}

%%%%%%%%%%%%%%%%%%%%%%%%%%%%%%%%%%%%%%%%%%%%%%%%%%%%%%%%%%
%% Alerm Classification %%%%%%%%%%%%%%%%%%%%%%%%%%%%%%%%%%
%%%%%%%%%%%%%%%%%%%%%%%%%%%%%%%%%%%%%%%%%%%%%%%%%%%%%%%%%%
\begin{2columnstable}{\DMname のアラームの分類(バンドルのプログラム)}{アラーム番号}{メッセージ}
001 & Pallet Alarm\\\hline
002 & Pallet Alarm\\\hline
003 & Pallet Alarm\\\hline
007 & Tool-Life-Check\\\hline
010 & Macro Pallet Check Alarm\\\hline
011 & No Program Selected\\\hline
012 & Wrong Pallet Alarm\\\hline
121 & Argument Is Not Assigned\\\hline
140 & A-Axis-Is-Not-Command\\\hline
141 & Tool-Measure-Alarm, Tool-Check, Measured-Value-Is-Over\\\hline
142 & Tool-Brake\\\hline
143 & Check-X,Y-Axis-Command-Value, Work-Coordinate-Setting-Error\\\hline
145 & MP10/MP12/MP60-Low-Battery\\\hline
146 & MP10/MP12/MP60-Alarm
\end{2columnstable}

%!TEX root = ../RPA_for_Creating_Program_Note.tex



ここでは\DMname の加工システムで使用している\index{コモンへんすう@コモン変数}コモン変数について述べる。


%%%%%%%%%%%%%%%%%%%%%%%%%%%%%%%%%%%%%%%%%%%%%%%%%%%%%%%%%%
%% section 11.1 %%%%%%%%%%%%%%%%%%%%%%%%%%%%%%%%%%%%%%%%%%
%%%%%%%%%%%%%%%%%%%%%%%%%%%%%%%%%%%%%%%%%%%%%%%%%%%%%%%%%%
\modHeadsection{コモン変数の範囲}
\DMname で使用可能なコモン変数は以下のとおりである。
\begin{enumerate}
\item \ttNum100\,-\ttNum199
\item \ttNum400\,-\ttNum999
\item \ttNum900000\,-\ttNum907399
\end{enumerate}



%%%%%%%%%%%%%%%%%%%%%%%%%%%%%%%%%%%%%%%%%%%%%%%%%%%%%%%%%%
%% section 18.2 %%%%%%%%%%%%%%%%%%%%%%%%%%%%%%%%%%%%%%%%%%
%%%%%%%%%%%%%%%%%%%%%%%%%%%%%%%%%%%%%%%%%%%%%%%%%%%%%%%%%%
\modHeadsection{\ttNum100\,-\ttNum199}


%%%%%%%%%%%%%%%%%%%%%%%%%%%%%%%%%%%%%%%%%%%%%%%%%%%%%%%%%%
%% subsection 18.2.1 %%%%%%%%%%%%%%%%%%%%%%%%%%%%%%%%%%%%%
%%%%%%%%%%%%%%%%%%%%%%%%%%%%%%%%%%%%%%%%%%%%%%%%%%%%%%%%%%
\subsection{\ttNum100\,-\ttNum174:一時保存値}
\ttNum100\,-\ttNum174については、(機械設置時の)\index{バンドルのプログラム}バンドルのプログラムやカスタマイズされた\index{Mコード}Mコードで既に使用されているものが多いため、基本的には(\index{RHS(コモンへんすう)@RHS(コモン変数)}RHSとしては)使用しないものとし、一時的なもの(\index{LHS(コモンへんすう)@LHS(コモン変数)}LHS)として扱うものとする。
\newline


%\clearpage
\noindent\ttNum100\,-\ttNum110については、主に一時的な保存に用いるものとする。\\

\begin{2columnstable}[white]{\ttNum100\,-\ttNum110:一時保存値}{|Sc|Sl|}{番号}{内容}
\ttNum100 & 各工程 切削回数用 一時保存値(仕上げ前 全削り代$X$ or $Z$)\\\hline
\ttNum101 & 各工程 切削回数用 一時保存値(仕上げ前 全削り代$Y$)\\\hline
\ttNum102 & 各工程 切削回数用 一時保存値 (max[\ttNum100, \ttNum101])\\\hline
\ttNum103 & 各工程 切削回数用 一時保存値(仕上げ前 切削回数)\\\hline
\ttNum104 & 各工程 切削回数用 一時保存値(加工時 径$X$)\\\hline
\ttNum105 & 各工程 切削回数用 一時保存値(加工時 径$Y$)\\\hline
\ttNum106 & 各工程 切削回数用 一時保存値(仕上げ 切削回数)\\\hline
\rowcolor{unusingVariables}
$\cdots$ & (以下 予備)\\
\end{2columnstable}


\clearpage
%%%%%%%%%%%%%%%%%%%%%%%%%%%%%%%%%%%%%%%%%%%%%%%%%%%%%%%%%%
%% subsection 18.2.2 %%%%%%%%%%%%%%%%%%%%%%%%%%%%%%%%%%%%%
%%%%%%%%%%%%%%%%%%%%%%%%%%%%%%%%%%%%%%%%%%%%%%%%%%%%%%%%%%
\subsection{\ttNum175\,-\ttNum199:各工程後一時停止}
\noindent\ttNum175\,-\ttNum199については、主に各工程後の確認のためのものとする。\\

\begin{3columnstable}[white]{\ttNum175\,-\ttNum199:各工程後一時停止}{|Sc|Sl|Sc|}{番号}{内容}{初期値}
\rowcolor{unusingVariables}
\ttNum175 & (以下 予備) &\\\hline
\rowcolor{unusingVariables}
$\cdots$ & \qquad$\cdots$ &\\\hline
\rowcolor{unusingVariables}
\ttNum183 & (不使用) &\\\hline
\ttNum184 & 芯出し測定後 一時停止 (0:non-stop, 1: \verb|M00|) & 0\\\hline
\rowcolor{unusingVariables}
\ttNum185 & (不使用) &\\\hline
\ttNum186 & \dimple 測定後 一時停止 (0:non-stop, 1: \verb|M00|) & 0\\\hline
\ttNum187 & \dimple 加工後 一時停止 (0:non-stop, 1: \verb|M00|, 2:\verb|O900003|) & 0\\\hline
\rowcolor{unusingVariables}
\ttNum188 & (不使用) &\\\hline
\ttNum189 & トップ端面加工後 一時停止 (0:non-stop, 1: \verb|M00|, 2:\verb|O900003|) & 0\\\hline
\ttNum190 & トップ外削加工後 一時停止 (0:non-stop, 1: \verb|M00|, 2:\verb|O900003|) & 0\\\hline
\ttNum191 & トップ溝加工後 一時停止 (0:non-stop, 1: \verb|M00|, 2:\verb|O900003|) & 0\\\hline
\ttNum192 & トップ外面取加工後 一時停止 (0:non-stop, 1: \verb|M00|, 2:\verb|O900003|) & 0\\\hline
\ttNum193 & トップ内面取加工後 一時停止 (0:non-stop, 1: \verb|M00|, 2:\verb|O900003|) & 0\\\hline
\ttNum194 & トップ座ぐり加工後 一時停止 (0:non-stop, 1: \verb|M00|, 2:\verb|O900003|) & 0\\\hline
\rowcolor{unusingVariables}
\ttNum195 & (不使用) &\\\hline
\ttNum196 & ボトム端面加工後 一時停止 (0:non-stop, 1: \verb|M00|, 2:\verb|O900003|) & 0\\\hline
\ttNum197 & ボトム外削加工後 一時停止 (0:non-stop, 1: \verb|M00|, 2:\verb|O900003|) & 0\\\hline
\ttNum198 & ボトム外面取加工後 一時停止 (0:non-stop, 1: \verb|M00|, 2:\verb|O900003|) & 0\\\hline
\ttNum199 & ボトム内面取加工後 一時停止 (0:non-stop, 1: \verb|M00|, 2:\verb|O900003|) & 0
\end{3columnstable}



\clearpage
%%%%%%%%%%%%%%%%%%%%%%%%%%%%%%%%%%%%%%%%%%%%%%%%%%%%%%%%%%
%% section 18.3 %%%%%%%%%%%%%%%%%%%%%%%%%%%%%%%%%%%%%%%%%%
%%%%%%%%%%%%%%%%%%%%%%%%%%%%%%%%%%%%%%%%%%%%%%%%%%%%%%%%%%
\modHeadsection{\ttNum400\,-\ttNum474:加工時の調整}
\ttNum400\,-\ttNum474については、\index{さぎょうしゃ(コモンへんすう)@作業者(コモン変数)}作業者が入力・変更することが想定されるものとする。


%\clearpage
%%%%%%%%%%%%%%%%%%%%%%%%%%%%%%%%%%%%%%%%%%%%%%%%%%%%%%%%%%
%% subsection 18.3.1 %%%%%%%%%%%%%%%%%%%%%%%%%%%%%%%%%%%%%
%%%%%%%%%%%%%%%%%%%%%%%%%%%%%%%%%%%%%%%%%%%%%%%%%%%%%%%%%%
\subsection{\ttNum400\,-\ttNum424}

\begin{3columnstable}[white]{\ttNum400\,-\ttNum404:初期設定}{|Sc|Sl|Sc|}{番号}{内容}{初期値}
\ttNum400 & トップ端面 全削り代 &\\\hline
\ttNum401 & ボトム端面 全削り代(0 or \ttNum0: \ttNum400) &\\\hline
\ttNum402 & 計測・加工 開始N番号 & 0\\\hline
\ttNum403 & 通り芯測定(0:off, 1:on) & 0\\\hline
\rowcolor{unusingVariables}
\ttNum404 & (不使用) &
\end{3columnstable}

\begin{3columnstable}[white]{\ttNum405\,-\ttNum414:測定時の調整(\dimple 除く)}{|Sc|Sl|Sc|}{番号}{内容}{初期値}
\ttNum405 & (\verb|G54|$X$)ボトム外$X$芯出し(両側・片側測定)測定位置$Z-$補正($X$自動補正) & 0\\\hline
\ttNum406 & (\verb|G54|$Y$)ボトム外$Y$芯出し(両側測定)測定位置$Z-$補正 & 0\\\hline
\ttNum407 & (\verb|G55|$X$)ボトム内$X$芯出し(両側測定)測定位置$Z-$補正($X$自動補正) & 0\\\hline
\ttNum408 & (\verb|G55|$Y$)ボトム内$Y$芯出し(両側測定)測定位置$Z-$補正 & 0\\\hline
\rowcolor{unusingVariables}
\ttNum409 & (不使用) &\\\hline
\ttNum410 & (\verb|G56|$X$)トップ外$X$芯出し(両側・片側測定)測定位置$Z-$補正($X$自動補正) & 0\\\hline
\ttNum411 & (\verb|G56|$Y$)トップ外$Y$芯出し(両側測定)測定位置$Z-$補正 & 0\\\hline
\ttNum412 & (\verb|G57|$X$)トップ内$X$芯出し(両側測定)測定位置$Z-$補正($X$自動補正) & 0\\\hline
\ttNum413 & (\verb|G57|$Y$)トップ内$Y$芯出し(両側測定)測定位置$Z-$補正 & 0\\\hline
\rowcolor{unusingVariables}
\ttNum414 & (不使用) &
\end{3columnstable}
%%%%%%%%%%%%%%%%%%%%%%%%%%%%%%%%%%%%%%%%%%%%%%%%%%%%%%%%%%
%% marker %%%%%%%%%%%%%%%%%%%%%%%%%%%%%%%%%%%%%%%%%%%%%%%%
%%%%%%%%%%%%%%%%%%%%%%%%%%%%%%%%%%%%%%%%%%%%%%%%%%%%%%%%%%
\begin{marker}
\MXIface(\index{がいさくちゅうしん@外削中心}外削中心測定)の場合、測定位置$Z$は端面$Z$位置でないことに注意
\end{marker}
%%%%%%%%%%%%%%%%%%%%%%%%%%%%%%%%%%%%%%%%%%%%%%%%%%%%%%%%%%
%%%%%%%%%%%%%%%%%%%%%%%%%%%%%%%%%%%%%%%%%%%%%%%%%%%%%%%%%%
%%%%%%%%%%%%%%%%%%%%%%%%%%%%%%%%%%%%%%%%%%%%%%%%%%%%%%%%%%


\clearpage
\begin{3columnstable}[white]{\ttNum415\,-\ttNum424:加工時の調整(端面・\dimple 除く)}{|Sc|Sl|Sc|}{番号}{内容}{初期値}
\ttNum415 & トップ外削 A面肉厚$+$補正(外削中心$X-$補正) & 0\\\hline
\ttNum416 & トップ外削 仕上げ前 一時停止 (0:non-stop, 1:\verb|M00|, 2:扉前\verb|M00|) & 0\\\hline
\ttNum417 & トップ外削 仕上げ加工 追加回数 (上限3) & 0\\\hline
\rowcolor{unusingVariables}
\ttNum418 & (不使用) &\\\hline
\ttNum419 & 溝位置$+$補正(溝幅不変) & 0\\\hline
\ttNum420 & 溝幅$+$補正 & 0\\\hline
\ttNum421 & 溝A面深さ$+$補正(溝径中心$X-$補正) & 0\\\hline
\ttNum422 & 溝幅$Z$方向中央切削(3回加工)(0:off, 3:on) & 0\\\hline
\ttNum423 & 溝 仕上げ前 一時停止 (0:non-stop, 1:\verb|M00|, 2:扉前\verb|M00|) & 0\\\hline
\ttNum424 & 溝 仕上げ加工 追加回数 (上限3) & 0
\end{3columnstable}


\clearpage
%%%%%%%%%%%%%%%%%%%%%%%%%%%%%%%%%%%%%%%%%%%%%%%%%%%%%%%%%%
%% subsection 18.3.3 %%%%%%%%%%%%%%%%%%%%%%%%%%%%%%%%%%%%%
%%%%%%%%%%%%%%%%%%%%%%%%%%%%%%%%%%%%%%%%%%%%%%%%%%%%%%%%%%
\subsection{\ttNum425\,-\ttNum449}

\begin{3columnstable}[white]{\ttNum425\,-\ttNum449:加工時の調整(続き)}{|Sc|Sl|Sc|}{番号}{内容}{初期値}
\ttNum425 & トップ外面取$X+$補正 & 0\\\hline
\ttNum426 & トップ外面取 仕上げ前 一時停止 (0:non-stop, 1:\verb|M00|, 2:扉前\verb|M00|) & 0\\\hline
\ttNum427 & トップ外面取 仕上げ加工 追加回数 (上限3) & 0\\\hline
\rowcolor{unusingVariables}
\ttNum428 & (不使用) &\\\hline
\ttNum429 & トップ内面取$X+$補正 & 0\\\hline
\ttNum430 & トップ内面取 仕上げ前 一時停止 (0:non-stop, 1:\verb|M00|, 2:扉前\verb|M00|) & 0\\\hline
\ttNum431 & トップ内面取 仕上げ加工 追加回数 (上限3) & 0\\\hline
\rowcolor{unusingVariables}
\ttNum432 & (不使用) &\\\hline
\rowcolor{unusingVariables}
\ttNum433 & (座ぐり$X+$補正用 予備) &\\\hline
\rowcolor{unusingVariables}
\ttNum434 & (座ぐり 仕上げ前 一時停止用 予備) &\\\hline
\rowcolor{unusingVariables}
\ttNum435 & (座ぐり 仕上げ加工 追加回数用 予備) &\\\hline
\rowcolor{unusingVariables}
\ttNum436 & (不使用) &\\\hline
\ttNum437 & ボトム外削 A面肉厚$+$補正(外削中心$X+$補正) & 0\\\hline
\ttNum438 & ボトム外削 仕上げ前 一時停止 (0:non-stop, 1:\verb|M00|, 2:扉前\verb|M00|) & 0\\\hline
\ttNum439 & ボトム外削 仕上げ加工 追加回数 (上限3) & 0\\\hline
\rowcolor{unusingVariables}
\ttNum440 & (不使用) &\\\hline
\ttNum441 & ボトム外面取$X+$補正 & 0\\\hline
\ttNum442 & ボトム外面取 仕上げ前 一時停止 (0:non-stop, 1:\verb|M00|, 2:扉前\verb|M00|) & 0\\\hline
\ttNum443 & ボトム外面取 仕上げ加工 追加回数 (上限3) & 0\\\hline
\rowcolor{unusingVariables}
\ttNum444 & (不使用) &\\\hline
\ttNum445 & ボトム内面取$X+$補正 & 0\\\hline
\ttNum446 & ボトム内面取 仕上げ前 一時停止 (0:non-stop, 1:\verb|M00|, 2:扉前\verb|M00|) & 0\\\hline
\ttNum447 & ボトム内面取 仕上げ加工 追加回数 (上限3) & 0\\\hline
\rowcolor{unusingVariables}
\ttNum448 & (不使用) &\\\hline
\rowcolor{unusingVariables}
\ttNum449 & (予備) &
\end{3columnstable}


\clearpage
%%%%%%%%%%%%%%%%%%%%%%%%%%%%%%%%%%%%%%%%%%%%%%%%%%%%%%%%%%
%% subsection 18.3.2 %%%%%%%%%%%%%%%%%%%%%%%%%%%%%%%%%%%%%
%%%%%%%%%%%%%%%%%%%%%%%%%%%%%%%%%%%%%%%%%%%%%%%%%%%%%%%%%%
\subsection{\ttNum450\,-\ttNum474:\dimple 深さの調整}
%%%%%%%%%%%%%%%%%%%%%%%%%%%%%%%%%%%%%%%%%%%%%%%%%%%%%%%%%%
%% marker %%%%%%%%%%%%%%%%%%%%%%%%%%%%%%%%%%%%%%%%%%%%%%%%
%%%%%%%%%%%%%%%%%%%%%%%%%%%%%%%%%%%%%%%%%%%%%%%%%%%%%%%%%%
\begin{marker}
\ttNum450-\ttNum453の値は2024/01/16時点のもの
\end{marker}
%%%%%%%%%%%%%%%%%%%%%%%%%%%%%%%%%%%%%%%%%%%%%%%%%%%%%%%%%%
%%%%%%%%%%%%%%%%%%%%%%%%%%%%%%%%%%%%%%%%%%%%%%%%%%%%%%%%%%
%%%%%%%%%%%%%%%%%%%%%%%%%%%%%%%%%%%%%%%%%%%%%%%%%%%%%%%%%%

\begin{3columnstable}[white]{\ttNum450\,-\ttNum474:\dimple 深さの調整}{|Sc|Sl|Sc|}{番号}{内容}{設定値}
\ttNum450 & 工具\verb|T31|(Tスロット)A側\dimple~深さ補正値(深さに$+$補正) & 0.06\\\hline
\ttNum451 & 工具\verb|T31|(Tスロット)C側\dimple~深さ補正値(深さに$+$補正) & 0.03\\\hline
\ttNum452 & 工具\verb|T31|(Tスロット)B側\dimple~深さ補正値(深さに$+$補正) & 0.06\\\hline
\ttNum453 & 工具\verb|T31|(Tスロット)D側\dimple~深さ補正値(深さに$+$補正) & 0.03\\\hline
\rowcolor{unusingVariables}
\ttNum454 & (不使用) &\\\hline
\ttNum455 & 工具\verb|T32|(Tスロット)A側\dimple~深さ補正値(深さに$+$補正) &\\\hline
\ttNum456 & 工具\verb|T32|(Tスロット)C側\dimple~深さ補正値(深さに$+$補正) &\\\hline
\ttNum457 & 工具\verb|T32|(Tスロット)B側\dimple~深さ補正値(深さに$+$補正) &\\\hline
\ttNum458 & 工具\verb|T32|(Tスロット)D側\dimple~深さ補正値(深さに$+$補正) &\\\hline
\rowcolor{unusingVariables}
\ttNum459 & (不使用) &\\\hline
\ttNum460 & 工具\verb|T33|(Tスロット)A側\dimple~深さ補正値(深さに$+$補正) &\\\hline
\ttNum461 & 工具\verb|T33|(Tスロット)C側\dimple~深さ補正値(深さに$+$補正) &\\\hline
\ttNum462 & 工具\verb|T33|(Tスロット)B側\dimple~深さ補正値(深さに$+$補正) &\\\hline
\ttNum463 & 工具\verb|T33|(Tスロット)D側\dimple~深さ補正値(深さに$+$補正) &\\\hline
\rowcolor{unusingVariables}
\ttNum464 & (不使用) &\\\hline
\rowcolor{unusingVariables}
$\cdots$ & (以下予備) &
\end{3columnstable}



\clearpage
%%%%%%%%%%%%%%%%%%%%%%%%%%%%%%%%%%%%%%%%%%%%%%%%%%%%%%%%%%
%% section 18.4 %%%%%%%%%%%%%%%%%%%%%%%%%%%%%%%%%%%%%%%%%%
%%%%%%%%%%%%%%%%%%%%%%%%%%%%%%%%%%%%%%%%%%%%%%%%%%%%%%%%%%
\modHeadsection{\ttNum500\,-\ttNum574:バンドルプログラムの使用コモン変数}
\ttNum500\,-\ttNum574については、\index{バンドルのプログラム}バンドルのプログラム\prgbox{O910x}\prgbox{O93xx}で使用されており、作成したプログラムには用いていない。
%%%%%%%%%%%%%%%%%%%%%%%%%%%%%%%%%%%%%%%%%%%%%%%%%%%%%%%%%%
%% marker %%%%%%%%%%%%%%%%%%%%%%%%%%%%%%%%%%%%%%%%%%%%%%%%
%%%%%%%%%%%%%%%%%%%%%%%%%%%%%%%%%%%%%%%%%%%%%%%%%%%%%%%%%%
\begin{marker}
これらの値は2023/09/26時点のもの
\end{marker}
%%%%%%%%%%%%%%%%%%%%%%%%%%%%%%%%%%%%%%%%%%%%%%%%%%%%%%%%%%
%%%%%%%%%%%%%%%%%%%%%%%%%%%%%%%%%%%%%%%%%%%%%%%%%%%%%%%%%%
%%%%%%%%%%%%%%%%%%%%%%%%%%%%%%%%%%%%%%%%%%%%%%%%%%%%%%%%%%

\begin{3columnstable}[white]{\ttNum500\,-\ttNum524:\prgbox{O910x}\prgbox{O93xx}用}{|Sc|Sl|Sc|}{番号}{内容\hspace*{0.71\textwidth}}{設定値}
\ttNum500 & 芯ずれ許容差 \prgbox{O93xx} & 5\\\hline
\ttNum501 & タッチセンサー信号遅れ補正 \prgbox{O93xx} & 0.040\\\hline
\ttNum502 & タッチセンサープローブ中心$X$補正 \prgbox{O93xx} & -0.016507\\\hline
\ttNum503 & タッチセンサープローブ中心$Y$補正 \prgbox{O93xx} & -0.068371\\\hline
\ttNum504 & 測定距離 \prgbox{O910x} & 5\\\hline
\ttNum505 & プローブ表面からプログラムの加工原点($Z$0)までの距離 \prgbox{O910x} & 785.529\\\hline
\ttNum506 & 工具長の変化の許容差 \prgbox{O910x} & 1.0\\\hline
\ttNum507 & 工具破損検出の許容差 \prgbox{O910x} & 1.0\\\hline
\ttNum508 & (不明) & \ttNum0\\\hline
\ttNum509 & $Z$座標系設定 \prgbox{O93xx} & 441.432\\\hline
\ttNum510 & (不明) & 270\\\hline
\ttNum511 & インチ/ミリ切替 \prgbox{O910x} & \ttNum0\\\hline
\ttNum512 & タッチセンサープローブ半径$\mathrm{mm}$値 \prgbox{O93xx} & 5.0\\\hline
\ttNum513 & 移動時用の送り速さ値 \prgbox{O910x} & 1000\\\hline
\ttNum514 & スキップ(\verb|G31|)測定時用 送り速さ値 \prgbox{O910x}\prgbox{O93xx} & 50\\\hline
\ttNum515 & (不明) & \ttNum0\\\hline
\ttNum516 & センサーの位置$X$座標 \prgbox{O910x} & -30.374\\\hline
\ttNum517 & センサーの位置$Y$座標 \prgbox{O910x} & -913.761\\\hline
\ttNum518 & センサーの位置$Z$座標 \prgbox{O910x} & -785.529\\\hline
\ttNum519 & (不明) & 6\\\hline
\ttNum520 & 拡張ワーク座標系 \prgbox{O910x} & 1861\\\hline
\ttNum521 & (不明) & 0\\\hline
\ttNum522 & (不明) & 0\\\hline
\ttNum523 & アプローチ時用の送り速さ値 \prgbox{O910x} & 30\\\hline
\ttNum524 & 測定時用の送り速さ値 \prgbox{O910x} & 3\\\hline
$\cdots$ & (以下不明) & \ttNum0\\\hline
\ttNum533 & (不明) & 0\\\hline
\ttNum534 & (不明) & 0\\\hline
$\cdots$ & (以下不明) & \ttNum0
\end{3columnstable}
%%%%%%%%%%%%%%%%%%%%%%%%%%%%%%%%%%%%%%%%%%%%%%%%%%%%%%%%%%
%% marker %%%%%%%%%%%%%%%%%%%%%%%%%%%%%%%%%%%%%%%%%%%%%%%%
%%%%%%%%%%%%%%%%%%%%%%%%%%%%%%%%%%%%%%%%%%%%%%%%%%%%%%%%%%
\begin{marker}
\ttNum501, \ttNum502, \ttNum503, \ttNum504, \ttNum505, \ttNum507, \ttNum512, \ttNum516, \ttNum517は作成したプログラムでもRHSとして使用していることに注意
\end{marker}
%%%%%%%%%%%%%%%%%%%%%%%%%%%%%%%%%%%%%%%%%%%%%%%%%%%%%%%%%%
%%%%%%%%%%%%%%%%%%%%%%%%%%%%%%%%%%%%%%%%%%%%%%%%%%%%%%%%%%
%%%%%%%%%%%%%%%%%%%%%%%%%%%%%%%%%%%%%%%%%%%%%%%%%%%%%%%%%%



\clearpage
%%%%%%%%%%%%%%%%%%%%%%%%%%%%%%%%%%%%%%%%%%%%%%%%%%%%%%%%%%
%% section 11.5 %%%%%%%%%%%%%%%%%%%%%%%%%%%%%%%%%%%%%%%%%%
%%%%%%%%%%%%%%%%%%%%%%%%%%%%%%%%%%%%%%%%%%%%%%%%%%%%%%%%%%
\modHeadsection{\ttNum600\,-\ttNum699}


%%%%%%%%%%%%%%%%%%%%%%%%%%%%%%%%%%%%%%%%%%%%%%%%%%%%%%%%%%
%% subsection 11.5.1 %%%%%%%%%%%%%%%%%%%%%%%%%%%%%%%%%%%%%
%%%%%%%%%%%%%%%%%%%%%%%%%%%%%%%%%%%%%%%%%%%%%%%%%%%%%%%%%%
\subsection{\ttNum600\,-\ttNum624:ワークと工具間の距離の調整}
\ttNum600\,-\ttNum624については、主に製品と工具間の距離に関するものとする。\\

\begin{3columnstable}[white]{\ttNum600\,-\ttNum624:ワークと工具間の距離の調整}{|Sc|Sl|Sc|}{番号}{内容}{設定値}
\ttNum600 & 工具 - 端面間 $Z$方向クリアランス平面間距離 & 100.0\\\hline
\rowcolor{unusingVariables}
\ttNum601 & (予備) &\\\hline
\ttNum602 & タッチセンサー計測時の近付き量 & 9.0\\\hline
\ttNum603 & タッチセンサー計測時の行過ぎ量 & 3.0\\\hline
\rowcolor{unusingVariables}
\ttNum604 & (予備) &\\\hline
\ttNum605 & 外側加工面 法線方向クリアランス平面間距離 最小値 & 30.0\\\hline
\ttNum606 & 内側加工面 法線方向クリアランス平面間距離 最小値 & 15.0\\\hline
\rowcolor{unusingVariables}
\ttNum607 & (予備) &\\\hline
\ttNum608 & 端面加工用 内径輪郭径$-$補正量 & 5.0\\\hline
\rowcolor{unusingVariables}
\ttNum609 & (予備) &\\\hline
\rowcolor{unusingVariables}
\ttNum610 & (外削加工用予備) & \\\hline
\rowcolor{unusingVariables}
\ttNum611 & (予備) &\\\hline
\rowcolor{unusingVariables}
\ttNum612 & (溝加工用予備) & \\\hline
\rowcolor{unusingVariables}
\ttNum613 & (予備) &\\\hline
\rowcolor{unusingVariables}
\ttNum614 & (外面取加工用予備) & \\\hline
\rowcolor{unusingVariables}
\ttNum615 & (予備) &\\\hline
\rowcolor{unusingVariables}
\ttNum616 & (内面取加工用予備) &\\\hline
\rowcolor{unusingVariables}
\ttNum617 & (予備) &\\\hline
\rowcolor{unusingVariables}
\ttNum618 & (座ぐり加工用予備) &\\\hline
\rowcolor{unusingVariables}
\ttNum619 & (予備) &\\\hline
\ttNum620 & \dimple 加工用 加工近付き量 & 1.0 \\\hline
\rowcolor{unusingVariables}
$\cdots$ & (以下予備) &
\end{3columnstable}


\clearpage
%%%%%%%%%%%%%%%%%%%%%%%%%%%%%%%%%%%%%%%%%%%%%%%%%%%%%%%%%%
%% subsection 11.5.2 %%%%%%%%%%%%%%%%%%%%%%%%%%%%%%%%%%%%%
%%%%%%%%%%%%%%%%%%%%%%%%%%%%%%%%%%%%%%%%%%%%%%%%%%%%%%%%%%
\subsection{\ttNum625\,-\ttNum649:残り代および1回あたりの削り代}
\ttNum625\,-\ttNum649については、各加工の残り代や1回あたりの削り代に関するものとする。\\

\begin{3columnstable}[white]{\ttNum625\,-\ttNum649:残り代および1回あたりの削り代}{|Sc|Sl|Sc|}{番号}{内容}{設定値}
\ttNum625 & 端面加工1回あたりの$Z$方向削り代 & 4.0\\\hline
\rowcolor{unusingVariables}
\ttNum626 & (予備) & \\\hline
\ttNum627 & 外削加工1回あたりの削り代(直径) & 2.0\\\hline
\ttNum628 & 外削加工 仕上げ前 残り削り代(直径) & 1.0\\\hline
\ttNum629 & 溝加工1回あたりの削り代(溝深さ) & 5.0\\\hline
\ttNum630 & 溝加工 仕上げ前 残り削り代(直径) & 1.0\\\hline
\ttNum631 & 外面取加工1回あたりの削り代(直径) & 2.0\\\hline
\ttNum632 & 外面取加工 仕上げ前 残り削り代(直径) & 1.0\\\hline
\ttNum633 & 内面取加工1回あたりの削り代(直径) & 2.0\\\hline
\ttNum634 & 内面取加工 仕上げ前 残り削り代(直径) & 1.0\\\hline
\rowcolor{unusingVariables}
\ttNum635 & (座ぐり加工用予備) & \\\hline
\rowcolor{unusingVariables}
\ttNum636 & (座ぐり加工用予備) & \\\hline
\rowcolor{unusingVariables}
$\cdots$ & (以下予備) &
\end{3columnstable}


\clearpage
%%%%%%%%%%%%%%%%%%%%%%%%%%%%%%%%%%%%%%%%%%%%%%%%%%%%%%%%%%
%% subsection 11.5.2 %%%%%%%%%%%%%%%%%%%%%%%%%%%%%%%%%%%%%
%%%%%%%%%%%%%%%%%%%%%%%%%%%%%%%%%%%%%%%%%%%%%%%%%%%%%%%%%%
\subsection{\ttNum650\,-\ttNum674:工具の送り速さ}
\ttNum650\,-\ttNum674については、工具の送り速さに関するものとする。\\

\begin{3columnstable}[white]{\ttNum650\,-\ttNum674:工具の送り速さ\TBW}{|Sc|Sl|Sc|}{番号}{内容}{設定値}
\ttNum650 & 早送り$Z$:\verb|T50|アプローチ以外 & 12000\\\hline
\ttNum651 & アプローチ・$XY$リトラクト:\verb|T50|以外 & 6000\\\hline
\ttNum652 & 早送り$XY$:\verb|T50|, 早送り$Z$:\verb|T50|アプローチ & 5400\\\hline
\ttNum653 & アプローチ:\verb|T50| & 1500\\\hline
\ttNum654 & アプローチ:工具長測定 & 1000\\\hline
\ttNum655 & 測定時アプローチ:\verb|T50|(両側測定) & 200\\\hline
\ttNum656 & 測定時アプローチ:\verb|T50|(片側測定) & 50\\\hline
\rowcolor{unusingVariables}
\ttNum681 & (予備) &\\\hline
\ttNum682 & 端面加工:直線 & 900\\\hline
\ttNum683 & 端面加工:コーナー & 900\\\hline
\ttNum684 & 外削加工:直線 & 400\\\hline
\ttNum685 & 外削加工:コーナー & 400\\\hline
\ttNum686 & 溝加工:直線 & 500\\\hline
\ttNum687 & 溝加工:コーナー & 400\\\hline
\ttNum688 & 外面取加工:直線 & 500\\\hline
\ttNum689 & 外面取加工:コーナー & 400\\\hline
\ttNum690 & 内面取加工:直線 & 400\\\hline
\ttNum691 & 内面取加工:コーナー & 400\\\hline
\ttNum692 & 座ぐり加工:直線 & 40\\\hline
\ttNum693 & 座ぐり加工:コーナー & 40\\\hline
\ttNum694 & \dimple 加工:表面アプローチ & 540\\\hline
\ttNum695 & \dimple 加工:加工 & 100\\\hline
\rowcolor{unusingVariables}
$\cdots$ & (以下予備) &
\end{3columnstable}



\clearpage
%%%%%%%%%%%%%%%%%%%%%%%%%%%%%%%%%%%%%%%%%%%%%%%%%%%%%%%%%%
%% section 11.6 %%%%%%%%%%%%%%%%%%%%%%%%%%%%%%%%%%%%%%%%%%
%%%%%%%%%%%%%%%%%%%%%%%%%%%%%%%%%%%%%%%%%%%%%%%%%%%%%%%%%%
\modHeadsection{\ttNum700\,-\ttNum750:\dimple}
\ttNum700\,-\ttNum750については、主に\expandafterindex{\dimplekana そくていようサブプログラム@\dimple 測定用サブプログラム}\dimple 用サブプログラム\prgbox{O2x000x}で使用されるものとする。


%%%%%%%%%%%%%%%%%%%%%%%%%%%%%%%%%%%%%%%%%%%%%%%%%%%%%%%%%%
%% subsection 18.6.1 %%%%%%%%%%%%%%%%%%%%%%%%%%%%%%%%%%%%%
%%%%%%%%%%%%%%%%%%%%%%%%%%%%%%%%%%%%%%%%%%%%%%%%%%%%%%%%%%
\subsection{\ttNum700\,-\ttNum724:内面溝レベル1サブプログラム用}

\begin{2columnstable}[white]{\ttNum700\,-\ttNum724:\dimple~レベル1サブプログラム用 \DLone}{|Sc|Sl|}{番号}{内容}
\rowcolor{unusingVariables}
\ttNum700 & (予備)\\\hline
\ttNum701 & プログラム読込み時のワーク座標系(\ttNum4012)\\\hline
\ttNum702 & 工具別$Z$補正(\verb|T50|:\ttNum512, \verb|T3|x:0)\\\hline
\ttNum703 & 工具別$XY$補正(\verb|T50|:\ttNum512, \verb|T3|x:\ttNum[2400+\ttNum4111]+\ttNum[2600+\ttNum4111])\\\hline
\ttNum704 & 工具別移動\verb|G#| (\verb|T50|:31, \verb|T3|x:1)\\\hline
\ttNum705 & テーブル中心からワーク座標(\ttNum701)原点までの$X$距離\\\hline
\ttNum706 & 傾き後のトップ端面中心(機械座標)$X$ (\cf\pageeqref{eq:afterPhiTCenterFromO})\\\hline
\ttNum707 & テーブル中心から傾き後のトップ端面中心までの$Z$距離 (\cf\pageeqref{eq:afterPhiTCenterFromO})\\\hline
\ttNum708 & 傾き後トップ端中心(ブロックエンド)$X$座標(\ttNum5001)\\\hline
\ttNum709 & 傾き後トップ端中心(ブロックエンド)$Z$座標(\ttNum5003)\\\hline
\ttNum710 & テーブル中心から\dimple1列目までの$Z$距離$Z-q$\\\hline
\ttNum711 & トップ端中心から\dimple1列目中心までの$X$距離(\cf\pageeqref{eq:dimpleCenterDistance})\\\hline
\ttNum712 & 傾き後\dimple1列目中心$X$移動距離(\cf\pageeqref{eq:afterPhidimpleCenterDistance})\\\hline
\ttNum713 & 傾き後\dimple1列目中心$Z$移動距離(\cf\pageeqref{eq:afterPhidimpleCenterDistance})\\\hline
\ttNum714 & 傾き後\dimple1列目中心(ブロックエンド)$X$座標 (\ttNum5001)\\\hline
\ttNum715 & 傾き後\dimple1列目中心(ブロックエンド)$Y$座標 (\ttNum5002)\\\hline
\ttNum716 & 傾き後\dimple1列目中心(ブロックエンド)$Z$座標 (\ttNum5003)\\\hline
\ttNum717\color{red}$^*$ & 各面 ループ用数値(1:A, 2:C, 3:B, 4:D)\\\hline
\ttNum718 & BD内半径$-\text{\ttNum703}-10$\\\hline
\ttNum719 & (AC内半径$-\text{\ttNum703}-10)\cos\phi$\\\hline
\rowcolor{unusingVariables}
$\cdots$ & (以下予備)
\end{2columnstable}
%%%%%%%%%%%%%%%%%%%%%%%%%%%%%%%%%%%%%%%%%%%%%%%%%%%%%%%%%%
%% hosoku %%%%%%%%%%%%%%%%%%%%%%%%%%%%%%%%%%%%%%%%%%%%%%%%
%%%%%%%%%%%%%%%%%%%%%%%%%%%%%%%%%%%%%%%%%%%%%%%%%%%%%%%%%%
\begin{marker}
\ttNum717はレベル2サブプログラム\DLtwoAC および\DLtwoBD で\index{RHS(コモンへんすう)@RHS(コモン変数)}RHSとして使用していることに注意
\end{marker}
%%%%%%%%%%%%%%%%%%%%%%%%%%%%%%%%%%%%%%%%%%%%%%%%%%%%%%%%%%
%%%%%%%%%%%%%%%%%%%%%%%%%%%%%%%%%%%%%%%%%%%%%%%%%%%%%%%%%%
%%%%%%%%%%%%%%%%%%%%%%%%%%%%%%%%%%%%%%%%%%%%%%%%%%%%%%%%%%


\clearpage
%%%%%%%%%%%%%%%%%%%%%%%%%%%%%%%%%%%%%%%%%%%%%%%%%%%%%%%%%%
%% subsection 18.6.2 %%%%%%%%%%%%%%%%%%%%%%%%%%%%%%%%%%%%%
%%%%%%%%%%%%%%%%%%%%%%%%%%%%%%%%%%%%%%%%%%%%%%%%%%%%%%%%%%
\subsection{\ttNum725\,-\ttNum749:内面溝レベル2, 3サブプログラム用}

\begin{2columnstable}[white]{\ttNum725\,-\ttNum744:\dimple~レベル2サブプログラム用 \DLtwoAC\DLtwoBD}{|Sc|Sl|}{番号}{内容}
\ttNum725 & プログラム読込時ブロックエンド$Y$ or $X$ (\ttNum5002, \ttNum5001)\\\hline
\ttNum726 & プログラム読込時ブロックエンド$Z$ (\ttNum5003)\\\hline
\ttNum727 & \dimple~偶数列の列数\\\hline
\ttNum728 & \dimple~偶数列(一列)の\dimple~数\\\hline
\ttNum729 & \dimple~奇数列(一列)の\dimple~数\\\hline
\ttNum730 & \dimple~現在の列の\dimple~数\\\hline
\rowcolor{unusingVariables}
$\cdots$ & (以下予備)
\end{2columnstable}


%\clearpage
\begin{2columnstable}[white]{\ttNum745\,-\ttNum749:\dimple~測定用 \DMLthreeAC\DMLthreeBD}{|Sc|Sl|}{番号}{内容}
\rowcolor{unusingVariables}
\ttNum745 & (以下 予備)\\\hline
\rowcolor{unusingVariables}
$\cdots$ & \qquad$\cdots$\\\hline
\ttNum748 & プログラム読込時ブロックエンド$X$ or $Y$ (\ttNum5001, \ttNum5002)\\\hline
\ttNum749\color{red}$^*$ & \dimple~表面位置$X$ or $Y$測定値
\end{2columnstable}
%%%%%%%%%%%%%%%%%%%%%%%%%%%%%%%%%%%%%%%%%%%%%%%%%%%%%%%%%%
%% hosoku %%%%%%%%%%%%%%%%%%%%%%%%%%%%%%%%%%%%%%%%%%%%%%%%
%%%%%%%%%%%%%%%%%%%%%%%%%%%%%%%%%%%%%%%%%%%%%%%%%%%%%%%%%%
\begin{marker}
\ttNum749はレベル2サブプログラム\DLtwoAC および\DLtwoBD で\index{RHS(コモンへんすう)@RHS(コモン変数)}RHSとして使用していることに注意
\end{marker}
%%%%%%%%%%%%%%%%%%%%%%%%%%%%%%%%%%%%%%%%%%%%%%%%%%%%%%%%%%
%%%%%%%%%%%%%%%%%%%%%%%%%%%%%%%%%%%%%%%%%%%%%%%%%%%%%%%%%%
%%%%%%%%%%%%%%%%%%%%%%%%%%%%%%%%%%%%%%%%%%%%%%%%%%%%%%%%%%



\clearpage
%%%%%%%%%%%%%%%%%%%%%%%%%%%%%%%%%%%%%%%%%%%%%%%%%%%%%%%%%%
%% section 19.7 %%%%%%%%%%%%%%%%%%%%%%%%%%%%%%%%%%%%%%%%%%
%%%%%%%%%%%%%%%%%%%%%%%%%%%%%%%%%%%%%%%%%%%%%%%%%%%%%%%%%%
\modHeadsection{\ttNum900001\,-\ttNum900031, \ttNum900101\,-\ttNum900500:実測値\TBW}
\ttNum900001\,-\ttNum900500については、\index{じっそくち@実測値}実測値または計算値を格納する。


%%%%%%%%%%%%%%%%%%%%%%%%%%%%%%%%%%%%%%%%%%%%%%%%%%%%%%%%%%
%% subsection 19.7.1 %%%%%%%%%%%%%%%%%%%%%%%%%%%%%%%%%%%%%
%%%%%%%%%%%%%%%%%%%%%%%%%%%%%%%%%%%%%%%%%%%%%%%%%%%%%%%%%%
\subsection{\ttNum900001\,-\ttNum900049:\dimple 以外}

\begin{2columnstable}[white]{\ttNum900001\,-\ttNum900005:外中心$X$ 両側測定用 \MXOThickness}{|Sc|Sl|}{番号}{内容}
\ttNum900001 & $X$外中心測定 $-X$側測定値\\\hline
\ttNum900002 & $X$外中心測定 $+X$側測定値\\\hline
\ttNum900003 & $X$外中心測定値\\\hline
\ttNum900004 & $X$外中心測定 厚さ測定値\\\hline
\rowcolor{unusingVariables}
\ttNum900005 & (予備)\\
\end{2columnstable}


\begin{2columnstable}[white]{\ttNum900006\,-\ttNum900010:外中心$Y$ 両側測定用 \MYOThickness}{|Sc|Sl|}{番号}{内容}
\ttNum900006 & $Y$外中心測定 $-Y$側測定値\\\hline
\ttNum900007 & $Y$外中心測定 $+Y$側測定値\\\hline
\ttNum900008 & $Y$外中心測定値\\\hline
\ttNum900009 & $Y$外中心測定 厚さ測定値\\\hline
\rowcolor{unusingVariables}
\ttNum900010 & (予備)\\
\end{2columnstable}


%\clearpage
\begin{2columnstable}[white]{\ttNum900011\,-\ttNum900013:溝中心$X$ 片側測定用 \MXOface}{|Sc|Sl|}{番号}{内容}
\ttNum900011 & $X$溝中心測定 A側外面測定値\\\hline
\rowcolor{unusingVariables}
\ttNum900012 & (予備)\\\hline
\rowcolor{unusingVariables}
\ttNum900013 & (予備)\\
\end{2columnstable}


%\clearpage
\begin{2columnstable}[white]{\ttNum900014\,-\ttNum900018:内中心$X$ 両側測定用 \MXIWidth}{|Sc|Sl|}{番号}{内容}
\ttNum900014 & $X$内中心測定 $-X$側測定値\\\hline
\ttNum900015 & $X$内中心測定 $+X$側測定値\\\hline
\ttNum900016 & $X$内中心測定値\\\hline
\ttNum900017 & $X$内中心測定 厚さ測定値\\\hline
\rowcolor{unusingVariables}
\ttNum900018 & (予備)\\
\end{2columnstable}


\clearpage
\begin{2columnstable}[white]{\ttNum900019\,-\ttNum900023:内中心$Y$ 両側測定用 \MYIWidth}{|Sc|Sl|}{番号}{内容}
\ttNum900019 & $Y$内中心測定 $-Y$側測定値\\\hline
\ttNum900020 & $Y$内中心測定 $+Y$側測定値\\\hline
\ttNum900021 & $Y$内中心測定値\\\hline
\ttNum900022 & $Y$内中心測定 厚さ測定値\\\hline
\rowcolor{unusingVariables}
\ttNum900023 & (予備)\\
\end{2columnstable}


%\clearpage
\begin{2columnstable}[white]{\ttNum900024\,-\ttNum900026:外削中心$X$ 片側測定用 \MXIface}{|Sc|Sl|}{番号}{内容}
\ttNum900024 & $X$外削中心測定 内面測定値\\\hline
\rowcolor{unusingVariables}
\ttNum900025 & (予備)\\\hline
\rowcolor{unusingVariables}
\ttNum900026 & (予備)\\
\end{2columnstable}


%\clearpage
\begin{2columnstable}[white]{\ttNum900027\,-\ttNum900030:通り芯$Y$ 片側測定用 \MYcenterline}{|Sc|Sl|}{番号}{内容}
\ttNum900027 & $Y$通り芯 ボトム側測定値\\\hline
\ttNum900028 & $Y$通り芯 トップ側測定値\\\hline
\ttNum900029 & $Y$通り芯 測定値\\\hline
\rowcolor{unusingVariables}
\ttNum900030 & (予備)\\
\end{2columnstable}


\begin{2columnstable}[white]{\ttNum900031\,-\ttNum900034:通り芯$X$ 片側測定用 \MXcenterline}{|Sc|Sl|}{番号}{内容}
\ttNum900031 & $X$通り芯 トップ側測定値\\\hline
\ttNum900032 & $X$通り芯 ボトム側測定値\\\hline
\ttNum900033 & $X$通り芯 測定値\\\hline
\rowcolor{unusingVariables}
\ttNum900034 & (予備)\\
\end{2columnstable}


\clearpage
%%%%%%%%%%%%%%%%%%%%%%%%%%%%%%%%%%%%%%%%%%%%%%%%%%%%%%%%%%
%% subsection 18.7.2 %%%%%%%%%%%%%%%%%%%%%%%%%%%%%%%%%%%%%
%%%%%%%%%%%%%%%%%%%%%%%%%%%%%%%%%%%%%%%%%%%%%%%%%%%%%%%%%%
\subsection{\ttNum900101\,-\ttNum900500:\dimple}

%%%%%%%%%%%%%%%%%%%%%%%%%%%%%%%%%%%%%%%%%%%%%%%%%%%%%%%%%%
%% subsubsection 18.7.2.1 %%%%%%%%%%%%%%%%%%%%%%%%%%%%%%%%
%%%%%%%%%%%%%%%%%%%%%%%%%%%%%%%%%%%%%%%%%%%%%%%%%%%%%%%%%%
\subsubsection{\ttNum900101\,-\ttNum900300:\dimple AC面}

\begin{2columnstable}[white]{\ttNum900101\,-\ttNum900300:\dimple AC表面位置 測定値 \DMLthreeAC}{|Sc|Sl|}{番号}{内容}
\ttNum900101\,-\ttNum900200 & A側\dimple~表面位置$X$ 測定値\\\hline
\ttNum900201\,-\ttNum900300 & C側\dimple~表面位置$X$ 測定値
\end{2columnstable}

%%%%%%%%%%%%%%%%%%%%%%%%%%%%%%%%%%%%%%%%%%%%%%%%%%%%%%%%%%
%% subsubsection 18.7.2.2 %%%%%%%%%%%%%%%%%%%%%%%%%%%%%%%%
%%%%%%%%%%%%%%%%%%%%%%%%%%%%%%%%%%%%%%%%%%%%%%%%%%%%%%%%%%
\subsubsection{\ttNum900301\,-\ttNum900500:\dimple BD面}

\begin{2columnstable}[white]{\ttNum900301\,-\ttNum900500:\dimple BD表面位置 測定値 \DMLthreeBD}{|Sc|Sl|}{番号}{内容}
\ttNum900301\,-\ttNum900400 & B側\dimple~表面位置$Y$ 測定値\\\hline
\ttNum900401\,-\ttNum900500 & D側\dimple~表面位置$Y$ 測定値
\end{2columnstable}



\clearpage
%%%%%%%%%%%%%%%%%%%%%%%%%%%%%%%%%%%%%%%%%%%%%%%%%%%%%%%%%%
%% section 11.8 %%%%%%%%%%%%%%%%%%%%%%%%%%%%%%%%%%%%%%%%%%
%%%%%%%%%%%%%%%%%%%%%%%%%%%%%%%%%%%%%%%%%%%%%%%%%%%%%%%%%%
\modHeadsection{\ttNum901000\,-\ttNum901024:パレット・ジグ}
\ttNum901000\,-\ttNum901024については、主に\index{パレット}パレットや\index{ジグ}ジグに関するものとする。\\

\begin{3columnstable}[white]{\ttNum901000\,-\ttNum901024:主にパレット・ジグ}{|Sc|Sl|Sc|}{番号}{内容}{設定値}
\rowcolor{unusingVariables}
\ttNum901000 & (予備) &\\\hline
\ttNum901001 & パレット\ttNum1 ジグ中心機械座標$X$ & -550.019\\\hline
\ttNum901002 & パレット\ttNum1 ジグ中心機械座標$Y$ & -740.0\\\hline
\ttNum901003 & パレット\ttNum1 ジグ中心機械座標$Z$ & -1149.974\\\hline
\ttNum901004 & パレット\ttNum1 ジグ中心機械座標$B$ & 0.073\\\hline
\ttNum901005 & パレット\ttNum2 ジグ中心機械座標$X$ & -550.019\\\hline
\ttNum901006 & パレット\ttNum2 ジグ中心機械座標$Y$ & -740.0\\\hline
\ttNum901007 & パレット\ttNum2 ジグ中心機械座標$Z$ & -1149.974\\\hline
\ttNum901008 & パレット\ttNum2 ジグ中心機械座標$B$ & 0.073\\\hline
\ttNum901009 & 工具中心機械座標$C$ & 0\\\hline
\rowcolor{unusingVariables}
\ttNum901010 & (予備) &\\\hline
\hline
\ttNum901011 & パレット\ttNum1 ジグ外側幅$2l$(機械座標系$B$0における$Z$方向) & 660.0\\\hline
\ttNum901012 & パレット\ttNum1 ジグ内側幅(機械座標系$B$0における$Z$方向) & 410.0\\\hline
\ttNum901013 & パレット\ttNum1 ジグ幅(機械座標系$B$0における$X$方向) & 455.0\\\hline
\rowcolor{unusingVariables}
\ttNum901014 & (予備) &\\\hline
\ttNum901015 & パレット\ttNum2 ジグ外側幅$2l$(機械座標系$B$0における$Z$方向) & 660.0\\\hline
\ttNum901016 & パレット\ttNum2 ジグ内側幅(機械座標系$B$0における$Z$方向) & 410.0\\\hline
\ttNum901017 & パレット\ttNum2 ジグ幅(機械座標系$B$0における$X$方向) & 455.0\\\hline
\rowcolor{unusingVariables}
$\cdots$ & (以下予備) &
\end{3columnstable}
%%%%%%%%%%%%%%%%%%%%%%%%%%%%%%%%%%%%%%%%%%%%%%%%%%%%%%%%%%
%% hosoku %%%%%%%%%%%%%%%%%%%%%%%%%%%%%%%%%%%%%%%%%%%%%%%%
%%%%%%%%%%%%%%%%%%%%%%%%%%%%%%%%%%%%%%%%%%%%%%%%%%%%%%%%%%
\begin{hosoku}
\index{ジグのちゅうしん@ジグの中心}ジグ中心\index{きかいざひょう@機械座標}機械座標については2023/09/26時点のもの。
その他の(\index{ずめん@図面}図面上の)\index{すんぽう@寸法}寸法として、
\begin{enumerate}
\item テーブル中心 と C面側ジグ端 との水平距離:196.5
\item 受板の円の半径$\rho$:100
\item 受板の鉛直方向の幅$\sigma$:40
\item テーブル中心 と 受板の円の中心 との水平距離$\varDelta$:201.5
\item 受板の円の中心 と 受板の水平方向の底 との距離:70
\end{enumerate}
\end{hosoku}
%%%%%%%%%%%%%%%%%%%%%%%%%%%%%%%%%%%%%%%%%%%%%%%%%%%%%%%%%%
%%%%%%%%%%%%%%%%%%%%%%%%%%%%%%%%%%%%%%%%%%%%%%%%%%%%%%%%%%
%%%%%%%%%%%%%%%%%%%%%%%%%%%%%%%%%%%%%%%%%%%%%%%%%%%%%%%%%%



\clearpage
%%%%%%%%%%%%%%%%%%%%%%%%%%%%%%%%%%%%%%%%%%%%%%%%%%%%%%%%%%
%% section 11.9 %%%%%%%%%%%%%%%%%%%%%%%%%%%%%%%%%%%%%%%%%%
%%%%%%%%%%%%%%%%%%%%%%%%%%%%%%%%%%%%%%%%%%%%%%%%%%%%%%%%%%
\modHeadsection{\ttNum901100\,-\ttNum901149:工具}
\ttNum901100\,-\ttNum901149については、\index{こうぐ@工具}工具に関するもの(工具長や工具径およびその摩耗量を除く)とする。\\

\begin{3columnstable}[white]{\ttNum901100\,-\ttNum901149:工具}{|Sc|Sl|Sc|}{番号}{内容\hspace*{0.71\textwidth}}{設定値}
\rowcolor{unusingVariables}
\ttNum901100 & (予備) &\\\hline
\ttNum901101 & 工具\verb|T02|(フェイスミル)最大刃径(直径)DCX公称値$\phi'_\mathrm D$ & 113.5\\\hline
\rowcolor{unusingVariables}
$\cdots$ & (端面加工工具\verb|T02|-\verb|T05|用 予備) &\\\hline
\ttNum901105 & 工具\verb|T06|(サイドカッター)厚さ$t$ & 7.0\\\hline
\rowcolor{unusingVariables}
$\cdots$ & (溝加工工具\verb|T06|, \verb|T07|用 予備) &\\\hline
\ttNum901109 & 工具\verb|T08|(サイドカッター)厚さ$t$ & 5.0\\\hline
\rowcolor{unusingVariables}
$\cdots$ & (溝加工工具\verb|T08|, \verb|T10|用 予備) &\\\hline
\ttNum901113 & 工具\verb|T11|(テーパエンドミル)参照直径用 工具長補正値 & 2.0\\\hline
\rowcolor{unusingVariables}
\ttNum901114 & (工具\verb|T11|用 予備) &\\\hline
\ttNum901115 & 工具\verb|T12|(テーパエンドミル)参照直径用 工具長補正値 & 2.0\\\hline
\rowcolor{unusingVariables}
\ttNum901116 & (工具\verb|T12|用 予備) &\\\hline
\ttNum901117 & 工具\verb|T13|(テーパエンドミル)参照直径用 工具長補正値 & 2.0\\\hline
\rowcolor{unusingVariables}
$\cdots$ & (面取加工工具\verb|T13|-\verb|T15|用 予備) &\\\hline
\rowcolor{unusingVariables}
\ttNum901121 & (以下 外削加工工具用 予備) &\\\hline
\rowcolor{unusingVariables}
$\cdots$ & $\cdots$ &\\\hline
\ttNum901127 & 工具\verb|T31|(Tスロットカッター)厚さ & 8.0\\\hline
\ttNum901128 & 工具\verb|T31|(Tスロットカッター)シャンク直径(公称値) & 25.0\\\hline
\ttNum901129 & 工具\verb|T32|(Tスロットカッター)厚さ & 8.0\\\hline
\ttNum901130 & 工具\verb|T32|(Tスロットカッター)シャンク直径(公称値) & 25.0\\\hline
\rowcolor{unusingVariables}
$\cdots$ & (以下 \dimple 加工工具用 予備) &\\
\end{3columnstable}
%%%%%%%%%%%%%%%%%%%%%%%%%%%%%%%%%%%%%%%%%%%%%%%%%%%%%%%%%%
%% hosoku %%%%%%%%%%%%%%%%%%%%%%%%%%%%%%%%%%%%%%%%%%%%%%%%
%%%%%%%%%%%%%%%%%%%%%%%%%%%%%%%%%%%%%%%%%%%%%%%%%%%%%%%%%%
\begin{hosoku}
\index{タッチセンサープローブのじく@タッチセンサープローブの軸}タッチセンサープローブの軸の半径:3.75
\end{hosoku}
%%%%%%%%%%%%%%%%%%%%%%%%%%%%%%%%%%%%%%%%%%%%%%%%%%%%%%%%%%
%%%%%%%%%%%%%%%%%%%%%%%%%%%%%%%%%%%%%%%%%%%%%%%%%%%%%%%%%%
%%%%%%%%%%%%%%%%%%%%%%%%%%%%%%%%%%%%%%%%%%%%%%%%%%%%%%%%%%
%%%%%%%%%%%%%%%%%%%%%%%%%%%%%%%%%%%%%%%%%%%%%%%%%%%%%%%%%%
%% hosoku %%%%%%%%%%%%%%%%%%%%%%%%%%%%%%%%%%%%%%%%%%%%%%%%
%%%%%%%%%%%%%%%%%%%%%%%%%%%%%%%%%%%%%%%%%%%%%%%%%%%%%%%%%%
\begin{hosoku}
工具長・工具径・工具の摩耗量といったオフセットの値は、\pageautoref{sec:IV.A.2}を参照。
\end{hosoku}
%%%%%%%%%%%%%%%%%%%%%%%%%%%%%%%%%%%%%%%%%%%%%%%%%%%%%%%%%%
%%%%%%%%%%%%%%%%%%%%%%%%%%%%%%%%%%%%%%%%%%%%%%%%%%%%%%%%%%
%%%%%%%%%%%%%%%%%%%%%%%%%%%%%%%%%%%%%%%%%%%%%%%%%%%%%%%%%%



\clearpage
%%%%%%%%%%%%%%%%%%%%%%%%%%%%%%%%%%%%%%%%%%%%%%%%%%%%%%%%%%
%% section 18.10 %%%%%%%%%%%%%%%%%%%%%%%%%%%%%%%%%%%%%%%%%%
%%%%%%%%%%%%%%%%%%%%%%%%%%%%%%%%%%%%%%%%%%%%%%%%%%%%%%%%%%
\modHeadsection{未使用(使用可)のコモン変数}
この章の各々の表に載せていないコモン変数は(\dateUnusedVariables 時点において)未使用であり、新たに用いても問題ない(他のプログラムと競合しない)。
これらのコモン変数を、以下にまとめておく。
\begin{enumerate}
\item[-] \ttNum475-\ttNum499
\item[-] \ttNum575-\ttNum599
\item[-] \ttNum750-\ttNum999
\item[-] \ttNum900050-\ttNum900100
\item[-] \ttNum900501-\ttNum900999
\item[-] \ttNum901025-\ttNum901999
\item[-] \ttNum901150-\ttNum907399
\end{enumerate}


%\clearpage
\vfill
%%%%%%%%%%%%%%%%%%%%%%%%%%%%%%%%%%%%%%%%%%%%%%%%%%%%%%%%%%
%%%%%%%%%%%%%%%%%%%%%%%%%%%%%%%%%%%%%%%%%%%%%%%%%%%%%%%%%%
%%%%%%%%%%%%%%%%%%%%%%%%%%%%%%%%%%%%%%%%%%%%%%%%%%%%%%%%%%
\begin{tcolorbox}[title={2023/07/28時点の\MMname 実測値}, fonttitle=\gtfamily\bfseries]
\begin{align*}
  \text{Bot ($B=0$)}
  \left\{
  \begin{array}{rl}
    X: & 97.790 \sim 99.930\\
    Y: & -823.850\\
    Z: & -634.620
  \end{array}
  \right.\quad
  \text{Top ($B=180.$)}
  \left\{
  \begin{array}{rl}
    X: & -97.980 \sim -99.570\\
    Y: & -823.780\\
    Z: & -634.720
  \end{array}
  \right.
\end{align*}\\
・$X$については、ジグの当たる点の凸部と端部($Z$方向は目分量)\\
・$Y$については、モールドの底が当たる面\\
・$Z$については、$X0$ $Y-850.$における、ジグとの接点\\
※これらの値に、\index{タッチセンサーせんたん@タッチセンサー先端}タッチセンサー先端球の半径を加減する必要がある
\end{tcolorbox}
%%%%%%%%%%%%%%%%%%%%%%%%%%%%%%%%%%%%%%%%%%%%%%%%%%%%%%%%%%
%%%%%%%%%%%%%%%%%%%%%%%%%%%%%%%%%%%%%%%%%%%%%%%%%%%%%%%%%%
%%%%%%%%%%%%%%%%%%%%%%%%%%%%%%%%%%%%%%%%%%%%%%%%%%%%%%%%%%

%%!TEX root = ../RPA_for_Creating_Program_Note.tex


\modHeadchapter[]{コモン変数(\MMC)\TBW}
ここでは\MMC の加工システムで使用している\expandafterindex{コモンへんすう(\yomiMMC)@コモン変数(\nameMMC)}コモン変数について述べる。



%%%%%%%%%%%%%%%%%%%%%%%%%%%%%%%%%%%%%%%%%%%%%%%%%%%%%%%%%%
%% section 17.1 %%%%%%%%%%%%%%%%%%%%%%%%%%%%%%%%%%%%%%%%%%
%%%%%%%%%%%%%%%%%%%%%%%%%%%%%%%%%%%%%%%%%%%%%%%%%%%%%%%%%%
\modHeadsection{コモン変数 (コモン変数の範囲)}
(to be written...)



%\clearpage
\vfill
%%%%%%%%%%%%%%%%%%%%%%%%%%%%%%%%%%%%%%%%%%%%%%%%%%%%%%%%%%
%%%%%%%%%%%%%%%%%%%%%%%%%%%%%%%%%%%%%%%%%%%%%%%%%%%%%%%%%%
%%%%%%%%%%%%%%%%%%%%%%%%%%%%%%%%%%%%%%%%%%%%%%%%%%%%%%%%%%
\begin{tcolorbox}[title={2023/07/28時点の\MMC 実測値}, fonttitle=\gtfamily\bfseries]
\begin{align*}
  \text{Bot ($B=0$)}
  \left\{
  \begin{array}{rl}
    X: & 97.790 \sim 99.930\\
    Y: & -823.850\\
    Z: & -634.620
  \end{array}
  \right.\quad
  \text{Top ($B=180.$)}
  \left\{
  \begin{array}{rl}
    X: & -97.980 \sim -99.570\\
    Y: & -823.780\\
    Z: & -634.720
  \end{array}
  \right.
\end{align*}\\
・$X$については、ジグの当たる点の凸部と端部($Z$方向は目分量)\\
・$Y$については、モールドの底が当たる面\\
・$Z$については、$X0$ $Y-850.$における、ジグとの接点\\
※これらの値に、\index{タッチセンサープローブせんたんきゅう@タッチセンサープローブ先端球}タッチセンサープローブ先端球の半径を加減する必要がある
\end{tcolorbox}
%%%%%%%%%%%%%%%%%%%%%%%%%%%%%%%%%%%%%%%%%%%%%%%%%%%%%%%%%%
%%%%%%%%%%%%%%%%%%%%%%%%%%%%%%%%%%%%%%%%%%%%%%%%%%%%%%%%%%
%%%%%%%%%%%%%%%%%%%%%%%%%%%%%%%%%%%%%%%%%%%%%%%%%%%%%%%%%%

%!TEX root = ../RPA_for_Creating_Program_Note.tex


作成された著作物をオンライン上に公表することで、その開発や保守等における生産性が大きく向上することができる。
実際、\DMname におけるソフトウェアに関して、\index{バージョンかんり@バージョン管理}バージョン管理や\index{イシューかんり@イシュー管理}イシュー管理はオンライン上の\index{バージョンかんりシステム@バージョン管理システム}バージョン管理システム, \index{ソースコードかんりシステム@ソースコード管理システム}ソースコード管理(\index{SCM}SCM)システム, \index{リポジトリホスティングサービス}リポジトリホスティングサービスを用いて行われている
%% footnote %%%%%%%%%%%%%%%%%%%%%
\footnote{コードの共有・バージョン管理・イシュー管理・ビルド・テストなどの機能を用いることができ、生産性が大きく向上している。}。
%%%%%%%%%%%%%%%%%%%%%%%%%%%%%%%%%
一方で、サーバ停止等によるリスクや公表による情報漏洩等の\index{セキュリティ}セキュリティリスクも考えられる。

こうしたことを踏まえ、ここでは作成されたソフトウェア関連の\index{ちょさくぶつ@著作物}著作物の取り扱いについて述べる。



%%%%%%%%%%%%%%%%%%%%%%%%%%%%%%%%%%%%%%%%%%%%%%%%%%%%%%%%%%
%% section 20.1 %%%%%%%%%%%%%%%%%%%%%%%%%%%%%%%%%%%%%%%%%%
%%%%%%%%%%%%%%%%%%%%%%%%%%%%%%%%%%%%%%%%%%%%%%%%%%%%%%%%%%
\modHeadsection{関連する著作物}
マシニングセンタによるモールドの加工に対して作成された(ソフトウェア関連の)著作物として、主に以下のものが挙げられる。
\begin{enumerate}
\item 本書
\item 位置情報等の数値計算用プログラム
\item 使用スペーサ計算用プログラム
\item バンドルのプログラムを除いたNCプログラム
\item モールドのRDB
\item モールドの3次元CADモデリング用テンプレート
\item 内面テーパの3次元CADモデリング用テンプレート
\end{enumerate}
\href{https://elaws.e-gov.go.jp/document?lawid=345AC0000000048#Mp-At_2}{著作権法第2条}第1項(著作物の定義)\cite{eGovCopyrightLaw}より、これらは「思想又は感情を創作的に表現したもの」であり、著作権法の保護対象となる
%% footnote %%%%%%%%%%%%%%%%%%%%%
\footnote{なお、\href{https://elaws.e-gov.go.jp/document?lawid=345AC0000000048\#Mp-At_10}{著作権法第10条}第1項第9号(プログラムの著作物)\cite{eGovCopyrightLaw}より、プログラムは著作物の一種として明示的に規程されている。
プログラムの定義については\href{https://elaws.e-gov.go.jp/document?lawid=345AC0000000048\#Mp-At_2}{著作権法第2条}第1項第10号の2 \cite{eGovCopyrightLaw}を参照。}。
%%%%%%%%%%%%%%%%%%%%%%%%%%%%%%%%%



%%%%%%%%%%%%%%%%%%%%%%%%%%%%%%%%%%%%%%%%%%%%%%%%%%%%%%%%%%
%% section 20.2 %%%%%%%%%%%%%%%%%%%%%%%%%%%%%%%%%%%%%%%%%%
%%%%%%%%%%%%%%%%%%%%%%%%%%%%%%%%%%%%%%%%%%%%%%%%%%%%%%%%%%
\modHeadsection{関連著作物の著作権および著作権者}


%%%%%%%%%%%%%%%%%%%%%%%%%%%%%%%%%%%%%%%%%%%%%%%%%%%%%%%%%%
%% subsection 20.2.1 %%%%%%%%%%%%%%%%%%%%%%%%%%%%%%%%%%%%%
%%%%%%%%%%%%%%%%%%%%%%%%%%%%%%%%%%%%%%%%%%%%%%%%%%%%%%%%%%
\subsection{著作人格権}
\index{ちょさくけんほう@著作権法}\href{https://elaws.e-gov.go.jp/document?lawid=345AC0000000048\#Mp-At_59}{著作権法第59条}(著作者人格権の一身専属性)\cite{eGovCopyrightLaw}より、すべての\index{かんれんちょさくぶつ@関連著作物}関連著作物の\index{ちょさくじんかくけん@著作人格権}著作人格権は、その\index{ちょさくしゃ@著作者}著作者に帰属する。
また、原則として著作人格権の行使の判断・決定は、著作者に委れられるものとする。


%%%%%%%%%%%%%%%%%%%%%%%%%%%%%%%%%%%%%%%%%%%%%%%%%%%%%%%%%%
%% subsection 20.2.2 %%%%%%%%%%%%%%%%%%%%%%%%%%%%%%%%%%%%%
%%%%%%%%%%%%%%%%%%%%%%%%%%%%%%%%%%%%%%%%%%%%%%%%%%%%%%%%%%
\subsection{著作財産権}
\index{ちょさくけんほう@著作権法}\href{https://elaws.e-gov.go.jp/document?lawid=345AC0000000048#Mp-At_15}{著作権法第15条}(職務上作成する著作物の著作者)\cite{eGovCopyrightLaw}より、関連著作物が職務上作成された著作物(\index{しょくむちょさくぶつ@職務著作物}職務著作物)に該当する場合、その\index{ちょさくざいさんけん@著作財産権}著作財産権はその業務を指示した法人に帰属する。

関連著作物が職務著作物に該当しない場合、その著作財産権は著作者個人に帰属する。



\clearpage
%%%%%%%%%%%%%%%%%%%%%%%%%%%%%%%%%%%%%%%%%%%%%%%%%%%%%%%%%%
%% section 20.2 %%%%%%%%%%%%%%%%%%%%%%%%%%%%%%%%%%%%%%%%%%
%%%%%%%%%%%%%%%%%%%%%%%%%%%%%%%%%%%%%%%%%%%%%%%%%%%%%%%%%%
\modHeadsection{職務著作物}
\index{ちょさくけんほう@著作権法}\href{https://elaws.e-gov.go.jp/document?lawid=345AC0000000048\#Mp-At_15}{著作権法第15条}(職務上作成する著作物の著作者)より、作成された\index{ちょさくぶつ@著作物}著作物が以下の4つの要件をすべて満たす場合に限り、その著作物は\index{しょくむちょさくぶつ@職務著作物}職務著作物に該当する
\begin{enumerate}[label=\Roman*, ref=\Roman*]
\item 法人等の発意に基づくこと
\item 法人等の業務に従事する者が職務上作成するものであること
\item\label{item:copyrightrule3} 法人等の名義の下に公表するものであること
\item 作成の時における契約、勤務規則その他に別段の定めがないこと
\end{enumerate}
この4つの要件をもう少し具体的に述べると、
\begin{enumerate}[label=\Roman*$'$]
\item 法人等がある目的を持って構想した著作物の具体的な作成を従業員に命じることを意味する
\item その著作物が従業員の通常の業務範囲内で作成されたものであれば、それは依然として職務著作物に該当する可能性がある
\item その著作物が作成した業務従事者の名前で公表されれば職務著作物とは認められない
\item その著作物についての契約や勤務規則等に、別段の(適切な)定めがある場合は、その定めに従う
\end{enumerate}
ただし一般に、\ref{item:copyrightrule3}{}については、プログラムの著作物に関してはこの要件は不要とされる。



\clearpage
%%%%%%%%%%%%%%%%%%%%%%%%%%%%%%%%%%%%%%%%%%%%%%%%%%%%%%%%%%
%% section 20.4 %%%%%%%%%%%%%%%%%%%%%%%%%%%%%%%%%%%%%%%%%%
%%%%%%%%%%%%%%%%%%%%%%%%%%%%%%%%%%%%%%%%%%%%%%%%%%%%%%%%%%
\modHeadsection{関連著作物の公表}
\href{https://elaws.e-gov.go.jp/document?lawid=345AC0000000048\#Mp-At_18}{著作権法第18条}(\index{こうひょうけん@公表権}公表権)\cite{eGovCopyrightLaw}より、著作者は、その著作物でまだ公表されていないもの(その同意を得ないで公表された著作物を含む)を公衆に提供し、または提示する権利を有する(当該著作物を原著作物とする二次的著作物についても同様)。


%%%%%%%%%%%%%%%%%%%%%%%%%%%%%%%%%%%%%%%%%%%%%%%%%%%%%%%%%%
%% subsection 20.4.1 %%%%%%%%%%%%%%%%%%%%%%%%%%%%%%%%%%%%%
%%%%%%%%%%%%%%%%%%%%%%%%%%%%%%%%%%%%%%%%%%%%%%%%%%%%%%%%%%
\subsection{公表する関連著作物}

%%%%%%%%%%%%%%%%%%%%%%%%%%%%%%%%%%%%%%%%%%%%%%%%%%%%%%%%%%
%% subsubsection 20.4.2.1 %%%%%%%%%%%%%%%%%%%%%%%%%%%%%%%%
%%%%%%%%%%%%%%%%%%%%%%%%%%%%%%%%%%%%%%%%%%%%%%%%%%%%%%%%%%
\subsubsection{生産性の向上および著作権者の同意}
関連著作物については、開発・保守等の生産性の向上を目的に、原則としてすべてオンライン上に公表する。
ただし公表は、\href{https://elaws.e-gov.go.jp/document?lawid=345AC0000000048\#Mp-At_2}{著作権法第18条}に基づき、その著作物におけるすべての\index{ちょさくじんかくけん@著作人格権}著作人格権の保有者(\index{ちょさくしゃ@著作者}著作者)およびすべての\index{ちょさくざいさんけん@著作財産権}著作財産権の保有者の、全員の同意の下で行われることを前提とする。

なお、次節(非公表にする関連著作物)に該当する著作物に関してはその限りではない。

%%%%%%%%%%%%%%%%%%%%%%%%%%%%%%%%%%%%%%%%%%%%%%%%%%%%%%%%%%
%% subsubsection 20.4.2.2 %%%%%%%%%%%%%%%%%%%%%%%%%%%%%%%%
%%%%%%%%%%%%%%%%%%%%%%%%%%%%%%%%%%%%%%%%%%%%%%%%%%%%%%%%%%
\subsubsection{個人の著作権者\label{subsec:individualholder}}
著作人格権および著作財産権の保有者が同一人物であり、かつ1人の個人のみである場合は、その個人が公表の行為を行ってよいものとする。

%%%%%%%%%%%%%%%%%%%%%%%%%%%%%%%%%%%%%%%%%%%%%%%%%%%%%%%%%%
%% subsubsection 20.4.2.3 %%%%%%%%%%%%%%%%%%%%%%%%%%%%%%%%
%%%%%%%%%%%%%%%%%%%%%%%%%%%%%%%%%%%%%%%%%%%%%%%%%%%%%%%%%%
\subsubsection{データ保護とプライバシー}
公表の際は、\index{こじんじょうほうほごほう@個人情報保護法}\href{https://elaws.e-gov.go.jp/document?lawid=415AC0000000057}{個人情報の保護に関する法律}(個人情報保護法)\cite{eGovPersonalInfoProtectionLaw}に基づいて、データ保護・プライバシー保護に十分に配慮しなければならない。


%%%%%%%%%%%%%%%%%%%%%%%%%%%%%%%%%%%%%%%%%%%%%%%%%%%%%%%%%%
%% subsection 20.4.2 %%%%%%%%%%%%%%%%%%%%%%%%%%%%%%%%%%%%%
%%%%%%%%%%%%%%%%%%%%%%%%%%%%%%%%%%%%%%%%%%%%%%%%%%%%%%%%%%
\subsection{非公表にする関連著作物}

%%%%%%%%%%%%%%%%%%%%%%%%%%%%%%%%%%%%%%%%%%%%%%%%%%%%%%%%%%
%% subsubsection 20.4.2.1 %%%%%%%%%%%%%%%%%%%%%%%%%%%%%%%%
%%%%%%%%%%%%%%%%%%%%%%%%%%%%%%%%%%%%%%%%%%%%%%%%%%%%%%%%%%
\subsubsection{機密情報の保護および公表の範囲\label{subsec:notopenwork}}
たとえば作成したメインプログラムやモールドのデータベースについては、個々の明細の情報(機密事項)を推察できるデータを含む。
このような機密事項を含む著作物については、原則として公表しないものとする。
あるいは、公表する場合でも(個々のものすべてでなく)代表的なものに留めるものとする。

%%%%%%%%%%%%%%%%%%%%%%%%%%%%%%%%%%%%%%%%%%%%%%%%%%%%%%%%%%
%% subsubsection 20.4.2.2 %%%%%%%%%%%%%%%%%%%%%%%%%%%%%%%%
%%%%%%%%%%%%%%%%%%%%%%%%%%%%%%%%%%%%%%%%%%%%%%%%%%%%%%%%%%
\subsubsection{外部作成の著作物および著作権者の同意\label{subsec:copyrightsSubcontractor}}
プログラムの中には外注先で作成されたものも存在する。
このような著作権者(特に著作財産権者)が当社の従業員または当社自体ではない著作物については、原則として公表しない。
公表する場合は、すべての著作人格権者およびすべての著作財産権者の同意の下に行われるものとする。



\clearpage
~\vfill
\begin{Column}{\DMname の関連著作物}
\DMname の社内で作成されたソフトウェア関連著作物については、先にも述べた通りその一部がオンライン上に公開されている。
これは、その\DMname の立上げに関わる必要な(ソフトウェアにおける)社内の業務の一切が、関連著作物の作成開始時から作成終了後に至るまで、1人の一般職である著作者個人に(管理職・スタッフにより)一任されており、
\begin{enumerate}[label=\Roman*]
\item 法人等がある目的を持って構想した著作物は(少なくともソフトウェアに関しては)全く存在しない。
\item
著作者(一般職)は主に以下のような作業を行っており、通常の業務範囲および業務量から大きく逸脱しているのは明らかである。
  \begin{enumerate}
  \item[-] ソフトウェア開発に関わる諸規定の策定および作成
  \item[-] ソフトウェア開発に関わる諸標準の策定および作成
  \item[-] ソフトウェア開発に関わる業務フローの調査・要件定義・システム設計・詳細設計・実装・統合試運転を除く試運転・修正保守・機能追加保守等
  \item[-] 開発したソフトウェアに関わる技術書の作成
  \end{enumerate}
\item 関連著作物は著作者個人の名前あるいはアカウントの下に、オンライン上に公表されている。
\item ソフトウェア作成時において、その著作物についての別段の定め等は本章を除いて一切ない。
\end{enumerate}
したがって、いずれの要件も満たしていないため、その\index{ちょさくざいさんけん@著作財産権}著作財産権は著作者個人に帰属する。
関連著作物の一部が(開発・保守等の生産性の向上を目的に)オンライン上に公開されているのは、\pageautoref{subsec:individualholder}に基づくものである。
\end{Column}



%%%%%%%%%%%%%%%%%%%%%%%%%%%%%%%%%%%%%%%%%%%%%%%%%%%%%%%%%
%% Appendices %%%%%%%%%%%%%%%%%%%%%%%%%%%%%%%%%%%%%%%%%%%
%%%%%%%%%%%%%%%%%%%%%%%%%%%%%%%%%%%%%%%%%%%%%%%%%%%%%%%%%
\begin{appendices}
\Appendixpart
%!TEX root = ../RfCPN.tex


\modHeadchapter[loColumn,lot]{引数指定}



%%%%%%%%%%%%%%%%%%%%%%%%%%%%%%%%%%%%%%%%%%%%%%%%%%%%%%%%%%
%% section F.1 %%%%%%%%%%%%%%%%%%%%%%%%%%%%%%%%%%%%%%%%%%%
%%%%%%%%%%%%%%%%%%%%%%%%%%%%%%%%%%%%%%%%%%%%%%%%%%%%%%%%%%
\modHeadsection{引数の指定}
\index{ひきすう@引数}引数の指定の仕方は2通りある。
ここではそれらを\index{ひきすうしてい@引数指定}引数指定Iおよび引数指定IIと呼ぶことにする。


%%%%%%%%%%%%%%%%%%%%%%%%%%%%%%%%%%%%%%%%%%%%%%%%%%%%%%%%%%
%% subsection 10.4.1 %%%%%%%%%%%%%%%%%%%%%%%%%%%%%%%%%%%%%
%%%%%%%%%%%%%%%%%%%%%%%%%%%%%%%%%%%%%%%%%%%%%%%%%%%%%%%%%%
\subsection{\index{ひきすうしていI@引数指定I}引数指定I}
\index{ひきすうしていI@引数指定I}引数指定Iでは、\index{ひきすうアドレス@引数アドレス}引数アドレスとして{\ttfamily A}-{\ttfamily Z}のアルファベットをそれぞれ1回ずつ用いることができる。
そのため使用できる引数の数は、アルファベットの数(26個)である。
ただし、通常は{\ttfamily G}, {\ttfamily L}, {\ttfamily N}, {\ttfamily O}, {\ttfamily P}は用いることができず、実質的に21個が使用可能な数である。


%%%%%%%%%%%%%%%%%%%%%%%%%%%%%%%%%%%%%%%%%%%%%%%%%%%%%%%%%%
%% subsection 10.4.1 %%%%%%%%%%%%%%%%%%%%%%%%%%%%%%%%%%%%%
%%%%%%%%%%%%%%%%%%%%%%%%%%%%%%%%%%%%%%%%%%%%%%%%%%%%%%%%%%
\subsection{\index{ひきすうしていII@引数指定II}引数指定II}
\index{ひきすうしていII@引数指定II}引数指定IIでは、\index{ひきすうアドレス@引数アドレス}引数アドレスとして{\ttfamily A}, {\ttfamily B}, {\ttfamily C}を1回と{\ttfamily I}, {\ttfamily J}, {\ttfamily K}を10組まで用いることができる。
そのため使用できる引数の数は、33個である。

{\ttfamily A}, {\ttfamily B}, {\ttfamily C}にはそれぞれ\ttNum01, \ttNum02, \ttNum03が割り当てられ、{\ttfamily I}, {\ttfamily J}, {\ttfamily K}は入力の順序でマクロの\ttNum04から順番に\ttNum33まで割り当てられる
%% footnote %%%%%%%%%%%%%%%%%%%%%
\footnote{たとえば、{\ttfamily K}の後に{\ttfamily I}を用いた場合でも、{\ttfamily K}: {\ttfamily\ttNum04}, {\ttfamily I}: {\ttfamily\ttNum07}のようになることに注意。
これは\index{ひきすうしていI@引数指定I}引数指定Iでも同様である。}。
%%%%%%%%%%%%%%%%%%%%%%%%%%%%%%%%%



\clearpage
%%%%%%%%%%%%%%%%%%%%%%%%%%%%%%%%%%%%%%%%%%%%%%%%%%%%%%%%%%
%% section F.2 %%%%%%%%%%%%%%%%%%%%%%%%%%%%%%%%%%%%%%%%%%%
%%%%%%%%%%%%%%%%%%%%%%%%%%%%%%%%%%%%%%%%%%%%%%%%%%%%%%%%%%
\modHeadsection{\index{ひきすうアドレス@引数アドレス}引数アドレスとその\index{ローカルへんすう(ひきすう)@ローカル変数(引数)}ローカル変数}
原則として、引数の数が22個以上必要でない限り、\DMC では\index{ひきすうしていI@引数指定I}引数指定Iを用いるものとする。


%%%%%%%%%%%%%%%%%%%%%%%%%%%%%%%%%%%%%%%%%%%%%%%%%%%%%%%%%%
%% subsection C.2.1 %%%%%%%%%%%%%%%%%%%%%%%%%%%%%%%%%%%%%%
%%%%%%%%%%%%%%%%%%%%%%%%%%%%%%%%%%%%%%%%%%%%%%%%%%%%%%%%%%
\subsection{\index{ひきすうしていI@引数指定I}引数指定I 一覧}
\index{ひきすうアドレス(ひきすうしていI)@引数アドレス(引数指定I)}引数指定Iの引数アドレスとそれに対応する\index{ローカルへんすう(ひきすう)@ローカル変数(引数)}ローカル変数は以下の通りである。\\
\noindent%
\begin{minipage}[t]{0.66\textwidth}
%%%%%%%%%%%%%%%%%%%%%%%%%%%%%%%%%%%%%%%%%%%%%%%%%%%%%%%%%%
%% captionof %%%%%%%%%%%%%%%%%%%%%%%%%%%%%%%%%%%%%%%%%%%%%
%%%%%%%%%%%%%%%%%%%%%%%%%%%%%%%%%%%%%%%%%%%%%%%%%%%%%%%%%%
\begin{twocolbreaktblr}{\index{ひきすうしていI@引数指定I}引数指定I 一覧}{cc|[3pt, white!0!]|cc|[3pt, white!0!]|cc|[3pt, white!0!]|cc}
\cmidrule[r=0]{1-2}\cmidrule[lr=0]{3-4}\cmidrule[lr=0]{5-6}\cmidrule[l=0]{7-8}
記号 & 変数 & 記号 & 変数 & 記号 & 変数 & 記号 & 変数\\
\cmidrule[r=0]{1-2}\cmidrule[lr=0]{3-4}\cmidrule[lr=0]{5-6}\cmidrule[l=0]{7-8}
A & \ttfamily\#01 & H & \ttfamily\#11 & R & \ttfamily\#18 & X & \ttfamily\#24\\
\cmidrule[r=0]{1-2}\cmidrule[lr=0]{3-4}\cmidrule[lr=0]{5-6}\cmidrule[l=0]{7-8}
B & \ttfamily\#02 & I & \ttfamily\#04 & S & \ttfamily\#19 & Y & \ttfamily\#25\\
\cmidrule[r=0]{1-2}\cmidrule[lr=0]{3-4}\cmidrule[lr=0]{5-6}\cmidrule[l=0]{7-8}
C & \ttfamily\#03 & J & \ttfamily\#05 & T & \ttfamily\#20 & Z & \ttfamily\#26\\
\cmidrule[r=0]{1-2}\cmidrule[lr=0]{3-4}\cmidrule[lr=0]{5-6}\cmidrule[l=0]{7-8}
D & \ttfamily\#07 & K & \ttfamily\#06 & U & \ttfamily\#21\\
\cmidrule[r=0]{1-2}\cmidrule[lr=0]{3-4}\cmidrule[lr=0]{5-6}\cmidrule[l=0]{7-8}
E & \ttfamily\#08 & M & \ttfamily\#13 & V & \ttfamily\#22\\
\cmidrule[r=0]{1-2}\cmidrule[lr=0]{3-4}\cmidrule[lr=0]{5-6}\cmidrule[l=0]{7-8}
F & \ttfamily\#09 & Q & \ttfamily\#17 & W & \ttfamily\#23\\
\cmidrule[r=0]{1-2}\cmidrule[lr=0]{3-4}\cmidrule[lr=0]{5-6}\cmidrule[l=0]{7-8}
\end{twocolbreaktblr}%
\end{minipage}%
\begin{minipage}[t]{0.34\textwidth}
%%%%%%%%%%%%%%%%%%%%%%%%%%%%%%%%%%%%%%%%%%%%%%%%%%%%%%%%%%
%% captionof %%%%%%%%%%%%%%%%%%%%%%%%%%%%%%%%%%%%%%%%%%%%%
%%%%%%%%%%%%%%%%%%%%%%%%%%%%%%%%%%%%%%%%%%%%%%%%%%%%%%%%%%
\begin{twocolbreaktblr}{通常指定不可な引数}{cc}
\cmidrule{1-Z}
記号 & 変数\\
\cmidrule{1-Z}
G & \ttfamily\#10\\
\cmidrule{1-Z}
L & \ttfamily\#12\\
\cmidrule{1-Z}
N & \ttfamily\#14\\
\cmidrule{1-Z}
O & \ttfamily\#15\\
\cmidrule{1-Z}
P & \ttfamily\#16\\
\cmidrule{1-Z}
\end{twocolbreaktblr}%
\end{minipage}


%%%%%%%%%%%%%%%%%%%%%%%%%%%%%%%%%%%%%%%%%%%%%%%%%%%%%%%%%%
%% subsection C.2.2 %%%%%%%%%%%%%%%%%%%%%%%%%%%%%%%%%%%%%%
%%%%%%%%%%%%%%%%%%%%%%%%%%%%%%%%%%%%%%%%%%%%%%%%%%%%%%%%%%
\subsection{\index{ひきすうしていII@引数指定II}引数指定II 一覧}
\index{ひきすうアドレス(ひきすうしていII)@引数アドレス(引数指定II)}引数指定IIの引数アドレスとそれに対応する\index{ローカルへんすう(ひきすう)@ローカル変数(引数)}ローカル変数は以下の通りである。
なお、{\ttfamily I}, {\ttfamily J}, {\ttfamily K}の添字は便宜上記述したものであり、実際の\index{NCプログラム}NCプログラムにはすべて同じ{\ttfamily I}, {\ttfamily J}, {\ttfamily K}の記号を用いる。\\

%%%%%%%%%%%%%%%%%%%%%%%%%%%%%%%%%%%%%%%%%%%%%%%%%%%%%%%%%%
%% captionof %%%%%%%%%%%%%%%%%%%%%%%%%%%%%%%%%%%%%%%%%%%%%
%%%%%%%%%%%%%%%%%%%%%%%%%%%%%%%%%%%%%%%%%%%%%%%%%%%%%%%%%%
\begin{twocolbreaktblr}{\index{ひきすうしていII@引数指定II}引数指定II 一覧}{cc|[3pt, white!0!]|cc|[3pt, white!0!]|cc|[3pt, white!0!]|cc}
\cmidrule[r=0]{1-2}\cmidrule[lr=0]{3-4}\cmidrule[lr=0]{5-6}\cmidrule[l=0]{7-8}
記号 & 変数 & 記号 & 変数 & 記号 & 変数 & 記号 & 変数\\
\cmidrule[r=0]{1-2}\cmidrule[lr=0]{3-4}\cmidrule[lr=0]{5-6}\cmidrule[l=0]{7-8}
A & \ttfamily\#01 & I$_3$ & \ttfamily\#10 & I$_6$ & \ttfamily\#19 & I$_9$ & \ttfamily\#28\\
\cmidrule[r=0]{1-2}\cmidrule[lr=0]{3-4}\cmidrule[lr=0]{5-6}\cmidrule[l=0]{7-8}
B & \ttfamily\#02 & J$_3$ & \ttfamily\#11 & J$_6$ & \ttfamily\#20 & J$_9$ & \ttfamily\#29\\
\cmidrule[r=0]{1-2}\cmidrule[lr=0]{3-4}\cmidrule[lr=0]{5-6}\cmidrule[l=0]{7-8}
C & \ttfamily\#03 & K$_3$ & \ttfamily\#12 & K$_6$ & \ttfamily\#21 & K$_9$ & \ttfamily\#30\\
\cmidrule[r=0]{1-2}\cmidrule[lr=0]{3-4}\cmidrule[lr=0]{5-6}\cmidrule[l=0]{7-8}
I$_1$ & \ttfamily\#04 & I$_4$ & \ttfamily\#13 & I$_7$ & \ttfamily\#22 & I$_{10}$ & \ttfamily\#31\\
\cmidrule[r=0]{1-2}\cmidrule[lr=0]{3-4}\cmidrule[lr=0]{5-6}\cmidrule[l=0]{7-8}
J$_1$ & \ttfamily\#05 & J$_4$ & \ttfamily\#14 & J$_7$ & \ttfamily\#23 & J$_{10}$ & \ttfamily\#32\\
\cmidrule[r=0]{1-2}\cmidrule[lr=0]{3-4}\cmidrule[lr=0]{5-6}\cmidrule[l=0]{7-8}
K$_1$ & \ttfamily\#06 & K$_4$ & \ttfamily\#15 & K$_7$ & \ttfamily\#24 & K$_{10}$ & \ttfamily\#33\\
\cmidrule[r=0]{1-2}\cmidrule[lr=0]{3-4}\cmidrule[lr=0]{5-6}\cmidrule[l=0]{7-8}
I$_2$ & \ttfamily\#07 & I$_5$ & \ttfamily\#16 & I$_8$ & \ttfamily\#25\\
\cmidrule[r=0]{1-2}\cmidrule[lr=0]{3-4}\cmidrule[lr=0]{5-6}\cmidrule[l=0]{7-8}
J$_2$ & \ttfamily\#08 & J$_5$ & \ttfamily\#17 & J$_8$ & \ttfamily\#26\\
\cmidrule[r=0]{1-2}\cmidrule[lr=0]{3-4}\cmidrule[lr=0]{5-6}\cmidrule[l=0]{7-8}
K$_2$ & \ttfamily\#09 & K$_5$ & \ttfamily\#18 & K$_8$ & \ttfamily\#27\\
\cmidrule[r=0]{1-2}\cmidrule[lr=0]{3-4}\cmidrule[lr=0]{5-6}\cmidrule[l=0]{7-8}
\end{twocolbreaktblr}%


\clearpage
~\vfill
%%%%%%%%%%%%%%%%%%%%%%%%%%%%%%%%%%%%%%%%%%%%%%%%%%%%%%%%%%
%% Column %%%%%%%%%%%%%%%%%%%%%%%%%%%%%%%%%%%%%%%%%%%%%%%%
%%%%%%%%%%%%%%%%%%%%%%%%%%%%%%%%%%%%%%%%%%%%%%%%%%%%%%%%%%
\begin{\Columnname}{コード記述 注意点・ミスの例}
\begin{enumerate}
\item \ttNum\hx の付け忘れ
\item \verb|[ ]|の数の不整合
\item \verb|IF[...]THEN M00|
\item \verb|G.. GOTO...|
\item \verb|G90 G53 Z0|\\
      \verb|G91 G28 Z0|(\TLCorrection があるとエラー)
\item \verb|(...) O...|(最初の\index{プログラムばんごう@プログラム番号}プログラム番号の前に\index{コメント(NCプログラム)}コメントがあると\index{エラー}エラー)
\end{enumerate}
\tcbline*
\begin{enumerate}
\item
{\ttfamily G31}はすべての工具に適用できる。\\
ただし、\index{スキップきのう@スキップ機能}スキップ機能を有するのは\index{タッチセンサープローブ}タッチセンサープローブのみ。\\
また、タッチセンサープローブの\index{でんげん(タッチセンサープローブ)@電源(タッチセンサープローブ)}電源が入っていない状態で用いると工具は移動せず、\index{いどうかいしてん@移動開始点}移動開始点がそのまま\index{ブロックエンド}ブロックエンドとなり、\index{NCプログラム}NCプログラムは進行する。
\item {\ttfamily G00}では\verb|F|値の記述はできるが適用されない
\item {\ttfamily M98}で呼び出したNCプログラムは、レベル1の\index{かいそう(NCプログラム)@階層(NCプログラム)}階層(\index{メインプログラム}メインプログラム)として実行される。
\end{enumerate}
\end{\Columnname}
%%%%%%%%%%%%%%%%%%%%%%%%%%%%%%%%%%%%%%%%%%%%%%%%%%%%%%%%%%
%%%%%%%%%%%%%%%%%%%%%%%%%%%%%%%%%%%%%%%%%%%%%%%%%%%%%%%%%%
%%%%%%%%%%%%%%%%%%%%%%%%%%%%%%%%%%%%%%%%%%%%%%%%%%%%%%%%%%
\end{appendices}

\addtocontents{toc}{\protect\end{tocBox}}
