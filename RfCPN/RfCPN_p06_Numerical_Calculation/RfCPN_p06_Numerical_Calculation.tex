%!TEX root = ./RfCPN.tex


\addtocontents{toc}{\protect\cleardoublepage}
%%%%%%%%%%%%%%%%%%%%%%%%%%%%%%%%%%%%%%%%%%%%%%%%%%%%%%%%%
%% Part Numerical Calculation %%%%%%%%%%%%%%%%%%%%%%%%%%%
%%%%%%%%%%%%%%%%%%%%%%%%%%%%%%%%%%%%%%%%%%%%%%%%%%%%%%%%%
\addtocontents{toc}{\protect\begin{tocBox}{\tmppartnum}}%
\tPart{解析計算に基づく数値解析\label{part:NC}}{%
\paragraph*{\tpartgoal}
\index{めいさい(モールド)@明細(モールド)}明細ごとに異なる\index{すんぽう@寸法}寸法・形状を持つすべての\index{ワーク}ワークに対し、\index{NCプログラム}NCプログラムの作成に必要な数値情報および\index{じょうけんぶんきじょうほう@条件分岐情報}条件分岐情報等が自動的に得られるシステムを構築する。
\tcbline*
\paragraph*{\tpartmethod}
前段階で導出した解析的な情報を用いて、各明細における具体的な数値的な情報に自動的に変換するシステムの構築を試みる。
\tcbline*
\paragraph*{\tpartbackground}
一般に、ワークの形状や使用する工具は明細ごとに異なり、固有の寸法・形状を持つ。
NCプログラムの作成の際には、それらをすべて考慮した具体的な数値情報が必要となる。
こうした数値情報は明細ごとに膨大にあるが、現時点(\DMC 設置時点)において、こうした手続きは明細ごとに手作業で行われている。。

 したがって、こうした手続きのシステム化を行い、可能な限り自動化することが喫緊の課題である。
そうすることで、\textbf{危険を伴う作業の削減(\index{あんぜんせい@安全性}安全性の向上)}や、\textbf{品質の低下の防止}に大きく寄与できることが期待される。
また副次的効果として、作業効率・人的資源・\index{ほしゅ@保守}保守などのいずれの観点からみた\textbf{能率の低下の防止}にも大きく貢献することも自ずと期待される。
}{%
\paragraph*{\tpartconclusion}
各明細のワークにおける固有数値情報の入力により、\index{NCプログラム}NCプログラムの作成に必要な数値情報および\index{じょうけんぶんきじょうほう@条件分岐情報}条件分岐情報等が(手動計算を介することなく)自動的に得られるシステムを構築した。
\tcbline*
\paragraph*{\tpartnextstep}
加工システムの具体的な設計を行う。
}

%%%%%%%%%%%%%%%%%%%%%%%%%%%%%%%%%%%%%%%%%%%%%%%%%%%%%%%%%%
%% chapters %%%%%%%%%%%%%%%%%%%%%%%%%%%%%%%%%%%%%%%%%%%%%%
%%%%%%%%%%%%%%%%%%%%%%%%%%%%%%%%%%%%%%%%%%%%%%%%%%%%%%%%%%
%!TEX root = ./RfCPN.tex


\modHeadchapter[lot]{入力する数値情報・パラメタ}
\index{すんぽう@寸法}寸法・形状等の数値情報は、\index{めいさい(モールド)@明細(モールド)}明細ごとに固有である。
そのため、その\index{こゆうじょうほう(ワーク)@固有情報(ワーク)}固有情報は入力する必要がある。
ここではそうした入力する必要のある情報をまとめておく。
なお、\index{にゅうりょくするすうちじょうほう@入力する数値情報}入力する数値情報に関しては、原則として\index{ずめん@図面}図面上の\index{すんぽう@寸法}寸法をそのまま入力する形となるような方針とする
%% footnote %%%%%%%%%%%%%%%%%%%%%
\footnote{ただし、これらの値にはそれぞれの\index{こうさ@公差}公差が考慮されている。}。
%%%%%%%%%%%%%%%%%%%%%%%%%%%%%%%%%



%%%%%%%%%%%%%%%%%%%%%%%%%%%%%%%%%%%%%%%%%%%%%%%%%%%%%%%%%%
%% section 30.1 %%%%%%%%%%%%%%%%%%%%%%%%%%%%%%%%%%%%%%%%%%
%%%%%%%%%%%%%%%%%%%%%%%%%%%%%%%%%%%%%%%%%%%%%%%%%%%%%%%%%%
\modHeadsection{湾曲・\Alocation に関する入力数値}

\begin{multicollongtblr}{入力情報:湾曲・\Alocation}{X[l]c}
内容 & 型\\
\CenterCurvatureExists & boolean\\
\CenterCurvatureRadius & float\\
\TopAlocationLength & float\\
\BottomAlocationLength & float\\
\end{multicollongtblr}



%%%%%%%%%%%%%%%%%%%%%%%%%%%%%%%%%%%%%%%%%%%%%%%%%%%%%%%%%%
%% section 30.2 %%%%%%%%%%%%%%%%%%%%%%%%%%%%%%%%%%%%%%%%%%
%%%%%%%%%%%%%%%%%%%%%%%%%%%%%%%%%%%%%%%%%%%%%%%%%%%%%%%%%%
\modHeadsection{外形・内形に関する入力数値}

\begin{multicollongtblr}{入力情報:外形}{X[l]c}
内容 & 型\\
\ACOD & float\\
\BDOD & float\\
\ODCornerR & float\\
\end{multicollongtblr}

\begin{multicollongtblr}{入力情報:内形}{X[l]c}
内容 & 型\\
\PlatingThk & float\\
\IDTaperTableNum & list\\
\end{multicollongtblr}
%%%%%%%%%%%%%%%%%%%%%%%%%%%%%%%%%%%%%%%%%%%%%%%%%%%%%%%%%%
%% marker %%%%%%%%%%%%%%%%%%%%%%%%%%%%%%%%%%%%%%%%%%%%%%%%
%%%%%%%%%%%%%%%%%%%%%%%%%%%%%%%%%%%%%%%%%%%%%%%%%%%%%%%%%%
\begin{marker}
\IDTaperTable には、\InnerCornerR の\index{すんぽう(\yomiInnerCornerR)@寸法(\nameInnerCornerR)}寸法情報も記載されているものとする。
\end{marker}
%%%%%%%%%%%%%%%%%%%%%%%%%%%%%%%%%%%%%%%%%%%%%%%%%%%%%%%%%%
%%%%%%%%%%%%%%%%%%%%%%%%%%%%%%%%%%%%%%%%%%%%%%%%%%%%%%%%%%
%%%%%%%%%%%%%%%%%%%%%%%%%%%%%%%%%%%%%%%%%%%%%%%%%%%%%%%%%%



\clearpage
%%%%%%%%%%%%%%%%%%%%%%%%%%%%%%%%%%%%%%%%%%%%%%%%%%%%%%%%%%
%% section 30.3 %%%%%%%%%%%%%%%%%%%%%%%%%%%%%%%%%%%%%%%%%%
%%%%%%%%%%%%%%%%%%%%%%%%%%%%%%%%%%%%%%%%%%%%%%%%%%%%%%%%%%
\modHeadsection{\Outcut に関する入力数値}

\begin{multicollongtblr}{入力情報:\BottomOutcut}{X[l]c}
内容 & 型\\
\BottomOutcutExists & boolean\\
\BottomOutcutTaperExists & boolean\\
\BottomCurvedOutcutExists & boolean\\
\BottomOutcutAsideThickness & float\\
\BottomOutcutACWidth & float\\
\BottomOutcutBDWidth & float\\
\BottomOutcutLength & float\\
\BottomOutcutCornerR & float\\
\end{multicollongtblr}

\begin{multicollongtblr}{入力情報:\TopOutcut}{X[l]c}
内容 & 型\\
\TopOutcutExists & boolean\\
\TopOutcutTaperExists & boolean\\
\TopCurvedOutcutExists & boolean\\
\TopOutcutAsideThickness & float\\
\TopOutcutACWidth & float\\
\TopOutcutBDWidth & float\\
\TopOutcutLength & float\\
\TopOutcutCornerR & float\\
\end{multicollongtblr}

\begin{multicollongtblr}{入力情報:両外削}{X[l]c}
内容 & 型\\
\OutcutCenter 基準 & enum\\
\CenterlineEndFaceDif & float\\
\end{multicollongtblr}



\clearpage
%%%%%%%%%%%%%%%%%%%%%%%%%%%%%%%%%%%%%%%%%%%%%%%%%%%%%%%%%%
%% section 30.4 %%%%%%%%%%%%%%%%%%%%%%%%%%%%%%%%%%%%%%%%%%
%%%%%%%%%%%%%%%%%%%%%%%%%%%%%%%%%%%%%%%%%%%%%%%%%%%%%%%%%%
\modHeadsection{\Keyway に関する入力数値}

\begin{multicollongtblr}{入力情報:\Keyway}{X[l]c}
内容 & 型\\
\KeywayCornerType & enum\\
\AsideKeywayDepth 指定 有無 & boolean\\
\KeywayACOD & float\\
\KeywayBDOD & float\\
\KeywayPos & float\\
\KeywayWidth & float\\
\AsideKeywayDepth & float\\
\KeywayCornerR & float\\
\KeywayCornerC & float\\
\end{multicollongtblr}



%\clearpage
%%%%%%%%%%%%%%%%%%%%%%%%%%%%%%%%%%%%%%%%%%%%%%%%%%%%%%%%%%
%% section 30.5 %%%%%%%%%%%%%%%%%%%%%%%%%%%%%%%%%%%%%%%%%%
%%%%%%%%%%%%%%%%%%%%%%%%%%%%%%%%%%%%%%%%%%%%%%%%%%%%%%%%%%
\modHeadsection{\Dimple に関する入力数値}

\begin{multicollongtblr}{入力情報:\Dimple}{X[l]c}
内容 & 型\\
\DimpleExists & boolean\\
トップ端と\DimpleFirstRow までの距離 & float\\
\DimpleVerticalPitch & float\\
\DimpleHorizontalPitch & float\\
\DimpleOddRowLength & float\\
\DimpleEvenRowLength & float\\
\DimpleRowNum & enum\\
\DimpleDepth & float\\
\DimpleRadius(工具小半径) & float\\
\end{multicollongtblr}



\clearpage
%%%%%%%%%%%%%%%%%%%%%%%%%%%%%%%%%%%%%%%%%%%%%%%%%%%%%%%%%%
%% section 30.6 %%%%%%%%%%%%%%%%%%%%%%%%%%%%%%%%%%%%%%%%%%
%%%%%%%%%%%%%%%%%%%%%%%%%%%%%%%%%%%%%%%%%%%%%%%%%%%%%%%%%%
\modHeadsection{\EndFaceChamfer に関する入力数値}


%%%%%%%%%%%%%%%%%%%%%%%%%%%%%%%%%%%%%%%%%%%%%%%%%%%%%%%%%%
%% subsection 30.6.1 %%%%%%%%%%%%%%%%%%%%%%%%%%%%%%%%%%%%%
%%%%%%%%%%%%%%%%%%%%%%%%%%%%%%%%%%%%%%%%%%%%%%%%%%%%%%%%%%
\subsection{\BottomEndFaceChamfer に関する入力情報}

\begin{multicollongtblr}{入力情報:\BottomEndFaceOutCChamfer}{X[l]c}
内容 & 型\\
\BottomEndFaceOutCChamferExists & boolean\\
\BottomEndFaceOutCChamferLength & float\\
\BottomEndFaceOutCChamferAngle & float\\
\end{multicollongtblr}

\begin{multicollongtblr}{入力情報:\BottomEndFaceOutRChamfer}{X[l]c}
内容 & 型\\
\BottomEndFaceOutRChamferExists & boolean\\
\BottomEndFaceOutRChamferRadius & float\\
\end{multicollongtblr}

\begin{multicollongtblr}{入力情報:\BottomEndFaceInCChamfer}{X[l]c}
内容 & 型\\
\BottomEndFaceInCChamferExists & boolean\\
\BottomEndFaceInCChamferLength & float\\
\BottomEndFaceInCChamferAngle & integer\\
\end{multicollongtblr}

\begin{multicollongtblr}{入力情報:\BottomEndFaceInRChamfer}{X[l]c}
内容 & 型\\
\BottomEndFaceInRChamferExsits & boolean\\
\BottomEndFaceInRChamferRadius & float\\
\end{multicollongtblr}



\clearpage
%%%%%%%%%%%%%%%%%%%%%%%%%%%%%%%%%%%%%%%%%%%%%%%%%%%%%%%%%%
%% subsection 30.6.2 %%%%%%%%%%%%%%%%%%%%%%%%%%%%%%%%%%%%%
%%%%%%%%%%%%%%%%%%%%%%%%%%%%%%%%%%%%%%%%%%%%%%%%%%%%%%%%%%
\subsection{\TopEndFaceChamfer に関する入力情報}

\begin{multicollongtblr}{入力情報:\TopEndFaceOutCChamfer}{X[l]c}
内容 & 型\\
\TopEndFaceOutCChamferExists & boolean\\
\TopEndFaceOutCChamferLength & float\\
\TopEndFaceOutCChamferAngle & integer\\
\end{multicollongtblr}

\begin{multicollongtblr}{入力情報:\TopEndFaceOutRChamfer}{X[l]c}
内容 & 型\\
\TopEndFaceOutRChamferExists & boolean\\
\TopEndFaceOutRChamferRadius & float\\
\end{multicollongtblr}

\begin{multicollongtblr}{入力情報:\TopEndFaceInCChamfer}{X[l]c}
内容 & 型\\
\TopEndFaceInCChamferExists & boolean\\
\TopEndFaceInCChamferLength & float\\
\TopEndFaceInCChamferAngle & integer\\
\end{multicollongtblr}

\begin{multicollongtblr}{入力情報:\TopEndFaceInRChamfer}{X[l]c}
内容 & 型\\
\TopEndFaceInRChamferExists & boolean\\
\TopEndFaceInRChamferRadius & float\\
\end{multicollongtblr}



%\clearpage
%%%%%%%%%%%%%%%%%%%%%%%%%%%%%%%%%%%%%%%%%%%%%%%%%%%%%%%%%%
%% section 30.6 %%%%%%%%%%%%%%%%%%%%%%%%%%%%%%%%%%%%%%%%%%
%%%%%%%%%%%%%%%%%%%%%%%%%%%%%%%%%%%%%%%%%%%%%%%%%%%%%%%%%%
\modHeadsection{\EndFaceBoring に関する入力情報}

\begin{multicollongtblr}{入力情報:\EndFaceBoring}{X[l]c}
内容 & 型\\
\EndFaceBoringExists & boolean\\
\EndFaceBoringWidth & float\\
\EndFaceBoringDepth & float\\
\EndFaceBoringCornerR & float\\
\EndFaceBoringLength & float\\
\end{multicollongtblr}



\clearpage
%%%%%%%%%%%%%%%%%%%%%%%%%%%%%%%%%%%%%%%%%%%%%%%%%%%%%%%%%%
%% section 30.6 %%%%%%%%%%%%%%%%%%%%%%%%%%%%%%%%%%%%%%%%%%
%%%%%%%%%%%%%%%%%%%%%%%%%%%%%%%%%%%%%%%%%%%%%%%%%%%%%%%%%%
\modHeadsection{\IncutBoring に関する入力情報}

\begin{multicollongtblr}{入力情報:\IncutBoring}{X[l]c}
内容 & 型\\
\IncutBoringExists & boolean\\
\IncutBoringACWidth & float\\
\IncutBoringBDWidth & float\\
\IncutBoringCornerR & float\\
\IncutBoringLength & float\\
\end{multicollongtblr}


\clearrightpage

%!TEX root = ./RfCPN.tex


\modHeadchapter[lot]{必要な条件分岐情報\TBW}
(to be written...)


%%%%%%%%%%%%%%%%%%%%%%%%%%%%%%%%%%%%%%%%%%%%%%%%%%%%%%%%%%
%% section 30.1 %%%%%%%%%%%%%%%%%%%%%%%%%%%%%%%%%%%%%%%%%%
%%%%%%%%%%%%%%%%%%%%%%%%%%%%%%%%%%%%%%%%%%%%%%%%%%%%%%%%%%
\modHeadsection{\TBW}
(to be written...)

\begin{multicollongtblr}{\TBW}{cX[l]}
条件分岐 & 内容\\
(to be written...) & (to be written..)\\
\end{multicollongtblr}


\clearrightpage

%!TEX root = ./RPA_for_Creating_Program_Note.tex


基本的に、\index{すうちじょうほう@数値情報}数値情報については数値計算用の言語を用いて行うため、その詳細は別ドキュメントに譲る。
ここでは各明細用の\index{メインプログラム}メインプログラムの記述に際して、\index{すうちけいさん@数値計算}数値計算に必要な部分をピックアップする。
なお、ここでは主に\DMname について述べるため、\index{スペーサ}スペーサに関するものは省略する。


%%%%%%%%%%%%%%%%%%%%%%%%%%%%%%%%%%%%%%%%%%%%%%%%%%%%%%%%%%
%% section 30.1 %%%%%%%%%%%%%%%%%%%%%%%%%%%%%%%%%%%%%%%%%%
%%%%%%%%%%%%%%%%%%%%%%%%%%%%%%%%%%%%%%%%%%%%%%%%%%%%%%%%%%
\modHeadsection{再振分長・再張出長・均等振分角の数値情報}
各パラメータを以下とする。
\begin{align*}
  \varDelta_x' = \varDelta_x+\sqrt{R_\mathrm i'-\bar l^2}\ , \quad
  R_\mathrm i' = R_\mathrm c-\frac{W_x}2-\rho\ ,\quad
  \bar l = l-\frac\sigma2\ ,\quad
  f_d = \frac{f_\mathrm B-f_\mathrm T}2\ .
\end{align*}


%%%%%%%%%%%%%%%%%%%%%%%%%%%%%%%%%%%%%%%%%%%%%%%%%%%%%%%%%%
%% subsection 30.1.1 %%%%%%%%%%%%%%%%%%%%%%%%%%%%%%%%%%%%%
%%%%%%%%%%%%%%%%%%%%%%%%%%%%%%%%%%%%%%%%%%%%%%%%%%%%%%%%%%
\subsection{再振分長}
\index{テーブル}テーブルを$-\theta$だけ回転させて調整したトップ・ボトム側の\index{さいふりわけちょう@再振分長}振分長$f'_\mathrm T$, $f'_\mathrm B$は、\pageeqref{eq:saifuriwake}より、
\begin{align*}
  \text{トップ側:}\quad
  & \HLbox{f_\mathrm T' = f_\mathrm T+\varDelta_x'\!\sin\theta}\ ,\\
  \text{ボトム側:}\quad
  & \HLbox{f_\mathrm B' = (f_\mathrm T+f_\mathrm B)-f_\mathrm T'}\ .
\end{align*}


%%%%%%%%%%%%%%%%%%%%%%%%%%%%%%%%%%%%%%%%%%%%%%%%%%%%%%%%%%
%% subsection 30.1.2 %%%%%%%%%%%%%%%%%%%%%%%%%%%%%%%%%%%%%
%%%%%%%%%%%%%%%%%%%%%%%%%%%%%%%%%%%%%%%%%%%%%%%%%%%%%%%%%%
\subsection{再張出長}
テーブルを$-\theta$だけ回転させた後の\index{ジグ}ジグ(長さ$2l$)からの\index{さいはりだしちょう@再張出長}張出長に換算すると、それぞれ
\begin{align*}
  \text{トップ側:}\quad
  & \HLbox{f_\mathrm T'-l}\ ,\\
  \text{ボトム側:}\quad
  & \HLbox{f_\mathrm B'-l}\ .
\end{align*}


%%%%%%%%%%%%%%%%%%%%%%%%%%%%%%%%%%%%%%%%%%%%%%%%%%%%%%%%%%
%% subsection 30.1.3 %%%%%%%%%%%%%%%%%%%%%%%%%%%%%%%%%%%%%
%%%%%%%%%%%%%%%%%%%%%%%%%%%%%%%%%%%%%%%%%%%%%%%%%%%%%%%%%%
\subsection{均等振分角}
$f'_\mathrm T$および$f'_\mathrm B$が均等になり、かつトップ側およびボトム側の\index{たんめん@端面}端面が$X$軸方向に平行になるときの\index{かたむきかく(ふりわけちょうせい)@傾き角(振分調整)}回転角$\theta_\mathrm T'$, $\theta_\mathrm B'$は、\pageeqref{eq:saifuriwakeangle}よりそれぞれ、
\begin{align*}
  \text{トップ側:}\quad
  & \HLbox{\theta_\mathrm T' = -\sin^{-1}\frac{f_d}{\varDelta_x'}}\ ,\\
  \text{ボトム側:}\quad
  & \HLbox{\theta_\mathrm B' = \pi-\sin^{-1}\frac{f_d}{\varDelta_x'}}\ .
\end{align*}



\clearpage
%%%%%%%%%%%%%%%%%%%%%%%%%%%%%%%%%%%%%%%%%%%%%%%%%%%%%%%%%%
%% section 30.2 %%%%%%%%%%%%%%%%%%%%%%%%%%%%%%%%%%%%%%%%%%
%%%%%%%%%%%%%%%%%%%%%%%%%%%%%%%%%%%%%%%%%%%%%%%%%%%%%%%%%%
\modHeadsection{原点設定の数値情報}


%%%%%%%%%%%%%%%%%%%%%%%%%%%%%%%%%%%%%%%%%%%%%%%%%%%%%%%%%%
%% subsection 30.2.1 %%%%%%%%%%%%%%%%%%%%%%%%%%%%%%%%%%%%%
%%%%%%%%%%%%%%%%%%%%%%%%%%%%%%%%%%%%%%%%%%%%%%%%%%%%%%%%%%
\subsection{ボトム側の外側中心\texorpdfstring{$X$}{X}}

%%%%%%%%%%%%%%%%%%%%%%%%%%%%%%%%%%%%%%%%%%%%%%%%%%%%%%%%%%
%% subsubsection 30.2.1.1 %%%%%%%%%%%%%%%%%%%%%%%%%%%%%%%%
%%%%%%%%%%%%%%%%%%%%%%%%%%%%%%%%%%%%%%%%%%%%%%%%%%%%%%%%%%
\subsubsection{ボトム端の外側中心\texorpdfstring{$X$}{X}}
\index{テーブルちゅうしん@テーブル中心}テーブル中心\index{P(テーブルちゅうしん)@P(テーブル中心)}Pを\index{げんてんP@原点P}原点とした、($-\theta$回転後の)\index{ボトムたんのそとがわちゅうしん@ボトム端の外側中心}ボトム端の外径中心の$X$位置は、\pageeqref{eq:tableBc}より、
\begin{align*}
  \HLbox{%
    \varDelta_x'\cos\theta
    -\frac{\sqrt{R_\mathrm o^2-f_\mathrm B^2}+\sqrt{R_\mathrm i^2-f_\mathrm B^2}}2%
  }\ .
\end{align*}

%%%%%%%%%%%%%%%%%%%%%%%%%%%%%%%%%%%%%%%%%%%%%%%%%%%%%%%%%%
%% subsubsection 30.2.1.2 %%%%%%%%%%%%%%%%%%%%%%%%%%%%%%%%
%%%%%%%%%%%%%%%%%%%%%%%%%%%%%%%%%%%%%%%%%%%%%%%%%%%%%%%%%%
\subsubsection{ボトム側の外削中心\texorpdfstring{$X$}{X}(ボトムA側肉厚基準)}
\index{テーブルちゅうしん@テーブル中心}テーブル中心\index{P(テーブルちゅうしん)@P(テーブル中心)}Pを\index{げんてんP@原点P}原点とした、($-\theta$回転後の)\index{ボトムAがわにくあつ@ボトムA側肉厚}ボトムA側肉厚を\index{きじゅん(ボトムAがわにくあつ)@基準(ボトムA側肉厚)}基準とした\index{ボトムがわのがいさくちゅうしん@ボトム側の外削中心}ボトム側の外削中心$\mathfrak B_\mathrm c'$の(おおよその)$X$座標は、\pageeqref{eq:gaisakucenterBt}より、
\begin{align*}
  \HLbox{%
    \varDelta_x'\cos\theta
    -\frac{\sqrt{R_\mathrm o^2-f_\mathrm B^2}+\sqrt{R_\mathrm i^2-f_\mathrm B^2}}2
    -\frac{w_\mathrm B}2
    -\tau_\mathrm B
    +\frac{\mathfrak W_\mathrm B}2
  }\ .
\end{align*}

%%%%%%%%%%%%%%%%%%%%%%%%%%%%%%%%%%%%%%%%%%%%%%%%%%%%%%%%%%
%% subsubsection 30.2.1.2 %%%%%%%%%%%%%%%%%%%%%%%%%%%%%%%%
%%%%%%%%%%%%%%%%%%%%%%%%%%%%%%%%%%%%%%%%%%%%%%%%%%%%%%%%%%
\subsubsection{ボトム側の外削中心\texorpdfstring{$X$}{X}(トップA側肉厚基準)}
\index{テーブルちゅうしん@テーブル中心}テーブル中心\index{P(テーブルちゅうしん)@P(テーブル中心)}Pを\index{げんてんP@原点P}原点とした、($-\theta$回転後の)\index{トップAがわにくあつ@トップA側肉厚}トップA側肉厚を\index{きじゅん(トップAがわにくあつ)@基準(トップA側肉厚)}基準とした\index{トップがわのがいさくちゅうしん@トップ側の外削中心}ボトム側の外削中心$\mathfrak B_\mathrm c'$の(おおよその)$X$座標は、\pageeqref{eq:gaisakucenterTt}より、
\begin{align*}
  \HLbox{%
    -\left(
      \frac{\sqrt{R_\mathrm o^2-f_\mathrm T^2}+\sqrt{R_\mathrm i^2-f_\mathrm T^2}}2
      -\varDelta_x'\cos\theta
      +\frac{w_\mathrm T}2
      +\tau_\mathrm T
      -\frac{\mathfrak W_\mathrm T}2
    \right)
    +T_x
  }\ .
\end{align*}


%%%%%%%%%%%%%%%%%%%%%%%%%%%%%%%%%%%%%%%%%%%%%%%%%%%%%%%%%%
%% subsection 30.2.2 %%%%%%%%%%%%%%%%%%%%%%%%%%%%%%%%%%%%%
%%%%%%%%%%%%%%%%%%%%%%%%%%%%%%%%%%%%%%%%%%%%%%%%%%%%%%%%%%
\subsection{ボトム側の内側中心\texorpdfstring{$X$}{X}}

%%%%%%%%%%%%%%%%%%%%%%%%%%%%%%%%%%%%%%%%%%%%%%%%%%%%%%%%%%
%% subsubsection 30.2.2.1 %%%%%%%%%%%%%%%%%%%%%%%%%%%%%%%%
%%%%%%%%%%%%%%%%%%%%%%%%%%%%%%%%%%%%%%%%%%%%%%%%%%%%%%%%%%
\subsubsection{ボトム端の湾曲中心\texorpdfstring{$X$}{X}}
\index{テーブルちゅうしん@テーブル中心}テーブル中心\index{P(テーブルちゅうしん)@P(テーブル中心)}Pを\index{げんてんP@原点P}原点とした、($-\theta$回転後の)\index{ボトムたんのわんきょくちゅうしん@ボトム端の湾曲中心}ボトム端の湾曲中心の$X$値は、\pageeqref{eq:tableBRc}より、
\begin{align*}
  \HLbox{\varDelta_x'\!\cos\theta-\sqrt{R_\mathrm c^2-f_\mathrm B^2}}~.
\end{align*}

%%%%%%%%%%%%%%%%%%%%%%%%%%%%%%%%%%%%%%%%%%%%%%%%%%%%%%%%%%
%% subsubsection 30.2.2.2 %%%%%%%%%%%%%%%%%%%%%%%%%%%%%%%%
%%%%%%%%%%%%%%%%%%%%%%%%%%%%%%%%%%%%%%%%%%%%%%%%%%%%%%%%%%
\subsubsection{ボトム端の内側中心\texorpdfstring{$X$}{X}}
\index{テーブルちゅうしん@テーブル中心}テーブル中心\index{P(テーブルちゅうしん)@P(テーブル中心)}Pを\index{げんてんP@原点P}原点とした、($-\theta$回転後の)\index{ボトムたんのうちがわちゅうしん@ボトム端の内側中心}ボトム端の内側中心は、ボトム端の湾曲中心をもって代用してもよいものとする。


\clearpage
%%%%%%%%%%%%%%%%%%%%%%%%%%%%%%%%%%%%%%%%%%%%%%%%%%%%%%%%%%
%% subsection 30.2.3 %%%%%%%%%%%%%%%%%%%%%%%%%%%%%%%%%%%%%
%%%%%%%%%%%%%%%%%%%%%%%%%%%%%%%%%%%%%%%%%%%%%%%%%%%%%%%%%%
\subsection{トップ側の外側中心\texorpdfstring{$X$}{X}}

%%%%%%%%%%%%%%%%%%%%%%%%%%%%%%%%%%%%%%%%%%%%%%%%%%%%%%%%%%
%% subsubsection 30.2.3.1 %%%%%%%%%%%%%%%%%%%%%%%%%%%%%%%%
%%%%%%%%%%%%%%%%%%%%%%%%%%%%%%%%%%%%%%%%%%%%%%%%%%%%%%%%%%
\subsubsection{トップ端の外側中心\texorpdfstring{$X$}{X}}
\index{テーブルちゅうしん@テーブル中心}テーブル中心\index{P(テーブルちゅうしん)@P(テーブル中心)}Pを\index{げんてんP@原点P}原点とした、($-\theta$回転後の\index{トップたんのそとがわちゅうしん@トップ端の外側中心}外側中心の$X$位置は、\pageeqref{eq:tableTc}より、
\begin{align*}
  \HLbox{%
    \frac{\sqrt{R_\mathrm o^2-f_\mathrm T^2}+\sqrt{R_\mathrm i^2-f_\mathrm T^2}}2-\varDelta_x'\cos\theta%
  }~.
\end{align*}

%%%%%%%%%%%%%%%%%%%%%%%%%%%%%%%%%%%%%%%%%%%%%%%%%%%%%%%%%%
%% subsubsection 30.2.3.2 %%%%%%%%%%%%%%%%%%%%%%%%%%%%%%%%
%%%%%%%%%%%%%%%%%%%%%%%%%%%%%%%%%%%%%%%%%%%%%%%%%%%%%%%%%%
\subsubsection{トップ側の外削中心\texorpdfstring{$X$}{X}(ボトムA側肉厚基準)}
\index{テーブルちゅうしん@テーブル中心}テーブル中心\index{P(テーブルちゅうしん)@P(テーブル中心)}Pを\index{げんてんP@原点P}原点とした、($-\theta$回転後の)\index{ボトムAがわにくあつ@ボトムA側肉厚}ボトムA側肉厚を\index{きじゅん(ボトムAがわにくあつ)@基準(ボトムA側肉厚)}基準とした\index{トップがわのがいさくちゅうしん@トップ側の外削中心}トップ側の外削中心$\mathfrak T_\mathrm c'$の(おおよその)$X$座標は、\pageeqref{eq:gaisakucenterBt}より、
\begin{align*}
  \HLbox{%
    -\left(
      \varDelta_x'\cos\theta
      -\frac{\sqrt{R_\mathrm o^2-f_\mathrm B^2}+\sqrt{R_\mathrm i^2-f_\mathrm B^2}}2
      -\frac{w_\mathrm B}2
      -\tau_\mathrm B
      +\frac{\mathfrak W_\mathrm B}2
    \right)
    +T_x
  }\ .
\end{align*}

%%%%%%%%%%%%%%%%%%%%%%%%%%%%%%%%%%%%%%%%%%%%%%%%%%%%%%%%%%
%% subsubsection 30.2.3.3 %%%%%%%%%%%%%%%%%%%%%%%%%%%%%%%%
%%%%%%%%%%%%%%%%%%%%%%%%%%%%%%%%%%%%%%%%%%%%%%%%%%%%%%%%%%
\subsubsection{トップ側の外削中心\texorpdfstring{$X$}{X}(トップA側肉厚基準)}
\index{テーブルちゅうしん@テーブル中心}テーブル中心\index{P(テーブルちゅうしん)@P(テーブル中心)}Pを\index{げんてんP@原点P}原点とした、($-\theta$回転後の)\index{トップAがわにくあつ@トップA側肉厚}トップA側肉厚を\index{きじゅん(トップAがわにくあつ)@基準(トップA側肉厚)}基準とした\index{トップがわのがいさくちゅうしん@トップ側の外削中心}トップ側の外削中心$\mathfrak T_\mathrm c'$の(おおよその)$X$座標は、\pageeqref{eq:gaisakucenterTt}より、
\begin{align*}
  \HLbox{%
    \frac{\sqrt{R_\mathrm o^2-f_\mathrm T^2}+\sqrt{R_\mathrm i^2-f_\mathrm T^2}}2
    -\varDelta_x'\cos\theta
    +\frac{w_\mathrm T}2
    +\tau_\mathrm T
    -\frac{\mathfrak W_\mathrm T}2
  }\ .
\end{align*}

%%%%%%%%%%%%%%%%%%%%%%%%%%%%%%%%%%%%%%%%%%%%%%%%%%%%%%%%%%
%% subsubsection 30.2.3.4 %%%%%%%%%%%%%%%%%%%%%%%%%%%%%%%%
%%%%%%%%%%%%%%%%%%%%%%%%%%%%%%%%%%%%%%%%%%%%%%%%%%%%%%%%%%
\subsubsection{溝中心\texorpdfstring{$X$}{X}(A側溝深さ基準・外削なし)}
\index{みぞいち@溝位置}溝位置$\kappa_p$, \index{みぞはば@溝幅}溝幅$\kappa_w$, \index{Aがわみぞふかさ@A側溝深さ}A側溝深さ$\kappa_d'$, \index{みぞACけい@溝AC径}溝AC径$W_{mx}$に対し、\index{みぞちゅうしん@溝中心}溝中心M$'$の$X$座標は\pageeqref{eq:mizocenterA}より、
\begin{gather*}
  \HLbox{%
    \sqrt{R_\mathrm o^2-\left(f_\mathrm T-\kappa_p-\frac{\kappa_w}2\right)^{\!2}}
    -\kappa_d
    -\frac{W_{mx}}2
    -\varDelta_x%
  }\\[9pt]
  \left(
  \kappa_d
  = \frac{2\kappa_d'-\kappa_w\sin\zeta}{1+\cos^2\zeta}\cos\zeta
    +\sqrt{R_\mathrm o^2-\left(f_\mathrm T-\kappa_p-\frac{\kappa_w}2\right)^{\!2}}
    -\sqrt{R_\mathrm o^2-\left(f_\mathrm T-\kappa_p\right)^2}
  \right).
\end{gather*}


\clearpage
%%%%%%%%%%%%%%%%%%%%%%%%%%%%%%%%%%%%%%%%%%%%%%%%%%%%%%%%%%
%% subsection 30.2.4 %%%%%%%%%%%%%%%%%%%%%%%%%%%%%%%%%%%%%
%%%%%%%%%%%%%%%%%%%%%%%%%%%%%%%%%%%%%%%%%%%%%%%%%%%%%%%%%%
\subsection{トップ側の内側中心\texorpdfstring{$X$}{X}}

%%%%%%%%%%%%%%%%%%%%%%%%%%%%%%%%%%%%%%%%%%%%%%%%%%%%%%%%%%
%% subsubsection 30.4.1.1 %%%%%%%%%%%%%%%%%%%%%%%%%%%%%%%%
%%%%%%%%%%%%%%%%%%%%%%%%%%%%%%%%%%%%%%%%%%%%%%%%%%%%%%%%%%
\subsubsection{トップ端の湾曲中心\texorpdfstring{$X$}{X}}
\index{テーブルちゅうしん@テーブル中心}テーブル中心\index{P(テーブルちゅうしん)@P(テーブル中心)}Pを\index{げんてんP@原点P}原点とした、($-\theta$回転後の)\index{トップたんのわんきょくちゅうしん@トップ端の湾曲中心}トップ端の湾曲中心の$X$値は、\pageeqref{eq:tableTRc}より、
\begin{align*}
  \HLbox{\sqrt{R_\mathrm c^2-f_\mathrm T^2}-\varDelta_x'\!\cos\theta}~.
\end{align*}

%%%%%%%%%%%%%%%%%%%%%%%%%%%%%%%%%%%%%%%%%%%%%%%%%%%%%%%%%%
%% subsubsection 30.4.1.2 %%%%%%%%%%%%%%%%%%%%%%%%%%%%%%%%
%%%%%%%%%%%%%%%%%%%%%%%%%%%%%%%%%%%%%%%%%%%%%%%%%%%%%%%%%%
\subsubsection{トップ端の内側中心\texorpdfstring{$X$}{X}}
\index{テーブルちゅうしん@テーブル中心}テーブル中心\index{P(テーブルちゅうしん)@P(テーブル中心)}Pを\index{げんてんP@原点P}原点とした、($-\theta$回転後の)\index{トップたんのうちがわちゅうしん@トップ端の内側中心}トップ端の内側中心は、トップ端の湾曲中心をもって代用してもよいものとする。


%%%%%%%%%%%%%%%%%%%%%%%%%%%%%%%%%%%%%%%%%%%%%%%%%%%%%%%%%%
%% subsection 30.2.1 %%%%%%%%%%%%%%%%%%%%%%%%%%%%%%%%%%%%%
%%%%%%%%%%%%%%%%%%%%%%%%%%%%%%%%%%%%%%%%%%%%%%%%%%%%%%%%%%
\subsection{外側中心・内側中心\texorpdfstring{$Y$}{Y}}
\index{ジグ}ジグの底の$Y$座標を$\varDelta_y$とすると、\index{そとがわちゅうしんY@外側中心$Y$}外側中心および\index{うちがわちゅうしんY@内側中心$Y$}内側中心$Y$座標は、
\begin{align*}
  \HLbox{\varDelta_y+\frac{W_y}2}\ .
\end{align*}

%!TEX root = ./RfCPN.tex


\modHeadchapter{各工程用\index{NCサブプログラム}NCサブプログラムに必要な数値情報}
ここでは主に各工程用の\index{NCサブプログラム}NCサブプログラムの記述に際して、\index{すうちけいさん@数値計算}数値計算の必要な部分をピックアップする。



%%%%%%%%%%%%%%%%%%%%%%%%%%%%%%%%%%%%%%%%%%%%%%%%%%%%%%%%%%
%% section 41.01 %%%%%%%%%%%%%%%%%%%%%%%%%%%%%%%%%%%%%%%%%
%%%%%%%%%%%%%%%%%%%%%%%%%%%%%%%%%%%%%%%%%%%%%%%%%%%%%%%%%%
\modHeadsection{\TBW}
(to be written...)

%%%%%%%%%%%%%%%%%%%%%%%%%%%%%%%%%%%%%%%%%%%%%%%%%%%%%%%%%
%% Appendiodes %%%%%%%%%%%%%%%%%%%%%%%%%%%%%%%%%%%%%%%%%%
%%%%%%%%%%%%%%%%%%%%%%%%%%%%%%%%%%%%%%%%%%%%%%%%%%%%%%%%%
\begin{appendices}
%\Appendixpart
\end{appendices}

\addtocontents{toc}{\protect\end{tocBox}}
