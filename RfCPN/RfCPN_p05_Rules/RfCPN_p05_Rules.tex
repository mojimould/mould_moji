%!TEX root = ../RfCPN.tex


\addtocontents{toc}{\protect\cleardoublepage}
%%%%%%%%%%%%%%%%%%%%%%%%%%%%%%%%%%%%%%%%%%%%%%%%%%%%%%%%%
%% Part Rules %%%%%%%%%%%%%%%%%%%%%%%%%%%%%%%%%%%%%%%%%%%
%%%%%%%%%%%%%%%%%%%%%%%%%%%%%%%%%%%%%%%%%%%%%%%%%%%%%%%%%
\addtocontents{toc}{\protect\begin{tocBox}{\tmppartnum}}%
\tPart{機械の稼働に向けた諸規程の策定}{%
\paragraph*{\tpartgoal}
\index{ソフトウェア}ソフトウェアおよび\index{ソフトウェアエンジニアリング}ソフトウェアエンジニアリングに関する\textbf{業務を管理状態に}おく。
\tcbline*
\paragraph*{\tpartmethod}
\index{よこがたマシニングセンタ@横型マシニングセンタ}横型マシニングセンタを用いる業務が関わる(ソフトウェア視点による)\textbf{規程の策定}を試みる。
\tcbline*
\paragraph*{\tpartbackground}
すべての業務において、何の決まりごともなく管理状態におくことは不可能である。

 しかし現時点(\DMC 設置時点)において、「ソフトウェアエンジニアリングに関する管理・業務を行う部門」は存在しておらず、専任の担当者さえも存在しない。
そのため、\textbf{ソフトウェアエンジニアリング視点による規程さえ存在しない}
%% footnote %%%%%%%%%%%%%%%%%%%%%
\footnote{「ソフトウェア」を「\index{ハードウェア}ハードウェア」に置き換えると、これがどれほどの異常事態であるかがわかる。
安全・環境・品質に関する部門さえ全く関与していない状態にある。}。
%%%%%%%%%%%%%%%%%%%%%%%%%%%%%%%%%
事実、社内のほとんどの業務におけるシステム化や自動化について、その\index{ようけんていぎ@要件定義}\textbf{要件定義さえ行えない}状態が長く放置されている。

 (この業務に限らず)ソフトウェアに関する業務を内製化するにあたり、こうした視点による規程を設けることは不可欠の課題である。
}{%
\paragraph*{\tpartconclusion}
ソフトウェアの観点のもと、\DMC 関連の業務が関わる\index{ソフトウェアかいはつ@ソフトウェア開発}ソフトウェア開発・\index{ほしゅ@保守}保守等の規程を策定した。
\tcbline*
\paragraph*{\tpartnextstep}
策定した規程に基づき、マシニングセンタ関連のソフトウェア開発における諸標準の策定を行う。
}

%%%%%%%%%%%%%%%%%%%%%%%%%%%%%%%%%%%%%%%%%%%%%%%%%%%%%%%%%%
%% chapters %%%%%%%%%%%%%%%%%%%%%%%%%%%%%%%%%%%%%%%%%%%%%%
%%%%%%%%%%%%%%%%%%%%%%%%%%%%%%%%%%%%%%%%%%%%%%%%%%%%%%%%%%
%!TEX root = ../RPA_for_Creating_Program_Note.tex



%%%%%%%%%%%%%%%%%%%%%%%%%%%%%%%%%%%%%%%%%%%%%%%%%%%%%%%%%%
%% section 07.1 %%%%%%%%%%%%%%%%%%%%%%%%%%%%%%%%%%%%%%%%%%
%%%%%%%%%%%%%%%%%%%%%%%%%%%%%%%%%%%%%%%%%%%%%%%%%%%%%%%%%%
\modHeadsection{目的}
\begin{enumerate}
\item 情報処理技術者(利活用者を含む)として備えるべき能力についての水準を示すことにより、企業内教育等における水準の確保に資すること
\item 情報処理技術者(利活用者を含む)の起用を行う際に有用となる、客観的な評価の尺度を提供すること
\end{enumerate}


%%%%%%%%%%%%%%%%%%%%%%%%%%%%%%%%%%%%%%%%%%%%%%%%%%%%%%%%%%
%% section 07.2 %%%%%%%%%%%%%%%%%%%%%%%%%%%%%%%%%%%%%%%%%%
%%%%%%%%%%%%%%%%%%%%%%%%%%%%%%%%%%%%%%%%%%%%%%%%%%%%%%%%%%
\section{利用者および技術者の区分}

%%%%%%%%%%%%%%%%%%%%%%%%%%%%%%%%%%%%%%%%%%%%%%%%%%%%%%%%%%
%% subsection 07.2.1 %%%%%%%%%%%%%%%%%%%%%%%%%%%%%%%%%%%%%
%%%%%%%%%%%%%%%%%%%%%%%%%%%%%%%%%%%%%%%%%%%%%%%%%%%%%%%%%%
\subsection{ITを利活用する者}
\begin{enumerate}[label*=\Roman*., ref=\Roman*]
\item\label{item:ITseg1}
職業人として備えておくべきITに関する共通的な基礎知識をもち、担当業務に対してITを活用していこうとする者
\item\label{item:ITseg2}
担当業務の遂行に必要な情報セキュリティ対策を適切に理解し、情報および情報システムを安全に活用するために、情報セキュリティが確保された状況を実現し、維持・改善する者
\end{enumerate}

%%%%%%%%%%%%%%%%%%%%%%%%%%%%%%%%%%%%%%%%%%%%%%%%%%%%%%%%%%
%% subsection 07.2.2 %%%%%%%%%%%%%%%%%%%%%%%%%%%%%%%%%%%%%
%%%%%%%%%%%%%%%%%%%%%%%%%%%%%%%%%%%%%%%%%%%%%%%%%%%%%%%%%%
\subsection{情報処理技術者}
\begin{enumerate}[start=3, label*=\Roman*., ref=\Roman*]
\item\label{item:ITseg3}
ITを活用したサービス・製品・システムおよびソフトウェアを作る人材に必要な基本的知識・技能をもち、実践的な活用能力を身に付けた者
\item\label{item:ITseg4}
ITを活用したサービス・製品・システムおよびソフトウェアを作る人材に必要な応用的知識・技能をもち、高度IT人材としての方向性を確立した者
\end{enumerate}


\clearpage
%%%%%%%%%%%%%%%%%%%%%%%%%%%%%%%%%%%%%%%%%%%%%%%%%%%%%%%%%%
%% section 07.3 %%%%%%%%%%%%%%%%%%%%%%%%%%%%%%%%%%%%%%%%%%
%%%%%%%%%%%%%%%%%%%%%%%%%%%%%%%%%%%%%%%%%%%%%%%%%%%%%%%%%%
\section{期待される技術水準}
上記の区分\ref{item:ITseg1}, \ref{item:ITseg2}, \ref{item:ITseg3}, \ref{item:ITseg4}\hx に該当する者は、それぞれ以下のような技術水準を有することが期待される。

%%%%%%%%%%%%%%%%%%%%%%%%%%%%%%%%%%%%%%%%%%%%%%%%%%%%%%%%%%
%% subsection 07.3.1 %%%%%%%%%%%%%%%%%%%%%%%%%%%%%%%%%%%%%
%%%%%%%%%%%%%%%%%%%%%%%%%%%%%%%%%%%%%%%%%%%%%%%%%%%%%%%%%%
\subsection{区分\ref{item:ITseg1}}
\begin{enumerate}
\item 利用する情報機器やシステムを把握するために、コンピュータシステム・データベース・ネットワーク・情報セキュリティ・情報デザイン・情報メディアに関する知識をもち、オフィスツールを活用できる
\item 担当業務の問題把握および必要な解決を図るためにデータを利活用し、システム的な考え方や論理的な思考力(プログラミング的思考力等)を有し、かつ問題分析および問題解決手法に関する知識をもつ
\item 安全に情報を収集し、効果的に活用するために、関連法規・情報セキュリティに関する各種規程・情報倫理に従って活動できる
\item 業務の分析やシステム化の支援を行うために、情報システムの開発・運用に関する知識をもつ
\item 新しい技術や新しい手法の概要に関する知識をもつ
\end{enumerate}

%%%%%%%%%%%%%%%%%%%%%%%%%%%%%%%%%%%%%%%%%%%%%%%%%%%%%%%%%%
%% subsection 07.3.2 %%%%%%%%%%%%%%%%%%%%%%%%%%%%%%%%%%%%%
%%%%%%%%%%%%%%%%%%%%%%%%%%%%%%%%%%%%%%%%%%%%%%%%%%%%%%%%%%
\subsection{区分\ref{item:ITseg2}}
\begin{enumerate}
\item 所属部門の情報セキュリティマネジメントの一部を独力で遂行できる
\item 情報セキュリティインシデントの発生またはそのおそれがあるときに、情報セキュリティリーダーとして適切に対処できる
\item IT全般に関する基本的な用語・内容を理解できる
\item 情報セキュリティ技術や情報セキュリティ諸規程に関する基本的な知識をもち、部門の情報セキュリティ対策の一部を独力、または上位者の指導のもとに実現できる
\item 情報セキュリティ機関や他の企業等の動向や事例を収集し、部門の環境への適用の必要性を評価できる
\end{enumerate}
なお、区分\ref{item:ITseg2}\hx に該当する者は、区分\ref{item:ITseg1}\hx の技術水準も有するものとする。

%%%%%%%%%%%%%%%%%%%%%%%%%%%%%%%%%%%%%%%%%%%%%%%%%%%%%%%%%%
%% subsection 07.3.3 %%%%%%%%%%%%%%%%%%%%%%%%%%%%%%%%%%%%%
%%%%%%%%%%%%%%%%%%%%%%%%%%%%%%%%%%%%%%%%%%%%%%%%%%%%%%%%%%
\subsection{区分\ref{item:ITseg3}}
\begin{enumerate}
\item IT全般に関する基本的事項を理解し、担当する業務に活用できる
\item 上位者の指導のもとに、IT戦略に関する予測・分析・評価に参加できる
\item 上位者の指導のもとに、システムまたはサービスの提案活動に参加できる
\item 上位者の指導のもとに、システムの企画・要件定義に参加できる
\item 上位者の指導のもとに、システムの設計・開発・運用が行える
\item 上位者の指導のもとに、ソフトウェアを設計できる
\item 上位者の方針を理解し、自らプログラムを作成できる
\end{enumerate}

\clearpage
%%%%%%%%%%%%%%%%%%%%%%%%%%%%%%%%%%%%%%%%%%%%%%%%%%%%%%%%%%
%% subsection 07.3.4 %%%%%%%%%%%%%%%%%%%%%%%%%%%%%%%%%%%%%
%%%%%%%%%%%%%%%%%%%%%%%%%%%%%%%%%%%%%%%%%%%%%%%%%%%%%%%%%%
\subsection{区分\ref{item:ITseg4}}
\begin{enumerate}
\item 経営戦略・IT戦略の策定に際して、経営者の方針を理解し、経営を取巻く外部環境を正確に捉え、動向や事例を収集できる
\item 経営戦略・IT戦略の評価に際して、定められたモニタリング指標に基づいて差異分析などを行える
\item システムまたはサービスの提案活動に際して、提案討議に参加し、提案書の一部を作成できる
\item システムの企画・要件定義, アーキテクチャの設計において、システムに対する要求を整理し適用できる技術の調査が行える
\item 運用管理チーム・オペレーションチーム等の一員として、担当分野におけるサービス提供と安定稼働の確保が行える
\item プロジェクトメンバーとして、プロジェクトマネージャまたはリーダーのもとでスコープ・予算・工程・品質等の管理ができる
\item 情報システム・ネットワーク・データベース・組込みシステム等の設計・開発・運用・保守において、上位者の方針を理解し、自ら技術的問題を解決できる
\end{enumerate}
なお、区分\ref{item:ITseg4}\hx に該当する者は、区分\ref{item:ITseg3}\hx の技術水準も有するものとする。


%!TEX root = ../RPA_for_Creating_Program_Note.tex



%%%%%%%%%%%%%%%%%%%%%%%%%%%%%%%%%%%%%%%%%%%%%%%%%%%%%%%%%%
%% section 12.1 %%%%%%%%%%%%%%%%%%%%%%%%%%%%%%%%%%%%%%%%%%
%%%%%%%%%%%%%%%%%%%%%%%%%%%%%%%%%%%%%%%%%%%%%%%%%%%%%%%%%%
\modHeadsection{開発プロセス}

%%%%%%%%%%%%%%%%%%%%%%%%%%%%%%%%%%%%%%%%%%%%%%%%%%%%%%%%%%
%% subsection 12.1.1 %%%%%%%%%%%%%%%%%%%%%%%%%%%%%%%%%%%%%
%%%%%%%%%%%%%%%%%%%%%%%%%%%%%%%%%%%%%%%%%%%%%%%%%%%%%%%%%%
\subsection{要件定義}
新規機能の開発または既存機能の改修を行う際は、要件定義を行い、要件定義書の作成を行うものとする。
\index{ようけんていぎ@要件定義}要件定義については関係者全員でレビューを実施し、必要に応じて要件定義書の更新を行う。

なおこのプロセスは、区分\ref{item:ITseg4}\hx に該当する者が行うものとする。

%%%%%%%%%%%%%%%%%%%%%%%%%%%%%%%%%%%%%%%%%%%%%%%%%%%%%%%%%%
%% subsection 12.1.2 %%%%%%%%%%%%%%%%%%%%%%%%%%%%%%%%%%%%%
%%%%%%%%%%%%%%%%%%%%%%%%%%%%%%%%%%%%%%%%%%%%%%%%%%%%%%%%%%
\subsection{基本設計}
要件定義書の内容に基づいて基本設計を行い、基本設計書の作成を行うものとする。
\index{きほんせっけい@基本設計}基本設計については関係者全員でレビューを実施し、必要に応じて基本設計書の更新を行う。

なおこのプロセスは、区分\ref{item:ITseg4}\hx に該当する者、あるいは区分\ref{item:ITseg4}\hx に該当する者の指導のもとで区分\ref{item:ITseg3}\hx に該当する者が行うものとする。

%%%%%%%%%%%%%%%%%%%%%%%%%%%%%%%%%%%%%%%%%%%%%%%%%%%%%%%%%%
%% subsection 12.1.3 %%%%%%%%%%%%%%%%%%%%%%%%%%%%%%%%%%%%%
%%%%%%%%%%%%%%%%%%%%%%%%%%%%%%%%%%%%%%%%%%%%%%%%%%%%%%%%%%
\subsection{詳細設計}
基本設計書の内容に基づいて詳細設計を行い、詳細設計書の作成を行うものとする。
詳細設計書を作成し、関係者全員でレビューを実施し、必要に応じて詳細設計書を更新する。

なおこのプロセスは、区分\ref{item:ITseg4}\hx に該当する者、あるいは区分\ref{item:ITseg4}\hx に該当する者の指導のもとで区分\ref{item:ITseg3}\hx に該当する者が行うものとする。

%%%%%%%%%%%%%%%%%%%%%%%%%%%%%%%%%%%%%%%%%%%%%%%%%%%%%%%%%%
%% subsection 12.1.4 %%%%%%%%%%%%%%%%%%%%%%%%%%%%%%%%%%%%%
%%%%%%%%%%%%%%%%%%%%%%%%%%%%%%%%%%%%%%%%%%%%%%%%%%%%%%%%%%
\subsection{コードの記述}
詳細設計書の内容に基づいて、コードの記述を行うものとする。
記述は、コーディングルールに従って記述を行う。

なおこのプロセスは、区分\ref{item:ITseg3}\hx に該当する者が行うものとする。

%%%%%%%%%%%%%%%%%%%%%%%%%%%%%%%%%%%%%%%%%%%%%%%%%%%%%%%%%%
%% subsection 12.1.5 %%%%%%%%%%%%%%%%%%%%%%%%%%%%%%%%%%%%%
%%%%%%%%%%%%%%%%%%%%%%%%%%%%%%%%%%%%%%%%%%%%%%%%%%%%%%%%%%
\subsection{コードレビュー}
コードの記述の完了に伴い、そのコードは\index{コードレビュー}コードレビューを受けるものとする。
レビューの結果にもと、必要があれば修正・改善を行う。

%%%%%%%%%%%%%%%%%%%%%%%%%%%%%%%%%%%%%%%%%%%%%%%%%%%%%%%%%%
%% subsection 12.1.6 %%%%%%%%%%%%%%%%%%%%%%%%%%%%%%%%%%%%%
%%%%%%%%%%%%%%%%%%%%%%%%%%%%%%%%%%%%%%%%%%%%%%%%%%%%%%%%%%
\subsection{テスト}
コードレビューを受けたコードについて、正しく動作することを確認するためにテストを行うものとする。
テストの方法は予め定めておき、その方法に従ってテストを実施する。
テスト結果に基づいて、必要があれば修正・改善を行う。

なおこのプロセスは、区分\ref{item:ITseg3}\hx に該当する者が行うものとする。

\clearpage
%%%%%%%%%%%%%%%%%%%%%%%%%%%%%%%%%%%%%%%%%%%%%%%%%%%%%%%%%%
%% subsection 12.1.7 %%%%%%%%%%%%%%%%%%%%%%%%%%%%%%%%%%%%%
%%%%%%%%%%%%%%%%%%%%%%%%%%%%%%%%%%%%%%%%%%%%%%%%%%%%%%%%%%
\subsection{テスト環境での動作確認}
テスト結果に問題のないことが確認された場合、コードを\index{テストかんきょう@テスト環境}テスト環境にデプロイし、動作の確認を実施する。
確認事項は予め定めておき、それに従って確認を実施する。
動作確認の結果に基づいて、必要があれば修正・改善を行う。

なおこのプロセスは、区分\ref{item:ITseg3}\hx に該当する者が行うものとする。

%%%%%%%%%%%%%%%%%%%%%%%%%%%%%%%%%%%%%%%%%%%%%%%%%%%%%%%%%%
%% subsection 12.1.8 %%%%%%%%%%%%%%%%%%%%%%%%%%%%%%%%%%%%%
%%%%%%%%%%%%%%%%%%%%%%%%%%%%%%%%%%%%%%%%%%%%%%%%%%%%%%%%%%
\subsection{本番環境へのリリース}
\index{テストかんきょう@テスト環境}テスト環境による動作に問題のないことが確認された場合、\index{ほんばんかんきょう@本番環境}本番環境にリリースを行うものとする。

なおこのプロセスは、区分\ref{item:ITseg4}\hx に該当する者、あるいは区分\ref{item:ITseg4}\hx に該当する者の指導のもとで区分\ref{item:ITseg3}\hx に該当する者が行うものとする。

%%%%%%%%%%%%%%%%%%%%%%%%%%%%%%%%%%%%%%%%%%%%%%%%%%%%%%%%%%
%% subsection 12.1.9 %%%%%%%%%%%%%%%%%%%%%%%%%%%%%%%%%%%%%
%%%%%%%%%%%%%%%%%%%%%%%%%%%%%%%%%%%%%%%%%%%%%%%%%%%%%%%%%%
\subsection{変更管理}
開発プロセス中に要件・設計・コード等が変更される場合、変更要求書を作成し、関係者全員でレビューを行うものとする。
変更要求書には、変更の理由・影響範囲・必要なリソース等を明記する。
変更が承認された場合、変更要求書に基づいて関連する文書やコードの更新を行う。

なお、区分\ref{item:ITseg4}\hx に該当する者が管理を行うものとする。

%%%%%%%%%%%%%%%%%%%%%%%%%%%%%%%%%%%%%%%%%%%%%%%%%%%%%%%%%%
%% subsection 12.1.10 %%%%%%%%%%%%%%%%%%%%%%%%%%%%%%%%%%%%
%%%%%%%%%%%%%%%%%%%%%%%%%%%%%%%%%%%%%%%%%%%%%%%%%%%%%%%%%%
\subsection{リスク管理}
プロジェクトの開始時に、\index{リスクかんり@リスク管理}リスク管理計画を行うものとする。
リスク管理計画には、潜在的なリスクの特定、リスクの評価、リスク対策の策定、リスクの監視と管理が含まれる。
リスク管理計画はプロジェクト期間中に定期的に見直し、更新を実施する。

なお、区分\ref{item:ITseg4}\hx に該当する者が管理を行うものとする。

%%%%%%%%%%%%%%%%%%%%%%%%%%%%%%%%%%%%%%%%%%%%%%%%%%%%%%%%%%
%% subsection 12.1.10 %%%%%%%%%%%%%%%%%%%%%%%%%%%%%%%%%%%%
%%%%%%%%%%%%%%%%%%%%%%%%%%%%%%%%%%%%%%%%%%%%%%%%%%%%%%%%%%
\subsection{ドキュメンテーション}
各々の業務において、その全体像を理解するために、適切な\index{ドキュメンテーション}ドキュメンテーションの作成を行うものとする。
これには、要件定義書・設計書・テスト計画・ユーザーマニュアル・作業手順書等が含まれる。
ドキュメンテーションは常に最新の状態を保ち、関係者が容易にアクセスできる場所に保存を行う。

なお、これらはそのプロジェクトの各段階の責任者に相当する者が作成し、区分\ref{item:ITseg4}\hx に該当する者が管理を行うものとする。

%%%%%%%%%%%%%%%%%%%%%%%%%%%%%%%%%%%%%%%%%%%%%%%%%%%%%%%%%%
%% subsection 12.1.11 %%%%%%%%%%%%%%%%%%%%%%%%%%%%%%%%%%%%
%%%%%%%%%%%%%%%%%%%%%%%%%%%%%%%%%%%%%%%%%%%%%%%%%%%%%%%%%%
\subsection{緊急時の対応}
本番環境で重大な問題が発生した際は、速やかに直属の上司に報告する。
報告を受けた上司はシステム管理者に速やかに連絡する。
システム管理者は速やかに問題の原因の特定を試み、適切な対応策の実施を行う。



\clearpage
%%%%%%%%%%%%%%%%%%%%%%%%%%%%%%%%%%%%%%%%%%%%%%%%%%%%%%%%%%
%% section 20.2 %%%%%%%%%%%%%%%%%%%%%%%%%%%%%%%%%%%%%%%%%%
%%%%%%%%%%%%%%%%%%%%%%%%%%%%%%%%%%%%%%%%%%%%%%%%%%%%%%%%%%
\modHeadsection{コードレビュー}
コードレビューとは、ソースコードの作成者とは別の人物がコードを詳しく調べて問題がないか検討することであり、プログラムの品質を維持するために欠かせない工程である。

%%%%%%%%%%%%%%%%%%%%%%%%%%%%%%%%%%%%%%%%%%%%%%%%%%%%%%%%%%
%% subsection 08.2.1 %%%%%%%%%%%%%%%%%%%%%%%%%%%%%%%%%%%%%
%%%%%%%%%%%%%%%%%%%%%%%%%%%%%%%%%%%%%%%%%%%%%%%%%%%%%%%%%%
\subsection{レビューおよびレビュアー}
新規に作成または修正したソースコードは、コードレビューを受けるものとする。
コードレビューを行う者(レビュアー)は、そのコードに関連する技術水準を有することが要求され、可能な限り作成者とは異なる他の開発者であることが望ましい。

なお、レビュアーの選定は区分\ref{item:ITseg3}\hx に該当する者が行うものとする。

%%%%%%%%%%%%%%%%%%%%%%%%%%%%%%%%%%%%%%%%%%%%%%%%%%%%%%%%%%
%% subsection 08.2.2 %%%%%%%%%%%%%%%%%%%%%%%%%%%%%%%%%%%%%
%%%%%%%%%%%%%%%%%%%%%%%%%%%%%%%%%%%%%%%%%%%%%%%%%%%%%%%%%%
\subsection{レビューの範囲}
コードレビューの対象となるコードは、新規に作成されたコード、修正されたコード、およびそれらのコードに直接影響を与える可能性のある既存のコードとする。
レビューの範囲は、レビュアーとコードの作成者が協議して決定する。

%%%%%%%%%%%%%%%%%%%%%%%%%%%%%%%%%%%%%%%%%%%%%%%%%%%%%%%%%%
%% subsection 12.3.2 %%%%%%%%%%%%%%%%%%%%%%%%%%%%%%%%%%%%%
%%%%%%%%%%%%%%%%%%%%%%%%%%%%%%%%%%%%%%%%%%%%%%%%%%%%%%%%%%
\subsection{コードレビューの承認}
コードレビューでは、以下のすべての観点について確認を行い、これらをすべて満たしている場合にのみ承認を行うものとする。
\begin{enumerate}
\item コーディングルールに従っていること
\item 意図したとおりの実装が行われていること
\item 不具合につながる部分が存在しないこと
\end{enumerate}

%%%%%%%%%%%%%%%%%%%%%%%%%%%%%%%%%%%%%%%%%%%%%%%%%%%%%%%%%%
%% subsection 12.3.2 %%%%%%%%%%%%%%%%%%%%%%%%%%%%%%%%%%%%%
%%%%%%%%%%%%%%%%%%%%%%%%%%%%%%%%%%%%%%%%%%%%%%%%%%%%%%%%%%
\subsection{フィードバックの提供}
レビュアーは、レビューの結果を明確に伝えるために、具体的なコメントや改善提案の提供を行う。
フィードバックは建設的であるべきであり、問題だけでなく良い点や改善の提案も含めることが推奨される。

%%%%%%%%%%%%%%%%%%%%%%%%%%%%%%%%%%%%%%%%%%%%%%%%%%%%%%%%%%
%% subsection 12.3.3 %%%%%%%%%%%%%%%%%%%%%%%%%%%%%%%%%%%%%
%%%%%%%%%%%%%%%%%%%%%%%%%%%%%%%%%%%%%%%%%%%%%%%%%%%%%%%%%%
\subsection{コードレビューの時期}
開発プロセスにおけるコードレビューの実施タイミングは、予め定められたスケジュールに従うものとする。


\clearpage
%%%%%%%%%%%%%%%%%%%%%%%%%%%%%%%%%%%%%%%%%%%%%%%%%%%%%%%%%%
%% section 12.3 %%%%%%%%%%%%%%%%%%%%%%%%%%%%%%%%%%%%%%%%%%
%%%%%%%%%%%%%%%%%%%%%%%%%%%%%%%%%%%%%%%%%%%%%%%%%%%%%%%%%%
\modHeadsection{バージョン管理}
バージョン管理とは、ソースコードの変更履歴を追跡し、必要に応じて以前のバージョンに戻すことができるシステムである。
これにより複数の開発者による同時作業や、過去のバージョンに戻すこと等ができ、生産性の向上に大きく寄与する。

%%%%%%%%%%%%%%%%%%%%%%%%%%%%%%%%%%%%%%%%%%%%%%%%%%%%%%%%%%
%% subsection 12.3.1 %%%%%%%%%%%%%%%%%%%%%%%%%%%%%%%%%%%%%
%%%%%%%%%%%%%%%%%%%%%%%%%%%%%%%%%%%%%%%%%%%%%%%%%%%%%%%%%%
\subsection{バージョン管理の目的}
ソフトウェア開発の際は、ソースコードの変更履歴を追跡し、必要に応じて以前のバージョンに戻すことができるように、バージョン管理を行うものとする。

%%%%%%%%%%%%%%%%%%%%%%%%%%%%%%%%%%%%%%%%%%%%%%%%%%%%%%%%%%
%% subsection 12.3.2 %%%%%%%%%%%%%%%%%%%%%%%%%%%%%%%%%%%%%
%%%%%%%%%%%%%%%%%%%%%%%%%%%%%%%%%%%%%%%%%%%%%%%%%%%%%%%%%%
\subsection{バージョン管理の手順}
各々の文書またはソースコードは、それぞれ1つのバージョン管理システムを用いて管理を行うものとする。
新規に作成または修正したソースコードは、バージョン管理システムにコミットする。
各コミットには、変更内容を説明する文言を必ず含める。

%%%%%%%%%%%%%%%%%%%%%%%%%%%%%%%%%%%%%%%%%%%%%%%%%%%%%%%%%%
%% subsection 12.3.3 %%%%%%%%%%%%%%%%%%%%%%%%%%%%%%%%%%%%%
%%%%%%%%%%%%%%%%%%%%%%%%%%%%%%%%%%%%%%%%%%%%%%%%%%%%%%%%%%
\subsection{コミットおよびコードレビュー}
バージョン管理システムにコミットする前に、そのソースコードはコードレビューを受けるものとする。
コードレビューの結果に基づいて、必要に応じてソースコードの修正・改善を行う。
問題がないと判断された場合、コミットを行う。

%%%%%%%%%%%%%%%%%%%%%%%%%%%%%%%%%%%%%%%%%%%%%%%%%%%%%%%%%%
%% subsection 12.3.4 %%%%%%%%%%%%%%%%%%%%%%%%%%%%%%%%%%%%%
%%%%%%%%%%%%%%%%%%%%%%%%%%%%%%%%%%%%%%%%%%%%%%%%%%%%%%%%%%
\subsection{変更およびコミットの時期}
バージョン変更およびコミットの実施時期は、ソースコードの新規作成または修正が完了した時点に行うものとする。


\clearpage
%%%%%%%%%%%%%%%%%%%%%%%%%%%%%%%%%%%%%%%%%%%%%%%%%%%%%%%%%%
%% section 12.4 %%%%%%%%%%%%%%%%%%%%%%%%%%%%%%%%%%%%%%%%%%
%%%%%%%%%%%%%%%%%%%%%%%%%%%%%%%%%%%%%%%%%%%%%%%%%%%%%%%%%%
\modHeadsection{テスト}

%%%%%%%%%%%%%%%%%%%%%%%%%%%%%%%%%%%%%%%%%%%%%%%%%%%%%%%%%%
%% subsection 12.4.1 %%%%%%%%%%%%%%%%%%%%%%%%%%%%%%%%%%%%%
%%%%%%%%%%%%%%%%%%%%%%%%%%%%%%%%%%%%%%%%%%%%%%%%%%%%%%%%%%
\subsection{テストの目的}
ソフトウェア開発の際は、ソフトウェアが期待通りの動作をすることを確認するために、テストを行うものとする。

%%%%%%%%%%%%%%%%%%%%%%%%%%%%%%%%%%%%%%%%%%%%%%%%%%%%%%%%%%
%% subsection 12.4.2 %%%%%%%%%%%%%%%%%%%%%%%%%%%%%%%%%%%%%
%%%%%%%%%%%%%%%%%%%%%%%%%%%%%%%%%%%%%%%%%%%%%%%%%%%%%%%%%%
\subsection{テストケースの作成}
テストケースは、要件定義書および設計書に基づいて作成するものとする。
すべての機能・シナリオをカバーするように、テストケースを作成する。

なお、テストケースの作成は、品質管理の能力を有するものが行うものとする。

%%%%%%%%%%%%%%%%%%%%%%%%%%%%%%%%%%%%%%%%%%%%%%%%%%%%%%%%%%
%% subsection 12.4.3 %%%%%%%%%%%%%%%%%%%%%%%%%%%%%%%%%%%%%
%%%%%%%%%%%%%%%%%%%%%%%%%%%%%%%%%%%%%%%%%%%%%%%%%%%%%%%%%%
\subsection{テストの手順}
新規に作成または修正したソースコードは、テストを行うものとする。
テストは、予め定められたテスト計画に従って実施する。
テストの結果に基づいて、必要に応じてソースコードの修正・改善を行う。

%%%%%%%%%%%%%%%%%%%%%%%%%%%%%%%%%%%%%%%%%%%%%%%%%%%%%%%%%%
%% subsection 12.4.4 %%%%%%%%%%%%%%%%%%%%%%%%%%%%%%%%%%%%%
%%%%%%%%%%%%%%%%%%%%%%%%%%%%%%%%%%%%%%%%%%%%%%%%%%%%%%%%%%
\subsection{テストの承認}
すべてのテストケースが成功した場合にのみ、そのソースコードは承認されるものとする。

%%%%%%%%%%%%%%%%%%%%%%%%%%%%%%%%%%%%%%%%%%%%%%%%%%%%%%%%%%
%% subsection 12.4.5 %%%%%%%%%%%%%%%%%%%%%%%%%%%%%%%%%%%%%
%%%%%%%%%%%%%%%%%%%%%%%%%%%%%%%%%%%%%%%%%%%%%%%%%%%%%%%%%%
\subsection{テストの時期}
開発プロセスにおけるテストの実施タイミングは、ソースコードの新規作成または修正が完了した時点とする。

%%%%%%%%%%%%%%%%%%%%%%%%%%%%%%%%%%%%%%%%%%%%%%%%%%%%%%%%%%
%% subsection 12.4.5 %%%%%%%%%%%%%%%%%%%%%%%%%%%%%%%%%%%%%
%%%%%%%%%%%%%%%%%%%%%%%%%%%%%%%%%%%%%%%%%%%%%%%%%%%%%%%%%%
\subsection{リグレッションテスト}
プログラムの変更や修正等により新たなコードを追加した際は、その新たなコードが既存の機能に影響を与えていないことを確認するため、必要に応じてリグレッションテストを行うものとする。


\clearpage
%%%%%%%%%%%%%%%%%%%%%%%%%%%%%%%%%%%%%%%%%%%%%%%%%%%%%%%%%%
%% section 12.5 %%%%%%%%%%%%%%%%%%%%%%%%%%%%%%%%%%%%%%%%%%
%%%%%%%%%%%%%%%%%%%%%%%%%%%%%%%%%%%%%%%%%%%%%%%%%%%%%%%%%%
\modHeadsection{ドキュメンテーション}

%%%%%%%%%%%%%%%%%%%%%%%%%%%%%%%%%%%%%%%%%%%%%%%%%%%%%%%%%%
%% subsection 12.5.1 %%%%%%%%%%%%%%%%%%%%%%%%%%%%%%%%%%%%%
%%%%%%%%%%%%%%%%%%%%%%%%%%%%%%%%%%%%%%%%%%%%%%%%%%%%%%%%%%
\subsection{ドキュメンテーションの目的}
ソフトウェア開発の際は、ソフトウェアの使用方法、設計、変更履歴などを記録し、開発者やユーザーが理解しやすくするためにドキュメンテーションを行うものとする。

%%%%%%%%%%%%%%%%%%%%%%%%%%%%%%%%%%%%%%%%%%%%%%%%%%%%%%%%%%
%% subsection 12.5.2 %%%%%%%%%%%%%%%%%%%%%%%%%%%%%%%%%%%%%
%%%%%%%%%%%%%%%%%%%%%%%%%%%%%%%%%%%%%%%%%%%%%%%%%%%%%%%%%%
\subsection{ドキュメンテーションの作成}
各々の業務において、その全体像を理解するために、適切なドキュメンテーションの作成を行うものとする。
これには、要件定義書、基本設計書、詳細設計書、ユーザーマニュアル等が含まれる。

%%%%%%%%%%%%%%%%%%%%%%%%%%%%%%%%%%%%%%%%%%%%%%%%%%%%%%%%%%
%% subsection 12.5.3 %%%%%%%%%%%%%%%%%%%%%%%%%%%%%%%%%%%%%
%%%%%%%%%%%%%%%%%%%%%%%%%%%%%%%%%%%%%%%%%%%%%%%%%%%%%%%%%%
\subsection{ドキュメンテーションの更新}
ドキュメンテーションは常に最新の状態を保つものとする。
新規に作成または修正したソースコードに関連するドキュメンテーションは、ソースコードの新規作成または修正が完了した時点で更新する。

%%%%%%%%%%%%%%%%%%%%%%%%%%%%%%%%%%%%%%%%%%%%%%%%%%%%%%%%%%
%% subsection 12.5.4 %%%%%%%%%%%%%%%%%%%%%%%%%%%%%%%%%%%%%
%%%%%%%%%%%%%%%%%%%%%%%%%%%%%%%%%%%%%%%%%%%%%%%%%%%%%%%%%%
\subsection{ドキュメンテーションの保存}
ドキュメンテーションは関係者が容易にアクセスできる場所に保存または提示を行うものとする。
また、最新でないドキュメンテーションは原則として提示せず、紙媒体(印刷物)のものは必要のない限り速やかに破棄をする。


%!TEX root = ../RPA_for_Creating_Program_Note.tex


\modHeadchapter{マシニングセンタにおける工具の取扱い}



%%%%%%%%%%%%%%%%%%%%%%%%%%%%%%%%%%%%%%%%%%%%%%%%%%%%%%%%%%
%% section 06.1 %%%%%%%%%%%%%%%%%%%%%%%%%%%%%%%%%%%%%%%%%%
%%%%%%%%%%%%%%%%%%%%%%%%%%%%%%%%%%%%%%%%%%%%%%%%%%%%%%%%%%
\modHeadsection{工具番号}

%%%%%%%%%%%%%%%%%%%%%%%%%%%%%%%%%%%%%%%%%%%%%%%%%%%%%%%%%%
%% subsection 06.1.1 %%%%%%%%%%%%%%%%%%%%%%%%%%%%%%%%%%%%%
%%%%%%%%%%%%%%%%%%%%%%%%%%%%%%%%%%%%%%%%%%%%%%%%%%%%%%%%%%
\subsection{工具番号の設定}
各々の工具は、一意の\index{こうぐばんごう@工具番号}工具番号を持つものとする。
また、工具番号の設定は、予め定められた規則にしたがって行うものとする。

%%%%%%%%%%%%%%%%%%%%%%%%%%%%%%%%%%%%%%%%%%%%%%%%%%%%%%%%%%
%% subsection 05.1.2 %%%%%%%%%%%%%%%%%%%%%%%%%%%%%%%%%%%%%
%%%%%%%%%%%%%%%%%%%%%%%%%%%%%%%%%%%%%%%%%%%%%%%%%%%%%%%%%%
\subsection{工具の登録}
工具番号が設定された工具は、その番号を用いて\index{マシニングセンタ}マシニングセンタに登録を行うものとする。

%%%%%%%%%%%%%%%%%%%%%%%%%%%%%%%%%%%%%%%%%%%%%%%%%%%%%%%%%%
%% subsection 05.1.3 %%%%%%%%%%%%%%%%%%%%%%%%%%%%%%%%%%%%%
%%%%%%%%%%%%%%%%%%%%%%%%%%%%%%%%%%%%%%%%%%%%%%%%%%%%%%%%%%
\subsection{工具番号の管理}
すべての\index{とうろくこうぐ@登録工具}登録工具とそれらの工具番号は、定期的に確認および更新を行うものとする。
新規に工具が追加された場合や、交換・修理・廃棄等により既存の工具番号の変更が生じた場合は、速やかに工具の\index{とうろくじょうほう@登録情報}登録情報を更新する。
また現在の登録工具番号の情報は、関係者が容易にアクセスできる場所に保存または提示を行う。



%%%%%%%%%%%%%%%%%%%%%%%%%%%%%%%%%%%%%%%%%%%%%%%%%%%%%%%%%%
%% section 05.2 %%%%%%%%%%%%%%%%%%%%%%%%%%%%%%%%%%%%%%%%%%
%%%%%%%%%%%%%%%%%%%%%%%%%%%%%%%%%%%%%%%%%%%%%%%%%%%%%%%%%%
\modHeadsection{工具補正値の設定}
登録された工具は、工具の設置時に\index{こうぐちょう@工具長}工具長および\index{こうぐけい@工具径}工具径を測定し、該当する登録番号の工具に対し適切な\index{こうぐちょうほせいち@工具長補正値}工具長補正値および\index{こうぐけいほせいち@工具径補正値}工具径補正値を設定する。
\index{まもう@摩耗}摩耗等で工具長または工具径に変更が生じた場合は、該当する登録番号の工具に対して適切な\index{こうぐまもうりょう@工具摩耗量}摩耗量を設定する。

また、この他に必要な補正量が存在する場合は、適当な\index{コモンへんすう@コモン変数}コモン変数を用いて設定を行うものとする。



%%%%%%%%%%%%%%%%%%%%%%%%%%%%%%%%%%%%%%%%%%%%%%%%%%%%%%%%%%
%% section 05.2 %%%%%%%%%%%%%%%%%%%%%%%%%%%%%%%%%%%%%%%%%%
%%%%%%%%%%%%%%%%%%%%%%%%%%%%%%%%%%%%%%%%%%%%%%%%%%%%%%%%%%
\modHeadsection{工具の送り速さ値および主軸回転数}

%%%%%%%%%%%%%%%%%%%%%%%%%%%%%%%%%%%%%%%%%%%%%%%%%%%%%%%%%%
%% subsection 05.2.1 %%%%%%%%%%%%%%%%%%%%%%%%%%%%%%%%%%%%%
%%%%%%%%%%%%%%%%%%%%%%%%%%%%%%%%%%%%%%%%%%%%%%%%%%%%%%%%%%
\subsection{送り速さ値および主軸回転数の設定}
各々の工具の\index{おくりはやさち@送り速さ値}送り速さ値および\index{しゅじくかいてんすう@主軸回転数}主軸回転数は、工具の最適なパフォーマンスを確保するために適切に設定するものとする。
送り速さ値および主軸回転数の設定は工具の種類・材料・加工条件等に基づいて行う。
また、現在の送り速さ値ならびに主軸回転数の情報は、関係者が容易にアクセスできる場所に保存または提示を行う。

%%%%%%%%%%%%%%%%%%%%%%%%%%%%%%%%%%%%%%%%%%%%%%%%%%%%%%%%%%
%% subsection 05.2.2 %%%%%%%%%%%%%%%%%%%%%%%%%%%%%%%%%%%%%
%%%%%%%%%%%%%%%%%%%%%%%%%%%%%%%%%%%%%%%%%%%%%%%%%%%%%%%%%%
\subsection{送り速さ値および主軸回転数の変更}
各々の工程における工具の送り速さ値および主軸回転数は、加工状況等に応じて必要があれば適時変更を行う。
送り速さ値または主軸回転数が変更が生じた場合は、速やかに現在の送り速さ値および主軸回転数の情報を更新する。

%!TEX root = ../RPA_for_Creating_Program_Note.tex


著作物をオンライン上に公表することで、その開発や保守等における生産性が大きく向上する。
実際、\DMname におけるバージョン管理やイシュー管理は、オンライン上の\index{バージョンかんりシステム@バージョン管理システム}バージョン管理システム, \index{ソースコードかんりシステム@ソースコード管理システム}ソースコード管理(\index{SCM}SCM)システム, \index{リポジトリホスティングサービス}リポジトリホスティングサービスを用いて行われている
%% footnote %%%%%%%%%%%%%%%%%%%%%
\footnote{これによりコードの共有・バージョン管理・ビルド・テスト・デプロイ・問題追跡・ドキュメンテーションなどの機能を用いることで、開発・保守の生産性が大きく向上している。}。
%%%%%%%%%%%%%%%%%%%%%%%%%%%%%%%%%
一方で、サーバ停止等によるリスクや公表による情報漏洩等の\index{セキュリティ}セキュリティリスクも考えられる。

こうしたことを踏まえ、ここでは作成されたソフトウェア関連の\index{ちょさくぶつ@著作物}著作物の取り扱いについて述べる。



%%%%%%%%%%%%%%%%%%%%%%%%%%%%%%%%%%%%%%%%%%%%%%%%%%%%%%%%%%
%% section 20.1 %%%%%%%%%%%%%%%%%%%%%%%%%%%%%%%%%%%%%%%%%%
%%%%%%%%%%%%%%%%%%%%%%%%%%%%%%%%%%%%%%%%%%%%%%%%%%%%%%%%%%
\modHeadsection{関連する著作物}
マシニングセンタによるモールドの加工に対して作成された(ソフトウェア関連の)著作物として、主に以下のものが挙げられる。
\begin{enumerate}
\item 本書
\item 位置情報等の数値計算用プログラム
\item 使用スペーサ計算用プログラム
\item バンドルのプログラムを除いたNCプログラム
\item モールドのRDB
\item モールドの3次元CADモデリング用テンプレート
\item 内面テーパの3次元CADモデリング用テンプレート
\end{enumerate}
著作権法第2条第1項(著作物の定義)より、これらは「思想又は感情を創作的に表現したもの」であり、著作権法の保護対象となる
%% footnote %%%%%%%%%%%%%%%%%%%%%
\footnote{なお、著作権法第10条第1項第9号(プログラムの著作物)より、プログラムは著作物の一種として明示的に規程されている。}。
%%%%%%%%%%%%%%%%%%%%%%%%%%%%%%%%%



%%%%%%%%%%%%%%%%%%%%%%%%%%%%%%%%%%%%%%%%%%%%%%%%%%%%%%%%%%
%% section 20.2 %%%%%%%%%%%%%%%%%%%%%%%%%%%%%%%%%%%%%%%%%%
%%%%%%%%%%%%%%%%%%%%%%%%%%%%%%%%%%%%%%%%%%%%%%%%%%%%%%%%%%
\modHeadsection{関連著作物の著作権および著作権者}


%%%%%%%%%%%%%%%%%%%%%%%%%%%%%%%%%%%%%%%%%%%%%%%%%%%%%%%%%%
%% subsection 20.2.1 %%%%%%%%%%%%%%%%%%%%%%%%%%%%%%%%%%%%%
%%%%%%%%%%%%%%%%%%%%%%%%%%%%%%%%%%%%%%%%%%%%%%%%%%%%%%%%%%
\subsection{著作人格権}
\index{ちょさくけんほう@著作権法}著作権法第59条より、すべての\index{かんれんちょさくぶつ@関連著作物}関連著作物の\index{ちょさくじんかくけん@著作人格権}著作人格権は、その\index{ちょさくしゃ@著作者}著作者に帰属する。
また、原則として著作人格権の行使の判断・決定は、著作者に委れられるものとする。


%%%%%%%%%%%%%%%%%%%%%%%%%%%%%%%%%%%%%%%%%%%%%%%%%%%%%%%%%%
%% subsection 20.2.2 %%%%%%%%%%%%%%%%%%%%%%%%%%%%%%%%%%%%%
%%%%%%%%%%%%%%%%%%%%%%%%%%%%%%%%%%%%%%%%%%%%%%%%%%%%%%%%%%
\subsection{著作財産権}
\index{ちょさくけんほう@著作権法}著作権法第15条(職務著作物)より、関連著作物が職務上作成された著作物(\index{しょくむちょさくぶつ@職務著作物}職務著作物)に該当する場合、その\index{ちょさくざいさんけん@著作財産権}著作財産権はその業務を指示した法人に帰属する。

関連著作物が職務著作物に該当しない場合、その著作財産権は著作者個人に帰属する。



\clearpage
%%%%%%%%%%%%%%%%%%%%%%%%%%%%%%%%%%%%%%%%%%%%%%%%%%%%%%%%%%
%% section 20.2 %%%%%%%%%%%%%%%%%%%%%%%%%%%%%%%%%%%%%%%%%%
%%%%%%%%%%%%%%%%%%%%%%%%%%%%%%%%%%%%%%%%%%%%%%%%%%%%%%%%%%
\modHeadsection{職務著作物}
\index{ちょさくけんほう@著作権法}著作権法第15条(職務著作物)より、作成された\index{ちょさくぶつ@著作物}著作物が以下の4つの要件をすべて満たす場合に限り、その著作物は\index{しょくむちょさくぶつ@職務著作物}職務著作物に該当する
\begin{enumerate}[label=\Roman*, ref=\Roman*]
\item 法人等の発意に基づくこと
\item 法人等の業務に従事する者が職務上作成するものであること
\item\label{item:copyrightrule3} 法人等の名義の下に公表するものであること
\item 作成の時における契約、勤務規則その他に別段の定めがないこと
\end{enumerate}
この4つの要件をもう少し具体的に述べると、
\begin{enumerate}[label=\Roman*$'$]
\item 法人等がある目的を持って構想した著作物の具体的な作成を従業員に命じることを意味する
\item その著作物が従業員の通常の業務範囲内で作成されたものであれば、それは依然として職務著作物に該当する可能性がある
\item その著作物が作成した業務従事者の名前で公表されれば職務著作物とは認められない
\item その著作物についての契約や勤務規則等に、別段の(適切な)定めがある場合は、その定めに従う
\end{enumerate}
ただし一般に、\ref{item:copyrightrule3}{}について、プログラムの著作物に関してはこの要件は不要とされる。



%\clearpage
%%%%%%%%%%%%%%%%%%%%%%%%%%%%%%%%%%%%%%%%%%%%%%%%%%%%%%%%%%
%% section 20.4 %%%%%%%%%%%%%%%%%%%%%%%%%%%%%%%%%%%%%%%%%%
%%%%%%%%%%%%%%%%%%%%%%%%%%%%%%%%%%%%%%%%%%%%%%%%%%%%%%%%%%
\modHeadsection{関連著作物の公表}
著作権法第18条(公表権)より、著作者は、その著作物でまだ公表されていないもの(その同意を得ないで公表された著作物を含む)を公衆に提供し、または提示する権利を有する(当該著作物を原著作物とする二次的著作物についても同様)。


%%%%%%%%%%%%%%%%%%%%%%%%%%%%%%%%%%%%%%%%%%%%%%%%%%%%%%%%%%
%% subsection 20.4.1 %%%%%%%%%%%%%%%%%%%%%%%%%%%%%%%%%%%%%
%%%%%%%%%%%%%%%%%%%%%%%%%%%%%%%%%%%%%%%%%%%%%%%%%%%%%%%%%%
\subsection{公表する関連著作物}
関連著作物については、開発・保守等の生産性の向上を目的に、原則としてすべてオンライン上に公表する。
ただし、著作権法第18条に基づき、公表の判断はその\index{ちょさくじんかくけん@著作人格権}著作人格権の保有者(\index{ちょさくしゃ@著作者}著作者)および\index{ちょさくざいさんけん@著作財産権}著作財産権の保有者の双方の同意の下で行われることを前提とする。
また、公表の際は個人情報保護法(データ保護・プライバシー保護)にも配慮しなければならない。


%%%%%%%%%%%%%%%%%%%%%%%%%%%%%%%%%%%%%%%%%%%%%%%%%%%%%%%%%%
%% subsection 20.4.2 %%%%%%%%%%%%%%%%%%%%%%%%%%%%%%%%%%%%%
%%%%%%%%%%%%%%%%%%%%%%%%%%%%%%%%%%%%%%%%%%%%%%%%%%%%%%%%%%
\subsection{非公表にする関連著作物}
たとえば作成したメインプログラムやモールドのデータベースについては、個々の明細の情報(機密事項)を推察できるデータを含む。
このような機密事項を含む著作物については、原則として公表しないものとする。
あるいは、公表する場合でも(個々のものすべてでなく)代表的なものに留めるものとする。



\clearpage
\begin{Column}{\DMname の関連著作物}
\DMname のソフトウェアにおける関連著作物の場合、その一切が著作者個人に(管理職・スタッフにより)一任されており、
\begin{enumerate}[label=\Roman*]
\item 法人等がある目的を持って構想した著作物が(少なくともソフトウェアに関しては)全く存在しない。
\item 著作者は、ソフトウェア開発・諸規定等の策定・技術書の作成などを行っており、通常の業務範囲から明らかに逸脱している。
\item 関連著作物は著作者個人の名前あるいはアカウントの下にオンライン上で公表されている。
\item ソフトウェア作成時において、その著作物についての別段の定め等は一切ない。
\end{enumerate}
したがって、いずれの要件も満たしていないため、その\index{ちょさくざいさんけん@著作財産権}著作財産権は著作者個人に帰属する。
\end{Column}






%%%%%%%%%%%%%%%%%%%%%%%%%%%%%%%%%%%%%%%%%%%%%%%%%%%%%%%%%
%% Appendices %%%%%%%%%%%%%%%%%%%%%%%%%%%%%%%%%%%%%%%%%%%
%%%%%%%%%%%%%%%%%%%%%%%%%%%%%%%%%%%%%%%%%%%%%%%%%%%%%%%%%
\begin{appendices}
%\Appendixpart
\end{appendices}

\addtocontents{toc}{\protect\end{tocBox}}
