%!TEX root = ./RfCPN.tex


\addtocontents{toc}{\protect\cleardoublepage}
%%%%%%%%%%%%%%%%%%%%%%%%%%%%%%%%%%%%%%%%%%%%%%%%%%%%%%%%%
%% Part Requirement Definition %%%%%%%%%%%%%%%%%%%%%%%%%%
%%%%%%%%%%%%%%%%%%%%%%%%%%%%%%%%%%%%%%%%%%%%%%%%%%%%%%%%%
\addtocontents{toc}{\protect\begin{tocBox}{\tmppartnum}}%
\tPart{機械の稼働における\index{ようけんていぎ@要件定義}要件定義}{%
\paragraph*{\tpartgoal}
新たな横型マシニングセンタの稼働に関して、その具体的な\index{ようけんていぎ@要件定義}要件定義を行う。

\tcbline*
\paragraph*{\tpartmethod}
マシニングセンタを稼働し製品を生産することに焦点を絞る。
現状のマシニングセンタの仕様を基に、前段階で見た要改善点をすべて考慮した上で、\index{ようけん@要件}要件を洗い出す。

\tcbline*
\paragraph*{\tpartbackground}
先にも述べた通り、(機械設置時点において)ソフトウェアの観点では、新たなマシニングセンタではとても生産ができるような状態にない
%% footnote %%%%%%%%%%%%%%%%%%%%%
\footnote{外注により作成された\index{NCプログラム(がいちゅう)@NCプログラム(外注)}NCプログラムは存在する。
しかし、当社が具体的な\index{ようけんていぎ@要件定義}要件定義を行えなかったため、実用に至っていない。
\index{NCプログラム(がいちゅう)@NCプログラム(外注)}NCプログラムそのものは尤もな内容でありレベルも十分に高いものであるが、(当部門に限らず)当社の\index{ソフトウェアエンジニアリング}ソフトウェアエンジニアリングに関する無関心による影響が顕在化した帰結となっている。}。
%%%%%%%%%%%%%%%%%%%%%%%%%%%%%%%%%
そのため、とにかく生産のできる状態にすることが目下かつ焦眉の課題である。\\
 その一環として、\index{システムかいはつプロセス@システム開発プロセス}システム開発プロセスの計画の策定を進めるために、その最初期の段階である\index{ようけんていぎ@要件定義}要件定義を早急に行う必要がある。
}{%
\paragraph*{\tpartconclusion}
\MMC における\expandafterindex{ぎょうむフロー(\yomiMMC)@業務フロー(\nameMMC)}業務フローを基に、\DMC の稼働に向けたソフトウェア視点における\index{ようけんていぎ@要件定義}要件定義を行った
%% footnote %%%%%%%%%%%%%%%%%%%%%
\footnote{\index{ようけんていぎ@要件定義}要件定義は一度だけ行われるものではない。
開発プロセスの進行に伴い新たな\index{ようけん@要件}要件の追加あるいは\index{ようけん@要件}要件の変更がされしだい、反復的に\index{ようけんていぎ@要件定義}要件定義が行われる。}。
%%%%%%%%%%%%%%%%%%%%%%%%%%%%%%%%%
\tcbline*
\paragraph*{\tpartnextstep}
\index{ようけんていぎ@要件定義}要件定義で特定された\index{ようけん@要件}要件を基に、次の段階である\index{システムせっけい@システム設計}システム設計を行う。
}

%%%%%%%%%%%%%%%%%%%%%%%%%%%%%%%%%%%%%%%%%%%%%%%%%%%%%%%%%%
%% chapters %%%%%%%%%%%%%%%%%%%%%%%%%%%%%%%%%%%%%%%%%%%%%%
%%%%%%%%%%%%%%%%%%%%%%%%%%%%%%%%%%%%%%%%%%%%%%%%%%%%%%%%%%
%!TEX root = ../RfCPN.tex


\modHeadchapter{加工システム作成における達成したい目標・解決すべき課題}
新たに導入したマシニングセンタで何を達成したいのか、その目的さえも明確にされていないのが現状である
%% footnote %%%%%%%%%%%%%%%%%%%%%
\footnote{\MMC に改善の余地が多くある中、どのような論理で導入にまで至ったのか、とても不思議である。}。
%%%%%%%%%%%%%%%%%%%%%%%%%%%%%%%%%
したがって、ここでは暫定的に目的を定め、具体的な目標の設定を行うことにする。



%%%%%%%%%%%%%%%%%%%%%%%%%%%%%%%%%%%%%%%%%%%%%%%%%%%%%%%%%%
%% section 04.01 %%%%%%%%%%%%%%%%%%%%%%%%%%%%%%%%%%%%%%%%%
%%%%%%%%%%%%%%%%%%%%%%%%%%%%%%%%%%%%%%%%%%%%%%%%%%%%%%%%%%
\modHeadsection{新たなマシニングセンタの導入の目的}
ここでは(暫定的な)目的として以下を採用する。
\begin{enumerate}[label=\sarrow]
\item \Dimple 加工の実現
\item \Dimple 加工に伴い増加する教育コストの削減
\item \Dimple 加工に伴い低下する作業効率%
%% footnote %%%%%%%%%%%%%%%%%%%%%
\footnote{機械の導入に伴う、\index{NCプログラム}NCプログラム作成や維持・管理の効率、別のマシニングセンタの移動に起因する効率の低下等が含まれる。}
%%%%%%%%%%%%%%%%%%%%%%%%%%%%%%%%%
の向上
\end{enumerate}
%%%%%%%%%%%%%%%%%%%%%%%%%%%%%%%%%%%%%%%%%%%%%%%%%%%%%%%%%%
%% hosoku %%%%%%%%%%%%%%%%%%%%%%%%%%%%%%%%%%%%%%%%%%%%%%%%
%%%%%%%%%%%%%%%%%%%%%%%%%%%%%%%%%%%%%%%%%%%%%%%%%%%%%%%%%%
\begin{hosoku}
後でも述べるように、とにかくマシニングセンタが稼働(生産)できる状態になることを優先して目指している。
そのため、実現の目処の立っていない\ReliefGrooveMilling についてはここでは盛り込まない。
\end{hosoku}
%%%%%%%%%%%%%%%%%%%%%%%%%%%%%%%%%%%%%%%%%%%%%%%%%%%%%%%%%%
%%%%%%%%%%%%%%%%%%%%%%%%%%%%%%%%%%%%%%%%%%%%%%%%%%%%%%%%%%
%%%%%%%%%%%%%%%%%%%%%%%%%%%%%%%%%%%%%%%%%%%%%%%%%%%%%%%%%%


\clearpage
%%%%%%%%%%%%%%%%%%%%%%%%%%%%%%%%%%%%%%%%%%%%%%%%%%%%%%%%%%
%% section 04.02 %%%%%%%%%%%%%%%%%%%%%%%%%%%%%%%%%%%%%%%%%
%%%%%%%%%%%%%%%%%%%%%%%%%%%%%%%%%%%%%%%%%%%%%%%%%%%%%%%%%%
\modHeadsection{加工システム作成における達成したい目標}


%%%%%%%%%%%%%%%%%%%%%%%%%%%%%%%%%%%%%%%%%%%%%%%%%%%%%%%%%%
%% subsection 04.02.01 %%%%%%%%%%%%%%%%%%%%%%%%%%%%%%%%%%%
%%%%%%%%%%%%%%%%%%%%%%%%%%%%%%%%%%%%%%%%%%%%%%%%%%%%%%%%%%
\subsection{\Dimple 加工の実現に対する目標}
\begin{enumerate}[label=\sarrow]
\item \Dimple に関する解析的な位置情報および\index{じょうけんぶんきじょうほう@条件分岐情報}条件分岐情報の把握および体系化
\item \Dimple 位置測定用\index{NCプログラム}NCプログラムの作成
\item \DimpleMilling 用\index{NCプログラム}NCプログラムの作成
\end{enumerate}


%%%%%%%%%%%%%%%%%%%%%%%%%%%%%%%%%%%%%%%%%%%%%%%%%%%%%%%%%%
%% subsection 04.02.02 %%%%%%%%%%%%%%%%%%%%%%%%%%%%%%%%%%%
%%%%%%%%%%%%%%%%%%%%%%%%%%%%%%%%%%%%%%%%%%%%%%%%%%%%%%%%%%
\subsection{\index{きょういくコスト@教育コスト}教育コストの削減に対する目標}
\begin{enumerate}[label=\sarrow]
\item 諸作業に対する属人性の大幅削減
\end{enumerate}
なお、属人性の削減については、以下のような方針をとるものとする
%% footnote %%%%%%%%%%%%%%%%%%%%%
\footnote{\textgt{書類生成の自動化}や\textgt{コード生成の自動化}などもあるが、これらについては\pageautoref{part:XI}以降で取り扱う。}。
%%%%%%%%%%%%%%%%%%%%%%%%%%%%%%%%%
\begin{enumerate}[label=\sarrow]
\item \textgt{加工の自動化}:手作業による測定・加工の削減・簡易化
\item \textgt{操作の簡易化}:手作業によるマシニングセンタ画面操作の削減・簡易化
\end{enumerate}


%%%%%%%%%%%%%%%%%%%%%%%%%%%%%%%%%%%%%%%%%%%%%%%%%%%%%%%%%%
%% subsection 04.02.03 %%%%%%%%%%%%%%%%%%%%%%%%%%%%%%%%%%%
%%%%%%%%%%%%%%%%%%%%%%%%%%%%%%%%%%%%%%%%%%%%%%%%%%%%%%%%%%
\subsection{全体の作業効率の向上に対する目標}
\begin{enumerate}[label=\sarrow]
\item 諸作業に対する属人性の大幅削減
\item 機内のワークの解析的な位置情報の把握および体系化
\item ワークの加工に関する条件分岐情報の把握および体系化
\end{enumerate}



%%%%%%%%%%%%%%%%%%%%%%%%%%%%%%%%%%%%%%%%%%%%%%%%%%%%%%%%%%
%% subsection 04.02.04 %%%%%%%%%%%%%%%%%%%%%%%%%%%%%%%%%%%
%%%%%%%%%%%%%%%%%%%%%%%%%%%%%%%%%%%%%%%%%%%%%%%%%%%%%%%%%%
\subsection{その他の目標}
\begin{enumerate}[label=\sarrow]
\item 作業員の安全性の考慮
\item 製品・コードの品質の考慮
\item 機械の信頼性の考慮
\end{enumerate}


\clearpage
%%%%%%%%%%%%%%%%%%%%%%%%%%%%%%%%%%%%%%%%%%%%%%%%%%%%%%%%%%
%% section 04.03 %%%%%%%%%%%%%%%%%%%%%%%%%%%%%%%%%%%%%%%%%
%%%%%%%%%%%%%%%%%%%%%%%%%%%%%%%%%%%%%%%%%%%%%%%%%%%%%%%%%%
\modHeadsection{加工システム作成における解決すべき課題\TBW}
赤色(\,\sarrow[red]\!)の項目は、(当社のこれまでの姿勢・振舞いから鑑みて)解決が事実上不可能あるいはそれに近いと帰結されるものを示す。
\begin{enumerate}[label=\sarrow]
\item[{\sarrow[red]}] ソフトウェアエンジニアリングに関する管理職ないしは経営陣の理解力およびモラルの会得
\item[{\sarrow[red]}] ソフトウェアエンジニアリング部門の創設
\item[{\sarrow[red]}] ソフトウェアエンジニアリングに関する管理職の\index{リーダーシップ}リーダーシップの会得・向上
\item[{\sarrow[red]}] ソフトウェアエンジニアリングに関する安全部門の機能化
\item[{\sarrow[red]}] ソフトウェアエンジニアリングに関する環境部門の機能化
\item[{\sarrow[red]}] ソフトウェアエンジニアリングに関する品質部門の機能化
\item[{\sarrow[red]}] プログラマ育成者の増員
\item[{\sarrow[red]}] プログラマの育成・増員
\item[{\sarrow[red]}] 管理職・スタッフの論理的思考力の向上
\item[{\sarrow[red]}] マシニングセンタに関わる管理職・スタッフのプログラミングに関する初等スキルの会得
\item[{\sarrow[red]}] マシニングセンタに関わる管理職・スタッフの初等数学(初等幾何学)能力の向上
\item ソフトウェアエンジニアリングに関する諸規程の策定
\item 諸規程に則った、マシニングセンタに関する諸標準の策定
\item マシニングセンタ内の幾何的情報の解析的な把握および体系化
\item 諸標準に則った、各明細・各工程に対する\index{NCプログラム}NCプログラムの作成
\end{enumerate}



%!TEX root = ../RfCPN.tex


\modHeadchapter{機械の稼働における\index{ようけんしゅうしゅう@要件収集}要件の収集}
新たに導入するマシニングセンタの稼働における\index{ようけんしゅうしゅう@要件収集}要件の収集については、本書の著者が\MMC で実際に作業した経験から行う
%% footnote %%%%%%%%%%%%%%%%%%%%%
\footnote{なお、本書の著者は管理者でも責任者でもない、ただの作業員の1人である。}。
%%%%%%%%%%%%%%%%%%%%%%%%%%%%%%%%%

また、新たに導入するマシニングセンタはすでに設置されており、用いる工具等や内蔵ソフトウェアもすでに提供されているため、それらに対する\index{ようけん@要件}要件については触れない。



%%%%%%%%%%%%%%%%%%%%%%%%%%%%%%%%%%%%%%%%%%%%%%%%%%%%%%%%%%
%% section 05.01 %%%%%%%%%%%%%%%%%%%%%%%%%%%%%%%%%%%%%%%%%
%%%%%%%%%%%%%%%%%%%%%%%%%%%%%%%%%%%%%%%%%%%%%%%%%%%%%%%%%%
\modHeadsection{安全性への考慮}
・作業者の安全について考慮したい


%%%%%%%%%%%%%%%%%%%%%%%%%%%%%%%%%%%%%%%%%%%%%%%%%%%%%%%%%%
%% subsection 05.01.01 %%%%%%%%%%%%%%%%%%%%%%%%%%%%%%%%%%%
%%%%%%%%%%%%%%%%%%%%%%%%%%%%%%%%%%%%%%%%%%%%%%%%%%%%%%%%%%
\subsection{肉体的負担に対する考慮}
\begin{enumerate}[label=\sarrow]
\item 機内に侵入する作業の削減
\item 手持ち研磨機による加工の廃止
\end{enumerate}


%%%%%%%%%%%%%%%%%%%%%%%%%%%%%%%%%%%%%%%%%%%%%%%%%%%%%%%%%%
%% subsection 05.01.02 %%%%%%%%%%%%%%%%%%%%%%%%%%%%%%%%%%%
%%%%%%%%%%%%%%%%%%%%%%%%%%%%%%%%%%%%%%%%%%%%%%%%%%%%%%%%%%
\subsection{精神的負担に対する考慮}
\begin{enumerate}[label=\sarrow]
\item 手動による操作盤を用いた調整作業の削減
\item 手動による\TouchSensorProbe を用いた測定作業の削減
\end{enumerate}



%%%%%%%%%%%%%%%%%%%%%%%%%%%%%%%%%%%%%%%%%%%%%%%%%%%%%%%%%%
%% section 05.02 %%%%%%%%%%%%%%%%%%%%%%%%%%%%%%%%%%%%%%%%%
%%%%%%%%%%%%%%%%%%%%%%%%%%%%%%%%%%%%%%%%%%%%%%%%%%%%%%%%%%
\modHeadsection{作業環境への考慮}
・作業環境について考慮したい
\begin{enumerate}[label=\sarrow]
\item 機内に侵入する作業の削減
\item 手持ち研磨機による加工の廃止
\end{enumerate}



\clearpage
%%%%%%%%%%%%%%%%%%%%%%%%%%%%%%%%%%%%%%%%%%%%%%%%%%%%%%%%%%
%% section 05.03 %%%%%%%%%%%%%%%%%%%%%%%%%%%%%%%%%%%%%%%%%
%%%%%%%%%%%%%%%%%%%%%%%%%%%%%%%%%%%%%%%%%%%%%%%%%%%%%%%%%%
\modHeadsection{品質への考慮}
・品質について考慮したい
\begin{enumerate}[label=\sarrow]
\item 手作業による書類記入作業の削減
\item 手作業による計算および入力作業の削減
\item 手動による\TouchSensorProbe を用いた測定作業の削減
\item 手動による加工作業の削減
\item \Spacer による振分調整作業の廃止
\item \CurvedOutcutMilling の自動化
\item ワーク\FixtureBolt の選定作業の簡易化
\end{enumerate}



%\clearpage
%%%%%%%%%%%%%%%%%%%%%%%%%%%%%%%%%%%%%%%%%%%%%%%%%%%%%%%%%%
%% section 05.04 %%%%%%%%%%%%%%%%%%%%%%%%%%%%%%%%%%%%%%%%%
%%%%%%%%%%%%%%%%%%%%%%%%%%%%%%%%%%%%%%%%%%%%%%%%%%%%%%%%%%
\modHeadsection{信頼性への考慮}
・機械の信頼性について考慮したい
\begin{enumerate}[label=\sarrow]
\item 異常測定値の検出機能の実装
\item 衝突防止策の実装
\end{enumerate}



%\clearpage
%%%%%%%%%%%%%%%%%%%%%%%%%%%%%%%%%%%%%%%%%%%%%%%%%%%%%%%%%%
%% section 05.04 %%%%%%%%%%%%%%%%%%%%%%%%%%%%%%%%%%%%%%%%%
%%%%%%%%%%%%%%%%%%%%%%%%%%%%%%%%%%%%%%%%%%%%%%%%%%%%%%%%%%
\modHeadsection{ワーク設置の段取り作業の簡易化}
・ワーク\FixtureBolt の選定の指標がほしい
\begin{enumerate}[label=\sarrow]
\item ワーク\FixtureBolt 長さの指標の解析的導出
\end{enumerate}
・\Spacer の取付け・取外し作業をなくしたい
\begin{enumerate}[label=\sarrow]
\item $B$軸回転を用いた\AlocationLength 調整の導入
\end{enumerate}



%%%%%%%%%%%%%%%%%%%%%%%%%%%%%%%%%%%%%%%%%%%%%%%%%%%%%%%%%%
%% section 05.02 %%%%%%%%%%%%%%%%%%%%%%%%%%%%%%%%%%%%%%%%%
%%%%%%%%%%%%%%%%%%%%%%%%%%%%%%%%%%%%%%%%%%%%%%%%%%%%%%%%%%
\modHeadsection{\TouchSensorProbe 測定の調整作業の簡易化}
・\index{げんてんせってい@原点設定}原点設定時の測定箇所変更時における諸調整作業を容易にしたい
\begin{enumerate}[label=\sarrow]
\item 測定箇所変更時の諸計算の機械化
\end{enumerate}
・\CenterlineEndFaceDifMeasurement を自動化したい
\begin{enumerate}[label=\sarrow]
\item \CenterlineEndFaceDifMeasurement の機械化
\end{enumerate}
・\CurvedOutcutMilling 時の測定作業をなくしたい
\begin{enumerate}[label=\sarrow]
\item \CurvedOutcutMilling の自動化
\end{enumerate}



\clearpage
%%%%%%%%%%%%%%%%%%%%%%%%%%%%%%%%%%%%%%%%%%%%%%%%%%%%%%%%%%
%% section 05.03 %%%%%%%%%%%%%%%%%%%%%%%%%%%%%%%%%%%%%%%%%
%%%%%%%%%%%%%%%%%%%%%%%%%%%%%%%%%%%%%%%%%%%%%%%%%%%%%%%%%%
\modHeadsection{各加工工程の調整作業の簡易化}


%%%%%%%%%%%%%%%%%%%%%%%%%%%%%%%%%%%%%%%%%%%%%%%%%%%%%%%%%%
%% subsection 05.03.01 %%%%%%%%%%%%%%%%%%%%%%%%%%%%%%%%%%%
%%%%%%%%%%%%%%%%%%%%%%%%%%%%%%%%%%%%%%%%%%%%%%%%%%%%%%%%%%
\subsection{加工全般に対する調整作業の簡易化}
・\index{NCメインプログラム}NCメインプログラム
%% footnote %%%%%%%%%%%%%%%%%%%%%
\footnote{ここでいう\index{メインプログラム}メインプログラムとは、\DrawingNumber と同一の\index{プログラムばんごう@プログラム番号}プログラム番号のものを指し、\index{サブプログラム}サブプログラムとはそれ以外のものを指す。}
%%%%%%%%%%%%%%%%%%%%%%%%%%%%%%%%%
の直接編集作業をなくしたい
\begin{enumerate}[label=\sarrow]
\item \index{NCメインプログラム}NCメインプログラムの全面見直し
\item \index{NCサブプログラム}NCサブプログラムの全面見直し
\item 必要変更箇所のコモン変数化
\end{enumerate}
・衝突防止策を備え付けたい
\begin{enumerate}[label=\sarrow]
\item 各工程用\index{NCサブプログラム}NCサブプログラムに機能追加
\end{enumerate}



%%%%%%%%%%%%%%%%%%%%%%%%%%%%%%%%%%%%%%%%%%%%%%%%%%%%%%%%%%
%% subsection 05.03.02 %%%%%%%%%%%%%%%%%%%%%%%%%%%%%%%%%%%
%%%%%%%%%%%%%%%%%%%%%%%%%%%%%%%%%%%%%%%%%%%%%%%%%%%%%%%%%%
\subsection{\EndFacecutMilling の調整作業の簡易化}
・\TDCorrection の変更作業をなくしたい
\begin{enumerate}[label=\sarrow]
\item \IDCenter を基準とした\EndFacecutMilling に変更
\item \ODCornerR への対応
\end{enumerate}
・\index{けずりしろ@削り代}削り代や加工回数の変更を簡易化したい
\begin{enumerate}[label=\sarrow]
\item \EndFacecutMilling 用\index{NCサブプログラム}NCサブプログラムに機能追加
\end{enumerate}


%%%%%%%%%%%%%%%%%%%%%%%%%%%%%%%%%%%%%%%%%%%%%%%%%%%%%%%%%%
%% subsection 05.03.03 %%%%%%%%%%%%%%%%%%%%%%%%%%%%%%%%%%%
%%%%%%%%%%%%%%%%%%%%%%%%%%%%%%%%%%%%%%%%%%%%%%%%%%%%%%%%%%
\subsection{\OutcutMilling の調整作業の簡易化}
\begin{enumerate}[label=\sarrow]
\item \CurvedOutcutMilling の機械化
\end{enumerate}


%%%%%%%%%%%%%%%%%%%%%%%%%%%%%%%%%%%%%%%%%%%%%%%%%%%%%%%%%%
%% subsection 05.03.04 %%%%%%%%%%%%%%%%%%%%%%%%%%%%%%%%%%%
%%%%%%%%%%%%%%%%%%%%%%%%%%%%%%%%%%%%%%%%%%%%%%%%%%%%%%%%%%
\subsection{\KeywayMilling の調整作業の簡易化}
・\KeywayPos・\KeywayWidth 補正を直感的にしたい
\begin{enumerate}[label=\sarrow]
\item \KeywayMilling 用\index{NCサブプログラム}NCサブプログラムに機能追加
\end{enumerate}
・四角形や六角形の\Keyway の角におけるかえりをなくしたい
\begin{enumerate}[label=\sarrow]
\item \KeywayMilling 用\index{NCサブプログラム}NCサブプログラムの見直し
\end{enumerate}
・R面取付き六角形の\Keyway の加工に対応したい
\begin{enumerate}[label=\sarrow]
\item \KeywayMilling 用\index{NCサブプログラム}NCサブプログラムに機能追加
\end{enumerate}


%%%%%%%%%%%%%%%%%%%%%%%%%%%%%%%%%%%%%%%%%%%%%%%%%%%%%%%%%%
%% subsection 05.03.05 %%%%%%%%%%%%%%%%%%%%%%%%%%%%%%%%%%%
%%%%%%%%%%%%%%%%%%%%%%%%%%%%%%%%%%%%%%%%%%%%%%%%%%%%%%%%%%
\subsection{\EndFaceChamferMilling 作業の簡易化}
・\index{てもちけんまき@手持ち研磨機}手持ち研磨機を使用した手動加工作業をなくしたい
\begin{enumerate}[label=\sarrow]
\item \EndFaceChamferMilling の機械化
\end{enumerate}
・AC方向の位置調整を目分量で調整する作業をなくしたい
\begin{enumerate}[label=\sarrow]
\item AC方向の位置調整作業の機械化
\end{enumerate}



\clearpage
%%%%%%%%%%%%%%%%%%%%%%%%%%%%%%%%%%%%%%%%%%%%%%%%%%%%%%%%%%
%% section 05.04 %%%%%%%%%%%%%%%%%%%%%%%%%%%%%%%%%%%%%%%%%
%%%%%%%%%%%%%%%%%%%%%%%%%%%%%%%%%%%%%%%%%%%%%%%%%%%%%%%%%%
\modHeadsection{\index{NCプログラム}NCプログラム作成作業の簡易化\TBW}


%%%%%%%%%%%%%%%%%%%%%%%%%%%%%%%%%%%%%%%%%%%%%%%%%%%%%%%%%%
%% subsection 05.04.01 %%%%%%%%%%%%%%%%%%%%%%%%%%%%%%%%%%%
%%%%%%%%%%%%%%%%%%%%%%%%%%%%%%%%%%%%%%%%%%%%%%%%%%%%%%%%%%
\subsection{各工程用\index{NCメインプログラム}NCメインプログラムの見直し\TBW}
(to be written...)


%%%%%%%%%%%%%%%%%%%%%%%%%%%%%%%%%%%%%%%%%%%%%%%%%%%%%%%%%%
%% subsection 05.04.02 %%%%%%%%%%%%%%%%%%%%%%%%%%%%%%%%%%%
%%%%%%%%%%%%%%%%%%%%%%%%%%%%%%%%%%%%%%%%%%%%%%%%%%%%%%%%%%
\subsection{各工程用\index{NCサブプログラム}NCサブプログラムの見直し\TBW}
(to be written...)




%!TEX root = ../RfCPN.tex


\modHeadchapter{機械の稼働における\index{きのうようけん@機能要件}機能要件の洗い出し\TBW}



%%%%%%%%%%%%%%%%%%%%%%%%%%%%%%%%%%%%%%%%%%%%%%%%%%%%%%%%%%
%% section 06.01 %%%%%%%%%%%%%%%%%%%%%%%%%%%%%%%%%%%%%%%%%
%%%%%%%%%%%%%%%%%%%%%%%%%%%%%%%%%%%%%%%%%%%%%%%%%%%%%%%%%%
\modHeadsection{ワークの幾何的情報の解析計算および体系化\TBW}
\begin{enumerate}[label=\sarrow]
\item
\end{enumerate}



%%%%%%%%%%%%%%%%%%%%%%%%%%%%%%%%%%%%%%%%%%%%%%%%%%%%%%%%%%
%% section 06.02 %%%%%%%%%%%%%%%%%%%%%%%%%%%%%%%%%%%%%%%%%
%%%%%%%%%%%%%%%%%%%%%%%%%%%%%%%%%%%%%%%%%%%%%%%%%%%%%%%%%%
\modHeadsection{\index{NCプログラム}NCプログラム作成に諸基準の規格化\TBW}
\begin{enumerate}[label=\sarrow]
\item
\end{enumerate}



%%%%%%%%%%%%%%%%%%%%%%%%%%%%%%%%%%%%%%%%%%%%%%%%%%%%%%%%%%
%% section 06.03 %%%%%%%%%%%%%%%%%%%%%%%%%%%%%%%%%%%%%%%%%
%%%%%%%%%%%%%%%%%%%%%%%%%%%%%%%%%%%%%%%%%%%%%%%%%%%%%%%%%%
\modHeadsection{\index{げんてんせってい@原点設定}原点設定用NCサブプログラムの作成\TBW}


%%%%%%%%%%%%%%%%%%%%%%%%%%%%%%%%%%%%%%%%%%%%%%%%%%%%%%%%%%
%% subsection 06.03.01 %%%%%%%%%%%%%%%%%%%%%%%%%%%%%%%%%%%
%%%%%%%%%%%%%%%%%%%%%%%%%%%%%%%%%%%%%%%%%%%%%%%%%%%%%%%%%%
\subsection{\TBW}
\begin{enumerate}[label=\sarrow]
\item
\end{enumerate}



%%%%%%%%%%%%%%%%%%%%%%%%%%%%%%%%%%%%%%%%%%%%%%%%%%%%%%%%%%
%% section 06.04 %%%%%%%%%%%%%%%%%%%%%%%%%%%%%%%%%%%%%%%%%
%%%%%%%%%%%%%%%%%%%%%%%%%%%%%%%%%%%%%%%%%%%%%%%%%%%%%%%%%%
\modHeadsection{\EndFacecutMilling 用NCサブプログラムの作成\TBW}
\begin{enumerate}[label=\sarrow]
\item \IDCenter を基準とした\EndFacecutMilling に変更
\item \ODCornerR への対応
\item \index{けずりしろ@削り代}削り代および加工回数の変更の簡易化
\end{enumerate}



%%%%%%%%%%%%%%%%%%%%%%%%%%%%%%%%%%%%%%%%%%%%%%%%%%%%%%%%%%
%% section 06.05 %%%%%%%%%%%%%%%%%%%%%%%%%%%%%%%%%%%%%%%%%
%%%%%%%%%%%%%%%%%%%%%%%%%%%%%%%%%%%%%%%%%%%%%%%%%%%%%%%%%%
\modHeadsection{\OutcutMilling 用NCサブプログラムの作成\TBW}
\begin{enumerate}[label=\sarrow]
\item \CurvedOutcutMilling の機械化
\end{enumerate}



%%%%%%%%%%%%%%%%%%%%%%%%%%%%%%%%%%%%%%%%%%%%%%%%%%%%%%%%%%
%% section 06.06 %%%%%%%%%%%%%%%%%%%%%%%%%%%%%%%%%%%%%%%%%
%%%%%%%%%%%%%%%%%%%%%%%%%%%%%%%%%%%%%%%%%%%%%%%%%%%%%%%%%%
\modHeadsection{\KeywayMilling 用NCサブプログラムの作成\TBW}
\begin{enumerate}[label=\sarrow]
\item \KeywayPos・\KeywayWidth 補正変更作業の簡易化
\item \KeywayWidth に応じた加工回数の変更作業の機械化
\end{enumerate}



%%%%%%%%%%%%%%%%%%%%%%%%%%%%%%%%%%%%%%%%%%%%%%%%%%%%%%%%%%
%% section 06.07 %%%%%%%%%%%%%%%%%%%%%%%%%%%%%%%%%%%%%%%%%
%%%%%%%%%%%%%%%%%%%%%%%%%%%%%%%%%%%%%%%%%%%%%%%%%%%%%%%%%%
\modHeadsection{\EndFaceChamferMilling 用NCサブプログラムの作成\TBW}
\begin{enumerate}[label=\sarrow]
\item \EndFaceChamferMilling の機械化
\item AC方向の位置調整作業の機械化
\end{enumerate}



%%%%%%%%%%%%%%%%%%%%%%%%%%%%%%%%%%%%%%%%%%%%%%%%%%%%%%%%%%
%% section 06.08 %%%%%%%%%%%%%%%%%%%%%%%%%%%%%%%%%%%%%%%%%
%%%%%%%%%%%%%%%%%%%%%%%%%%%%%%%%%%%%%%%%%%%%%%%%%%%%%%%%%%
\modHeadsection{\Dimple 用NCサブプログラムの作成\TBW}
\begin{enumerate}[label=\sarrow]
\item
\end{enumerate}



%!TEX root = ../RfCPN.tex


\modHeadchapter{加工システム作成における\index{ひきのうようけん@非機能要件}非機能要件\TBW}
% システムの性能、信頼性、拡張性など、ソフトウェアの「どのように動作するべきか」に関する要件を定義


\begin{NFR}{}
\begin{enumerate}[label=\sarrow]
\item \index{NCプログラム}NCプログラムは保守・管理がしやすいように作成するように努める
\item \index{NCプログラム}NCプログラム作成者の都合を、マシニングセンタ管理者や作業者における操作の利便性より優先しない
%% footnote %%%%%%%%%%%%%%%%%%%%%
\footnote{自らの怠慢を理由に作業者らに負担を押し付けないようにする、という道徳的心得である。}
%%%%%%%%%%%%%%%%%%%%%%%%%%%%%%%%%
\item 作業者の操作の利便性を最優先にし、次いでマシニングセンタ管理者の操作の利便性を優先する。
\end{enumerate}
\end{NFR}

\begin{NFR}{}
\begin{enumerate}[label=\sarrow]
\item NCプログラムの\index{ひきすう@引数}引数は、原則として\index{ひきすうしていI@引数指定I}引数指定I(\pageautoref{chap:argumentSpecification}参照)を用いる
\item \index{NCメインプログラム}NCメインプログラムで用いる各工程用\index{NCサブプログラム}NCサブプログラムの\index{ひきすう@引数}引数の値は、原則として\index{ワーク}ワークの\index{ずめん(モールド)@図面(モールド)}図面に用いられている寸法値(数値)で与える
\end{enumerate}
\end{NFR}

\begin{NFR}{}
\begin{enumerate}[label=\sarrow]
\item NCプログラムの\index{いれここうぞう@入れ子構造}入れ子構造(\index{ネスティング}ネスティング)は、NCメインプログラムを最上位の階層として、原則として3重までとする。
\end{enumerate}
\end{NFR}

\begin{NFR}{}
\begin{enumerate}[label=\sarrow]
\item NCプログラム内での座標について、\index{3じげんユークリッドくうかんざひょう@3次元ユークリッド空間座標}3次元ユークリッド空間座標の使用を優先する。
\end{enumerate}
\end{NFR}






\modHeadchapter{制約の特定\TBW}
% 予算、時間、既存のシステムとの互換性など、プロジェクトに影響を与える可能性のある制約を特定



\modHeadsection{予算\TBW}
(to be written...)



\modHeadsection{時間\TBW}
(to be written...)



\modHeadsection{既存のシステムとの互換性\TBW}
(to be written...)




\modHeadchapter{要件の検証\TBW}
% すべての要件を文書化し、関係者全員が理解し、合意できることを確認



\modHeadsection{要件の検証\TBW}
(to be written...)

%%%%%%%%%%%%%%%%%%%%%%%%%%%%%%%%%%%%%%%%%%%%%%%%%%%%%%%%%
%% Appendices %%%%%%%%%%%%%%%%%%%%%%%%%%%%%%%%%%%%%%%%%%%
%%%%%%%%%%%%%%%%%%%%%%%%%%%%%%%%%%%%%%%%%%%%%%%%%%%%%%%%%
\begin{appendices}
%\Appendixpart
\end{appendices}

\addtocontents{toc}{\protect\end{tocBox}}
