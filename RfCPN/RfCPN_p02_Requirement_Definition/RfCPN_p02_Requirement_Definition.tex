%!TEX root = ./RfCPN.tex


\addtocontents{toc}{\protect\cleardoublepage}
%%%%%%%%%%%%%%%%%%%%%%%%%%%%%%%%%%%%%%%%%%%%%%%%%%%%%%%%%
%% Part Requirement Definition %%%%%%%%%%%%%%%%%%%%%%%%%%
%%%%%%%%%%%%%%%%%%%%%%%%%%%%%%%%%%%%%%%%%%%%%%%%%%%%%%%%%
\addtocontents{toc}{\protect\begin{tocBox}{\tmppartnum}}%
\tPart{機械の稼働における\index{ようけんていぎ@要件定義}要件定義}{%
\paragraph*{\tpartgoal}
新たな横型マシニングセンタの稼働に関して、その具体的な\index{ようけんていぎ@要件定義}要件定義を行う。

\tcbline*
\paragraph*{\tpartmethod}
マシニングセンタを稼働し製品を生産することに焦点を絞る。
現状のマシニングセンタの仕様を基に、前段階で見た改善案をすべて考慮した上で、\index{ようけんていぎ@要件定義}要件定義を行う。

\tcbline*
\paragraph*{\tpartbackground}
先にも述べた通り、(機械設置時点において)ソフトウェアの観点では、新たなマシニングセンタではとても生産ができるような状態にないことは明らかである
%% footnote %%%%%%%%%%%%%%%%%%%%%
\footnote{外注により作成された\index{NCプログラム(がいちゅう)@NCプログラム(外注)}NCプログラムは一応存在する。
しかし、当社が具体的な\index{ようけんていぎ@要件定義}要件定義さえ行えなかったため、全く実用に至っていない。
NCプログラムそのものは尤もな内容でありレベルも十分なものであるが、(当部門に限らず)当社の\index{ソフトウェアエンジニアリング}ソフトウェアエンジニアリングに関する致命的なほどの無関心が顕わに露呈したものであり、機械導入以前から予想されていたとおりの自明な帰結である。}。
%%%%%%%%%%%%%%%%%%%%%%%%%%%%%%%%%
そのため、とにかく生産のできる状態にすることが目下かつ焦眉の課題である。\\
 その一環として、\index{システムかいはつプロセス@システム開発プロセス}システム開発プロセスの計画において、その最初期の段階である\index{ようけんていぎ@要件定義}要件定義から始める必要がある。
}{%
\paragraph*{\tpartconclusion}
\MMC における業務の流れを基に、\DMC の稼働に向けたソフトウェア視点における\index{ようけんていぎ@要件定義}要件定義を行った
%% footnote %%%%%%%%%%%%%%%%%%%%%
\footnote{\index{ようけんていぎ@要件定義}要件定義は一度だけ行われるものではない。
開発プロセスの進行に伴い新たな\index{要件}要件あるいは\index{要件}要件の変更がされしだい、反復的に\index{ようけんていぎ@要件定義}要件定義が行われる。}。
%%%%%%%%%%%%%%%%%%%%%%%%%%%%%%%%%
\tcbline*
\paragraph*{\tpartnextstep}
\index{ようけんていぎ@要件定義}要件定義で特定された要件を基に、次の段階である\index{システムせっけい@システム設計}システム設計を行う。
}

%%%%%%%%%%%%%%%%%%%%%%%%%%%%%%%%%%%%%%%%%%%%%%%%%%%%%%%%%
%% chapters %%%%%%%%%%%%%%%%%%%%%%%%%%%%%%%%%%%%%%%%%%%%%%
%%%%%%%%%%%%%%%%%%%%%%%%%%%%%%%%%%%%%%%%%%%%%%%%%%%%%%%%%%
%!TEX root = ../RfCPN.tex


\modHeadchapter{目的・目標の明確化}
新たに導入するマシニングセンタで何を達成したいのか、その目的さえも明確にされていないのが現状である
%% footnote %%%%%%%%%%%%%%%%%%%%%
\footnote{\MMC を蔑ろに扱っている現状にもかかわらず、そもそもどのような論理で導入にまで至ったのか、常識的に考えて不思議でならない。}。
%%%%%%%%%%%%%%%%%%%%%%%%%%%%%%%%%
したがって、ここでは暫定的に目的を定め、具体的な目標の設定を行うことにする。



%%%%%%%%%%%%%%%%%%%%%%%%%%%%%%%%%%%%%%%%%%%%%%%%%%%%%%%%%%
%% section 04.01 %%%%%%%%%%%%%%%%%%%%%%%%%%%%%%%%%%%%%%%%%
%%%%%%%%%%%%%%%%%%%%%%%%%%%%%%%%%%%%%%%%%%%%%%%%%%%%%%%%%%
\modHeadsection{機械導入の目的}
(暫定的な)目的として、以下を採用する。
\begin{enumerate}[label=\sarrow]
\item 作業効率(\MMC 比)の大幅向上
\item 教育コスト(\MMC 比)の大幅削減
\item \Dimple 加工の内製化
\end{enumerate}



%%%%%%%%%%%%%%%%%%%%%%%%%%%%%%%%%%%%%%%%%%%%%%%%%%%%%%%%%%
%% section 04.02 %%%%%%%%%%%%%%%%%%%%%%%%%%%%%%%%%%%%%%%%%
%%%%%%%%%%%%%%%%%%%%%%%%%%%%%%%%%%%%%%%%%%%%%%%%%%%%%%%%%%
\modHeadsection{達成したい目標}
達成したい目標として主に以下が挙げられる。
\begin{enumerate}[label=\sarrow]
\item 諸規定の策定
\item 諸規定に則った、諸標準の策定
\item 諸標準に則った、各明細・各工程に対する\index{NCプログラム}NCプログラムの作成
\item 諸作業に対する属人性の大幅削減
\end{enumerate}
なお、属人性の削減については、以下のような方針をとるものとする。
\begin{enumerate}[label=\sarrow]
\item \textgt{加工の自動化}:手作業による測定・加工の削減・簡易化
\item \textgt{操作の自動化}:手作業によるマシニングセンタ画面操作の削減・簡易化
\item \textgt{書類生成の自動化}:手作業による書類記入の削減・簡易化
\item \textgt{コード生成の自動化}:手作業による\index{NCメインプログラム}NC(メイン)プログラム作成の簡易化
\end{enumerate}



\clearpage
%%%%%%%%%%%%%%%%%%%%%%%%%%%%%%%%%%%%%%%%%%%%%%%%%%%%%%%%%%
%% section 04.03 %%%%%%%%%%%%%%%%%%%%%%%%%%%%%%%%%%%%%%%%%
%%%%%%%%%%%%%%%%%%%%%%%%%%%%%%%%%%%%%%%%%%%%%%%%%%%%%%%%%%
\modHeadsection{諸作業の目標}


%%%%%%%%%%%%%%%%%%%%%%%%%%%%%%%%%%%%%%%%%%%%%%%%%%%%%%%%%%
%% subsection 04.03.01 %%%%%%%%%%%%%%%%%%%%%%%%%%%%%%%%%%%
%%%%%%%%%%%%%%%%%%%%%%%%%%%%%%%%%%%%%%%%%%%%%%%%%%%%%%%%%%
\subsection{\index{ワークのせっち@ワークの設置}ワークの設置における目標}
\begin{enumerate}
\item \Spacer による振分調整作業の廃止
\item ワーク\FixtureBolt の規格化
\end{enumerate}


%%%%%%%%%%%%%%%%%%%%%%%%%%%%%%%%%%%%%%%%%%%%%%%%%%%%%%%%%%
%% subsection 04.03.02 %%%%%%%%%%%%%%%%%%%%%%%%%%%%%%%%%%%
%%%%%%%%%%%%%%%%%%%%%%%%%%%%%%%%%%%%%%%%%%%%%%%%%%%%%%%%%%
\subsection{測定(原点設定・\CenterlineEndFaceDif)における目標}
\begin{enumerate}
\item \index{ワークざひょうげんてん@ワーク座標原点}ワーク座標原点概算値導出作業の廃止(解析的導出)
\item 測定箇所変更時の数値変更作業の自動化
\item AC方向\KeywayCenter 座標計算作業の廃止・自動化
\item \CenterlineEndFaceDifMeasurement 作業の廃止・自動化
\end{enumerate}


%%%%%%%%%%%%%%%%%%%%%%%%%%%%%%%%%%%%%%%%%%%%%%%%%%%%%%%%%%
%% subsection 04.03.03 %%%%%%%%%%%%%%%%%%%%%%%%%%%%%%%%%%%
%%%%%%%%%%%%%%%%%%%%%%%%%%%%%%%%%%%%%%%%%%%%%%%%%%%%%%%%%%
\subsection{\DimpleMeasurement における目標}
\begin{enumerate}
\item 短時間による\DimpleMeasurement(概ね6s/個 程度)
\end{enumerate}


%%%%%%%%%%%%%%%%%%%%%%%%%%%%%%%%%%%%%%%%%%%%%%%%%%%%%%%%%%
%% subsection 04.03.04 %%%%%%%%%%%%%%%%%%%%%%%%%%%%%%%%%%%
%%%%%%%%%%%%%%%%%%%%%%%%%%%%%%%%%%%%%%%%%%%%%%%%%%%%%%%%%%
\subsection{\DimpleMilling における目標\TBW}
(to be written...)


%%%%%%%%%%%%%%%%%%%%%%%%%%%%%%%%%%%%%%%%%%%%%%%%%%%%%%%%%%
%% subsection 04.03.05 %%%%%%%%%%%%%%%%%%%%%%%%%%%%%%%%%%%
%%%%%%%%%%%%%%%%%%%%%%%%%%%%%%%%%%%%%%%%%%%%%%%%%%%%%%%%%%
\subsection{\EndFacecutMilling における目標}
\begin{enumerate}
\item 加工回数変更作業の簡易化(半自動化)
\item \TDCValue 変更作業の廃止
\end{enumerate}


%%%%%%%%%%%%%%%%%%%%%%%%%%%%%%%%%%%%%%%%%%%%%%%%%%%%%%%%%%
%% subsection 04.03.06 %%%%%%%%%%%%%%%%%%%%%%%%%%%%%%%%%%%
%%%%%%%%%%%%%%%%%%%%%%%%%%%%%%%%%%%%%%%%%%%%%%%%%%%%%%%%%%
\subsection{\OutcutMilling における目標}
\begin{enumerate}
\item \CurvedOutcut 用測定作業の廃止
\item \CurvedOutcutMilling の自動化
\end{enumerate}


%%%%%%%%%%%%%%%%%%%%%%%%%%%%%%%%%%%%%%%%%%%%%%%%%%%%%%%%%%
%% subsection 04.03.07 %%%%%%%%%%%%%%%%%%%%%%%%%%%%%%%%%%%
%%%%%%%%%%%%%%%%%%%%%%%%%%%%%%%%%%%%%%%%%%%%%%%%%%%%%%%%%%
\subsection{\KeywayMilling における目標}
\begin{enumerate}
\item \KeywayPos・\KeywayWidth 調整作業の簡易化
\item $Z$方向加工回数変更作業の廃止・自動化
\end{enumerate}


%%%%%%%%%%%%%%%%%%%%%%%%%%%%%%%%%%%%%%%%%%%%%%%%%%%%%%%%%%
%% subsection 04.03.08 %%%%%%%%%%%%%%%%%%%%%%%%%%%%%%%%%%%
%%%%%%%%%%%%%%%%%%%%%%%%%%%%%%%%%%%%%%%%%%%%%%%%%%%%%%%%%%
\subsection{\EndFaceChamferMilling における目標}
\begin{enumerate}
\item 手作業による\EndFaceCChamferMilling の廃止・自動化
\item 手作業による\EndFaceRChamferMilling の簡易化・半自動化
\end{enumerate}


%%%%%%%%%%%%%%%%%%%%%%%%%%%%%%%%%%%%%%%%%%%%%%%%%%%%%%%%%%
%% subsection 04.03.09 %%%%%%%%%%%%%%%%%%%%%%%%%%%%%%%%%%%
%%%%%%%%%%%%%%%%%%%%%%%%%%%%%%%%%%%%%%%%%%%%%%%%%%%%%%%%%%
\subsection{\EndFaceBoringMilling における目標}
\begin{enumerate}
\item \EndFaceBoringWidth の計算間違いの訂正
\end{enumerate}


%%%%%%%%%%%%%%%%%%%%%%%%%%%%%%%%%%%%%%%%%%%%%%%%%%%%%%%%%%
%% subsection 04.03.10 %%%%%%%%%%%%%%%%%%%%%%%%%%%%%%%%%%%
%%%%%%%%%%%%%%%%%%%%%%%%%%%%%%%%%%%%%%%%%%%%%%%%%%%%%%%%%%
\subsection{\IncutBoringMilling における目標\TBW}
(to be written...)



\clearpage
%%%%%%%%%%%%%%%%%%%%%%%%%%%%%%%%%%%%%%%%%%%%%%%%%%%%%%%%%%
%% section 04.02 %%%%%%%%%%%%%%%%%%%%%%%%%%%%%%%%%%%%%%%%%
%%%%%%%%%%%%%%%%%%%%%%%%%%%%%%%%%%%%%%%%%%%%%%%%%%%%%%%%%%
\modHeadsection{目標の優先順位\TBW}
(to be written...)


\modHeadchapter{機能要件の洗い出し\TBW}
% 新たなマシニングセンタがどのような機能を持つべきかをリストアップ



\modHeadsection{必要な加工能力\TBW}
(to be written...)



\modHeadsection{必要な精度\TBW}
(to be written...)



\modHeadsection{必要な速度\TBW}
(to be written...)




\modHeadchapter{ソフトウェア要件の定義\TBW}
% 新たなマシニングセンタの操作や管理に必要なソフトウェアの要件を定義



\modHeadsection{ユーザーインターフェース\TBW}
(to be written...)



\modHeadsection{データ管理\TBW}
(to be written...)



\modHeadsection{セキュリティ\TBW}
(to be written...)



\modHeadsection{互換性\TBW}
(to be written...)




\modHeadchapter{非機能要件の考慮\TBW}
% システムの性能、信頼性、拡張性など、ソフトウェアの「どのように動作するべきか」に関する要件を定義



\modHeadsection{システムの性能\TBW}
(to be written...)



\modHeadsection{システムの信頼性\TBW}
(to be written...)



\modHeadsection{システムの拡張性\TBW}
(to be written...)




\modHeadchapter{制約の特定\TBW}
% 予算、時間、既存のシステムとの互換性など、プロジェクトに影響を与える可能性のある制約を特定



\modHeadsection{予算\TBW}
(to be written...)



\modHeadsection{時間\TBW}
(to be written...)



\modHeadsection{既存のシステムとの互換性\TBW}
(to be written...)




\modHeadchapter{要件の文書化と検証\TBW}
% すべての要件を文書化し、関係者全員が理解し、合意できることを確認



\modHeadsection{要件の文書化\TBW}
(to be written...)



\modHeadsection{要件の検証\TBW}
(to be written...)

%%%%%%%%%%%%%%%%%%%%%%%%%%%%%%%%%%%%%%%%%%%%%%%%%%%%%%%%%
%% Appendices %%%%%%%%%%%%%%%%%%%%%%%%%%%%%%%%%%%%%%%%%%%
%%%%%%%%%%%%%%%%%%%%%%%%%%%%%%%%%%%%%%%%%%%%%%%%%%%%%%%%%
\begin{appendices}
%\Appendixpart
\end{appendices}

\addtocontents{toc}{\protect\end{tocBox}}
