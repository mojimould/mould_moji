%!TEX root = ./RfCPN.tex


\addtocontents{toc}{\protect\cleardoublepage}
%%%%%%%%%%%%%%%%%%%%%%%%%%%%%%%%%%%%%%%%%%%%%%%%%%%%%%%%%
%% Part Requirement Definition %%%%%%%%%%%%%%%%%%%%%%%%%%
%%%%%%%%%%%%%%%%%%%%%%%%%%%%%%%%%%%%%%%%%%%%%%%%%%%%%%%%%
\addtocontents{toc}{\protect\begin{tocBox}{\tmppartnum}}%
\tPart{機械の稼働における\index{ようけんていぎ@要件定義}要件定義}{%
\paragraph*{\tpartgoal}
新たな横型マシニングセンタの稼働に関して、その具体的な\index{ようけんていぎ@要件定義}要件定義を行う。

\tcbline*
\paragraph*{\tpartmethod}
マシニングセンタを稼働し製品を生産することに焦点を絞る。
現状のマシニングセンタの仕様を基に、前段階で見た要改善点をすべて考慮した上で、\index{ようけんていぎ@要件定義}要件定義を行う。

\tcbline*
\paragraph*{\tpartbackground}
先にも述べた通り、(機械設置時点において)ソフトウェアの観点では、新たなマシニングセンタではとても生産ができるような状態にない
%% footnote %%%%%%%%%%%%%%%%%%%%%
\footnote{外注により作成された\index{NCプログラム(がいちゅう)@NCプログラム(外注)}NCプログラムは存在する。
しかし、当社が具体的な\index{ようけんていぎ@要件定義}要件定義を行えなかったため、実用に至っていない。
\index{NCプログラム(がいちゅう)@NCプログラム(外注)}NCプログラムそのものは尤もな内容でありレベルも十分に高いものであるが、(当部門に限らず)当社の\index{ソフトウェアエンジニアリング}ソフトウェアエンジニアリングに関する無関心による影響が顕在化した帰結となっている。}。
%%%%%%%%%%%%%%%%%%%%%%%%%%%%%%%%%
そのため、とにかく生産のできる状態にすることが目下かつ焦眉の課題である。\\
 その一環として、\index{システムかいはつプロセス@システム開発プロセス}システム開発プロセスの計画の策定を進めるために、その最初期の段階である\index{ようけんていぎ@要件定義}要件定義を早急に行う必要がある。
}{%
\paragraph*{\tpartconclusion}
\MMC における\expandafterindex{ぎょうむフロー(\yomiMMC)@業務フロー(\nameMMC)}業務フローを基に、\DMC の稼働に向けたソフトウェア視点における\index{ようけんていぎ@要件定義}要件定義を行った
%% footnote %%%%%%%%%%%%%%%%%%%%%
\footnote{\index{ようけんていぎ@要件定義}要件定義は一度だけ行われるものではない。
開発プロセスの進行に伴い新たな\index{ようけん@要件}要件の追加あるいは\index{ようけん@要件}要件の変更がされしだい、反復的に\index{ようけんていぎ@要件定義}要件定義が行われる。}。
%%%%%%%%%%%%%%%%%%%%%%%%%%%%%%%%%
\tcbline*
\paragraph*{\tpartnextstep}
\index{ようけんていぎ@要件定義}要件定義で特定された\index{ようけん@要件}要件を基に、次の段階である\index{システムせっけい@システム設計}システム設計を行う。
}

%%%%%%%%%%%%%%%%%%%%%%%%%%%%%%%%%%%%%%%%%%%%%%%%%%%%%%%%%%
%% chapters %%%%%%%%%%%%%%%%%%%%%%%%%%%%%%%%%%%%%%%%%%%%%%
%%%%%%%%%%%%%%%%%%%%%%%%%%%%%%%%%%%%%%%%%%%%%%%%%%%%%%%%%%
%!TEX root = ../RfCPN.tex


\modHeadchapter{機械稼働における達成したい目標・解決すべき課題}
新たに導入したマシニングセンタで何を達成したいのか、その目的さえも明確にされていないのが現状である
%% footnote %%%%%%%%%%%%%%%%%%%%%
\footnote{\MMC に改善の余地が多くある中、そもそもどのような論理で導入にまで至ったのか、とても不思議である。}。
%%%%%%%%%%%%%%%%%%%%%%%%%%%%%%%%%
したがって、ここでは暫定的に目的を定め、具体的な目標の設定を行うことにする。



%%%%%%%%%%%%%%%%%%%%%%%%%%%%%%%%%%%%%%%%%%%%%%%%%%%%%%%%%%
%% section 04.01 %%%%%%%%%%%%%%%%%%%%%%%%%%%%%%%%%%%%%%%%%
%%%%%%%%%%%%%%%%%%%%%%%%%%%%%%%%%%%%%%%%%%%%%%%%%%%%%%%%%%
\modHeadsection{\DMC 導入の目的}
主な目的としては\Dimple 加工を内製化することではあるが、それに伴って作業効率の低下や教育コストの増加が多大になってしまっては本末転倒である。
そのためここでは、(暫定的な)目的として以下を採用する。
なお、これらは\DMC 設置時点での\MMC の現状と比較したものである。
\begin{enumerate}[label=\sarrow]
\item 作業効率の大幅向上
\item 教育コストの大幅削減
\item \Dimple 加工の内製化
\end{enumerate}



%%%%%%%%%%%%%%%%%%%%%%%%%%%%%%%%%%%%%%%%%%%%%%%%%%%%%%%%%%
%% section 04.02 %%%%%%%%%%%%%%%%%%%%%%%%%%%%%%%%%%%%%%%%%
%%%%%%%%%%%%%%%%%%%%%%%%%%%%%%%%%%%%%%%%%%%%%%%%%%%%%%%%%%
\modHeadsection{達成したい目標}
達成したい目標として主に以下が挙げられる。
なお、これらは\DMC 設置時点での\MMC の現状と比較したものである。
\begin{enumerate}[label=\sarrow]
\item 作業員の安全性の考慮
\item 製品・コードの品質の考慮
\item 機械の信頼性の考慮
\item 諸作業に対する属人性の大幅削減
\end{enumerate}
なお、属人性の削減については、以下のような方針をとるものとする
%% footnote %%%%%%%%%%%%%%%%%%%%%
\footnote{\textgt{書類生成の自動化}や\textgt{コード生成の自動化}などもあるが、これらについては\pageautoref{part:XI}以降で取り扱う。}。
%%%%%%%%%%%%%%%%%%%%%%%%%%%%%%%%%
\begin{enumerate}[label=\sarrow]
\item \textgt{加工の自動化}:手作業による測定・加工の削減・簡易化
\item \textgt{操作の自動化}:手作業によるマシニングセンタ画面操作の削減・簡易化
\end{enumerate}


\clearpage
%%%%%%%%%%%%%%%%%%%%%%%%%%%%%%%%%%%%%%%%%%%%%%%%%%%%%%%%%%
%% section 04.03 %%%%%%%%%%%%%%%%%%%%%%%%%%%%%%%%%%%%%%%%%
%%%%%%%%%%%%%%%%%%%%%%%%%%%%%%%%%%%%%%%%%%%%%%%%%%%%%%%%%%
\modHeadsection{解決すべき課題\TBW}
解決すべき課題として、主に以下が挙げられる。
\begin{enumerate}[label=\sarrow]
\item 諸規程の策定
\item 諸規程に則った、諸標準の策定
\item 機内の幾何的情報の一般化および体系化
\item 諸標準に則った、各明細・各工程に対する\index{NCプログラム}NCプログラムの作成
\item モールドの関係データベースの作成
\end{enumerate}


%!TEX root = ../RfCPN.tex


\modHeadchapter{機械の稼働における\index{きのうようけん@機能要件}機能要件の洗い出し\TBW}
新たに導入するマシニングセンタはすでに決定している。
用いる工具等や内蔵ソフトウェアもすでに提供されているため、それらの\index{きのうようけん@機能要件}機能要件については触れない。
ここではその他の機能要件を挙げていく。



%%%%%%%%%%%%%%%%%%%%%%%%%%%%%%%%%%%%%%%%%%%%%%%%%%%%%%%%%%
%% section 05.01 %%%%%%%%%%%%%%%%%%%%%%%%%%%%%%%%%%%%%%%%%
%%%%%%%%%%%%%%%%%%%%%%%%%%%%%%%%%%%%%%%%%%%%%%%%%%%%%%%%%%
\modHeadsection{測定に必要な能力\TBW}
(to be written...)



%%%%%%%%%%%%%%%%%%%%%%%%%%%%%%%%%%%%%%%%%%%%%%%%%%%%%%%%%%
%% section 05.02 %%%%%%%%%%%%%%%%%%%%%%%%%%%%%%%%%%%%%%%%%
%%%%%%%%%%%%%%%%%%%%%%%%%%%%%%%%%%%%%%%%%%%%%%%%%%%%%%%%%%
\modHeadsection{加工に必要な能力\TBW}
(to be written...)



%%%%%%%%%%%%%%%%%%%%%%%%%%%%%%%%%%%%%%%%%%%%%%%%%%%%%%%%%%
%% section 05.03 %%%%%%%%%%%%%%%%%%%%%%%%%%%%%%%%%%%%%%%%%
%%%%%%%%%%%%%%%%%%%%%%%%%%%%%%%%%%%%%%%%%%%%%%%%%%%%%%%%%%
\modHeadsection{必要な精度\TBW}
(to be written...)



%%%%%%%%%%%%%%%%%%%%%%%%%%%%%%%%%%%%%%%%%%%%%%%%%%%%%%%%%%
%% section 05.04 %%%%%%%%%%%%%%%%%%%%%%%%%%%%%%%%%%%%%%%%%
%%%%%%%%%%%%%%%%%%%%%%%%%%%%%%%%%%%%%%%%%%%%%%%%%%%%%%%%%%
\modHeadsection{必要な速さ\TBW}
(to be written...)





%!TEX root = ../RfCPN.tex


\modHeadchapter{ソフトウェア要件の洗い出し\TBW}
導入するマシニングセンタ内にPCに専用ソフトウェアが内蔵されている。
つまり、すでに使用するソフトウェアは決定されているため、それについてのソフトウェア要件は挙げない。
ここでは作成する\index{NCプログラム}NCプログラムについての要件を挙げていく。





\modHeadchapter{非機能要件の考慮\TBW}
% システムの性能、信頼性、拡張性など、ソフトウェアの「どのように動作するべきか」に関する要件を定義



\modHeadsection{システムの性能\TBW}
(to be written...)



\modHeadsection{システムの信頼性\TBW}
(to be written...)



\modHeadsection{システムの拡張性\TBW}
(to be written...)




\modHeadchapter{制約の特定\TBW}
% 予算、時間、既存のシステムとの互換性など、プロジェクトに影響を与える可能性のある制約を特定



\modHeadsection{予算\TBW}
(to be written...)



\modHeadsection{時間\TBW}
(to be written...)



\modHeadsection{既存のシステムとの互換性\TBW}
(to be written...)




\modHeadchapter{要件の文書化と検証\TBW}
% すべての要件を文書化し、関係者全員が理解し、合意できることを確認



\modHeadsection{要件の文書化\TBW}
(to be written...)



\modHeadsection{要件の検証\TBW}
(to be written...)

%%%%%%%%%%%%%%%%%%%%%%%%%%%%%%%%%%%%%%%%%%%%%%%%%%%%%%%%%
%% Appendices %%%%%%%%%%%%%%%%%%%%%%%%%%%%%%%%%%%%%%%%%%%
%%%%%%%%%%%%%%%%%%%%%%%%%%%%%%%%%%%%%%%%%%%%%%%%%%%%%%%%%
\begin{appendices}
%\Appendixpart
\end{appendices}

\addtocontents{toc}{\protect\end{tocBox}}
