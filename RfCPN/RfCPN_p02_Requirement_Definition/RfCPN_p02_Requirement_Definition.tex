%!TEX root = ./RfCPN.tex


\addtocontents{toc}{\protect\cleardoublepage}
%%%%%%%%%%%%%%%%%%%%%%%%%%%%%%%%%%%%%%%%%%%%%%%%%%%%%%%%%
%% Part Requirement Definition %%%%%%%%%%%%%%%%%%%%%%%%%%
%%%%%%%%%%%%%%%%%%%%%%%%%%%%%%%%%%%%%%%%%%%%%%%%%%%%%%%%%
\addtocontents{toc}{\protect\begin{tocBox}{\tmppartnum}}%
\tPart{機械の稼働における\index{ようけんていぎ@要件定義}要件定義}{%
\paragraph*{目標(なにがしたいか?)}
新たな横型マシニングセンタの稼働に関して、その具体的な\index{ようけんていぎ@要件定義}要件定義を行う。

\tcbline*
\paragraph*{手段(どうやって?)}
マシニングセンタを稼働し製品を生産することに焦点を絞る。
現状のマシニングセンタの仕様を基に、前段階で見た改善案をすべて考慮した上で、\index{ようけんていぎ@要件定義}要件定義を行う。

\tcbline*
\paragraph*{背景(なぜ?)}
先にも述べた通り、(機械設置時点において)ソフトウェアの観点では、新たなマシニングセンタではとても生産ができるような状態にないことは明らかである
%% footnote %%%%%%%%%%%%%%%%%%%%%
\footnote{外注により作成された\index{NCプログラム(がいちゅう)@NCプログラム(外注)}NCプログラムは一応存在する。
しかし、当社が具体的な\index{ようけんていぎ@要件定義}要件定義さえ行えなかったため、全く実用に至っていない。
NCプログラムそのものは尤もな内容でありレベルも十分なものであるが、(当部門に限らず)当社の\index{ソフトウェアエンジニアリング}ソフトウェアエンジニアリングに関する致命的なほどの無関心が顕わに露呈したものであり、機械導入以前から予想されていたとおりの自明な帰結である。}。
%%%%%%%%%%%%%%%%%%%%%%%%%%%%%%%%%
そのため、とにかく生産のできる状態にすることが目下かつ焦眉の課題である。\\
 その一環として、\index{システムかいはつプロセス@システム開発プロセス}システム開発プロセスの計画において、その最初期の段階である\index{ようけんていぎ@要件定義}要件定義から始める必要がある。
}{%
\paragraph*{結論(どうなった?)}
\MMC における業務の流れを基に、\DMC の稼働に向けたソフトウェア視点における\index{ようけんていぎ@要件定義}要件定義を行った
%% footnote %%%%%%%%%%%%%%%%%%%%%
\footnote{\index{ようけんていぎ@要件定義}要件定義は一度だけ行われるものではない。
開発プロセスの進行に伴い新たな\index{要件}要件あるいは\index{要件}要件の変更がされしだい、反復的に\index{ようけんていぎ@要件定義}要件定義が行われる。}。
%%%%%%%%%%%%%%%%%%%%%%%%%%%%%%%%%
\tcbline*
\paragraph*{次の段階(それで?)}
\index{ようけんていぎ@要件定義}要件定義で特定された要件を基に、次の段階である\index{システムせっけい@システム設計}システム設計を行う。
}

%%%%%%%%%%%%%%%%%%%%%%%%%%%%%%%%%%%%%%%%%%%%%%%%%%%%%%%%%
%% chapters %%%%%%%%%%%%%%%%%%%%%%%%%%%%%%%%%%%%%%%%%%%%%%
%%%%%%%%%%%%%%%%%%%%%%%%%%%%%%%%%%%%%%%%%%%%%%%%%%%%%%%%%%
%!TEX root = ../RPA_for_Creating_Program_Note.tex


\modHeadchapter{はじめに\TBW}
% 本文書の目的と範囲について説明



%%%%%%%%%%%%%%%%%%%%%%%%%%%%%%%%%%%%%%%%%%%%%%%%%%%%%%%%%%
%% section 8.1 %%%%%%%%%%%%%%%%%%%%%%%%%%%%%%%%%%%%%%%%%%%
%%%%%%%%%%%%%%%%%%%%%%%%%%%%%%%%%%%%%%%%%%%%%%%%%%%%%%%%%%
\modHeadsection{文書の目的\TBW}



%%%%%%%%%%%%%%%%%%%%%%%%%%%%%%%%%%%%%%%%%%%%%%%%%%%%%%%%%%
%% section 8.2 %%%%%%%%%%%%%%%%%%%%%%%%%%%%%%%%%%%%%%%%%%%
%%%%%%%%%%%%%%%%%%%%%%%%%%%%%%%%%%%%%%%%%%%%%%%%%%%%%%%%%%
\modHeadsection{文書の範囲\TBW}


\modHeadchapter{機能要件の洗い出し\TBW}
% 新たなマシニングセンタがどのような機能を持つべきかをリストアップ



\modHeadsection{必要な加工能力\TBW}
(to be written...)



\modHeadsection{必要な精度\TBW}
(to be written...)



\modHeadsection{必要な速度\TBW}
(to be written...)




\modHeadchapter{ソフトウェア要件の定義\TBW}
% 新たなマシニングセンタの操作や管理に必要なソフトウェアの要件を定義



\modHeadsection{ユーザーインターフェース\TBW}
(to be written...)



\modHeadsection{データ管理\TBW}
(to be written...)



\modHeadsection{セキュリティ\TBW}
(to be written...)



\modHeadsection{互換性\TBW}
(to be written...)




\modHeadchapter{非機能要件の考慮\TBW}
% システムの性能、信頼性、拡張性など、ソフトウェアの「どのように動作するべきか」に関する要件を定義



\modHeadsection{システムの性能\TBW}
(to be written...)



\modHeadsection{システムの信頼性\TBW}
(to be written...)



\modHeadsection{システムの拡張性\TBW}
(to be written...)




\modHeadchapter{制約の特定\TBW}
% 予算、時間、既存のシステムとの互換性など、プロジェクトに影響を与える可能性のある制約を特定



\modHeadsection{予算\TBW}
(to be written...)



\modHeadsection{時間\TBW}
(to be written...)



\modHeadsection{既存のシステムとの互換性\TBW}
(to be written...)




\modHeadchapter{要件の文書化と検証\TBW}
% すべての要件を文書化し、関係者全員が理解し、合意できることを確認



\modHeadsection{要件の文書化\TBW}
(to be written...)



\modHeadsection{要件の検証\TBW}
(to be written...)

%%%%%%%%%%%%%%%%%%%%%%%%%%%%%%%%%%%%%%%%%%%%%%%%%%%%%%%%%
%% Appendices %%%%%%%%%%%%%%%%%%%%%%%%%%%%%%%%%%%%%%%%%%%
%%%%%%%%%%%%%%%%%%%%%%%%%%%%%%%%%%%%%%%%%%%%%%%%%%%%%%%%%
\begin{appendices}
%\Appendixpart
\end{appendices}

\addtocontents{toc}{\protect\end{tocBox}}
