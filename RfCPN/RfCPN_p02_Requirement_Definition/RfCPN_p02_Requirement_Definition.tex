%!TEX root = ./RfCPN.tex


\addtocontents{toc}{\protect\cleardoublepage}
%%%%%%%%%%%%%%%%%%%%%%%%%%%%%%%%%%%%%%%%%%%%%%%%%%%%%%%%%
%% Part Requirement Definition %%%%%%%%%%%%%%%%%%%%%%%%%%
%%%%%%%%%%%%%%%%%%%%%%%%%%%%%%%%%%%%%%%%%%%%%%%%%%%%%%%%%
\addtocontents{toc}{\protect\begin{tocBox}{\tmppartnum}}%
\tPart{加工システム作成における\index{ようけん@要件}要件}{%
\paragraph*{\tpartgoal}
新たな横型マシニングセンタの加工システムの作成に関して、その具体的な\index{ようけん@要件}要件を定める。

\tcbline*
\paragraph*{\tpartmethod}
前段階で見た要改善点をすべて考慮した上で、\DimpleMilling についても含めて\index{ようけん@要件}要件を洗い出す。

\tcbline*
\paragraph*{\tpartbackground}
当社の開発プロセスでは、その要件さえまともに定められていない。
そのため、加工システムの開発にあたって、まずはその要件を洗い出すことが目下かつ焦眉の課題である。
}{%
\paragraph*{\tpartconclusion}
\MMC における\expandafterindex{ぎょうむフロー(\yomiMMC)@業務フロー(\nameMMC)}業務フローを基に、\DMC の稼働に向けたソフトウェア視点における\index{ようけん@要件}要件を定めた
%% footnote %%%%%%%%%%%%%%%%%%%%%
\footnote{\index{ようけんていぎ@要件定義}要件の定義は一度だけ行われるものではない。
開発プロセスの進行に伴い新たな\index{ようけん@要件}要件の追加あるいは\index{ようけん@要件}要件の変更がされしだい、反復的に\index{ようけんていぎ@要件定義}要件が定められる。}。
%%%%%%%%%%%%%%%%%%%%%%%%%%%%%%%%%
\tcbline*
\paragraph*{\tpartnextstep}
定められた\index{ようけん@要件}要件を基に、\index{システムせっけい@システム設計}システム設計を行う。
}

%%%%%%%%%%%%%%%%%%%%%%%%%%%%%%%%%%%%%%%%%%%%%%%%%%%%%%%%%%
%% chapters %%%%%%%%%%%%%%%%%%%%%%%%%%%%%%%%%%%%%%%%%%%%%%
%%%%%%%%%%%%%%%%%%%%%%%%%%%%%%%%%%%%%%%%%%%%%%%%%%%%%%%%%%
%!TEX root = ../RfCPN.tex


\modHeadchapter{機械稼働における達成したい目標・解決すべき課題}
新たに導入したマシニングセンタで何を達成したいのか、その目的さえも明確にされていないのが現状である
%% footnote %%%%%%%%%%%%%%%%%%%%%
\footnote{\MMC に改善の余地が多くある中、そもそもどのような論理で導入にまで至ったのか、とても不思議である。}。
%%%%%%%%%%%%%%%%%%%%%%%%%%%%%%%%%
したがって、ここでは暫定的に目的を定め、具体的な目標の設定を行うことにする。



%%%%%%%%%%%%%%%%%%%%%%%%%%%%%%%%%%%%%%%%%%%%%%%%%%%%%%%%%%
%% section 04.01 %%%%%%%%%%%%%%%%%%%%%%%%%%%%%%%%%%%%%%%%%
%%%%%%%%%%%%%%%%%%%%%%%%%%%%%%%%%%%%%%%%%%%%%%%%%%%%%%%%%%
\modHeadsection{\DMC 導入の目的}
主な目的としては\Dimple 加工を内製化することではあるが、それに伴って作業効率の低下や教育コストの増加が多大になってしまっては本末転倒である。
そのためここでは、(暫定的な)目的として以下を採用する。
なお、これらは\DMC 設置時点での\MMC の現状と比較したものである。
\begin{enumerate}[label=\sarrow]
\item 作業効率の大幅向上
\item 教育コストの大幅削減
\item \Dimple 加工の内製化
\end{enumerate}



%%%%%%%%%%%%%%%%%%%%%%%%%%%%%%%%%%%%%%%%%%%%%%%%%%%%%%%%%%
%% section 04.02 %%%%%%%%%%%%%%%%%%%%%%%%%%%%%%%%%%%%%%%%%
%%%%%%%%%%%%%%%%%%%%%%%%%%%%%%%%%%%%%%%%%%%%%%%%%%%%%%%%%%
\modHeadsection{達成したい目標}
達成したい目標として主に以下が挙げられる。
なお、これらは\DMC 設置時点での\MMC の現状と比較したものである。
\begin{enumerate}[label=\sarrow]
\item 作業員の安全性の考慮
\item 製品・コードの品質の考慮
\item 機械の信頼性の考慮
\item 諸作業に対する属人性の大幅削減
\end{enumerate}
なお、属人性の削減については、以下のような方針をとるものとする
%% footnote %%%%%%%%%%%%%%%%%%%%%
\footnote{\textgt{書類生成の自動化}や\textgt{コード生成の自動化}などもあるが、これらについては\pageautoref{part:XI}以降で取り扱う。}。
%%%%%%%%%%%%%%%%%%%%%%%%%%%%%%%%%
\begin{enumerate}[label=\sarrow]
\item \textgt{加工の自動化}:手作業による測定・加工の削減・簡易化
\item \textgt{操作の自動化}:手作業によるマシニングセンタ画面操作の削減・簡易化
\end{enumerate}


\clearpage
%%%%%%%%%%%%%%%%%%%%%%%%%%%%%%%%%%%%%%%%%%%%%%%%%%%%%%%%%%
%% section 04.03 %%%%%%%%%%%%%%%%%%%%%%%%%%%%%%%%%%%%%%%%%
%%%%%%%%%%%%%%%%%%%%%%%%%%%%%%%%%%%%%%%%%%%%%%%%%%%%%%%%%%
\modHeadsection{解決すべき課題\TBW}
解決すべき課題として、主に以下が挙げられる。
\begin{enumerate}[label=\sarrow]
\item 諸規程の策定
\item 諸規程に則った、諸標準の策定
\item 機内の幾何的情報の一般化および体系化
\item 諸標準に則った、各明細・各工程に対する\index{NCプログラム}NCプログラムの作成
\item モールドの関係データベースの作成
\end{enumerate}


%!TEX root = ../RfCPN.tex


\modHeadchapter{機械の稼働における\index{きのうようけん@機能要件}機能要件の洗い出し\TBW}
新たに導入するマシニングセンタはすでに決定している。
用いる工具等や内蔵ソフトウェアもすでに提供されているため、それらの\index{きのうようけん@機能要件}機能要件については触れない。
ここではその他の機能要件を挙げていく。



%%%%%%%%%%%%%%%%%%%%%%%%%%%%%%%%%%%%%%%%%%%%%%%%%%%%%%%%%%
%% section 05.01 %%%%%%%%%%%%%%%%%%%%%%%%%%%%%%%%%%%%%%%%%
%%%%%%%%%%%%%%%%%%%%%%%%%%%%%%%%%%%%%%%%%%%%%%%%%%%%%%%%%%
\modHeadsection{測定に必要な能力\TBW}
(to be written...)



%%%%%%%%%%%%%%%%%%%%%%%%%%%%%%%%%%%%%%%%%%%%%%%%%%%%%%%%%%
%% section 05.02 %%%%%%%%%%%%%%%%%%%%%%%%%%%%%%%%%%%%%%%%%
%%%%%%%%%%%%%%%%%%%%%%%%%%%%%%%%%%%%%%%%%%%%%%%%%%%%%%%%%%
\modHeadsection{加工に必要な能力\TBW}
(to be written...)



%%%%%%%%%%%%%%%%%%%%%%%%%%%%%%%%%%%%%%%%%%%%%%%%%%%%%%%%%%
%% section 05.03 %%%%%%%%%%%%%%%%%%%%%%%%%%%%%%%%%%%%%%%%%
%%%%%%%%%%%%%%%%%%%%%%%%%%%%%%%%%%%%%%%%%%%%%%%%%%%%%%%%%%
\modHeadsection{必要な精度\TBW}
(to be written...)



%%%%%%%%%%%%%%%%%%%%%%%%%%%%%%%%%%%%%%%%%%%%%%%%%%%%%%%%%%
%% section 05.04 %%%%%%%%%%%%%%%%%%%%%%%%%%%%%%%%%%%%%%%%%
%%%%%%%%%%%%%%%%%%%%%%%%%%%%%%%%%%%%%%%%%%%%%%%%%%%%%%%%%%
\modHeadsection{必要な速さ\TBW}
(to be written...)





%!TEX root = ../RfCPN.tex


\modHeadchapter{加工システム作成における\index{ひきのうようけん@非機能要件}非機能要件}
% システムの性能、信頼性、拡張性など、ソフトウェアの「どのように動作するべきか」に関する要件を定義



%%%%%%%%%%%%%%%%%%%%%%%%%%%%%%%%%%%%%%%%%%%%%%%%%%%%%%%%%%
%% section 06.01 %%%%%%%%%%%%%%%%%%%%%%%%%%%%%%%%%%%%%%%%%
%%%%%%%%%%%%%%%%%%%%%%%%%%%%%%%%%%%%%%%%%%%%%%%%%%%%%%%%%%
\modHeadsection{加工システムの利便性}
\begin{enumerate}[label=\alph*)]
\item \index{NCプログラム}NCプログラムは保守・管理がしやすいように作成するように努める
\item \index{NCプログラム}NCプログラム作成者の都合を、マシニングセンタ管理者や作業者における操作の利便性より優先しない
%% footnote %%%%%%%%%%%%%%%%%%%%%
\footnote{自らの怠慢を理由に作業者らに負担を押し付けないようにする、という道徳的心得である。}
%%%%%%%%%%%%%%%%%%%%%%%%%%%%%%%%%
\item 作業者の操作の利便性を最優先し、次いでマシニングセンタ管理者の操作の利便性を優先する
\end{enumerate}



%%%%%%%%%%%%%%%%%%%%%%%%%%%%%%%%%%%%%%%%%%%%%%%%%%%%%%%%%%
%% section 06.02 %%%%%%%%%%%%%%%%%%%%%%%%%%%%%%%%%%%%%%%%%
%%%%%%%%%%%%%%%%%%%%%%%%%%%%%%%%%%%%%%%%%%%%%%%%%%%%%%%%%%
\modHeadsection{NCプログラムの引数の指定}
\begin{enumerate}[label=\alph*)]
\item NCプログラムの\index{ひきすう@引数}引数の指定は、原則として\index{ひきすうしていI@引数指定I}引数指定I(\pageautoref{chap:argumentSpecification}参照)を用いる
\item \index{NCメインプログラム}NCメインプログラムで用いる\index{NCサブプログラム}NCサブプログラムの\index{ひきすう@引数}引数の値は、原則として\index{ワーク}ワークの\index{ずめん(モールド)@図面(モールド)}図面に用いられる寸法値(数値)で与える
\end{enumerate}



%%%%%%%%%%%%%%%%%%%%%%%%%%%%%%%%%%%%%%%%%%%%%%%%%%%%%%%%%%
%% section 06.03 %%%%%%%%%%%%%%%%%%%%%%%%%%%%%%%%%%%%%%%%%
%%%%%%%%%%%%%%%%%%%%%%%%%%%%%%%%%%%%%%%%%%%%%%%%%%%%%%%%%%
\modHeadsection{NCプログラムのネスティング}
\begin{enumerate}[label=\alph*)]
\item NCプログラムの\index{いれここうぞう@入れ子構造}入れ子構造(\index{ネスティング}ネスティング)は、NCメインプログラムを最上位の階層として、原則として3重までとする
\end{enumerate}



%%%%%%%%%%%%%%%%%%%%%%%%%%%%%%%%%%%%%%%%%%%%%%%%%%%%%%%%%%
%% section 06.04 %%%%%%%%%%%%%%%%%%%%%%%%%%%%%%%%%%%%%%%%%
%%%%%%%%%%%%%%%%%%%%%%%%%%%%%%%%%%%%%%%%%%%%%%%%%%%%%%%%%%
\modHeadsection{NCプログラムに用いる座標}
\begin{enumerate}[label=\alph*)]
\item NCプログラム内での座標について、\index{3じげんユークリッドくうかんざひょう@3次元ユークリッド空間座標}3次元ユークリッド空間座標の使用を優先する
\end{enumerate}



%%%%%%%%%%%%%%%%%%%%%%%%%%%%%%%%%%%%%%%%%%%%%%%%%%%%%%%%%%
%% section 06.05 %%%%%%%%%%%%%%%%%%%%%%%%%%%%%%%%%%%%%%%%%
%%%%%%%%%%%%%%%%%%%%%%%%%%%%%%%%%%%%%%%%%%%%%%%%%%%%%%%%%%
\modHeadsection{異常値の検知}
\begin{enumerate}[label*=\alph*)]
\item NCプログラムの\index{ひきすう@引数}引数値や測定値, 計算値等に対する異常値を検知する機能を施す
\item 原則として、異常値を検知した場合はアラームを発生させ停止する
\end{enumerate}



%%%%%%%%%%%%%%%%%%%%%%%%%%%%%%%%%%%%%%%%%%%%%%%%%%%%%%%%%%
%% section 06.06 %%%%%%%%%%%%%%%%%%%%%%%%%%%%%%%%%%%%%%%%%
%%%%%%%%%%%%%%%%%%%%%%%%%%%%%%%%%%%%%%%%%%%%%%%%%%%%%%%%%%
\modHeadsection{加工システムのバックアップ}
\begin{enumerate}[label*=\alph*)]
\item 加工システムのファイル群は、原則としてすべてオンライン(クラウド上)に保存する
\item 保存先は、バージョン管理, ソースコード管理等の機能を有するものとする
\end{enumerate}



\clearpage
%%%%%%%%%%%%%%%%%%%%%%%%%%%%%%%%%%%%%%%%%%%%%%%%%%%%%%%%%%
%% section 06.07 %%%%%%%%%%%%%%%%%%%%%%%%%%%%%%%%%%%%%%%%%
%%%%%%%%%%%%%%%%%%%%%%%%%%%%%%%%%%%%%%%%%%%%%%%%%%%%%%%%%%
\modHeadsection{NCプログラムの公開}
\begin{enumerate}[label*=\alph*)]
\item 加工システムのファイル群は、すべてその\index{ちょさくけん@著作権}著作権の所有者を明らかにする
\item 原則として、加工システムのファイル群はすべてオンラインで一般公開する
\item 図面の寸法等を含むものについては非公開、あるいは代表的・典型的なものの公開に留める
\item 当社の職務著作物以外については、無断で公開しない
\end{enumerate}



%!TEX root = ../RfCPN.tex


\modHeadchapter{制約の特定}
% 予算、時間、既存のシステムとの互換性など、プロジェクトに影響を与える可能性のある制約を特定



\modHeadsection{人員の制約}
前述のとおり、加工システムの作成に関しては、事実上(管理職レベルのものも含め)職務がすべて放棄されている
%% footnote %%%%%%%%%%%%%%%%%%%%%
\footnote{ある管理職いわく、「この業務は会社の重要事項ではない」と断言された。}。
%%%%%%%%%%%%%%%%%%%%%%%%%%%%%%%%%
したがって加工システム作成作業は、そのすべてにおいて、1人(本書の著者)のみで行わなければならない状態にある。



\modHeadsection{開発期間の制約}
開発期間に関する制約は何も与えられていない。
そこで、ここでは稼働(製品の加工)までの概ねの目安として、\Dimple についてのNCプログラム作成に1年、その他の作業に1年の、計2年程度を目指すものとする。
%%%%%%%%%%%%%%%%%%%%%%%%%%%%%%%%%%%%%%%%%%%%%%%%%%%%%%%%%%
%% marker %%%%%%%%%%%%%%%%%%%%%%%%%%%%%%%%%%%%%%%%%%%%%%%%
%%%%%%%%%%%%%%%%%%%%%%%%%%%%%%%%%%%%%%%%%%%%%%%%%%%%%%%%%%
\begin{marker}
追記:2023年9月下旬に着手し、2024年3月下旬頃に\index{ほんばんかんきょう@本番環境}本番環境での製品の加工に至った。
\end{marker}
%%%%%%%%%%%%%%%%%%%%%%%%%%%%%%%%%%%%%%%%%%%%%%%%%%%%%%%%%%
%%%%%%%%%%%%%%%%%%%%%%%%%%%%%%%%%%%%%%%%%%%%%%%%%%%%%%%%%%
%%%%%%%%%%%%%%%%%%%%%%%%%%%%%%%%%%%%%%%%%%%%%%%%%%%%%%%%%%



\modHeadsection{加工の空間的制約\TBW}
(to be written...)



\modHeadsection{内蔵メモリの記憶容量の制約\TBW}
(to be written...)



%%%%%%%%%%%%%%%%%%%%%%%%%%%%%%%%%%%%%%%%%%%%%%%%%%%%%%%%%
%% Appendices %%%%%%%%%%%%%%%%%%%%%%%%%%%%%%%%%%%%%%%%%%%
%%%%%%%%%%%%%%%%%%%%%%%%%%%%%%%%%%%%%%%%%%%%%%%%%%%%%%%%%
\begin{appendices}
%\Appendixpart
\end{appendices}

\addtocontents{toc}{\protect\end{tocBox}}
