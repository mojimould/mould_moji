%!TEX root = ../RfCPN.tex


\modHeadchapter{加工システム作成における\index{きのうようけん@機能要件}機能要件}
ここでは加工システム(\index{NCプログラム}NCプログラム群)を作成する際の\index{きのうようけん@機能要件}機能要件を挙げる。
なお、青色(\,\sarrow\!)の項目は、優先度が低いものを示す。


%%%%%%%%%%%%%%%%%%%%%%%%%%%%%%%%%%%%%%%%%%%%%%%%%%%%%%%%%%
%% section 05.01 %%%%%%%%%%%%%%%%%%%%%%%%%%%%%%%%%%%%%%%%%
%%%%%%%%%%%%%%%%%%%%%%%%%%%%%%%%%%%%%%%%%%%%%%%%%%%%%%%%%%
\modHeadsection{ワークの幾何的情報の解析計算および体系化}


%%%%%%%%%%%%%%%%%%%%%%%%%%%%%%%%%%%%%%%%%%%%%%%%%%%%%%%%%%
%% subsection 05.01.01 %%%%%%%%%%%%%%%%%%%%%%%%%%%%%%%%%%%
%%%%%%%%%%%%%%%%%%%%%%%%%%%%%%%%%%%%%%%%%%%%%%%%%%%%%%%%%%
\subsection{\Table 回転による振分調整における幾何的情報の体系化}
\begin{enumerate}[label={\sarrow[red]}]
\item \Table 回転後の\AlocationLength(\ReAlocationLength)の解析的導出
\item \ReAlocationLength 指定時の\Table 回転角度の解析的導出
\end{enumerate}


%%%%%%%%%%%%%%%%%%%%%%%%%%%%%%%%%%%%%%%%%%%%%%%%%%%%%%%%%%
%% subsection 05.01.02 %%%%%%%%%%%%%%%%%%%%%%%%%%%%%%%%%%%
%%%%%%%%%%%%%%%%%%%%%%%%%%%%%%%%%%%%%%%%%%%%%%%%%%%%%%%%%%
\subsection{ワーク座標系原点設定における幾何的情報の体系化}
\begin{enumerate}[label={\sarrow[red]}]
\item \Table 回転後の、図面上の端面外側中心座標の解析的導出
\item \Table 回転後の、図面上の\OutcutCenter 座標の解析的導出
\item \Table 回転後の、図面上の端面内側中心座標の解析的導出
\item \Table 回転後の、図面上の\KeywayCenter 座標の解析的導出
\item 測定箇所変更時に対する補正の解析的導出
\end{enumerate}


%%%%%%%%%%%%%%%%%%%%%%%%%%%%%%%%%%%%%%%%%%%%%%%%%%%%%%%%%%
%% subsection 05.01.03 %%%%%%%%%%%%%%%%%%%%%%%%%%%%%%%%%%%
%%%%%%%%%%%%%%%%%%%%%%%%%%%%%%%%%%%%%%%%%%%%%%%%%%%%%%%%%%
\subsection{\CenterlineEndFaceDifMeasurement における幾何的情報の体系化}
\begin{enumerate}[label={\sarrow[red]}]
\item \CenterlineEndFaceDifMeasurement 位置座標の解析的導出
\end{enumerate}


%%%%%%%%%%%%%%%%%%%%%%%%%%%%%%%%%%%%%%%%%%%%%%%%%%%%%%%%%%
%% subsection 05.01.05 %%%%%%%%%%%%%%%%%%%%%%%%%%%%%%%%%%%
%%%%%%%%%%%%%%%%%%%%%%%%%%%%%%%%%%%%%%%%%%%%%%%%%%%%%%%%%%
\subsection{\KeywayMilling における幾何的情報の体系化}
\begin{enumerate}[label={\sarrow[red]}]
\item[\sarrow] \index{デプスゲージ}デプスゲージでの測定における\AsideKeywayDepth の補正の解析的導出
\end{enumerate}


%%%%%%%%%%%%%%%%%%%%%%%%%%%%%%%%%%%%%%%%%%%%%%%%%%%%%%%%%%
%% subsection 05.01.06 %%%%%%%%%%%%%%%%%%%%%%%%%%%%%%%%%%%
%%%%%%%%%%%%%%%%%%%%%%%%%%%%%%%%%%%%%%%%%%%%%%%%%%%%%%%%%%
\subsection{\EndFaceOutChamferMilling における幾何的情報の体系化}
\begin{enumerate}[label={\sarrow[red]}]
\item \EndFaceOutChamferLength に対する加工中心座標の補正の解析的導出
\item[\sarrow] 傾いた\EndFaceOutChamfer におけるコーナー部の位置座標の解析的導出
\item \EndFaceOutRChamfer の接面の解析的導出
\end{enumerate}


%%%%%%%%%%%%%%%%%%%%%%%%%%%%%%%%%%%%%%%%%%%%%%%%%%%%%%%%%%
%% subsection 05.01.07 %%%%%%%%%%%%%%%%%%%%%%%%%%%%%%%%%%%
%%%%%%%%%%%%%%%%%%%%%%%%%%%%%%%%%%%%%%%%%%%%%%%%%%%%%%%%%%
\subsection{\EndFaceInChamferMilling における幾何的情報の体系化}
\begin{enumerate}[label={\sarrow[red]}]
\item \EndFaceInChamferLength に対する加工中心座標の補正の解析的導出
\item[\sarrow] 傾いた\EndFaceInChamfer におけるコーナー部の位置座標の解析的導出
\item \EndFaceInRChamfer の接面の解析的導出
\end{enumerate}


\clearpage
%%%%%%%%%%%%%%%%%%%%%%%%%%%%%%%%%%%%%%%%%%%%%%%%%%%%%%%%%%
%% subsection 05.01.08 %%%%%%%%%%%%%%%%%%%%%%%%%%%%%%%%%%%
%%%%%%%%%%%%%%%%%%%%%%%%%%%%%%%%%%%%%%%%%%%%%%%%%%%%%%%%%%
\subsection{\DimpleMilling における幾何的情報の体系化}
\begin{enumerate}[label={\sarrow[red]}]
\item \Dimple 部分の\InnerDiameter の寸法の解析的導出
\item \index{アンダーカット}アンダーカット回避用\Table 回転角の解析的導出
\item \Table 回転前における任意の\Dimple の位置座標の解析的導出
\item \Table 回転後における任意の\Dimple の位置座標の解析的導出
\item \Table 回転後における、B, D面の\Dimple 列間距離の解析的導出
\item \Table 回転後における、A, C面の\Dimple 列間距離の解析的導出
\item \Table 回転後における、B, D面の同列内の\Dimple 間距離の解析的導出
\item \Table 回転後における、A, C面の同列内の\Dimple 間距離の解析的導出
\end{enumerate}


%%%%%%%%%%%%%%%%%%%%%%%%%%%%%%%%%%%%%%%%%%%%%%%%%%%%%%%%%%
%% subsection 05.01.09 %%%%%%%%%%%%%%%%%%%%%%%%%%%%%%%%%%%
%%%%%%%%%%%%%%%%%%%%%%%%%%%%%%%%%%%%%%%%%%%%%%%%%%%%%%%%%%
\subsection{その他の幾何的情報の体系化}

%%%%%%%%%%%%%%%%%%%%%%%%%%%%%%%%%%%%%%%%%%%%%%%%%%%%%%%%%%
%% subsubsection 05.01.09.1 %%%%%%%%%%%%%%%%%%%%%%%%%%%%%%
%%%%%%%%%%%%%%%%%%%%%%%%%%%%%%%%%%%%%%%%%%%%%%%%%%%%%%%%%%
\subsubsection{\EndFaceBoringMilling における幾何的情報の体系化}
\begin{enumerate}[label={\sarrow[red]}]
\item[\sarrow] \EndFaceBoring におけるコーナー円の中心座標の解析的導出
\end{enumerate}

%%%%%%%%%%%%%%%%%%%%%%%%%%%%%%%%%%%%%%%%%%%%%%%%%%%%%%%%%%
%% subsubsection 05.01.09.2 %%%%%%%%%%%%%%%%%%%%%%%%%%%%%%
%%%%%%%%%%%%%%%%%%%%%%%%%%%%%%%%%%%%%%%%%%%%%%%%%%%%%%%%%%
\subsubsection{\CurvedOutcutMilling における幾何的情報の体系化}
\begin{enumerate}[label={\sarrow[red]}]
\item[\sarrow] \CurvedOutcut における加工中心位置座標および加工径の解析的導出
\item[\sarrow] \CurvedOutcut における\EndFace の位置座標の解析的導出
\end{enumerate}

%%%%%%%%%%%%%%%%%%%%%%%%%%%%%%%%%%%%%%%%%%%%%%%%%%%%%%%%%%
%% subsubsection 05.01.09.3 %%%%%%%%%%%%%%%%%%%%%%%%%%%%%%
%%%%%%%%%%%%%%%%%%%%%%%%%%%%%%%%%%%%%%%%%%%%%%%%%%%%%%%%%%
\subsubsection{\InnerDiameter における幾何的情報の体系化}
\begin{enumerate}[label={\sarrow[red]}]
\item[\sarrow] \index{テーパ}テーパを考慮した\InnerDiameter の解析的導出
\end{enumerate}



\clearpage
%%%%%%%%%%%%%%%%%%%%%%%%%%%%%%%%%%%%%%%%%%%%%%%%%%%%%%%%%%
%% section 05.02 %%%%%%%%%%%%%%%%%%%%%%%%%%%%%%%%%%%%%%%%%
%%%%%%%%%%%%%%%%%%%%%%%%%%%%%%%%%%%%%%%%%%%%%%%%%%%%%%%%%%
\modHeadsection{\index{NCプログラム}NCプログラム作成に関する諸基準の規格化}
%%%%%%%%%%%%%%%%%%%%%%%%%%%%%%%%%%%%%%%%%%%%%%%%%%%%%%%%%%
%% marker %%%%%%%%%%%%%%%%%%%%%%%%%%%%%%%%%%%%%%%%%%%%%%%%
%%%%%%%%%%%%%%%%%%%%%%%%%%%%%%%%%%%%%%%%%%%%%%%%%%%%%%%%%%
\begin{marker}
大前提として、\index{NCプログラム}NCプログラムの作成に関しては、原則として\textbf{経営陣・管理職らの方針・意見等は一切考慮しない}。
ここではマシニングセンタの直接的な関係者ら(特に作業者)の意見のみを尊重・優先する。
\tcbline*
これは、経営陣・管理職らの(四半世紀以上にわたる)これまでの姿勢・モラル・能力を鑑みて、極めて自然かつ当然かつ必然の措置である。
これを考慮してしまうと、安全性・品質・費用・作業効率・維持管理・信頼性など、どの観点においてもその機能が著しく損なわれ、そもそも作成自体が不可能になることが容易に想定される
%% footnote %%%%%%%%%%%%%%%%%%%%%
\footnote{事実、機械設置時点までの当社の開発計画は支離滅裂で論理が破綻しており、すぐさま頓挫し、一切の反省もされないまま放置されている。
これは事前に(容易に)予測されていたことである。}。
%%%%%%%%%%%%%%%%%%%%%%%%%%%%%%%%%
\end{marker}
%%%%%%%%%%%%%%%%%%%%%%%%%%%%%%%%%%%%%%%%%%%%%%%%%%%%%%%%%%
%%%%%%%%%%%%%%%%%%%%%%%%%%%%%%%%%%%%%%%%%%%%%%%%%%%%%%%%%%
%%%%%%%%%%%%%%%%%%%%%%%%%%%%%%%%%%%%%%%%%%%%%%%%%%%%%%%%%%


%%%%%%%%%%%%%%%%%%%%%%%%%%%%%%%%%%%%%%%%%%%%%%%%%%%%%%%%%%
%% subsection 05.02.01 %%%%%%%%%%%%%%%%%%%%%%%%%%%%%%%%%%%
%%%%%%%%%%%%%%%%%%%%%%%%%%%%%%%%%%%%%%%%%%%%%%%%%%%%%%%%%%
\subsection{ソフトウェアエンジニアリングに関する諸規程の策定}
\begin{enumerate}[label={\sarrow[red]}]
\item \index{じょうほうぎじゅつしゃ@情報技術者}情報技術者およびその\index{ぎじゅつすいじゅん@技術水準}技術水準に関する規程の策定
\item システムおよびソフトウェアの作成に関する規程の策定
\item \index{ちょさくぶつ@著作物}著作物および\index{ちょさくぶつのていじ@著作物の提示}その提示に関する規程の策定
\end{enumerate}


%%%%%%%%%%%%%%%%%%%%%%%%%%%%%%%%%%%%%%%%%%%%%%%%%%%%%%%%%%
%% subsection 05.02.01 %%%%%%%%%%%%%%%%%%%%%%%%%%%%%%%%%%%
%%%%%%%%%%%%%%%%%%%%%%%%%%%%%%%%%%%%%%%%%%%%%%%%%%%%%%%%%%
\subsection{マシニングセンタに関する諸標準の策定}
\begin{enumerate}[label={\sarrow[red]}]
\item マシニングセンタにおけるワークの寸法に関する標準の策定
\item \index{NCプログラムばんごう@NCプログラム番号}NCプログラムの番号付けに関する標準の策定
\item 諸工具の\index{おくりはやさ@送り速さ}送り速さに関する標準の策定
\item 諸工具の\index{しゅじくかいてんすう@主軸回転数}主軸回転数に関する標準の策定
\item \index{NCプログラム}NCプログラムの\index{シーケンスばんごう@シーケンス番号}シーケンス番号に関する標準の策定
\item[\sarrow] \index{NCプログラム}NCプログラムの\index{アラームばんごう@アラーム番号}アラーム番号に関する標準の策定
\item マシニングセンタごとの\index{コモンへんすう@コモン変数}コモン変数に関する標準の策定
\item マシニングセンタについての\index{ちょさくぶつ@著作物}著作物および\index{ちょさくぶつのていじ@著作物の提示}その提示に関する標準の策定
\end{enumerate}



\clearpage
%%%%%%%%%%%%%%%%%%%%%%%%%%%%%%%%%%%%%%%%%%%%%%%%%%%%%%%%%%
%% section 05.03 %%%%%%%%%%%%%%%%%%%%%%%%%%%%%%%%%%%%%%%%%
%%%%%%%%%%%%%%%%%%%%%%%%%%%%%%%%%%%%%%%%%%%%%%%%%%%%%%%%%%
\modHeadsection{具体的なNCプログラムの作成}


%%%%%%%%%%%%%%%%%%%%%%%%%%%%%%%%%%%%%%%%%%%%%%%%%%%%%%%%%%
%% subsection 05.03.02 %%%%%%%%%%%%%%%%%%%%%%%%%%%%%%%%%%%
%%%%%%%%%%%%%%%%%%%%%%%%%%%%%%%%%%%%%%%%%%%%%%%%%%%%%%%%%%
\subsection{NCプログラム作成の基本方針}
具体的な\index{NCプログラム}NCプログラムの作成に際して、基本的には以下のような方針を取ることにする。
\begin{enumerate}[label*=\alph*)]
\item \textgt{利便性の優先度}
%% footnote %%%%%%%%%%%%%%%%%%%%%
\footnote{自らの怠慢を理由に作業者らに負担を押し付けないようにする、というような道徳的心得である}\\
%%%%%%%%%%%%%%%%%%%%%%%%%%%%%%%%%
\index{NCプログラム}NCプログラムは保守・管理がしやすいように作成するように努める。
ただし、\index{NCプログラム}NCプログラム作成者の都合を、マシニングセンタ管理者や作業者における操作の利便性より優先することのないよう作成を行うものとする。
なお、作業者における操作の利便性を最優先にし、次いでマシニングセンタ管理者における操作の利便性を優先する。
\item \textgt{\index{3D CAD}\textbf{3D CAD}・\index{CAM}\textbf{CAM}の使用について}\\
\index{3D CAD}3D CAD・\index{CAM}CAMの使用は、\index{そせいかこうひん@塑性加工品}塑性加工品の加工との相性が悪く非現実的である。
また、当社の能力から見ても使用は非現実的であることが明白である。
さらには、加工の対象となるワークはそのほとんどが直線と円弧のみで表される単純な構造であり、幾何的情報は\index{しょとうきかがく@初等幾何学}初等幾何学で解析的導出が可能である。
こうした事情から、\index{3D CAD}3D CAD・\index{CAM}CAMを用いた\index{NCプログラム}作成は行わない。
\item \textgt{\Spacer の使用について}\\
\Spacer を用いた\AlocationAdjustment は、明細ごとに\Spacer を手配する必要があり、その維持管理や費用の観点から非現実的である。
したがって、\Spacer を用いた\AlocationAdjustment は原則として行わない。
代替として、\Table の回転を用いた\AlocationAdjustment を行う
%% footnote %%%%%%%%%%%%%%%%%%%%%
\footnote{\Spacer の使用は、本質的に(B軸方向の)回転と同義である。}。
%%%%%%%%%%%%%%%%%%%%%%%%%%%%%%%%%
\item \textgt{\EndFace 部の荒削り加工工程について}\\
\DimpleMilling に伴う\index{アンダーカット}アンダーカットを回避する手段として、\EndFace 部の荒削り加工が考えられる。
しかし、\DMC には荒削り加工した後の清掃機能が備わっておらず、必然的に人手による清掃が必要となる。
したがって作業効率が大きく低下するため、\EndFace 部の荒削り加工は行わない。
\item \textgt{\textbf{NC}プログラムに用いる引数}\\
NCプログラムの\index{ひきすう@引数}は、原則として\index{ひきすうしていI@引数指定I}引数指定Iを用いる。\\
 また、\index{NCメインプログラム}NCメインプログラムで用いる各工程用\index{NCサブプログラム}NCサブプログラムの\index{ひきすう@引数}引数の値は、原則として\index{ワーク}ワークの\index{ずめん(モールド)@図面(モールド)}図面に用いられている寸法値(数値)で与えるものとする。
\index{NCサブプログラム}NCサブプログラム内で用いるNCサブプログラムの引数の値については問わない。
\item \textgt{\index{NCプログラム}\textbf{NC}プログラムの\index{ネスティング}ネスティング}\\
NCプログラムの\index{いれここうぞう@入れ子構造}入れ子構造(\index{ネスティング}ネスティング)は、NCメインプログラムを最上位の階層として、原則として3重までとする。
\item \textgt{衝突予防策}\\
どのNCプログラムにも、原則として衝突予防の機能を施す。
\item \textgt{異常値の検知}\\
どのNCプログラムにも、原則として引数や変数, 測定値, 計算値等に対する異常値を検知する機能を施す。
\end{enumerate}


\clearpage
%%%%%%%%%%%%%%%%%%%%%%%%%%%%%%%%%%%%%%%%%%%%%%%%%%%%%%%%%%
%% subsection 05.03.01 %%%%%%%%%%%%%%%%%%%%%%%%%%%%%%%%%%%
%%%%%%%%%%%%%%%%%%%%%%%%%%%%%%%%%%%%%%%%%%%%%%%%%%%%%%%%%%
\subsection{\index{げんてんせってい@原点設定}原点設定用NCサブプログラムの作成\TBW}


%%%%%%%%%%%%%%%%%%%%%%%%%%%%%%%%%%%%%%%%%%%%%%%%%%%%%%%%%%
%% subsection 05.03.02 %%%%%%%%%%%%%%%%%%%%%%%%%%%%%%%%%%%
%%%%%%%%%%%%%%%%%%%%%%%%%%%%%%%%%%%%%%%%%%%%%%%%%%%%%%%%%%
\subsection{\EndFacecutMilling 用NCサブプログラムの作成\TBW}
\begin{enumerate}[label=\sarrow]
\item \IDCenter を基準とした\EndFacecutMilling に変更
\item \ODCornerR への対応
\item \index{けずりしろ@削り代}削り代および加工回数の変更の簡易化
\end{enumerate}


%%%%%%%%%%%%%%%%%%%%%%%%%%%%%%%%%%%%%%%%%%%%%%%%%%%%%%%%%%
%% subsection 05.03.03 %%%%%%%%%%%%%%%%%%%%%%%%%%%%%%%%%%%
%%%%%%%%%%%%%%%%%%%%%%%%%%%%%%%%%%%%%%%%%%%%%%%%%%%%%%%%%%
\subsection{\OutcutMilling 用NCサブプログラムの作成\TBW}
\begin{enumerate}[label=\sarrow]
\item \CurvedOutcutMilling の機械化
\end{enumerate}


%%%%%%%%%%%%%%%%%%%%%%%%%%%%%%%%%%%%%%%%%%%%%%%%%%%%%%%%%%
%% subsection 05.03.04 %%%%%%%%%%%%%%%%%%%%%%%%%%%%%%%%%%%
%%%%%%%%%%%%%%%%%%%%%%%%%%%%%%%%%%%%%%%%%%%%%%%%%%%%%%%%%%
\subsection{\KeywayMilling 用NCサブプログラムの作成\TBW}
\begin{enumerate}[label=\sarrow]
\item \KeywayPos・\KeywayWidth 補正変更作業の簡易化
\item \KeywayWidth に応じた加工回数の変更作業の機械化
\end{enumerate}


%%%%%%%%%%%%%%%%%%%%%%%%%%%%%%%%%%%%%%%%%%%%%%%%%%%%%%%%%%
%% subsection 05.03.05 %%%%%%%%%%%%%%%%%%%%%%%%%%%%%%%%%%%
%%%%%%%%%%%%%%%%%%%%%%%%%%%%%%%%%%%%%%%%%%%%%%%%%%%%%%%%%%
\subsection{\EndFaceChamferMilling 用NCサブプログラムの作成\TBW}
\begin{enumerate}[label=\sarrow]
\item \EndFaceChamferMilling の機械化
\item AC方向の位置調整作業の機械化
\end{enumerate}


%%%%%%%%%%%%%%%%%%%%%%%%%%%%%%%%%%%%%%%%%%%%%%%%%%%%%%%%%%
%% subsection 05.03.05 %%%%%%%%%%%%%%%%%%%%%%%%%%%%%%%%%%%
%%%%%%%%%%%%%%%%%%%%%%%%%%%%%%%%%%%%%%%%%%%%%%%%%%%%%%%%%%
\subsection{\Dimple 用NCサブプログラムの作成\TBW}
\begin{enumerate}[label=\sarrow]
\item
\end{enumerate}


