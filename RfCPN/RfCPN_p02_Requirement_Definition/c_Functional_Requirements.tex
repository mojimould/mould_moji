%!TEX root = ../RfCPN.tex


\modHeadchapter{加工システム作成における\index{ようけん@要件}要件}
ここでは加工システム(\index{NCプログラム}NCプログラム群)を作成する際の\index{ようけん@要件}要件を挙げる。
なお、青色(\,\sarrow\!)の項目は優先度が低いものを示す。


%%%%%%%%%%%%%%%%%%%%%%%%%%%%%%%%%%%%%%%%%%%%%%%%%%%%%%%%%%
%% section 05.01 %%%%%%%%%%%%%%%%%%%%%%%%%%%%%%%%%%%%%%%%%
%%%%%%%%%%%%%%%%%%%%%%%%%%%%%%%%%%%%%%%%%%%%%%%%%%%%%%%%%%
\modHeadsection{ワークの幾何的情報の体系的把握}


%%%%%%%%%%%%%%%%%%%%%%%%%%%%%%%%%%%%%%%%%%%%%%%%%%%%%%%%%%
%% subsection 05.01.01 %%%%%%%%%%%%%%%%%%%%%%%%%%%%%%%%%%%
%%%%%%%%%%%%%%%%%%%%%%%%%%%%%%%%%%%%%%%%%%%%%%%%%%%%%%%%%%
\subsection{\Table 回転による振分調整における幾何的情報の体系化}
\begin{enumerate}[label={\sarrow[red]}]
\item \Table 回転後の\AlocationLength(\ReAlocationLength)の解析的導出
\item \ReAlocationLength 指定時の\Table 回転角度の解析的導出
\end{enumerate}


%%%%%%%%%%%%%%%%%%%%%%%%%%%%%%%%%%%%%%%%%%%%%%%%%%%%%%%%%%
%% subsection 05.01.02 %%%%%%%%%%%%%%%%%%%%%%%%%%%%%%%%%%%
%%%%%%%%%%%%%%%%%%%%%%%%%%%%%%%%%%%%%%%%%%%%%%%%%%%%%%%%%%
\subsection{ワーク座標系原点設定における幾何的情報の体系化}
\begin{enumerate}[label={\sarrow[red]}]
\item \Table 回転後の、図面上の端面外側中心座標の解析的導出
\item \Table 回転後の、図面上の\OutcutCenter 座標の解析的導出
\item \Table 回転後の、図面上の端面内側中心座標の解析的導出
\item \Table 回転後の、図面上の\KeywayCenter 座標の解析的導出
\item 測定箇所変更時に対する補正の解析的導出
\end{enumerate}


%%%%%%%%%%%%%%%%%%%%%%%%%%%%%%%%%%%%%%%%%%%%%%%%%%%%%%%%%%
%% subsection 05.01.03 %%%%%%%%%%%%%%%%%%%%%%%%%%%%%%%%%%%
%%%%%%%%%%%%%%%%%%%%%%%%%%%%%%%%%%%%%%%%%%%%%%%%%%%%%%%%%%
\subsection{\CenterlineEndFaceDifMeasurement における幾何的情報の体系化}
\begin{enumerate}[label={\sarrow[red]}]
\item \CenterlineEndFaceDifMeasurement 位置座標の解析的導出
\end{enumerate}


%%%%%%%%%%%%%%%%%%%%%%%%%%%%%%%%%%%%%%%%%%%%%%%%%%%%%%%%%%
%% subsection 05.01.05 %%%%%%%%%%%%%%%%%%%%%%%%%%%%%%%%%%%
%%%%%%%%%%%%%%%%%%%%%%%%%%%%%%%%%%%%%%%%%%%%%%%%%%%%%%%%%%
\subsection{\KeywayMilling における幾何的情報の体系化}
\begin{enumerate}[label={\sarrow[red]}]
\item[\sarrow] \index{デプスゲージ}デプスゲージでの測定における\AsideKeywayDepth の補正の解析的導出
\end{enumerate}


%%%%%%%%%%%%%%%%%%%%%%%%%%%%%%%%%%%%%%%%%%%%%%%%%%%%%%%%%%
%% subsection 05.01.06 %%%%%%%%%%%%%%%%%%%%%%%%%%%%%%%%%%%
%%%%%%%%%%%%%%%%%%%%%%%%%%%%%%%%%%%%%%%%%%%%%%%%%%%%%%%%%%
\subsection{\EndFaceOutChamferMilling における幾何的情報の体系化}
\begin{enumerate}[label={\sarrow[red]}]
\item \EndFaceOutChamferLength に対する加工中心座標の補正の解析的導出
\item[\sarrow] 傾いた\EndFaceOutChamfer におけるコーナー部の位置座標の解析的導出
\item \EndFaceOutRChamfer の接面の解析的導出
\end{enumerate}


%%%%%%%%%%%%%%%%%%%%%%%%%%%%%%%%%%%%%%%%%%%%%%%%%%%%%%%%%%
%% subsection 05.01.07 %%%%%%%%%%%%%%%%%%%%%%%%%%%%%%%%%%%
%%%%%%%%%%%%%%%%%%%%%%%%%%%%%%%%%%%%%%%%%%%%%%%%%%%%%%%%%%
\subsection{\EndFaceInChamferMilling における幾何的情報の体系化}
\begin{enumerate}[label={\sarrow[red]}]
\item \EndFaceInChamferLength に対する加工中心座標の補正の解析的導出
\item[\sarrow] 傾いた\EndFaceInChamfer におけるコーナー部の位置座標の解析的導出
\item \EndFaceInRChamfer の接面の解析的導出
\end{enumerate}


\clearpage
%%%%%%%%%%%%%%%%%%%%%%%%%%%%%%%%%%%%%%%%%%%%%%%%%%%%%%%%%%
%% subsection 05.01.08 %%%%%%%%%%%%%%%%%%%%%%%%%%%%%%%%%%%
%%%%%%%%%%%%%%%%%%%%%%%%%%%%%%%%%%%%%%%%%%%%%%%%%%%%%%%%%%
\subsection{\DimpleMilling における幾何的情報の体系化}
\begin{enumerate}[label={\sarrow[red]}]
\item \Dimple 部分の\InnerDiameter の寸法の解析的導出
\item \index{アンダーカット}アンダーカット回避用\Table 回転角の解析的導出
\item \Table 回転前における任意の\Dimple の位置座標の解析的導出
\item \Table 回転後における任意の\Dimple の位置座標の解析的導出
\item \Table 回転後における、B, D面の\Dimple 列間距離の解析的導出
\item \Table 回転後における、A, C面の\Dimple 列間距離の解析的導出
\item \Table 回転後における、B, D面の同列内の\Dimple 間距離の解析的導出
\item \Table 回転後における、A, C面の同列内の\Dimple 間距離の解析的導出
\end{enumerate}


%%%%%%%%%%%%%%%%%%%%%%%%%%%%%%%%%%%%%%%%%%%%%%%%%%%%%%%%%%
%% subsection 05.01.09 %%%%%%%%%%%%%%%%%%%%%%%%%%%%%%%%%%%
%%%%%%%%%%%%%%%%%%%%%%%%%%%%%%%%%%%%%%%%%%%%%%%%%%%%%%%%%%
\subsection{その他の幾何的情報の体系化}

%%%%%%%%%%%%%%%%%%%%%%%%%%%%%%%%%%%%%%%%%%%%%%%%%%%%%%%%%%
%% subsubsection 05.01.09.1 %%%%%%%%%%%%%%%%%%%%%%%%%%%%%%
%%%%%%%%%%%%%%%%%%%%%%%%%%%%%%%%%%%%%%%%%%%%%%%%%%%%%%%%%%
\subsubsection{\EndFaceBoringMilling における幾何的情報の体系化}
\begin{enumerate}[label={\sarrow[red]}]
\item[\sarrow] \EndFaceBoring におけるコーナー円の中心座標の解析的導出
\end{enumerate}

%%%%%%%%%%%%%%%%%%%%%%%%%%%%%%%%%%%%%%%%%%%%%%%%%%%%%%%%%%
%% subsubsection 05.01.09.2 %%%%%%%%%%%%%%%%%%%%%%%%%%%%%%
%%%%%%%%%%%%%%%%%%%%%%%%%%%%%%%%%%%%%%%%%%%%%%%%%%%%%%%%%%
\subsubsection{\CurvedOutcutMilling における幾何的情報の体系化}
\begin{enumerate}[label={\sarrow[red]}]
\item[\sarrow] \CurvedOutcut における加工中心位置座標および加工径の解析的導出
\item[\sarrow] \CurvedOutcut における\EndFace の位置座標の解析的導出
\end{enumerate}

%%%%%%%%%%%%%%%%%%%%%%%%%%%%%%%%%%%%%%%%%%%%%%%%%%%%%%%%%%
%% subsubsection 05.01.09.3 %%%%%%%%%%%%%%%%%%%%%%%%%%%%%%
%%%%%%%%%%%%%%%%%%%%%%%%%%%%%%%%%%%%%%%%%%%%%%%%%%%%%%%%%%
\subsubsection{\InnerDiameter における幾何的情報の体系化}
\begin{enumerate}[label={\sarrow[red]}]
\item[\sarrow] \index{テーパ(ないめん)@テーパ(内面)}テーパを考慮した\InnerDiameter の解析的導出
\end{enumerate}



\clearpage
%%%%%%%%%%%%%%%%%%%%%%%%%%%%%%%%%%%%%%%%%%%%%%%%%%%%%%%%%%
%% section 05.02 %%%%%%%%%%%%%%%%%%%%%%%%%%%%%%%%%%%%%%%%%
%%%%%%%%%%%%%%%%%%%%%%%%%%%%%%%%%%%%%%%%%%%%%%%%%%%%%%%%%%
\modHeadsection{\index{NCプログラム}NCプログラム作成に関する諸基準の規格化}
%%%%%%%%%%%%%%%%%%%%%%%%%%%%%%%%%%%%%%%%%%%%%%%%%%%%%%%%%%
%% marker %%%%%%%%%%%%%%%%%%%%%%%%%%%%%%%%%%%%%%%%%%%%%%%%
%%%%%%%%%%%%%%%%%%%%%%%%%%%%%%%%%%%%%%%%%%%%%%%%%%%%%%%%%%
\begin{marker}
大前提として、\index{NCプログラム}NCプログラムの作成に関しては、原則として\textbf{経営陣・管理職らの方針・意見等は一切考慮しない}。
\tcbline*
これは当社の過去の経緯を鑑みた必然の措置である
%% footnote %%%%%%%%%%%%%%%%%%%%%
\footnote{事実、機械設置時点までの当社の開発計画は支離滅裂で論理が破綻しており、すぐに頓挫し、一切の反省もされないまま放置されている。}。
%%%%%%%%%%%%%%%%%%%%%%%%%%%%%%%%%
ここではマシニングセンタの直接的な関係者ら(特に作業者)の意見のみを尊重・優先する。
\end{marker}
%%%%%%%%%%%%%%%%%%%%%%%%%%%%%%%%%%%%%%%%%%%%%%%%%%%%%%%%%%
%%%%%%%%%%%%%%%%%%%%%%%%%%%%%%%%%%%%%%%%%%%%%%%%%%%%%%%%%%
%%%%%%%%%%%%%%%%%%%%%%%%%%%%%%%%%%%%%%%%%%%%%%%%%%%%%%%%%%


%%%%%%%%%%%%%%%%%%%%%%%%%%%%%%%%%%%%%%%%%%%%%%%%%%%%%%%%%%
%% subsection 05.02.01 %%%%%%%%%%%%%%%%%%%%%%%%%%%%%%%%%%%
%%%%%%%%%%%%%%%%%%%%%%%%%%%%%%%%%%%%%%%%%%%%%%%%%%%%%%%%%%
\subsection{ソフトウェアエンジニアリングに関する諸規程の策定}
\begin{enumerate}[label={\sarrow[red]}]
\item \index{じょうほうぎじゅつしゃ@情報技術者}情報技術者およびその\index{ぎじゅつすいじゅん@技術水準}技術水準に関する規程の策定
\item システムおよびソフトウェアの作成に関する規程の策定
\item \index{ちょさくぶつ@著作物}著作物および\index{ちょさくぶつのていじ@著作物の提示}その提示に関する規程の策定
\end{enumerate}


%%%%%%%%%%%%%%%%%%%%%%%%%%%%%%%%%%%%%%%%%%%%%%%%%%%%%%%%%%
%% subsection 05.02.02 %%%%%%%%%%%%%%%%%%%%%%%%%%%%%%%%%%%
%%%%%%%%%%%%%%%%%%%%%%%%%%%%%%%%%%%%%%%%%%%%%%%%%%%%%%%%%%
\subsection{マシニングセンタに関する諸標準の策定}
\begin{enumerate}[label={\sarrow[red]}]
\item マシニングセンタにおけるワークの寸法に関する標準の策定
\item \index{NCプログラムばんごう@NCプログラム番号}NCプログラムの番号付けに関する標準の策定
\item 諸工具の\index{おくりはやさ@送り速さ}送り速さに関する標準の策定
\item 諸工具の\index{しゅじくかいてんすう@主軸回転数}主軸回転数に関する標準の策定
\item \index{NCプログラム}NCプログラムの\index{シーケンスばんごう@シーケンス番号}シーケンス番号に関する標準の策定
\item[\sarrow] \index{NCプログラム}NCプログラムの\index{アラームばんごう@アラーム番号}アラーム番号に関する標準の策定
\item マシニングセンタごとの\index{コモンへんすう@コモン変数}コモン変数に関する標準の策定
\item マシニングセンタについての\index{ちょさくぶつ@著作物}著作物および\index{ちょさくぶつのていじ@著作物の提示}その提示に関する標準の策定
\end{enumerate}



%%%%%%%%%%%%%%%%%%%%%%%%%%%%%%%%%%%%%%%%%%%%%%%%%%%%%%%%%%
%% section 05.03 %%%%%%%%%%%%%%%%%%%%%%%%%%%%%%%%%%%%%%%%%
%%%%%%%%%%%%%%%%%%%%%%%%%%%%%%%%%%%%%%%%%%%%%%%%%%%%%%%%%%
\modHeadsection{\index{3D CAD}3D CAD・\index{CAM}CAMの使用について}
\begin{enumerate}[label*=\alph*)]
\item \index{そせいかこうひん@塑性加工品}塑性加工品の加工との相性が悪く非現実的である
\item 当社の能力から鑑みて使用は明らかに非現実的である
\item 各明細ごとに、各工程部分ごとの専用NCプログラム群が必要となり、大量にあるファイルの維持・管理および記憶容量の観点から望ましくない
\item 加工の対象となるワークはそのほとんどが直線と円弧のみで表される単純な構造であり、幾何的情報は\index{しょとうきかがく@初等幾何学}初等幾何学で解析的導出が可能である
\end{enumerate}
こうした事情から、\index{3D CAD}3D CAD・\index{CAM}CAMを用いた\index{NCプログラム}NCプログラムの作成は行わない。
代替として、明細の違いに依存しない形の各工程用NCプログラム群の作成を行う。


\clearpage
%%%%%%%%%%%%%%%%%%%%%%%%%%%%%%%%%%%%%%%%%%%%%%%%%%%%%%%%%%
%% section 05.04 %%%%%%%%%%%%%%%%%%%%%%%%%%%%%%%%%%%%%%%%%
%%%%%%%%%%%%%%%%%%%%%%%%%%%%%%%%%%%%%%%%%%%%%%%%%%%%%%%%%%
\modHeadsection{\AlocationAdjustment 用\Spacer の使用について}
\begin{enumerate}[label*=\alph*)]
\item \index{めいさい(モールド)@明細(モールド)}明細ごとに\Spacer を手配する必要があり、その維持・管理や費用の観点から非現実的である
\item \Spacer の使用は、本質的にB軸方向の回転と等価である
\end{enumerate}
こうした事情から、\Spacer を用いた\AlocationAdjustment は原則として行わない。
代替として、\Table の回転を用いた\AlocationAdjustment を行う。



%%%%%%%%%%%%%%%%%%%%%%%%%%%%%%%%%%%%%%%%%%%%%%%%%%%%%%%%%%
%% section 05.05 %%%%%%%%%%%%%%%%%%%%%%%%%%%%%%%%%%%%%%%%%
%%%%%%%%%%%%%%%%%%%%%%%%%%%%%%%%%%%%%%%%%%%%%%%%%%%%%%%%%%
\modHeadsection{\EndFace 部の荒削り加工工程について}
\DimpleMilling 時の\index{アンダーカット}アンダーカットの発生を防ぐための手段として、\EndFace 部の荒削り加工が考えられている。
しかし、
\begin{enumerate}[label*=\alph*)]
\item \DMC には荒削り加工した後に必要な清掃機能が備わっておらず、必然的に人手による清掃が必要となり、作業効率が大きく低下する
\item 前工程である\index{せつだん@切断}切断の工程で、ある程度の対応が可能である
\item \DimpleMilling 時の傾き角を適切に設定することで、ある程度の対応が可能である
\end{enumerate}
こうした事情から、\EndFace 部の荒削り加工は行わない。



%%%%%%%%%%%%%%%%%%%%%%%%%%%%%%%%%%%%%%%%%%%%%%%%%%%%%%%%%%
%% section 05.06 %%%%%%%%%%%%%%%%%%%%%%%%%%%%%%%%%%%%%%%%%
%%%%%%%%%%%%%%%%%%%%%%%%%%%%%%%%%%%%%%%%%%%%%%%%%%%%%%%%%%
\modHeadsection{異常値の検知}
\begin{enumerate}[label*={\sarrow[red]}]
\item NCサブプログラムの\index{ひきすう@引数}引数の値に対して、異常値を検知する
\item \TouchSensorProbe による測定値に対して、異常値を検知する
\item NCサブプログラム内の計算値の値に対して、異常値を検知する
\item 異常値を検知した際に、\index{アラーム}アラームを発生し停止する
\item 原則として、異常値を検知した場合はアラームを発生させ停止する
\end{enumerate}



\clearpage
%%%%%%%%%%%%%%%%%%%%%%%%%%%%%%%%%%%%%%%%%%%%%%%%%%%%%%%%%%
%% section 05.07 %%%%%%%%%%%%%%%%%%%%%%%%%%%%%%%%%%%%%%%%%
%%%%%%%%%%%%%%%%%%%%%%%%%%%%%%%%%%%%%%%%%%%%%%%%%%%%%%%%%%
\modHeadsection{各工程に必要な機能}


%%%%%%%%%%%%%%%%%%%%%%%%%%%%%%%%%%%%%%%%%%%%%%%%%%%%%%%%%%
%% subsection 05.03.02 %%%%%%%%%%%%%%%%%%%%%%%%%%%%%%%%%%%
%%%%%%%%%%%%%%%%%%%%%%%%%%%%%%%%%%%%%%%%%%%%%%%%%%%%%%%%%%
\subsection{測定における工程の機能}
\begin{enumerate}[label={\sarrow[red]}]
\item ワークの\EndFace 外側中心座標の測定
\item ワークの\EndFace 内側中心座標の測定
\item ワークの\OutcutCenter 座標の測定
\item ワークの\EndFace 外側径の測定
\item ワークの\EndFace 内側径の測定
\item ワークの\OutcutCenter 座標測定の機能を有するNCサブプログラム
\item \Dimple の表面位置座標測定の機能を有するNCサブプログラム
\item \CenterlineEndFaceDifMeasurement の機能を有するNCサブプログラム
\end{enumerate}


%%%%%%%%%%%%%%%%%%%%%%%%%%%%%%%%%%%%%%%%%%%%%%%%%%%%%%%%%%
%% subsection 05.03.03 %%%%%%%%%%%%%%%%%%%%%%%%%%%%%%%%%%%
%%%%%%%%%%%%%%%%%%%%%%%%%%%%%%%%%%%%%%%%%%%%%%%%%%%%%%%%%%
\subsection{加工における工程の機能}
\begin{enumerate}[label={\sarrow[red]}]
\item \EndFacecutMilling の基準点へ移動する機能を有するNCサブプログラム
\item \OutcutMilling の基準点へ移動する機能を有するNCサブプログラム
\item[\sarrow] \CurvedOutcutMilling の基準点へ移動する機能を有するNCサブプログラム
\item \KeywayMilling の基準点へ移動する機能を有するNCサブプログラム
\item \EndFaceOutCChamferMilling の基準点へ移動する機能を有するNCサブプログラム
\item[\sarrow] \CurvedOutcut に沿った\EndFaceOutCChamferMilling の基準点へ移動する機能を有するNCサブプログラム
\item \EndFaceInCChamferMilling の基準点へ移動する機能を有するNCサブプログラム
\item \DimpleMilling の機能を有するNCサブプログラム
\item[\sarrow] \EndFaceBoringMilling の基準点へ移動する機能を有するNCサブプログラム
\item[\sarrow] \IncutBoringMilling の基準点へ移動する機能を有するNCサブプログラム
\end{enumerate}



\clearpage
%%%%%%%%%%%%%%%%%%%%%%%%%%%%%%%%%%%%%%%%%%%%%%%%%%%%%%%%%%
%% section 04.08 %%%%%%%%%%%%%%%%%%%%%%%%%%%%%%%%%%%%%%%%%
%%%%%%%%%%%%%%%%%%%%%%%%%%%%%%%%%%%%%%%%%%%%%%%%%%%%%%%%%%
\modHeadsection{その他の要件}


%%%%%%%%%%%%%%%%%%%%%%%%%%%%%%%%%%%%%%%%%%%%%%%%%%%%%%%%%%
%% subsection 04.08.01 %%%%%%%%%%%%%%%%%%%%%%%%%%%%%%%%%%%
%%%%%%%%%%%%%%%%%%%%%%%%%%%%%%%%%%%%%%%%%%%%%%%%%%%%%%%%%%
\subsection{加工システムの利便性}
\begin{enumerate}[label=\alph*)]
\item \index{NCプログラム}NCプログラムは保守・管理がしやすいように作成するように努める
\item \index{NCプログラム}NCプログラム作成者の都合を、マシニングセンタ管理者や作業者における操作の利便性より優先しない
%% footnote %%%%%%%%%%%%%%%%%%%%%
\footnote{自らの怠慢を理由に作業者らに負担を押し付けないようにする、という道徳的心得である。}
%%%%%%%%%%%%%%%%%%%%%%%%%%%%%%%%%
\item 作業者の操作の利便性を最優先し、次いでマシニングセンタ管理者の操作の利便性を優先する
\end{enumerate}


%%%%%%%%%%%%%%%%%%%%%%%%%%%%%%%%%%%%%%%%%%%%%%%%%%%%%%%%%%
%% subsection 04.08.02 %%%%%%%%%%%%%%%%%%%%%%%%%%%%%%%%%%%
%%%%%%%%%%%%%%%%%%%%%%%%%%%%%%%%%%%%%%%%%%%%%%%%%%%%%%%%%%
\subsection{NCプログラムの引数の指定}
\begin{enumerate}[label=\alph*)]
\item NCプログラムの\index{ひきすう@引数}引数の指定は、原則として\index{ひきすうしていI@引数指定I}引数指定I(\pageautoref{chap:argumentSpecification}参照)を用いる
\item \index{NCメインプログラム}NCメインプログラムで用いる\index{NCサブプログラム}NCサブプログラムの\index{ひきすう@引数}引数の値は、原則として\index{ワーク}ワークの\index{ずめん(モールド)@図面(モールド)}図面に用いられる寸法値(数値)で与える
\end{enumerate}


%%%%%%%%%%%%%%%%%%%%%%%%%%%%%%%%%%%%%%%%%%%%%%%%%%%%%%%%%%
%% subsection 04.08.02 %%%%%%%%%%%%%%%%%%%%%%%%%%%%%%%%%%%
%%%%%%%%%%%%%%%%%%%%%%%%%%%%%%%%%%%%%%%%%%%%%%%%%%%%%%%%%%
\subsection{NCプログラムのネスティング}
\begin{enumerate}[label=\alph*)]
\item NCプログラムの\index{いれここうぞう@入れ子構造}入れ子構造(\index{ネスティング}ネスティング)は、NCメインプログラムを最上位の階層として、原則として3重までとする
\end{enumerate}


%%%%%%%%%%%%%%%%%%%%%%%%%%%%%%%%%%%%%%%%%%%%%%%%%%%%%%%%%%
%% subsection 04.08.02 %%%%%%%%%%%%%%%%%%%%%%%%%%%%%%%%%%%
%%%%%%%%%%%%%%%%%%%%%%%%%%%%%%%%%%%%%%%%%%%%%%%%%%%%%%%%%%
\subsection{NCプログラムに用いる座標}
\begin{enumerate}[label=\alph*)]
\item NCプログラム内での座標について、\index{3じげんユークリッドくうかんざひょう@3次元ユークリッド空間座標}3次元ユークリッド空間座標の使用を優先する
\end{enumerate}


%%%%%%%%%%%%%%%%%%%%%%%%%%%%%%%%%%%%%%%%%%%%%%%%%%%%%%%%%%
%% subsection 04.08.02 %%%%%%%%%%%%%%%%%%%%%%%%%%%%%%%%%%%
%%%%%%%%%%%%%%%%%%%%%%%%%%%%%%%%%%%%%%%%%%%%%%%%%%%%%%%%%%
\subsection{加工システムのバックアップ}
\begin{enumerate}[label*=\alph*)]
\item 加工システムのファイル群は、原則としてすべてオンライン(クラウド上)に保存する
\item 保存先は、バージョン管理, ソースコード管理等の機能を有するものとする
\end{enumerate}


%%%%%%%%%%%%%%%%%%%%%%%%%%%%%%%%%%%%%%%%%%%%%%%%%%%%%%%%%%
%% subsection 04.08.02 %%%%%%%%%%%%%%%%%%%%%%%%%%%%%%%%%%%
%%%%%%%%%%%%%%%%%%%%%%%%%%%%%%%%%%%%%%%%%%%%%%%%%%%%%%%%%%
\subsection{加工システムの公開}
\begin{enumerate}[label*=\alph*)]
\item 加工システムのファイル群は、すべてその\index{ちょさくけん@著作権}著作権の所有者を明らかにする
\item 原則として、加工システムのファイル群はすべてオンライン上に一般公開する
\item 図面の寸法等を含むものについては非公開、あるいは代表的・典型的なものの公開に留める
\item 当社の職務著作物以外については、無断で公開しない
\end{enumerate}




