%!TEX root = ../RfCPN.tex


\modHeadchapter{加工システム作成における\index{きのうようけん@機能要件}機能要件}
ここでは加工システム(\index{NCプログラム}NCプログラム群)を作成する際の\index{きのうようけん@機能要件}機能要件を挙げる。
なお、青色(\,\sarrow\!)の項目は、優先度が低いものを示す。


%%%%%%%%%%%%%%%%%%%%%%%%%%%%%%%%%%%%%%%%%%%%%%%%%%%%%%%%%%
%% section 05.01 %%%%%%%%%%%%%%%%%%%%%%%%%%%%%%%%%%%%%%%%%
%%%%%%%%%%%%%%%%%%%%%%%%%%%%%%%%%%%%%%%%%%%%%%%%%%%%%%%%%%
\modHeadsection{ワークの幾何的情報の解析計算および体系化}


%%%%%%%%%%%%%%%%%%%%%%%%%%%%%%%%%%%%%%%%%%%%%%%%%%%%%%%%%%
%% subsection 05.01.01 %%%%%%%%%%%%%%%%%%%%%%%%%%%%%%%%%%%
%%%%%%%%%%%%%%%%%%%%%%%%%%%%%%%%%%%%%%%%%%%%%%%%%%%%%%%%%%
\subsection{\Table 回転による振分調整における幾何的情報の体系化}
\begin{enumerate}[label={\sarrow[red]}]
\item \Table 回転後の\AlocationLength(\ReAlocationLength)の解析的導出
\item \ReAlocationLength 指定時の\Table 回転角度の解析的導出
\end{enumerate}


%%%%%%%%%%%%%%%%%%%%%%%%%%%%%%%%%%%%%%%%%%%%%%%%%%%%%%%%%%
%% subsection 05.01.02 %%%%%%%%%%%%%%%%%%%%%%%%%%%%%%%%%%%
%%%%%%%%%%%%%%%%%%%%%%%%%%%%%%%%%%%%%%%%%%%%%%%%%%%%%%%%%%
\subsection{ワーク座標系原点設定における幾何的情報の体系化}
\begin{enumerate}[label={\sarrow[red]}]
\item \Table 回転後の、図面上の端面外側中心座標の解析的導出
\item \Table 回転後の、図面上の\OutcutCenter 座標の解析的導出
\item \Table 回転後の、図面上の端面内側中心座標の解析的導出
\item \Table 回転後の、図面上の\KeywayCenter 座標の解析的導出
\item 測定箇所変更時に対する補正の解析的導出
\end{enumerate}


%%%%%%%%%%%%%%%%%%%%%%%%%%%%%%%%%%%%%%%%%%%%%%%%%%%%%%%%%%
%% subsection 05.01.03 %%%%%%%%%%%%%%%%%%%%%%%%%%%%%%%%%%%
%%%%%%%%%%%%%%%%%%%%%%%%%%%%%%%%%%%%%%%%%%%%%%%%%%%%%%%%%%
\subsection{\CenterlineEndFaceDifMeasurement における幾何的情報の体系化}
\begin{enumerate}[label={\sarrow[red]}]
\item \CenterlineEndFaceDifMeasurement 位置座標の解析的導出
\end{enumerate}


%%%%%%%%%%%%%%%%%%%%%%%%%%%%%%%%%%%%%%%%%%%%%%%%%%%%%%%%%%
%% subsection 05.01.05 %%%%%%%%%%%%%%%%%%%%%%%%%%%%%%%%%%%
%%%%%%%%%%%%%%%%%%%%%%%%%%%%%%%%%%%%%%%%%%%%%%%%%%%%%%%%%%
\subsection{\KeywayMilling における幾何的情報の体系化}
\begin{enumerate}[label={\sarrow[red]}]
\item[\sarrow] \index{デプスゲージ}デプスゲージでの測定における\AsideKeywayDepth の補正の解析的導出
\end{enumerate}


%%%%%%%%%%%%%%%%%%%%%%%%%%%%%%%%%%%%%%%%%%%%%%%%%%%%%%%%%%
%% subsection 05.01.06 %%%%%%%%%%%%%%%%%%%%%%%%%%%%%%%%%%%
%%%%%%%%%%%%%%%%%%%%%%%%%%%%%%%%%%%%%%%%%%%%%%%%%%%%%%%%%%
\subsection{\EndFaceOutChamferMilling における幾何的情報の体系化}
\begin{enumerate}[label={\sarrow[red]}]
\item \EndFaceOutChamferLength に対する加工中心座標の補正の解析的導出
\item[\sarrow] 傾いた\EndFaceOutChamfer におけるコーナー部の位置座標の解析的導出
\item \EndFaceOutRChamfer の接面の解析的導出
\end{enumerate}


%%%%%%%%%%%%%%%%%%%%%%%%%%%%%%%%%%%%%%%%%%%%%%%%%%%%%%%%%%
%% subsection 05.01.07 %%%%%%%%%%%%%%%%%%%%%%%%%%%%%%%%%%%
%%%%%%%%%%%%%%%%%%%%%%%%%%%%%%%%%%%%%%%%%%%%%%%%%%%%%%%%%%
\subsection{\EndFaceInChamferMilling における幾何的情報の体系化}
\begin{enumerate}[label={\sarrow[red]}]
\item \EndFaceInChamferLength に対する加工中心座標の補正の解析的導出
\item[\sarrow] 傾いた\EndFaceInChamfer におけるコーナー部の位置座標の解析的導出
\item \EndFaceInRChamfer の接面の解析的導出
\end{enumerate}


\clearpage
%%%%%%%%%%%%%%%%%%%%%%%%%%%%%%%%%%%%%%%%%%%%%%%%%%%%%%%%%%
%% subsection 05.01.08 %%%%%%%%%%%%%%%%%%%%%%%%%%%%%%%%%%%
%%%%%%%%%%%%%%%%%%%%%%%%%%%%%%%%%%%%%%%%%%%%%%%%%%%%%%%%%%
\subsection{\DimpleMilling における幾何的情報の体系化}
\begin{enumerate}[label={\sarrow[red]}]
\item \Dimple 部分の\InnerDiameter の寸法の解析的導出
\item \index{アンダーカット}アンダーカット回避用\Table 回転角の解析的導出
\item \Table 回転前における任意の\Dimple の位置座標の解析的導出
\item \Table 回転後における任意の\Dimple の位置座標の解析的導出
\item \Table 回転後における、B, D面の\Dimple 列間距離の解析的導出
\item \Table 回転後における、A, C面の\Dimple 列間距離の解析的導出
\item \Table 回転後における、B, D面の同列内の\Dimple 間距離の解析的導出
\item \Table 回転後における、A, C面の同列内の\Dimple 間距離の解析的導出
\end{enumerate}


%%%%%%%%%%%%%%%%%%%%%%%%%%%%%%%%%%%%%%%%%%%%%%%%%%%%%%%%%%
%% subsection 05.01.09 %%%%%%%%%%%%%%%%%%%%%%%%%%%%%%%%%%%
%%%%%%%%%%%%%%%%%%%%%%%%%%%%%%%%%%%%%%%%%%%%%%%%%%%%%%%%%%
\subsection{その他の幾何的情報の体系化}

%%%%%%%%%%%%%%%%%%%%%%%%%%%%%%%%%%%%%%%%%%%%%%%%%%%%%%%%%%
%% subsubsection 05.01.09.1 %%%%%%%%%%%%%%%%%%%%%%%%%%%%%%
%%%%%%%%%%%%%%%%%%%%%%%%%%%%%%%%%%%%%%%%%%%%%%%%%%%%%%%%%%
\subsubsection{\EndFaceBoringMilling における幾何的情報の体系化}
\begin{enumerate}[label={\sarrow[red]}]
\item[\sarrow] \EndFaceBoring におけるコーナー円の中心座標の解析的導出
\end{enumerate}

%%%%%%%%%%%%%%%%%%%%%%%%%%%%%%%%%%%%%%%%%%%%%%%%%%%%%%%%%%
%% subsubsection 05.01.09.2 %%%%%%%%%%%%%%%%%%%%%%%%%%%%%%
%%%%%%%%%%%%%%%%%%%%%%%%%%%%%%%%%%%%%%%%%%%%%%%%%%%%%%%%%%
\subsubsection{\CurvedOutcutMilling における幾何的情報の体系化}
\begin{enumerate}[label={\sarrow[red]}]
\item[\sarrow] \CurvedOutcut における加工中心位置座標および加工径の解析的導出
\item[\sarrow] \CurvedOutcut における\EndFace の位置座標の解析的導出
\end{enumerate}

%%%%%%%%%%%%%%%%%%%%%%%%%%%%%%%%%%%%%%%%%%%%%%%%%%%%%%%%%%
%% subsubsection 05.01.09.3 %%%%%%%%%%%%%%%%%%%%%%%%%%%%%%
%%%%%%%%%%%%%%%%%%%%%%%%%%%%%%%%%%%%%%%%%%%%%%%%%%%%%%%%%%
\subsubsection{\InnerDiameter における幾何的情報の体系化}
\begin{enumerate}[label={\sarrow[red]}]
\item[\sarrow] \index{テーパ}テーパを考慮した\InnerDiameter の解析的導出
\end{enumerate}



\clearpage
%%%%%%%%%%%%%%%%%%%%%%%%%%%%%%%%%%%%%%%%%%%%%%%%%%%%%%%%%%
%% section 05.02 %%%%%%%%%%%%%%%%%%%%%%%%%%%%%%%%%%%%%%%%%
%%%%%%%%%%%%%%%%%%%%%%%%%%%%%%%%%%%%%%%%%%%%%%%%%%%%%%%%%%
\modHeadsection{\index{NCプログラム}NCプログラム作成に関する諸基準の規格化}


%%%%%%%%%%%%%%%%%%%%%%%%%%%%%%%%%%%%%%%%%%%%%%%%%%%%%%%%%%
%% subsection 05.02.01 %%%%%%%%%%%%%%%%%%%%%%%%%%%%%%%%%%%
%%%%%%%%%%%%%%%%%%%%%%%%%%%%%%%%%%%%%%%%%%%%%%%%%%%%%%%%%%
\subsection{ソフトウェアエンジニアリングに関する諸規程の策定}
\begin{enumerate}[label={\sarrow[red]}]
\item 技術者およびその\index{ぎじゅつすいじゅん@技術水準}技術水準に関する規程の策定
\item システムおよびソフトウェアの作成に関する規程の策定
\item \index{ちょさくぶつ@著作物}著作物および\index{ちょさくぶつのていじ@著作物の提示}その提示に関する規程の策定
\end{enumerate}


%%%%%%%%%%%%%%%%%%%%%%%%%%%%%%%%%%%%%%%%%%%%%%%%%%%%%%%%%%
%% subsection 05.02.01 %%%%%%%%%%%%%%%%%%%%%%%%%%%%%%%%%%%
%%%%%%%%%%%%%%%%%%%%%%%%%%%%%%%%%%%%%%%%%%%%%%%%%%%%%%%%%%
\subsection{マシニングセンタに関する諸標準の策定}
\begin{enumerate}[label={\sarrow[red]}]
\item マシニングセンタにおけるワークの寸法に関する標準の策定
\item \index{NCプログラムばんごう@NCプログラム番号}NCプログラムの番号付けに関する標準の策定
\item 諸工具の\index{おくりはやさ@送り速さ}送り速さに関する標準の策定
\item 諸工具の\index{しゅじくかいてんすう@主軸回転数}主軸回転数に関する標準の策定
\item \index{NCプログラム}NCプログラムの\index{シーケンスばんごう@シーケンス番号}シーケンス番号に関する標準の策定
\item[\sarrow] \index{NCプログラム}NCプログラムの\index{アラームばんごう@アラーム番号}アラーム番号に関する標準の策定
\item マシニングセンタごとの\index{コモンへんすう@コモン変数}コモン変数に関する標準の策定
\item マシニングセンタについての\index{ちょさくぶつ@著作物}著作物および\index{ちょさくぶつのていじ@著作物の提示}その提示に関する標準の策定
\end{enumerate}



\clearpage
%%%%%%%%%%%%%%%%%%%%%%%%%%%%%%%%%%%%%%%%%%%%%%%%%%%%%%%%%%
%% section 05.03 %%%%%%%%%%%%%%%%%%%%%%%%%%%%%%%%%%%%%%%%%
%%%%%%%%%%%%%%%%%%%%%%%%%%%%%%%%%%%%%%%%%%%%%%%%%%%%%%%%%%
\modHeadsection{具体的なNCプログラムの作成}
具体的な\index{NCプログラム}NCプログラムの作成に際して、基本的には以下のような方針を取ることにする。
\begin{enumerate}[label*=\alph*)]
\item 工程の順は原則として以下のとおりとする(\pageautoref{fig:flowchart}参照)
  \begin{enumerate}[label={\arabic*.}]
  \item ワークの搬入
  \item \BottomEndFace 部AC方向外側中心測定 or \BottomOutcut AC方向中心測定
  \item \BottomEndFace 部BD方向外側中心測定 or \BottomOutcut BD方向中心測定
  \item \BottomEndFace 部AC方向内側中心測定
  \item \BottomEndFace 部BD方向内側中心測定
  \item \TopEndFace 部AC方向外側中心測定 or \TopOutcut AC方向中心測定
  \item \TopEndFace 部BD方向外側中心測定 or \TopOutcut BD方向中心測定
  \item \TopEndFace 部AC方向内側中心測定
  \item \TopEndFace 部BD方向内側中心測定
  \item \Keyway AC方向中心測定
  \item \Keyway BD方向中心測定
  \item 全\Dimple の表面位置測定
  \item \DimpleMilling
  \item \TopEndFacecutMilling
  \item \TopOutcutMilling or \EndFaceBoringMilling or \IncutBoringMilling
  \item \KeywayMilling
  \item \TopEndFaceOutCChamferMilling
  \item \TopEndFaceInCChamferMilling
  \item \BottomEndFacecutMilling
  \item \BottomOutcutMilling
  \item \BottomEndFaceOutCChamferMilling
  \item \BottomEndFaceInCChamferMilling
  \item \CenterlineEndFaceDifMeasurement
  \item ワークの搬出
  \end{enumerate}
\item 作業者にとっての利便性を、NCプログラム作成者にとっての利便性より優先する
\item \index{NCメインプログラム}NCメインプログラムで用いる\index{NCサブプログラム}NCサブプログラムの\index{ひきすう@引数}引数は、原則として\index{ワーク}ワークの\index{ずめん(モールド)@図面(モールド)}図面に用いられている寸法値で与えるものとする
\item NCプログラムの\index{いれここうぞう@入れ子構造}入れ子構造(\index{ネスティング}ネスティング)は原則として3重までとする
\end{enumerate}


\clearpage
%%%%%%%%%%%%%%%%%%%%%%%%%%%%%%%%%%%%%%%%%%%%%%%%%%%%%%%%%%
%% subsection 05.03.01 %%%%%%%%%%%%%%%%%%%%%%%%%%%%%%%%%%%
%%%%%%%%%%%%%%%%%%%%%%%%%%%%%%%%%%%%%%%%%%%%%%%%%%%%%%%%%%
\subsection{\index{げんてんせってい@原点設定}原点設定用NCサブプログラムの作成\TBW}



%%%%%%%%%%%%%%%%%%%%%%%%%%%%%%%%%%%%%%%%%%%%%%%%%%%%%%%%%%
%% section 05.04 %%%%%%%%%%%%%%%%%%%%%%%%%%%%%%%%%%%%%%%%%
%%%%%%%%%%%%%%%%%%%%%%%%%%%%%%%%%%%%%%%%%%%%%%%%%%%%%%%%%%
\modHeadsection{\EndFacecutMilling 用NCサブプログラムの作成\TBW}
\begin{enumerate}[label=\sarrow]
\item \IDCenter を基準とした\EndFacecutMilling に変更
\item \ODCornerR への対応
\item \index{けずりしろ@削り代}削り代および加工回数の変更の簡易化
\end{enumerate}



%%%%%%%%%%%%%%%%%%%%%%%%%%%%%%%%%%%%%%%%%%%%%%%%%%%%%%%%%%
%% section 06.05 %%%%%%%%%%%%%%%%%%%%%%%%%%%%%%%%%%%%%%%%%
%%%%%%%%%%%%%%%%%%%%%%%%%%%%%%%%%%%%%%%%%%%%%%%%%%%%%%%%%%
\modHeadsection{\OutcutMilling 用NCサブプログラムの作成\TBW}
\begin{enumerate}[label=\sarrow]
\item \CurvedOutcutMilling の機械化
\end{enumerate}



%%%%%%%%%%%%%%%%%%%%%%%%%%%%%%%%%%%%%%%%%%%%%%%%%%%%%%%%%%
%% section 06.06 %%%%%%%%%%%%%%%%%%%%%%%%%%%%%%%%%%%%%%%%%
%%%%%%%%%%%%%%%%%%%%%%%%%%%%%%%%%%%%%%%%%%%%%%%%%%%%%%%%%%
\modHeadsection{\KeywayMilling 用NCサブプログラムの作成\TBW}
\begin{enumerate}[label=\sarrow]
\item \KeywayPos・\KeywayWidth 補正変更作業の簡易化
\item \KeywayWidth に応じた加工回数の変更作業の機械化
\end{enumerate}



%%%%%%%%%%%%%%%%%%%%%%%%%%%%%%%%%%%%%%%%%%%%%%%%%%%%%%%%%%
%% section 06.07 %%%%%%%%%%%%%%%%%%%%%%%%%%%%%%%%%%%%%%%%%
%%%%%%%%%%%%%%%%%%%%%%%%%%%%%%%%%%%%%%%%%%%%%%%%%%%%%%%%%%
\modHeadsection{\EndFaceChamferMilling 用NCサブプログラムの作成\TBW}
\begin{enumerate}[label=\sarrow]
\item \EndFaceChamferMilling の機械化
\item AC方向の位置調整作業の機械化
\end{enumerate}



%%%%%%%%%%%%%%%%%%%%%%%%%%%%%%%%%%%%%%%%%%%%%%%%%%%%%%%%%%
%% section 06.08 %%%%%%%%%%%%%%%%%%%%%%%%%%%%%%%%%%%%%%%%%
%%%%%%%%%%%%%%%%%%%%%%%%%%%%%%%%%%%%%%%%%%%%%%%%%%%%%%%%%%
\modHeadsection{\Dimple 用NCサブプログラムの作成\TBW}
\begin{enumerate}[label=\sarrow]
\item
\end{enumerate}


