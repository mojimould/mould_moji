%!TEX root = ../RfCPN.tex


\modHeadchapter{加工システム作成における\index{ひきのうようけん@非機能要件}非機能要件\TBW}
% システムの性能、信頼性、拡張性など、ソフトウェアの「どのように動作するべきか」に関する要件を定義


\begin{NFR}{}
\begin{enumerate}[label=\sarrow]
\item \index{NCプログラム}NCプログラムは保守・管理がしやすいように作成するように努める
\item \index{NCプログラム}NCプログラム作成者の都合を、マシニングセンタ管理者や作業者における操作の利便性より優先しない
%% footnote %%%%%%%%%%%%%%%%%%%%%
\footnote{自らの怠慢を理由に作業者らに負担を押し付けないようにする、という道徳的心得である。}
%%%%%%%%%%%%%%%%%%%%%%%%%%%%%%%%%
\item 作業者の操作の利便性を最優先にし、次いでマシニングセンタ管理者の操作の利便性を優先する。
\end{enumerate}
\end{NFR}

\begin{NFR}{}
\begin{enumerate}[label=\sarrow]
\item NCプログラムの\index{ひきすう@引数}引数は、原則として\index{ひきすうしていI@引数指定I}引数指定I(\pageautoref{chap:argumentSpecification}参照)を用いる
\item \index{NCメインプログラム}NCメインプログラムで用いる各工程用\index{NCサブプログラム}NCサブプログラムの\index{ひきすう@引数}引数の値は、原則として\index{ワーク}ワークの\index{ずめん(モールド)@図面(モールド)}図面に用いられている寸法値(数値)で与える
\end{enumerate}
\end{NFR}

\begin{NFR}{}
\begin{enumerate}[label=\sarrow]
\item NCプログラムの\index{いれここうぞう@入れ子構造}入れ子構造(\index{ネスティング}ネスティング)は、NCメインプログラムを最上位の階層として、原則として3重までとする。
\end{enumerate}
\end{NFR}

\begin{NFR}{}
\begin{enumerate}[label=\sarrow]
\item NCプログラム内での座標について、\index{3じげんユークリッドくうかんざひょう@3次元ユークリッド空間座標}3次元ユークリッド空間座標の使用を優先する。
\end{enumerate}
\end{NFR}




