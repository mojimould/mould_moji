%!TEX root = ../RfCPN.tex


\modHeadchapter{加工システム作成における\index{ひきのうようけん@非機能要件}非機能要件}
% システムの性能、信頼性、拡張性など、ソフトウェアの「どのように動作するべきか」に関する要件を定義



%%%%%%%%%%%%%%%%%%%%%%%%%%%%%%%%%%%%%%%%%%%%%%%%%%%%%%%%%%
%% section 06.01 %%%%%%%%%%%%%%%%%%%%%%%%%%%%%%%%%%%%%%%%%
%%%%%%%%%%%%%%%%%%%%%%%%%%%%%%%%%%%%%%%%%%%%%%%%%%%%%%%%%%
\modHeadsection{加工システムの利便性}
\begin{enumerate}[label=\alph*)]
\item \index{NCプログラム}NCプログラムは保守・管理がしやすいように作成するように努める
\item \index{NCプログラム}NCプログラム作成者の都合を、マシニングセンタ管理者や作業者における操作の利便性より優先しない
%% footnote %%%%%%%%%%%%%%%%%%%%%
\footnote{自らの怠慢を理由に作業者らに負担を押し付けないようにする、という道徳的心得である。}
%%%%%%%%%%%%%%%%%%%%%%%%%%%%%%%%%
\item 作業者の操作の利便性を最優先し、次いでマシニングセンタ管理者の操作の利便性を優先する
\end{enumerate}



%%%%%%%%%%%%%%%%%%%%%%%%%%%%%%%%%%%%%%%%%%%%%%%%%%%%%%%%%%
%% section 06.02 %%%%%%%%%%%%%%%%%%%%%%%%%%%%%%%%%%%%%%%%%
%%%%%%%%%%%%%%%%%%%%%%%%%%%%%%%%%%%%%%%%%%%%%%%%%%%%%%%%%%
\modHeadsection{NCプログラムの引数の指定}
\begin{enumerate}[label=\alph*)]
\item NCプログラムの\index{ひきすう@引数}引数の指定は、原則として\index{ひきすうしていI@引数指定I}引数指定I(\pageautoref{chap:argumentSpecification}参照)を用いる
\item \index{NCメインプログラム}NCメインプログラムで用いる\index{NCサブプログラム}NCサブプログラムの\index{ひきすう@引数}引数の値は、原則として\index{ワーク}ワークの\index{ずめん(モールド)@図面(モールド)}図面に用いられる寸法値(数値)で与える
\end{enumerate}



%%%%%%%%%%%%%%%%%%%%%%%%%%%%%%%%%%%%%%%%%%%%%%%%%%%%%%%%%%
%% section 06.03 %%%%%%%%%%%%%%%%%%%%%%%%%%%%%%%%%%%%%%%%%
%%%%%%%%%%%%%%%%%%%%%%%%%%%%%%%%%%%%%%%%%%%%%%%%%%%%%%%%%%
\modHeadsection{NCプログラムのネスティング}
\begin{enumerate}[label=\alph*)]
\item NCプログラムの\index{いれここうぞう@入れ子構造}入れ子構造(\index{ネスティング}ネスティング)は、NCメインプログラムを最上位の階層として、原則として3重までとする
\end{enumerate}



%%%%%%%%%%%%%%%%%%%%%%%%%%%%%%%%%%%%%%%%%%%%%%%%%%%%%%%%%%
%% section 06.04 %%%%%%%%%%%%%%%%%%%%%%%%%%%%%%%%%%%%%%%%%
%%%%%%%%%%%%%%%%%%%%%%%%%%%%%%%%%%%%%%%%%%%%%%%%%%%%%%%%%%
\modHeadsection{NCプログラムに用いる座標}
\begin{enumerate}[label=\alph*)]
\item NCプログラム内での座標について、\index{3じげんユークリッドくうかんざひょう@3次元ユークリッド空間座標}3次元ユークリッド空間座標の使用を優先する
\end{enumerate}



%%%%%%%%%%%%%%%%%%%%%%%%%%%%%%%%%%%%%%%%%%%%%%%%%%%%%%%%%%
%% section 06.05 %%%%%%%%%%%%%%%%%%%%%%%%%%%%%%%%%%%%%%%%%
%%%%%%%%%%%%%%%%%%%%%%%%%%%%%%%%%%%%%%%%%%%%%%%%%%%%%%%%%%
\modHeadsection{異常値の検知}
\begin{enumerate}[label*=\alph*)]
\item NCプログラムの\index{ひきすう@引数}引数値や測定値, 計算値等に対する異常値を検知する機能を施す
\item 原則として、異常値を検知した場合はアラームを発生させ停止する
\end{enumerate}



%%%%%%%%%%%%%%%%%%%%%%%%%%%%%%%%%%%%%%%%%%%%%%%%%%%%%%%%%%
%% section 06.06 %%%%%%%%%%%%%%%%%%%%%%%%%%%%%%%%%%%%%%%%%
%%%%%%%%%%%%%%%%%%%%%%%%%%%%%%%%%%%%%%%%%%%%%%%%%%%%%%%%%%
\modHeadsection{加工システムのバックアップ}
\begin{enumerate}[label*=\alph*)]
\item 加工システムのファイル群は、原則としてすべてオンライン(クラウド上)に保存する
\item 保存先は、バージョン管理, ソースコード管理等の機能を有するものとする
\end{enumerate}



\clearpage
%%%%%%%%%%%%%%%%%%%%%%%%%%%%%%%%%%%%%%%%%%%%%%%%%%%%%%%%%%
%% section 06.07 %%%%%%%%%%%%%%%%%%%%%%%%%%%%%%%%%%%%%%%%%
%%%%%%%%%%%%%%%%%%%%%%%%%%%%%%%%%%%%%%%%%%%%%%%%%%%%%%%%%%
\modHeadsection{NCプログラムの公開}
\begin{enumerate}[label*=\alph*)]
\item 加工システムのファイル群は、すべてその\index{ちょさくけん@著作権}著作権の所有者を明らかにする
\item 原則として、加工システムのファイル群はすべてオンラインで一般公開する
\item 図面の寸法等を含むものについては非公開、あるいは代表的・典型的なものの公開に留める
\item 当社の職務著作物以外については、無断で公開しない
\end{enumerate}


