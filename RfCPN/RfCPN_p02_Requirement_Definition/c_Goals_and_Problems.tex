%!TEX root = ../RfCPN.tex


\modHeadchapter{機械稼働における達成したい目標・解決すべき課題}
新たに導入したマシニングセンタで何を達成したいのか、その目的さえも明確にされていないのが現状である
%% footnote %%%%%%%%%%%%%%%%%%%%%
\footnote{\MMC に改善の余地が多くある中、そもそもどのような論理で導入にまで至ったのか、とても不思議である。}。
%%%%%%%%%%%%%%%%%%%%%%%%%%%%%%%%%
したがって、ここでは暫定的に目的を定め、具体的な目標の設定を行うことにする。



%%%%%%%%%%%%%%%%%%%%%%%%%%%%%%%%%%%%%%%%%%%%%%%%%%%%%%%%%%
%% section 04.01 %%%%%%%%%%%%%%%%%%%%%%%%%%%%%%%%%%%%%%%%%
%%%%%%%%%%%%%%%%%%%%%%%%%%%%%%%%%%%%%%%%%%%%%%%%%%%%%%%%%%
\modHeadsection{\DMC 導入の目的}
主な目的としては\Dimple 加工を内製化することではあるが、それに伴って作業効率の低下や教育コストの増加が多大になってしまっては本末転倒である。
そのためここでは、(暫定的な)目的として以下を採用する。
なお、これらは\DMC 設置時点での\MMC の現状と比較したものである。
\begin{enumerate}[label=\sarrow]
\item 作業効率の大幅向上
\item 教育コストの大幅削減
\item \Dimple 加工の内製化
\end{enumerate}



%%%%%%%%%%%%%%%%%%%%%%%%%%%%%%%%%%%%%%%%%%%%%%%%%%%%%%%%%%
%% section 04.02 %%%%%%%%%%%%%%%%%%%%%%%%%%%%%%%%%%%%%%%%%
%%%%%%%%%%%%%%%%%%%%%%%%%%%%%%%%%%%%%%%%%%%%%%%%%%%%%%%%%%
\modHeadsection{達成したい目標}
達成したい目標として主に以下が挙げられる。
なお、これらは\DMC 設置時点での\MMC の現状と比較したものである。
\begin{enumerate}[label=\sarrow]
\item 作業員の安全性の考慮
\item 製品・コードの品質の考慮
\item 機械の信頼性の考慮
\item 諸作業に対する属人性の大幅削減
\end{enumerate}
なお、属人性の削減については、以下のような方針をとるものとする
%% footnote %%%%%%%%%%%%%%%%%%%%%
\footnote{\textgt{書類生成の自動化}や\textgt{コード生成の自動化}などもあるが、これらについては\pageautoref{part:XI}以降で取り扱う。}。
%%%%%%%%%%%%%%%%%%%%%%%%%%%%%%%%%
\begin{enumerate}[label=\sarrow]
\item \textgt{加工の自動化}:手作業による測定・加工の削減・簡易化
\item \textgt{操作の自動化}:手作業によるマシニングセンタ画面操作の削減・簡易化
\end{enumerate}


\clearpage
%%%%%%%%%%%%%%%%%%%%%%%%%%%%%%%%%%%%%%%%%%%%%%%%%%%%%%%%%%
%% section 04.03 %%%%%%%%%%%%%%%%%%%%%%%%%%%%%%%%%%%%%%%%%
%%%%%%%%%%%%%%%%%%%%%%%%%%%%%%%%%%%%%%%%%%%%%%%%%%%%%%%%%%
\modHeadsection{解決すべき課題\TBW}
解決すべき課題として、主に以下が挙げられる。
\begin{enumerate}[label=\sarrow]
\item 諸規程の策定
\item 諸規程に則った、諸標準の策定
\item 機内の幾何的情報の一般化および体系化
\item 諸標準に則った、各明細・各工程に対する\index{NCプログラム}NCプログラムの作成
\item モールドの関係データベースの作成
\end{enumerate}

