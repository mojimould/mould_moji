%!TEX root = ../RfCPN.tex


\modHeadchapter{加工システム作成における達成したい目標・解決すべき課題}
新たに導入したマシニングセンタで何を達成したいのか、その目的さえも明確にされていないのが現状である
%% footnote %%%%%%%%%%%%%%%%%%%%%
\footnote{\MMC に改善の余地が多くある中、どのような論理で導入にまで至ったのか、とても不思議である。}。
%%%%%%%%%%%%%%%%%%%%%%%%%%%%%%%%%
したがって、ここでは暫定的に目的を定め、具体的な目標の設定を行うことにする。



%%%%%%%%%%%%%%%%%%%%%%%%%%%%%%%%%%%%%%%%%%%%%%%%%%%%%%%%%%
%% section 04.01 %%%%%%%%%%%%%%%%%%%%%%%%%%%%%%%%%%%%%%%%%
%%%%%%%%%%%%%%%%%%%%%%%%%%%%%%%%%%%%%%%%%%%%%%%%%%%%%%%%%%
\modHeadsection{\DMC 導入の目的}
ここでは(暫定的な)目的として以下を採用する。
\begin{enumerate}[label=\sarrow]
\item \Dimple 加工の実現
\item \Dimple 加工に伴い増加する教育コストの削減
\item \Dimple 加工に伴い低下する作業効率%
%% footnote %%%%%%%%%%%%%%%%%%%%%
\footnote{機械の導入に伴う、\index{NCプログラム}NCプログラム作成や維持・管理の効率、別のマシニングセンタの移動に起因する効率の低下等が含まれる。}
%%%%%%%%%%%%%%%%%%%%%%%%%%%%%%%%%
の向上
\end{enumerate}



%%%%%%%%%%%%%%%%%%%%%%%%%%%%%%%%%%%%%%%%%%%%%%%%%%%%%%%%%%
%% section 04.02 %%%%%%%%%%%%%%%%%%%%%%%%%%%%%%%%%%%%%%%%%
%%%%%%%%%%%%%%%%%%%%%%%%%%%%%%%%%%%%%%%%%%%%%%%%%%%%%%%%%%
\modHeadsection{加工システム作成における達成したい目標}


%%%%%%%%%%%%%%%%%%%%%%%%%%%%%%%%%%%%%%%%%%%%%%%%%%%%%%%%%%
%% subsection 04.02.01 %%%%%%%%%%%%%%%%%%%%%%%%%%%%%%%%%%%
%%%%%%%%%%%%%%%%%%%%%%%%%%%%%%%%%%%%%%%%%%%%%%%%%%%%%%%%%%
\subsection{\Dimple 加工の実現に対する目標}
\begin{enumerate}[label=\sarrow]
\item \Dimple に関する解析的な位置情報および\index{じょうけんぶんきじょうほう@条件分岐情報}条件分岐情報の把握および体系化
\item \Dimple 位置測定用\index{NCプログラム}NCプログラムの作成
\item \DimpleMilling 用\index{NCプログラム}NCプログラムの作成
\end{enumerate}


%%%%%%%%%%%%%%%%%%%%%%%%%%%%%%%%%%%%%%%%%%%%%%%%%%%%%%%%%%
%% subsection 04.02.02 %%%%%%%%%%%%%%%%%%%%%%%%%%%%%%%%%%%
%%%%%%%%%%%%%%%%%%%%%%%%%%%%%%%%%%%%%%%%%%%%%%%%%%%%%%%%%%
\subsection{\index{きょういくコスト@教育コスト}教育コストの削減に対する目標}
\begin{enumerate}[label=\sarrow]
\item 諸作業に対する属人性の大幅削減
\end{enumerate}
なお、属人性の削減については、以下のような方針をとるものとする
%% footnote %%%%%%%%%%%%%%%%%%%%%
\footnote{\textgt{書類生成の自動化}や\textgt{コード生成の自動化}などもあるが、これらについては\pageautoref{part:XI}以降で取り扱う。}。
%%%%%%%%%%%%%%%%%%%%%%%%%%%%%%%%%
\begin{enumerate}[label=\sarrow]
\item \textgt{加工の自動化}:手作業による測定・加工の削減・簡易化
\item \textgt{操作の簡易化}:手作業によるマシニングセンタ画面操作の削減・簡易化
\end{enumerate}


%%%%%%%%%%%%%%%%%%%%%%%%%%%%%%%%%%%%%%%%%%%%%%%%%%%%%%%%%%
%% subsection 04.02.03 %%%%%%%%%%%%%%%%%%%%%%%%%%%%%%%%%%%
%%%%%%%%%%%%%%%%%%%%%%%%%%%%%%%%%%%%%%%%%%%%%%%%%%%%%%%%%%
\subsection{全体の作業効率の向上に対する目標}
\begin{enumerate}[label=\sarrow]
\item 諸作業に対する属人性の大幅削減
\item 機内のワークの解析的な位置情報の把握および体系化
\item ワークの加工に関する条件分岐情報の把握および体系化
\end{enumerate}



\clearpage
%%%%%%%%%%%%%%%%%%%%%%%%%%%%%%%%%%%%%%%%%%%%%%%%%%%%%%%%%%
%% subsection 04.02.04 %%%%%%%%%%%%%%%%%%%%%%%%%%%%%%%%%%%
%%%%%%%%%%%%%%%%%%%%%%%%%%%%%%%%%%%%%%%%%%%%%%%%%%%%%%%%%%
\subsection{その他の目標}
\begin{enumerate}[label=\sarrow]
\item 作業員の安全性の考慮
\item 製品・コードの品質の考慮
\item 機械の信頼性の考慮
\end{enumerate}


%\clearpage
%%%%%%%%%%%%%%%%%%%%%%%%%%%%%%%%%%%%%%%%%%%%%%%%%%%%%%%%%%
%% section 04.03 %%%%%%%%%%%%%%%%%%%%%%%%%%%%%%%%%%%%%%%%%
%%%%%%%%%%%%%%%%%%%%%%%%%%%%%%%%%%%%%%%%%%%%%%%%%%%%%%%%%%
\modHeadsection{加工システム作成における解決すべき課題\TBW}
赤色(\,\sarrow[red]\!)の項目は、(当社のこれまでの姿勢・振舞いから鑑みて)解決が事実上不可能あるいはそれに近いと帰結されるものを示す。
\begin{enumerate}[label=\sarrow]
\item[{\sarrow[red]}] ソフトウェアエンジニアリングに関する管理職ないしは経営陣の理解力およびモラルの会得
\item[{\sarrow[red]}] ソフトウェアエンジニアリング部門の創設
\item[{\sarrow[red]}] ソフトウェアエンジニアリングに関する管理職の\index{リーダーシップ}リーダーシップの会得・向上
\item[{\sarrow[red]}] ソフトウェアエンジニアリングに関する安全部門の機能化
\item[{\sarrow[red]}] ソフトウェアエンジニアリングに関する環境部門の機能化
\item[{\sarrow[red]}] ソフトウェアエンジニアリングに関する品質部門の機能化
\item[{\sarrow[red]}] プログラマ育成者の増員
\item[{\sarrow[red]}] プログラマの育成・増員
\item[{\sarrow[red]}] 管理職・スタッフの論理的思考力の向上
\item[{\sarrow[red]}] マシニングセンタに関わる管理職・スタッフのプログラミングに関する初等スキルの会得
\item[{\sarrow[red]}] マシニングセンタに関わる管理職・スタッフの初等数学(初等幾何学)能力の向上
\item ソフトウェアエンジニアリングに関する諸規程の策定
\item 諸規程に則った、マシニングセンタに関する諸標準の策定
\item マシニングセンタ内の幾何的情報の解析的な把握および体系化
\item 諸標準に則った、各明細・各工程に対する\index{NCプログラム}NCプログラムの作成
\end{enumerate}


