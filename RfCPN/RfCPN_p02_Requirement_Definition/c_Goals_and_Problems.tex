%!TEX root = ../RfCPN.tex


\modHeadchapter{機械稼働における達成したい目標・解決すべき課題}
新たに導入したマシニングセンタで何を達成したいのか、その目的さえも明確にされていないのが現状である
%% footnote %%%%%%%%%%%%%%%%%%%%%
\footnote{\MMC に改善の余地が多く残っている中、そもそもどのような論理で導入にまで至ったのか、不思議でならない。}。
%%%%%%%%%%%%%%%%%%%%%%%%%%%%%%%%%
したがって、ここでは暫定的に目的を定め、具体的な目標の設定を行うことにする。



%%%%%%%%%%%%%%%%%%%%%%%%%%%%%%%%%%%%%%%%%%%%%%%%%%%%%%%%%%
%% section 04.01 %%%%%%%%%%%%%%%%%%%%%%%%%%%%%%%%%%%%%%%%%
%%%%%%%%%%%%%%%%%%%%%%%%%%%%%%%%%%%%%%%%%%%%%%%%%%%%%%%%%%
\modHeadsection{\DMC 導入の目的}
主な目的としては\Dimple 加工を内製化することではあるが、それに伴って作業効率の低下や教育コストの増加が多大になってしまっては本末転倒である。
そのためここでは、(暫定的な)目的として以下を採用する。
なお、これらは\DMC 設置時点での\MMC の現状と比較したものである。
\begin{enumerate}[label=\sarrow]
\item 作業効率の大幅向上
\item 教育コストの大幅削減
\item \Dimple 加工の内製化
\end{enumerate}



%%%%%%%%%%%%%%%%%%%%%%%%%%%%%%%%%%%%%%%%%%%%%%%%%%%%%%%%%%
%% section 04.02 %%%%%%%%%%%%%%%%%%%%%%%%%%%%%%%%%%%%%%%%%
%%%%%%%%%%%%%%%%%%%%%%%%%%%%%%%%%%%%%%%%%%%%%%%%%%%%%%%%%%
\modHeadsection{達成したい目標}
達成したい目標として主に以下が挙げられる。
なお、これらは\DMC 設置時点での\MMC の現状と比較したものである。
\begin{enumerate}[label=\sarrow]
\item 作業員の安全性の向上
\item 製品の品質の向上
\item 機械の信頼性の大幅向上
\item 諸作業に対する属人性の大幅削減
\end{enumerate}
なお、属人性の削減については、以下のような方針をとるものとする
%% footnote %%%%%%%%%%%%%%%%%%%%%
\footnote{\textgt{書類生成の自動化}や\textgt{コード生成の自動化}などもあるが、これらについては\pageautoref{part:XI}以降で取り扱う。}。
%%%%%%%%%%%%%%%%%%%%%%%%%%%%%%%%%
\begin{enumerate}[label=\sarrow]
\item \textgt{加工の自動化}:手作業による測定・加工の削減・簡易化
\item \textgt{操作の自動化}:手作業によるマシニングセンタ画面操作の削減・簡易化
\end{enumerate}


\clearpage
%%%%%%%%%%%%%%%%%%%%%%%%%%%%%%%%%%%%%%%%%%%%%%%%%%%%%%%%%%
%% subsection 04.02.01 %%%%%%%%%%%%%%%%%%%%%%%%%%%%%%%%%%%
%%%%%%%%%%%%%%%%%%%%%%%%%%%%%%%%%%%%%%%%%%%%%%%%%%%%%%%%%%
\subsection{安全性の向上の目標}

%%%%%%%%%%%%%%%%%%%%%%%%%%%%%%%%%%%%%%%%%%%%%%%%%%%%%%%%%%
%% subsubsection 04.02.01.1 %%%%%%%%%%%%%%%%%%%%%%%%%%%%%%
%%%%%%%%%%%%%%%%%%%%%%%%%%%%%%%%%%%%%%%%%%%%%%%%%%%%%%%%%%
\subsubsection{肉体的負担のある作業の削減}
\begin{enumerate}[label=\sarrow]
\item 機内に侵入する作業の削減
\item 手持ち研磨機による加工の廃止
\end{enumerate}

%%%%%%%%%%%%%%%%%%%%%%%%%%%%%%%%%%%%%%%%%%%%%%%%%%%%%%%%%%
%% subsubsection 04.02.01.1 %%%%%%%%%%%%%%%%%%%%%%%%%%%%%%
%%%%%%%%%%%%%%%%%%%%%%%%%%%%%%%%%%%%%%%%%%%%%%%%%%%%%%%%%%
\subsubsection{精神的負担のある作業の削減}
\begin{enumerate}[label=\sarrow]
\item 手動による操作盤を用いた調整作業の大幅削減・自動化
\item 手動による\TouchSensorProbe を用いた測定作業の大幅削減・自動化
\end{enumerate}


%%%%%%%%%%%%%%%%%%%%%%%%%%%%%%%%%%%%%%%%%%%%%%%%%%%%%%%%%%
%% subsection 04.02.02 %%%%%%%%%%%%%%%%%%%%%%%%%%%%%%%%%%%
%%%%%%%%%%%%%%%%%%%%%%%%%%%%%%%%%%%%%%%%%%%%%%%%%%%%%%%%%%
\subsection{品質の向上の目標}

%%%%%%%%%%%%%%%%%%%%%%%%%%%%%%%%%%%%%%%%%%%%%%%%%%%%%%%%%%
%% subsubsection 04.02.02.1 %%%%%%%%%%%%%%%%%%%%%%%%%%%%%%
%%%%%%%%%%%%%%%%%%%%%%%%%%%%%%%%%%%%%%%%%%%%%%%%%%%%%%%%%%
\subsubsection{ヒューマンエラーの削減}
\begin{enumerate}[label=\sarrow]
\item 手作業による計算作業の大幅削減・自動化
\item 手作業による書類記入作業の大幅削減・自動化
\item 衝突防止策の実装
\end{enumerate}

%%%%%%%%%%%%%%%%%%%%%%%%%%%%%%%%%%%%%%%%%%%%%%%%%%%%%%%%%%
%% subsubsection 04.02.02.2 %%%%%%%%%%%%%%%%%%%%%%%%%%%%%%
%%%%%%%%%%%%%%%%%%%%%%%%%%%%%%%%%%%%%%%%%%%%%%%%%%%%%%%%%%
\subsubsection{測定・加工作業の機械化}
\begin{enumerate}[label=\sarrow]
\item 手動による測定作業の大幅削減・機械化
\item 手動による加工作業の大幅削減・機械化
\end{enumerate}


%%%%%%%%%%%%%%%%%%%%%%%%%%%%%%%%%%%%%%%%%%%%%%%%%%%%%%%%%%
%% subsection 04.02.03 %%%%%%%%%%%%%%%%%%%%%%%%%%%%%%%%%%%
%%%%%%%%%%%%%%%%%%%%%%%%%%%%%%%%%%%%%%%%%%%%%%%%%%%%%%%%%%
\subsection{信頼性の向上の目標}
\begin{enumerate}[label=\sarrow]
\item 異常測定値の検出機能の実装
\item 衝突防止策の実装
\end{enumerate}


%\clearpage
%%%%%%%%%%%%%%%%%%%%%%%%%%%%%%%%%%%%%%%%%%%%%%%%%%%%%%%%%%
%% subsection 04.02.05 %%%%%%%%%%%%%%%%%%%%%%%%%%%%%%%%%%%
%%%%%%%%%%%%%%%%%%%%%%%%%%%%%%%%%%%%%%%%%%%%%%%%%%%%%%%%%%
\subsection{属人性の削減の目標}
\begin{enumerate}[label=\sarrow]
\item 手作業による書類記入作業の大幅削減・自動化
\item 手作業による計算および入力作業の大幅削減・自動化
\item 手動による測定作業の大幅削減・機械化
\item 手動による加工作業の大幅削減・機械化
\item \Spacer による振分調整作業の廃止
\item \CurvedOutcutMilling の自動化
\item ワーク\FixtureBolt の選定作業の簡易化
\end{enumerate}



\clearpage
%%%%%%%%%%%%%%%%%%%%%%%%%%%%%%%%%%%%%%%%%%%%%%%%%%%%%%%%%%
%% section 04.03 %%%%%%%%%%%%%%%%%%%%%%%%%%%%%%%%%%%%%%%%%
%%%%%%%%%%%%%%%%%%%%%%%%%%%%%%%%%%%%%%%%%%%%%%%%%%%%%%%%%%
\modHeadsection{解決すべき課題\TBW}
解決すべき課題として、主に以下が挙げられる。
\begin{enumerate}[label=\sarrow]
\item 諸規程の策定
\item 諸規程に則った、諸標準の策定
\item 機内の幾何的情報の一般化および体系化
\item 諸標準に則った、各明細・各工程に対する\index{NCプログラム}NCプログラムの作成
\item モールドの関係データベースの作成
\end{enumerate}

