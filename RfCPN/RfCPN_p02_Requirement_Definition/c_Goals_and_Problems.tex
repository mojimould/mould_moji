%!TEX root = ../RfCPN.tex


\modHeadchapter{加工システム作成における達成したい目標・解決すべき課題}
新たに導入したマシニングセンタで何を達成したいのか、その目的さえも明確にされていないのが現状である。
したがって、ここでは暫定的に目的を定め、具体的な目標の設定を行うことにする。



%%%%%%%%%%%%%%%%%%%%%%%%%%%%%%%%%%%%%%%%%%%%%%%%%%%%%%%%%%
%% section 04.01 %%%%%%%%%%%%%%%%%%%%%%%%%%%%%%%%%%%%%%%%%
%%%%%%%%%%%%%%%%%%%%%%%%%%%%%%%%%%%%%%%%%%%%%%%%%%%%%%%%%%
\modHeadsection{新たなマシニングセンタの導入の目的}
ここでは(暫定的な)目的として以下を採用する。
\begin{enumerate}[label=\sarrow]
\item \MMC における\index{NCプログラム}NCプログラムの全面的見直し
\item マシニングセンタの導入に伴い増加する教育コストの抑制
\item マシニングセンタの導入に伴い低下する作業効率%
%% footnote %%%%%%%%%%%%%%%%%%%%%
\footnote{機械の導入に伴う、\index{NCプログラム}NCプログラム作成や維持管理の効率、機械の空間的制約に起因する効率の低下等が含まれる。}
%%%%%%%%%%%%%%%%%%%%%%%%%%%%%%%%%
の向上
\item \Dimple 加工の実現
\end{enumerate}
%%%%%%%%%%%%%%%%%%%%%%%%%%%%%%%%%%%%%%%%%%%%%%%%%%%%%%%%%%
%% hosoku %%%%%%%%%%%%%%%%%%%%%%%%%%%%%%%%%%%%%%%%%%%%%%%%
%%%%%%%%%%%%%%%%%%%%%%%%%%%%%%%%%%%%%%%%%%%%%%%%%%%%%%%%%%
\begin{hosoku}
後でも述べるように、とにかくマシニングセンタが稼働(生産)できる状態になることを目指している。
そのため、実現の目処の立っていない\ReliefGrooveMilling についてはここでは盛り込まない。

 \Dimple については実現の目処が立つ(幾何的情報が簡単な\index{しょとうきかがく@初等幾何学}初等幾何学で解析的に導出できると想定される)ため、盛り込むことにする。
\end{hosoku}
%%%%%%%%%%%%%%%%%%%%%%%%%%%%%%%%%%%%%%%%%%%%%%%%%%%%%%%%%%
%%%%%%%%%%%%%%%%%%%%%%%%%%%%%%%%%%%%%%%%%%%%%%%%%%%%%%%%%%
%%%%%%%%%%%%%%%%%%%%%%%%%%%%%%%%%%%%%%%%%%%%%%%%%%%%%%%%%%


\clearpage
%%%%%%%%%%%%%%%%%%%%%%%%%%%%%%%%%%%%%%%%%%%%%%%%%%%%%%%%%%
%% section 04.02 %%%%%%%%%%%%%%%%%%%%%%%%%%%%%%%%%%%%%%%%%
%%%%%%%%%%%%%%%%%%%%%%%%%%%%%%%%%%%%%%%%%%%%%%%%%%%%%%%%%%
\modHeadsection{加工システム作成における達成したい目標}


%%%%%%%%%%%%%%%%%%%%%%%%%%%%%%%%%%%%%%%%%%%%%%%%%%%%%%%%%%
%% subsection 04.02.02 %%%%%%%%%%%%%%%%%%%%%%%%%%%%%%%%%%%
%%%%%%%%%%%%%%%%%%%%%%%%%%%%%%%%%%%%%%%%%%%%%%%%%%%%%%%%%%
\subsection{\index{きょういくコスト@教育コスト}教育コストの削減に対する目標}
\begin{enumerate}[label=\sarrow]
\item 諸作業に対する属人性の大幅削減
\end{enumerate}
なお、属人性の削減については、以下のような方針をとるものとする
%% footnote %%%%%%%%%%%%%%%%%%%%%
\footnote{\textgt{書類生成の自動化}や\textgt{コード生成の自動化}などもあるが、これらについては\pageautoref{part:XI}以降で取り扱う。}。
%%%%%%%%%%%%%%%%%%%%%%%%%%%%%%%%%
\begin{enumerate}[label=\sarrow]
\item \textgt{加工の自動化}:手作業による測定・加工の削減・簡易化
\item \textgt{操作の簡易化}:手作業によるマシニングセンタ画面操作の削減・簡易化
\end{enumerate}


%%%%%%%%%%%%%%%%%%%%%%%%%%%%%%%%%%%%%%%%%%%%%%%%%%%%%%%%%%
%% subsection 04.02.03 %%%%%%%%%%%%%%%%%%%%%%%%%%%%%%%%%%%
%%%%%%%%%%%%%%%%%%%%%%%%%%%%%%%%%%%%%%%%%%%%%%%%%%%%%%%%%%
\subsection{全体の作業効率の向上に対する目標}
\begin{enumerate}[label=\sarrow]
\item 諸作業に対する属人性の大幅削減
\item 機内のワークの解析的な位置情報の把握および体系化
\item ワークの加工に関する条件分岐情報の把握および体系化
\end{enumerate}


%%%%%%%%%%%%%%%%%%%%%%%%%%%%%%%%%%%%%%%%%%%%%%%%%%%%%%%%%%
%% subsection 04.02.01 %%%%%%%%%%%%%%%%%%%%%%%%%%%%%%%%%%%
%%%%%%%%%%%%%%%%%%%%%%%%%%%%%%%%%%%%%%%%%%%%%%%%%%%%%%%%%%
\subsection{\Dimple 加工の実現に対する目標}
\begin{enumerate}[label=\sarrow]
\item \Dimple に関する解析的な位置情報および\index{じょうけんぶんきじょうほう@条件分岐情報}条件分岐情報の把握および体系化
\item \Dimple 位置測定用\index{NCプログラム}NCプログラムの作成
\item \DimpleMilling 用\index{NCプログラム}NCプログラムの作成
\end{enumerate}


%%%%%%%%%%%%%%%%%%%%%%%%%%%%%%%%%%%%%%%%%%%%%%%%%%%%%%%%%%
%% subsection 04.02.04 %%%%%%%%%%%%%%%%%%%%%%%%%%%%%%%%%%%
%%%%%%%%%%%%%%%%%%%%%%%%%%%%%%%%%%%%%%%%%%%%%%%%%%%%%%%%%%
\subsection{その他の目標}
\begin{enumerate}[label=\sarrow]
\item 作業員の安全性の考慮
\item 製品・コードの品質の考慮
\item 機械の信頼性の考慮
\end{enumerate}



\clearpage
%%%%%%%%%%%%%%%%%%%%%%%%%%%%%%%%%%%%%%%%%%%%%%%%%%%%%%%%%%
%% section 04.03 %%%%%%%%%%%%%%%%%%%%%%%%%%%%%%%%%%%%%%%%%
%%%%%%%%%%%%%%%%%%%%%%%%%%%%%%%%%%%%%%%%%%%%%%%%%%%%%%%%%%
\modHeadsection{加工システム作成における解決すべき課題}
\begin{enumerate}[label=\sarrow]
\item ソフトウェアエンジニアリングに関連する諸規程の策定
\item マシニングセンタ内におけるワークの幾何的情報の解析的な把握および体系化
\item 諸規程に則った、マシニングセンタ用のNCプログラム作成に関する諸標準の策定
\item 諸標準に則った、各明細・各工程に対する\index{NCプログラム}NCプログラムの作成
\end{enumerate}
%%%%%%%%%%%%%%%%%%%%%%%%%%%%%%%%%%%%%%%%%%%%%%%%%%%%%%%%%%
%% hosoku %%%%%%%%%%%%%%%%%%%%%%%%%%%%%%%%%%%%%%%%%%%%%%%%
%%%%%%%%%%%%%%%%%%%%%%%%%%%%%%%%%%%%%%%%%%%%%%%%%%%%%%%%%%
\begin{hosoku}
これらの前提として、以下のようなより根本的な課題を解決する必要がある。
\begin{enumerate}[label=\sarrow]
\item[{\sarrow[red]}] 当社の、科学的事実に基づく論理的判断をする能力の修得
\item[{\sarrow[red]}] 当社の、ソフトウェアエンジニアリングに関する理解力およびモラルの会得
\item[{\sarrow[red]}] ソフトウェアエンジニアリング部門の創設
\item[{\sarrow[red]}] 当社の、ソフトウェアエンジニアリングに関する\index{リーダーシップ}リーダーシップの会得
\item[{\sarrow[red]}] 当社のプログラマ育成能力の会得
\item[{\sarrow[red]}] プログラマの育成・増員
\item[{\sarrow[red]}] 当社の数学応用能力(中学3年~高校1年程度)の修得
\item[{\sarrow[red]}] 当社のプログラミングに関わる初等スキル(論理的思考力・読解力)の修得
\end{enumerate}
しかし、これらは事実上解決が不可能なので、ここでは盛り込まない。
\end{hosoku}
%%%%%%%%%%%%%%%%%%%%%%%%%%%%%%%%%%%%%%%%%%%%%%%%%%%%%%%%%%
%%%%%%%%%%%%%%%%%%%%%%%%%%%%%%%%%%%%%%%%%%%%%%%%%%%%%%%%%%
%%%%%%%%%%%%%%%%%%%%%%%%%%%%%%%%%%%%%%%%%%%%%%%%%%%%%%%%%%


