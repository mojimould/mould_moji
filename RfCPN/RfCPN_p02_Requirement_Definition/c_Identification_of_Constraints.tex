%!TEX root = ../RfCPN.tex


\modHeadchapter{制約の特定}
% 予算、時間、既存のシステムとの互換性など、プロジェクトに影響を与える可能性のある制約を特定



\modHeadsection{人員の制約}
前述のとおり、加工システムの作成に関しては、事実上(管理職レベルのもの含め)職務がすべて放棄されている。
したがって加工システム作成作業は、そのすべてにおいて、1人(本書の著者)のみで行う。



\modHeadsection{開発期間の制約}
開発期間に関する制約は何も与えられていない。
そこで、ここでは稼働(製品の加工)までの概ねの目安として、\Dimple についてのNCプログラム作成に1年、その他の作業に1年の、計2年程度を目指すものとする
%% footnote %%%%%%%%%%%%%%%%%%%%%
\footnote{追記:2023年9月下旬(機械設置時)に着手し、2024年3月下旬頃に\index{ほんばんかんきょう@本番環境}本番環境での稼働に至った。}。
%%%%%%%%%%%%%%%%%%%%%%%%%%%%%%%%%



\modHeadsection{加工の空間的制約\TBW}
(to be written...)



\modHeadsection{既存のシステムとの互換性\TBW}
(to be written...)


