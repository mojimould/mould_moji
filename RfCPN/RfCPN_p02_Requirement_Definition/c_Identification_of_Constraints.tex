%!TEX root = ../RfCPN.tex


\modHeadchapter{制約の特定}
% 予算、時間、既存のシステムとの互換性など、プロジェクトに影響を与える可能性のある制約を特定



\modHeadsection{人員の制約}
前述のとおり、加工システムの作成に関しては、事実上(管理職レベルの内容を始め)職務がすべて放棄されている
%% footnote %%%%%%%%%%%%%%%%%%%%%
\footnote{ある管理職は、「この業務は当社の優先事項ではない」と断言された。}。
%%%%%%%%%%%%%%%%%%%%%%%%%%%%%%%%%
さらに、こうした状況の中、加工システム作成作業に携わることのできる人員は、1人(本書の著者)のみである。



\modHeadsection{開発期間の制約}
開発期間に関する制約についても何も与えられていない。
そこで、ここでは概ねの目安として、(\index{しゅうせいほしゅ@修正保守}修正保守を含め)1年半程度で行うものとする。
%%%%%%%%%%%%%%%%%%%%%%%%%%%%%%%%%%%%%%%%%%%%%%%%%%%%%%%%%%
%% marker %%%%%%%%%%%%%%%%%%%%%%%%%%%%%%%%%%%%%%%%%%%%%%%%
%%%%%%%%%%%%%%%%%%%%%%%%%%%%%%%%%%%%%%%%%%%%%%%%%%%%%%%%%%
\begin{marker}
追記:2023年9月下旬に着手し、2024年3月下旬頃に\index{ほんばんかんきょう@本番環境}本番環境での製品の加工に至った。
\end{marker}
%%%%%%%%%%%%%%%%%%%%%%%%%%%%%%%%%%%%%%%%%%%%%%%%%%%%%%%%%%
%%%%%%%%%%%%%%%%%%%%%%%%%%%%%%%%%%%%%%%%%%%%%%%%%%%%%%%%%%
%%%%%%%%%%%%%%%%%%%%%%%%%%%%%%%%%%%%%%%%%%%%%%%%%%%%%%%%%%



\modHeadsection{加工の空間的制約\TBW}
(to be written...)


