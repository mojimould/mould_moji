%!TEX root = ./RfCPN.tex


\addtocontents{toc}{\protect\cleardoublepage}
\addtocontents{lot}{\protect\tcbline*}
%%%%%%%%%%%%%%%%%%%%%%%%%%%%%%%%%%%%%%%%%%%%%%%%%%%%%%%%%
%% Part MANUAL %%%%%%%%%%%%%%%%%%%%%%%%%%%%%%%%%%%%%%%%%%
%%%%%%%%%%%%%%%%%%%%%%%%%%%%%%%%%%%%%%%%%%%%%%%%%%%%%%%%%
\addtocontents{toc}{\protect\begin{tocBox}{\tmppartnum}}%
\tPart[lot,locode]{各工程用\index{NCプログラム}NCプログラムの作成}{%
\paragraph*{\tpartgoal}
具体的な\Dimple 用の\index{NCプログラム}NCプログラムを作成する。
加えて、その他の工程に対するNCプログラムの新たな作成も行う。
\tcbline*
\paragraph*{\tpartmethod}
前段階までに整理した\index{こゆうすうちじょうほう@固有数値情報}固有数値情報および\index{じょうけんぶんきじょうほう@条件分岐情報}条件分岐情報を\index{ひきすう@引数}引数とする形でコードを記述する。
\tcbline*
\paragraph*{\tpartbackground}
\Dimple 用のNCプログラムが必要なことは言うまでもない。
その他の工程に対しては、従来のNCプログラムでは人手による煩雑な計算や\index{メインプログラム}メインプログラムの直接編集といったことをしなければならない。
すなわち、作業員に対して多くの複雑かつ不必要な操作を強いている状態にある。
こうした状態が数十年にわたって放置され続けており、\index{ひんしつ@品質}品質・\index{しんらいせい@信頼性}信頼性の低下(不良率・トラブル等の増加)や\index{きょういくコスト@教育コスト}教育コストの増加をもたらし続けている
%% footnote %%%%%%%%%%%%%%%%%%%%%
\footnote{そしてその皺寄せをすべて作業員に押し付けている。
これはさすがに従来の管理・監督者らの怠慢と言わざるを得ない。}。
%%%%%%%%%%%%%%%%%%%%%%%%%%%%%%%%%

 このような事態の解消は管理・監督者の責務であるが、それが全く期待できないことは明白である。
\Dimple 用のNCプログラムの作成だけにとどまらず、その他の工程用プログラムの見直しという高度な業務でさえ、一般職に強いている状態にある。
}{%
\paragraph*{\tpartconclusion}
加工に必要なすべてのNCプログラムを作成した。
\tcbline*
\paragraph*{\tpartnextstep}
作成したNCプログラムや関連操作等に対するマニュアルの作成を行う。
}

%%%%%%%%%%%%%%%%%%%%%%%%%%%%%%%%%%%%%%%%%%%%%%%%%%%%%%%%%%
%% chapters %%%%%%%%%%%%%%%%%%%%%%%%%%%%%%%%%%%%%%%%%%%%%%
%%%%%%%%%%%%%%%%%%%%%%%%%%%%%%%%%%%%%%%%%%%%%%%%%%%%%%%%%%
%!TEX root = ./RfCPN.tex


\modHeadchapter[lot]{\CreatedNCPrg の取扱説明\TBW}



%%%%%%%%%%%%%%%%%%%%%%%%%%%%%%%%%%%%%%%%%%%%%%%%%%%%%%%%%%
%% section 46.01 %%%%%%%%%%%%%%%%%%%%%%%%%%%%%%%%%%%%%%%%%
%%%%%%%%%%%%%%%%%%%%%%%%%%%%%%%%%%%%%%%%%%%%%%%%%%%%%%%%%%
\modHeadsection{\CreatedNCPrg{} 一覧}


%%%%%%%%%%%%%%%%%%%%%%%%%%%%%%%%%%%%%%%%%%%%%%%%%%%%%%%%%%
%% subsection 46.01.01 %%%%%%%%%%%%%%%%%%%%%%%%%%%%%%%%%%%
%%%%%%%%%%%%%%%%%%%%%%%%%%%%%%%%%%%%%%%%%%%%%%%%%%%%%%%%%%
\subsection{\CreatedNCMainPrg の例 一覧}
\DMC において\CreatedNCMainPrg は以下のとおりである。
%%%%%%%%%%%%%%%%%%%%%%%%%%%%%%%%%%%%%%%%%%%%%%%%%%%%%%%%%%
%% marker %%%%%%%%%%%%%%%%%%%%%%%%%%%%%%%%%%%%%%%%%%%%%%%%
%%%%%%%%%%%%%%%%%%%%%%%%%%%%%%%%%%%%%%%%%%%%%%%%%%%%%%%%%%
\begin{marker}
ここでいう\index{メインプログラム}メインプログラムとは、\DrawingNumber と同一の\index{プログラムばんごう@プログラム番号}プログラム番号のものを指す。
ただし、ここでは例として挙げており、\pageautoref{subsec:notopenwork}に伴い、\index{プログラムばんごう@プログラム番号}プログラム番号は\pageautoref{chap:VI.24}の規則には則っていない。
\end{marker}
%%%%%%%%%%%%%%%%%%%%%%%%%%%%%%%%%%%%%%%%%%%%%%%%%%%%%%%%%%
%%%%%%%%%%%%%%%%%%%%%%%%%%%%%%%%%%%%%%%%%%%%%%%%%%%%%%%%%%
%%%%%%%%%%%%%%%%%%%%%%%%%%%%%%%%%%%%%%%%%%%%%%%%%%%%%%%%%%

\begin{multicollongtblr}{\CreatedNCPrg 一覧:\index{メインプログラムのれい@メインプログラムの例}メインプログラムの例}{ccX[l]}
{\ttfamily O}番号 & \SetCell{c}工程 & 使用subprg\\
\SetCell[r=9]{c}
\MainExOne & 外側芯出し・幅 & \MYOThickness\MXIface\\
           & 内側芯出し・幅 & \MXIWidth\MYIWidth\\
           & \expandafterindex{\yomiCenterlineEndFaceDifMeasurement@\nameCenterlineEndFaceDifMeasurement}\nameCenterlineEndFaceDif & \MCenterline\\
           & \Dimple & \DLone\\
           & \EndFacecut & \KEndFaceRight\\
           & \Outcut & \KOutcutRLeft\\
           & \Keyway & \KKeywayConerLeft\\
           & \EndFaceChamfer & \KEndFaceOutCChamferRLeft\KEndFaceInCChamferRLeft\\
           & その他 & \OpauseCheck\OsensorOn\OsensorOff\\
\hline
\SetCell[r=10]{c}
\MainExTwo & 外側芯出し・幅 & \MXOThickness\MYOThickness\MXOface\\
           & 内側芯出し・幅 & \MXIWidth\MYIWidth\\
           & \Dimple & \DLone\\
           & \EndFacecut & \KEndFaceRight\\
           & \CurvedOutcut & \KCurvedOutcutRLeft\\
           & \Keyway & \KKeywayConerLeft\\
           & \EndFaceChamfer & \KEndFaceOutCChamferRLeft\KEndFaceInCChamferRLeft\KEndFaceCurvedOutCChamferRLeft\\
           & \EndFaceBoring & \KEndFaceBoring\\
           & \IncutBoring & \KIncutBoring\\
           & その他 & \OpauseCheck\OsensorOn\OsensorOff\\
\end{multicollongtblr}


\clearpage
%%%%%%%%%%%%%%%%%%%%%%%%%%%%%%%%%%%%%%%%%%%%%%%%%%%%%%%%%%
%% subsection 46.01.02 %%%%%%%%%%%%%%%%%%%%%%%%%%%%%%%%%%%
%%%%%%%%%%%%%%%%%%%%%%%%%%%%%%%%%%%%%%%%%%%%%%%%%%%%%%%%%%
\subsection{\CreatedNCSubPrg{} 一覧}
\DMC において\CreatedNCSubPrg は以下のとおりである
%%%%%%%%%%%%%%%%%%%%%%%%%%%%%%%%%%%%%%%%%%%%%%%%%%%%%%%%%%
%% marker %%%%%%%%%%%%%%%%%%%%%%%%%%%%%%%%%%%%%%%%%%%%%%%%
%%%%%%%%%%%%%%%%%%%%%%%%%%%%%%%%%%%%%%%%%%%%%%%%%%%%%%%%%%
\begin{marker}
ここでいう\index{サブプログラム}サブプログラムとは、\DrawingNumber と同一の\index{プログラムばんごう@プログラム番号}プログラム番号以外のものを指す
%% footnote %%%%%%%%%%%%%%%%%%%%%
\footnote{\OwarmingupA や\OtoolLengthA はレベル1の階層で用いるため、\index{メインプログラム}メインプログラムと称すのがより適切であるが、ここでは便宜上\index{サブプログラム}サブプログラムと称している。}。
%%%%%%%%%%%%%%%%%%%%%%%%%%%%%%%%%
\end{marker}
%%%%%%%%%%%%%%%%%%%%%%%%%%%%%%%%%%%%%%%%%%%%%%%%%%%%%%%%%%
%%%%%%%%%%%%%%%%%%%%%%%%%%%%%%%%%%%%%%%%%%%%%%%%%%%%%%%%%%
%%%%%%%%%%%%%%%%%%%%%%%%%%%%%%%%%%%%%%%%%%%%%%%%%%%%%%%%%%

\begin{multicollongtblr}{\CreatedNCSubPrg 一覧:芯出し・幅・\CenterlineEndFaceDifMeasurement}{cX[l]l}
{\ttfamily O}番号 & 内容 & 使用subprg\\
\MXOThickness & 測定 両側 外側中心・幅$X$ & \OsensorOff\\
\MYOThickness & 測定 両側 外側中心・幅$Y$ & \OsensorOff\\
\MXOface      & 測定 片側 \KeywayCenter$X$(トップA側外面測定) & \OsensorOff\\
\MXIWidth     & 測定 両側 内側中心・幅$X$ & \OsensorOff\\
\MYIWidth     & 測定 両側 内側中心・幅$Y$ & \OsensorOff\\
\MXIface      & 測定 片側 \OutcutCenter$X$(C側内面方向測定) & \OsensorOff\\
\MCenterline  & 測定 片側 \expandafterindex{\yomiCenterlineEndFaceDifMeasurement@\nameCenterlineEndFaceDifMeasurement}\nameCenterlineEndFaceDif(C側外削面 $Z$方向, B側外削面 $Y$方向測定) & \OsensorOff\\
\end{multicollongtblr}

\begin{multicollongtblr}{\CreatedNCSubPrg 一覧:\Dimple}{cX[l]l}
{\ttfamily O}番号 & 内容 & 使用subprg\\
\DLone      & 移動 各列の中心上 & \DLtwoAC\DLtwoBD\\
\DLtwoAC    & 移動 AC面 列内の各\Dimple 上 & \DMLthreeAC\DKLthreeAC\\
\DLtwoBD    & 移動 BC面 列内の各\Dimple 上 & \DMLthreeBD\DKLthreeBD\\
\DMLthreeAC & 測定 AC内表面$X$ & \OsensorOff\\
\DMLthreeBD & 測定 BD内表面$Y$ & \OsensorOff\\
\DKLthreeAC & 加工 AC内表面$X$ & -\\
\DKLthreeBD & 加工 BD内表面$Y$ & -\\
\end{multicollongtblr}

\clearpage
\begin{multicollongtblr}{\CreatedNCSubPrg 一覧:加工(\Dimple 以外)}{cX[l]l}
{\ttfamily O}番号 & 内容 & 使用subprg\\
\KEndFaceRight                  & 加工 \EndFacecut{} コーナーR 右回り1周 & \KOLeftFS\\
\KOutcutRLeft                   & 加工 \Outcut{} コーナーR 左回り1周 & \KOLeftFS\OpauseCheck\\
\KCurvedOutcutRLeft             & 加工 \CurvedOutcut{} コーナーR 左回り1周 & \KOLeftFSZ\OpauseCheck\\
\KKeywayConerLeft               & 加工 \Keyway{} 左回り1周 & \KOLeftFS\OpauseCheck\\
\KEndFaceOutCChamferRLeft       & 加工 \EndFaceOutChamfer{} コーナーR 左回り1周 & \KOLeftFS\OpauseCheck\\
\KEndFaceCurvedOutCChamferRLeft & 加工 \CurvedOutcut 用\EndFaceOutChamfer{} コーナーR 左回り1周 & \KOLeftFS\OpauseCheck\\
\KEndFaceInCChamferRLeft        & 加工 \EndFaceInChamfer{} コーナーR 左回り1周 & \KILeftFF\OpauseCheck\\
\KEndFaceBoring                 & 加工 \EndFaceBoring{} コーナーR 左回り1周 & \KOLeftFF\OpauseCheck\\
\KIncutBoring                   & 加工 \IncutBoring{} コーナーR 左回り1周 & \KILeftFF\OpauseCheck\\
\KOLeftFF  & 外側 {\ttfamily G41} 左回り1周 右始まり & -\\
\KOLeftFS  & 外側 {\ttfamily G42} 左回り1周 右上始まり & -\\
\KOLeftFSZ & 外側 {\ttfamily G42} 左回り1周 右上始まり $Z$変動 & -\\
\KILeftFF  & 内側 {\ttfamily G41} 左回り1周 中央上始まり & -\\
\end{multicollongtblr}

%\clearpage
\begin{multicollongtblr}{\CreatedNCSubPrg 一覧:その他}{cX[l]l}
{\ttfamily O}番号 & 内容 & 使用subprg\\
\OpauseCheck  & 移動・加工後確認用:90$^\circ$回転 扉前一時停止 & -\\
\OsensorOn    & \index{タッチセンサーでんげん@タッチセンサー電源}タッチセンサー電源ON & -\\
\OsensorOff   & \index{タッチセンサーでんげん@タッチセンサー電源}タッチセンサー電源OFF & -\\
\OwarmingupA  & \index{だんきうんてん@暖機運転}暖機運転 & \Owarmingup\\
\Owarmingup   & \index{だんきうんてんようサブプログラム@暖機運転用サブプログラム}暖機運転用サブプログラム & -\\
\OtoolLengthA & \index{とうろくこうぐ@登録工具}登録工具 \TLCorrection & \OtoolLength\\
\OtoolLength  & \expandafterindex{\yomiTLCorrection ようサブプログラム@\nameTLCorrection 用サブプログラム}\nameTLCorrection 用サブプログラム & -\\
\end{multicollongtblr}


\clearpage
%%%%%%%%%%%%%%%%%%%%%%%%%%%%%%%%%%%%%%%%%%%%%%%%%%%%%%%%%%
%% section 46.02 %%%%%%%%%%%%%%%%%%%%%%%%%%%%%%%%%%%%%%%%%
%%%%%%%%%%%%%%%%%%%%%%%%%%%%%%%%%%%%%%%%%%%%%%%%%%%%%%%%%%
\modHeadsection{\MXOThickness :測定 両側 外側中心・幅\texorpdfstring{$X$}{X}}


%%%%%%%%%%%%%%%%%%%%%%%%%%%%%%%%%%%%%%%%%%%%%%%%%%%%%%%%%%
%% subsection 46.02.01 %%%%%%%%%%%%%%%%%%%%%%%%%%%%%%%%%%%
%%%%%%%%%%%%%%%%%%%%%%%%%%%%%%%%%%%%%%%%%%%%%%%%%%%%%%%%%%
\subsection{\MXOThicknessArguments}

\begin{multicollongtblr}{\MXOThicknessArguments}{ccX[l]}
引数 & 変数 & 内容\\
{\ttfamily H} & {\ttfamily\ttNum11} & \KeywayWidth\\
{\ttfamily M} & {\ttfamily\ttNum13} & \KeywayPos\\
{\ttfamily R} & {\ttfamily\ttNum18} & \CenterCurvatureRadius\\
{\ttfamily W} & {\ttfamily\ttNum23} & \AlocationLength\\
{\ttfamily X} & {\ttfamily\ttNum24} & \ACOD\\
{\ttfamily Z} & {\ttfamily\ttNum26} & \ReAlocationLength\\
\end{multicollongtblr}


%%%%%%%%%%%%%%%%%%%%%%%%%%%%%%%%%%%%%%%%%%%%%%%%%%%%%%%%%%
%% subsection 46.02.02 %%%%%%%%%%%%%%%%%%%%%%%%%%%%%%%%%%%
%%%%%%%%%%%%%%%%%%%%%%%%%%%%%%%%%%%%%%%%%%%%%%%%%%%%%%%%%%
\subsection{\MXOThickness の取扱説明\TBW}
(to be written...)


%%%%%%%%%%%%%%%%%%%%%%%%%%%%%%%%%%%%%%%%%%%%%%%%%%%%%%%%%%
%% subsection 46.02.02 %%%%%%%%%%%%%%%%%%%%%%%%%%%%%%%%%%%
%%%%%%%%%%%%%%%%%%%%%%%%%%%%%%%%%%%%%%%%%%%%%%%%%%%%%%%%%%
\subsection{\MXOThickness の注意事項\TBW}
(to be written...)



\clearpage
%%%%%%%%%%%%%%%%%%%%%%%%%%%%%%%%%%%%%%%%%%%%%%%%%%%%%%%%%%
%% section 46.03 %%%%%%%%%%%%%%%%%%%%%%%%%%%%%%%%%%%%%%%%%
%%%%%%%%%%%%%%%%%%%%%%%%%%%%%%%%%%%%%%%%%%%%%%%%%%%%%%%%%%
\modHeadsection{\MYOThickness :測定 両側 外側中心・幅\texorpdfstring{$Y$}{Y}}


%%%%%%%%%%%%%%%%%%%%%%%%%%%%%%%%%%%%%%%%%%%%%%%%%%%%%%%%%%
%% subsection 46.02.01 %%%%%%%%%%%%%%%%%%%%%%%%%%%%%%%%%%%
%%%%%%%%%%%%%%%%%%%%%%%%%%%%%%%%%%%%%%%%%%%%%%%%%%%%%%%%%%
\subsection{\MYOThicknessArguments}

\begin{multicollongtblr}{\MYOThicknessArguments}{ccX[l]}
引数 & 変数 & 内容\\
{\ttfamily H} & {\ttfamily\ttNum11} & \KeywayWidth\\
{\ttfamily M} & {\ttfamily\ttNum13} & \KeywayPos\\
{\ttfamily Y} & {\ttfamily\ttNum25} & \BDOD\\
{\ttfamily Z} & {\ttfamily\ttNum26} & \ReAlocationLength\\
\end{multicollongtblr}


%%%%%%%%%%%%%%%%%%%%%%%%%%%%%%%%%%%%%%%%%%%%%%%%%%%%%%%%%%
%% subsection 46.02.02 %%%%%%%%%%%%%%%%%%%%%%%%%%%%%%%%%%%
%%%%%%%%%%%%%%%%%%%%%%%%%%%%%%%%%%%%%%%%%%%%%%%%%%%%%%%%%%
\subsection{\MYOThickness の取扱説明\TBW}
(to be written...)


%%%%%%%%%%%%%%%%%%%%%%%%%%%%%%%%%%%%%%%%%%%%%%%%%%%%%%%%%%
%% subsection 46.02.02 %%%%%%%%%%%%%%%%%%%%%%%%%%%%%%%%%%%
%%%%%%%%%%%%%%%%%%%%%%%%%%%%%%%%%%%%%%%%%%%%%%%%%%%%%%%%%%
\subsection{\MYOThickness の注意事項\TBW}
(to be written...)



\clearpage
%%%%%%%%%%%%%%%%%%%%%%%%%%%%%%%%%%%%%%%%%%%%%%%%%%%%%%%%%%
%% section 46.04 %%%%%%%%%%%%%%%%%%%%%%%%%%%%%%%%%%%%%%%%%
%%%%%%%%%%%%%%%%%%%%%%%%%%%%%%%%%%%%%%%%%%%%%%%%%%%%%%%%%%
\modHeadsection{\MXOface :測定 片側 \KeywayCenter\texorpdfstring{$X$}{X}}


%%%%%%%%%%%%%%%%%%%%%%%%%%%%%%%%%%%%%%%%%%%%%%%%%%%%%%%%%%
%% subsection 46.02.01 %%%%%%%%%%%%%%%%%%%%%%%%%%%%%%%%%%%
%%%%%%%%%%%%%%%%%%%%%%%%%%%%%%%%%%%%%%%%%%%%%%%%%%%%%%%%%%
\subsection{\MXOfaceArguments}

\begin{multicollongtblr}{\MXOfaceArguments}{ccX[l]}
引数 & 変数 & 内容\\
{\ttfamily I} & {\ttfamily\ttNum04} & \AsideKeywayDepth\\
{\ttfamily H} & {\ttfamily\ttNum11} & \KeywayWidth\\
{\ttfamily M} & {\ttfamily\ttNum13} & \KeywayPos\\
{\ttfamily U} & {\ttfamily\ttNum21} & \ACOD\\
{\ttfamily X} & {\ttfamily\ttNum24} & \KeywayACOD\\
{\ttfamily Z} & {\ttfamily\ttNum26} & \ReAlocationLength\\
\end{multicollongtblr}


%%%%%%%%%%%%%%%%%%%%%%%%%%%%%%%%%%%%%%%%%%%%%%%%%%%%%%%%%%
%% subsection 46.02.02 %%%%%%%%%%%%%%%%%%%%%%%%%%%%%%%%%%%
%%%%%%%%%%%%%%%%%%%%%%%%%%%%%%%%%%%%%%%%%%%%%%%%%%%%%%%%%%
\subsection{\MXOface の取扱説明\TBW}
(to be written...)


%%%%%%%%%%%%%%%%%%%%%%%%%%%%%%%%%%%%%%%%%%%%%%%%%%%%%%%%%%
%% subsection 46.02.02 %%%%%%%%%%%%%%%%%%%%%%%%%%%%%%%%%%%
%%%%%%%%%%%%%%%%%%%%%%%%%%%%%%%%%%%%%%%%%%%%%%%%%%%%%%%%%%
\subsection{\MXOface の注意事項\TBW}
(to be written...)



\clearpage
%%%%%%%%%%%%%%%%%%%%%%%%%%%%%%%%%%%%%%%%%%%%%%%%%%%%%%%%%%
%% section 46.2 %%%%%%%%%%%%%%%%%%%%%%%%%%%%%%%%%%%%%%%%%%
%%%%%%%%%%%%%%%%%%%%%%%%%%%%%%%%%%%%%%%%%%%%%%%%%%%%%%%%%%
\modHeadsection{\MXIWidth :測定 両側 外側中心・幅\texorpdfstring{$X$}{X}}


%%%%%%%%%%%%%%%%%%%%%%%%%%%%%%%%%%%%%%%%%%%%%%%%%%%%%%%%%%
%% subsection 46.02.01 %%%%%%%%%%%%%%%%%%%%%%%%%%%%%%%%%%%
%%%%%%%%%%%%%%%%%%%%%%%%%%%%%%%%%%%%%%%%%%%%%%%%%%%%%%%%%%
\subsection{\MXIWidthArguments}

\begin{multicollongtblr}{\MXIWidthArguments}{ccX[l]}
引数 & 変数 & 内容\\
{\ttfamily M} & {\ttfamily\ttNum13} & \PlatingThk\\
{\ttfamily R} & {\ttfamily\ttNum18} & \CenterCurvatureRadius\\
{\ttfamily W} & {\ttfamily\ttNum23} & \AlocationLength\\
{\ttfamily X} & {\ttfamily\ttNum24} & \ACID\\
{\ttfamily Z} & {\ttfamily\ttNum26} & \ReAlocationLength\\
\end{multicollongtblr}


%%%%%%%%%%%%%%%%%%%%%%%%%%%%%%%%%%%%%%%%%%%%%%%%%%%%%%%%%%
%% subsection 46.02.02 %%%%%%%%%%%%%%%%%%%%%%%%%%%%%%%%%%%
%%%%%%%%%%%%%%%%%%%%%%%%%%%%%%%%%%%%%%%%%%%%%%%%%%%%%%%%%%
\subsection{\MXIWidth の取扱説明\TBW}
(to be written...)


%%%%%%%%%%%%%%%%%%%%%%%%%%%%%%%%%%%%%%%%%%%%%%%%%%%%%%%%%%
%% subsection 46.02.02 %%%%%%%%%%%%%%%%%%%%%%%%%%%%%%%%%%%
%%%%%%%%%%%%%%%%%%%%%%%%%%%%%%%%%%%%%%%%%%%%%%%%%%%%%%%%%%
\subsection{\MXIWidth の注意事項\TBW}
(to be written...)



\clearpage
%%%%%%%%%%%%%%%%%%%%%%%%%%%%%%%%%%%%%%%%%%%%%%%%%%%%%%%%%%
%% section 46.2 %%%%%%%%%%%%%%%%%%%%%%%%%%%%%%%%%%%%%%%%%%
%%%%%%%%%%%%%%%%%%%%%%%%%%%%%%%%%%%%%%%%%%%%%%%%%%%%%%%%%%
\modHeadsection{\MYIWidth :測定 両側 外側中心・幅\texorpdfstring{$Y$}{Y}}


%%%%%%%%%%%%%%%%%%%%%%%%%%%%%%%%%%%%%%%%%%%%%%%%%%%%%%%%%%
%% subsection 46.02.01 %%%%%%%%%%%%%%%%%%%%%%%%%%%%%%%%%%%
%%%%%%%%%%%%%%%%%%%%%%%%%%%%%%%%%%%%%%%%%%%%%%%%%%%%%%%%%%
\subsection{\MYIWidthArguments}

\begin{multicollongtblr}{\MYIWidthArguments}{ccX[l]}
引数 & 変数 & 内容\\
{\ttfamily M} & {\ttfamily\ttNum13} & \PlatingThk\\
{\ttfamily Y} & {\ttfamily\ttNum25} & \BDID\\
{\ttfamily Z} & {\ttfamily\ttNum26} & \ReAlocationLength\\
\end{multicollongtblr}


%%%%%%%%%%%%%%%%%%%%%%%%%%%%%%%%%%%%%%%%%%%%%%%%%%%%%%%%%%
%% subsection 46.02.02 %%%%%%%%%%%%%%%%%%%%%%%%%%%%%%%%%%%
%%%%%%%%%%%%%%%%%%%%%%%%%%%%%%%%%%%%%%%%%%%%%%%%%%%%%%%%%%
\subsection{\MYIWidth の取扱説明\TBW}
(to be written...)


%%%%%%%%%%%%%%%%%%%%%%%%%%%%%%%%%%%%%%%%%%%%%%%%%%%%%%%%%%
%% subsection 46.02.02 %%%%%%%%%%%%%%%%%%%%%%%%%%%%%%%%%%%
%%%%%%%%%%%%%%%%%%%%%%%%%%%%%%%%%%%%%%%%%%%%%%%%%%%%%%%%%%
\subsection{\MYIWidth の注意事項\TBW}
(to be written...)



\clearpage
%%%%%%%%%%%%%%%%%%%%%%%%%%%%%%%%%%%%%%%%%%%%%%%%%%%%%%%%%%
%% section 46.2 %%%%%%%%%%%%%%%%%%%%%%%%%%%%%%%%%%%%%%%%%%
%%%%%%%%%%%%%%%%%%%%%%%%%%%%%%%%%%%%%%%%%%%%%%%%%%%%%%%%%%
\modHeadsection{\MXIface :測定 片側 \OutcutCenter\texorpdfstring{$X$}{X}}


%%%%%%%%%%%%%%%%%%%%%%%%%%%%%%%%%%%%%%%%%%%%%%%%%%%%%%%%%%
%% subsection 46.02.01 %%%%%%%%%%%%%%%%%%%%%%%%%%%%%%%%%%%
%%%%%%%%%%%%%%%%%%%%%%%%%%%%%%%%%%%%%%%%%%%%%%%%%%%%%%%%%%
\subsection{\MXIfaceArguments}

\begin{multicollongtblr}{\MXIfaceArguments}{ccX[l]}
引数 & 変数 & 内容\\
{\ttfamily C} & {\ttfamily\ttNum03} & \EndFaceInChamferLength\\
{\ttfamily I} & {\ttfamily\ttNum04} & \ACID\\
{\ttfamily M} & {\ttfamily\ttNum13} & \PlatingThk\\
{\ttfamily R} & {\ttfamily\ttNum18} & \CenterCurvatureRadius\\
{\ttfamily T} & {\ttfamily\ttNum20} & \AsideThickness\\
{\ttfamily W} & {\ttfamily\ttNum23} & \AlocationLength\\
{\ttfamily X} & {\ttfamily\ttNum24} & \OutcutACWidth\\
{\ttfamily Z} & {\ttfamily\ttNum26} & \ReAlocationLength\\
\end{multicollongtblr}


%%%%%%%%%%%%%%%%%%%%%%%%%%%%%%%%%%%%%%%%%%%%%%%%%%%%%%%%%%
%% subsection 46.02.02 %%%%%%%%%%%%%%%%%%%%%%%%%%%%%%%%%%%
%%%%%%%%%%%%%%%%%%%%%%%%%%%%%%%%%%%%%%%%%%%%%%%%%%%%%%%%%%
\subsection{\MXIface の取扱説明\TBW}
(to be written...)


%%%%%%%%%%%%%%%%%%%%%%%%%%%%%%%%%%%%%%%%%%%%%%%%%%%%%%%%%%
%% subsection 46.02.02 %%%%%%%%%%%%%%%%%%%%%%%%%%%%%%%%%%%
%%%%%%%%%%%%%%%%%%%%%%%%%%%%%%%%%%%%%%%%%%%%%%%%%%%%%%%%%%
\subsection{\MXIface の注意事項\TBW}
(to be written...)



\clearpage
%%%%%%%%%%%%%%%%%%%%%%%%%%%%%%%%%%%%%%%%%%%%%%%%%%%%%%%%%%
%% section 46.2 %%%%%%%%%%%%%%%%%%%%%%%%%%%%%%%%%%%%%%%%%%
%%%%%%%%%%%%%%%%%%%%%%%%%%%%%%%%%%%%%%%%%%%%%%%%%%%%%%%%%%
\modHeadsection{\MCenterline :\expandafterindex{\yomiCenterlineEndFaceDifMeasurement@\nameCenterlineEndFaceDifMeasurement}測定 片側 \nameCenterlineEndFaceDif}


%%%%%%%%%%%%%%%%%%%%%%%%%%%%%%%%%%%%%%%%%%%%%%%%%%%%%%%%%%
%% subsection 46.02.01 %%%%%%%%%%%%%%%%%%%%%%%%%%%%%%%%%%%
%%%%%%%%%%%%%%%%%%%%%%%%%%%%%%%%%%%%%%%%%%%%%%%%%%%%%%%%%%
\subsection{\MCenterlineArguments}

\begin{multicollongtblr}{\MCenterlineArguments}{ccX[l]}
引数 & 変数 & 内容\\
{\ttfamily A} & {\ttfamily\ttNum01} & \AlocationAngle\\
{\ttfamily B} & {\ttfamily\ttNum02} & \BottomOutcutACWidth\\
{\ttfamily C} & {\ttfamily\ttNum03} & \BottomOutcutBDWidth\\
{\ttfamily I} & {\ttfamily\ttNum04} & \CenterlineEndFaceDif\\
{\ttfamily K} & {\ttfamily\ttNum06} & \BottomOutcutLength\\
{\ttfamily M} & {\ttfamily\ttNum13} & \KeywayPos\\
{\ttfamily T} & {\ttfamily\ttNum20} & \TopOutcutACWidth\\
{\ttfamily U} & {\ttfamily\ttNum21} & \TopOutcutBDWidth\\
{\ttfamily W} & {\ttfamily\ttNum23} & \TopReAlocationLength\\
{\ttfamily Z} & {\ttfamily\ttNum26} & \BottomReAlocationLength\\
\end{multicollongtblr}


%%%%%%%%%%%%%%%%%%%%%%%%%%%%%%%%%%%%%%%%%%%%%%%%%%%%%%%%%%
%% subsection 46.02.02 %%%%%%%%%%%%%%%%%%%%%%%%%%%%%%%%%%%
%%%%%%%%%%%%%%%%%%%%%%%%%%%%%%%%%%%%%%%%%%%%%%%%%%%%%%%%%%
\subsection{\MCenterline の取扱説明\TBW}
(to be written...)


%%%%%%%%%%%%%%%%%%%%%%%%%%%%%%%%%%%%%%%%%%%%%%%%%%%%%%%%%%
%% subsection 46.02.02 %%%%%%%%%%%%%%%%%%%%%%%%%%%%%%%%%%%
%%%%%%%%%%%%%%%%%%%%%%%%%%%%%%%%%%%%%%%%%%%%%%%%%%%%%%%%%%
\subsection{\MCenterline の注意事項\TBW}
(to be written...)



\clearpage
%%%%%%%%%%%%%%%%%%%%%%%%%%%%%%%%%%%%%%%%%%%%%%%%%%%%%%%%%%
%% section 46.2 %%%%%%%%%%%%%%%%%%%%%%%%%%%%%%%%%%%%%%%%%%
%%%%%%%%%%%%%%%%%%%%%%%%%%%%%%%%%%%%%%%%%%%%%%%%%%%%%%%%%%
\modHeadsection{\KEndFaceRight :加工 \EndFacecut{} コーナーR 右回り1周}


%%%%%%%%%%%%%%%%%%%%%%%%%%%%%%%%%%%%%%%%%%%%%%%%%%%%%%%%%%
%% subsection 46.02.01 %%%%%%%%%%%%%%%%%%%%%%%%%%%%%%%%%%%
%%%%%%%%%%%%%%%%%%%%%%%%%%%%%%%%%%%%%%%%%%%%%%%%%%%%%%%%%%
\subsection{\KEndFaceRightArguments}

\begin{multicollongtblr}{\KEndFaceRightArguments}{ccX[l]}
引数 & 変数 & 内容\\
{\ttfamily R} & {\ttfamily\ttNum18} & \CenterCurvatureRadius\\
{\ttfamily Z} & {\ttfamily\ttNum26} & \ReAlocationLength\\
\end{multicollongtblr}


%%%%%%%%%%%%%%%%%%%%%%%%%%%%%%%%%%%%%%%%%%%%%%%%%%%%%%%%%%
%% subsection 46.02.02 %%%%%%%%%%%%%%%%%%%%%%%%%%%%%%%%%%%
%%%%%%%%%%%%%%%%%%%%%%%%%%%%%%%%%%%%%%%%%%%%%%%%%%%%%%%%%%
\subsection{\KEndFaceRight の取扱説明\TBW}
(to be written...)


%%%%%%%%%%%%%%%%%%%%%%%%%%%%%%%%%%%%%%%%%%%%%%%%%%%%%%%%%%
%% subsection 46.02.02 %%%%%%%%%%%%%%%%%%%%%%%%%%%%%%%%%%%
%%%%%%%%%%%%%%%%%%%%%%%%%%%%%%%%%%%%%%%%%%%%%%%%%%%%%%%%%%
\subsection{\KEndFaceRight の注意事項\TBW}
(to be written...)



\clearpage
%%%%%%%%%%%%%%%%%%%%%%%%%%%%%%%%%%%%%%%%%%%%%%%%%%%%%%%%%%
%% section 46.2 %%%%%%%%%%%%%%%%%%%%%%%%%%%%%%%%%%%%%%%%%%
%%%%%%%%%%%%%%%%%%%%%%%%%%%%%%%%%%%%%%%%%%%%%%%%%%%%%%%%%%
\modHeadsection{\KOutcutRLeft :加工 \Outcut{} コーナーR 左回り1周}


%%%%%%%%%%%%%%%%%%%%%%%%%%%%%%%%%%%%%%%%%%%%%%%%%%%%%%%%%%
%% subsection 46.02.01 %%%%%%%%%%%%%%%%%%%%%%%%%%%%%%%%%%%
%%%%%%%%%%%%%%%%%%%%%%%%%%%%%%%%%%%%%%%%%%%%%%%%%%%%%%%%%%
\subsection{\KOutcutRLeftArguments}

\begin{multicollongtblr}{\KOutcutRLeftArguments}{ccX[l]}
引数 & 変数 & 内容\\
{\ttfamily K}     & {\ttfamily\ttNum06} & \OutcutLength\\
{\ttfamily M$^*$} & {\ttfamily\ttNum13} & \KeywayPos\\
{\ttfamily R}     & {\ttfamily\ttNum18} & \OutcutCornerR\\
{\ttfamily U}     & {\ttfamily\ttNum21} & \ACOD\\
{\ttfamily V}     & {\ttfamily\ttNum22} & \BDOD\\
{\ttfamily W$^*$} & {\ttfamily\ttNum23} & \KeywayWidth\\
{\ttfamily X}     & {\ttfamily\ttNum24} & \OutcutACWidth\\
{\ttfamily Y}     & {\ttfamily\ttNum25} & \OutcutBDWidth\\
{\ttfamily Z}     & {\ttfamily\ttNum26} & \ReAlocationLength\\
\end{multicollongtblr}


%%%%%%%%%%%%%%%%%%%%%%%%%%%%%%%%%%%%%%%%%%%%%%%%%%%%%%%%%%
%% subsection 46.02.02 %%%%%%%%%%%%%%%%%%%%%%%%%%%%%%%%%%%
%%%%%%%%%%%%%%%%%%%%%%%%%%%%%%%%%%%%%%%%%%%%%%%%%%%%%%%%%%
\subsection{\KOutcutRLeft の取扱説明\TBW}
(to be written...)


%%%%%%%%%%%%%%%%%%%%%%%%%%%%%%%%%%%%%%%%%%%%%%%%%%%%%%%%%%
%% subsection 46.02.02 %%%%%%%%%%%%%%%%%%%%%%%%%%%%%%%%%%%
%%%%%%%%%%%%%%%%%%%%%%%%%%%%%%%%%%%%%%%%%%%%%%%%%%%%%%%%%%
\subsection{\KOutcutRLeft の注意事項\TBW}
(to be written...)



\clearpage
%%%%%%%%%%%%%%%%%%%%%%%%%%%%%%%%%%%%%%%%%%%%%%%%%%%%%%%%%%
%% section 46.2 %%%%%%%%%%%%%%%%%%%%%%%%%%%%%%%%%%%%%%%%%%
%%%%%%%%%%%%%%%%%%%%%%%%%%%%%%%%%%%%%%%%%%%%%%%%%%%%%%%%%%
\modHeadsection{\KCurvedOutcutRLeft :加工 \CurvedOutcut{} コーナーR 左回り1周}


%%%%%%%%%%%%%%%%%%%%%%%%%%%%%%%%%%%%%%%%%%%%%%%%%%%%%%%%%%
%% subsection 46.02.01 %%%%%%%%%%%%%%%%%%%%%%%%%%%%%%%%%%%
%%%%%%%%%%%%%%%%%%%%%%%%%%%%%%%%%%%%%%%%%%%%%%%%%%%%%%%%%%
\subsection{\KCurvedOutcutRLeftArguments}

\begin{multicollongtblr}{\KCurvedOutcutRLeftArguments}{ccX[l]}
引数 & 変数 & 内容\\
{\ttfamily K}     & {\ttfamily\ttNum06} & \OutcutLength\\
{\ttfamily M$^*$} & {\ttfamily\ttNum13} & \KeywayPos\\
{\ttfamily Q}     & {\ttfamily\ttNum17} & \CenterCurvatureRadius\\
{\ttfamily R}     & {\ttfamily\ttNum18} & \OutcutCornerR\\
{\ttfamily U}     & {\ttfamily\ttNum21} & \ACOD\\
{\ttfamily V}     & {\ttfamily\ttNum22} & \BDOD\\
{\ttfamily W$^*$} & {\ttfamily\ttNum23} & \KeywayWidth\\
{\ttfamily X}     & {\ttfamily\ttNum24} & \OutcutACWidth\\
{\ttfamily Y}     & {\ttfamily\ttNum25} & \OutcutBDWidth\\
{\ttfamily Z}     & {\ttfamily\ttNum26} & \ReAlocationLength\\
\end{multicollongtblr}


%%%%%%%%%%%%%%%%%%%%%%%%%%%%%%%%%%%%%%%%%%%%%%%%%%%%%%%%%%
%% subsection 46.02.02 %%%%%%%%%%%%%%%%%%%%%%%%%%%%%%%%%%%
%%%%%%%%%%%%%%%%%%%%%%%%%%%%%%%%%%%%%%%%%%%%%%%%%%%%%%%%%%
\subsection{\KCurvedOutcutRLeft の取扱説明\TBW}
(to be written...)


%%%%%%%%%%%%%%%%%%%%%%%%%%%%%%%%%%%%%%%%%%%%%%%%%%%%%%%%%%
%% subsection 46.02.02 %%%%%%%%%%%%%%%%%%%%%%%%%%%%%%%%%%%
%%%%%%%%%%%%%%%%%%%%%%%%%%%%%%%%%%%%%%%%%%%%%%%%%%%%%%%%%%
\subsection{\KCurvedOutcutRLeft の注意事項\TBW}
(to be written...)



\clearpage
%%%%%%%%%%%%%%%%%%%%%%%%%%%%%%%%%%%%%%%%%%%%%%%%%%%%%%%%%%
%% section 46.2 %%%%%%%%%%%%%%%%%%%%%%%%%%%%%%%%%%%%%%%%%%
%%%%%%%%%%%%%%%%%%%%%%%%%%%%%%%%%%%%%%%%%%%%%%%%%%%%%%%%%%
\modHeadsection{\KKeywayConerLeft :加工 \Keyway{} 左回り1周}


%%%%%%%%%%%%%%%%%%%%%%%%%%%%%%%%%%%%%%%%%%%%%%%%%%%%%%%%%%
%% subsection 46.02.01 %%%%%%%%%%%%%%%%%%%%%%%%%%%%%%%%%%%
%%%%%%%%%%%%%%%%%%%%%%%%%%%%%%%%%%%%%%%%%%%%%%%%%%%%%%%%%%
\subsection{\KKeywayConerLeftArguments}

\begin{multicollongtblr}{\KKeywayConerLeftArguments}{ccX[l]}
引数 & 変数 & 内容\\
{\ttfamily A} & {\ttfamily\ttNum01} & \TopOutcutExists\\
{\ttfamily B} & {\ttfamily\ttNum02} & \AKDToleranceExists\\
{\ttfamily C} & {\ttfamily\ttNum03} & \KeywayCornerC\\
{\ttfamily K} & {\ttfamily\ttNum06} & \KeywayCornerType\\
{\ttfamily H} & {\ttfamily\ttNum11} & \KeywayWidth\\
{\ttfamily M} & {\ttfamily\ttNum13} & \KeywayPos\\
{\ttfamily Q} & {\ttfamily\ttNum17} & \CenterCurvatureRadius\\
{\ttfamily R} & {\ttfamily\ttNum18} & \KeywayCornerR\\
{\ttfamily U} & {\ttfamily\ttNum21} & \ACOD または\OutcutACWidth\\
{\ttfamily V} & {\ttfamily\ttNum22} & \BDOD または\OutcutBDWidth\\
{\ttfamily W} & {\ttfamily\ttNum23} & \TopAlocationLength\\
{\ttfamily X} & {\ttfamily\ttNum24} & \KeywayACOD\\
{\ttfamily Y} & {\ttfamily\ttNum25} & \KeywayBDOD\\
{\ttfamily Z} & {\ttfamily\ttNum26} & \ReAlocationLength\\
\end{multicollongtblr}


%%%%%%%%%%%%%%%%%%%%%%%%%%%%%%%%%%%%%%%%%%%%%%%%%%%%%%%%%%
%% subsection 46.02.02 %%%%%%%%%%%%%%%%%%%%%%%%%%%%%%%%%%%
%%%%%%%%%%%%%%%%%%%%%%%%%%%%%%%%%%%%%%%%%%%%%%%%%%%%%%%%%%
\subsection{\KKeywayConerLeft の取扱説明\TBW}
(to be written...)


%%%%%%%%%%%%%%%%%%%%%%%%%%%%%%%%%%%%%%%%%%%%%%%%%%%%%%%%%%
%% subsection 46.02.02 %%%%%%%%%%%%%%%%%%%%%%%%%%%%%%%%%%%
%%%%%%%%%%%%%%%%%%%%%%%%%%%%%%%%%%%%%%%%%%%%%%%%%%%%%%%%%%
\subsection{\KKeywayConerLeft の注意事項\TBW}
(to be written...)



\clearpage
%%%%%%%%%%%%%%%%%%%%%%%%%%%%%%%%%%%%%%%%%%%%%%%%%%%%%%%%%%
%% section 46.2 %%%%%%%%%%%%%%%%%%%%%%%%%%%%%%%%%%%%%%%%%%
%%%%%%%%%%%%%%%%%%%%%%%%%%%%%%%%%%%%%%%%%%%%%%%%%%%%%%%%%%
\modHeadsection{\KEndFaceOutCChamferRLeft :加工 \EndFaceOutChamfer{} コーナーR 左回り1周}


%%%%%%%%%%%%%%%%%%%%%%%%%%%%%%%%%%%%%%%%%%%%%%%%%%%%%%%%%%
%% subsection 46.02.01 %%%%%%%%%%%%%%%%%%%%%%%%%%%%%%%%%%%
%%%%%%%%%%%%%%%%%%%%%%%%%%%%%%%%%%%%%%%%%%%%%%%%%%%%%%%%%%
\subsection{\KEndFaceOutCChamferRLeftArguments}

\begin{multicollongtblr}{\KEndFaceOutCChamferRLeftArguments}{ccX[l]}
引数 & 変数 & 内容\\
{\ttfamily A}     & {\ttfamily\ttNum01} & \OutcutExists\\
{\ttfamily B}     & {\ttfamily\ttNum02} & \ChamferType\\
{\ttfamily K}     & {\ttfamily\ttNum06} & \EndFaceChamferLength\\
{\ttfamily Q}     & {\ttfamily\ttNum17} & \CenterCurvatureRadius\\
{\ttfamily R}     & {\ttfamily\ttNum18} & \ODCornerR または\OutcutCornerR\\
{\ttfamily W}     & {\ttfamily\ttNum23} & \AlocationLength\\
{\ttfamily X$^*$} & {\ttfamily\ttNum24} & \OutcutACWidth\\
{\ttfamily Y$^*$} & {\ttfamily\ttNum25} & \OutcutBDWidth\\
{\ttfamily Z}     & {\ttfamily\ttNum26} & \TopReAlocationLength\\
\end{multicollongtblr}


%%%%%%%%%%%%%%%%%%%%%%%%%%%%%%%%%%%%%%%%%%%%%%%%%%%%%%%%%%
%% subsection 46.02.02 %%%%%%%%%%%%%%%%%%%%%%%%%%%%%%%%%%%
%%%%%%%%%%%%%%%%%%%%%%%%%%%%%%%%%%%%%%%%%%%%%%%%%%%%%%%%%%
\subsection{\KEndFaceOutCChamferRLeft の取扱説明\TBW}
(to be written...)


%%%%%%%%%%%%%%%%%%%%%%%%%%%%%%%%%%%%%%%%%%%%%%%%%%%%%%%%%%
%% subsection 46.02.02 %%%%%%%%%%%%%%%%%%%%%%%%%%%%%%%%%%%
%%%%%%%%%%%%%%%%%%%%%%%%%%%%%%%%%%%%%%%%%%%%%%%%%%%%%%%%%%
\subsection{\KEndFaceOutCChamferRLeft の注意事項\TBW}
(to be written...)



\clearpage
%%%%%%%%%%%%%%%%%%%%%%%%%%%%%%%%%%%%%%%%%%%%%%%%%%%%%%%%%%
%% section 46.2 %%%%%%%%%%%%%%%%%%%%%%%%%%%%%%%%%%%%%%%%%%
%%%%%%%%%%%%%%%%%%%%%%%%%%%%%%%%%%%%%%%%%%%%%%%%%%%%%%%%%%
\modHeadsection{\KEndFaceInCChamferRLeft :加工 \EndFaceInChamfer{} コーナーR 左回り1周}


%%%%%%%%%%%%%%%%%%%%%%%%%%%%%%%%%%%%%%%%%%%%%%%%%%%%%%%%%%
%% subsection 46.02.01 %%%%%%%%%%%%%%%%%%%%%%%%%%%%%%%%%%%
%%%%%%%%%%%%%%%%%%%%%%%%%%%%%%%%%%%%%%%%%%%%%%%%%%%%%%%%%%
\subsection{\KEndFaceInCChamferRLeftArguments}

\begin{multicollongtblr}{\KEndFaceInCChamferRLeftArguments}{ccX[l]}
引数 & 変数 & 内容\\
{\ttfamily K}     & {\ttfamily\ttNum06} & \EndFaceChamferLength\\
{\ttfamily M}     & {\ttfamily\ttNum13} & \PlatingThk\\
{\ttfamily Q}     & {\ttfamily\ttNum17} & \CenterCurvatureRadius\\
{\ttfamily R}     & {\ttfamily\ttNum18} & \IDCornerR または\IncutBoringCornerR\\
{\ttfamily W}     & {\ttfamily\ttNum23} & \AlocationLength\\
{\ttfamily X$^*$} & {\ttfamily\ttNum24} & \IncutBoringACWidth\\
{\ttfamily Y$^*$} & {\ttfamily\ttNum25} & \IncutBoringBDWidth\\
{\ttfamily Z}     & {\ttfamily\ttNum26} & \ReAlocationLength\\
\end{multicollongtblr}


%%%%%%%%%%%%%%%%%%%%%%%%%%%%%%%%%%%%%%%%%%%%%%%%%%%%%%%%%%
%% subsection 46.02.02 %%%%%%%%%%%%%%%%%%%%%%%%%%%%%%%%%%%
%%%%%%%%%%%%%%%%%%%%%%%%%%%%%%%%%%%%%%%%%%%%%%%%%%%%%%%%%%
\subsection{\KEndFaceInCChamferRLeft の取扱説明\TBW}
(to be written...)


%%%%%%%%%%%%%%%%%%%%%%%%%%%%%%%%%%%%%%%%%%%%%%%%%%%%%%%%%%
%% subsection 46.02.02 %%%%%%%%%%%%%%%%%%%%%%%%%%%%%%%%%%%
%%%%%%%%%%%%%%%%%%%%%%%%%%%%%%%%%%%%%%%%%%%%%%%%%%%%%%%%%%
\subsection{\KEndFaceInCChamferRLeft の注意事項\TBW}
(to be written...)



\clearpage
%%%%%%%%%%%%%%%%%%%%%%%%%%%%%%%%%%%%%%%%%%%%%%%%%%%%%%%%%%
%% section 46.2 %%%%%%%%%%%%%%%%%%%%%%%%%%%%%%%%%%%%%%%%%%
%%%%%%%%%%%%%%%%%%%%%%%%%%%%%%%%%%%%%%%%%%%%%%%%%%%%%%%%%%
\modHeadsection{\KEndFaceBoring :加工 \EndFaceBoring{} コーナーR 左回り1周}


%%%%%%%%%%%%%%%%%%%%%%%%%%%%%%%%%%%%%%%%%%%%%%%%%%%%%%%%%%
%% subsection 46.02.01 %%%%%%%%%%%%%%%%%%%%%%%%%%%%%%%%%%%
%%%%%%%%%%%%%%%%%%%%%%%%%%%%%%%%%%%%%%%%%%%%%%%%%%%%%%%%%%
\subsection{\KEndFaceBoringArguments}

\begin{multicollongtblr}{\KEndFaceBoringArguments}{ccX[l]}
引数 & 変数 & 内容\\
{\ttfamily I} & {\ttfamily\ttNum04} & \EndFaceBoringAsideDistance\\
{\ttfamily R} & {\ttfamily\ttNum18} & \EndFaceBoringCornerR\\
{\ttfamily U} & {\ttfamily\ttNum21} & \EndFaceBoringWidth\\
{\ttfamily V} & {\ttfamily\ttNum22} & \EndFaceBoringDepth\\
{\ttfamily W} & {\ttfamily\ttNum23} & \EndFaceBoringLength\\
{\ttfamily Z} & {\ttfamily\ttNum26} & \TopReAlocationLength\\
\end{multicollongtblr}


%%%%%%%%%%%%%%%%%%%%%%%%%%%%%%%%%%%%%%%%%%%%%%%%%%%%%%%%%%
%% subsection 46.02.02 %%%%%%%%%%%%%%%%%%%%%%%%%%%%%%%%%%%
%%%%%%%%%%%%%%%%%%%%%%%%%%%%%%%%%%%%%%%%%%%%%%%%%%%%%%%%%%
\subsection{\KEndFaceBoring の取扱説明\TBW}
(to be written...)


%%%%%%%%%%%%%%%%%%%%%%%%%%%%%%%%%%%%%%%%%%%%%%%%%%%%%%%%%%
%% subsection 46.02.02 %%%%%%%%%%%%%%%%%%%%%%%%%%%%%%%%%%%
%%%%%%%%%%%%%%%%%%%%%%%%%%%%%%%%%%%%%%%%%%%%%%%%%%%%%%%%%%
\subsection{\KEndFaceBoring の注意事項\TBW}
(to be written...)



\clearpage
%%%%%%%%%%%%%%%%%%%%%%%%%%%%%%%%%%%%%%%%%%%%%%%%%%%%%%%%%%
%% section 46.2 %%%%%%%%%%%%%%%%%%%%%%%%%%%%%%%%%%%%%%%%%%
%%%%%%%%%%%%%%%%%%%%%%%%%%%%%%%%%%%%%%%%%%%%%%%%%%%%%%%%%%
\modHeadsection{\KIncutBoring :加工 \IncutBoring{} コーナーR 左回り1周}


%%%%%%%%%%%%%%%%%%%%%%%%%%%%%%%%%%%%%%%%%%%%%%%%%%%%%%%%%%
%% subsection 46.02.01 %%%%%%%%%%%%%%%%%%%%%%%%%%%%%%%%%%%
%%%%%%%%%%%%%%%%%%%%%%%%%%%%%%%%%%%%%%%%%%%%%%%%%%%%%%%%%%
\subsection{\KIncutBoringArguments}

\begin{multicollongtblr}{\KIncutBoringArguments}{ccX[l]}
引数 & 変数 & 内容\\
{\ttfamily K} & {\ttfamily\ttNum06} & \IncutBoringLength\\
{\ttfamily R} & {\ttfamily\ttNum18} & \IncutBoringCornerR\\
{\ttfamily X} & {\ttfamily\ttNum24} & \IncutBoringACWidth\\
{\ttfamily Y} & {\ttfamily\ttNum25} & \IncutBoringBDWidth\\
{\ttfamily Z} & {\ttfamily\ttNum26} & \TopReAlocationLength\\
\end{multicollongtblr}


%%%%%%%%%%%%%%%%%%%%%%%%%%%%%%%%%%%%%%%%%%%%%%%%%%%%%%%%%%
%% subsection 46.02.02 %%%%%%%%%%%%%%%%%%%%%%%%%%%%%%%%%%%
%%%%%%%%%%%%%%%%%%%%%%%%%%%%%%%%%%%%%%%%%%%%%%%%%%%%%%%%%%
\subsection{\KIncutBoring の取扱説明\TBW}
(to be written...)


%%%%%%%%%%%%%%%%%%%%%%%%%%%%%%%%%%%%%%%%%%%%%%%%%%%%%%%%%%
%% subsection 46.02.02 %%%%%%%%%%%%%%%%%%%%%%%%%%%%%%%%%%%
%%%%%%%%%%%%%%%%%%%%%%%%%%%%%%%%%%%%%%%%%%%%%%%%%%%%%%%%%%
\subsection{\KIncutBoring の注意事項\TBW}
(to be written...)



\clearpage
%%%%%%%%%%%%%%%%%%%%%%%%%%%%%%%%%%%%%%%%%%%%%%%%%%%%%%%%%%
%% section 46.2 %%%%%%%%%%%%%%%%%%%%%%%%%%%%%%%%%%%%%%%%%%
%%%%%%%%%%%%%%%%%%%%%%%%%%%%%%%%%%%%%%%%%%%%%%%%%%%%%%%%%%
\modHeadsection{\DLone :測定・加工 \Dimple}


%%%%%%%%%%%%%%%%%%%%%%%%%%%%%%%%%%%%%%%%%%%%%%%%%%%%%%%%%%
%% subsection 46.02.01 %%%%%%%%%%%%%%%%%%%%%%%%%%%%%%%%%%%
%%%%%%%%%%%%%%%%%%%%%%%%%%%%%%%%%%%%%%%%%%%%%%%%%%%%%%%%%%
\subsection{\DLoneArguments}

\begin{multicollongtblr}{\KKeywayConerLeftArguments}{ccX[l]}
引数 & 変数 & 内容\\
{\ttfamily A} & {\ttfamily\ttNum01} & \AlocationAngle\\
{\ttfamily B} & {\ttfamily\ttNum02} & \DimpleAngle\\
{\ttfamily I} & {\ttfamily\ttNum04} & \DimpleHorizontalPitch\\
{\ttfamily K} & {\ttfamily\ttNum06} & \DimpleVerticalPitch\\
{\ttfamily F} & {\ttfamily\ttNum09} & \DimpleOddRowLength\\
{\ttfamily M} & {\ttfamily\ttNum13} & \DimpleRowNum\\
{\ttfamily Q} & {\ttfamily\ttNum17} & \DistanceTopEndFaceDimpleFirstRow\\
{\ttfamily R} & {\ttfamily\ttNum18} & \CenterCurvatureRadius\\
{\ttfamily S} & {\ttfamily\ttNum19} & \DimpleEvenRowLength\\
{\ttfamily T} & {\ttfamily\ttNum20} & \PlatingThk\\
{\ttfamily U} & {\ttfamily\ttNum21} & \DimpleDepth\\
{\ttfamily W} & {\ttfamily\ttNum23} & \TopAlocationLength\\
{\ttfamily Z} & {\ttfamily\ttNum26} & \TopReAlocationLength\\
\end{multicollongtblr}


%%%%%%%%%%%%%%%%%%%%%%%%%%%%%%%%%%%%%%%%%%%%%%%%%%%%%%%%%%
%% subsection 46.02.02 %%%%%%%%%%%%%%%%%%%%%%%%%%%%%%%%%%%
%%%%%%%%%%%%%%%%%%%%%%%%%%%%%%%%%%%%%%%%%%%%%%%%%%%%%%%%%%
\subsection{\DLone の取扱説明\TBW}
(to be written...)


%%%%%%%%%%%%%%%%%%%%%%%%%%%%%%%%%%%%%%%%%%%%%%%%%%%%%%%%%%
%% subsection 46.02.02 %%%%%%%%%%%%%%%%%%%%%%%%%%%%%%%%%%%
%%%%%%%%%%%%%%%%%%%%%%%%%%%%%%%%%%%%%%%%%%%%%%%%%%%%%%%%%%
\subsection{\DLone の注意事項\TBW}
(to be written...)



\clearpage
%%%%%%%%%%%%%%%%%%%%%%%%%%%%%%%%%%%%%%%%%%%%%%%%%%%%%%%%%%
%% section 46.2 %%%%%%%%%%%%%%%%%%%%%%%%%%%%%%%%%%%%%%%%%%
%%%%%%%%%%%%%%%%%%%%%%%%%%%%%%%%%%%%%%%%%%%%%%%%%%%%%%%%%%
\modHeadsection{\Owarmingup :\index{だんきうんてん@暖機運転}暖機運転}


%%%%%%%%%%%%%%%%%%%%%%%%%%%%%%%%%%%%%%%%%%%%%%%%%%%%%%%%%%
%% subsection 46.02.01 %%%%%%%%%%%%%%%%%%%%%%%%%%%%%%%%%%%
%%%%%%%%%%%%%%%%%%%%%%%%%%%%%%%%%%%%%%%%%%%%%%%%%%%%%%%%%%
\subsection{\OwarmingupArguments}
引数なし


%%%%%%%%%%%%%%%%%%%%%%%%%%%%%%%%%%%%%%%%%%%%%%%%%%%%%%%%%%
%% subsection 46.02.02 %%%%%%%%%%%%%%%%%%%%%%%%%%%%%%%%%%%
%%%%%%%%%%%%%%%%%%%%%%%%%%%%%%%%%%%%%%%%%%%%%%%%%%%%%%%%%%
\subsection{\Owarmingup の取扱説明\TBW}
(to be written...)


%%%%%%%%%%%%%%%%%%%%%%%%%%%%%%%%%%%%%%%%%%%%%%%%%%%%%%%%%%
%% subsection 46.02.02 %%%%%%%%%%%%%%%%%%%%%%%%%%%%%%%%%%%
%%%%%%%%%%%%%%%%%%%%%%%%%%%%%%%%%%%%%%%%%%%%%%%%%%%%%%%%%%
\subsection{\Owarmingup の注意事項\TBW}
(to be written...)



\clearpage
%%%%%%%%%%%%%%%%%%%%%%%%%%%%%%%%%%%%%%%%%%%%%%%%%%%%%%%%%%
%% section 46.2 %%%%%%%%%%%%%%%%%%%%%%%%%%%%%%%%%%%%%%%%%%
%%%%%%%%%%%%%%%%%%%%%%%%%%%%%%%%%%%%%%%%%%%%%%%%%%%%%%%%%%
\modHeadsection{\OtoolLength :\TLCorrection}


%%%%%%%%%%%%%%%%%%%%%%%%%%%%%%%%%%%%%%%%%%%%%%%%%%%%%%%%%%
%% subsection 46.02.01 %%%%%%%%%%%%%%%%%%%%%%%%%%%%%%%%%%%
%%%%%%%%%%%%%%%%%%%%%%%%%%%%%%%%%%%%%%%%%%%%%%%%%%%%%%%%%%
\subsection{\OtoolLengthArguments}

\begin{multicollongtblr}{\KIncutBoringArguments}{ccX[l]}
引数 & 変数 & 内容\\
{\ttfamily I$^*$} & {\ttfamily\ttNum04} & 刃先$X$方向のシフト量\\
{\ttfamily J$^*$} & {\ttfamily\ttNum05} & 刃先$Y$方向のシフト量\\
{\ttfamily H}     & {\ttfamily\ttNum11} & 工具番号\\
\end{multicollongtblr}


%%%%%%%%%%%%%%%%%%%%%%%%%%%%%%%%%%%%%%%%%%%%%%%%%%%%%%%%%%
%% subsection 46.02.02 %%%%%%%%%%%%%%%%%%%%%%%%%%%%%%%%%%%
%%%%%%%%%%%%%%%%%%%%%%%%%%%%%%%%%%%%%%%%%%%%%%%%%%%%%%%%%%
\subsection{\OtoolLength の取扱説明\TBW}
(to be written...)


%%%%%%%%%%%%%%%%%%%%%%%%%%%%%%%%%%%%%%%%%%%%%%%%%%%%%%%%%%
%% subsection 46.02.02 %%%%%%%%%%%%%%%%%%%%%%%%%%%%%%%%%%%
%%%%%%%%%%%%%%%%%%%%%%%%%%%%%%%%%%%%%%%%%%%%%%%%%%%%%%%%%%
\subsection{\OtoolLengthA の注意事項\TBW}
(to be written...)


%%%%%%%%%%%%%%%%%%%%%%%%%%%%%%%%%%%%%%%%%%%%%%%%%%%%%%%%%
%% Appendicies %%%%%%%%%%%%%%%%%%%%%%%%%%%%%%%%%%%%%%%%%%
%%%%%%%%%%%%%%%%%%%%%%%%%%%%%%%%%%%%%%%%%%%%%%%%%%%%%%%%%
\begin{appendices}
\Appendixpart
%!TEX root = ../RPA_for_Creating_Program_Note.tex
\setcounter{lstlisting}{0}


\modHeadchapter[lot,lol]{作成したG-codeプログラム\label{chap:createdGcodeDM}}



%%%%%%%%%%%%%%%%%%%%%%%%%%%%%%%%%%%%%%%%%%%%%%%%%%%%%%%%%%
%% section K.1 %%%%%%%%%%%%%%%%%%%%%%%%%%%%%%%%%%%%%%%%%%%
%%%%%%%%%%%%%%%%%%%%%%%%%%%%%%%%%%%%%%%%%%%%%%%%%%%%%%%%%%
\modHeadsection{作成したプログラム 一覧}


%%%%%%%%%%%%%%%%%%%%%%%%%%%%%%%%%%%%%%%%%%%%%%%%%%%%%%%%%%
%% subsection K.1.1 %%%%%%%%%%%%%%%%%%%%%%%%%%%%%%%%%%%%%%
%%%%%%%%%%%%%%%%%%%%%%%%%%%%%%%%%%%%%%%%%%%%%%%%%%%%%%%%%%
\subsection{メインプログラムの例 一覧}
\DMC において作成した\index{メインプログラム}メインプログラムは以下のとおりである。
%%%%%%%%%%%%%%%%%%%%%%%%%%%%%%%%%%%%%%%%%%%%%%%%%%%%%%%%%%
%% marker %%%%%%%%%%%%%%%%%%%%%%%%%%%%%%%%%%%%%%%%%%%%%%%%
%%%%%%%%%%%%%%%%%%%%%%%%%%%%%%%%%%%%%%%%%%%%%%%%%%%%%%%%%%
\begin{marker}
ここでいうメインプログラムとは、\index{ずめんばんごう@図面番号}図面場号と同一の\index{プログラムばんごう@プログラム番号}プログラム番号のものを指す。
ただし、ここでは例として挙げているので、\pageautoref{subsec:notopenwork}に伴い、プログラム番号は本来のものから変更している。
\end{marker}
%%%%%%%%%%%%%%%%%%%%%%%%%%%%%%%%%%%%%%%%%%%%%%%%%%%%%%%%%%
%%%%%%%%%%%%%%%%%%%%%%%%%%%%%%%%%%%%%%%%%%%%%%%%%%%%%%%%%%
%%%%%%%%%%%%%%%%%%%%%%%%%%%%%%%%%%%%%%%%%%%%%%%%%%%%%%%%%%

\begin{multicollongtblr}{作成したプログラム一覧:メインプログラムの例}{clX[l]}
{\ttfamily O}番号 & \SetCell[c=2]{l}使用prg\\
\SetCell[r=10]{c}
\MainExOne & 外側芯出し・幅 & \MYOThickness\MXIface\\
           & 内側芯出し・幅 & \MXIWidth\MYIWidth\\
           & 通り芯 & \MXcenterline\MYcenterline\\
           & \Dimple & \DLone\\
           & 端面 & \KTanmenRight\\
           & 外削 & \KGaisakuRLeft\\
           & \nameKeyway & \KMizoConerLeft\\
           & 外C面取 & \KSotoMentoriRLeft\\
           & 内C面取 & \KUchiMentoriRLeft\\
           & その他 & \OpauseCheck\OsensorOn\OsensorOff\\
\hline
\SetCell[r=8]{c}
\MainExTwo & 外側芯出し・幅 & \MXOThickness\MYOThickness\MXOface\\
           & 内側芯出し・幅 & \MXIWidth\MYIWidth\\
           & \Dimple & \DLone\\
           & 端面 & \KTanmenRight\\
           & \nameKeyway & \KMizoConerLeft\\
           & 外C面取 & \KSotoMentoriRLeft\\
           & 内C面取 & \KUchiMentoriRLeft\\
           & その他 & \OpauseCheck\OsensorOn\OsensorOff\\
\end{multicollongtblr}


\clearpage
%%%%%%%%%%%%%%%%%%%%%%%%%%%%%%%%%%%%%%%%%%%%%%%%%%%%%%%%%%
%% subsection K.1.2 %%%%%%%%%%%%%%%%%%%%%%%%%%%%%%%%%%%%%%
%%%%%%%%%%%%%%%%%%%%%%%%%%%%%%%%%%%%%%%%%%%%%%%%%%%%%%%%%%
\subsection{サブプログラム 一覧}
\DMC において作成した\index{サブプログラム}サブプログラムは以下のとおりである
%%%%%%%%%%%%%%%%%%%%%%%%%%%%%%%%%%%%%%%%%%%%%%%%%%%%%%%%%%
%% marker %%%%%%%%%%%%%%%%%%%%%%%%%%%%%%%%%%%%%%%%%%%%%%%%
%%%%%%%%%%%%%%%%%%%%%%%%%%%%%%%%%%%%%%%%%%%%%%%%%%%%%%%%%%
\begin{marker}
ここでいうサブプログラムとは、\index{せいひんばんごう@製品番号}製品場号と同一の\index{プログラムばんごう@プログラム番号}プログラム番号以外のものを指す。
\end{marker}
%%%%%%%%%%%%%%%%%%%%%%%%%%%%%%%%%%%%%%%%%%%%%%%%%%%%%%%%%%
%%%%%%%%%%%%%%%%%%%%%%%%%%%%%%%%%%%%%%%%%%%%%%%%%%%%%%%%%%
%%%%%%%%%%%%%%%%%%%%%%%%%%%%%%%%%%%%%%%%%%%%%%%%%%%%%%%%%%

\begin{multicollongtblr}{作成したプログラム一覧:芯出し・幅測定}{cX[l]l}
{\ttfamily O}番号 & 内容 & 使用prg\\
\MXOThickness & 測定 両側 外側中心・幅$X$ & \OsensorOff\\
\MYOThickness & 測定 両側 外側中心・幅$Y$ & \OsensorOff\\
\MXOface      & 測定 片側 溝中心$X$(トップA側外面測定) & \OsensorOff\\
\MXIWidth     & 測定 両側 内側中心・幅$X$ & \OsensorOff\\
\MYIWidth     & 測定 両側 内側中心・幅$Y$ & \OsensorOff\\
\MXIface      & 測定 片側 外削中心$X$(C側内面方向測定) & \OsensorOff\\
\end{multicollongtblr}

%\clearpage
\begin{multicollongtblr}{作成したプログラム一覧:通り芯測定}{cX[l]l}
{\ttfamily O}番号 & 内容 & 使用prg\\
\MYcenterline & 測定 片側 通り芯$Y$(B側外削面測定) & \OsensorOff\\
\MXcenterline & 測定 片側 通り芯$X$(C側外削面 $Z$方向測定) & \OsensorOff\\
\end{multicollongtblr}

\begin{multicollongtblr}{作成したプログラム一覧:\Dimple}{cX[l]l}
{\ttfamily O}番号 & 内容 & 使用prg\\
\DLone      & \Dimple :移動 各列の中心上 & \DLtwoAC\DLtwoBD\\
\DLtwoAC    & \Dimple :移動 AC面 列内の各\Dimple 上 & \DMLthreeAC\DKLthreeAC\\
\DLtwoBD    & \Dimple :移動 BC面 列内の各\Dimple 上 & \DMLthreeBD\DKLthreeBD\\
\DMLthreeAC & \Dimple :測定 AC内表面$X$ & \OsensorOff\\
\DMLthreeBD & \Dimple :測定 BD内表面$Y$ & \OsensorOff\\
\DKLthreeAC & \Dimple :加工 AC内表面$X$ & \\
\DKLthreeBD & \Dimple :加工 BD内表面$Y$ & \\
\end{multicollongtblr}

\clearpage
\begin{multicollongtblr}{作成したプログラム一覧:加工(\Dimple 以外)}{cX[l]l}
{\ttfamily O}番号 & 内容 & 使用prg\\
\KTanmenRight      & 加工 端面 コーナーR 右回り1周 & \KOLeftAR\\
\KGaisakuRLeft     & 加工 外削 コーナーR 左回り1周 & \KOLeftAR\OpauseCheck\\
\KMizoConerLeft    & 加工 溝 左回り1周 & \KOLeftAR\OpauseCheck\\
\KSotoMentoriRLeft & 加工 外面取 コーナーR 左回り1周 & \KOLeftAR\OpauseCheck\\
\KUchiMentoriRLeft & 加工 内面取 コーナーR 左回り1周 & \KILeftAC\OpauseCheck\\
\KOLeftAR   & 外側 左回り1周 右上始まり & \\
\KILeftAC   & 内側 左回り1周 中央上始まり & \\
\end{multicollongtblr}

\begin{multicollongtblr}{作成したプログラム一覧:その他}{cX[l]l}
{\ttfamily O}番号 & 内容 & 使用prg\\
\OpauseCheck  & 移動・加工後確認用:90$^\circ$回転 扉前一時停止 & \\
\OsensorOn    & タッチセンサー電源ON & \\
\OsensorOff   & タッチセンサー電源OFF &\\
\OwarmingupA  & 暖機運転 & \Owarmingup\\
\Owarmingup   & 暖機運転用サブプログラム & \\
\OtoolLengthA & 登録工具 工具長自動測定 & \OtoolLength\\
\OtoolLength  & 工具長 自動測定用サブプログラム &\\
\end{multicollongtblr}



\clearrightpage
%%%%%%%%%%%%%%%%%%%%%%%%%%%%%%%%%%%%%%%%%%%%%%%%%%%%%%%%%%
%% section G.1 %%%%%%%%%%%%%%%%%%%%%%%%%%%%%%%%%%%%%%%%%%%
%%%%%%%%%%%%%%%%%%%%%%%%%%%%%%%%%%%%%%%%%%%%%%%%%%%%%%%%%%
\modHeadsection{メインプログラムの例}
\index{メインプログラム}メインプログラムについては、個々の\index{めいさい(モールド)@明細(モールド)}明細の情報(社内機密情報)を含む。
そのため、\pageautoref{subsec:notopenwork}に則り、記載は代表的・典型的なものに留める。\\

%%%%%%%%%%%%%%%%%%%%%%%%%%%%%%%%%%%%%%%%%%%%%%%%%%%%%%%%%%
%% Prg. \MainExOne %%%%%%%%%%%%%%%%%%%%%%%%%%%%%%%%%%%%%%%
%%%%%%%%%%%%%%%%%%%%%%%%%%%%%%%%%%%%%%%%%%%%%%%%%%%%%%%%%%
\modcaptionof{lstlisting}{\MainExOne:メインプログラムの例1}
\lstinputlisting[style=Gcode-more]{../Mould_Machining_Programs/main_program_examples/\nameMainExOne}


\clearrightpage
%%%%%%%%%%%%%%%%%%%%%%%%%%%%%%%%%%%%%%%%%%%%%%%%%%%%%%%%%%
%% Prg. \MainExTwo %%%%%%%%%%%%%%%%%%%%%%%%%%%%%%%%%%%%%%%
%%%%%%%%%%%%%%%%%%%%%%%%%%%%%%%%%%%%%%%%%%%%%%%%%%%%%%%%%%
\modcaptionof{lstlisting}{\MainExTwo:メインプログラムの例2}
\lstinputlisting[style=Gcode-more]{../Mould_Machining_Programs/main_program_examples/\nameMainExTwo}



\clearrightpage
%%%%%%%%%%%%%%%%%%%%%%%%%%%%%%%%%%%%%%%%%%%%%%%%%%%%%%%%%%
%% section E.2 %%%%%%%%%%%%%%%%%%%%%%%%%%%%%%%%%%%%%%%%%%%
%%%%%%%%%%%%%%%%%%%%%%%%%%%%%%%%%%%%%%%%%%%%%%%%%%%%%%%%%%
\modHeadsection{測定用(\Dimple 除く)サブプログラム}


%%%%%%%%%%%%%%%%%%%%%%%%%%%%%%%%%%%%%%%%%%%%%%%%%%%%%%%%%%
%% subsection E.2.1 %%%%%%%%%%%%%%%%%%%%%%%%%%%%%%%%%%%%%%
%%%%%%%%%%%%%%%%%%%%%%%%%%%%%%%%%%%%%%%%%%%%%%%%%%%%%%%%%%
\subsection{芯出し測定用サブプログラム}

%%%%%%%%%%%%%%%%%%%%%%%%%%%%%%%%%%%%%%%%%%%%%%%%%%%%%%%%%%
%% Prg. \MXOThickness %%%%%%%%%%%%%%%%%%%%%%%%%%%%%%%%%%%%
%%%%%%%%%%%%%%%%%%%%%%%%%%%%%%%%%%%%%%%%%%%%%%%%%%%%%%%%%%
\modcaptionof{lstlisting}{\MXOThickness\,:測定 両側 外側中心\texorpdfstring{$X$}{X}}
\lstinputlisting[style=Gcode-more]{../Mould_Machining_Programs/sub_programs/\nameMXOThickness.nc}

\clearrightpage
%%%%%%%%%%%%%%%%%%%%%%%%%%%%%%%%%%%%%%%%%%%%%%%%%%%%%%%%%%
%% Prg. \MYOThickness %%%%%%%%%%%%%%%%%%%%%%%%%%%%%%%%%%%%
%%%%%%%%%%%%%%%%%%%%%%%%%%%%%%%%%%%%%%%%%%%%%%%%%%%%%%%%%%
\modcaptionof{lstlisting}{\MYOThickness\,:測定 両側 外側中心\texorpdfstring{$Y$}{Y}}
\lstinputlisting[style=Gcode-more]{../Mould_Machining_Programs/sub_programs/\nameMYOThickness.nc}

\clearrightpage
%%%%%%%%%%%%%%%%%%%%%%%%%%%%%%%%%%%%%%%%%%%%%%%%%%%%%%%%%%
%% Prg. \MYOThickness %%%%%%%%%%%%%%%%%%%%%%%%%%%%%%%%%%%%
%%%%%%%%%%%%%%%%%%%%%%%%%%%%%%%%%%%%%%%%%%%%%%%%%%%%%%%%%%
\modcaptionof{lstlisting}{\MXOface\,:測定 片側 溝中心\texorpdfstring{$X$}{X}(A側外面側測定)}
\lstinputlisting[style=Gcode-more]{../Mould_Machining_Programs/sub_programs/\nameMXOface.nc}

\clearrightpage
%%%%%%%%%%%%%%%%%%%%%%%%%%%%%%%%%%%%%%%%%%%%%%%%%%%%%%%%%%
%% Prg. \MXIWidth %%%%%%%%%%%%%%%%%%%%%%%%%%%%%%%%%%%%%%%%
%%%%%%%%%%%%%%%%%%%%%%%%%%%%%%%%%%%%%%%%%%%%%%%%%%%%%%%%%%
\modcaptionof{lstlisting}{\MXIWidth\,:測定 両側 内側中心\texorpdfstring{$X$}{X}}
\lstinputlisting[style=Gcode-more]{../Mould_Machining_Programs/sub_programs/\nameMXIWidth.nc}

\clearrightpage
%%%%%%%%%%%%%%%%%%%%%%%%%%%%%%%%%%%%%%%%%%%%%%%%%%%%%%%%%%
%% Prg. \MYIWidth %%%%%%%%%%%%%%%%%%%%%%%%%%%%%%%%%%%%%%%%
%%%%%%%%%%%%%%%%%%%%%%%%%%%%%%%%%%%%%%%%%%%%%%%%%%%%%%%%%%
\modcaptionof{lstlisting}{\MYIWidth\,:測定 両側 内側中心\texorpdfstring{$Y$}{Y}}
\lstinputlisting[style=Gcode-more]{../Mould_Machining_Programs/sub_programs/\nameMYIWidth.nc}

\clearrightpage
%%%%%%%%%%%%%%%%%%%%%%%%%%%%%%%%%%%%%%%%%%%%%%%%%%%%%%%%%%
%% Prg. \MXface %%%%%%%%%%%%%%%%%%%%%%%%%%%%%%%%%%%%%%%%%%
%%%%%%%%%%%%%%%%%%%%%%%%%%%%%%%%%%%%%%%%%%%%%%%%%%%%%%%%%%
\modcaptionof{lstlisting}{\MXIface\,:測定 片側 外削中心\texorpdfstring{$X$}{X}(C側内面測定)}
\lstinputlisting[style=Gcode-more]{../Mould_Machining_Programs/sub_programs/\nameMXIface.nc}


\clearrightpage
%%%%%%%%%%%%%%%%%%%%%%%%%%%%%%%%%%%%%%%%%%%%%%%%%%%%%%%%%%
%% subsection E.2.1 %%%%%%%%%%%%%%%%%%%%%%%%%%%%%%%%%%%%%%
%%%%%%%%%%%%%%%%%%%%%%%%%%%%%%%%%%%%%%%%%%%%%%%%%%%%%%%%%%
\subsection{通り芯測定用サブプログラム}

%%%%%%%%%%%%%%%%%%%%%%%%%%%%%%%%%%%%%%%%%%%%%%%%%%%%%%%%%%
%% Prg. \MYcenterline %%%%%%%%%%%%%%%%%%%%%%%%%%%%%%%%%%%%
%%%%%%%%%%%%%%%%%%%%%%%%%%%%%%%%%%%%%%%%%%%%%%%%%%%%%%%%%%
\modcaptionof{lstlisting}{\MYcenterline\,:測定 片側 通り芯\texorpdfstring{$Y$}{Y}(B側外削面測定)}
\lstinputlisting[style=Gcode-more]{../Mould_Machining_Programs/sub_programs/\nameMYcenterline.nc}


\clearrightpage
%%%%%%%%%%%%%%%%%%%%%%%%%%%%%%%%%%%%%%%%%%%%%%%%%%%%%%%%%%
%% Prg. \MXcenterline  %%%%%%%%%%%%%%%%%%%%%%%%%%%%%%%%%%%
%%%%%%%%%%%%%%%%%%%%%%%%%%%%%%%%%%%%%%%%%%%%%%%%%%%%%%%%%%
\modcaptionof{lstlisting}{\MXcenterline\,:測定 片側 通り芯\texorpdfstring{$X$}{X}(C側外削面$Z$測定)}
\lstinputlisting[style=Gcode-more]{../Mould_Machining_Programs/sub_programs/\nameMXcenterline.nc}


\clearrightpage
%%%%%%%%%%%%%%%%%%%%%%%%%%%%%%%%%%%%%%%%%%%%%%%%%%%%%%%%%%
%% section E.3 %%%%%%%%%%%%%%%%%%%%%%%%%%%%%%%%%%%%%%%%%%%
%%%%%%%%%%%%%%%%%%%%%%%%%%%%%%%%%%%%%%%%%%%%%%%%%%%%%%%%%%
\modHeadsection{加工用(\Dimple 除く)サブプログラム}


%%%%%%%%%%%%%%%%%%%%%%%%%%%%%%%%%%%%%%%%%%%%%%%%%%%%%%%%%%
%% Prg. \KTanmenRight %%%%%%%%%%%%%%%%%%%%%%%%%%%%%%%%%%%%
%%%%%%%%%%%%%%%%%%%%%%%%%%%%%%%%%%%%%%%%%%%%%%%%%%%%%%%%%%
\modcaptionof{lstlisting}{\KTanmenRight\,:加工 端面 コーナーR 右回り1周}
\lstinputlisting[style=Gcode-more]{../Mould_Machining_Programs/sub_programs/\nameKTanmenRight.nc}


\clearrightpage
%%%%%%%%%%%%%%%%%%%%%%%%%%%%%%%%%%%%%%%%%%%%%%%%%%%%%%%%%%
%% Prg. \KGaisakuRLeft %%%%%%%%%%%%%%%%%%%%%%%%%%%%%%%%%%%
%%%%%%%%%%%%%%%%%%%%%%%%%%%%%%%%%%%%%%%%%%%%%%%%%%%%%%%%%%
\modcaptionof{lstlisting}{\KGaisakuRLeft\,:加工 外削 コーナーR 左回り1周}
\lstinputlisting[style=Gcode-more]{../Mould_Machining_Programs/sub_programs/\nameKGaisakuRLeft.nc}


\clearrightpage
%%%%%%%%%%%%%%%%%%%%%%%%%%%%%%%%%%%%%%%%%%%%%%%%%%%%%%%%%%
%% Prg. \KMizoConerLeft %%%%%%%%%%%%%%%%%%%%%%%%%%%%%%%%%%
%%%%%%%%%%%%%%%%%%%%%%%%%%%%%%%%%%%%%%%%%%%%%%%%%%%%%%%%%%
\modcaptionof{lstlisting}{\KMizoConerLeft\,:加工 溝 左回り1周}
\lstinputlisting[style=Gcode-more]{../Mould_Machining_Programs/sub_programs/\nameKMizoConerLeft.nc}


\clearrightpage
%%%%%%%%%%%%%%%%%%%%%%%%%%%%%%%%%%%%%%%%%%%%%%%%%%%%%%%%%%
%% Prg. \KSotoMentoriRLeft %%%%%%%%%%%%%%%%%%%%%%%%%%%%%%%
%%%%%%%%%%%%%%%%%%%%%%%%%%%%%%%%%%%%%%%%%%%%%%%%%%%%%%%%%%
\modcaptionof{lstlisting}{\KSotoMentoriRLeft\,:加工 外面取 コーナーR 左回り1周}
\lstinputlisting[style=Gcode-more]{../Mould_Machining_Programs/sub_programs/\nameKSotoMentoriRLeft.nc}


\clearrightpage
%%%%%%%%%%%%%%%%%%%%%%%%%%%%%%%%%%%%%%%%%%%%%%%%%%%%%%%%%%
%% Prg. \KUchiMentoriRLeft %%%%%%%%%%%%%%%%%%%%%%%%%%%%%%%
%%%%%%%%%%%%%%%%%%%%%%%%%%%%%%%%%%%%%%%%%%%%%%%%%%%%%%%%%%
\modcaptionof{lstlisting}{\KUchiMentoriRLeft\,:加工 内面取 コーナーR 左回り1周}
\lstinputlisting[style=Gcode-more]{../Mould_Machining_Programs/sub_programs/\nameKUchiMentoriRLeft.nc}


\clearrightpage
%%%%%%%%%%%%%%%%%%%%%%%%%%%%%%%%%%%%%%%%%%%%%%%%%%%%%%%%%%
%% Prg. \KUchiMentoriRLeft %%%%%%%%%%%%%%%%%%%%%%%%%%%%%%%
%%%%%%%%%%%%%%%%%%%%%%%%%%%%%%%%%%%%%%%%%%%%%%%%%%%%%%%%%%
\modcaptionof{lstlisting}{\KOLeftAR\,:外側 左回り1周 右上始まり}
\lstinputlisting[style=Gcode-more]{../Mould_Machining_Programs/sub_programs/\nameKOLeftAR.nc}


\clearrightpage
%%%%%%%%%%%%%%%%%%%%%%%%%%%%%%%%%%%%%%%%%%%%%%%%%%%%%%%%%%
%% Prg. \KUchiMentoriRLeft %%%%%%%%%%%%%%%%%%%%%%%%%%%%%%%
%%%%%%%%%%%%%%%%%%%%%%%%%%%%%%%%%%%%%%%%%%%%%%%%%%%%%%%%%%
\modcaptionof{lstlisting}{\KILeftAC\,:内側 左回り1周 中央上始まり}
\lstinputlisting[style=Gcode-more]{../Mould_Machining_Programs/sub_programs/\nameKILeftAC.nc}


\clearrightpage
%%%%%%%%%%%%%%%%%%%%%%%%%%%%%%%%%%%%%%%%%%%%%%%%%%%%%%%%%%
%% section E.4 %%%%%%%%%%%%%%%%%%%%%%%%%%%%%%%%%%%%%%%%%%%
%%%%%%%%%%%%%%%%%%%%%%%%%%%%%%%%%%%%%%%%%%%%%%%%%%%%%%%%%%
\modHeadsection{\Dimple 用 移動・測定・加工用サブプログラム}


%%%%%%%%%%%%%%%%%%%%%%%%%%%%%%%%%%%%%%%%%%%%%%%%%%%%%%%%%%
%% Prg. \DLone %%%%%%%%%%%%%%%%%%%%%%%%%%%%%%%%%%%%%%%%%%%
%%%%%%%%%%%%%%%%%%%%%%%%%%%%%%%%%%%%%%%%%%%%%%%%%%%%%%%%%%
\modcaptionof{lstlisting}{\DLone\,:\Dimple :移動 各列の中心上}
\lstinputlisting[style=Gcode-more]{../Mould_Machining_Programs/sub_programs/\nameDLone.nc}


\clearrightpage
%%%%%%%%%%%%%%%%%%%%%%%%%%%%%%%%%%%%%%%%%%%%%%%%%%%%%%%%%%
%% Prg. \DLtwoAC %%%%%%%%%%%%%%%%%%%%%%%%%%%%%%%%%%%%%%%%%
%%%%%%%%%%%%%%%%%%%%%%%%%%%%%%%%%%%%%%%%%%%%%%%%%%%%%%%%%%
\modcaptionof{lstlisting}{\DLtwoAC\,:\Dimple :移動 AC面 列内の各\Dimple 上}
\lstinputlisting[style=Gcode-more]{../Mould_Machining_Programs/sub_programs/\nameDLtwoAC.nc}


\clearrightpage
%%%%%%%%%%%%%%%%%%%%%%%%%%%%%%%%%%%%%%%%%%%%%%%%%%%%%%%%%%
%% Prg. \DLtwoBD %%%%%%%%%%%%%%%%%%%%%%%%%%%%%%%%%%%%%%%%%
%%%%%%%%%%%%%%%%%%%%%%%%%%%%%%%%%%%%%%%%%%%%%%%%%%%%%%%%%%
\modcaptionof{lstlisting}{\DLtwoBD\,:\Dimple :移動 BC面 列内の各\Dimple 上}
\lstinputlisting[style=Gcode-more]{../Mould_Machining_Programs/sub_programs/\nameDLtwoBD.nc}


\clearrightpage
%%%%%%%%%%%%%%%%%%%%%%%%%%%%%%%%%%%%%%%%%%%%%%%%%%%%%%%%%%
%% Prg. \DMLthreeAC %%%%%%%%%%%%%%%%%%%%%%%%%%%%%%%%%%%%%%
%%%%%%%%%%%%%%%%%%%%%%%%%%%%%%%%%%%%%%%%%%%%%%%%%%%%%%%%%%
\modcaptionof{lstlisting}{\DMLthreeAC\,:\Dimple :測定 AC内表面\texorpdfstring{$X$}{X}}
\lstinputlisting[style=Gcode-more]{../Mould_Machining_Programs/sub_programs/\nameDMLthreeAC.nc}


\clearrightpage
%%%%%%%%%%%%%%%%%%%%%%%%%%%%%%%%%%%%%%%%%%%%%%%%%%%%%%%%%%
%% Prg. \DMLthreeBD %%%%%%%%%%%%%%%%%%%%%%%%%%%%%%%%%%%%%%
%%%%%%%%%%%%%%%%%%%%%%%%%%%%%%%%%%%%%%%%%%%%%%%%%%%%%%%%%%
\modcaptionof{lstlisting}{\DMLthreeBD\,:\Dimple :測定 BD内表面\texorpdfstring{$Y$}{Y}}
\lstinputlisting[style=Gcode-more]{../Mould_Machining_Programs/sub_programs/\nameDMLthreeBD.nc}


\clearrightpage
%%%%%%%%%%%%%%%%%%%%%%%%%%%%%%%%%%%%%%%%%%%%%%%%%%%%%%%%%%
%% Prg. \DKLthreeAC %%%%%%%%%%%%%%%%%%%%%%%%%%%%%%%%%%%%%%
%%%%%%%%%%%%%%%%%%%%%%%%%%%%%%%%%%%%%%%%%%%%%%%%%%%%%%%%%%
\modcaptionof{lstlisting}{\DKLthreeAC\,:\Dimple :加工 AC内表面\texorpdfstring{$X$}{X}}
\lstinputlisting[style=Gcode-more]{../Mould_Machining_Programs/sub_programs/\nameDKLthreeAC.nc}


\clearrightpage
%%%%%%%%%%%%%%%%%%%%%%%%%%%%%%%%%%%%%%%%%%%%%%%%%%%%%%%%%%
%% Prg. \DKLthreeBD %%%%%%%%%%%%%%%%%%%%%%%%%%%%%%%%%%%%%%
%%%%%%%%%%%%%%%%%%%%%%%%%%%%%%%%%%%%%%%%%%%%%%%%%%%%%%%%%%
\modcaptionof{lstlisting}{\DKLthreeBD\,:\Dimple :加工 BD内表面\texorpdfstring{$Y$}{Y}}
\lstinputlisting[style=Gcode-more]{../Mould_Machining_Programs/sub_programs/\nameDKLthreeBD.nc}



\clearrightpage
%%%%%%%%%%%%%%%%%%%%%%%%%%%%%%%%%%%%%%%%%%%%%%%%%%%%%%%%%%
%% section I.5 %%%%%%%%%%%%%%%%%%%%%%%%%%%%%%%%%%%%%%%%%%%
%%%%%%%%%%%%%%%%%%%%%%%%%%%%%%%%%%%%%%%%%%%%%%%%%%%%%%%%%%
\modHeadsection{その他のサブプログラム}


%%%%%%%%%%%%%%%%%%%%%%%%%%%%%%%%%%%%%%%%%%%%%%%%%%%%%%%%%%
%% subsection I.5.1 %%%%%%%%%%%%%%%%%%%%%%%%%%%%%%%%%%%%%%
%%%%%%%%%%%%%%%%%%%%%%%%%%%%%%%%%%%%%%%%%%%%%%%%%%%%%%%%%%
\subsection{一時停止およびワーク確認}


%%%%%%%%%%%%%%%%%%%%%%%%%%%%%%%%%%%%%%%%%%%%%%%%%%%%%%%%%%
%% Prg. \OpauseCheck %%%%%%%%%%%%%%%%%%%%%%%%%%%%%%%%%%%%%
%%%%%%%%%%%%%%%%%%%%%%%%%%%%%%%%%%%%%%%%%%%%%%%%%%%%%%%%%%
\modcaptionof{lstlisting}{\OpauseCheck\,:90\texorpdfstring{$\boldsymbol{^\circ}$}{°}回転 扉前一時停止}
\lstinputlisting[style=Gcode-more]{../Mould_Machining_Programs/sub_programs/\nameOpauseCheck.nc}


\clearrightpage
%%%%%%%%%%%%%%%%%%%%%%%%%%%%%%%%%%%%%%%%%%%%%%%%%%%%%%%%%%
%% subsection I.5.2 %%%%%%%%%%%%%%%%%%%%%%%%%%%%%%%%%%%%%%
%%%%%%%%%%%%%%%%%%%%%%%%%%%%%%%%%%%%%%%%%%%%%%%%%%%%%%%%%%
\subsection{タッチセンサープローブ電源スイッチ}
%%%%%%%%%%%%%%%%%%%%%%%%%%%%%%%%%%%%%%%%%%%%%%%%%%%%%%%%%%
%% Prg. \OsensorOn %%%%%%%%%%%%%%%%%%%%%%%%%%%%%%%%%%%%%%%
%%%%%%%%%%%%%%%%%%%%%%%%%%%%%%%%%%%%%%%%%%%%%%%%%%%%%%%%%%
\modcaptionof{lstlisting}{\OsensorOn\,:タッチセンサー電源ON}
\lstinputlisting[style=Gcode-more]{../Mould_Machining_Programs/sub_programs/\nameOsensorOn.nc}


\clearpage
%%%%%%%%%%%%%%%%%%%%%%%%%%%%%%%%%%%%%%%%%%%%%%%%%%%%%%%%%%
%% Prg. \OsensorOff %%%%%%%%%%%%%%%%%%%%%%%%%%%%%%%%%%%%%%
%%%%%%%%%%%%%%%%%%%%%%%%%%%%%%%%%%%%%%%%%%%%%%%%%%%%%%%%%%
\modcaptionof{lstlisting}{\OsensorOff\,:タッチセンサー電源OFF}
\lstinputlisting[style=Gcode-more]{../Mould_Machining_Programs/sub_programs/\nameOsensorOff.nc}


\clearrightpage
%%%%%%%%%%%%%%%%%%%%%%%%%%%%%%%%%%%%%%%%%%%%%%%%%%%%%%%%%%
%% subsection I.5.2 %%%%%%%%%%%%%%%%%%%%%%%%%%%%%%%%%%%%%%
%%%%%%%%%%%%%%%%%%%%%%%%%%%%%%%%%%%%%%%%%%%%%%%%%%%%%%%%%%
\subsection{暖機運転}
%%%%%%%%%%%%%%%%%%%%%%%%%%%%%%%%%%%%%%%%%%%%%%%%%%%%%%%%%%
%% Prg. \OsensorOn %%%%%%%%%%%%%%%%%%%%%%%%%%%%%%%%%%%%%%%
%%%%%%%%%%%%%%%%%%%%%%%%%%%%%%%%%%%%%%%%%%%%%%%%%%%%%%%%%%
\modcaptionof{lstlisting}{\OwarmingupA\,:暖機運転}
\lstinputlisting[style=Gcode-more]{../Mould_Machining_Programs/sub_programs/\nameOwarmingupA.nc}


\clearrightpage
%%%%%%%%%%%%%%%%%%%%%%%%%%%%%%%%%%%%%%%%%%%%%%%%%%%%%%%%%%
%% Prg. \OsensorOn %%%%%%%%%%%%%%%%%%%%%%%%%%%%%%%%%%%%%%%
%%%%%%%%%%%%%%%%%%%%%%%%%%%%%%%%%%%%%%%%%%%%%%%%%%%%%%%%%%
\modcaptionof{lstlisting}{\Owarmingup\,:暖機運転用サブプログラム}
\lstinputlisting[style=Gcode-more]{../Mould_Machining_Programs/sub_programs/\nameOwarmingup.nc}


\clearrightpage
%%%%%%%%%%%%%%%%%%%%%%%%%%%%%%%%%%%%%%%%%%%%%%%%%%%%%%%%%%
%% subsection I.5.2 %%%%%%%%%%%%%%%%%%%%%%%%%%%%%%%%%%%%%%
%%%%%%%%%%%%%%%%%%%%%%%%%%%%%%%%%%%%%%%%%%%%%%%%%%%%%%%%%%
\subsection{工具長自動測定}
%%%%%%%%%%%%%%%%%%%%%%%%%%%%%%%%%%%%%%%%%%%%%%%%%%%%%%%%%%
%% Prg. \OsensorOn %%%%%%%%%%%%%%%%%%%%%%%%%%%%%%%%%%%%%%%
%%%%%%%%%%%%%%%%%%%%%%%%%%%%%%%%%%%%%%%%%%%%%%%%%%%%%%%%%%
\modcaptionof{lstlisting}{\OtoolLengthA\,:指定工具 工具長自動測定}
\lstinputlisting[style=Gcode-more]{../Mould_Machining_Programs/sub_programs/\nameOtoolLengthA.nc}


\clearrightpage
%%%%%%%%%%%%%%%%%%%%%%%%%%%%%%%%%%%%%%%%%%%%%%%%%%%%%%%%%%
%% Prg. \OsensorOn %%%%%%%%%%%%%%%%%%%%%%%%%%%%%%%%%%%%%%%
%%%%%%%%%%%%%%%%%%%%%%%%%%%%%%%%%%%%%%%%%%%%%%%%%%%%%%%%%%
\modcaptionof{lstlisting}{\OtoolLength\,:工具長 自動測定用サブプログラム}
\lstinputlisting[style=Gcode-more]{../Mould_Machining_Programs/sub_programs/\nameOtoolLength.nc}

\end{appendices}

\addtocontents{toc}{\protect\end{tocBox}}
