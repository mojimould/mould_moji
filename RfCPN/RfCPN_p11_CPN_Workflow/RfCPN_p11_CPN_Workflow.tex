%!TEX root = ./RfCPN.tex


\addtocontents{toc}{\protect\cleardoublepage}
%%%%%%%%%%%%%%%%%%%%%%%%%%%%%%%%%%%%%%%%%%%%%%%%%%%%%%%%%
%% Part RDB %%%%%%%%%%%%%%%%%%%%%%%%%%%%%%%%%%%%%%%%%%%%%
%%%%%%%%%%%%%%%%%%%%%%%%%%%%%%%%%%%%%%%%%%%%%%%%%%%%%%%%%
\addtocontents{toc}{\protect\begin{tocBox}{\tmppartnum}}%
\tPart{現状の関連ドキュメント作成作業の見直し\TBW}{%
\paragraph*{\tpartgoal}
現状の関連ドキュメントの作成作業に対して、ソフトウェアの視点から作成状況を把握する
\tcbline*
\paragraph*{\tpartmethod}
現状の関連ドキュメントの内容から作成状況の推察を行う
\tcbline*
\paragraph*{\tpartbackground}
当社のソフトウェアに関する業務については惨憺たる状態であることは先にも述べたとおりである。

 機械化(自動化)でき得る内容の多くが手動でなされている。
すなわち、作業者に精神的・体力的負担を課した上で、工数・教育コストを多くかけ、品質・信頼性を低下させている
%% footnote %%%%%%%%%%%%%%%%%%%%%
\footnote{さらにいうと、これが永き(少なくとも\MMC が稼働し始めた時点から)にわたって放置されている。}。
%%%%%%%%%%%%%%%%%%%%%%%%%%%%%%%%%

 したがって、ソフトウェア関連業務の見直しは極めて重要な課題である。
作業の機械化に向けて、まず関連ドキュメントの内容の把握から始める必要がある。
}{%
\paragraph*{\tpartconclusion}
(to be written...)
\tcbline*
\paragraph*{\tpartnextstep}
(to be written...)
}

%%%%%%%%%%%%%%%%%%%%%%%%%%%%%%%%%%%%%%%%%%%%%%%%%%%%%%%%%%
%% chaptera %%%%%%%%%%%%%%%%%%%%%%%%%%%%%%%%%%%%%%%%%%%%%%
%%%%%%%%%%%%%%%%%%%%%%%%%%%%%%%%%%%%%%%%%%%%%%%%%%%%%%%%%%
\modHeadchapter{業務フローの取決め}
(to be written...)

%%%%%%%%%%%%%%%%%%%%%%%%%%%%%%%%%%%%%%%%%%%%%%%%%%%%%%%%%%
%% section D.1 %%%%%%%%%%%%%%%%%%%%%%%%%%%%%%%%%%%%%%%%%%%
%%%%%%%%%%%%%%%%%%%%%%%%%%%%%%%%%%%%%%%%%%%%%%%%%%%%%%%%%%
\modHeadsection{列}
(to be written...)
%%%%%%%%%%%%%%%%%%%%%%%%%%%%%%%%%%%%%%%%%%%%%%%%%%%%%%%%%
%% Appendices %%%%%%%%%%%%%%%%%%%%%%%%%%%%%%%%%%%%%%%%%%%
%%%%%%%%%%%%%%%%%%%%%%%%%%%%%%%%%%%%%%%%%%%%%%%%%%%%%%%%%
\begin{appendices}
%\Appendixpart
\end{appendices}

\addtocontents{toc}{\protect\end{tocBox}}
