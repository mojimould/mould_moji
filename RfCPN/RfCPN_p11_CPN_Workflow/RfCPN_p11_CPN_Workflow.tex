%!TEX root = ./RfCPN.tex


\addtocontents{toc}{\protect\cleardoublepage}
%%%%%%%%%%%%%%%%%%%%%%%%%%%%%%%%%%%%%%%%%%%%%%%%%%%%%%%%%
%% Part RDB %%%%%%%%%%%%%%%%%%%%%%%%%%%%%%%%%%%%%%%%%%%%%
%%%%%%%%%%%%%%%%%%%%%%%%%%%%%%%%%%%%%%%%%%%%%%%%%%%%%%%%%
\addtocontents{toc}{\protect\begin{tocBox}{\tmppartnum}}%
\tPart{現状の関連ドキュメント作成作業\TBW}{%
\paragraph*{\tpartgoal}
現状の関連ドキュメントの作成作業に対して、ソフトウェアの視点から作成状況を把握する
\tcbline*
\paragraph*{\tpartmethod}
現状の関連ドキュメントの内容から作成状況の推察を行う
\tcbline*
\paragraph*{\tpartbackground}
当社のソフトウェアに関する業務については誠に厳しい状態であることは先に述べたとおりである。

 実際、機械化(自動化)でき得る内容の多くが手動でなされている。
これはすなわち、作業者に精神的・体力的負担を課し、工数・教育コストを多くかけた上で、品質・信頼性を低下させていることを意味する
%% footnote %%%%%%%%%%%%%%%%%%%%%
\footnote{さらにいうと、これが永き(少なく見積もって四半世紀以上)にわたって放置されている。}。
%%%%%%%%%%%%%%%%%%%%%%%%%%%%%%%%%

 したがって、ソフトウェア関連業務の見直しは極めて重要な課題である。
作業の機械化に向けて、まず関連ドキュメントの内容の把握から始める必要がある。
}{%
\paragraph*{\tpartconclusion}
(to be written...)
\tcbline*
\paragraph*{\tpartnextstep}
(to be written...)
}

%%%%%%%%%%%%%%%%%%%%%%%%%%%%%%%%%%%%%%%%%%%%%%%%%%%%%%%%%%
%% chaptera %%%%%%%%%%%%%%%%%%%%%%%%%%%%%%%%%%%%%%%%%%%%%%
%%%%%%%%%%%%%%%%%%%%%%%%%%%%%%%%%%%%%%%%%%%%%%%%%%%%%%%%%%
\modHeadchapter{作成するドキュメント\TBW}
\begin{enumerate}
\item NCメインプログラム
\item マシニングセンタ加工後寸法確認表
\item 図面
\end{enumerate}



%%%%%%%%%%%%%%%%%%%%%%%%%%%%%%%%%%%%%%%%%%%%%%%%%%%%%%%%%%
%% section 04.02 %%%%%%%%%%%%%%%%%%%%%%%%%%%%%%%%%%%%%%%%%
%%%%%%%%%%%%%%%%%%%%%%%%%%%%%%%%%%%%%%%%%%%%%%%%%%%%%%%%%%
\modHeadsection{達成したい目標\TBW}
\begin{enumerate}[label=\sarrow]
\item \textgt{書類生成の自動化}:手作業による書類記入の削減・簡易化
\item \textgt{コード生成の自動化}:手作業による\index{NCメインプログラム}NC(メイン)プログラム作成の簡易化
\end{enumerate}
\begin{enumerate}[label=\sarrow]
\item 手作業による書類記入作業の削減
\end{enumerate}
(to be written...)



%%%%%%%%%%%%%%%%%%%%%%%%%%%%%%%%%%%%%%%%%%%%%%%%%%%%%%%%%
%% Appendices %%%%%%%%%%%%%%%%%%%%%%%%%%%%%%%%%%%%%%%%%%%
%%%%%%%%%%%%%%%%%%%%%%%%%%%%%%%%%%%%%%%%%%%%%%%%%%%%%%%%%
\begin{appendices}
%\Appendixpart
\end{appendices}

\addtocontents{toc}{\protect\end{tocBox}}
