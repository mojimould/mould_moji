%!TEX root = ../RfCPN.tex


\modHeadchapter{イシュー・問題の特定}
先に述べた\expandafterindex{ぎょうむフロー(\yomiMMC)@業務フロー(\nameMMC)}業務フローを通して、\MMC に関する\index{イシュー}イシュー(issue)および\index{もんだい(problem)@問題(problem)}問題(problem)の特定を試みる。



%%%%%%%%%%%%%%%%%%%%%%%%%%%%%%%%%%%%%%%%%%%%%%%%%%%%%%%%%%
%% section 02.01 %%%%%%%%%%%%%%%%%%%%%%%%%%%%%%%%%%%%%%%%%
%%%%%%%%%%%%%%%%%%%%%%%%%%%%%%%%%%%%%%%%%%%%%%%%%%%%%%%%%%
\modHeadsection{安全に関するイシュー・問題}

%%%%%%%%%%%%%%%%%%%%%%%%%%%%%%%%%%%%%%%%%%%%%%%%%%%%%%%%%%
%% Issues %%%%%%%%%%%%%%%%%%%%%%%%%%%%%%%%%%%%%%%%%%%%%%%%
%%%%%%%%%%%%%%%%%%%%%%%%%%%%%%%%%%%%%%%%%%%%%%%%%%%%%%%%%%
\begin{Issues}{作業員の機内の侵入に伴うリスク}
一般に、\index{きないへのしんにゅう@機内への侵入}機内への侵入は、転倒・巻き込まれ等のリスクを伴う
\begin{enumerate}[label=\sarrow]
\item[{\sarrow[red]}]
機内に侵入し直接測定をしないと、ワークの\index{かこうげんてん(がいさんち)@加工原点(概算値)}加工原点の概算値が見出だせない状態にある
\item[{\sarrow[red]}] 機内に侵入しないと、\index{はのこうかん(フェイスミル)@刃の交換(フェイスミル)}フェイスミルの刃の交換ができない状態にある
\item 機内に侵入しないと、内部の掃除ができない状態にある
\end{enumerate}
\end{Issues}
%%%%%%%%%%%%%%%%%%%%%%%%%%%%%%%%%%%%%%%%%%%%%%%%%%%%%%%%%%
%%%%%%%%%%%%%%%%%%%%%%%%%%%%%%%%%%%%%%%%%%%%%%%%%%%%%%%%%%
%%%%%%%%%%%%%%%%%%%%%%%%%%%%%%%%%%%%%%%%%%%%%%%%%%%%%%%%%%
%%%%%%%%%%%%%%%%%%%%%%%%%%%%%%%%%%%%%%%%%%%%%%%%%%%%%%%%%%
%% Issues %%%%%%%%%%%%%%%%%%%%%%%%%%%%%%%%%%%%%%%%%%%%%%%%
%%%%%%%%%%%%%%%%%%%%%%%%%%%%%%%%%%%%%%%%%%%%%%%%%%%%%%%%%%
\begin{Issues}{\index{てもちけんまき@手持ち研磨機}手持ち研磨機の使用に伴うリスク}
一般に、\index{てもちけんまき@手持ち研磨機}手持ち研磨機による加工は、巻き込まれや粉塵の付着・吸引等のリスクを伴う
\begin{enumerate}[label=\sarrow]
\item[{\sarrow[red]}]
寸法の小さな\EndFaceCChamferMilling が\MMC で行われず、\index{てもちけんまき@手持ち研磨機}手持ち研磨機により人手で行われている状態にある
\item[{\sarrow[red]}] \EndFaceChamferMilling に関する解析的な幾何情報を導出しないまま放置され続けている
\end{enumerate}
\end{Issues}
%%%%%%%%%%%%%%%%%%%%%%%%%%%%%%%%%%%%%%%%%%%%%%%%%%%%%%%%%%
%%%%%%%%%%%%%%%%%%%%%%%%%%%%%%%%%%%%%%%%%%%%%%%%%%%%%%%%%%
%%%%%%%%%%%%%%%%%%%%%%%%%%%%%%%%%%%%%%%%%%%%%%%%%%%%%%%%%%
%%%%%%%%%%%%%%%%%%%%%%%%%%%%%%%%%%%%%%%%%%%%%%%%%%%%%%%%%%
%% Issues %%%%%%%%%%%%%%%%%%%%%%%%%%%%%%%%%%%%%%%%%%%%%%%%
%%%%%%%%%%%%%%%%%%%%%%%%%%%%%%%%%%%%%%%%%%%%%%%%%%%%%%%%%%
\begin{Issues}{柵への衝突に伴うリスク}
\begin{enumerate}[label=\sarrow]
\item 後付けされた\expandafterindex{あんぜんさく(\yomiMMC)@安全柵(\nameMMC)}安全柵(と称されている柵)が、衝突の\index{リスク(しょうとつ)@リスク(衝突)}リスクを生み出している
\item 一部のみ(出入口のみ)を着目し、周囲(柵の周辺)が軽視され、\index{あんぜんたいさく@安全対策}安全対策が機能せず反対に危険リスクを生み出している
\end{enumerate}
\end{Issues}
%%%%%%%%%%%%%%%%%%%%%%%%%%%%%%%%%%%%%%%%%%%%%%%%%%%%%%%%%%
%%%%%%%%%%%%%%%%%%%%%%%%%%%%%%%%%%%%%%%%%%%%%%%%%%%%%%%%%%
%%%%%%%%%%%%%%%%%%%%%%%%%%%%%%%%%%%%%%%%%%%%%%%%%%%%%%%%%%
%%%%%%%%%%%%%%%%%%%%%%%%%%%%%%%%%%%%%%%%%%%%%%%%%%%%%%%%%%
%% Issues %%%%%%%%%%%%%%%%%%%%%%%%%%%%%%%%%%%%%%%%%%%%%%%%
%%%%%%%%%%%%%%%%%%%%%%%%%%%%%%%%%%%%%%%%%%%%%%%%%%%%%%%%%%
\begin{Issues}{\index{リフター}リフターへの衝突に伴うリスク}
\begin{enumerate}[label=\sarrow]
\item ワークの\index{うけいれけんさ(ワーク)@受入検査(ワーク)}受入検査の際に\index{リフター}リフターや\index{クレーン}クレーンが必ず\index{あんぜんつうろ@安全通路}安全通路を通る構造にある
\item \index{じぎょうぶりねん@事業部理念}事業部理念が機能しておらず、安全性より生産性を優先する\index{けいえいほうしん(とうしゃ)@経営方針(当社)}経営方針となっている
\end{enumerate}
\end{Issues}
%%%%%%%%%%%%%%%%%%%%%%%%%%%%%%%%%%%%%%%%%%%%%%%%%%%%%%%%%%
%%%%%%%%%%%%%%%%%%%%%%%%%%%%%%%%%%%%%%%%%%%%%%%%%%%%%%%%%%
%%%%%%%%%%%%%%%%%%%%%%%%%%%%%%%%%%%%%%%%%%%%%%%%%%%%%%%%%%


\clearpage
%%%%%%%%%%%%%%%%%%%%%%%%%%%%%%%%%%%%%%%%%%%%%%%%%%%%%%%%%%
%% section 02.02 %%%%%%%%%%%%%%%%%%%%%%%%%%%%%%%%%%%%%%%%%
%%%%%%%%%%%%%%%%%%%%%%%%%%%%%%%%%%%%%%%%%%%%%%%%%%%%%%%%%%
\modHeadsection{\index{ひんしつ@品質}品質に関するイシュー・問題\TBW}
\begin{Issues}{\index{てもちけんまき@手持ち研磨機}手持ち研磨機の使用}
多くの場合、人手(\index{てもちけんまき@手持ち研磨機}手持ち研磨機)による\EndFaceChamferMilling は、マシニングセンタによる\EndFaceChamferMilling に比べて品質が低下する
\begin{enumerate}[label=\sarrow]
\item[{\sarrow[red]}]
寸法の小さな\EndFaceCChamferMilling が\MMC で行われず、\index{てもちけんまき@手持ち研磨機}手持ち研磨機により人手で行われている状態にある
\item[{\sarrow[red]}] \EndFaceChamferMilling に関する解析的な幾何情報を導出しないまま放置され続けている
\end{enumerate}
\end{Issues}

\begin{Issues}{仕上げ前加工・仕上げ加工の分離\TBW}
\begin{enumerate}[label=\sarrow]
\item[{\sarrow[red]}]
\end{enumerate}
\end{Issues}

\begin{Issues}{\KeywayMilling のばりの除去\TBW}
\begin{enumerate}[label=\sarrow]
\item[{\sarrow[red]}]
\end{enumerate}
\end{Issues}

\begin{Issues}{\EndFaceInCChamferMilling の始点・終点の変更\TBW}
\begin{enumerate}[label=\sarrow]
\item[{\sarrow[red]}]
\end{enumerate}
\end{Issues}

\begin{Issues}{\EndFaceInCChamferMilling のコーナーRの補正\TBW}
\begin{enumerate}[label=\sarrow]
\item[{\sarrow[red]}]
\end{enumerate}
\end{Issues}


\clearpage
%%%%%%%%%%%%%%%%%%%%%%%%%%%%%%%%%%%%%%%%%%%%%%%%%%%%%%%%%%
%% section 02.03 %%%%%%%%%%%%%%%%%%%%%%%%%%%%%%%%%%%%%%%%%
%%%%%%%%%%%%%%%%%%%%%%%%%%%%%%%%%%%%%%%%%%%%%%%%%%%%%%%%%%
\modHeadsection{\index{さぎょうこうりつ@作業効率}作業効率に関するイシュー・問題\TBW}


%%%%%%%%%%%%%%%%%%%%%%%%%%%%%%%%%%%%%%%%%%%%%%%%%%%%%%%%%%
%% subsection 02.03.01 %%%%%%%%%%%%%%%%%%%%%%%%%%%%%%%%%%%
%%%%%%%%%%%%%%%%%%%%%%%%%%%%%%%%%%%%%%%%%%%%%%%%%%%%%%%%%%
\subsection{\Drawing の作成における作業効率}

\begin{Issues}{\TBW}
\begin{enumerate}[label=\sarrow]
\item[{\sarrow[red]}]
\end{enumerate}
\end{Issues}


%%%%%%%%%%%%%%%%%%%%%%%%%%%%%%%%%%%%%%%%%%%%%%%%%%%%%%%%%%
%% subsection 02.03.02 %%%%%%%%%%%%%%%%%%%%%%%%%%%%%%%%%%%
%%%%%%%%%%%%%%%%%%%%%%%%%%%%%%%%%%%%%%%%%%%%%%%%%%%%%%%%%%
\subsection{\index{NCプログラム}NCプログラムの作成における作業効率}

\begin{Issues}{\KeywayCenter 位置の自動計算\TBW}
\begin{enumerate}[label=\sarrow]
\item[{\sarrow[red]}]
\end{enumerate}
\end{Issues}

\begin{Issues}{\AsideKeywayDepth 指定時の中心位置の自動測定\TBW}
\begin{enumerate}[label=\sarrow]
\item[{\sarrow[red]}]
\end{enumerate}
\end{Issues}

\begin{Issues}{NCプログラムの自動生成\TBW}
\begin{enumerate}[label=\sarrow]
\item[{\sarrow[red]}]
\end{enumerate}
\end{Issues}


%%%%%%%%%%%%%%%%%%%%%%%%%%%%%%%%%%%%%%%%%%%%%%%%%%%%%%%%%%
%% subsection 02.03.03 %%%%%%%%%%%%%%%%%%%%%%%%%%%%%%%%%%%
%%%%%%%%%%%%%%%%%%%%%%%%%%%%%%%%%%%%%%%%%%%%%%%%%%%%%%%%%%
\subsection{測定における作業効率}

\begin{Issues}{測定基準値の手動による実測\TBW}
\begin{enumerate}[label=\sarrow]
\item[{\sarrow[red]}]
\end{enumerate}
\end{Issues}


%%%%%%%%%%%%%%%%%%%%%%%%%%%%%%%%%%%%%%%%%%%%%%%%%%%%%%%%%%
%% subsection 02.03.04 %%%%%%%%%%%%%%%%%%%%%%%%%%%%%%%%%%%
%%%%%%%%%%%%%%%%%%%%%%%%%%%%%%%%%%%%%%%%%%%%%%%%%%%%%%%%%%
\subsection{\EndFacecutMilling における作業効率}

\begin{Issues}{\EndFacecutMilling の手動による\TDCorrection\TBW}
\begin{enumerate}[label=\sarrow]
\item[{\sarrow[red]}]
\end{enumerate}
\end{Issues}


%%%%%%%%%%%%%%%%%%%%%%%%%%%%%%%%%%%%%%%%%%%%%%%%%%%%%%%%%%
%% subsection 02.03.05 %%%%%%%%%%%%%%%%%%%%%%%%%%%%%%%%%%%
%%%%%%%%%%%%%%%%%%%%%%%%%%%%%%%%%%%%%%%%%%%%%%%%%%%%%%%%%%
\subsection{\KeywayMilling における作業効率}

\begin{Issues}{加工回数($Z$方向)の自動特定\TBW}
\begin{enumerate}[label=\sarrow]
\item[{\sarrow[red]}]
\end{enumerate}
\end{Issues}


%%%%%%%%%%%%%%%%%%%%%%%%%%%%%%%%%%%%%%%%%%%%%%%%%%%%%%%%%%
%% subsection 02.03.06 %%%%%%%%%%%%%%%%%%%%%%%%%%%%%%%%%%%
%%%%%%%%%%%%%%%%%%%%%%%%%%%%%%%%%%%%%%%%%%%%%%%%%%%%%%%%%%
\subsection{\EndFaceChamferMilling における作業効率}

\begin{Issues}{マシニングセンタによる\EndFaceChamferMilling の追加\TBW}
\begin{enumerate}[label=\sarrow]
\item[{\sarrow[red]}]
\end{enumerate}
\end{Issues}


%%%%%%%%%%%%%%%%%%%%%%%%%%%%%%%%%%%%%%%%%%%%%%%%%%%%%%%%%%
%% subsection 02.03.07 %%%%%%%%%%%%%%%%%%%%%%%%%%%%%%%%%%%
%%%%%%%%%%%%%%%%%%%%%%%%%%%%%%%%%%%%%%%%%%%%%%%%%%%%%%%%%%
\subsection{加工全般における作業効率}

\begin{Issues}{加工回数の自動決定\TBW}
\begin{enumerate}[label=\sarrow]
\item[{\sarrow[red]}]
\end{enumerate}
\end{Issues}

\begin{Issues}{\TBW}
\begin{enumerate}[label=\sarrow]
\item[{\sarrow[red]}]
\end{enumerate}
\end{Issues}


\clearpage
%%%%%%%%%%%%%%%%%%%%%%%%%%%%%%%%%%%%%%%%%%%%%%%%%%%%%%%%%%
%% section 02.03 %%%%%%%%%%%%%%%%%%%%%%%%%%%%%%%%%%%%%%%%%
%%%%%%%%%%%%%%%%%%%%%%%%%%%%%%%%%%%%%%%%%%%%%%%%%%%%%%%%%%
\modHeadsection{信頼性に関するイシュー・問題\TBW}

\begin{Issues}{パラメータ誤入力の防止\TBW}
\begin{enumerate}[label=\sarrow]
\item[{\sarrow[red]}]
\end{enumerate}
\end{Issues}

\begin{Issues}{\KeywayMilling{} 外部ワーク座標系による衝突の防止\TBW}
\begin{enumerate}[label=\sarrow]
\item[{\sarrow[red]}]
\end{enumerate}
\end{Issues}



\clearpage
%%%%%%%%%%%%%%%%%%%%%%%%%%%%%%%%%%%%%%%%%%%%%%%%%%%%%%%%%%
%% section 02.04 %%%%%%%%%%%%%%%%%%%%%%%%%%%%%%%%%%%%%%%%%
%%%%%%%%%%%%%%%%%%%%%%%%%%%%%%%%%%%%%%%%%%%%%%%%%%%%%%%%%%
\modHeadsection{根源的な問題点\TBW}
そもそも、「\expandafterindex{\yomiDrawing(モールド)@\nameDrawing(モールド)}\nameDrawing を見て\index{NCプログラム}NCプログラムを作成」するという時点で、明らかに尋常な状態でないことは言うまでもない
%% footnote %%%%%%%%%%%%%%%%%%%%%
\footnote{さらに言及すると、これが問題点(改善可能な点)だということがこれまで一切(スタッフ・管理職を含めて)認識されなかったことが(非常に大きな・根本的な)問題点として挙げられる。
つまり、(\index{せいぞうぎょう@製造業}製造業にも関わらず)当社が\index{ソフトウェアエンジニアリング}ソフトウェアエンジニアリングをあからさまに蔑ろにし続けていることが根本に存在する。
ソフトウェアエンジニアリングに対する意識の低さ、ならびにモラル・マナーの欠如が顕著に露呈している。
このような事態は、この業務に限らず社内のほぼ全ての業務において同様である。

なお、これは客観的な事実を述べたものであるということを、念のために付記しておく。}。
%%%%%%%%%%%%%%%%%%%%%%%%%%%%%%%%%
\begin{Issues}{NCプログラムの作成}
NCプログラム作成より圧倒的に情報量の多い\expandafterindex{\yomiDrawing のさくせい)@\nameDrawing の作成}\nameDrawing の作成がスタッフまたは管理職によりなされているにも関わらず、NCプログラム作成はその後工程としてなされている
\tcbline*
NCプログラム作成が、現場作業員(一般職)により行われている
\end{Issues}
その\expandafterindex{\yomiDrawing のさくせい)@\nameDrawing の作成}\nameDrawing の作成についても以下のような状態にある。
\begin{Issues}{\expandafterindex{\yomiDrawing のさくせい)@\nameDrawing の作成}\nameDrawing の作成}
\expandafterindex{\yomiDrawing のさくせい)@\nameDrawing の作成}\nameDrawing の作成が、\index{CADソフトウェア}CADソフトウェアを用いて明細ごとに手動で描かれている
\end{Issues}
これらは単に、モールドに関するデータが管理状態にないことを示している。
\begin{Issues}{データベースの非存在}
\index{モールド}モールドに関する\index{かんけいデータベース@関係データベース}関係データベース(\index{RDB}RDB)に相当するものが存在しない
\end{Issues}
