%!TEX root = ../RfCPN.tex


\modHeadchapter{イシュー・問題の特定}
先に述べた\expandafterindex{ぎょうむフロー(\yomiMMC)@業務フロー(\nameMMC)}業務フローを通して、\MMC に関する\index{イシュー}イシュー(issue)および\index{もんだい(problem)@問題(problem)}問題(problem)の特定を試みる。



%%%%%%%%%%%%%%%%%%%%%%%%%%%%%%%%%%%%%%%%%%%%%%%%%%%%%%%%%%
%% section 02.01 %%%%%%%%%%%%%%%%%%%%%%%%%%%%%%%%%%%%%%%%%
%%%%%%%%%%%%%%%%%%%%%%%%%%%%%%%%%%%%%%%%%%%%%%%%%%%%%%%%%%
\modHeadsection{安全(safety)に関するイシュー・問題}

\begin{Issues}{オペレータの精神的高ストレス下での作業の強要}
通常、\index{NCプログラム}NCプログラムの編集はシステム管理者・設計者・\index{プログラマ}プログラマ等が行う高度な業務である
\begin{enumerate}[label=\sarrow]
\item[{\sarrow[red]}]\index{オペレータ}オペレータに対してメインプログラムの直接的な編集を強要させている
\item[{\sarrow[red]}]専門の担当者が存在せず、オペレータが兼任している状態になっている
\end{enumerate}
\end{Issues}

\begin{Issues}{\index{オペレータ}オペレータの機内の侵入に伴うリスク}
一般に、\index{きないへのしんにゅう@機内への侵入}機内への侵入は、転倒・巻き込まれ等のリスクを伴う
\begin{enumerate}[label=\sarrow]
\item[{\sarrow[red]}]機内に侵入し直接測定をしないと、ワークの\index{かこうげんてん(がいさんち)@加工原点(概算値)}加工原点の概算値が見出だせない状態にある
\item[{\sarrow[red]}]機内に侵入しないと、\index{はのこうかん(フェイスミル)@刃の交換(フェイスミル)}フェイスミルの刃の交換ができない状態にある
\item 機内に侵入しないと、内部の掃除ができない状態にある
\end{enumerate}
\end{Issues}

\begin{Issues}{\index{てもちけんまき@手持ち研磨機}手持ち研磨機の使用に伴うリスク}
一般に、\index{てもちけんまき@手持ち研磨機}手持ち研磨機による加工は、巻き込まれや粉塵の付着・吸引等のリスクを伴う
\begin{enumerate}[label=\sarrow]
\item[{\sarrow[red]}]寸法の小さな\EndFaceChamferMilling が\MMC で行われず、\index{てもちけんまき@手持ち研磨機}手持ち研磨機により手作業で行われている状態にある
\item[{\sarrow[red]}]\EndFaceChamferMilling に関する解析的な幾何情報を導出しないまま放置され続けている
\end{enumerate}
\end{Issues}

\begin{Issues}{柵への衝突に伴うリスク}
\begin{enumerate}[label=\sarrow]
\item 後付けされた\expandafterindex{あんぜんさく(\yomiMMC)@安全柵(\nameMMC)}安全柵(と称されている柵)が、衝突の\index{リスク(しょうとつ)@リスク(衝突)}リスクを生み出している
\item 一部のみ(出入口のみ)を着目し、周囲(柵の周辺)が軽視され、\index{あんぜんたいさく@安全対策}安全対策が機能せず反対に危険リスクを生み出している
\end{enumerate}
\end{Issues}

\begin{Issues}{\index{リフター}リフターへの衝突に伴うリスク}
\begin{enumerate}[label=\sarrow]
\item ワークの\index{うけいれけんさ(ワーク)@受入検査(ワーク)}受入検査の際に\index{リフター}リフターや\index{クレーン}クレーンが必ず\index{あんぜんつうろ@安全通路}安全通路を通る構造にある
\item \index{じぎょうぶりねん@事業部理念}事業部理念が機能しておらず、安全性より生産性を優先する\index{けいえいほうしん(とうしゃ)@経営方針(当社)}経営方針となっている
\end{enumerate}
\end{Issues}


\clearpage
%%%%%%%%%%%%%%%%%%%%%%%%%%%%%%%%%%%%%%%%%%%%%%%%%%%%%%%%%%
%% section 02.02 %%%%%%%%%%%%%%%%%%%%%%%%%%%%%%%%%%%%%%%%%
%%%%%%%%%%%%%%%%%%%%%%%%%%%%%%%%%%%%%%%%%%%%%%%%%%%%%%%%%%
\modHeadsection{\index{ひんしつ@品質}品質に関するイシュー・問題}


%%%%%%%%%%%%%%%%%%%%%%%%%%%%%%%%%%%%%%%%%%%%%%%%%%%%%%%%%%
%% subsection 02.02.01 %%%%%%%%%%%%%%%%%%%%%%%%%%%%%%%%%%%
%%%%%%%%%%%%%%%%%%%%%%%%%%%%%%%%%%%%%%%%%%%%%%%%%%%%%%%%%%
\subsection{測定における品質}

\begin{Issues}{\KeywayCenterMeasurement(AC方向)の不安定性}
\begin{enumerate}[label=\sarrow]
\item[{\sarrow[red]}]\KeywayCenter(AC方向)の測定が、計算による関節的な方法で導出されている
\item[{\sarrow[red]}]\CenterCurvature に伴う誤差を含み、位置(特に\AsideKeywayDepth)が安定しない
\end{enumerate}
\end{Issues}

\begin{Issues}{\KeywayCenterMeasurement(BD方向)の不安定性}
\begin{enumerate}[label=\sarrow]
\item[{\sarrow[red]}]\KeywayCenter(BD方向)が、端面におけるそれとして与えられている
\item[{\sarrow[red]}]BD方向の真直度に伴う誤差を含み、位置が安定しない
\end{enumerate}
\end{Issues}

\begin{Issues}{\CenterlineEndFaceDifMeasurement の不安定性}
\begin{enumerate}[label=\sarrow]
\item[{\sarrow[red]}]\CenterlineEndFaceDifMeasurement が、手動による\index{ハンドルそうさ@ハンドル操作}ハンドル操作で行われている
\item[{\sarrow[red]}]手作業によるため、測定位置や送り速さが安定しない
\item[{\sarrow[red]}]\Outcut・\Keyway・\CenterCurvature の寸法から自動化が可能にもかかわらず、事態が放置され続けている
\end{enumerate}
\end{Issues}

\begin{Issues}{\TLMeasurement の故障}
\begin{enumerate}[label=\sarrow]
\item\TLMeasurement 用装置が、物理的に壊れている
\item 現物合わせで\TLMeasurement を行っている
\end{enumerate}
\end{Issues}


%%%%%%%%%%%%%%%%%%%%%%%%%%%%%%%%%%%%%%%%%%%%%%%%%%%%%%%%%%
%% subsection 02.02.02 %%%%%%%%%%%%%%%%%%%%%%%%%%%%%%%%%%%
%%%%%%%%%%%%%%%%%%%%%%%%%%%%%%%%%%%%%%%%%%%%%%%%%%%%%%%%%%
\subsection{\OutcutMilling における品質}

\begin{Issues}{\CurvedOutcutMilling の不安定性}
\begin{enumerate}[label=\sarrow]
\item[{\sarrow[red]}]手作業による\index{ハンドルそうさ@ハンドル操作}ハンドル操作により測定が行われている
\item[{\sarrow[red]}]\index{NCプログラム}NCプログラム内の\OutcutLength の寸法に誤りが存在している
\end{enumerate}
\end{Issues}


%%%%%%%%%%%%%%%%%%%%%%%%%%%%%%%%%%%%%%%%%%%%%%%%%%%%%%%%%%
%% subsection 02.02.03 %%%%%%%%%%%%%%%%%%%%%%%%%%%%%%%%%%%
%%%%%%%%%%%%%%%%%%%%%%%%%%%%%%%%%%%%%%%%%%%%%%%%%%%%%%%%%%
\subsection{\KeywayMilling における品質}

\begin{Issues}{\KeywayMilling のかえりの\index{てさぎょう@手作業}手作業による除去}
\begin{enumerate}[label=\sarrow]
\item[{\sarrow[red]}]長方形・8角形(C面取)の\KeywayMilling の一部の頂点にかえりが残る
\item[{\sarrow[red]}]手作業でかえりを除去するため、頂点部が歪な形状になる
\end{enumerate}
\end{Issues}


\clearpage
%%%%%%%%%%%%%%%%%%%%%%%%%%%%%%%%%%%%%%%%%%%%%%%%%%%%%%%%%%
%% subsection 02.02.04 %%%%%%%%%%%%%%%%%%%%%%%%%%%%%%%%%%%
%%%%%%%%%%%%%%%%%%%%%%%%%%%%%%%%%%%%%%%%%%%%%%%%%%%%%%%%%%
\subsection{\EndFaceChamferMilling における品質}

\begin{Issues}{\index{てさぎょう@手作業}手作業による\EndFaceChamferMilling}
\begin{enumerate}[label=\sarrow]
\item[{\sarrow[red]}]
寸法の小さな\EndFaceChamferMilling が\index{てもちけんまき@手持ち研磨機}手持ち研磨機により\index{てさぎょう@手作業}手作業で行われている
\item[{\sarrow[red]}]\EndFaceChamferMilling に関する解析計算がなされないまま放置され続けている
\end{enumerate}
\end{Issues}

%%%%%%%%%%%%%%%%%%%%%%%%%%%%%%%%%%%%%%%%%%%%%%%%%%%%%%%%%%
%% subsection 02.02.04.1 %%%%%%%%%%%%%%%%%%%%%%%%%%%%%%%%%
%%%%%%%%%%%%%%%%%%%%%%%%%%%%%%%%%%%%%%%%%%%%%%%%%%%%%%%%%%
\subsubsection{\EndFaceOutCChamferMilling における品質}

\begin{Issues}{\EndFaceOutCChamferMilling の位置調整}
\begin{enumerate}[label=\sarrow]
\item[{\sarrow[red]}]\EndFaceOutCChamferMilling のAC方向の位置調整が手作業により行われている
\item[{\sarrow[red]}]目分量により位置調整がなされているため、位置が安定しない
\end{enumerate}
\end{Issues}

%%%%%%%%%%%%%%%%%%%%%%%%%%%%%%%%%%%%%%%%%%%%%%%%%%%%%%%%%%
%% subsection 02.02.04.2 %%%%%%%%%%%%%%%%%%%%%%%%%%%%%%%%%
%%%%%%%%%%%%%%%%%%%%%%%%%%%%%%%%%%%%%%%%%%%%%%%%%%%%%%%%%%
\subsubsection{\EndFaceInCChamferMilling における品質}

\begin{Issues}{\EndFaceInCChamferMilling の位置調整}
\begin{enumerate}[label=\sarrow]
\item[{\sarrow[red]}]\EndFaceInCChamferMilling のAC方向の位置調整が手作業により行われている
\item[{\sarrow[red]}]目分量により位置調整がなされているため、位置が安定しない
\end{enumerate}
\end{Issues}

\begin{Issues}{\EndFaceInCChamferMilling の始点・終点の位置}
\begin{enumerate}[label=\sarrow]
\item[{\sarrow[red]}]\EndFaceInCChamferMilling の始点および終点が直線部分にあり、加工の跡が残りやすい
\item[{\sarrow[red]}]始点および終点が同じ点になっているため、工具が少し摩耗しただけでも跡が残る
\end{enumerate}
\end{Issues}

\begin{Issues}{\EndFaceInCChamferMilling のコーナーRの補正}
\begin{enumerate}[label=\sarrow]
\item\index{しんがね@芯金}芯金の摩耗等により、\EndFaceInCChamfer のコーナーが変化しうる
\item[{\sarrow[red]}]\EndFaceInCChamferMilling のコーナーの値が一定なため、変化に対応ができない
\end{enumerate}
\end{Issues}


%%%%%%%%%%%%%%%%%%%%%%%%%%%%%%%%%%%%%%%%%%%%%%%%%%%%%%%%%%
%% subsection 02.02.02 %%%%%%%%%%%%%%%%%%%%%%%%%%%%%%%%%%%
%%%%%%%%%%%%%%%%%%%%%%%%%%%%%%%%%%%%%%%%%%%%%%%%%%%%%%%%%%
\subsection{\EndFaceBoringMilling における品質}

\begin{Issues}{\EndFaceBoringMilling の寸法の誤り}
\begin{enumerate}[label=\sarrow]
\item[{\sarrow[red]}]\EndFaceBoringMilling のAC方向の寸法が、図面のものと異なっている
\item 後工程で用いる端面座ぐり部の穴あけ用ジグを、誤った寸法のものに手作業で修正する必要がある
\end{enumerate}
\end{Issues}


\clearpage
%%%%%%%%%%%%%%%%%%%%%%%%%%%%%%%%%%%%%%%%%%%%%%%%%%%%%%%%%%
%% subsection 02.02.02 %%%%%%%%%%%%%%%%%%%%%%%%%%%%%%%%%%%
%%%%%%%%%%%%%%%%%%%%%%%%%%%%%%%%%%%%%%%%%%%%%%%%%%%%%%%%%%
\subsection{加工全般における品質}

\begin{Issues}{仕上げ前加工・仕上げ加工の分離の非統一}
一般にどの加工においても、削り代がより少ないほど仕上がりがきれいなものになる
\begin{enumerate}[label=\sarrow]
\item[{\sarrow[red]}]仕上げ前の加工と仕上げの加工とで、削り代が統一されていない
\end{enumerate}
\end{Issues}


\clearpage
%%%%%%%%%%%%%%%%%%%%%%%%%%%%%%%%%%%%%%%%%%%%%%%%%%%%%%%%%%
%% section 02.03 %%%%%%%%%%%%%%%%%%%%%%%%%%%%%%%%%%%%%%%%%
%%%%%%%%%%%%%%%%%%%%%%%%%%%%%%%%%%%%%%%%%%%%%%%%%%%%%%%%%%
\modHeadsection{\index{さぎょうこうりつ@作業効率}作業効率に関するイシュー・問題}


%%%%%%%%%%%%%%%%%%%%%%%%%%%%%%%%%%%%%%%%%%%%%%%%%%%%%%%%%%
%% subsection 02.03.01 %%%%%%%%%%%%%%%%%%%%%%%%%%%%%%%%%%%
%%%%%%%%%%%%%%%%%%%%%%%%%%%%%%%%%%%%%%%%%%%%%%%%%%%%%%%%%%
\subsection{\index{NCプログラム}NCプログラムの作成における作業効率}

\begin{Issues}{\KeywayCenter 位置の手動計算による導出}
\begin{enumerate}[label=\sarrow]
\item[{\sarrow[red]}]\KeywayCenter の$X$座標を、電卓により手作業で計算して導出している
\end{enumerate}
\end{Issues}

\begin{Issues}{\AsideKeywayDepth 指定時の\KeywayCenter 位置の関節的導出}
\begin{enumerate}[label=\sarrow]
\item[{\sarrow[red]}]\AsideKeywayDepth 指定時の\KeywayCenter 位置を、直接的な測定ではなく、端面部の外側中心を基準に関節的に導出している
\item[{\sarrow[red]}]\CenterCurvature に伴う誤差を含み、\AsideKeywayDepth が安定しない
\item[{\sarrow[red]}]位置調整(手入力)の頻度が必然的に多くなっている
\end{enumerate}
\end{Issues}

\begin{Issues}{\index{NCメインプログラム}NCメインプログラムの手動作成}
\begin{enumerate}[label=\sarrow]
\item[{\sarrow[red]}]\index{NCメインプログラム}NCメインプログラムが人手により手動で作成されている
\item[{\sarrow[red]}]\index{NCメインプログラム}NCメインプログラムのパターン化がなされていない
\item[{\sarrow[red]}]\index{プログラマ}プログラマとしての能力が必要な\index{NCメインプログラム}NCメインプログラムの作成業務が、(スタッフや管理職でなく)作業者に課されている
\item[{\sarrow[red]}]作業者に\index{プログラマ}プログラマとしての能力が問われ、\index{きょういくコスト@教育コスト}教育コストが多くかかっている
\end{enumerate}
\end{Issues}


%%%%%%%%%%%%%%%%%%%%%%%%%%%%%%%%%%%%%%%%%%%%%%%%%%%%%%%%%%
%% subsection 02.03.02 %%%%%%%%%%%%%%%%%%%%%%%%%%%%%%%%%%%
%%%%%%%%%%%%%%%%%%%%%%%%%%%%%%%%%%%%%%%%%%%%%%%%%%%%%%%%%%
\subsection{測定における作業効率}

\begin{Issues}{測定基準値の手動による実測}
\begin{enumerate}[label=\sarrow]
\item[{\sarrow[red]}]\index{ワークざひょうけいげんてん@ワーク座標系原点}ワーク座標系原点の概算値を見出すために、オペレータが直接機内で測定している
\item[{\sarrow[red]}]ジグ・ワークの寸法から机上で見出だせるにかかわらず、放置され続けている
\end{enumerate}
\end{Issues}

\begin{Issues}{測定時の送り速さ}
\begin{enumerate}[label=\sarrow]
\item[{\sarrow[red]}]すべての測定時の送り速さが1つの値で制御されている状態にある
\item[{\sarrow[red]}]すべての測定に対し、最も小さい送り速さに合わせる必要があり、必然的に測定に時間がかかってしまう
\end{enumerate}
\end{Issues}

\begin{Issues}{\CenterlineEndFaceDifMeasurement の作業効率}
\begin{enumerate}[label=\sarrow]
\item[{\sarrow[red]}]\CenterlineEndFaceDifMeasurement が、手動による\index{ハンドルそうさ@ハンドル操作}ハンドル操作で行われている
\item[{\sarrow[red]}]\Outcut・\Keyway・\CenterCurvature の寸法から自動化が可能にもかかわらず、事態が放置され続けている
\end{enumerate}
\end{Issues}


\clearpage
%%%%%%%%%%%%%%%%%%%%%%%%%%%%%%%%%%%%%%%%%%%%%%%%%%%%%%%%%%
%% subsection 02.03.03 %%%%%%%%%%%%%%%%%%%%%%%%%%%%%%%%%%%
%%%%%%%%%%%%%%%%%%%%%%%%%%%%%%%%%%%%%%%%%%%%%%%%%%%%%%%%%%
\subsection{\EndFacecutMilling における作業効率}

\begin{Issues}{\EndFacecutMilling の手動による\TDCorrection}
\begin{enumerate}[label=\sarrow]
\item[{\sarrow[red]}]\EndFacecutMilling の基準点が外側中心として与えられており、\index{めいさい@明細}明細に依存する\indexTDFaceMill\nameTDCorrection を行わなければならない
\item[{\sarrow[red]}]オペレータが手動で\indexTDFaceMill\nameTDCorrection 値を編集しなければならない状態にある
\end{enumerate}
\end{Issues}

\begin{Issues}{\EndFacecutMilling の手動による加工回数の変更}
\begin{enumerate}[label=\sarrow]
\item[{\sarrow[red]}]\EndFacecutMilling の回数を変更する際、オペレータが手動により直接メインプログラムを編集しなければならない状態にある
\end{enumerate}
\end{Issues}


%%%%%%%%%%%%%%%%%%%%%%%%%%%%%%%%%%%%%%%%%%%%%%%%%%%%%%%%%%
%% subsection 02.03.05 %%%%%%%%%%%%%%%%%%%%%%%%%%%%%%%%%%%
%%%%%%%%%%%%%%%%%%%%%%%%%%%%%%%%%%%%%%%%%%%%%%%%%%%%%%%%%%
\subsection{\KeywayMilling における作業効率}

\begin{Issues}{\KeywayMilling 回数($Z$方向)の手動による設定}
\begin{enumerate}[label=\sarrow]
\item[{\sarrow[red]}]\KeywayWidth・工具幅に応じた\KeywayMilling の回数を、作業者が決定しなければならない状態にある
\item[{\sarrow[red]}]\KeywayMilling の回数の変更は、作業者が\index{NCメインプログラム}NCメインプログラムを直接編集しなければならない状態にある
\end{enumerate}
\end{Issues}


%%%%%%%%%%%%%%%%%%%%%%%%%%%%%%%%%%%%%%%%%%%%%%%%%%%%%%%%%%
%% subsection 02.03.06 %%%%%%%%%%%%%%%%%%%%%%%%%%%%%%%%%%%
%%%%%%%%%%%%%%%%%%%%%%%%%%%%%%%%%%%%%%%%%%%%%%%%%%%%%%%%%%
\subsection{\EndFaceChamferMilling における作業効率}

\begin{Issues}{手作業による\EndFaceChamferMilling の加工}
\begin{enumerate}[label=\sarrow]
\item[{\sarrow[red]}]寸法の小さな\EndFaceChamferMilling が、手持ち研磨機を用いて手作業で加工されている
\item[{\sarrow[red]}]\EndFaceChamferMilling に関する解析計算がなされないまま放置され続けている
\end{enumerate}
\end{Issues}


%%%%%%%%%%%%%%%%%%%%%%%%%%%%%%%%%%%%%%%%%%%%%%%%%%%%%%%%%%
%% subsection 02.03.07 %%%%%%%%%%%%%%%%%%%%%%%%%%%%%%%%%%%
%%%%%%%%%%%%%%%%%%%%%%%%%%%%%%%%%%%%%%%%%%%%%%%%%%%%%%%%%%
\subsection{\index{とくしゅなかこう@特殊な加工}特殊な加工における作業効率}

\begin{Issues}{\CurvedOutcutMilling の\index{ハンドルそうさ@ハンドル操作}ハンドル操作による測定}
\begin{enumerate}[label=\sarrow]
\item[{\sarrow[red]}]手作業による\index{ハンドルそうさ@ハンドル操作}ハンドル操作により測定が行われている
\item[{\sarrow[red]}]\CurvedOutcutMilling に関する解析計算がなされないまま放置され続けている
\end{enumerate}
\end{Issues}

\begin{Issues}{\Keyway 8角形コーナーRの加工の未対応}
\begin{enumerate}[label=\sarrow]
\item[{\sarrow[red]}]8角形コーナーRの形状をした\KeywayMilling に対応できていない
\end{enumerate}
\end{Issues}


\clearpage
%%%%%%%%%%%%%%%%%%%%%%%%%%%%%%%%%%%%%%%%%%%%%%%%%%%%%%%%%%
%% subsection 02.03.08 %%%%%%%%%%%%%%%%%%%%%%%%%%%%%%%%%%%
%%%%%%%%%%%%%%%%%%%%%%%%%%%%%%%%%%%%%%%%%%%%%%%%%%%%%%%%%%
\subsection{加工全般における作業効率}

\begin{Issues}{\Table 回転による振分調整の未対応}
\begin{enumerate}[label=\sarrow]
\item[{\sarrow[red]}]振分調整が\Spacer を用いた方法でのみしかできない状態にある
\item[{\sarrow[red]}]再振分けを行う際に、必ず\Spacer の取付けまたは取外し作業を行わなければならない
\end{enumerate}
\end{Issues}

\begin{Issues}{メインプログラム編集による\index{おくりそくど@送り速度}送り速度の変更}
\begin{enumerate}[label=\sarrow]
\item[{\sarrow[red]}]\index{おくりそくど@送り速度}送り速度を変更するために、\index{NCメインプログラム}NCメインプログラムを直接編集しなければならない
\end{enumerate}
\end{Issues}

\begin{Issues}{メインプログラム編集による\index{しゅじくかいてんすう@主軸回転数}主軸回転数の変更}
\begin{enumerate}[label=\sarrow]
\item[{\sarrow[red]}]\index{しゅじくかいてんすう@主軸回転数}主軸回転数を変更するために、\index{NCメインプログラム}NCメインプログラムを直接編集しなければならない
\end{enumerate}
\end{Issues}


\clearpage
%%%%%%%%%%%%%%%%%%%%%%%%%%%%%%%%%%%%%%%%%%%%%%%%%%%%%%%%%%
%% section 02.03 %%%%%%%%%%%%%%%%%%%%%%%%%%%%%%%%%%%%%%%%%
%%%%%%%%%%%%%%%%%%%%%%%%%%%%%%%%%%%%%%%%%%%%%%%%%%%%%%%%%%
\modHeadsection{\index{しんらいせい@信頼性}信頼性に関するイシュー・問題}

\begin{Issues}{\index{だんきうんてん@暖機運転}暖機運転}
\begin{enumerate}[label=\sarrow]
\item[{\sarrow[red]}]十分な\index{だんきうんてん@暖機運転}暖機運転がなされていない
\end{enumerate}
\end{Issues}

\begin{Issues}{パラメタ誤入力の予防策}
\begin{enumerate}[label=\sarrow]
\item[{\sarrow[red]}]\index{NCメインプログラム}NCメインプログラム・\index{NCサブプログラム}NCサブプログラムともに、\index{ひきすう@引数}引数に対する誤入力の予防措置がなされていない
\item[{\sarrow[red]}]\index{NCメインプログラム}NCメインプログラム・\index{NCサブプログラム}NCサブプログラムともに、\index{コモンへんすう@コモン変数}コモン変数に対する誤入力の予防措置がなされていない
\end{enumerate}
\end{Issues}

\begin{Issues}{\KeywayMilling 時 \index{がいぶワークざひょうけい@外部ワーク座標系}外部ワーク座標系による衝突の予防策}
\begin{enumerate}[label=\sarrow]
\item[{\sarrow[red]}]\index{がいぶワークざひょうけい@外部ワーク座標系}外部ワーク座標系による衝突の予防措置がなされていない
\item[{\sarrow[red]}]特に影響の大きい\KeywayMilling にさえなされていない
\end{enumerate}
\end{Issues}



\clearpage
%%%%%%%%%%%%%%%%%%%%%%%%%%%%%%%%%%%%%%%%%%%%%%%%%%%%%%%%%%
%% section 02.04 %%%%%%%%%%%%%%%%%%%%%%%%%%%%%%%%%%%%%%%%%
%%%%%%%%%%%%%%%%%%%%%%%%%%%%%%%%%%%%%%%%%%%%%%%%%%%%%%%%%%
\modHeadsection{全般的な問題点\TBW}
そもそも、「\expandafterindex{\yomiDrawing(モールド)@\nameDrawing(モールド)}\nameDrawing を見て\index{NCプログラム}NCプログラムを作成」するという時点で、明らかに尋常な状態でないことは言うまでもない
%% footnote %%%%%%%%%%%%%%%%%%%%%
\footnote{さらに言及すると、これが問題点(改善可能な点)だということがこれまで一切(スタッフ・管理職を含めて)認識されていないことが大きな問題点として挙げられる。
つまり、(\index{せいぞうぎょう@製造業}製造業にも関わらず)\index{ソフトウェアエンジニアリング}ソフトウェアエンジニアリングをあからさまに蔑ろにし続けていることが根本に存在する。
\textbf{客観的事実として}、ソフトウェアエンジニアリングに対する意識の低さ、ならびにモラル・マナーの欠如が顕著に露呈している。
このような事態は、この業務に限らず社内のほぼ全ての業務において同様である。}。
その\expandafterindex{\yomiDrawing のさくせい)@\nameDrawing の作成}\nameDrawing の作成についても以下のような状態にある。
\begin{Issues}{\expandafterindex{\yomiDrawing のさくせい)@\nameDrawing の作成}\nameDrawing の作成}
\begin{enumerate}[label=\sarrow]
\item[{\sarrow[red]}] \expandafterindex{\yomiDrawing のさくせい)@\nameDrawing の作成}\nameDrawing の作成が、\index{CADソフトウェア}CADソフトウェアを用いて明細ごとに手動で描かれている
\end{enumerate}
\end{Issues}
これらは単に、モールドに関するデータが管理状態にないことを示している。
\begin{Issues}{データベースの非存在}
\begin{enumerate}[label=\sarrow]
\item[{\sarrow[red]}] \index{モールド}モールドに関する\index{データベース@データベース}データベースに相当するものが存在しない
\end{enumerate}
\end{Issues}


%%%%%%%%%%%%%%%%%%%%%%%%%%%%%%%%%%%%%%%%%%%%%%%%%%%%%%%%%%
%% subsection 02.03.01 %%%%%%%%%%%%%%%%%%%%%%%%%%%%%%%%%%%
%%%%%%%%%%%%%%%%%%%%%%%%%%%%%%%%%%%%%%%%%%%%%%%%%%%%%%%%%%
\subsection{\Drawing の作成における作業効率}

\begin{Issues}{\TBW}
\begin{enumerate}[label=\sarrow]
\item[{\sarrow[red]}]
\end{enumerate}
\end{Issues}
