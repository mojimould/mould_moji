%!TEX root = ../RfCPN.tex


\modHeadchapter[lot]{現状の横型マシニングセンタの業務フロー}
新たに導入する\index{よこがたマシニングセンタ@横型マシニングセンタ}横型マシニングセンタ(以下、\textbf{\DMC})での\expandafterindex{こうてい(\yomiDMC)@工程(\yomiDMC)}工程は、\Dimple の測定・加工を除けば三菱製横型マシニングセンタ(以下、\textbf{\MMC})と大まかには同様である。
そこで、まずは\MMC ではどのようなフローで業務が行われているかを(ソフトウェアの観点から)みることにする。
%%%%%%%%%%%%%%%%%%%%%%%%%%%%%%%%%%%%%%%%%%%%%%%%%%%%%%%%%%
%% marker %%%%%%%%%%%%%%%%%%%%%%%%%%%%%%%%%%%%%%%%%%%%%%%%
%%%%%%%%%%%%%%%%%%%%%%%%%%%%%%%%%%%%%%%%%%%%%%%%%%%%%%%%%%
\begin{marker}
ここでは主に\MMC の\expandafterindex{No.1パレット(\yomiMMC)@No.1パレット(\nameMMC)}No.1パレットで加工を行うものを対象とする
%% footnote %%%%%%%%%%%%%%%%%%%%%
\footnote{\expandafterindex{No.2パレット(\yomiMMC)@No.2パレット(\nameMMC)}No.2パレットでは、\index{おおがたのモールド@大型のモールド}径の大きなものや\index{まるがたのモールド@丸型のモールド}丸形のもの等の加工が主に行われる。}。
%%%%%%%%%%%%%%%%%%%%%%%%%%%%%%%%%
\end{marker}
%%%%%%%%%%%%%%%%%%%%%%%%%%%%%%%%%%%%%%%%%%%%%%%%%%%%%%%%%%
%%%%%%%%%%%%%%%%%%%%%%%%%%%%%%%%%%%%%%%%%%%%%%%%%%%%%%%%%%
%%%%%%%%%%%%%%%%%%%%%%%%%%%%%%%%%%%%%%%%%%%%%%%%%%%%%%%%%%



%%%%%%%%%%%%%%%%%%%%%%%%%%%%%%%%%%%%%%%%%%%%%%%%%%%%%%%%%%
%% section 1.1 %%%%%%%%%%%%%%%%%%%%%%%%%%%%%%%%%%%%%%%%%%%
%%%%%%%%%%%%%%%%%%%%%%%%%%%%%%%%%%%%%%%%%%%%%%%%%%%%%%%%%%
\modHeadsection{\MMC における\expandafterindex{こうてい(\yomiMMC)@工程(\nameMMC)}工程および\expandafterindex{しようツール(\yomiMMC)@使用ツール(\nameMMC)}使用ツール}


%%%%%%%%%%%%%%%%%%%%%%%%%%%%%%%%%%%%%%%%%%%%%%%%%%%%%%%%%%
%% subsection 01.01.1 %%%%%%%%%%%%%%%%%%%%%%%%%%%%%%%%%%%%
%%%%%%%%%%%%%%%%%%%%%%%%%%%%%%%%%%%%%%%%%%%%%%%%%%%%%%%%%%
\subsection{\expandafterindex{こうていのしゅるい(\yomiMMC)@工程の種類(\nameMMC)}工程の種類(\yomiMMC)}
\MMC について、直接的なワークに対する測定・加工に関するものに着目すると、主に以下のような\expandafterindex{こうてい(\yomiMMC)@工程(\nameMMC)}工程が行われる。\\

\begin{multicollongtblr}{ワークに直接関わる主な\expandafterindex{こうていのしゅるい(\yomiMMC)@工程の種類(\nameMMC)}工程の種類(\yomiMMC)}{X[l] X[l]}
測定 & 加工\\
\EndFace 外側AC方向 芯出し(外側 両側$X$測定) & \EndFacecutMilling\\
\EndFace 外側BD方向 芯出し(外側 両側$Y$測定) & \OutcutMilling\\
\OutcutWidth AC方向 芯出し(内側 片側$X$測定) & \KeywayMilling\\
\EndFace 内側AC方向 芯出し(内側 両側$X$測定) & \EndFaceOutCChamferMilling\\
\EndFace 内側BD方向 芯出し(内側 両側$Y$測定) & \EndFaceInCChamferMilling\\
\CenterlineEndFaceDifAC 測定(外側 片側$Z$測定) & \EndFaceBoringMilling\\
\CenterlineEndFaceDifBD 測定(外側 片側$Y$測定) & \IncutBoringMilling\\
\end{multicollongtblr}


\clearpage
%%%%%%%%%%%%%%%%%%%%%%%%%%%%%%%%%%%%%%%%%%%%%%%%%%%%%%%%%%
%% subsection 01.01.1 %%%%%%%%%%%%%%%%%%%%%%%%%%%%%%%%%%%%
%%%%%%%%%%%%%%%%%%%%%%%%%%%%%%%%%%%%%%%%%%%%%%%%%%%%%%%%%%
\subsection{\expandafterindex{しようツール(\yomiMMC)@使用ツール(\nameMMC)}使用ツール(\yomiMMC)}
ソフトウェアに着目すると、\MMC による加工では主に以下のようなツールが使用されている。\\

\begin{multicollongtblr}{\expandafterindex{しようソフトウェア(\yomiMMC)@使用ソフトウェア(\nameMMC)}使用ソフトウェアおよび\expandafterindex{しようツール(\yomiMMC)@使用ツール(\nameMMC)}ツール(\yomiMMC)}{l X[l]}
ツール & 主な用途\\
\MMC 操作盤 & \index{NCプログラム}NCプログラムや各種\index{へんすう(NCプログラム)@変数(NCプログラム)}変数の編集\\
\index{みつびしマシニングセンタむじんかシステム@三菱マシニングセンタ無人化システム}三菱マシニングセンタ無人化システム & \index{タッチセンサープローブ}タッチセンサープローブ測定\\
\expandafterindex{さくせいしたNCプログラム(\yomiMMC)@作成したNCプログラム(\nameMMC)}作成したNCプログラム & 各々の加工\\
振分調整用スペーサ計算プログラム & 振分調整用スペーサの選定 および \ReAlocationLength の決定\\
\index{かんすうでんたく@関数電卓}関数電卓 & 四則演算・根号の計算\\
\end{multicollongtblr}

\noindent
\MMC では、各々の\index{こうてい(しんだしそくてい)@工程(芯出し測定)}芯出し測定の工程では「\index{みつびしマシニングセンタむじんかシステム@三菱マシニングセンタ無人化システム}三菱マシニングセンタ無人化システム」を、各々の加工工程では当社の従業員(一般職)により作成された\index{サブプログラム}サブプログラムを用いている。
これらの\index{サブプログラム}サブプログラムを用いることにより、各々の\index{めいさい(モールド)@明細(モールド)}明細における必要な\index{すんぽう@寸法}寸法値等を\index{ひきすう(NCプログラム)@引数(NCプログラム)}引数に格納して用いればよい状態になっている。
%%%%%%%%%%%%%%%%%%%%%%%%%%%%%%%%%%%%%%%%%%%%%%%%%%%%%%%%%%
%% marker %%%%%%%%%%%%%%%%%%%%%%%%%%%%%%%%%%%%%%%%%%%%%%%%
%%%%%%%%%%%%%%%%%%%%%%%%%%%%%%%%%%%%%%%%%%%%%%%%%%%%%%%%%%
\begin{marker}
つまり、\MMC において「NCプログラムの作成」とは、主にこの具体的な\index{ひきすう(NCプログラム)@引数(NCプログラム)}引数等を計算することを意味している。
\end{marker}
%%%%%%%%%%%%%%%%%%%%%%%%%%%%%%%%%%%%%%%%%%%%%%%%%%%%%%%%%%
%%%%%%%%%%%%%%%%%%%%%%%%%%%%%%%%%%%%%%%%%%%%%%%%%%%%%%%%%%
%%%%%%%%%%%%%%%%%%%%%%%%%%%%%%%%%%%%%%%%%%%%%%%%%%%%%%%%%%



\clearpage
%%%%%%%%%%%%%%%%%%%%%%%%%%%%%%%%%%%%%%%%%%%%%%%%%%%%%%%%%%
%% section 1.3 %%%%%%%%%%%%%%%%%%%%%%%%%%%%%%%%%%%%%%%%%%%
%%%%%%%%%%%%%%%%%%%%%%%%%%%%%%%%%%%%%%%%%%%%%%%%%%%%%%%%%%
\modHeadsection{加工の流れ(加工前)}
\MMC において、ある\index{めいさい(モールド)@明細(モールド)}明細のワークを加工をする際に、以下のような流れで作業が行われる。
%%%%%%%%%%%%%%%%%%%%%%%%%%%%%%%%%%%%%%%%%%%%%%%%%%%%%%%%%%
%% marker %%%%%%%%%%%%%%%%%%%%%%%%%%%%%%%%%%%%%%%%%%%%%%%%
%%%%%%%%%%%%%%%%%%%%%%%%%%%%%%%%%%%%%%%%%%%%%%%%%%%%%%%%%%
\begin{marker}
ここで挙げている必要なパラメータ(\index{すんぽう@寸法}寸法)には、その\index{こうさ@公差}公差も考慮されているものとする。
\end{marker}
%%%%%%%%%%%%%%%%%%%%%%%%%%%%%%%%%%%%%%%%%%%%%%%%%%%%%%%%%%
%%%%%%%%%%%%%%%%%%%%%%%%%%%%%%%%%%%%%%%%%%%%%%%%%%%%%%%%%%
%%%%%%%%%%%%%%%%%%%%%%%%%%%%%%%%%%%%%%%%%%%%%%%%%%%%%%%%%%


%%%%%%%%%%%%%%%%%%%%%%%%%%%%%%%%%%%%%%%%%%%%%%%%%%%%%%%%%%
%% subsection 01.1.1 %%%%%%%%%%%%%%%%%%%%%%%%%%%%%%%%%%%%%
%%%%%%%%%%%%%%%%%%%%%%%%%%%%%%%%%%%%%%%%%%%%%%%%%%%%%%%%%%
\subsection{図面の確認}
\begin{enumerate}[label=\sarrow]
\item 対象となる明細の\index{ずめん(モールド)@図面(モールド)}図面を用意する
\item 他に内容が類似する明細の図面があれば、それも併せて用意する
\end{enumerate}
%%%%%%%%%%%%%%%%%%%%%%%%%%%%%%%%%%%%%%%%%%%%%%%%%%%%%%%%%%
%% PARAMETER %%%%%%%%%%%%%%%%%%%%%%%%%%%%%%%%%%%%%%%%%%%%%
%%%%%%%%%%%%%%%%%%%%%%%%%%%%%%%%%%%%%%%%%%%%%%%%%%%%%%%%%%
\begin{Parameter}{必要なパラメータ}
\PMDrawingExists%
\PMDrawingNumber%
\end{Parameter}
%%%%%%%%%%%%%%%%%%%%%%%%%%%%%%%%%%%%%%%%%%%%%%%%%%%%%%%%%%
%%%%%%%%%%%%%%%%%%%%%%%%%%%%%%%%%%%%%%%%%%%%%%%%%%%%%%%%%%
%%%%%%%%%%%%%%%%%%%%%%%%%%%%%%%%%%%%%%%%%%%%%%%%%%%%%%%%%%


%%%%%%%%%%%%%%%%%%%%%%%%%%%%%%%%%%%%%%%%%%%%%%%%%%%%%%%%%%
%% subsection 01.1.2 %%%%%%%%%%%%%%%%%%%%%%%%%%%%%%%%%%%%%
%%%%%%%%%%%%%%%%%%%%%%%%%%%%%%%%%%%%%%%%%%%%%%%%%%%%%%%%%%
\subsection{加工部分の有無の確認}

%%%%%%%%%%%%%%%%%%%%%%%%%%%%%%%%%%%%%%%%%%%%%%%%%%%%%%%%%%
%% subsubsection 01.1.2.2 %%%%%%%%%%%%%%%%%%%%%%%%%%%%%%%%
%%%%%%%%%%%%%%%%%%%%%%%%%%%%%%%%%%%%%%%%%%%%%%%%%%%%%%%%%%
\subsubsection{\EndFacecutMilling 部分}
\EndFacecutMilling については、全明細に共通の形で存在する。

%%%%%%%%%%%%%%%%%%%%%%%%%%%%%%%%%%%%%%%%%%%%%%%%%%%%%%%%%%
%% subsubsection 01.1.2.2 %%%%%%%%%%%%%%%%%%%%%%%%%%%%%%%%
%%%%%%%%%%%%%%%%%%%%%%%%%%%%%%%%%%%%%%%%%%%%%%%%%%%%%%%%%%
\subsubsection{\OutcutMilling 部分}
\OutcutMilling については、明細により\OutcutExists または\OutcutType の違いが存在する。
\begin{enumerate}[label=\sarrow]
\item \TopOutcutExists または\BottomOutcutExists を確認する
\item \TopOutcutType および\BottomOutcutType を確認し、使用する\expandafterindex{こうぐ(\yomiOutcut)@工具(\nameOutcut)}工具を決定する
\item \nameOutcut が\CurvedOutcut かどうかも確認する
\end{enumerate}
%\clearpage
%%%%%%%%%%%%%%%%%%%%%%%%%%%%%%%%%%%%%%%%%%%%%%%%%%%%%%%%%%
%% PARAMETER %%%%%%%%%%%%%%%%%%%%%%%%%%%%%%%%%%%%%%%%%%%%%
%%%%%%%%%%%%%%%%%%%%%%%%%%%%%%%%%%%%%%%%%%%%%%%%%%%%%%%%%%
\begin{Parameter}{必要なパラメータ}
\PMTopOutcutExists%
\PMBottomOutcutExists%
\PMTopOutcutType%
\PMBottomOutcutType%
\end{Parameter}
%%%%%%%%%%%%%%%%%%%%%%%%%%%%%%%%%%%%%%%%%%%%%%%%%%%%%%%%%%
%%%%%%%%%%%%%%%%%%%%%%%%%%%%%%%%%%%%%%%%%%%%%%%%%%%%%%%%%%
%%%%%%%%%%%%%%%%%%%%%%%%%%%%%%%%%%%%%%%%%%%%%%%%%%%%%%%%%%

%\clearpage
%%%%%%%%%%%%%%%%%%%%%%%%%%%%%%%%%%%%%%%%%%%%%%%%%%%%%%%%%%
%% subsubsection 01.1.2.3 %%%%%%%%%%%%%%%%%%%%%%%%%%%%%%%%
%%%%%%%%%%%%%%%%%%%%%%%%%%%%%%%%%%%%%%%%%%%%%%%%%%%%%%%%%%
\subsubsection{\KeywayMilling 部分}
\KeywayMilling については、全明細のトップ側に存在し、明細により種類の違いが存在する。
\begin{enumerate}[label=\sarrow]
\item \nameKeywayType を確認し、使用する\expandafterindex{サブプログラム(\yomiKeyway)@サブプログラム(\nameKeyway)}サブプログラムの判断を行う
\item \nameKeywayWidth を確認し、使用する\expandafterindex{こうぐ(\yomiKeyway)@工具(\nameKeyway)}工具の判断を行う
\end{enumerate}
%%%%%%%%%%%%%%%%%%%%%%%%%%%%%%%%%%%%%%%%%%%%%%%%%%%%%%%%%%
%% PARAMETER %%%%%%%%%%%%%%%%%%%%%%%%%%%%%%%%%%%%%%%%%%%%%
%%%%%%%%%%%%%%%%%%%%%%%%%%%%%%%%%%%%%%%%%%%%%%%%%%%%%%%%%%
\begin{Parameter}{必要なパラメータ}
\PMKeywayType%
\PMTopOutcutExists%
\PMKeywayWidth%
\end{Parameter}
%%%%%%%%%%%%%%%%%%%%%%%%%%%%%%%%%%%%%%%%%%%%%%%%%%%%%%%%%%
%%%%%%%%%%%%%%%%%%%%%%%%%%%%%%%%%%%%%%%%%%%%%%%%%%%%%%%%%%
%%%%%%%%%%%%%%%%%%%%%%%%%%%%%%%%%%%%%%%%%%%%%%%%%%%%%%%%%%

\clearpage
%%%%%%%%%%%%%%%%%%%%%%%%%%%%%%%%%%%%%%%%%%%%%%%%%%%%%%%%%%
%% subsubsection 01.1.2.4 %%%%%%%%%%%%%%%%%%%%%%%%%%%%%%%%
%%%%%%%%%%%%%%%%%%%%%%%%%%%%%%%%%%%%%%%%%%%%%%%%%%%%%%%%%%
\subsubsection{\EndFaceChamferMilling 部分}
\EndFaceChamferMilling については、全明細に存在し、明細により種類の違いが存在する。
\begin{enumerate}[label=\sarrow]
\item 種類が\EndFaceCChamfer であれば、その\EndFaceCChamferLength により\index{マシニングセンタ}マシニングセンタによる加工を行うか判断を行う
\item \EndFaceCChamferAngle を確認し、使用する\expandafterindex{こうぐ(\yomiEndFaceCChamfer)@工具(\nameEndFaceCChamfer)}工具を決定する
\end{enumerate}
%%%%%%%%%%%%%%%%%%%%%%%%%%%%%%%%%%%%%%%%%%%%%%%%%%%%%%%%%%
%% PARAMETER %%%%%%%%%%%%%%%%%%%%%%%%%%%%%%%%%%%%%%%%%%%%%
%%%%%%%%%%%%%%%%%%%%%%%%%%%%%%%%%%%%%%%%%%%%%%%%%%%%%%%%%%
\begin{Parameter}{必要なパラメータ}
\PMChamferType%
\PMTopEndFaceOutCChamferLength
\PMTopEndFaceOutCChamferAngle%
\PMTopOutcutExists\\
\PMBottomEndFaceOutCChamferLength%
\PMBottomEndFaceOutCChamferAngle%
\PMBottomOutcutExists%
\end{Parameter}
%%%%%%%%%%%%%%%%%%%%%%%%%%%%%%%%%%%%%%%%%%%%%%%%%%%%%%%%%%
%%%%%%%%%%%%%%%%%%%%%%%%%%%%%%%%%%%%%%%%%%%%%%%%%%%%%%%%%%
%%%%%%%%%%%%%%%%%%%%%%%%%%%%%%%%%%%%%%%%%%%%%%%%%%%%%%%%%%

%\clearpage
%%%%%%%%%%%%%%%%%%%%%%%%%%%%%%%%%%%%%%%%%%%%%%%%%%%%%%%%%%
%% subsubsection 01.1.2.4 %%%%%%%%%%%%%%%%%%%%%%%%%%%%%%%%
%%%%%%%%%%%%%%%%%%%%%%%%%%%%%%%%%%%%%%%%%%%%%%%%%%%%%%%%%%
\subsubsection{\EndFaceBoringMilling 部分\TBW}
(to be written...)
%%%%%%%%%%%%%%%%%%%%%%%%%%%%%%%%%%%%%%%%%%%%%%%%%%%%%%%%%%
%% PARAMETER %%%%%%%%%%%%%%%%%%%%%%%%%%%%%%%%%%%%%%%%%%%%%
%%%%%%%%%%%%%%%%%%%%%%%%%%%%%%%%%%%%%%%%%%%%%%%%%%%%%%%%%%
\begin{Parameter}{必要なパラメータ}
\PMEndFaceBoringExists%
\PMEndFaceBoringCornerR%
\end{Parameter}
%%%%%%%%%%%%%%%%%%%%%%%%%%%%%%%%%%%%%%%%%%%%%%%%%%%%%%%%%%
%%%%%%%%%%%%%%%%%%%%%%%%%%%%%%%%%%%%%%%%%%%%%%%%%%%%%%%%%%
%%%%%%%%%%%%%%%%%%%%%%%%%%%%%%%%%%%%%%%%%%%%%%%%%%%%%%%%%%

%\clearpage
%%%%%%%%%%%%%%%%%%%%%%%%%%%%%%%%%%%%%%%%%%%%%%%%%%%%%%%%%%
%% subsubsection 01.02.2.5 %%%%%%%%%%%%%%%%%%%%%%%%%%%%%%%
%%%%%%%%%%%%%%%%%%%%%%%%%%%%%%%%%%%%%%%%%%%%%%%%%%%%%%%%%%
\subsubsection{\IncutBoringMilling 部分\TBW}
(to be written...)
%%%%%%%%%%%%%%%%%%%%%%%%%%%%%%%%%%%%%%%%%%%%%%%%%%%%%%%%%%
%% PARAMETER %%%%%%%%%%%%%%%%%%%%%%%%%%%%%%%%%%%%%%%%%%%%%
%%%%%%%%%%%%%%%%%%%%%%%%%%%%%%%%%%%%%%%%%%%%%%%%%%%%%%%%%%
\begin{Parameter}{必要なパラメータ}
\PMIncutBoringExists%
\end{Parameter}
%%%%%%%%%%%%%%%%%%%%%%%%%%%%%%%%%%%%%%%%%%%%%%%%%%%%%%%%%%
%%%%%%%%%%%%%%%%%%%%%%%%%%%%%%%%%%%%%%%%%%%%%%%%%%%%%%%%%%
%%%%%%%%%%%%%%%%%%%%%%%%%%%%%%%%%%%%%%%%%%%%%%%%%%%%%%%%%%


\clearpage
%%%%%%%%%%%%%%%%%%%%%%%%%%%%%%%%%%%%%%%%%%%%%%%%%%%%%%%%%%
%% subsection 01.1.3 %%%%%%%%%%%%%%%%%%%%%%%%%%%%%%%%%%%%%
%%%%%%%%%%%%%%%%%%%%%%%%%%%%%%%%%%%%%%%%%%%%%%%%%%%%%%%%%%
\subsection{加工部分の寸法の確認}

%%%%%%%%%%%%%%%%%%%%%%%%%%%%%%%%%%%%%%%%%%%%%%%%%%%%%%%%%%
%% subsubsection 01.1.3.1 %%%%%%%%%%%%%%%%%%%%%%%%%%%%%%%%
%%%%%%%%%%%%%%%%%%%%%%%%%%%%%%%%%%%%%%%%%%%%%%%%%%%%%%%%%%
\subsubsection{\EndFacecutMilling における寸法}
\begin{enumerate}[label=\sarrow]
\item \expandafterindex{こうさ(\yomiWorkTotalLength)@公差(\nameWorkTotalLength)}\nameWorkTotalLength の公差を確認し、\TopAlocationLength および\BottomAlocationLength の\expandafterindex{こうさ(\yomiAlocationLength)@公差(\nameAlocationLength)}公差の判断を行う
\item \TopAlocationLength・\BottomAlocationLength を確認し、\index{スペーサ}スペーサによる調整が必要か判断を行う
\item 使用するスペーサおよび\ReAlocationLength は、専用の計算\index{プログラム(Excel VBA)}プログラム(\index{Excel VBA}Excel VBA)を用いて決定する
\item \OuterDiameter・\expandafterindex{\yomiThickness(\yomiEndFace)@\nameThickness(\nameEndFace)}\nameEndFace 部の\nameThickness・\expandafterindex{\yomiEndFace そとがわコーナーR@\nameEndFace 外側コーナーR}\nameEndFace 外側コーナーRを確認し、それに応じて\index{こうぐけいほせいち@工具径補正値}工具径補正値を決定する
\item \ReAlocationLength に応じて、$Z$方向の\index{クリアランスへいめん(Zほうこう)@クリアランス平面($Z$方向)}クリアランス平面の位置を決定する
\end{enumerate}
%%%%%%%%%%%%%%%%%%%%%%%%%%%%%%%%%%%%%%%%%%%%%%%%%%%%%%%%%%
%% PARAMETER %%%%%%%%%%%%%%%%%%%%%%%%%%%%%%%%%%%%%%%%%%%%%
%%%%%%%%%%%%%%%%%%%%%%%%%%%%%%%%%%%%%%%%%%%%%%%%%%%%%%%%%%
\begin{Parameter}{必要なパラメータ}
\paragraph*{\ReAlocationLength}
\PMWorkTotalLength
\PMTopAlocationLength
\PMBottomAlocationLength
\PMACOD
\PMJigLength
\GPMbox{受板の幅}
\tcbline*
\paragraph*{\TopEndFacecut}
\PMTopReAlocationLength
\PMACOD
\PMBDOD
\PMODCornerR\\
\PMTopEndACID
\PMTopEndBDID
\GPMbox{トップ側$Z$方向クリアランス平面距離}
\tcbline*
\paragraph*{\BottomEndFacecut}
\PMBottomReAlocationLength
\PMACOD
\PMBDOD
\PMODCornerR\\
\PMBottomEndACID
\PMBottomEndBDID
\GPMbox{ボトム側$Z$方向クリアランス平面距離}
\tcbline*
\paragraph*{\TopEndFacecut・\BottomEndFacecut~共通}
\GPMbox{\EndFacecutMilling 用工具径補正量}
\end{Parameter}
%%%%%%%%%%%%%%%%%%%%%%%%%%%%%%%%%%%%%%%%%%%%%%%%%%%%%%%%%%
%%%%%%%%%%%%%%%%%%%%%%%%%%%%%%%%%%%%%%%%%%%%%%%%%%%%%%%%%%
%%%%%%%%%%%%%%%%%%%%%%%%%%%%%%%%%%%%%%%%%%%%%%%%%%%%%%%%%%

\clearpage
%%%%%%%%%%%%%%%%%%%%%%%%%%%%%%%%%%%%%%%%%%%%%%%%%%%%%%%%%%
%% subsubsection 01.1.3.2 %%%%%%%%%%%%%%%%%%%%%%%%%%%%%%%%
%%%%%%%%%%%%%%%%%%%%%%%%%%%%%%%%%%%%%%%%%%%%%%%%%%%%%%%%%%
\subsubsection{\OutcutMilling における寸法}
\begin{enumerate}[label=\sarrow]
\item \OutcutLength と\Keyway の位置を確認し、実際に加工する\OutcutLength の判断を行う
\item トップ側・ボトム側の両方に\Outcut のある場合は、どちら側が\expandafterindex{きじゅん(\yomiOutcutCenter)@基準(\nameOutcutCenter)}基準であるのかを確認する
\item \EndFaceID・\OutcutAsideThickness・内面の\PlatingThk から、\OutcutCenter$X$位置用のパラメータを手動による計算で決定する
\item \CurvedOutcut の場合は、\expandafterindex{かたむきかく(\yomiCurvedOutcut)@傾き角(\nameCurvedOutcut)}傾き角を手動による計算で決定する
\end{enumerate}
%%%%%%%%%%%%%%%%%%%%%%%%%%%%%%%%%%%%%%%%%%%%%%%%%%%%%%%%%%
%% PARAMETER %%%%%%%%%%%%%%%%%%%%%%%%%%%%%%%%%%%%%%%%%%%%%
%%%%%%%%%%%%%%%%%%%%%%%%%%%%%%%%%%%%%%%%%%%%%%%%%%%%%%%%%%
\begin{Parameter}{必要なパラメータ}
\paragraph*{ボトム側の\Outcut のみの場合}
\PMBottomOutcutACwidth
\PMBottomOutcutBDwidth
\PMBottomOutcutConerR
\PMBottomOutcutLength\\
\PMBottomEndACID
\PMBottomOutcutAsideThickness
\PMPlatingThk
\tcbline*
\paragraph*{トップ側の\Outcut のみの場合}
\PMTopOutcutACwidth
\PMTopOutcutBDwidth
\PMTopOutcutCornerR\\
\PMTopOutcutLength
\PMKeywayPos
\PMKeywayWidth\\
\PMTopEndACID
\PMTopOutcutAsideThickness
\PMPlatingThk
\tcbline*
\paragraph*{両方に\Outcut があり、ボトム側が基準の場合}
\PMBottomOutcutACwidth
\PMBottomOutcutBDwidth
\PMBottomOutcutConerR
\PMBottomOutcutLength\\
\PMBottomEndACID
\PMBottomOutcutAsideThickness
\PMPlatingThk\\
\PMTopOutcutACwidth
\PMTopOutcutBDwidth
\PMTopOutcutCornerR
\PMTopOutcutLength\\
\PMKeywayPos
\PMKeywayWidth
\PMCenterlineEndFaceDifAC
\tcbline*
\paragraph*{両方に\Outcut があり、トップ側が基準の場合}
\PMTopOutcutACwidth
\PMTopOutcutBDwidth
\PMTopOutcutCornerR\\
\PMTopOutcutLength
\PMKeywayPos
\PMKeywayWidth\\
\PMTopEndACID
\PMTopOutcutAsideThickness
\PMPlatingThk\\
\PMBottomOutcutACwidth
\PMBottomOutcutBDwidth
\PMBottomOutcutConerR
\PMBottomOutcutLength\\
\PMCenterlineEndFaceDifAC
\tcbline*
\paragraph*{\CurvedOutcut の場合}
(以上に加えて)\PMCenterCurvatureRadius
\end{Parameter}
%%%%%%%%%%%%%%%%%%%%%%%%%%%%%%%%%%%%%%%%%%%%%%%%%%%%%%%%%%
%%%%%%%%%%%%%%%%%%%%%%%%%%%%%%%%%%%%%%%%%%%%%%%%%%%%%%%%%%
%%%%%%%%%%%%%%%%%%%%%%%%%%%%%%%%%%%%%%%%%%%%%%%%%%%%%%%%%%

\clearpage
%%%%%%%%%%%%%%%%%%%%%%%%%%%%%%%%%%%%%%%%%%%%%%%%%%%%%%%%%%
%% subsubsection 01.1.3.3 %%%%%%%%%%%%%%%%%%%%%%%%%%%%%%%%
%%%%%%%%%%%%%%%%%%%%%%%%%%%%%%%%%%%%%%%%%%%%%%%%%%%%%%%%%%
\subsubsection{\KeywayMilling における寸法}
\begin{enumerate}[label=\sarrow]
\item \KeywayType を確認し、必要に応じて加工における径の決定する
\item \expandafterindex{きじゅん(\yomiKeywayCenter)@基準(\nameKeywayCenter)}\nameKeywayCenter の基準を確認し、\KeywayCenter の$X$位置を手動で計算し、決定する
\item \KeywayWidth を確認し、\expandafterindex{かこうかいすう(\yomiKeywayWidth)@加工回数(\nameKeywayWidth)}加工の回数を決定する
\end{enumerate}
%%%%%%%%%%%%%%%%%%%%%%%%%%%%%%%%%%%%%%%%%%%%%%%%%%%%%%%%%%
%% PARAMETER %%%%%%%%%%%%%%%%%%%%%%%%%%%%%%%%%%%%%%%%%%%%%
%%%%%%%%%%%%%%%%%%%%%%%%%%%%%%%%%%%%%%%%%%%%%%%%%%%%%%%%%%
\begin{Parameter}{必要なパラメータ}
\paragraph*{\CenterCurvatureLine が基準の場合}
\PMKeywayACOD
\PMKeywayBDOD
\PMKeywayPos
\PMKeywayWidth
\PMCenterCurvatureRadius\\
\PMKeywayCornerR または\PMKeywayCornerC
\tcbline*
\paragraph*{\OutcutCenter が基準の場合}
\PMKeywayACOD
\PMKeywayBDOD
\PMKeywayPos
\PMKeywayWidth\\
\PMKeywayCornerR または\PMKeywayCornerC
\tcbline*
\paragraph*{\AsideKeywayDepth が基準の場合}
(以上に加えて)\PMAsideKeywayDepth
\end{Parameter}
%%%%%%%%%%%%%%%%%%%%%%%%%%%%%%%%%%%%%%%%%%%%%%%%%%%%%%%%%%
%%%%%%%%%%%%%%%%%%%%%%%%%%%%%%%%%%%%%%%%%%%%%%%%%%%%%%%%%%
%%%%%%%%%%%%%%%%%%%%%%%%%%%%%%%%%%%%%%%%%%%%%%%%%%%%%%%%%%

\clearpage
%%%%%%%%%%%%%%%%%%%%%%%%%%%%%%%%%%%%%%%%%%%%%%%%%%%%%%%%%%
%% subsubsection 01.1.3.4 %%%%%%%%%%%%%%%%%%%%%%%%%%%%%%%%
%%%%%%%%%%%%%%%%%%%%%%%%%%%%%%%%%%%%%%%%%%%%%%%%%%%%%%%%%%
\subsubsection{\EndFaceChamferMilling における寸法}
\begin{enumerate}[label=\sarrow]
\item \EndFaceOutCChamfer の場合は、\OutcutExists を確認し、加工の径を決定する
\item \expandafterindex{Cめんとり(\yomiOutcut ようこうぐせんたん)@C面取(\nameOutcut 用工具先端)}\nameOutcut 用工具先端部がC面取の場合は、\expandafterindex{\yomiOutcut のけいじょう@\nameOutcut の形状}\nameOutcut の形状を確認し、\expandafterindex{こうぐ(\yomiOutcut)@工具(\nameOutcut)}工具を決定する
\end{enumerate}
%%%%%%%%%%%%%%%%%%%%%%%%%%%%%%%%%%%%%%%%%%%%%%%%%%%%%%%%%%
%% PARAMETER %%%%%%%%%%%%%%%%%%%%%%%%%%%%%%%%%%%%%%%%%%%%%
%%%%%%%%%%%%%%%%%%%%%%%%%%%%%%%%%%%%%%%%%%%%%%%%%%%%%%%%%%
\begin{Parameter}{必要なパラメータ}
\paragraph*{トップ\EndFaceOutCChamfer:\Outcut のない場合}
\PMACOD
\PMBDOD
\PMTopEndFaceOutCChamferLength
\PMODCornerR
\tcbline*
\paragraph*{トップ\EndFaceOutCChamfer:\Outcut のある場合}
\PMTopOutcutACwidth
\PMTopOutcutBDwidth
\PMTopOutcutCornerR
\PMTopEndFaceOutCChamferLength
\tcbline*
\paragraph*{ボトム\EndFaceOutCChamfer:\Outcut のない場合}
\PMACOD
\PMBDOD
\PMBottomEndFaceOutCChamferLength
\PMODCornerR
\tcbline*
\paragraph*{ボトム\EndFaceOutCChamfer:\Outcut のある場合}
\PMBottomOutcutACwidth
\PMBottomOutcutBDwidth
\PMBottomOutcutConerR
\PMBottomEndFaceOutCChamferLength
\tcbline*
\paragraph*{トップ\EndFaceInCChamfer}
\PMTopEndACID
\PMTopEndBDID
\PMTopEndIDCornerR\\
\PMTopEndFaceInCChamferLength
\PMPlatingThk
\tcbline*
\paragraph*{ボトム\EndFaceInCChamfer}
\PMBottomEndACID
\PMBottomEndBDID
\PMBottomEndIDCornerR\\
\PMBottomEndFaceInCChamferLength
\PMPlatingThk
\end{Parameter}
%%%%%%%%%%%%%%%%%%%%%%%%%%%%%%%%%%%%%%%%%%%%%%%%%%%%%%%%%%
%%%%%%%%%%%%%%%%%%%%%%%%%%%%%%%%%%%%%%%%%%%%%%%%%%%%%%%%%%
%%%%%%%%%%%%%%%%%%%%%%%%%%%%%%%%%%%%%%%%%%%%%%%%%%%%%%%%%%

%\clearpage
%%%%%%%%%%%%%%%%%%%%%%%%%%%%%%%%%%%%%%%%%%%%%%%%%%%%%%%%%%
%% subsubsection 01.1.3.4 %%%%%%%%%%%%%%%%%%%%%%%%%%%%%%%%
%%%%%%%%%%%%%%%%%%%%%%%%%%%%%%%%%%%%%%%%%%%%%%%%%%%%%%%%%%
\subsubsection{\EndFaceBoringMilling における寸法\TBW}
(to be written...)
%%%%%%%%%%%%%%%%%%%%%%%%%%%%%%%%%%%%%%%%%%%%%%%%%%%%%%%%%%
%% PARAMETER %%%%%%%%%%%%%%%%%%%%%%%%%%%%%%%%%%%%%%%%%%%%%
%%%%%%%%%%%%%%%%%%%%%%%%%%%%%%%%%%%%%%%%%%%%%%%%%%%%%%%%%%
\begin{Parameter}{必要なパラメータ}
\PMEndFaceBoringWidth
\PMEndFaceBoringCornerR
\PMEndFaceBoringDepth
\PMEndFaceBoringLength\\
\PMACOD
\PMBDOD
\end{Parameter}
%%%%%%%%%%%%%%%%%%%%%%%%%%%%%%%%%%%%%%%%%%%%%%%%%%%%%%%%%%
%%%%%%%%%%%%%%%%%%%%%%%%%%%%%%%%%%%%%%%%%%%%%%%%%%%%%%%%%%
%%%%%%%%%%%%%%%%%%%%%%%%%%%%%%%%%%%%%%%%%%%%%%%%%%%%%%%%%%

%\clearpage
%%%%%%%%%%%%%%%%%%%%%%%%%%%%%%%%%%%%%%%%%%%%%%%%%%%%%%%%%%
%% subsubsection 01.02.3.5 %%%%%%%%%%%%%%%%%%%%%%%%%%%%%%%
%%%%%%%%%%%%%%%%%%%%%%%%%%%%%%%%%%%%%%%%%%%%%%%%%%%%%%%%%%
\subsubsection{\IncutBoringMilling における寸法\TBW}
(to be written...)
%%%%%%%%%%%%%%%%%%%%%%%%%%%%%%%%%%%%%%%%%%%%%%%%%%%%%%%%%%
%% PARAMETER %%%%%%%%%%%%%%%%%%%%%%%%%%%%%%%%%%%%%%%%%%%%%
%%%%%%%%%%%%%%%%%%%%%%%%%%%%%%%%%%%%%%%%%%%%%%%%%%%%%%%%%%
\begin{Parameter}{必要なパラメータ}
\end{Parameter}
%%%%%%%%%%%%%%%%%%%%%%%%%%%%%%%%%%%%%%%%%%%%%%%%%%%%%%%%%%
%%%%%%%%%%%%%%%%%%%%%%%%%%%%%%%%%%%%%%%%%%%%%%%%%%%%%%%%%%
%%%%%%%%%%%%%%%%%%%%%%%%%%%%%%%%%%%%%%%%%%%%%%%%%%%%%%%%%%


\clearpage
%%%%%%%%%%%%%%%%%%%%%%%%%%%%%%%%%%%%%%%%%%%%%%%%%%%%%%%%%%
%% subsection 01.1.4 %%%%%%%%%%%%%%%%%%%%%%%%%%%%%%%%%%%%%
%%%%%%%%%%%%%%%%%%%%%%%%%%%%%%%%%%%%%%%%%%%%%%%%%%%%%%%%%%
\subsection{NCプログラムの入力}
\begin{enumerate}[label=\sarrow]
\item 原則として、\index{プログラムばんごう@プログラム番号}プログラム番号は\DrawingExists と一致させる\\
ただし、加工内容が同一のものである場合は、既存のNCプログラムをそのまま流用する
\item 各々の加工部分およびその形状に対する\index{サブプログラム}サブプログラムを決定する
\item 各々のサブプログラムに対し、適切な寸法値を手動で計算する
\item 各々のサブプログラムに対し、計算した寸法値・\index{こうぐばんごう@工具番号}工具番号・\index{おくりはやさ@送り速さ}送り速さ・\index{しゅじくかいてんすう@主軸回転数}主軸回転数を格納する
\item \ReAlocationLength の寸法に応じて、\index{クリアランスへいめん(Zほうこう)@クリアランス平面($Z$方向)}$Z$方向クリアランス平面の位置を決定する
\item マシニングセンタの操作画面にて\index{メインプログラム}メインプログラムを直接編集し、必要なコードまたは数値を記入する
\item 必要に応じて、\index{いちじていし(NCプログラム)@一時停止(NCプログラム)}一時停止用のコードを代入する
\item \index{こうぐけいほせい@工具径補正}工具径または\index{こうぐちょうほせい@工具長補正}工具長の補正が必要な場合は、別途専用画面にて手動で編集を行う
\end{enumerate}


%\clearpage
%%%%%%%%%%%%%%%%%%%%%%%%%%%%%%%%%%%%%%%%%%%%%%%%%%%%%%%%%%
%% subsection 01.1.4 %%%%%%%%%%%%%%%%%%%%%%%%%%%%%%%%%%%%%
%%%%%%%%%%%%%%%%%%%%%%%%%%%%%%%%%%%%%%%%%%%%%%%%%%%%%%%%%%
\subsection{ワークの設置}
\begin{enumerate}[label=\sarrow]
\item \index{スペーサ}スペーサが必要な場合は、適切なスペーサを\index{ジグ}ジグの\index{うけいた@受板}受板に設置する
\item \index{ワーク}ワークの大きさを考慮して、\index{ワークこていようボルト@ワーク固定用ボルト}ワーク固定用ボルトの長さを目分量で適宜決定し、ジグに設置する
\item \ReAlocationLength に応じた位置に\index{ワーク}ワークを設置し、固定する
\item トップ側およびボトム側の、ジグからの\index{はりだしちょう@張出長}張出長を\index{メジャー}メジャーを用いて測定する
\item 測定した張出長から、\EndFacecutMilling における\index{ぜんけずりしろ(\yomiEndFacecut)@全削り代(\nameEndFacecut)}全削り代を手動でおおまかに計算する
\end{enumerate}
%%%%%%%%%%%%%%%%%%%%%%%%%%%%%%%%%%%%%%%%%%%%%%%%%%%%%%%%%%
%% PARAMETER %%%%%%%%%%%%%%%%%%%%%%%%%%%%%%%%%%%%%%%%%%%%%
%%%%%%%%%%%%%%%%%%%%%%%%%%%%%%%%%%%%%%%%%%%%%%%%%%%%%%%%%%
\begin{Parameter}{必要なパラメータ}
\paragraph*{ワーク固定用ボルト}
\PMACOD
\PMBDOD\\
\GPMbox{ジグ床面とボルト取付具(上)間の距離}
\GPMbox{受板とボルト取付具(横)間の距離}
\tcbline*
\paragraph*{\EndFacecut の削り代}
\PMJigLength
\PMTopReAlocationLength
\PMBottomReAlocationLength
\GPMbox{\nameEndFacecutMilling 1回あたりの削り代}\\
\GPMbox{トップ側張出長実測値}
\GPMbox{ボトム側張出長実測値}
\end{Parameter}
%%%%%%%%%%%%%%%%%%%%%%%%%%%%%%%%%%%%%%%%%%%%%%%%%%%%%%%%%%
%%%%%%%%%%%%%%%%%%%%%%%%%%%%%%%%%%%%%%%%%%%%%%%%%%%%%%%%%%
%%%%%%%%%%%%%%%%%%%%%%%%%%%%%%%%%%%%%%%%%%%%%%%%%%%%%%%%%%


\clearpage
%%%%%%%%%%%%%%%%%%%%%%%%%%%%%%%%%%%%%%%%%%%%%%%%%%%%%%%%%%
%% subsection 01.1.5 %%%%%%%%%%%%%%%%%%%%%%%%%%%%%%%%%%%%%
%%%%%%%%%%%%%%%%%%%%%%%%%%%%%%%%%%%%%%%%%%%%%%%%%%%%%%%%%%
\subsection{ワーク設置後の調整}
\begin{enumerate}[label=\sarrow]
\item トップ側およびボトム側の\expandafterindex{ぜんけずりしろ(\yomiEndFacecut)@全削り代(\nameEndFacecut)}全削り代に応じて、\index{かこうかいすう(\yomiEndFacecutMilling)@加工回数(\nameEndFacecutMilling)}\nameEndFacecutMilling の回数を設定する
\item \TopODCenter および\BottomODCenter の位置を\index{メジャー}メジャーで測定する
\item 測定した中心位置を用いて、\index{ワークざひょうけいげんてん@ワーク座標系原点}ワーク座標系原点の設定を行う
\item \expandafterindex{テーブルのかいてんちゅうしん(\yomiMMC)@テーブルの回転中心(\nameMMC)}テーブルの回転中心とのずれを考慮して、\EndFace の$Z$方向の長さを調整する
\item \CenterlineEndFaceDif がある場合\expandafterindex{テーブルのかいてんちゅうしん(\yomiMMC)@テーブルの回転中心(\nameMMC)}テーブルの回転中心とのずれを考慮して、\OutcutCenter の$X$方向の位置を調整する
\end{enumerate}
これらの設定は、\MMC の操作盤から\index{メインプログラム}メインプログラムを直接手動で編集する形で行われる。



\clearpage
%%%%%%%%%%%%%%%%%%%%%%%%%%%%%%%%%%%%%%%%%%%%%%%%%%%%%%%%%%
%% section 1.2 %%%%%%%%%%%%%%%%%%%%%%%%%%%%%%%%%%%%%%%%%%%
%%%%%%%%%%%%%%%%%%%%%%%%%%%%%%%%%%%%%%%%%%%%%%%%%%%%%%%%%%
\modHeadsection{\expandafterindex{こうてい(\yomiMMC)@工程(\nameMMC)}\nameMMC における工程(加工中)}


%%%%%%%%%%%%%%%%%%%%%%%%%%%%%%%%%%%%%%%%%%%%%%%%%%%%%%%%%%
%% subsection 01.2.1 %%%%%%%%%%%%%%%%%%%%%%%%%%%%%%%%%%%%%
%%%%%%%%%%%%%%%%%%%%%%%%%%%%%%%%%%%%%%%%%%%%%%%%%%%%%%%%%%
\subsection{芯出し測定後}
\begin{enumerate}[label=\sarrow]
\item 各々の\index{ワークざひょうけいげんてん@ワーク座標系原点}ワーク座標系原点の測定後、必要に応じてワーク座標系原点の設定変更を行う
\item 各々の測定箇所の$Z$位置の変更を、必要に応じて行う
\end{enumerate}
これらの設定は、\index{マシニングセンタ}マシニングセンタの操作盤から\index{メインプログラム}メインプログラムを直接手動で編集する形で行われる。


%%%%%%%%%%%%%%%%%%%%%%%%%%%%%%%%%%%%%%%%%%%%%%%%%%%%%%%%%%
%% subsection 01.2.1 %%%%%%%%%%%%%%%%%%%%%%%%%%%%%%%%%%%%%
%%%%%%%%%%%%%%%%%%%%%%%%%%%%%%%%%%%%%%%%%%%%%%%%%%%%%%%%%%
\subsection{\EndFacecutMilling 中}
\begin{enumerate}[label=\sarrow]
\item 必要に応じて、\expandafterindex{1かいあたりのけずりしろ(\yomiEndFacecut)@1回あたりの削り代(\nameEndFacecut)}1回あたりの削り代を調整する
\end{enumerate}


%%%%%%%%%%%%%%%%%%%%%%%%%%%%%%%%%%%%%%%%%%%%%%%%%%%%%%%%%%
%% subsection 01.2.1 %%%%%%%%%%%%%%%%%%%%%%%%%%%%%%%%%%%%%
%%%%%%%%%%%%%%%%%%%%%%%%%%%%%%%%%%%%%%%%%%%%%%%%%%%%%%%%%%
\subsection{\OutcutMilling 中}
\begin{enumerate}[label=\sarrow]
\item 必要に応じて\expandafterindex{しあげかこう(\yomiOutcut)@仕上げ加工(\nameOutcut)}仕上加工前に\index{いちじていし(NCプログラム)@一時停止(NCプログラム)}一時停止を行い、\OutcutAsideThickness および\OutcutWidth の測定を行う
\item \OutcutAsideThickness を調整する場合は、該当する\expandafterindex{しんだしそくてい(\yomiOutcutCenter)@芯出し測定(\nameOutcutCenter)}芯出し測定部分のパラメータをメインプログラムから直接手動で編集する
\item \OutcutWidth を調整する場合は、該当する加工部分のパラメータを\MMC の操作盤から\index{メインプログラム}メインプログラムを直接手動で編集する
\item \expandafterindex{かこうかいすう(\yomiOutcut)@加工回数(\nameOutcut)}加工の回数を変更する場合は、該当する加工部分をマシニングセンタの操作盤からメインプログラムを直接手動で編集する
\end{enumerate}


%%%%%%%%%%%%%%%%%%%%%%%%%%%%%%%%%%%%%%%%%%%%%%%%%%%%%%%%%%
%% subsection 01.2.1 %%%%%%%%%%%%%%%%%%%%%%%%%%%%%%%%%%%%%
%%%%%%%%%%%%%%%%%%%%%%%%%%%%%%%%%%%%%%%%%%%%%%%%%%%%%%%%%%
\subsection{\KeywayMilling 中}
\begin{enumerate}[label=\sarrow]
\item 必要に応じて\expandafterindex{しあげかこう(\yomiKeyway)@仕上げ加工(\nameKeyway)}仕上加工前に\index{いちじていし(NCプログラム)@一時停止(NCプログラム)}一時停止を行い、\AsideKeywayDepth および\KeywayDiameter の測定を行う
\item \AsideKeywayDepth を調整する場合は、該当する\expandafterindex{しんだしそくてい(\yomiKeywayCenter)@芯出し測定(\nameKeywayCenter)}芯出し測定部分のパラメータをマシニングセンタの操作画面からメインプログラムを直接手動で編集する
\item \KeywayDiameter を調整する場合は、該当する加工部分のパラメータをマシニングセンタの操作画面からメインプログラムを直接手動で編集する
\item \expandafterindex{かこうかいすう(\yomiKeyway)@加工回数(\nameKeyway)}加工の回数を変更する場合は、該当する加工部分をマシニングセンタの操作画面からメインプログラムを直接手動で編集する
\item 必要に応じて、\index{ブロックゲージ}ブロックゲージによる\KeywayWidth の測定を行う
\end{enumerate}


%%%%%%%%%%%%%%%%%%%%%%%%%%%%%%%%%%%%%%%%%%%%%%%%%%%%%%%%%%
%% subsection 01.2.1 %%%%%%%%%%%%%%%%%%%%%%%%%%%%%%%%%%%%%
%%%%%%%%%%%%%%%%%%%%%%%%%%%%%%%%%%%%%%%%%%%%%%%%%%%%%%%%%%
\subsection{\EndFaceCChamferMilling 中}
\begin{enumerate}[label=\sarrow]
\item 必要に応じて\expandafterindex{しあげかこう(\yomiEndFaceOutCChamfer)@仕上げ加工(\nameEndFaceOutCChamfer)}仕上加工前に\index{いちじていし(NCプログラム)@一時停止(NCプログラム)}一時停止を行い、\EndFaceOutCChamfer の測定・位置の確認を行う
\item \EndFaceOutCChamfer の位置を調整する場合は、該当する加工部分のパラメータをマシニングセンタの操作画面からメインプログラムを直接手動で編集する
\item \expandafterindex{かこうかいすう(\yomiEndFaceOutCChamfer)@加工回数(\nameEndFaceOutCChamfer)}加工の回数を変更する場合は、該当する加工部分をマシニングセンタの操作画面からメインプログラムを直接手動で編集する
\end{enumerate}


\clearpage
%%%%%%%%%%%%%%%%%%%%%%%%%%%%%%%%%%%%%%%%%%%%%%%%%%%%%%%%%%
%% subsection 01.2.1 %%%%%%%%%%%%%%%%%%%%%%%%%%%%%%%%%%%%%
%%%%%%%%%%%%%%%%%%%%%%%%%%%%%%%%%%%%%%%%%%%%%%%%%%%%%%%%%%
\subsection{\EndFaceInCChamferMilling 中}
\begin{enumerate}[label=\sarrow]
\item 必要に応じて\expandafterindex{しあげかこう(\yomiEndFaceInCChamfer)@仕上げ加工(\nameEndFaceInCChamfer)}仕上加工前に\index{いちじていし(NCプログラム)@一時停止(NCプログラム)}一時停止を行い、\EndFaceInCChamfer の状態の確認を行う
\item \EndFaceInCChamfer の位置を調整する場合は、該当する加工部分のパラメータをマシニングセンタの操作画面からメインプログラムを直接手動で編集する
\item \expandafterindex{かこうかいすう(\yomiEndFaceInCChamfer)@加工回数(\nameEndFaceInCChamfer)}加工の回数を変更する場合は、該当する加工部分をマシニングセンタの操作画面からメインプログラムを直接手動で編集する
\end{enumerate}


%\clearpage
%%%%%%%%%%%%%%%%%%%%%%%%%%%%%%%%%%%%%%%%%%%%%%%%%%%%%%%%%%
%% subsection 01.2.1 %%%%%%%%%%%%%%%%%%%%%%%%%%%%%%%%%%%%%
%%%%%%%%%%%%%%%%%%%%%%%%%%%%%%%%%%%%%%%%%%%%%%%%%%%%%%%%%%
\subsection{\EndFaceBoringMilling 中\TBW}
(to be written...)


%\clearpage
%%%%%%%%%%%%%%%%%%%%%%%%%%%%%%%%%%%%%%%%%%%%%%%%%%%%%%%%%%
%% subsection 01.03.7 %%%%%%%%%%%%%%%%%%%%%%%%%%%%%%%%%%%%
%%%%%%%%%%%%%%%%%%%%%%%%%%%%%%%%%%%%%%%%%%%%%%%%%%%%%%%%%%
\subsection{\IncutBoringMilling 中\TBW}
(to be written...)



\clearpage
%%%%%%%%%%%%%%%%%%%%%%%%%%%%%%%%%%%%%%%%%%%%%%%%%%%%%%%%%%
%% section 01.3 %%%%%%%%%%%%%%%%%%%%%%%%%%%%%%%%%%%%%%%%%%
%%%%%%%%%%%%%%%%%%%%%%%%%%%%%%%%%%%%%%%%%%%%%%%%%%%%%%%%%%
\modHeadsection{\expandafterindex{こうてい(\yomiMMC)@工程(\nameMMC)}\nameMMC における工程(加工後)}


%%%%%%%%%%%%%%%%%%%%%%%%%%%%%%%%%%%%%%%%%%%%%%%%%%%%%%%%%%
%% subsection 01.3.1 %%%%%%%%%%%%%%%%%%%%%%%%%%%%%%%%%%%%%
%%%%%%%%%%%%%%%%%%%%%%%%%%%%%%%%%%%%%%%%%%%%%%%%%%%%%%%%%%
\subsection{ワークの取外し}
\begin{enumerate}[label=\sarrow]
\item 必要に応じて、\index{ワークこていようボルト@ワーク固定用ボルト}ワーク固定用ボルトを緩める前に、各種\index{そくていき@測定器}測定器で各々の\index{すんぽう@寸法}寸法を確認する
\item \index{クーラント}クーラントおよびエアーブローを用いて軽く洗浄を行い、ワーク固定用ボルトを緩めて\index{ワーク}ワークを取り出し、軽く乾拭きを行う
\item \index{リフター}リフターまたは\index{クレーン}クレーンを用いて、\index{めんとりようさぎょうだい@面取用作業台}面取用作業台にワークを移動する
\end{enumerate}


%%%%%%%%%%%%%%%%%%%%%%%%%%%%%%%%%%%%%%%%%%%%%%%%%%%%%%%%%%
%% subsection 01.3.2 %%%%%%%%%%%%%%%%%%%%%%%%%%%%%%%%%%%%%
%%%%%%%%%%%%%%%%%%%%%%%%%%%%%%%%%%%%%%%%%%%%%%%%%%%%%%%%%%
\subsection{外観の確認・寸法の測定}
\begin{enumerate}[label=\sarrow]
\item \index{がいかん(ワーク)@外観(ワーク)}外観に異常がないか確認を行う
\item \index{そくていき@測定器}測定器を用いて\index{すんぽう@寸法}寸法の確認を行う
\item 所定の用紙に、指定箇所の\index{こうさ@公差}公差を考慮した寸法値を、手動で計算を行い手動で記入する
\item 必要に応じて、所定の用紙に測定値の記入を行う
\end{enumerate}


%%%%%%%%%%%%%%%%%%%%%%%%%%%%%%%%%%%%%%%%%%%%%%%%%%%%%%%%%%
%% subsection 01.3.3 %%%%%%%%%%%%%%%%%%%%%%%%%%%%%%%%%%%%%
%%%%%%%%%%%%%%%%%%%%%%%%%%%%%%%%%%%%%%%%%%%%%%%%%%%%%%%%%%
\subsection{手動による仕上げ加工}
\begin{enumerate}[label=\sarrow]
\item 所定の寸法の\EndFaceChamfer を、\index{てもちけんまき@手持ち研磨機}手持ち研磨機を用いて手動で行う
\item \index{ばり}ばり等を除去を、\index{やすり}やすりを用いて全体的に手動で行う
\item \expandafterindex{Cがわにくあつ(\yomiBottomEndFace)@C側肉厚(\nameBottomEndFace)}\nameBottomEndFace のC側肉厚に応じて\index{こくいん@刻印}刻印の大きさを決定する
\item 明細のによる指定に応じて、刻印の位置を調整する
\item リフターまたはクレーンを用いて、所定の置き場に移動する
\end{enumerate}

