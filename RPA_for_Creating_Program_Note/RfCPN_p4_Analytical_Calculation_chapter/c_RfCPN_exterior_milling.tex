%!TEX root = ../RPA_for_Creating_Program_Note.tex



モールドに\index{がいさく@外削}\textbf{外削}があるときは、たいていの場合、\index{Aがわないめん@A側内面}A側内面の端面における位置を基準として考えることが多い。

トップ・ボトム端における\index{ないけい@内径}内径をそれぞれ$w_\mathrm T$, $w_\mathrm B$, \index{がいさくけい@外削径}外削径をそれぞれ$\mathfrak W_\mathrm T$, $\mathfrak W_\mathrm B$, \index{Aがわにくあつ@A側肉厚}A側\index{にくあつ@肉厚}\textbf{肉厚}をそれぞれ$\tau_\mathrm T$, $\tau_\mathrm B$とする。
また、内面の\index{めっきまくあつ@めっき膜厚}\textbf{めっき膜厚}を$\mu$とし、\index{とおりしん@通り芯}通り心(\index{トップがわのがいさくちゅうしん@トップ側の外削中心}トップ外削中心$\mathfrak T_\mathrm c$と\index{ボトムがわのがいさくちゅうしん@ボトム側の外削中心}ボトム外削中心$\mathfrak B_\mathrm c$の差)の$X$, $Y$成分をそれぞれ$T_x$, $T_y$とする。
ただし、$T_x \geqq 0$として、トップ外削中心$\mathfrak T_\mathrm c$はボトム外削中心$\mathfrak B_\mathrm c$よりA面方向にあるものとする。
%%%%%%%%%%%%%%%%%%%%%%%%%%%%%%%%%%%%%%%%%%%%%%%%%%%%%%%%%%
%% hosoku %%%%%%%%%%%%%%%%%%%%%%%%%%%%%%%%%%%%%%%%%%%%%%%%
%%%%%%%%%%%%%%%%%%%%%%%%%%%%%%%%%%%%%%%%%%%%%%%%%%%%%%%%%%
\begin{hosoku}
内径$w_\mathrm T$, $w_\mathrm B$は、湾曲の中心O(またはO$'$)に向かった方向にあることに注意。
\index{ないけいちゅうしん@内径中心}内径の中心がそれぞれの端に位置している。
\end{hosoku}
%%%%%%%%%%%%%%%%%%%%%%%%%%%%%%%%%%%%%%%%%%%%%%%%%%%%%%%%%%
%%%%%%%%%%%%%%%%%%%%%%%%%%%%%%%%%%%%%%%%%%%%%%%%%%%%%%%%%%
%%%%%%%%%%%%%%%%%%%%%%%%%%%%%%%%%%%%%%%%%%%%%%%%%%%%%%%%%%



%%%%%%%%%%%%%%%%%%%%%%%%%%%%%%%%%%%%%%%%%%%%%%%%%%%%%%%%%%
%% section 3.1 %%%%%%%%%%%%%%%%%%%%%%%%%%%%%%%%%%%%%%%%%%%
%%%%%%%%%%%%%%%%%%%%%%%%%%%%%%%%%%%%%%%%%%%%%%%%%%%%%%%%%%
\modHeadsection{ボトム側外削径の中心(ボトム基準)}
\index{Aがわないめん@A側内面}A側内面の端面における位置を基準とする場合、\index{ボトムたんのがいけいちゅうしん@ボトム端の外径中心}ボトム端の外径中心B$_\mathrm c'$から、\index{ボトムたんのないけい@ボトム端の内径}ボトム端の内径$w_\mathrm B$の半分を引き、さらに\index{Aがわにくあつ@A側肉厚}A側肉厚$\tau_\mathrm B$と\index{めっきまくあつ@めっき膜厚}めっき膜厚$\mu$との差を引いたものが(おおよその)\index{Aがわがいさくめん@A側外削面}A側外削面の位置$\mathfrak B_\mathrm o'$に相当する
%% footnote %%%%%%%%%%%%%%%%%%%%%
\footnote{ボトム側が工具側にある場合は、A面は$X$の負方向にあることに注意。}。
%%%%%%%%%%%%%%%%%%%%%%%%%%%%%%%%%


%%%%%%%%%%%%%%%%%%%%%%%%%%%%%%%%%%%%%%%%%%%%%%%%%%%%%%%%%%
%% subsubsection 3.1.1 %%%%%%%%%%%%%%%%%%%%%%%%%%%%%%%%%%%
%%%%%%%%%%%%%%%%%%%%%%%%%%%%%%%%%%%%%%%%%%%%%%%%%%%%%%%%%%
\subsection[スペーサを用いた場合の\texorpdfstring{$\mathfrak B_\mathrm c'$}{Bc'}]
           {スペーサを用いた場合の$\boldsymbol{\mathfrak B_\mathrm c'}$}
厚さ$\delta_\mathrm s$の\index{スペーサ}スペーサを用いた場合、\index{テーブルちゅうしん@テーブル中心}テーブル中心Pを原点とした
%% footnote %%%%%%%%%%%%%%%%%%%%%
\footnote{\index{マシニング}マシニングによって\index{きかいげんてん@機械原点}機械原点(の$X$座標)がテーブル中心Pと同じだったり異なったりする場合がある。}\relax
%%%%%%%%%%%%%%%%%%%%%%%%%%%%%%%%%
\index{ボトムがわのがいさくちゅうしん@ボトム側の外削中心}ボトム側外削径の中心$\mathfrak B_\mathrm c'$の(おおよその)$X$座標は、\pageeqref{eq:spacerBc}より、
\begin{align*}
  \varDelta-\frac{\sqrt{R_\mathrm o^2-f_\mathrm B^2}+\sqrt{R_\mathrm i^2-f_\mathrm B^2}}2-\frac{\delta_\mathrm s}2
  +\sqrt{R_\mathrm i'^2-\frac{\delta_\mathrm s^2+(2\bar l)^2}4}\frac{2\bar l}{\sqrt{\delta_\mathrm s^2+(2\bar l)^2}}
  -\frac{w_\mathrm B}2-\tau_\mathrm B+\frac{\mathfrak W_\mathrm B}2\ .
\end{align*}
%%%%%%%%%%%%%%%%%%%%%%%%%%%%%%%%%%%%%%%%%%%%%%%%%%%%%%%%%%
%% hosoku %%%%%%%%%%%%%%%%%%%%%%%%%%%%%%%%%%%%%%%%%%%%%%%%
%%%%%%%%%%%%%%%%%%%%%%%%%%%%%%%%%%%%%%%%%%%%%%%%%%%%%%%%%%
\begin{hosoku}
正確には、\index{ボトムたんめんのないめんちゅうしん@ボトム端面の内面中心}ボトム端面における(内径ではなく)A・C面側の内面中心b$_\mathrm c'$を見る必要がある。
しかし実際の作業においては、これは\index{タッチセンサー}タッチセンサーの\index{そくていかいしてん@測定開始点}測定開始点として用いるものであるため、おおよその値($\pm10$mm以内程度)で十分である。
そのため、ここでは単純に中心b$_\mathrm c'$の代わりに\index{ボトムたんのがいけいちゅうしん@ボトム端の外径中心}ボトム外径中心B$_\mathrm c'$とし、また\index{ボトムたんのないけい@ボトム端の内径}ボトム端における内径$w_\mathrm B$を用いている。
さらにいうと、外径中心B$_\mathrm c'$は\index{ボトムたんのわんきょくちゅうしん@ボトム端の湾曲中心}ボトム端の湾曲中心B$_{\mathrm R_\mathrm c}'$で代用してもまず問題はない。
\end{hosoku}
%%%%%%%%%%%%%%%%%%%%%%%%%%%%%%%%%%%%%%%%%%%%%%%%%%%%%%%%%%
%%%%%%%%%%%%%%%%%%%%%%%%%%%%%%%%%%%%%%%%%%%%%%%%%%%%%%%%%%
%%%%%%%%%%%%%%%%%%%%%%%%%%%%%%%%%%%%%%%%%%%%%%%%%%%%%%%%%%
これは飽くまで(\index{ずめん@図面}図面の数字をもとにした)計算値であり、\index{タッチセンサー}タッチセンサーでの\index{そくていかいしてん@測定開始点}測定開始点として用いる。
そして、現物のボトム端に相当する箇所のA面(負方向)側内面b$_\mathrm o'$の位置を直接計測し、その位置を基準として(\index{ワークざひょうけい@ワーク座標系}ワーク座標系の)\index{げんてん@原点}原点$\mathfrak B_\mathrm c'$を定める。
このとき、\index{Aがわないめん@A側内面}A側内面b$_\mathrm o'$の$X$座標が以下になるように、原点$\mathfrak B_\mathrm c'$を定める。
\begin{align*}
  -\left(\frac{\mathfrak W_\mathrm B}2-\tau_\mathrm B+\mu\right).
\end{align*}


%%%%%%%%%%%%%%%%%%%%%%%%%%%%%%%%%%%%%%%%%%%%%%%%%%%%%%%%%%
%% subsubsection 3.1.2 %%%%%%%%%%%%%%%%%%%%%%%%%%%%%%%%%%%
%%%%%%%%%%%%%%%%%%%%%%%%%%%%%%%%%%%%%%%%%%%%%%%%%%%%%%%%%%
\subsection[テーブルを傾けた場合の\texorpdfstring{$\mathfrak B_\mathrm c'$}{Bc'}]
           {テーブルを傾けた場合の$\boldsymbol{\mathfrak B_\mathrm c'}$}
\index{テーブル}テーブルを$-\theta$傾けた場合、\index{テーブルちゅうしん@テーブル中心}テーブル中心Pを原点としたボトム側外削径の中心$\mathfrak B_\mathrm c'$の(おおよその)$X$座標は、\pageeqref{eq:tableBc}より、
\begin{align}
  \label{eq:gaisakucenterBt}
  \varDelta'\cos\theta-\frac{\sqrt{R_\mathrm o^2-f_\mathrm B^2}+\sqrt{R_\mathrm i^2-f_\mathrm B^2}}2
  -\frac{w_\mathrm B}2-\tau_\mathrm B+\frac{\mathfrak W_\mathrm B}2\ .
\end{align}
このとき、計測した\index{Aがわないめん@A側内面}A側内面b$_\mathrm o'$の$X$座標が以下になるように、原点$\mathfrak B_\mathrm c'$を定める。
\begin{align}
  \label{eq:gaisakucenterBr}
  -\left(\frac{\mathfrak W_\mathrm B}2-\tau_\mathrm B+\mu\right).
\end{align}


%%%%%%%%%%%%%%%%%%%%%%%%%%%%%%%%%%%%%%%%%%%%%%%%%%%%%%%%%%
%% subsection 3.1.3 %%%%%%%%%%%%%%%%%%%%%%%%%%%%%%%%%%%%%%
%%%%%%%%%%%%%%%%%%%%%%%%%%%%%%%%%%%%%%%%%%%%%%%%%%%%%%%%%%
\subsection{トップ側外削径中心(ボトム基準)}
トップ側にも\index{がいさく@外削}外削がある場合、ボトム側外削から\index{とおりしん@通り芯}通り芯を指定する形でトップ外削の位置を決めるのが通常である。
このとき、テーブル中心Pを\index{げんてん@原点}原点とした\index{トップがわのがいさくちゅうしん@トップ側の外削中心}トップ側外削径中心$\mathfrak T_\mathrm c'$の$X$座標は、計測で定めた$\mathfrak B_\mathrm c'$の$X$座標$\mathcal G_{\mathrm Bx}$の符号を反転し
%% footnote %%%%%%%%%%%%%%%%%%%%%
\footnote{トップ側が工具側にある場合は、A面は$X$の正方向にある。
ボトム側と比べてテーブルを$B$軸($Y$軸まわり)に$180^\circ$回転する必要があるため、$X$座標の符号が反転する形になる。}、
%%%%%%%%%%%%%%%%%%%%%%%%%%%%%%%%%
通り芯$T_x$の分を加味すればよい。
したがって、
\begin{align}
  \label{eq:BbasedTx}
  -\mathcal G_{Bx}+T_x
\end{align}
で与えられる
%% footnote %%%%%%%%%%%%%%%%%%%%%
\footnote{$Y$座標については、$B$軸の回転に影響しないので、$\mathcal G_{\mathrm By}+T_y$となる。
なお、実際の作業においては、$T_y = 0$であることが通常である。}。
%%%%%%%%%%%%%%%%%%%%%%%%%%%%%%%%%
ただし実際の作業では、\index{テーブルちゅうしん@テーブル中心}テーブル中心Pの\index{かいてんちゅうしん@回転中心}回転中心からのずれも考慮する必要がある
%% footnote %%%%%%%%%%%%%%%%%%%%%
\footnote{\index{かいてんちゅうしん@回転中心}回転中心とテーブル中心は通常一致しているものとして考えるが、実際にはわずかにずれている。
特に$X$方向のずれは、$B$軸回転を伴う場合に効いてくる。}。
%%%%%%%%%%%%%%%%%%%%%%%%%%%%%%%%%



\clearpage
%%%%%%%%%%%%%%%%%%%%%%%%%%%%%%%%%%%%%%%%%%%%%%%%%%%%%%%%%%
%% section 3.2 %%%%%%%%%%%%%%%%%%%%%%%%%%%%%%%%%%%%%%%%%%%
%%%%%%%%%%%%%%%%%%%%%%%%%%%%%%%%%%%%%%%%%%%%%%%%%%%%%%%%%%
\modHeadsection{トップ側外削径の中心}


%%%%%%%%%%%%%%%%%%%%%%%%%%%%%%%%%%%%%%%%%%%%%%%%%%%%%%%%%%
%% subsection 3.2.1 %%%%%%%%%%%%%%%%%%%%%%%%%%%%%%%%%%%%%%
%%%%%%%%%%%%%%%%%%%%%%%%%%%%%%%%%%%%%%%%%%%%%%%%%%%%%%%%%%
\subsection[スペーサを用いた場合の\texorpdfstring{$\mathfrak T_\mathrm c'$}{Tc'}]
           {スペーサを用いた場合の$\boldsymbol{\mathfrak T_\mathrm c'}$}
\index{トップがわのAがわがいさくめん@トップ端のA側外削面}トップ端A側外削面が基準となる場合も考慮しておく。
この場合も考えかたはボトム基準のそれと同様である。
\index{テーブルちゅうしん@テーブル中心}テーブル中心Pを原点とした場合の、トップ側外削径の中心$\mathfrak T_\mathrm c'$のおおよその$X$座標は、\pageeqref{eq:spacerTc}より、
\begin{align*}
  -\varDelta+\frac{\sqrt{R_\mathrm o^2-f_\mathrm T^2}+\sqrt{R_\mathrm i^2-f_\mathrm T^2}}2+\frac{\delta_\mathrm s}2
  -\sqrt{R_\mathrm i'^2-\frac{\delta_\mathrm s^2+(2\bar l)^2}4}\frac{2\bar l}{\sqrt{\delta_\mathrm s^2+(2\bar l)^2}}
  +\frac{w_\mathrm T}2+\tau_\mathrm T-\frac{\mathfrak W_\mathrm T}2\ .
\end{align*}
これを\index{タッチセンサー}タッチセンサーでの\index{そくていかいしてん@測定開始点}測定開始点とし、計測した原点の$X$座標(実測値)を$\mathcal G_{Tx}$とすると、トップ端における\index{Aがわないめん@A側内面}A面側内面と$\mathcal G_{Tx}$との差の$X$座標は、
\begin{align*}
  \frac{\mathfrak W_\mathrm T}2-\tau_\mathrm T+\mu~.
\end{align*}


%%%%%%%%%%%%%%%%%%%%%%%%%%%%%%%%%%%%%%%%%%%%%%%%%%%%%%%%%%
%% subsection 3.2.2 %%%%%%%%%%%%%%%%%%%%%%%%%%%%%%%%%%%%%%
%%%%%%%%%%%%%%%%%%%%%%%%%%%%%%%%%%%%%%%%%%%%%%%%%%%%%%%%%%
\subsection[テーブルを傾けた場合の\texorpdfstring{$\mathfrak T_\mathrm c'$}{Tc'}]
           {テーブルを傾けた場合の$\boldsymbol{\mathfrak T_\mathrm c'}$}
テーブルを$-\theta$傾けた場合、\index{テーブルちゅうしん@テーブル中心}テーブル中心Pを原点としたトップ側外削径の中心$\mathfrak T_\mathrm c'$の(おおよその)$X$座標は、\pageeqref{eq:tableTc}より、
\begin{align}
  \label{eq:gaisakucenterTt}
  \frac{\sqrt{R_\mathrm o^2-f_\mathrm T^2}+\sqrt{R_\mathrm i^2-f_\mathrm T^2}}2-\varDelta'\cos\theta
  +\frac{w_\mathrm T}2+\tau_\mathrm T-\frac{\mathfrak W_\mathrm T}2\ .
\end{align}
計測して定めた原点$\mathfrak T_\mathrm c'$と、トップ端A側内面t$_\mathrm o'$との差の$X$座標は、
\begin{align}
  \label{eq:gaisakucenterTr}
  \frac{\mathfrak W_\mathrm T}2-\tau_\mathrm T+\mu~.
\end{align}




%%%%%%%%%%%%%%%%%%%%%%%%%%%%%%%%%%%%%%%%%%%%%%%%%%%%%%%%%%
%% subsection 3.2.3 %%%%%%%%%%%%%%%%%%%%%%%%%%%%%%%%%%%%%%
%%%%%%%%%%%%%%%%%%%%%%%%%%%%%%%%%%%%%%%%%%%%%%%%%%%%%%%%%%
\subsection{ボトム側外削径中心(トップ基準)}
ボトム側にも\index{がいさく@外削}外削がある場合、トップ側外削から\index{とおりしん@通り芯}通り芯を指定する形でボトム外削の位置を決めることが多い。
このとき、\index{テーブルちゅうしん@テーブル中心}テーブル中心Pを\index{げんてん@原点}原点とした\index{ボトムがわのがいさくちゅうしん@ボトム側の外削中心}ボトム側外削径中心$\mathfrak B_\mathrm c'$の$X$座標は、計測で定めた$\mathfrak T_\mathrm c'$の$X$座標$\mathcal G_{\mathrm Tx}$の符号を反転し、\index{とおりしん@通り芯}通り芯$T_x$の分を加味すればよい。
したがって、
\begin{align}
  \label{eq:TbasedTx}
  -\mathcal G_{Tx}+T_x
\end{align}
で与えられる。



\clearpage
%%%%%%%%%%%%%%%%%%%%%%%%%%%%%%%%%%%%%%%%%%%%%%%%%%%%%%%%%%
%% section 3.3 %%%%%%%%%%%%%%%%%%%%%%%%%%%%%%%%%%%%%%%%%%%
%%%%%%%%%%%%%%%%%%%%%%%%%%%%%%%%%%%%%%%%%%%%%%%%%%%%%%%%%%
\modHeadsection{トップ側の外削長}
トップ側の\index{がいさくちょう@外削長}外削長に関しては、基本的には\index{ふりわけちょう@振分長}振分長$f_\mathrm T$から外削長$h_\mathrm T$を引いた位置に$Z$座標を合わせればよい。
すなわち、\index{テーブルちゅうしん@テーブル中心}テーブル中心Pを\index{げんてん@原点}原点として$Z$座標を
\begin{align*}
  f_\mathrm T - h_\mathrm T
\end{align*}
とすればよい。
ただし、外削長$h_\mathrm T$が\index{みぞいち@溝位置}溝位置$\kappa_p$と\index{みぞはば@溝幅}溝幅$\kappa_w$の和に一致する場合は、\index{がいさくちょう@外削長}外削長を$\kappa_p+1$mmとして切削する
%% footnote %%%%%%%%%%%%%%%%%%%%%
\footnote{$f_\mathrm T-(\kappa_p+\kappa_w) < Z < f_\mathrm T-\kappa_p$であれば問題ない。
1mmとしているのは慣例によるものである。}。
%%%%%%%%%%%%%%%%%%%%%%%%%%%%%%%%%
すなわち、
\begin{align*}
  h_\mathrm T = \kappa_p+\kappa_w \quad \longrightarrow \quad f_\mathrm T-\kappa_p-1[\mathrm{mm}]
\end{align*}




\clearpage
%%%%%%%%%%%%%%%%%%%%%%%%%%%%%%%%%%%%%%%%%%%%%%%%%%%%%%%%%%
%% section 8.2 %%%%%%%%%%%%%%%%%%%%%%%%%%%%%%%%%%%%%%%%%%%
%%%%%%%%%%%%%%%%%%%%%%%%%%%%%%%%%%%%%%%%%%%%%%%%%%%%%%%%%%
\modHeadsection{湾曲に沿った外削\TBW}
外削が湾曲に沿った形である場合を考える。
ここでは\index{テーブルちゅうしん@テーブル中心}テーブル中心Pを原点とし、ボトム端面側が工具側に向いているとする。
端面のA面側・C面側・中心の$X$座標をそれぞれ$x_A$, $x_C$, $x_m$とする。
このとき\index{たんめん@端面}端面のそれぞれの位置は
%% footnote %%%%%%%%%%%%%%%%%%%%%
\footnote{ここでは$Z$の正方向を実軸、$X$の正方向を虚軸として考えている。}、
%%%%%%%%%%%%%%%%%%%%%%%%%%%%%%%%%
\begin{subequations}
\begin{align*}
  \text{中点:}&\quad \sqrt{z_b^2+x_m^2}e^{i\theta_m}, \quad \tan\theta_m = \frac{x_m}{z_b}\\
  \text{C面側端点:}&\quad \sqrt{z_b^2+x_C^2}e^{i\theta_C}, \quad \tan\theta_C = \frac{x_C}{z_b}\\
  \text{A面側端点:}&\quad \sqrt{z_b^2+x_A^2}e^{i\theta_A}, \quad \tan\theta_A = \frac{x_A}{z_b}.
\end{align*}
\end{subequations}
\index{ボトムがわのがいさくちょう@ボトム側の外削長}ボトムの外削長を$h_\mathrm B$とすると、外削部分の傾きはボトム端から距離$\nicefrac{h_\mathrm B}2$の断面に平行になる形にとるので、必要な回転角$\theta_b$は、
\begin{align*}
  \sin\theta_b = \frac{z_b-\nicefrac{h_\mathrm B}2}{R_\mathrm c}
\end{align*}
を満たす。
回転後の\index{ボトムたんめん@ボトム端面}ボトム端面の中心の位置は、
\begin{align*}
  \sqrt{z_b^2+x_m^2}e^{i(\theta_m-\theta_b)}
\end{align*}
となるので、これの虚部が$X$座標、実部が$Z$座標となる
%% footnote %%%%%%%%%%%%%%%%%%%%%
\footnote{ここでは\index{ふくそすうへいめん@複素数平面}複素数平面を考えているが、通常の直行座標系で単純に回転行列をかけているのと同義である。
\begin{align*}
  \left[
    \begin{array}{c}
      x'_m\\
      z'_b
    \end{array}
  \right]
  = \left[
    \begin{array}{cc}
      \cos\theta_b & -\sin\theta_b\\
      \sin\theta_b & \cos\theta_b
    \end{array}
  \right]\!\!
  \left[
    \begin{array}{c}
      x_m\\
      z_b
    \end{array}
  \right]
  = \left[
    \begin{array}{c}
      x_m\cos\theta_b-z_b\sin\theta_b\\
      x_m\sin\theta_b+z_b\cos\theta_b
    \end{array}
  \right].
\end{align*}%
}。
%%%%%%%%%%%%%%%%%%%%%%%%%%%%%%%%%
\begin{align*}
  \sqrt{z_b^2+x_m^2}\sin(\theta_m-\theta_b)
  &= \sqrt{z_b^2+x_m^2}(\sin\theta_m\cos\theta_b-\cos\theta_m\sin\theta_b)\\
  &= x_m\sqrt{1-\left(\frac{z_b-\nicefrac{h_\mathrm B}2}{R_\mathrm c}\right)^{\!2}}
     -z_b\cdot\frac{z_b-\nicefrac{h_\mathrm B}2}{R_\mathrm c}~,\\
  \sqrt{z_b^2+x_m^2}\cos(\theta_m-\theta_b)
  &= \sqrt{z_b^2+x_m^2}(\cos\theta_m\cos\theta_b+\sin\theta_m\sin\theta_b)\\
  &= z_b\sqrt{1-\left(\frac{z_b-\nicefrac{h_\mathrm B}2}{R_\mathrm c}\right)^{\!2}}
     +x_m\cdot\frac{z_b-\nicefrac{h_\mathrm B}2}{R_\mathrm c}~.
\end{align*}
端面のA面側・C面側の位置についても同様である。
まとめると、
\begin{align*}
  \text{中点:}&\quad
    \left[
      \begin{array}{c}
        x'_m\\
        z'_b
      \end{array}
    \right]
    = \left[
      \begin{array}{c}
        \displaystyle
        x_m\sqrt{1-\left(\frac{z_b-\nicefrac{h_\mathrm B}2}{R_\mathrm c}\right)^{\!2}}
        -z_b\cdot\frac{z_b-\nicefrac{h_\mathrm B}2}{R_\mathrm c}\\[15pt]
        \displaystyle
        z_b\sqrt{1-\left(\frac{z_b-\nicefrac{h_\mathrm B}2}{R_\mathrm c}\right)^{\!2}}
        +x_m\cdot\frac{z_b-\nicefrac{h_\mathrm B}2}{R_\mathrm c}
      \end{array}
    \right],\\
  \text{C面側端点:}&\quad
    \left[
      \begin{array}{c}
        x'_C\\
        z'_b
      \end{array}
    \right]
    = \left[
      \begin{array}{c}
        \displaystyle
        x_C\sqrt{1-\left(\frac{z_b-\nicefrac{h_\mathrm B}2}{R_\mathrm c}\right)^{\!2}}
        -z_b\cdot\frac{z_b-\nicefrac{h_\mathrm B}2}{R_\mathrm c}\\[15pt]
        \displaystyle
        z_b\sqrt{1-\left(\frac{z_b-\nicefrac{h_\mathrm B}2}{R_\mathrm c}\right)^{\!2}}
        +x_C\cdot\frac{z_b-\nicefrac{h_\mathrm B}2}{R_\mathrm c}
      \end{array}
    \right],\\
  \text{A面側端点:}&\quad
    \left[
      \begin{array}{c}
        x'_A\\
        z'_b
      \end{array}
    \right]
    = \left[
      \begin{array}{c}
        \displaystyle
        x_A\sqrt{1-\left(\frac{z_b-\nicefrac{h_\mathrm B}2}{R_\mathrm c}\right)^{\!2}}
        -z_b\cdot\frac{z_b-\nicefrac{h_\mathrm B}2}{R_\mathrm c}\\[15pt]
        \displaystyle
        z_b\sqrt{1-\left(\frac{z_b-\nicefrac{h_\mathrm B}2}{R_\mathrm c}\right)^{\!2}}
        +x_A\cdot\frac{z_b-\nicefrac{h_\mathrm B}2}{R_\mathrm c}
      \end{array}
    \right].
\end{align*}
