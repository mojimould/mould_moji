%!TEX root = ../RPA_for_Creating_Program_Note.tex


モールドの端面加工・外削加工・溝加工等は主に\MMname にて行われている。
新たに導入する横型マシニングセンタ(\DMname)での工程は、\dimple の測定・加工を除けば\MMname と同様である。
そこで、まずは\MMname ではどのようなフローで業務が行われているかを(ソフトウェアの観点から)みることにする。

なお、ここでは\MMname のNo.1パレット(山側)で加工を行うものを対象とする
%% footnote %%%%%%%%%%%%%%%%%%%%%
\footnote{No.2パレット(海側)では、径の大きなものや丸形のもの等の加工が主に行われる。}。
%%%%%%%%%%%%%%%%%%%%%%%%%%%%%%%%%



%%%%%%%%%%%%%%%%%%%%%%%%%%%%%%%%%%%%%%%%%%%%%%%%%%%%%%%%%%
%% section 1.1 %%%%%%%%%%%%%%%%%%%%%%%%%%%%%%%%%%%%%%%%%%%
%%%%%%%%%%%%%%%%%%%%%%%%%%%%%%%%%%%%%%%%%%%%%%%%%%%%%%%%%%
\modHeadsection{\MMname における工程(加工前)}
\MMname において、ある明細のモールドを加工をする際に、以下のような流れで作業が行われる。


%%%%%%%%%%%%%%%%%%%%%%%%%%%%%%%%%%%%%%%%%%%%%%%%%%%%%%%%%%
%% subsection 01.1.1 %%%%%%%%%%%%%%%%%%%%%%%%%%%%%%%%%%%%%
%%%%%%%%%%%%%%%%%%%%%%%%%%%%%%%%%%%%%%%%%%%%%%%%%%%%%%%%%%
\subsection{図面の確認}
\begin{enumerate}
\item 対象となる明細の図面を用意する
\item 他に内容が類似する明細の図面があれば、それも併せて用意する
\end{enumerate}


%%%%%%%%%%%%%%%%%%%%%%%%%%%%%%%%%%%%%%%%%%%%%%%%%%%%%%%%%%
%% subsection 01.1.2 %%%%%%%%%%%%%%%%%%%%%%%%%%%%%%%%%%%%%
%%%%%%%%%%%%%%%%%%%%%%%%%%%%%%%%%%%%%%%%%%%%%%%%%%%%%%%%%%
\subsection{加工部分の有無の確認}

%%%%%%%%%%%%%%%%%%%%%%%%%%%%%%%%%%%%%%%%%%%%%%%%%%%%%%%%%%
%% subsubsection 01.1.2.2 %%%%%%%%%%%%%%%%%%%%%%%%%%%%%%%%
%%%%%%%%%%%%%%%%%%%%%%%%%%%%%%%%%%%%%%%%%%%%%%%%%%%%%%%%%%
\subsubsection{端面部分}
端面の加工については、全明細に共通の形で存在する。

%%%%%%%%%%%%%%%%%%%%%%%%%%%%%%%%%%%%%%%%%%%%%%%%%%%%%%%%%%
%% subsubsection 01.1.2.2 %%%%%%%%%%%%%%%%%%%%%%%%%%%%%%%%
%%%%%%%%%%%%%%%%%%%%%%%%%%%%%%%%%%%%%%%%%%%%%%%%%%%%%%%%%%
\subsubsection{外削部分}
外削の加工については、明細により外削の有無または形状の違いが存在する。
\begin{enumerate}
\item トップ側またはボトム側の\index{がいさく@外削}外削の有無を確認する
\item 外削の形状を確認し、使用する工具を決定する
\item \index{わんきょくにそったがいさく@湾曲に沿った外削}外削が湾曲に沿ったものかどうかも確認する
\end{enumerate}

%%%%%%%%%%%%%%%%%%%%%%%%%%%%%%%%%%%%%%%%%%%%%%%%%%%%%%%%%%
%% subsubsection 01.1.2.3 %%%%%%%%%%%%%%%%%%%%%%%%%%%%%%%%
%%%%%%%%%%%%%%%%%%%%%%%%%%%%%%%%%%%%%%%%%%%%%%%%%%%%%%%%%%
\subsubsection{溝加工}
外削の加工については、全明細に存在し、明細により形状の違いが存在する。
\begin{enumerate}
\item 溝の形状を確認し、使用するサブプログラムの判断を行う
\item \index{みぞはば@溝幅}溝幅を確認し、使用する工具の判断を行う
\end{enumerate}

%\clearpage
%%%%%%%%%%%%%%%%%%%%%%%%%%%%%%%%%%%%%%%%%%%%%%%%%%%%%%%%%%
%% subsubsection 01.1.2.4 %%%%%%%%%%%%%%%%%%%%%%%%%%%%%%%%
%%%%%%%%%%%%%%%%%%%%%%%%%%%%%%%%%%%%%%%%%%%%%%%%%%%%%%%%%%
\subsubsection{面取加工}
外削の加工については、全明細に存在し、明細により形状の違いが存在する。
\begin{enumerate}
\item 溝の形状を確認し、C面取であれば、マシニングによる加工を行うか判断を行う
\item \index{Cめんとり@C面取}C面取の角度を確認し、使用する工具を決定する
\end{enumerate}


\clearpage
%%%%%%%%%%%%%%%%%%%%%%%%%%%%%%%%%%%%%%%%%%%%%%%%%%%%%%%%%%
%% subsection 01.1.3 %%%%%%%%%%%%%%%%%%%%%%%%%%%%%%%%%%%%%
%%%%%%%%%%%%%%%%%%%%%%%%%%%%%%%%%%%%%%%%%%%%%%%%%%%%%%%%%%
\subsection{加工部分の寸法の確認}

%%%%%%%%%%%%%%%%%%%%%%%%%%%%%%%%%%%%%%%%%%%%%%%%%%%%%%%%%%
%% subsubsection 01.1.3.1 %%%%%%%%%%%%%%%%%%%%%%%%%%%%%%%%
%%%%%%%%%%%%%%%%%%%%%%%%%%%%%%%%%%%%%%%%%%%%%%%%%%%%%%%%%%
\subsubsection{端面における寸法}
\begin{enumerate}
\item \index{ぜんちょう@全長}全長の公差を確認し、トップ振分長およびボトム振分長の公差の判断を行う
\item トップ側・ボトム側の\index{ふりわけちょう@振分長}振分長を確認し、\index{スペーサ}スペーサによる調整が必要か判断を行う
\item 使用するスペーサおよび再振分長は、専用の計算プログラム(\index{Excel VBA}Excel VBA)を用いて決定する
\item 外径・端面部の肉厚・コーナーRの大きさを確認し、それに応じて工具径補正量を決定する
\end{enumerate}
\begin{Tabbox}[title={必要な図面上のパラメタ}]\small
\paragraph*{再振分長}
全長・トップ振分長・ボトム振分長・AC外径・ジグの長さ・受板の幅
\tcbline*
\paragraph*{トップ端面}
トップ再振分長・AC外径・BD外径・トップ端外側コーナーR・トップ端AC内径・トップ端BD内径
\tcbline*
\paragraph*{ボトム端面}
ボトム再振分長・AC外径・BD外径・ボトム端外側コーナーR・ボトム端AC内径・ボトム端BD内径
\end{Tabbox}

%%%%%%%%%%%%%%%%%%%%%%%%%%%%%%%%%%%%%%%%%%%%%%%%%%%%%%%%%%
%% subsubsection 01.1.3.2 %%%%%%%%%%%%%%%%%%%%%%%%%%%%%%%%
%%%%%%%%%%%%%%%%%%%%%%%%%%%%%%%%%%%%%%%%%%%%%%%%%%%%%%%%%%
\subsubsection{外削における寸法}
\begin{enumerate}
\item \index{がいさくちょう@外削長}外削長と\index{みぞ@溝}溝の位置を確認し、実際に加工する外削の長さの判断を行う
\item トップ側・ボトム側の両方に外削のある場合は、どちら側が基準であるのかを確認する
\item 湾曲に沿った外削の場合は、傾き角を手動による計算で決定する
\end{enumerate}
\begin{Tabbox}[title={必要な図面上のパラメタ}]\small
\paragraph*{ボトム側の外削のみの場合}
ボトム外削AC径・ボトム外削BD径・ボトム端AC内径・ボトム端A側肉厚・めっき膜厚・ボトム外削長
\tcbline*
\paragraph*{トップ側の外削のみの場合}
トップ外削AC径・トップ外削BD径・トップ端AC内径・トップ端A側肉厚・めっき膜厚・トップ外削長・溝位置・溝幅
\tcbline*
\paragraph*{両方に外削があり、ボトム側が基準の場合}
ボトム外削AC径・ボトム外削BD径・ボトム端AC内径・ボトム端A側肉厚・めっき膜厚・ボトム外削長・
トップ外削AC径・トップ外削BD径・トップ外削長・溝位置・溝幅・
通り芯
\tcbline*
\paragraph*{両方に外削があり、トップ側が基準の場合}
トップ外削AC径・トップ外削BD径・トップ端AC内径・トップ端A側肉厚・トップ外削長・溝位置・溝幅・
ボトム外削AC径・ボトム外削BD径・めっき膜厚・ボトム外削長・
通り芯
\tcbline*
\paragraph*{湾曲に沿った外削の場合}
(以上に加えて)中心湾曲
\end{Tabbox}

\clearpage
%%%%%%%%%%%%%%%%%%%%%%%%%%%%%%%%%%%%%%%%%%%%%%%%%%%%%%%%%%
%% subsubsection 01.1.3.3 %%%%%%%%%%%%%%%%%%%%%%%%%%%%%%%%
%%%%%%%%%%%%%%%%%%%%%%%%%%%%%%%%%%%%%%%%%%%%%%%%%%%%%%%%%%
\subsubsection{溝における寸法}
\begin{enumerate}
\item 溝の形状を確認し、必要に応じて加工における径の決定する
\item \index{みぞちゅうしん@溝中心}溝中心の基準を確認し、溝中心の$X$位置を手動で計算し、決定する
\end{enumerate}
\begin{Tabbox}[title={必要な図面上のパラメタ}]\small
\paragraph*{湾曲中心が基準の場合}
溝AC径・溝BD径・溝位置・溝幅・中心湾曲・溝コーナーRまたはC
\tcbline*
\paragraph*{外削中心が基準の場合}
溝AC径・溝BD径・溝位置・溝幅・溝コーナーRまたはC
\tcbline*
\paragraph*{A側溝深さが基準の場合}
(以上に加えて)A側溝深さ
\end{Tabbox}

%%%%%%%%%%%%%%%%%%%%%%%%%%%%%%%%%%%%%%%%%%%%%%%%%%%%%%%%%%
%% subsubsection 01.1.3.4 %%%%%%%%%%%%%%%%%%%%%%%%%%%%%%%%
%%%%%%%%%%%%%%%%%%%%%%%%%%%%%%%%%%%%%%%%%%%%%%%%%%%%%%%%%%
\subsubsection{面取における寸法}
\begin{enumerate}
\item 外側C面取の場合は、外削の有無を確認し、加工の径を決定する
\end{enumerate}
\begin{Tabbox}[title={必要な図面上のパラメタ}]\small
\paragraph*{トップ端外側C面取}
AC外径またはトップ外削AC径・BD外径またはトップ外削BD径・トップ端外側C面取長・外径またはトップ端外削コーナーR
\tcbline*
\paragraph*{ボトム端外側C面取}
AC外径またはボトム外削AC径・BD外径またはボトム外削BD径・ボトム端外側C面取長・外径またはボトム端外削コーナーR
\tcbline*
\paragraph*{トップ端内側C面取}
トップ端AC内径・トップ端BD内径・トップ端内側C面取長・トップ端内径コーナーR
\tcbline*
\paragraph*{ボトム端内側C面取}
ボトム端AC内径・ボトム端BD内径・ボトム端内側C面取長・ボトム端内径コーナーR
\end{Tabbox}


\clearpage
%%%%%%%%%%%%%%%%%%%%%%%%%%%%%%%%%%%%%%%%%%%%%%%%%%%%%%%%%%
%% subsection 01.1.4 %%%%%%%%%%%%%%%%%%%%%%%%%%%%%%%%%%%%%
%%%%%%%%%%%%%%%%%%%%%%%%%%%%%%%%%%%%%%%%%%%%%%%%%%%%%%%%%%
\subsection{プログラムの入力\TBW}
(to be written ...)


%\clearpage
%%%%%%%%%%%%%%%%%%%%%%%%%%%%%%%%%%%%%%%%%%%%%%%%%%%%%%%%%%
%% subsection 01.1.4 %%%%%%%%%%%%%%%%%%%%%%%%%%%%%%%%%%%%%
%%%%%%%%%%%%%%%%%%%%%%%%%%%%%%%%%%%%%%%%%%%%%%%%%%%%%%%%%%
\subsection{ワークの設置および調整\TBW}
(to be written ...)



\clearpage
%%%%%%%%%%%%%%%%%%%%%%%%%%%%%%%%%%%%%%%%%%%%%%%%%%%%%%%%%%
%% section 1.2 %%%%%%%%%%%%%%%%%%%%%%%%%%%%%%%%%%%%%%%%%%%
%%%%%%%%%%%%%%%%%%%%%%%%%%%%%%%%%%%%%%%%%%%%%%%%%%%%%%%%%%
\modHeadsection{\MMname における工程(加工中)\TBW}
(to be written...)



\clearpage
%%%%%%%%%%%%%%%%%%%%%%%%%%%%%%%%%%%%%%%%%%%%%%%%%%%%%%%%%%
%% section 1.3 %%%%%%%%%%%%%%%%%%%%%%%%%%%%%%%%%%%%%%%%%%%
%%%%%%%%%%%%%%%%%%%%%%%%%%%%%%%%%%%%%%%%%%%%%%%%%%%%%%%%%%
\modHeadsection{\MMname における工程(加工後)\TBW}
(to be written...)
