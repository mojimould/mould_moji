%!TEX root = ../RPA_for_Creating_Program_Note.tex



\modHeadchapter[lot]{工具の送り速さ(Sコード値)}
工具の移動には、主に{\ttfamily G00}, {\ttfamily G01}, {\ttfamily G02}, {\ttfamily G03}, {\ttfamily G31}が用いられる。
一般に、
\begin{enumerate}
\item {\ttfamily G00}は主に\index{はやおくり@早送り}早送りに使用されることが多い
\item {\ttfamily G01}は直線状に移動し、主に切削の際の送りに使用されることが多い
\item {\ttfamily G02}, {\ttfamily G03}は円弧状に移動し、主に切削の際の送りに使用されることが多い
\item {\ttfamily G31}は主に測定の際の\index{スキップきのう@スキップ機能}スキップ機能に伴って使用されることが多い
\end{enumerate}
{\ttfamily G00}は\index{さいだいおくりはやさ@最大送り速さ}最大送り速さで移動し、その速さは(Fコード値では)指定することはできない。
{\ttfamily G01}, {\ttfamily G02}, {\ttfamily G03}, {\ttfamily G31}はその送りの速さを\index{Fコードち@Fコード値}Fコード値で指定することができる。

なお、{\ttfamily G28}, {\ttfamily G30}では{\ttfamily G00}によって移動する。
また、{\ttfamily G31}はすべての工具で指定はすることができるが、スキップ機能を有するものは\index{タッチセンサープローブ}タッチセンサープローブのみである。
%%%%%%%%%%%%%%%%%%%%%%%%%%%%%%%%%%%%%%%%%%%%%%%%%%%%%%%%%%
%% marker %%%%%%%%%%%%%%%%%%%%%%%%%%%%%%%%%%%%%%%%%%%%%%%%
%%%%%%%%%%%%%%%%%%%%%%%%%%%%%%%%%%%%%%%%%%%%%%%%%%%%%%%%%%
\begin{marker}
この章では工具の具体的な送り速さ値を記述している。
しかし、工具の送り速さ値の具体的な数値は、(ソフトウェアでなく)ハードウェアの標準に記載するほうが望ましい。
\end{marker}
%%%%%%%%%%%%%%%%%%%%%%%%%%%%%%%%%%%%%%%%%%%%%%%%%%%%%%%%%%
%%%%%%%%%%%%%%%%%%%%%%%%%%%%%%%%%%%%%%%%%%%%%%%%%%%%%%%%%%
%%%%%%%%%%%%%%%%%%%%%%%%%%%%%%%%%%%%%%%%%%%%%%%%%%%%%%%%%%



%%%%%%%%%%%%%%%%%%%%%%%%%%%%%%%%%%%%%%%%%%%%%%%%%%%%%%%%%%
%% section 07.1 %%%%%%%%%%%%%%%%%%%%%%%%%%%%%%%%%%%%%%%%%%
%%%%%%%%%%%%%%%%%%%%%%%%%%%%%%%%%%%%%%%%%%%%%%%%%%%%%%%%%%
\modHeadsection{送り速さの基本事項}
\begin{enumerate}
\item $X$, $Y$, $Z$方向の\index{いちぎめ@位置決め}位置決め(\index{はやおくり@早送り}早送り){\ttfamily G00}の\index{おくりはやさ@送り速さ}送り速さデフォルト値:60000mm/min
\item $B$方向の\index{はやおくり(Bじく)@早送り(B軸)}早送り{\ttfamily G00}の送り速さデフォルト値:$12000^\circ$/min ($\sim 33.33$回転/min)
\item $X$, $Y$, $Z$方向の指定できる送り速さ値の範囲:$0\sim60000$mm/min
\item 設定可能な最小単位:0.1mm/min
\end{enumerate}



\clearpage
%%%%%%%%%%%%%%%%%%%%%%%%%%%%%%%%%%%%%%%%%%%%%%%%%%%%%%%%%%
%% section 07.2 %%%%%%%%%%%%%%%%%%%%%%%%%%%%%%%%%%%%%%%%%%
%%%%%%%%%%%%%%%%%%%%%%%%%%%%%%%%%%%%%%%%%%%%%%%%%%%%%%%%%%
\modHeadsection{タッチセンサープローブ}
\DMname では、\index{ぜんちょう(モールド)@全長(モールド)}全長の長い\index{タッチセンサープローブ}タッチセンサープローブを用いる。
したがって、\index{おくりはやさ@送り速さ}送り速さを大きくして移動をすると、その加速度によって\index{タッチセンサー}タッチセンサーが反応してしまったり、タッチセンサープローブそのものに大きな負担がかかる。
そのため、タッチセンサープローブの\index{おくりはやさ(タッチセンサープローブ)@送り速さ(タッチセンサープローブ)}送り速さに関しては他の工具よりも低めに設定する。
\begin{enumerate}[label=\Roman*., ref=\Roman*]
\item \label{item:FDTS} 原則として、{\ttfamily G00}を用いた移動はしない
\item \ref{item:FDTS}に伴い、{\ttfamily G28}, {\ttfamily G30}を(直接的に)用いた移動はしない
\item {\ttfamily G01}を位置決め(早送り)として用いるものとし、送り速さは{\ttfamily F5400}以下とする
\item ワークへの\index{アプローチ}アプローチの際は、{\ttfamily G31}を用いるものとし、送り速さは{\ttfamily F1500}以下とする
\item 測定の際の\index{スキップ}スキップ({\ttfamily G31})の送り速さは、測定の仕方に応じて以下のものとする
  \begin{enumerate}
  \item \index{しんごうおくれほせい@信号遅れ補正}信号遅れ補正を考慮する必要があるような場合は、送り速さはF50とする
  \item 信号遅れ補正を考慮する必要がない場合は、送り速さは{\ttfamily F50}以上{\ttfamily F300}以下とする
  \end{enumerate}
\item $XY$方向におけるワークからの\index{リトラクト}リトラクトの際は{\ttfamily G01}を用いるものとする。
\end{enumerate}
なお、$Z$方向の移動は工具ではなくテーブルが移動するため、アプローチを除いて次節(タッチセンサー以外の工具)にしたがうものとする。

なお、{\ttfamily G31}で送る場合は、タッチセンサープローブの\index{でんげん(タッチセンサープローブ)@電源(タッチセンサープローブ)}電源を入れて行うこと。
電源が入っていなければ移動せず、その点が\index{ブロックエンド}ブロックエンドとなる。



%\clearpage
%%%%%%%%%%%%%%%%%%%%%%%%%%%%%%%%%%%%%%%%%%%%%%%%%%%%%%%%%%
%% section 10.2 %%%%%%%%%%%%%%%%%%%%%%%%%%%%%%%%%%%%%%%%%%
%%%%%%%%%%%%%%%%%%%%%%%%%%%%%%%%%%%%%%%%%%%%%%%%%%%%%%%%%%
\modHeadsection{タッチセンサープローブ以外の工具}
\DMname では、$Z$方向については\index{テーブル}テーブル(\index{パレット(\DMname)@パレット(\DMname)}パレット)が移動する。
したがって、\index{Zほうこうのおくりはやさ@$Z$方向の送り速さ}$Z$方向の送り速さを大きくして移動すると、その加速度によってテーブル上の\index{ジグ}ジグが動いてしまう恐れがある。
そのため、$Z$方向の送り速さに関しては$X$, $Y$方向の送り速さより低めに設定する。
\begin{enumerate}[label=\Roman*., ref=\Roman*]
\item $X$, $Y$方向の移動は、{\ttfamily G00}を用いてもよい
\item $Z$方向の移動は、原則として、{\ttfamily G00}を用いた移動はしない
\item $Z$方向の送り速さは、{\ttfamily F15000}以下とする
\item \index{ワーク}ワークへの\index{アプローチ}アプローチの際は、{\ttfamily G01}を用いるものとし、速さは{\ttfamily F6000}以下とする
\item 加工の際は、それぞれの加工や状況に応じて適切なFコード値を設定する
\item ワークからの\index{リトラクト}リトラクトの際は{\ttfamily G01}を用いるものとし、速さは{\ttfamily F12000}以下とする。
\end{enumerate}


\clearpage
\noindent
\dateKouguSpeed における\index{おくりはやさ@送り速さ}送り速さの\index{せっていち(おくりはやさ)@設定値(送り速さ)}設定値は、以下のとおりである。\\

\begin{multicollongtblr}{工具の送り速さ設定値 一覧}{X[l]cc}
内容 & 速さ値 & \ttfamily G\ttNum\\
早送り$XY$:{\ttfamily T50}以外 & 60000 & \ttfamily G00\\
早送り$XY$:{\ttfamily T50}    & 6000 & \\
早送り$Z$:{\ttfamily T50}アプローチ以外 & 18000 &\\
早送り$Z$:{\ttfamily T50}アプローチ    & 6000 & \\
アプローチ:{\ttfamily T50}以外 & 6600 & \\
アプローチ:{\ttfamily T50}    & 1500 & \ttfamily G31\\
アプローチ:工具長測定 & 1000 & \ttfamily G31\\
測定時アプローチ:{\ttfamily T50}片側測定 & 50 & \ttfamily G31\\
測定時アプローチ:{\ttfamily T50}両側測定 & 50 & \ttfamily G31\\
\hline
端面加工:直線             & 1200 &\\
端面加工:コーナー          & 1200 &\\
外削加工:直線             & 400 &\\
外削加工:コーナー          & 400 &\\
溝加工:直線               & 500 &\\
溝加工:コーナー            & 400 &\\
外面取加工:直線            & 500 &\\
外面取加工:コーナー         & 400 &\\
内面取加工:直線            & 400 &\\
内面取加工:コーナー         & 400 &\\
座ぐり加工:直線            & 40 &\\
座ぐり加工:コーナー         & 40 &\\
\dimple 加工:表面アプローチ & 600 &\\
\dimple 加工:加工         & 100 &\\
\end{multicollongtblr}


