%!TEX root = ../RPA_for_Creating_Program_Note.tex


\modHeadchapter[loC]{関連する著作物およびその提示}
\DMC における\index{ソフトウェア}ソフトウェアに関して、\index{バージョンかんり@バージョン管理}バージョン管理や\index{イシューかんり@イシュー管理}イシュー管理はオンライン上の\index{バージョンかんりシステム@バージョン管理システム}バージョン管理システム, \index{ソースコードかんりシステム@ソースコード管理システム}ソースコード管理(\index{SCM}SCM)システム, \index{リポジトリホスティングサービス}リポジトリホスティングサービスを用いて行われている。
これによりコードの共有・バージョン管理・イシュー管理・ビルド・テストなどの機能を用いることができ、生産性が大きく向上している。

このように、開発・\index{ほしゅ@保守}保守等の生産性の向上を目的としたとき、作成された\index{ちょさくぶつ@著作物}著作物をオンライン上に公開・提示することはメリットが大きい。
一方で、その著作物の中には\index{きみつじこう@機密事項}機密事項を含んでいたり、外部の法人・個人による著作物等も含まれるため、安易に公開してはならないものも存在する。

そのため、ここでは\DMC に関する著作物およびその公表・提示について一定の\index{きじゅん(こうかい・ていじ)@基準(公開・提示)}基準を設ける。



%%%%%%%%%%%%%%%%%%%%%%%%%%%%%%%%%%%%%%%%%%%%%%%%%%%%%%%%%%
%% section 20.1 %%%%%%%%%%%%%%%%%%%%%%%%%%%%%%%%%%%%%%%%%%
%%%%%%%%%%%%%%%%%%%%%%%%%%%%%%%%%%%%%%%%%%%%%%%%%%%%%%%%%%
\modHeadsection{関連する著作物}
\index{マシニングセンタ}マシニングセンタによる\index{ワーク}ワークの加工に対して、当社の従業員によって作成されたソフトウェア関連の著作物(以下、\index{かんれんちょさくぶつ@関連著作物}\textbf{関連著作物})として、主に以下のものが挙げられる。
\begin{enumerate}[label=\sarrow]
\item 本書
\item 位置情報等の数値計算用プログラム
\item \index{ないけいテーパひょう(すうちけいさんよう)@内径テーパ表(数値計算用)}数値計算用の内径テーパ(コーナーR含む)データ表
\item \index{しようスペーサけいさんようプログラム@使用スペーサ計算用プログラム}使用スペーサ計算用プログラム
\item 作成した\index{NCプログラム}NCプログラム
%\item モールドのRDB
\item \index{3D CADモデリングようひながた(ワーク)@3D CADモデリング用雛型(ワーク)}ワークの3D CADモデリング用雛型
\item \index{3D CADモデリングようひながた(ないけいテーパ)@3D CADモデリング用雛型(内径テーパ)}内径テーパの3D CADモデリング用雛型
\end{enumerate}
これらは\index{ちょさくぶつのていぎ@著作物の定義}著作物の定義に該当し、\index{ちょさくけんほう@著作権法}著作権法の保護対象となる。



%%%%%%%%%%%%%%%%%%%%%%%%%%%%%%%%%%%%%%%%%%%%%%%%%%%%%%%%%%
%% section 20.2 %%%%%%%%%%%%%%%%%%%%%%%%%%%%%%%%%%%%%%%%%%
%%%%%%%%%%%%%%%%%%%%%%%%%%%%%%%%%%%%%%%%%%%%%%%%%%%%%%%%%%
\modHeadsection{関連著作物の著作権および著作権者}


%%%%%%%%%%%%%%%%%%%%%%%%%%%%%%%%%%%%%%%%%%%%%%%%%%%%%%%%%%
%% subsection 20.2.1 %%%%%%%%%%%%%%%%%%%%%%%%%%%%%%%%%%%%%
%%%%%%%%%%%%%%%%%%%%%%%%%%%%%%%%%%%%%%%%%%%%%%%%%%%%%%%%%%
\subsection{著作人格権}
すべての\index{かんれんちょさくぶつ@関連著作物}関連著作物の\index{ちょさくじんかくけん@著作人格権}著作人格権は、その\index{ちょさくしゃ@著作者}著作者に帰属する。


%%%%%%%%%%%%%%%%%%%%%%%%%%%%%%%%%%%%%%%%%%%%%%%%%%%%%%%%%%
%% subsection 20.2.2 %%%%%%%%%%%%%%%%%%%%%%%%%%%%%%%%%%%%%
%%%%%%%%%%%%%%%%%%%%%%%%%%%%%%%%%%%%%%%%%%%%%%%%%%%%%%%%%%
\subsection{著作財産権}
当社において作成された\DMC における関連著作物が職務上作成された著作物(\index{しょくむちょさくぶつ@職務著作物}職務著作物)に該当する場合、その\index{ちょさくざいさんけん@著作財産権}著作財産権は当社に帰属する。

関連著作物が職務著作物に該当しない場合、その著作財産権は著作者個人に帰属する。



\clearpage
%%%%%%%%%%%%%%%%%%%%%%%%%%%%%%%%%%%%%%%%%%%%%%%%%%%%%%%%%%
%% section 20.4 %%%%%%%%%%%%%%%%%%%%%%%%%%%%%%%%%%%%%%%%%%
%%%%%%%%%%%%%%%%%%%%%%%%%%%%%%%%%%%%%%%%%%%%%%%%%%%%%%%%%%
\modHeadsection{関連著作物の公表}
\index{かんれんちょさくぶつ@関連著作物}関連著作物の\index{ちょさくしゃ@著作者}著作者は、その\index{ちょさくぶつ@著作物}著作物でまだ公表されていないもの(その同意を得ないで公表された著作物を含む)を公表・提示する権利を有する(当該著作物を原著作物とする二次的著作物についても同様)。


%%%%%%%%%%%%%%%%%%%%%%%%%%%%%%%%%%%%%%%%%%%%%%%%%%%%%%%%%%
%% subsection 20.4.1 %%%%%%%%%%%%%%%%%%%%%%%%%%%%%%%%%%%%%
%%%%%%%%%%%%%%%%%%%%%%%%%%%%%%%%%%%%%%%%%%%%%%%%%%%%%%%%%%
\subsection{公表する関連著作物}

%%%%%%%%%%%%%%%%%%%%%%%%%%%%%%%%%%%%%%%%%%%%%%%%%%%%%%%%%%
%% subsubsection 20.4.2.1 %%%%%%%%%%%%%%%%%%%%%%%%%%%%%%%%
%%%%%%%%%%%%%%%%%%%%%%%%%%%%%%%%%%%%%%%%%%%%%%%%%%%%%%%%%%
\subsubsection{生産性の向上および著作権者の同意}
関連著作物については、開発・保守等の生産性の向上を目的に、原則としてすべてオンライン上に提示する。
なお、次節(非公表にする関連著作物)に該当する著作物に関してはその限りではない。

ただし提示は、その著作物におけるすべての\index{ちょさくじんかくけん@著作人格権}著作人格権の保有者(\index{ちょさくしゃ@著作者}著作者)およびすべての\index{ちょさくざいさんけん@著作財産権}著作財産権の保有者の、全員の同意の下で行われることを前提とする。


%%%%%%%%%%%%%%%%%%%%%%%%%%%%%%%%%%%%%%%%%%%%%%%%%%%%%%%%%%
%% subsubsection 20.4.2.2 %%%%%%%%%%%%%%%%%%%%%%%%%%%%%%%%
%%%%%%%%%%%%%%%%%%%%%%%%%%%%%%%%%%%%%%%%%%%%%%%%%%%%%%%%%%
\subsubsection{個人の著作権者による提示にの権利\label{subsec:individualright}}
著作人格権および著作財産権の保有者が同一人物であり、かつ1人の個人のみである場合は、その個人が提示の行為を行ってよいものとする。

%%%%%%%%%%%%%%%%%%%%%%%%%%%%%%%%%%%%%%%%%%%%%%%%%%%%%%%%%%
%% subsubsection 20.4.2.3 %%%%%%%%%%%%%%%%%%%%%%%%%%%%%%%%
%%%%%%%%%%%%%%%%%%%%%%%%%%%%%%%%%%%%%%%%%%%%%%%%%%%%%%%%%%
\subsubsection{データ保護とプライバシー}
公表・提示の際は、\index{こじんじょうほうほごほう@個人情報保護法}\href{https://elaws.e-gov.go.jp/document?lawid=415AC0000000057}{個人情報の保護に関する法律}(個人情報保護法)\cite{online:eGovPersonalInfoProtectionLaw}に基づいて、\index{データほご@データ保護}データ保護・\index{プライバシーほご@プライバシー保護}プライバシー保護に十分に配慮しなければならない。


%%%%%%%%%%%%%%%%%%%%%%%%%%%%%%%%%%%%%%%%%%%%%%%%%%%%%%%%%%
%% subsection 20.4.2 %%%%%%%%%%%%%%%%%%%%%%%%%%%%%%%%%%%%%
%%%%%%%%%%%%%%%%%%%%%%%%%%%%%%%%%%%%%%%%%%%%%%%%%%%%%%%%%%
\subsection{非公表にする関連著作物}

%%%%%%%%%%%%%%%%%%%%%%%%%%%%%%%%%%%%%%%%%%%%%%%%%%%%%%%%%%
%% subsubsection 20.4.2.1 %%%%%%%%%%%%%%%%%%%%%%%%%%%%%%%%
%%%%%%%%%%%%%%%%%%%%%%%%%%%%%%%%%%%%%%%%%%%%%%%%%%%%%%%%%%
\subsubsection{機密情報の保護および提示の範囲\label{subsec:notopenwork}}
作成した\index{メインプログラム}メインプログラムやモールドの\index{データベース(モールド)}データベースについては、個々の\index{めいさい(モールド)@明細(モールド)}明細の情報(\index{きみつじこう@機密事項}機密事項)を推察できるデータを含む。
このような機密事項を含む著作物については、原則として提示はしないものとする。

提示する場合は(個々のものすべてでなく)代表的・典型的なものに留めるか、あるいは許可を得た者のみが閲覧可能な状態として提示を行うものとする。

%%%%%%%%%%%%%%%%%%%%%%%%%%%%%%%%%%%%%%%%%%%%%%%%%%%%%%%%%%
%% subsubsection 20.4.2.2 %%%%%%%%%%%%%%%%%%%%%%%%%%%%%%%%
%%%%%%%%%%%%%%%%%%%%%%%%%%%%%%%%%%%%%%%%%%%%%%%%%%%%%%%%%%
\subsubsection{外部作成の著作物および著作権者の同意\label{subsec:standardscopyrightsSubcontractor}}
\index{NCプログラム}NCプログラムの中には外注先で作成されたものも存在する。
このような\index{ちょさくけんしゃ@著作権者}著作権者(特に著作財産権者)が当社の従業員または当社自体ではない著作物については、原則として公表しない。
公表する場合は、すべての著作人格権者およびすべての著作財産権者の同意の下に行われるものとする。



\clearpage
~\vfill
\begin{Column}{\DMC の関連著作物}
\DMC の社内で作成された\index{ソフトウェアかんれんちょさくぶつ@ソフトウェア関連著作物}ソフトウェア関連著作物については、先にも述べた通りその一部がオンライン上に提示されている。
これは、その\DMC の立上げに関わる必要な(\index{ソフトウェア}ソフトウェアにおける)社内の業務の一切が、\index{かんれんちょさくぶつ@関連著作物}関連著作物の作成開始時から作成終了後(保守・管理を含む)に至るまで、1人の一般職である著作者個人の独力に(管理職・スタッフにより)一任されており、以下のような状態にあることが背景にある。
\tcbline*
\begin{enumerate}[label=\Roman*]
\item 当社がある目的を持って構想した著作物は(ソフトウェアに関しては)全く存在しない。
\item
著作者(一般職)は主に以下のような作業を単独かつ独力で行っており、通常の業務範囲および業務量から大きく逸脱しているのは明白である。
  \begin{enumerate}
  \item[-] ソフトウェアの観点からみた現状の業務フローの調査・整理
  \item[-] 現行の業務フローにおける問題点・改善可能点の抽出
  \item[-] マシニングセンタ導入後の業務フローの計画の策定
  \item[-] マシニングセンタの稼働に対する要件定義
  \item[-] マシニングセンタの稼働に対するシステム設計
  \item[-] マシニングセンタの稼働に対する詳細設計
  \item[-] ソフトウェア開発に関わる諸規定の策定および作成
  \item[-] ソフトウェア開発に関わる諸標準の策定および作成
  \item[-] マシニングセンタ内におけるワークの幾何学的形状の解析計算および体系化
  \item[-] 明細ごとの具体的な数値の自動取得システムの設計
  \item[-] 明細ごとの具体的な数値の自動取得システムの構築および作成
  \item[-] 加工用サブプログラムの作成
  \item[-] 加工用メインプログラムの作成
  \item[-] NCプログラムの実装
  \item[-] NCプログラムに対する統合試運転を除く試運転
  \item[-] NCプログラムの修正保守
  \item[-] NCプログラムの機能追加保守
  \item[-] モールドの関係データベースの設計
  \item[-] モールドの関係データベースの作成
  \item[-] 加工用メインプログラムの自動作成システムの設計
  \item[-] 加工用メインプログラムの自動作成システムの構築および作成
  \item[-] 関連ドキュメントの作成および管理
  \end{enumerate}
\item 関連著作物は著作者個人の氏名あるいはアカウントの下に、オンライン上に提示されている。
\item ソフトウェア作成時において、その\index{ちょさくぶつ@著作物}著作物についての別段の定め等は本書を除いて一切ない。
\end{enumerate}
\tcbline*
したがって、いずれの要件も満たしていないため、その\index{ちょさくざいさんけん@著作財産権}著作財産権は著作者個人に帰属する。
関連著作物(一部)のオンライン上の提示は、\pageautoref{subsec:individualright}に基づいて行われている。
\end{Column}
\clearrightpage

