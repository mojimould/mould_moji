%!TEX root = ../RPA_for_Creating_Program_Note.tex


\modHeadchapter{マシニングセンタにおける寸法}
ここでは\index{プログラム}プログラムを記述する際や、\index{ずめん(モールド)@図面(モールド)}図面・\index{3Dモデル(モールド)}3Dモデルの描画をする際に必要となる、\index{すんぽう@寸法}寸法や\index{こうさ@公差}公差等の取り扱いについて触れる。

なお、以降で述べる水平方向とは、端面のAC方向のことを指す。



%%%%%%%%%%%%%%%%%%%%%%%%%%%%%%%%%%%%%%%%%%%%%%%%%%%%%%%%%%
%% section 13.1 %%%%%%%%%%%%%%%%%%%%%%%%%%%%%%%%%%%%%%%%%%
%%%%%%%%%%%%%%%%%%%%%%%%%%%%%%%%%%%%%%%%%%%%%%%%%%%%%%%%%%
\modHeadsection{寸法における基本事項}


%%%%%%%%%%%%%%%%%%%%%%%%%%%%%%%%%%%%%%%%%%%%%%%%%%%%%%%%%%
%% subsection 13.1.1 %%%%%%%%%%%%%%%%%%%%%%%%%%%%%%%%%%%%%
%%%%%%%%%%%%%%%%%%%%%%%%%%%%%%%%%%%%%%%%%%%%%%%%%%%%%%%%%%
\subsection{寸法公差の取扱い}
全般的に、\index{すんぽうこうさ@寸法公差}寸法公差がある場合、\index{+こうさ@$+$公差}$+$公差と\index{-こうさ@$-$公差}$-$公差の中央(算術平均)を見るものとする。
ただし、\index{ないけいテーパひょう@内径テーパ表}内径テーパ表の数値については、この限りではない。
%%%%%%%%%%%%%%%%%%%%%%%%%%%%%%%%%%%%%%%%%%%%%%%%%%%%%%%%%%
%% hosoku %%%%%%%%%%%%%%%%%%%%%%%%%%%%%%%%%%%%%%%%%%%%%%%%
%%%%%%%%%%%%%%%%%%%%%%%%%%%%%%%%%%%%%%%%%%%%%%%%%%%%%%%%%%
\begin{hosoku}
たとえば、$100^{+0.5}_{\phantom -0}$であれば、100.25とみなす。
\end{hosoku}
%%%%%%%%%%%%%%%%%%%%%%%%%%%%%%%%%%%%%%%%%%%%%%%%%%%%%%%%%%
%%%%%%%%%%%%%%%%%%%%%%%%%%%%%%%%%%%%%%%%%%%%%%%%%%%%%%%%%%
%%%%%%%%%%%%%%%%%%%%%%%%%%%%%%%%%%%%%%%%%%%%%%%%%%%%%%%%%%


%%%%%%%%%%%%%%%%%%%%%%%%%%%%%%%%%%%%%%%%%%%%%%%%%%%%%%%%%%
%% subsection 13.1.2 %%%%%%%%%%%%%%%%%%%%%%%%%%%%%%%%%%%%%
%%%%%%%%%%%%%%%%%%%%%%%%%%%%%%%%%%%%%%%%%%%%%%%%%%%%%%%%%%
\subsection{寸法の優先度}
\index{こうさのあるすんぽう@公差のある寸法}公差のある寸法と\index{こうさのないすんぽう@公差のない寸法}公差のない寸法(\index{かっこすんぽう@括弧寸法}括弧寸法含む)とが共存して記載されている場合、公差のある寸法を優先する。

ただし、\index{とっきじこう(すんぽう)@特記事項(寸法)}特記事項等がある場合は、それを優先するものとする。
%%%%%%%%%%%%%%%%%%%%%%%%%%%%%%%%%%%%%%%%%%%%%%%%%%%%%%%%%%
%% hosoku %%%%%%%%%%%%%%%%%%%%%%%%%%%%%%%%%%%%%%%%%%%%%%%%
%%%%%%%%%%%%%%%%%%%%%%%%%%%%%%%%%%%%%%%%%%%%%%%%%%%%%%%%%%
\begin{hosoku}
たとえば、2つの線の寸法がそれぞれ$12^{+0.1}_{\phantom -0}$, 4.05と記述されていて、かつその和に相当する部分の寸法が16と記述されている場合は、16.10とみなす。
\end{hosoku}
%%%%%%%%%%%%%%%%%%%%%%%%%%%%%%%%%%%%%%%%%%%%%%%%%%%%%%%%%%
%%%%%%%%%%%%%%%%%%%%%%%%%%%%%%%%%%%%%%%%%%%%%%%%%%%%%%%%%%
%%%%%%%%%%%%%%%%%%%%%%%%%%%%%%%%%%%%%%%%%%%%%%%%%%%%%%%%%%



\clearpage
%%%%%%%%%%%%%%%%%%%%%%%%%%%%%%%%%%%%%%%%%%%%%%%%%%%%%%%%%%
%% section 10.2 %%%%%%%%%%%%%%%%%%%%%%%%%%%%%%%%%%%%%%%%%%
%%%%%%%%%%%%%%%%%%%%%%%%%%%%%%%%%%%%%%%%%%%%%%%%%%%%%%%%%%
\modHeadsection{全長および振分長に関する寸法}


%%%%%%%%%%%%%%%%%%%%%%%%%%%%%%%%%%%%%%%%%%%%%%%%%%%%%%%%%%
%% subsection 10.2.1 %%%%%%%%%%%%%%%%%%%%%%%%%%%%%%%%%%%%%
%%%%%%%%%%%%%%%%%%%%%%%%%%%%%%%%%%%%%%%%%%%%%%%%%%%%%%%%%%
\subsection{全長と振分長の公差の関係}
\index{ふりわけちょう@振分長}振分長の\index{こうさ(ふりわけちょう)@公差(振分長)}公差については、\index{ぜんちょう(モールド)@全長(モールド)}全長の\index{こうさ(ぜんちょう)@公差(全長)}公差を\index{トップふりわけちょう@トップ振分長}トップ振分長と\index{ボトムふりわけちょう@ボトム振分長}ボトム振分長とで等分配する。
%%%%%%%%%%%%%%%%%%%%%%%%%%%%%%%%%%%%%%%%%%%%%%%%%%%%%%%%%%
%% hosoku %%%%%%%%%%%%%%%%%%%%%%%%%%%%%%%%%%%%%%%%%%%%%%%%
%%%%%%%%%%%%%%%%%%%%%%%%%%%%%%%%%%%%%%%%%%%%%%%%%%%%%%%%%%
\begin{hosoku}
たとえば、全長が$1000^{\phantom +0}_{-1.0}$でトップ振分長が200であれば、全長の公差分$-0.5$を等分配し、それぞれ$-0.25$, $-0.25$とする。
つまり、トップ振分長は199.75, ボトム振分長は799.75とする
%% footnote %%%%%%%%%%%%%%%%%%%%%
\footnote{\index{ふりわけちゅうしん@振分中心}振分中心からのずれとして考えると、振分長に依らず等分配するのが自然、と捉えることができる。}。
%%%%%%%%%%%%%%%%%%%%%%%%%%%%%%%%%
\end{hosoku}
%%%%%%%%%%%%%%%%%%%%%%%%%%%%%%%%%%%%%%%%%%%%%%%%%%%%%%%%%%
%%%%%%%%%%%%%%%%%%%%%%%%%%%%%%%%%%%%%%%%%%%%%%%%%%%%%%%%%%
%%%%%%%%%%%%%%%%%%%%%%%%%%%%%%%%%%%%%%%%%%%%%%%%%%%%%%%%%%


%%%%%%%%%%%%%%%%%%%%%%%%%%%%%%%%%%%%%%%%%%%%%%%%%%%%%%%%%%
%% subsection 10.2.2 %%%%%%%%%%%%%%%%%%%%%%%%%%%%%%%%%%%%%
%%%%%%%%%%%%%%%%%%%%%%%%%%%%%%%%%%%%%%%%%%%%%%%%%%%%%%%%%%
\subsection{振分長が括弧寸法の場合}
片方の振分長が\index{かっこすんぽう@括弧寸法}括弧寸法の場合は、全長の公差をそのまま括弧寸法に割り当てる。
%%%%%%%%%%%%%%%%%%%%%%%%%%%%%%%%%%%%%%%%%%%%%%%%%%%%%%%%%%
%% hosoku %%%%%%%%%%%%%%%%%%%%%%%%%%%%%%%%%%%%%%%%%%%%%%%%
%%%%%%%%%%%%%%%%%%%%%%%%%%%%%%%%%%%%%%%%%%%%%%%%%%%%%%%%%%
\begin{hosoku}
たとえば、全長が$1000^{\phantom +0}_{-1.0}$でトップ振分長が200, ボトム振分長が(800)であれば、トップ振分長は200, ボトム振分長は799.5とする。
\end{hosoku}
%%%%%%%%%%%%%%%%%%%%%%%%%%%%%%%%%%%%%%%%%%%%%%%%%%%%%%%%%%
%%%%%%%%%%%%%%%%%%%%%%%%%%%%%%%%%%%%%%%%%%%%%%%%%%%%%%%%%%
%%%%%%%%%%%%%%%%%%%%%%%%%%%%%%%%%%%%%%%%%%%%%%%%%%%%%%%%%%


%%%%%%%%%%%%%%%%%%%%%%%%%%%%%%%%%%%%%%%%%%%%%%%%%%%%%%%%%%
%% subsection 11.2.3 %%%%%%%%%%%%%%%%%%%%%%%%%%%%%%%%%%%%%
%%%%%%%%%%%%%%%%%%%%%%%%%%%%%%%%%%%%%%%%%%%%%%%%%%%%%%%%%%
\subsection{振分の調整}
\index{ふりわけちょう@振分長}振分長の調整を行う場合は、\index{スペーサ}スペーサまたは\index{テーブルかいてん(ふりわけちょうせい)@テーブル回転(振分調整)}テーブル回転のどちらかを用いて行うものとする。

%%%%%%%%%%%%%%%%%%%%%%%%%%%%%%%%%%%%%%%%%%%%%%%%%%%%%%%%%%
%% subsubsection 01.1.2.3 %%%%%%%%%%%%%%%%%%%%%%%%%%%%%%%%
%%%%%%%%%%%%%%%%%%%%%%%%%%%%%%%%%%%%%%%%%%%%%%%%%%%%%%%%%%
\subsubsection{スペーサによる調整}
スペーサは原則として\index{ジグ}ジグのトップ側の\index{うけいた@受板}受板に設置する。
厚さ$\delta_\mathrm s$の\index{スペーサ}スペーサによる調整を行う場合、もともとのトップ振分長$f_\mathrm T$に対し、\index{トップふりわけちょう@トップ振分長}トップ側の振分長(\index{トップさいふりわけちょう@トップ再振分長}トップ再振分長)$f'_\mathrm T$を以下のように調整する。
\begin{align*}
  f'_\mathrm T
  = f_\mathrm T
    +\sqrt{R_\mathrm i'^2-\frac{\delta_\mathrm s^2+(2\bar l)^2}4}\frac{\delta_\mathrm s}{\sqrt{\delta_\mathrm s^2+(2\bar l)^2}}\qquad
    \left(R_\mathrm i' = R_\mathrm c-\frac{W_x}2-\rho~,~~\bar l = l-\frac\sigma2\right).
\end{align*}
$R_\textrm c$, $W_x$, $l$, $\rho$, $\sigma$はそれぞれ中心湾曲半径, AC方向の外径, ジグ幅の半分, 受板の半径, 受板の幅を示す。

%%%%%%%%%%%%%%%%%%%%%%%%%%%%%%%%%%%%%%%%%%%%%%%%%%%%%%%%%%
%% subsubsection 01.1.2.3 %%%%%%%%%%%%%%%%%%%%%%%%%%%%%%%%
%%%%%%%%%%%%%%%%%%%%%%%%%%%%%%%%%%%%%%%%%%%%%%%%%%%%%%%%%%
\subsubsection{テーブル回転による調整}
\index{かたむきかく(ふりわけちょうせい)@傾き角(振分調整)}角度$\theta$だけテーブル回転をして調整を行う場合は、もともとの\index{トップふりわけちょう@トップ振分長}トップ振分長$f_\mathrm T$に対し、トップ側の振分長(\index{トップさいふりわけちょう@トップ再振分長}再振分長)$f'_\mathrm T$を以下のように調整する。
\begin{align*}
  f_\mathrm T'
  = f_\mathrm T+\left(\Delta+\sqrt{R_\mathrm i'-\bar l^2}\right)\sin\theta\qquad
    \left(R_\mathrm i' = R_\mathrm c-\frac{W_x}2-\rho~,~~\bar l = l-\frac\sigma2\right).
\end{align*}
$R_\textrm c$, $W_x$, $l$, $\rho$, $\sigma$, $\Delta$はそれぞれ中心湾曲半径, AC方向の外径, ジグ幅の半分, 受板の半径, 受板の幅, 受板中心とテーブル中心との水平距離を示す。



\clearpage
%%%%%%%%%%%%%%%%%%%%%%%%%%%%%%%%%%%%%%%%%%%%%%%%%%%%%%%%%%
%% section 13.3 %%%%%%%%%%%%%%%%%%%%%%%%%%%%%%%%%%%%%%%%%%
%%%%%%%%%%%%%%%%%%%%%%%%%%%%%%%%%%%%%%%%%%%%%%%%%%%%%%%%%%
\modHeadsection{外径に関する寸法}
\nameCenterCurvature を$R_\mathrm c$, トップ振分長を$f_\mathrm T$, 外径を$W_x$とすると、\nameTopEndFace 部の水平方向の長さ$W_\mathrm T$は以下で与えられる。(\nameBottomEndFace 部も同様)
\begin{align*}
  W_\mathrm T
  = \sqrt{\left(R_\mathrm c+\frac{W_x}2\right)^2-f_\mathrm T^2}
    -\sqrt{\left(R_\mathrm c-\frac{W_x}2\right)^2-f_\mathrm T^2}\ .
\end{align*}
なお、$(\nicefrac{f_\mathrm T}{R_\mathrm c})^2$が十分小さい場合は、$W_\mathrm T$は
\begin{align*}
  W_\mathrm T \simeq W_x\left(1+\frac{f_\mathrm T^2}{2R^2}\right)
\end{align*}
とみなしてもよいものとする。



%%%%%%%%%%%%%%%%%%%%%%%%%%%%%%%%%%%%%%%%%%%%%%%%%%%%%%%%%%
%% section 10.4 %%%%%%%%%%%%%%%%%%%%%%%%%%%%%%%%%%%%%%%%%%
%%%%%%%%%%%%%%%%%%%%%%%%%%%%%%%%%%%%%%%%%%%%%%%%%%%%%%%%%%
\modHeadsection{内径に関する寸法}

%%%%%%%%%%%%%%%%%%%%%%%%%%%%%%%%%%%%%%%%%%%%%%%%%%%%%%%%%%
%% subsection 04.4.1 %%%%%%%%%%%%%%%%%%%%%%%%%%%%%%%%%%%%%
%%%%%%%%%%%%%%%%%%%%%%%%%%%%%%%%%%%%%%%%%%%%%%%%%%%%%%%%%%
\subsection{内径テーパ表の公差}
\index{ないけいテーパひょう@内径テーパ表}内径テーパ表を参照する際は、\index{ぜんちょう@全長}全長の\index{こうさ@公差}公差は考慮しないものとする。
また、トップ端からの距離の\index{ピッチ(ないけいテーパひょう)@ピッチ(内径テーパ表)}ピッチも、同様に公差は考慮しないものとする。
%%%%%%%%%%%%%%%%%%%%%%%%%%%%%%%%%%%%%%%%%%%%%%%%%%%%%%%%%%
%% hosoku %%%%%%%%%%%%%%%%%%%%%%%%%%%%%%%%%%%%%%%%%%%%%%%%
%%%%%%%%%%%%%%%%%%%%%%%%%%%%%%%%%%%%%%%%%%%%%%%%%%%%%%%%%%
\begin{hosoku}
たとえば、全長が$800^{+0.5}_{\phantom -0}$, トップ振分長が400, ピッチが25である場合を考える。
このとき、トップ端は\index{ふりわけちゅうしん@振分中心}振分中心から400の位置にあり、ピッチは25であるものとし、両端についてはそれを適宜延長して調整する。
\end{hosoku}
%%%%%%%%%%%%%%%%%%%%%%%%%%%%%%%%%%%%%%%%%%%%%%%%%%%%%%%%%%
%%%%%%%%%%%%%%%%%%%%%%%%%%%%%%%%%%%%%%%%%%%%%%%%%%%%%%%%%%
%%%%%%%%%%%%%%%%%%%%%%%%%%%%%%%%%%%%%%%%%%%%%%%%%%%%%%%%%%

%%%%%%%%%%%%%%%%%%%%%%%%%%%%%%%%%%%%%%%%%%%%%%%%%%%%%%%%%%
%% subsection 04.4.1 %%%%%%%%%%%%%%%%%%%%%%%%%%%%%%%%%%%%%
%%%%%%%%%%%%%%%%%%%%%%%%%%%%%%%%%%%%%%%%%%%%%%%%%%%%%%%%%%
\subsection{内径テーパ表にない内径}
内径テーパ表におけるトップ端からの距離$\lambda_i$ ($i = 0, 1, 2, \cdots$), それに対するAC方向の内径$w_{\mathrm Ai}$に対し、トップ端から$\lambda$の位置にある\index{ないけい(ACほうこう)@内径(AC方向)}内径$w_{\mathrm A\lambda}$は、
\begin{align*}
  w_{\mathrm A\lambda}
  = \frac{(\lambda-\lambda_i)w_{\mathrm Ai+1}+(\lambda_{i+1}-\lambda)w_{\mathrm Ai}}{\lambda_{i+1}-\lambda_i}
  \qquad
  \Big(\lambda_i \leq \lambda < \lambda_{i+1}\Big)
\end{align*}
とみなしてもよいものとする。
\index{ないけい(BDほうこう)@内径(BD方向)}BD方向の内径$w_{\mathrm B\lambda}$についても同様である。

%%%%%%%%%%%%%%%%%%%%%%%%%%%%%%%%%%%%%%%%%%%%%%%%%%%%%%%%%%
%% subsection 11.4.3 %%%%%%%%%%%%%%%%%%%%%%%%%%%%%%%%%%%%%
%%%%%%%%%%%%%%%%%%%%%%%%%%%%%%%%%%%%%%%%%%%%%%%%%%%%%%%%%%
\subsection{水平方向の内径}
中心湾曲線上のトップ端から$\lambda$の位置における水平方向の\index{ないけい(すいへいほうこう)@内径(水平方向)}内径は、トップ端から$\lambda$の位置におけるAC方向の内径$w_{A\lambda}$とみなしてもよいものとする。

%%%%%%%%%%%%%%%%%%%%%%%%%%%%%%%%%%%%%%%%%%%%%%%%%%%%%%%%%%
%% subsection 11.4.4 %%%%%%%%%%%%%%%%%%%%%%%%%%%%%%%%%%%%%
%%%%%%%%%%%%%%%%%%%%%%%%%%%%%%%%%%%%%%%%%%%%%%%%%%%%%%%%%%
\subsection{めっき膜厚の考慮}
後工程にて内面に\index{めっき}めっきを施す場合は、\index{めっきまくあつ@めっき膜厚}めっき膜厚$\mu$を考慮した上で内径の調整を行うものとする。
このとき、AC方向の内径を
\begin{align*}
  w_{\mathrm A\lambda}' = w_{\mathrm A\lambda}+2\mu
\end{align*}
とみなすものとする。
BD方向の内径$w_{\mathrm B\lambda}'$についても同様である。



\clearpage
%%%%%%%%%%%%%%%%%%%%%%%%%%%%%%%%%%%%%%%%%%%%%%%%%%%%%%%%%%
%% section 13.5 %%%%%%%%%%%%%%%%%%%%%%%%%%%%%%%%%%%%%%%%%%
%%%%%%%%%%%%%%%%%%%%%%%%%%%%%%%%%%%%%%%%%%%%%%%%%%%%%%%%%%
\modHeadsection{端面加工に関する寸法}


%%%%%%%%%%%%%%%%%%%%%%%%%%%%%%%%%%%%%%%%%%%%%%%%%%%%%%%%%%
%% subsection 10.5.1 %%%%%%%%%%%%%%%%%%%%%%%%%%%%%%%%%%%%%
%%%%%%%%%%%%%%%%%%%%%%%%%%%%%%%%%%%%%%%%%%%%%%%%%%%%%%%%%%
\subsection{端面加工の基準点}
\index{たんめんかこう@端面加工}端面加工の\index{きじゅん(たんめんかこう)@基準(端面加工)}基準は、\index{ないけいちゅうしん(たんめん)@内径中心(端面)}端面における内径中心を基準として行うものとする。


%%%%%%%%%%%%%%%%%%%%%%%%%%%%%%%%%%%%%%%%%%%%%%%%%%%%%%%%%%
%% subsection 11.5.1 %%%%%%%%%%%%%%%%%%%%%%%%%%%%%%%%%%%%%
%%%%%%%%%%%%%%%%%%%%%%%%%%%%%%%%%%%%%%%%%%%%%%%%%%%%%%%%%%
\subsection{工具補正:端面加工}

%%%%%%%%%%%%%%%%%%%%%%%%%%%%%%%%%%%%%%%%%%%%%%%%%%%%%%%%%%
%% subsubsection 10.7.3.1 %%%%%%%%%%%%%%%%%%%%%%%%%%%%%%%%
%%%%%%%%%%%%%%%%%%%%%%%%%%%%%%%%%%%%%%%%%%%%%%%%%%%%%%%%%%
\subsubsection{工具長補正:端面加工}
端面加工に使用する\index{フェイスミル}フェイスミルの\index{こうぐちょう(フェイスミル)@工具長(フェイスミル)}工具長は、その刃の\index{せんたんぶ(フェイスミル)@先端部(フェイスミル)}先端部を工具長として\index{オフセット(こうぐちょうほせい)@オフセット(工具長補正)}オフセット量の設定を行うものとする。

なお、このとき工具長の\index{まもうりょう(こうぐちょうほせい)@摩耗量(工具長補正)}摩耗量は0とする。

%%%%%%%%%%%%%%%%%%%%%%%%%%%%%%%%%%%%%%%%%%%%%%%%%%%%%%%%%%
%% subsubsection 10.7.3.1 %%%%%%%%%%%%%%%%%%%%%%%%%%%%%%%%
%%%%%%%%%%%%%%%%%%%%%%%%%%%%%%%%%%%%%%%%%%%%%%%%%%%%%%%%%%
\subsubsection{工具径補正:端面加工}
端面加工に使用するフェイスミルの\index{こうぐけい(フェイスミル)@工具径(フェイスミル)}工具長は、その刃の径方向の先端部を工具径として\index{オフセット(こうぐけいほせい)@オフセット(工具径補正)}オフセット量の設定を行うものとする。

なお、このとき工具径の\index{まもうりょう(こうぐけいほせい)@摩耗量(工具径補正)}摩耗量は0とする。


%%%%%%%%%%%%%%%%%%%%%%%%%%%%%%%%%%%%%%%%%%%%%%%%%%%%%%%%%%
%% subsection 11.5.1 %%%%%%%%%%%%%%%%%%%%%%%%%%%%%%%%%%%%%
%%%%%%%%%%%%%%%%%%%%%%%%%%%%%%%%%%%%%%%%%%%%%%%%%%%%%%%%%%
\subsection{端面加工のコーナーR}
端面加工の際の\index{コーナーR(たんめんかこう)@コーナーR(端面加工)}コーナーRは、\index{ないけいコーナーR(たんめん)@内径コーナーR(端面)}端面における内径のコーナーRとする。



\clearpage
%%%%%%%%%%%%%%%%%%%%%%%%%%%%%%%%%%%%%%%%%%%%%%%%%%%%%%%%%%
%% section 10.06 %%%%%%%%%%%%%%%%%%%%%%%%%%%%%%%%%%%%%%%%%
%%%%%%%%%%%%%%%%%%%%%%%%%%%%%%%%%%%%%%%%%%%%%%%%%%%%%%%%%%
\modHeadsection{外削加工に関する寸法}


%%%%%%%%%%%%%%%%%%%%%%%%%%%%%%%%%%%%%%%%%%%%%%%%%%%%%%%%%%
%% subsection 10.5.1 %%%%%%%%%%%%%%%%%%%%%%%%%%%%%%%%%%%%%
%%%%%%%%%%%%%%%%%%%%%%%%%%%%%%%%%%%%%%%%%%%%%%%%%%%%%%%%%%
\subsection{外削加工の基準点}
\index{がいさくかこう@外削加工}外削加工の\index{きじゅん(がいさくかこう)@基準(外削加工)}基準は、\index{がいさくちゅうしん@外削中心}外削中心を基準として行うものとする。


%%%%%%%%%%%%%%%%%%%%%%%%%%%%%%%%%%%%%%%%%%%%%%%%%%%%%%%%%%
%% subsection 10.06.1 %%%%%%%%%%%%%%%%%%%%%%%%%%%%%%%%%%%%
%%%%%%%%%%%%%%%%%%%%%%%%%%%%%%%%%%%%%%%%%%%%%%%%%%%%%%%%%%
\subsection{工具補正:外削加工}


%%%%%%%%%%%%%%%%%%%%%%%%%%%%%%%%%%%%%%%%%%%%%%%%%%%%%%%%%%
%% subsubsection 10.06.1.1 %%%%%%%%%%%%%%%%%%%%%%%%%%%%%%%
%%%%%%%%%%%%%%%%%%%%%%%%%%%%%%%%%%%%%%%%%%%%%%%%%%%%%%%%%%
\subsubsection{工具長補正:外削加工}
外削加工に使用する\index{スクエアエンドミル}スクエアエンドミルの\index{こうぐちょう(スクエアエンドミル)@工具長(スクエアエンドミル)}工具長は、その刃の\index{せんたんぶ(スクエアエンドミル)@先端部(スクエアエンドミル)}先端部を工具長として\index{オフセット(こうぐちょうほせい)@オフセット(工具長補正)}オフセット量の設定を行うものとする。

なお、このとき工具長の\index{まもうりょう(こうぐちょうほせい)@摩耗量(工具長補正)}摩耗量は0とする。


%%%%%%%%%%%%%%%%%%%%%%%%%%%%%%%%%%%%%%%%%%%%%%%%%%%%%%%%%%
%% subsubsection 10.06.1.2 %%%%%%%%%%%%%%%%%%%%%%%%%%%%%%%
%%%%%%%%%%%%%%%%%%%%%%%%%%%%%%%%%%%%%%%%%%%%%%%%%%%%%%%%%%
\subsubsection{工具径補正:外削加工}
外削加工に使用するスクエアエンドミルの\index{こうぐけい(スクエアエンドミル)@工具径(スクエアエンドミル)}工具長は、その刃の径方向の先端部を工具径として\index{オフセット(こうぐけいほせい)@オフセット(工具径補正)}オフセット量の設定を行うものとする。

なお、このとき工具径の\index{まもうりょう(こうぐけいほせい)@摩耗量(工具径補正)}摩耗量は0とする。


%%%%%%%%%%%%%%%%%%%%%%%%%%%%%%%%%%%%%%%%%%%%%%%%%%%%%%%%%%
%% subsection 10.06.2 %%%%%%%%%%%%%%%%%%%%%%%%%%%%%%%%%%%%
%%%%%%%%%%%%%%%%%%%%%%%%%%%%%%%%%%%%%%%%%%%%%%%%%%%%%%%%%%
\subsection{外削長}
\index{がいさくちょう@外削長}外削長の寸法は、\index{たんめん@端面}端面に垂直な方向の値とする。
また\index{トップがわのがいさくちょう@トップ側の外削長}トップ側の外削長については、\index{みぞはば@溝幅}溝幅の寸法も含むものとする。
このとき、外削長$h_\mathrm T$が、\index{みぞいち@溝位置}溝位置$\kappa_p$と溝幅$\kappa_w$の和に等しい場合は、
\begin{align*}
  h_\mathrm T = \kappa_p+1[\text{mm}]
\end{align*}
とみなして加工を行うものとする。


%%%%%%%%%%%%%%%%%%%%%%%%%%%%%%%%%%%%%%%%%%%%%%%%%%%%%%%%%%
%% subsection 10.06.3 %%%%%%%%%%%%%%%%%%%%%%%%%%%%%%%%%%%%
%%%%%%%%%%%%%%%%%%%%%%%%%%%%%%%%%%%%%%%%%%%%%%%%%%%%%%%%%%
\subsection{湾曲に沿った外削\TBW}
(to be written ...)


\clearpage
%%%%%%%%%%%%%%%%%%%%%%%%%%%%%%%%%%%%%%%%%%%%%%%%%%%%%%%%%%
%% section 11.6 %%%%%%%%%%%%%%%%%%%%%%%%%%%%%%%%%%%%%%%%%%
%%%%%%%%%%%%%%%%%%%%%%%%%%%%%%%%%%%%%%%%%%%%%%%%%%%%%%%%%%
\modHeadsection{溝加工に関する寸法}


%%%%%%%%%%%%%%%%%%%%%%%%%%%%%%%%%%%%%%%%%%%%%%%%%%%%%%%%%%
%% subsection 10.7.1 %%%%%%%%%%%%%%%%%%%%%%%%%%%%%%%%%%%%%
%%%%%%%%%%%%%%%%%%%%%%%%%%%%%%%%%%%%%%%%%%%%%%%%%%%%%%%%%%
\subsection{溝加工の基準点\TBW}
(to be written ...)


%%%%%%%%%%%%%%%%%%%%%%%%%%%%%%%%%%%%%%%%%%%%%%%%%%%%%%%%%%
%% subsection 10.06.1 %%%%%%%%%%%%%%%%%%%%%%%%%%%%%%%%%%%%
%%%%%%%%%%%%%%%%%%%%%%%%%%%%%%%%%%%%%%%%%%%%%%%%%%%%%%%%%%
\subsection{工具補正:溝加工}


%%%%%%%%%%%%%%%%%%%%%%%%%%%%%%%%%%%%%%%%%%%%%%%%%%%%%%%%%%
%% subsubsection 10.06.1.1 %%%%%%%%%%%%%%%%%%%%%%%%%%%%%%%
%%%%%%%%%%%%%%%%%%%%%%%%%%%%%%%%%%%%%%%%%%%%%%%%%%%%%%%%%%
\subsubsection{工具長補正:溝加工}
\index{みぞかこう@溝加工}溝加工に使用する\index{サイドカッター}サイドカッターの\index{こうぐちょう(サイドカッター)@工具長(サイドカッター)}工具長は、その切削部(刃の部分)の\index{せんたんぶ(サイドカッター)@先端部(サイドカッター)}先端部を工具長として\index{オフセット(こうぐちょうほせい)@オフセット(工具長補正)}オフセット量の設定を行うものとする。

なお、このとき工具長の\index{まもうりょう(こうぐちょうほせい)@摩耗量(工具長補正)}摩耗量は0とする。


%%%%%%%%%%%%%%%%%%%%%%%%%%%%%%%%%%%%%%%%%%%%%%%%%%%%%%%%%%
%% subsubsection 10.06.1.2 %%%%%%%%%%%%%%%%%%%%%%%%%%%%%%%
%%%%%%%%%%%%%%%%%%%%%%%%%%%%%%%%%%%%%%%%%%%%%%%%%%%%%%%%%%
\subsubsection{工具径補正:溝加工}
溝加工に使用するサイドカッターの\index{こうぐけい(サイドカッター)@工具径(サイドカッター)}工具長は、その刃の径方向の先端部を工具径として\index{オフセット(こうぐけいほせい)@オフセット(工具径補正)}オフセット量の設定を行うものとする。

なお、このとき工具径の\index{まもうりょう(こうぐけいほせい)@摩耗量(工具径補正)}摩耗量は0とする。


%%%%%%%%%%%%%%%%%%%%%%%%%%%%%%%%%%%%%%%%%%%%%%%%%%%%%%%%%%
%% subsection 11.6.2 %%%%%%%%%%%%%%%%%%%%%%%%%%%%%%%%%%%%%
%%%%%%%%%%%%%%%%%%%%%%%%%%%%%%%%%%%%%%%%%%%%%%%%%%%%%%%%%%
\subsection{溝位置および溝幅}
トップ端から垂直方向に、\index{みぞ@溝}溝のトップ側の端までの距離を\index{みぞいち@溝位置}溝位置とする。
また、同様の方向に、溝のトップ側の端から溝のボトム側の端までの距離を\index{みぞはば@溝幅}溝幅とする。


%%%%%%%%%%%%%%%%%%%%%%%%%%%%%%%%%%%%%%%%%%%%%%%%%%%%%%%%%%
%% subsection 11.6.2 %%%%%%%%%%%%%%%%%%%%%%%%%%%%%%%%%%%%%
%%%%%%%%%%%%%%%%%%%%%%%%%%%%%%%%%%%%%%%%%%%%%%%%%%%%%%%%%%
\subsection{溝深さ}
トップ側に外削がなく、かつ\index{Aがわみぞふかさ@A側溝深さ}A側溝深さが\index{こうさ@公差}公差のある寸法$\kappa_d'$を持つ場合、\index{みぞはばちゅうおう@溝幅中央}溝幅中央における溝A側面とA側外面との距離$\kappa_d$は以下のものとみなす。
\begin{gather*}
  \kappa_d
  = \frac{2\kappa_d'-\kappa_w\sin\zeta}{1+\cos^2\zeta}\cos\zeta
    +\sqrt{R_\mathrm o^2-\left(f_\mathrm T-\kappa_p-\frac{\kappa_w}2\right)^2}
    -\sqrt{R_\mathrm o^2-\left(f_\mathrm T-\kappa_p\right)^2}\\[3pt]
  \left(
  \tan\zeta
  = \frac{\sqrt{R_\mathrm o^2-\left(f_\mathrm T-\kappa_p-\kappa_w\right)^2}
          -\sqrt{R_\mathrm o^2-\left(f_\mathrm T-\kappa_p\right)^2}}
         {\kappa_w}
    ~~, \quad
    R_\mathrm o = R_\mathrm c+\frac{W_x}2
  \right).
\end{gather*}
$R_\mathrm c$, $W_x$, $f_\mathrm T$, $\kappa_p$, $\kappa_w$はそれぞれ中心湾曲半径, 外径, トップ振分長, 溝位置, 溝幅を示す。

なお、$(\nicefrac{f_\mathrm T}{R_\mathrm c})^2$が十分小さい場合は、
\begin{align*}
  \kappa_d \simeq \kappa_d'+\frac{\kappa_w^2}{8R_\mathrm o}
\end{align*}
とみなしてもよいものとする。



\clearpage
%%%%%%%%%%%%%%%%%%%%%%%%%%%%%%%%%%%%%%%%%%%%%%%%%%%%%%%%%%
%% section 12.7 %%%%%%%%%%%%%%%%%%%%%%%%%%%%%%%%%%%%%%%%%%
%%%%%%%%%%%%%%%%%%%%%%%%%%%%%%%%%%%%%%%%%%%%%%%%%%%%%%%%%%
\modHeadsection{端面C面取に関する寸法}


%%%%%%%%%%%%%%%%%%%%%%%%%%%%%%%%%%%%%%%%%%%%%%%%%%%%%%%%%%
%% subsection 10.8.1 %%%%%%%%%%%%%%%%%%%%%%%%%%%%%%%%%%%%%
%%%%%%%%%%%%%%%%%%%%%%%%%%%%%%%%%%%%%%%%%%%%%%%%%%%%%%%%%%
\subsection{端面C面取加工の基準点}

%%%%%%%%%%%%%%%%%%%%%%%%%%%%%%%%%%%%%%%%%%%%%%%%%%%%%%%%%%
%% subsubsection 10.8.1.1 %%%%%%%%%%%%%%%%%%%%%%%%%%%%%%%%
%%%%%%%%%%%%%%%%%%%%%%%%%%%%%%%%%%%%%%%%%%%%%%%%%%%%%%%%%%
\subsubsection{端面外側C面取加工の基準点}
\index{たんめんそとがわCめんとりかこう@端面外側C面取加工}端面外側C面取加工の\index{きじゅん(たんめんそとがわCめんとりかこう)@基準(端面外側C面取加工)}基準は、外削のある場合は\index{がいさくちゅうしん@外削中心}外削中心を基準とし、外削のない場合は\index{がいけいちゅうしん(たんめん)@外径中心(端面)}端面における外径中心を基準として行うものとする。

%%%%%%%%%%%%%%%%%%%%%%%%%%%%%%%%%%%%%%%%%%%%%%%%%%%%%%%%%%
%% subsubsection 10.8.1.1 %%%%%%%%%%%%%%%%%%%%%%%%%%%%%%%%
%%%%%%%%%%%%%%%%%%%%%%%%%%%%%%%%%%%%%%%%%%%%%%%%%%%%%%%%%%
\subsubsection{端面内側C面取加工の基準点}
\index{たんめんうちがわCめんとりかこう@端面内側C面取加工}端面内側C面取加工の\index{きじゅん(たんめんうちがわCめんとりかこう)@基準(端面内側C面取加工)}基準は、\index{ないけいちゅうしん(たんめん)@内径中心(端面)}端面における内径中心を基準として行うものとする。


%%%%%%%%%%%%%%%%%%%%%%%%%%%%%%%%%%%%%%%%%%%%%%%%%%%%%%%%%%
%% subsection 11.6.2 %%%%%%%%%%%%%%%%%%%%%%%%%%%%%%%%%%%%%
%%%%%%%%%%%%%%%%%%%%%%%%%%%%%%%%%%%%%%%%%%%%%%%%%%%%%%%%%%
\subsection{端面C面取の寸法}
\index{たんめんそとがわCめんとり@端面外側C面取}端面外側C面取の寸法$c_\mathrm o$ならびに\index{たんめんうちがわCめんとり@端面内側C面取}端面内側C面取の寸法$c_\mathrm i$は、\index{たんめん@端面}端面に垂直な方向の距離とみなす。
このとき、\index{かたかく(テーパエンドミル)@片角(テーパエンドミル)}片角が$\xi_\mathrm e$の\index{テーパエンドミル}テーパエンドミルに対して、\index{Cめんとり(たんめん)@C面取(端面)}C面取の$XY$方向の寸法は、$c_\mathrm o\tan\xi_\mathrm e$および$c_\mathrm i\tan\xi_\mathrm e$で与えられる。


%%%%%%%%%%%%%%%%%%%%%%%%%%%%%%%%%%%%%%%%%%%%%%%%%%%%%%%%%%
%% subsection 11.6.2 %%%%%%%%%%%%%%%%%%%%%%%%%%%%%%%%%%%%%
%%%%%%%%%%%%%%%%%%%%%%%%%%%%%%%%%%%%%%%%%%%%%%%%%%%%%%%%%%
\subsection{工具補正}

%%%%%%%%%%%%%%%%%%%%%%%%%%%%%%%%%%%%%%%%%%%%%%%%%%%%%%%%%%
%% subsubsection 10.7.3.1 %%%%%%%%%%%%%%%%%%%%%%%%%%%%%%%%
%%%%%%%%%%%%%%%%%%%%%%%%%%%%%%%%%%%%%%%%%%%%%%%%%%%%%%%%%%
\subsubsection{工具長補正}
\index{Cめんとり@C面取}C面取に使用する\index{テーパエンドミル}テーパエンドミルの\index{こうぐちょう(テーパエンドミル)@工具長(テーパエンドミル)}工具長は、その\index{せんたんぶ(テーパエンドミル)@先端部(テーパエンドミル)}先端部から軸方向に適当な距離$d_\mathrm e$を差し引いた長さを工具長として\index{オフセット(こうぐちょうほせい)@オフセット(工具長補正)}オフセット量の設定を行うものとする。

したがって、工具の先端部を測定し、先端から$d_\mathrm e$を引いた値を\index{こうぐちょうほせいち@工具長補正値}工具長補正値とする。
なお、このとき工具長の\index{まもうりょう(こうぐちょうほせい)@摩耗量(工具長補正)}摩耗量は0とする。

%%%%%%%%%%%%%%%%%%%%%%%%%%%%%%%%%%%%%%%%%%%%%%%%%%%%%%%%%%
%% subsubsection 10.7.3.1 %%%%%%%%%%%%%%%%%%%%%%%%%%%%%%%%
%%%%%%%%%%%%%%%%%%%%%%%%%%%%%%%%%%%%%%%%%%%%%%%%%%%%%%%%%%
\subsubsection{工具径補正}
\index{せんたんけい(テーパエンドミル)@先端径(テーパエンドミル)}先端径(直径)および先端の\index{かたかく(テーパエンドミル)@片角(テーパエンドミル)}片角がそれぞれ$D_\mathrm e$, $\xi_\mathrm e$の\index{テーパエンドミル}テーパエンドミルに対し、その\index{こうぐけいほせいち@工具径補正値}工具径補正値を$\nicefrac{D_\mathrm e}2$として設定を行い、さらに\index{こうぐけいまもうりょう@工具径摩耗量}工具径摩耗量を$d_\mathrm e\tan\xi_\mathrm e$として設定を行うものとする。
ここで$d_\mathrm e$は、\index{こうぐちょうほせい@工具長補正}工具長補正において用いたものとする。


%%%%%%%%%%%%%%%%%%%%%%%%%%%%%%%%%%%%%%%%%%%%%%%%%%%%%%%%%%
%% subsection 10.7.3 %%%%%%%%%%%%%%%%%%%%%%%%%%%%%%%%%%%%%
%%%%%%%%%%%%%%%%%%%%%%%%%%%%%%%%%%%%%%%%%%%%%%%%%%%%%%%%%%
\subsection{端面外側C面取加工}

%%%%%%%%%%%%%%%%%%%%%%%%%%%%%%%%%%%%%%%%%%%%%%%%%%%%%%%%%%
%% subsubsection 10.7.3.1 %%%%%%%%%%%%%%%%%%%%%%%%%%%%%%%%
%%%%%%%%%%%%%%%%%%%%%%%%%%%%%%%%%%%%%%%%%%%%%%%%%%%%%%%%%%
\subsubsection{面取Cが小さい場合}
\index{そとがわCめんとり(たんめん)@外側C面取(端面)}端面の外側C面取について、その大きさが小さい場合は、\index{てもちけんまき@手持ち研磨機}手持ち研磨機を用いた手動による加工で行ってもよいものとする。

%%%%%%%%%%%%%%%%%%%%%%%%%%%%%%%%%%%%%%%%%%%%%%%%%%%%%%%%%%
%% subsubsection 10.7.3.2 %%%%%%%%%%%%%%%%%%%%%%%%%%%%%%%%
%%%%%%%%%%%%%%%%%%%%%%%%%%%%%%%%%%%%%%%%%%%%%%%%%%%%%%%%%%
\subsubsection{面取Cが大きい場合}
マシニングセンタを用いて加工する場合、外削のない場合は、加工径の中心座標$X$をトップ側・ボトム側のそれぞれに対して以下だけ補正する。
\begin{align*}
  \text{トップ側:}&~~
  \sqrt{R_\mathrm c^2-\left(f_\mathrm T-c_\mathrm{To}\right)^2}-\sqrt{R_\mathrm c^2-f_\mathrm T^2}\ ,\\
  \text{ボトム側:}&~~
  \sqrt{R_\mathrm c^2-f_\mathrm B^2}-\sqrt{R_\mathrm c^2-\left(f_\mathrm B-c_\mathrm{Bo}\right)^2}\ .
\end{align*}
ここで$c_\mathrm{To}$, $c_\mathrm{Bo}$, $R_\mathrm c$, $f_\mathrm T$, $f_\mathrm B$はそれぞれトップ端の外側C面取の大きさ, ボトム端の外側C面取の大きさ, 中心湾曲半径, トップ振分長, ボトム振分長を示す。


\clearpage
%%%%%%%%%%%%%%%%%%%%%%%%%%%%%%%%%%%%%%%%%%%%%%%%%%%%%%%%%%
%% subsection 10.7.4 %%%%%%%%%%%%%%%%%%%%%%%%%%%%%%%%%%%%%
%%%%%%%%%%%%%%%%%%%%%%%%%%%%%%%%%%%%%%%%%%%%%%%%%%%%%%%%%%
\subsection{端面内側C面取加工}

%%%%%%%%%%%%%%%%%%%%%%%%%%%%%%%%%%%%%%%%%%%%%%%%%%%%%%%%%%
%% subsubsection 10.7.4.1 %%%%%%%%%%%%%%%%%%%%%%%%%%%%%%%%
%%%%%%%%%%%%%%%%%%%%%%%%%%%%%%%%%%%%%%%%%%%%%%%%%%%%%%%%%%
\subsubsection{面取Cが小さい場合}
\index{うちがわCめんとり(たんめん)@内側C面取(端面)}端面の内側C面取について、その大きさが小さい場合は、\index{てもちけんまき@手持ち研磨機}手持ち研磨機を用いた手動による加工で行ってもよいものとする。

%%%%%%%%%%%%%%%%%%%%%%%%%%%%%%%%%%%%%%%%%%%%%%%%%%%%%%%%%%
%% subsubsection 10.7.4.2 %%%%%%%%%%%%%%%%%%%%%%%%%%%%%%%%
%%%%%%%%%%%%%%%%%%%%%%%%%%%%%%%%%%%%%%%%%%%%%%%%%%%%%%%%%%
\subsubsection{面取Cが大きい場合}
マシニングセンタを用いて加工する場合、加工径の中心座標$X$をトップ側・ボトム側のそれぞれに対して以下だけ補正する。
\begin{align*}
  \text{トップ側:}&~~
  \sqrt{R_\mathrm c^2-\left(f_\mathrm T-c_\mathrm{Ti}\right)^2}-\sqrt{R_\mathrm c^2-f_\mathrm T^2}\ ,\\
  \text{ボトム側:}&~~
  \sqrt{R_\mathrm c^2-f_\mathrm B^2}-\sqrt{R_\mathrm c^2-\left(f_\mathrm B-c_\mathrm{Bi}\right)^2}\ .
\end{align*}
ここで$c_\mathrm{Ti}$, $c_\mathrm{Bi}$, $R_\mathrm c$, $f_\mathrm T$, $f_\mathrm B$はそれぞれトップ端の内側C面取の大きさ, ボトム端の内側C面取の大きさ, 中心湾曲半径, トップ振分長, ボトム振分長を示す。



\clearpage
%%%%%%%%%%%%%%%%%%%%%%%%%%%%%%%%%%%%%%%%%%%%%%%%%%%%%%%%%%
%% section 10.8 %%%%%%%%%%%%%%%%%%%%%%%%%%%%%%%%%%%%%%%%%%
%%%%%%%%%%%%%%%%%%%%%%%%%%%%%%%%%%%%%%%%%%%%%%%%%%%%%%%%%%
\modHeadsection{端面R面取に関する寸法}


%%%%%%%%%%%%%%%%%%%%%%%%%%%%%%%%%%%%%%%%%%%%%%%%%%%%%%%%%%
%% subsection 10.9.1 %%%%%%%%%%%%%%%%%%%%%%%%%%%%%%%%%%%%%
%%%%%%%%%%%%%%%%%%%%%%%%%%%%%%%%%%%%%%%%%%%%%%%%%%%%%%%%%%
\subsection{端面R面取加工の基準点}

%%%%%%%%%%%%%%%%%%%%%%%%%%%%%%%%%%%%%%%%%%%%%%%%%%%%%%%%%%
%% subsubsection 10.8.1.1 %%%%%%%%%%%%%%%%%%%%%%%%%%%%%%%%
%%%%%%%%%%%%%%%%%%%%%%%%%%%%%%%%%%%%%%%%%%%%%%%%%%%%%%%%%%
\subsubsection{端面外側R面取加工の基準点}
\index{たんめんそとがわRめんとりかこう@端面外側R面取加工}端面外側R面取加工の\index{きじゅん(たんめんそとがわRめんとりかこう)@基準(端面外側R面取加工)}基準は、外削のある場合は\index{がいさくちゅうしん@外削中心}外削中心を基準とし、外削のない場合は\index{がいけいちゅうしん(たんめん)@外径中心(端面)}端面における外径中心を基準として行うものとする。

%%%%%%%%%%%%%%%%%%%%%%%%%%%%%%%%%%%%%%%%%%%%%%%%%%%%%%%%%%
%% subsubsection 10.8.1.1 %%%%%%%%%%%%%%%%%%%%%%%%%%%%%%%%
%%%%%%%%%%%%%%%%%%%%%%%%%%%%%%%%%%%%%%%%%%%%%%%%%%%%%%%%%%
\subsubsection{端面内側R面取加工}
\index{たんめんうちがわRめんとりかこう@端面内側R面取加工}端面内側R面取加工の\index{きじゅん(たんめんうちがわRめんとりかこう)@基準(端面内側R面取加工)}基準は、\index{ないけいちゅうしん(たんめん)@内径中心(端面)}端面における内径中心を基準として行うものとする。


%%%%%%%%%%%%%%%%%%%%%%%%%%%%%%%%%%%%%%%%%%%%%%%%%%%%%%%%%%
%% subsection 10.8.1 %%%%%%%%%%%%%%%%%%%%%%%%%%%%%%%%%%%%%
%%%%%%%%%%%%%%%%%%%%%%%%%%%%%%%%%%%%%%%%%%%%%%%%%%%%%%%%%%
\subsection{端面外側R面取加工}

%%%%%%%%%%%%%%%%%%%%%%%%%%%%%%%%%%%%%%%%%%%%%%%%%%%%%%%%%%
%% subsubsection 10.8.1.1 %%%%%%%%%%%%%%%%%%%%%%%%%%%%%%%%
%%%%%%%%%%%%%%%%%%%%%%%%%%%%%%%%%%%%%%%%%%%%%%%%%%%%%%%%%%
\subsubsection{面取Rが小さい場合}
\index{そとがわRめんとり(たんめん)@外側R面取(端面)}端面の外側R面取については、基本的にはマシニングセンタでは行わず、\index{てもちけんまき@手持ち研磨機}手持ち研磨機を用いて手動で行うものとする。

%%%%%%%%%%%%%%%%%%%%%%%%%%%%%%%%%%%%%%%%%%%%%%%%%%%%%%%%%%
%% subsubsection 10.8.1.2 %%%%%%%%%%%%%%%%%%%%%%%%%%%%%%%%
%%%%%%%%%%%%%%%%%%%%%%%%%%%%%%%%%%%%%%%%%%%%%%%%%%%%%%%%%%
\subsubsection{面取Rが大きい場合}
\index{めんとりR@面取R}面取Rが大きい場合、\index{さぎょうしゃ@作業者}作業者の負担や\index{さぎょうじかん@作業時間}作業時間を考慮して、その一部を\index{テーパエンドミル}テーパエンドミルを用いて加工してもよいものとする。

このとき、トップ端およびボトム端における外側R面取の大きさ$r_\mathrm{To}$, $r_\mathrm{Bo}$, および片角$\xi_\mathrm e$のテーパエンドミルに対して、
\begin{align*}
  c_\mathrm{To} &= r_\mathrm{To}\left(1+\cot\xi_\mathrm e-\csc\xi_\mathrm e\right)\ ,\\
  c_\mathrm{Bo} &= r_\mathrm{Bo}\left(1+\cot\xi_\mathrm e-\csc\xi_\mathrm e\right)
\end{align*}
の\index{Cめんとり(たんめん)@C面取(端面)}C面取とみなして、\index{そとがわCめんとり(たんめん)@外側C面取(端面)}外側C面取加工を行うものとする。


%%%%%%%%%%%%%%%%%%%%%%%%%%%%%%%%%%%%%%%%%%%%%%%%%%%%%%%%%%
%% subsection 10.8.2 %%%%%%%%%%%%%%%%%%%%%%%%%%%%%%%%%%%%%
%%%%%%%%%%%%%%%%%%%%%%%%%%%%%%%%%%%%%%%%%%%%%%%%%%%%%%%%%%
\subsection{端面の内側R面取加工}

%%%%%%%%%%%%%%%%%%%%%%%%%%%%%%%%%%%%%%%%%%%%%%%%%%%%%%%%%%
%% subsubsection 10.8.2.1 %%%%%%%%%%%%%%%%%%%%%%%%%%%%%%%%
%%%%%%%%%%%%%%%%%%%%%%%%%%%%%%%%%%%%%%%%%%%%%%%%%%%%%%%%%%
\subsubsection{面取Rが小さい場合}
\index{うちがわRめんとり(たんめん)@内側R面取(端面)}トップ端およびボトム端の内側R面取については、基本的にはマシニングセンタでは行わず、\index{てもちけんまき@手持ち研磨機}手持ち研磨機を用いて手動で行うものとする。

%%%%%%%%%%%%%%%%%%%%%%%%%%%%%%%%%%%%%%%%%%%%%%%%%%%%%%%%%%
%% subsubsection 10.8.2.2 %%%%%%%%%%%%%%%%%%%%%%%%%%%%%%%%
%%%%%%%%%%%%%%%%%%%%%%%%%%%%%%%%%%%%%%%%%%%%%%%%%%%%%%%%%%
\subsubsection{面取Rが大きい場合}
面取Rが大きい場合、作業者の負担や作業時間を考慮して、その一部を\index{テーパエンドミル}テーパエンドミルを用いて加工してもよいものとする。

このとき、トップ端およびボトム端における内側R面取の大きさ$r_\mathrm{Ti}$, $r_\mathrm{Bi}$, および片角$\xi_\mathrm e$のテーパエンドミルに対して、
\begin{align*}
  c_\mathrm{Ti} &= r_\mathrm{Ti}\left(1+\cot\xi_\mathrm e-\csc\xi_\mathrm e\right)\ ,\\
  c_\mathrm{Bi} &= r_\mathrm{Bi}\left(1+\cot\xi_\mathrm e-\csc\xi_\mathrm e\right)
\end{align*}
の\index{Cめんとり(たんめん)@C面取(端面)}C面取とみなして、\index{うちがわCめんとり(たんめん)@内側C面取(端面)}内側C面取加工を行うものとする。



\clearpage
%%%%%%%%%%%%%%%%%%%%%%%%%%%%%%%%%%%%%%%%%%%%%%%%%%%%%%%%%%
%% section 10.9 %%%%%%%%%%%%%%%%%%%%%%%%%%%%%%%%%%%%%%%%%%
%%%%%%%%%%%%%%%%%%%%%%%%%%%%%%%%%%%%%%%%%%%%%%%%%%%%%%%%%%
\modHeadsection{\TanmenZaguri に関する寸法\TBW}
(to be written...)


\clearpage
%%%%%%%%%%%%%%%%%%%%%%%%%%%%%%%%%%%%%%%%%%%%%%%%%%%%%%%%%%
%% section 10.9 %%%%%%%%%%%%%%%%%%%%%%%%%%%%%%%%%%%%%%%%%%
%%%%%%%%%%%%%%%%%%%%%%%%%%%%%%%%%%%%%%%%%%%%%%%%%%%%%%%%%%
\modHeadsection{\Dimple に関する寸法}


%%%%%%%%%%%%%%%%%%%%%%%%%%%%%%%%%%%%%%%%%%%%%%%%%%%%%%%%%%
%% subsection 10.06.1 %%%%%%%%%%%%%%%%%%%%%%%%%%%%%%%%%%%%
%%%%%%%%%%%%%%%%%%%%%%%%%%%%%%%%%%%%%%%%%%%%%%%%%%%%%%%%%%
\subsection{工具補正:\Dimple 加工}

%%%%%%%%%%%%%%%%%%%%%%%%%%%%%%%%%%%%%%%%%%%%%%%%%%%%%%%%%%
%% subsubsection 10.06.1.1 %%%%%%%%%%%%%%%%%%%%%%%%%%%%%%%
%%%%%%%%%%%%%%%%%%%%%%%%%%%%%%%%%%%%%%%%%%%%%%%%%%%%%%%%%%
\subsubsection{工具長補正:\Dimple 加工}
\Dimple 加工に用いる\index{Tスロットカッター}Tスロットカッターの\index{こうぐちょう(Tスロットカッター)@工具長(Tスロットカッター)}工具長については、切削部(刃の部分)の中央を工具長として\index{オフセット(こうぐちょうほせい)@オフセット(工具長補正)}オフセット量の設定を行うものとする。

したがって、工具の切削部の先端、および切削部の厚みを測定し、切削部の先端から厚みの半分の値を引いた値を\index{こうぐちょうほせいち@工具長補正値}工具長補正値とする。
なお、このとき工具長の\index{まもうりょう(こうぐちょうほせい)@摩耗量(工具長補正)}摩耗量は0とする。

%%%%%%%%%%%%%%%%%%%%%%%%%%%%%%%%%%%%%%%%%%%%%%%%%%%%%%%%%%
%% subsubsection 10.06.1.2 %%%%%%%%%%%%%%%%%%%%%%%%%%%%%%%
%%%%%%%%%%%%%%%%%%%%%%%%%%%%%%%%%%%%%%%%%%%%%%%%%%%%%%%%%%
\subsubsection{工具径補正:\Dimple 加工}
\Dimple 加工に用いる\index{Tスロットカッター}Tスロットカッターの\index{こうぐけい(Tスロットカッター)@工具径(Tスロット)}工具長は、その刃の径方向の先端部を工具径として\index{オフセット(こうぐけいほせい)@オフセット(工具径補正)}オフセット量の設定を行うものとする。

なお、このとき工具径の\index{まもうりょう(こうぐけいほせい)@摩耗量(工具径補正)}摩耗量は0とする。


%%%%%%%%%%%%%%%%%%%%%%%%%%%%%%%%%%%%%%%%%%%%%%%%%%%%%%%%%%
%% subsection 10.9.2 %%%%%%%%%%%%%%%%%%%%%%%%%%%%%%%%%%%%%
%%%%%%%%%%%%%%%%%%%%%%%%%%%%%%%%%%%%%%%%%%%%%%%%%%%%%%%%%%
\subsection{\Dimple 加工の基準点}
\Dimple を加工する際は、トップ側の端面における内側の中心座標の\index{じっそくち@実測値}実測値を基準にして行うものとする。


%%%%%%%%%%%%%%%%%%%%%%%%%%%%%%%%%%%%%%%%%%%%%%%%%%%%%%%%%%
%% subsection 10.9.3 %%%%%%%%%%%%%%%%%%%%%%%%%%%%%%%%%%%%%
%%%%%%%%%%%%%%%%%%%%%%%%%%%%%%%%%%%%%%%%%%%%%%%%%%%%%%%%%%
\subsection{\Dimple 加工の傾き角}
\Dimple の測定および加工は、\index{アンダーカット}アンダーカットが生じないように適切にワークを$B$軸方向に傾けるものとする。
このとき\expandafterindex{かたむきかく(\yomiDimple)@傾き角(\nameDimple)}傾ける角度$\phi$は、\expandafterindex{\yomiDimple1れつめ@\nameDimple1列目}トップ端から1列目の\nameDimple までの距離$q$に対して、
\begin{align*}
  \tan\phi
  &= \frac{\displaystyle
           \sqrt{\left(R_\mathrm c+\frac{w'_{\mathrm Aq}}2\right)^2-(f_\mathrm T-q)^2}
           -\sqrt{\left(R_\mathrm c+\frac{w'_{\mathrm A0}}2\right)^2-f_\mathrm T^2}}q
\end{align*}
とする。
ここで$w'_{\mathrm A0}$, $w'_{\mathrm Aq}$は(\index{めっきまくあつ@めっき膜厚}めっき膜厚$\mu$を考慮した)トップ端におけるAC方向の内径およびトップ端から距離$q$におけるAC方向の内径を示し、$R_\mathrm c$, $f_\mathrm T$はそれぞれ中心湾曲半径, トップ振分長を示す。

ただし、$\phi < 0$となる場合は、$\phi = 0$とみなすものとする。


