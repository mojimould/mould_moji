%!TEX root = ../RPA_for_Creating_Program_Note.tex


\index{ひきすう@引数}引数の指定の仕方は2通りあるが、ここでは以下の表のものを使用する。

%%%%%%%%%%%%%%%%%%%%%%%%%%%%%%%%%%%%%%%%%%%%%%%%%%%%%%%%%%
%% section 13.1 %%%%%%%%%%%%%%%%%%%%%%%%%%%%%%%%%%%%%%%%%%
%%%%%%%%%%%%%%%%%%%%%%%%%%%%%%%%%%%%%%%%%%%%%%%%%%%%%%%%%%
\modHeadsection{引数アドレスとローカル変数\TBW}
\addtocontents{lot}{\protect\addvspace{3pt}}{}{}
\addcontentsline{lot}{section}{\numberline{\thesection}\Sectionname}
%%%%%%%%%%%%%%%%%%%%%%%%%%%%%%%%%%%%%%%%%%%%%%%%%%%%%%%%%%
%% common variables %%%%%%%%%%%%%%%%%%%%%%%%%%%%%%%%%%%%%%
%%%%%%%%%%%%%%%%%%%%%%%%%%%%%%%%%%%%%%%%%%%%%%%%%%%%%%%%%%
%%%%%%%%%%%%%%%%%%%%%%%%%%%%%%%%%%%%%%%%%%%%%%%%%%%%%%%%%%
%% captionof %%%%%%%%%%%%%%%%%%%%%%%%%%%%%%%%%%%%%%%%%%%%%
%%%%%%%%%%%%%%%%%%%%%%%%%%%%%%%%%%%%%%%%%%%%%%%%%%%%%%%%%%
\begin{Notation}{引数指定\TBW}{番号}
A & \verb|#01| &\\\hline
B & \verb|#02| &\\\hline
C & \verb|#03| &\\\hline
D & \verb|#07| &\\\hline
E & \verb|#08| &\\\hline
F & \verb|#09| &\\\hline
H & \verb|#11| &\\\hline
I & \verb|#04| &\\\hline
J & \verb|#05| &\\\hline
K & \verb|#06| &\\\hline
M & \verb|#13| &\\\hline
Q & \verb|#17| &\\\hline
R & \verb|#18| &\\\hline
S & \verb|#19| &\\\hline
T & \verb|#20| &\\\hline
U & \verb|#21| &\\\hline
V & \verb|#22| &\\\hline
W & \verb|#23| &\\\hline
X & \verb|#24| &\\\hline
Y & \verb|#25| &\\\hline
Z & \verb|#26| &\\\hline
 & \verb|#27| &\\\hline
 & \verb|#28| &\\\hline
 & \verb|#29| &\\\hline
 & \verb|#30| &\\\hline
 & \verb|#31| &\\\hline
 & \verb|#32| &\\\hline
 & \verb|#33| &\\\hline
\end{Notation}


\clearpage
%%%%%%%%%%%%%%%%%%%%%%%%%%%%%%%%%%%%%%%%%%%%%%%%%%%%%%%%%%
%% captionof %%%%%%%%%%%%%%%%%%%%%%%%%%%%%%%%%%%%%%%%%%%%%
%%%%%%%%%%%%%%%%%%%%%%%%%%%%%%%%%%%%%%%%%%%%%%%%%%%%%%%%%%
\begin{Notation}{通常は使用できない引数\TBW}{番号}
G & \verb|#10| &\\\hline
L & \verb|#12| &\\\hline
N & \verb|#14| &\\\hline
O & \verb|#15| &\\\hline
P & \verb|#16| &\\\hline
\end{Notation}


