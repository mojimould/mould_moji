%!TEX root = ../RPA_for_Creating_Program_Note.tex




%%%%%%%%%%%%%%%%%%%%%%%%%%%%%%%%%%%%%%%%%%%%%%%%%%%%%%%%%%
%% section 12.1 %%%%%%%%%%%%%%%%%%%%%%%%%%%%%%%%%%%%%%%%%%
%%%%%%%%%%%%%%%%%%%%%%%%%%%%%%%%%%%%%%%%%%%%%%%%%%%%%%%%%%
\modHeadsection{開発プロセス}

%%%%%%%%%%%%%%%%%%%%%%%%%%%%%%%%%%%%%%%%%%%%%%%%%%%%%%%%%%
%% subsection 12.1.1 %%%%%%%%%%%%%%%%%%%%%%%%%%%%%%%%%%%%%
%%%%%%%%%%%%%%%%%%%%%%%%%%%%%%%%%%%%%%%%%%%%%%%%%%%%%%%%%%
\subsection{要件定義}
新規機能の開発または既存機能の改修を行う際は、要件定義を行い、要件定義書の作成を行うものとする。
要件定義については関係者全員でレビューを実施し、必要に応じて要件定義書の更新を行う。

なお、このプロセスはプロジェクトリーダーまたはシステムアナリストとしての能力を有する者が行うものとする。

%%%%%%%%%%%%%%%%%%%%%%%%%%%%%%%%%%%%%%%%%%%%%%%%%%%%%%%%%%
%% subsection 12.1.2 %%%%%%%%%%%%%%%%%%%%%%%%%%%%%%%%%%%%%
%%%%%%%%%%%%%%%%%%%%%%%%%%%%%%%%%%%%%%%%%%%%%%%%%%%%%%%%%%
\subsection{基本設計}
要件定義書の内容に基づいて基本設計を行い、基本設計書の作成を行うものとする。
基本設計については関係者全員でレビューを実施し、必要に応じて基本設計書の更新を行う。

なお、このプロセスはシステムエンジニアとしての能力の有する者が行うものとする。

%%%%%%%%%%%%%%%%%%%%%%%%%%%%%%%%%%%%%%%%%%%%%%%%%%%%%%%%%%
%% subsection 12.1.3 %%%%%%%%%%%%%%%%%%%%%%%%%%%%%%%%%%%%%
%%%%%%%%%%%%%%%%%%%%%%%%%%%%%%%%%%%%%%%%%%%%%%%%%%%%%%%%%%
\subsection{詳細設計}
基本設計書の内容に基づいて詳細設計を行い、詳細設計書の作成を行うものとする。
詳細設計書を作成し、関係者全員でレビューを実施し、必要に応じて詳細設計書を更新する。

なお、このプロセスはシステムエンジニアとしての能力の有する者が行うものとする。

%%%%%%%%%%%%%%%%%%%%%%%%%%%%%%%%%%%%%%%%%%%%%%%%%%%%%%%%%%
%% subsection 12.1.4 %%%%%%%%%%%%%%%%%%%%%%%%%%%%%%%%%%%%%
%%%%%%%%%%%%%%%%%%%%%%%%%%%%%%%%%%%%%%%%%%%%%%%%%%%%%%%%%%
\subsection{コードの記述}
詳細設計書の内容に基づいて、コードの記述を行うものとする。
記述は、コーディングルールに従って記述を行う。

なお、このプロセスはシステムエンジニアまたはプログラマとしての能力の有する者が行うものとする。

%%%%%%%%%%%%%%%%%%%%%%%%%%%%%%%%%%%%%%%%%%%%%%%%%%%%%%%%%%
%% subsection 12.1.5 %%%%%%%%%%%%%%%%%%%%%%%%%%%%%%%%%%%%%
%%%%%%%%%%%%%%%%%%%%%%%%%%%%%%%%%%%%%%%%%%%%%%%%%%%%%%%%%%
\subsection{コードレビュー}
コードの記述の完了に伴い、そのコードはコードレビューを受けるものとする。
レビューの結果にもと、必要があれば修正・改善を行う。

%%%%%%%%%%%%%%%%%%%%%%%%%%%%%%%%%%%%%%%%%%%%%%%%%%%%%%%%%%
%% subsection 12.1.6 %%%%%%%%%%%%%%%%%%%%%%%%%%%%%%%%%%%%%
%%%%%%%%%%%%%%%%%%%%%%%%%%%%%%%%%%%%%%%%%%%%%%%%%%%%%%%%%%
\subsection{テスト}
コードレビューを受けたコードについて、正しく動作することを確認するためにテストを行うものとする。
テストの方法は予め定めておき、その方法に従ってテストを実施する。
テスト結果に基づいて、必要があれば修正・改善を行う。

なお、このプロセスはテストエンジニアとしての能力の有する者が行うものとする。

%%%%%%%%%%%%%%%%%%%%%%%%%%%%%%%%%%%%%%%%%%%%%%%%%%%%%%%%%%
%% subsection 12.1.7 %%%%%%%%%%%%%%%%%%%%%%%%%%%%%%%%%%%%%
%%%%%%%%%%%%%%%%%%%%%%%%%%%%%%%%%%%%%%%%%%%%%%%%%%%%%%%%%%
\subsection{テスト環境での動作確認}
テスト結果に問題のないことが確認された場合、コードをテスト環境にデプロイを行い、動作の確認を実施する。
確認事項は予め定めておき、それに従って確認を実施する。
動作確認の結果に基づいて、必要があれば修正・改善を行う。

なお、このプロセスはテストエンジニアとしての能力の有する者が行うものとする。

%%%%%%%%%%%%%%%%%%%%%%%%%%%%%%%%%%%%%%%%%%%%%%%%%%%%%%%%%%
%% subsection 12.1.8 %%%%%%%%%%%%%%%%%%%%%%%%%%%%%%%%%%%%%
%%%%%%%%%%%%%%%%%%%%%%%%%%%%%%%%%%%%%%%%%%%%%%%%%%%%%%%%%%
\subsection{リリース}
テスト環境による動作に問題のないことが確認された場合、本番環境にリリースを行うものとする。

なお、このプロセスはリリースマネージャとしての能力の有する者が行うものとする。

%%%%%%%%%%%%%%%%%%%%%%%%%%%%%%%%%%%%%%%%%%%%%%%%%%%%%%%%%%
%% subsection 12.1.9 %%%%%%%%%%%%%%%%%%%%%%%%%%%%%%%%%%%%%
%%%%%%%%%%%%%%%%%%%%%%%%%%%%%%%%%%%%%%%%%%%%%%%%%%%%%%%%%%
\subsection{変更管理}
開発プロセス中に要件・設計・コード等が変更される場合、変更要求書を作成し、関係者全員でレビューを行うものとする。
変更要求書には、変更の理由・影響範囲・必要なリソース等を明記する。
変更が承認された場合、変更要求書に基づいて関連する文書やコードの更新を行う。

なお、プロジェクトマネージャまたはプロジェクトリーダーとしての能力を有する者が管理を行うものとする。

%%%%%%%%%%%%%%%%%%%%%%%%%%%%%%%%%%%%%%%%%%%%%%%%%%%%%%%%%%
%% subsection 12.1.10 %%%%%%%%%%%%%%%%%%%%%%%%%%%%%%%%%%%%
%%%%%%%%%%%%%%%%%%%%%%%%%%%%%%%%%%%%%%%%%%%%%%%%%%%%%%%%%%
\subsection{リスク管理}
プロジェクトの開始時に、リスク管理計画を行うものとする。
リスク管理計画には、潜在的なリスクの特定、リスクの評価、リスク対策の策定、リスクの監視と管理が含まれる。
リスク管理計画はプロジェクト期間中に定期的に見直し、更新を実施する。

なお、プロジェクトマネージャまたはプロジェクトリーダーとしての能力を有する者が管理を行うものとする。

%%%%%%%%%%%%%%%%%%%%%%%%%%%%%%%%%%%%%%%%%%%%%%%%%%%%%%%%%%
%% subsection 12.1.10 %%%%%%%%%%%%%%%%%%%%%%%%%%%%%%%%%%%%
%%%%%%%%%%%%%%%%%%%%%%%%%%%%%%%%%%%%%%%%%%%%%%%%%%%%%%%%%%
\subsection{ドキュメンテーション}
各々の業務において、その全体像を理解するために、適切なドキュメンテーションの作成を行うものとする。
これには、要件定義書、設計書、テスト計画、ユーザーマニュアル等が含まれる。
ドキュメンテーションは常に最新の状態を保ち、関係者が容易にアクセスできる場所に保存を行う。

なお、これらはそのプロジェクトの各段階の責任者に相当する者が作成し、プロジェクトマネージャまたはプロジェクトリーダーとしての能力を有する者が管理を行うものとする。

%%%%%%%%%%%%%%%%%%%%%%%%%%%%%%%%%%%%%%%%%%%%%%%%%%%%%%%%%%
%% subsection 12.1.11 %%%%%%%%%%%%%%%%%%%%%%%%%%%%%%%%%%%%
%%%%%%%%%%%%%%%%%%%%%%%%%%%%%%%%%%%%%%%%%%%%%%%%%%%%%%%%%%
\subsection{緊急時の対応}
本番環境で重大な問題が発生した際は、速やかに直属の上司またはそのシステムの管理者に連絡する。
システム管理者は直ちに問題の原因を特定し、適切な対応策の実施を行う。



\clearpage
%%%%%%%%%%%%%%%%%%%%%%%%%%%%%%%%%%%%%%%%%%%%%%%%%%%%%%%%%%
%% section 20.2 %%%%%%%%%%%%%%%%%%%%%%%%%%%%%%%%%%%%%%%%%%
%%%%%%%%%%%%%%%%%%%%%%%%%%%%%%%%%%%%%%%%%%%%%%%%%%%%%%%%%%
\modHeadsection{コードレビュー}
コードレビューとは、ソースコードの作成者とは別の人物がコードを詳しく調べて問題がないか検討することであり、プログラムの品質を維持するために欠かせない工程である。

%%%%%%%%%%%%%%%%%%%%%%%%%%%%%%%%%%%%%%%%%%%%%%%%%%%%%%%%%%
%% subsection 12.3.1 %%%%%%%%%%%%%%%%%%%%%%%%%%%%%%%%%%%%%
%%%%%%%%%%%%%%%%%%%%%%%%%%%%%%%%%%%%%%%%%%%%%%%%%%%%%%%%%%
\subsection{コードレビューの目的}
ソフトウェア開発の際は、ソフトウェア開発の質を高めることを目的として、コードレビューを行うものとする。

%%%%%%%%%%%%%%%%%%%%%%%%%%%%%%%%%%%%%%%%%%%%%%%%%%%%%%%%%%
%% subsection 12.3.1 %%%%%%%%%%%%%%%%%%%%%%%%%%%%%%%%%%%%%
%%%%%%%%%%%%%%%%%%%%%%%%%%%%%%%%%%%%%%%%%%%%%%%%%%%%%%%%%%
\subsection{コードレビューの手順}
新規に作成または修正したソースコードは、可能な限りその作成者とは異なる他の開発者によるコードレビューを受けるものとする。

%%%%%%%%%%%%%%%%%%%%%%%%%%%%%%%%%%%%%%%%%%%%%%%%%%%%%%%%%%
%% subsection 12.3.1 %%%%%%%%%%%%%%%%%%%%%%%%%%%%%%%%%%%%%
%%%%%%%%%%%%%%%%%%%%%%%%%%%%%%%%%%%%%%%%%%%%%%%%%%%%%%%%%%
\subsection{レビュアー}
コードレビューを行う者(レビュアー)は、そのコードに関連する技術的スキルを持っていることが要求される。

なお、レビュアーの選定はプロジェクトマネージャまたはプロジェクトリーダーとしての能力を有する者が行うものとする。

%%%%%%%%%%%%%%%%%%%%%%%%%%%%%%%%%%%%%%%%%%%%%%%%%%%%%%%%%%
%% subsection 12.3.2 %%%%%%%%%%%%%%%%%%%%%%%%%%%%%%%%%%%%%
%%%%%%%%%%%%%%%%%%%%%%%%%%%%%%%%%%%%%%%%%%%%%%%%%%%%%%%%%%
\subsection{レビューの範囲}
コードレビューの対象となるコードは、新規に作成されたコード、修正されたコード、およびそれらのコードに直接影響を与える可能性のある既存のコードとする。
レビューの範囲は、レビュアーとコードの作成者が協議して決定する。

%%%%%%%%%%%%%%%%%%%%%%%%%%%%%%%%%%%%%%%%%%%%%%%%%%%%%%%%%%
%% subsection 12.3.2 %%%%%%%%%%%%%%%%%%%%%%%%%%%%%%%%%%%%%
%%%%%%%%%%%%%%%%%%%%%%%%%%%%%%%%%%%%%%%%%%%%%%%%%%%%%%%%%%
\subsection{コードレビューの承認}
コードレビューでは、以下のすべての観点について確認を行い、これらをすべて満たしている場合にのみ承認を行うものとする。
\begin{enumerate}
\item コーディングルールに従っていること
\item 意図したとおりの実装が行われていること
\item 不具合につながる部分が存在しないこと
\end{enumerate}

%%%%%%%%%%%%%%%%%%%%%%%%%%%%%%%%%%%%%%%%%%%%%%%%%%%%%%%%%%
%% subsection 12.3.2 %%%%%%%%%%%%%%%%%%%%%%%%%%%%%%%%%%%%%
%%%%%%%%%%%%%%%%%%%%%%%%%%%%%%%%%%%%%%%%%%%%%%%%%%%%%%%%%%
\subsection{フィードバックの提供}
レビュアーは、レビューの結果を明確に伝えるために、具体的なコメントや改善提案の提供を行う。
フィードバックは建設的であるべきであり、問題だけでなく良い点や改善の提案も含めることが推奨される。

%%%%%%%%%%%%%%%%%%%%%%%%%%%%%%%%%%%%%%%%%%%%%%%%%%%%%%%%%%
%% subsection 12.3.3 %%%%%%%%%%%%%%%%%%%%%%%%%%%%%%%%%%%%%
%%%%%%%%%%%%%%%%%%%%%%%%%%%%%%%%%%%%%%%%%%%%%%%%%%%%%%%%%%
\subsection{コードレビューの時期}
開発プロセスにおけるコードレビューの実施タイミングは、予め定められたスケジュールに従うものとする。


\clearpage
%%%%%%%%%%%%%%%%%%%%%%%%%%%%%%%%%%%%%%%%%%%%%%%%%%%%%%%%%%
%% section 12.3 %%%%%%%%%%%%%%%%%%%%%%%%%%%%%%%%%%%%%%%%%%
%%%%%%%%%%%%%%%%%%%%%%%%%%%%%%%%%%%%%%%%%%%%%%%%%%%%%%%%%%
\modHeadsection{バージョン管理}
バージョン管理とは、ソースコードの変更履歴を追跡し、必要に応じて以前のバージョンに戻すことができるシステムである。
これにより複数の開発者による同時作業や、過去のバージョンに戻すこと等ができ、生産性の向上に大きく寄与する。

%%%%%%%%%%%%%%%%%%%%%%%%%%%%%%%%%%%%%%%%%%%%%%%%%%%%%%%%%%
%% subsection 12.3.1 %%%%%%%%%%%%%%%%%%%%%%%%%%%%%%%%%%%%%
%%%%%%%%%%%%%%%%%%%%%%%%%%%%%%%%%%%%%%%%%%%%%%%%%%%%%%%%%%
\subsection{バージョン管理の目的}
ソフトウェア開発の際は、ソースコードの変更履歴を追跡し、必要に応じて以前のバージョンに戻すことができるように、バージョン管理を行うものとする。

%%%%%%%%%%%%%%%%%%%%%%%%%%%%%%%%%%%%%%%%%%%%%%%%%%%%%%%%%%
%% subsection 12.3.2 %%%%%%%%%%%%%%%%%%%%%%%%%%%%%%%%%%%%%
%%%%%%%%%%%%%%%%%%%%%%%%%%%%%%%%%%%%%%%%%%%%%%%%%%%%%%%%%%
\subsection{バージョン管理の手順}
各々の文書またはソースコードは、それぞれ1つのバージョン管理システムを用いて管理を行うものとする。
新規に作成または修正したソースコードは、バージョン管理システムにコミットする。
各コミットには、変更内容を説明するメッセージを必ず含める。

%%%%%%%%%%%%%%%%%%%%%%%%%%%%%%%%%%%%%%%%%%%%%%%%%%%%%%%%%%
%% subsection 12.3.3 %%%%%%%%%%%%%%%%%%%%%%%%%%%%%%%%%%%%%
%%%%%%%%%%%%%%%%%%%%%%%%%%%%%%%%%%%%%%%%%%%%%%%%%%%%%%%%%%
\subsection{バージョン管理の承認}
バージョン管理システムにコミットする前に、そのソースコードはコードレビューを受けるものとする。
コードレビューの結果に基づいて、必要に応じてソースコードの修正・改善を行う。

%%%%%%%%%%%%%%%%%%%%%%%%%%%%%%%%%%%%%%%%%%%%%%%%%%%%%%%%%%
%% subsection 12.3.4 %%%%%%%%%%%%%%%%%%%%%%%%%%%%%%%%%%%%%
%%%%%%%%%%%%%%%%%%%%%%%%%%%%%%%%%%%%%%%%%%%%%%%%%%%%%%%%%%
\subsection{バージョン管理の時期}
開発プロセスにおけるバージョン管理の実施タイミングは、ソースコードの新規作成または修正が完了した時点とする。


\clearpage
%%%%%%%%%%%%%%%%%%%%%%%%%%%%%%%%%%%%%%%%%%%%%%%%%%%%%%%%%%
%% section 12.4 %%%%%%%%%%%%%%%%%%%%%%%%%%%%%%%%%%%%%%%%%%
%%%%%%%%%%%%%%%%%%%%%%%%%%%%%%%%%%%%%%%%%%%%%%%%%%%%%%%%%%
\modHeadsection{テスト}

%%%%%%%%%%%%%%%%%%%%%%%%%%%%%%%%%%%%%%%%%%%%%%%%%%%%%%%%%%
%% subsection 12.4.1 %%%%%%%%%%%%%%%%%%%%%%%%%%%%%%%%%%%%%
%%%%%%%%%%%%%%%%%%%%%%%%%%%%%%%%%%%%%%%%%%%%%%%%%%%%%%%%%%
\subsection{テストの目的}
ソフトウェア開発の際は、ソフトウェアが期待通りの動作をすることを確認するために、テストを行うものとする。

%%%%%%%%%%%%%%%%%%%%%%%%%%%%%%%%%%%%%%%%%%%%%%%%%%%%%%%%%%
%% subsection 12.4.2 %%%%%%%%%%%%%%%%%%%%%%%%%%%%%%%%%%%%%
%%%%%%%%%%%%%%%%%%%%%%%%%%%%%%%%%%%%%%%%%%%%%%%%%%%%%%%%%%
\subsection{テストケースの作成}
テストケースは、要件定義書および設計書に基づいて作成するものとする。
すべての機能・シナリオをカバーするように、テストケースを作成する。

なお、テストケースの作成は、品質管理の能力を有するものが行うものとする。

%%%%%%%%%%%%%%%%%%%%%%%%%%%%%%%%%%%%%%%%%%%%%%%%%%%%%%%%%%
%% subsection 12.4.3 %%%%%%%%%%%%%%%%%%%%%%%%%%%%%%%%%%%%%
%%%%%%%%%%%%%%%%%%%%%%%%%%%%%%%%%%%%%%%%%%%%%%%%%%%%%%%%%%
\subsection{テストの手順}
新規に作成または修正したソースコードは、テストを行うものとする。
テストは、予め定められたテスト計画に従って実施する。
テストの結果に基づいて、必要に応じてソースコードの修正・改善を行う。

%%%%%%%%%%%%%%%%%%%%%%%%%%%%%%%%%%%%%%%%%%%%%%%%%%%%%%%%%%
%% subsection 12.4.4 %%%%%%%%%%%%%%%%%%%%%%%%%%%%%%%%%%%%%
%%%%%%%%%%%%%%%%%%%%%%%%%%%%%%%%%%%%%%%%%%%%%%%%%%%%%%%%%%
\subsection{テストの承認}
すべてのテストケースが成功した場合にのみ、そのソースコードは承認されるものとする。

%%%%%%%%%%%%%%%%%%%%%%%%%%%%%%%%%%%%%%%%%%%%%%%%%%%%%%%%%%
%% subsection 12.4.5 %%%%%%%%%%%%%%%%%%%%%%%%%%%%%%%%%%%%%
%%%%%%%%%%%%%%%%%%%%%%%%%%%%%%%%%%%%%%%%%%%%%%%%%%%%%%%%%%
\subsection{テストの時期}
開発プロセスにおけるテストの実施タイミングは、ソースコードの新規作成または修正が完了した時点とする。

%%%%%%%%%%%%%%%%%%%%%%%%%%%%%%%%%%%%%%%%%%%%%%%%%%%%%%%%%%
%% subsection 12.4.5 %%%%%%%%%%%%%%%%%%%%%%%%%%%%%%%%%%%%%
%%%%%%%%%%%%%%%%%%%%%%%%%%%%%%%%%%%%%%%%%%%%%%%%%%%%%%%%%%
\subsection{リグレッションテスト}
プログラムの変更や修正等により新たなコードを追加した際は、その新たなコードが既存の機能に影響を与えていないことを確認するため、必要に応じてリグレッションテストを行うものとする。


\clearpage
%%%%%%%%%%%%%%%%%%%%%%%%%%%%%%%%%%%%%%%%%%%%%%%%%%%%%%%%%%
%% section 12.5 %%%%%%%%%%%%%%%%%%%%%%%%%%%%%%%%%%%%%%%%%%
%%%%%%%%%%%%%%%%%%%%%%%%%%%%%%%%%%%%%%%%%%%%%%%%%%%%%%%%%%
\modHeadsection{ドキュメンテーション}

%%%%%%%%%%%%%%%%%%%%%%%%%%%%%%%%%%%%%%%%%%%%%%%%%%%%%%%%%%
%% subsection 12.5.1 %%%%%%%%%%%%%%%%%%%%%%%%%%%%%%%%%%%%%
%%%%%%%%%%%%%%%%%%%%%%%%%%%%%%%%%%%%%%%%%%%%%%%%%%%%%%%%%%
\subsection{ドキュメンテーションの目的}
ソフトウェア開発の際は、ソフトウェアの使用方法、設計、変更履歴などを記録し、開発者やユーザーが理解しやすくするためにドキュメンテーションを行うものとする。

%%%%%%%%%%%%%%%%%%%%%%%%%%%%%%%%%%%%%%%%%%%%%%%%%%%%%%%%%%
%% subsection 12.5.2 %%%%%%%%%%%%%%%%%%%%%%%%%%%%%%%%%%%%%
%%%%%%%%%%%%%%%%%%%%%%%%%%%%%%%%%%%%%%%%%%%%%%%%%%%%%%%%%%
\subsection{ドキュメンテーションの作成}
各々の業務において、その全体像を理解するために、適切なドキュメンテーションの作成を行うものとする。
これには、要件定義書、基本設計書、詳細設計書、ユーザーマニュアル等が含まれる。

%%%%%%%%%%%%%%%%%%%%%%%%%%%%%%%%%%%%%%%%%%%%%%%%%%%%%%%%%%
%% subsection 12.5.3 %%%%%%%%%%%%%%%%%%%%%%%%%%%%%%%%%%%%%
%%%%%%%%%%%%%%%%%%%%%%%%%%%%%%%%%%%%%%%%%%%%%%%%%%%%%%%%%%
\subsection{ドキュメンテーションの更新}
ドキュメンテーションは常に最新の状態を保つものとする。
新規に作成または修正したソースコードに関連するドキュメンテーションは、ソースコードの新規作成または修正が完了した時点で更新する。

%%%%%%%%%%%%%%%%%%%%%%%%%%%%%%%%%%%%%%%%%%%%%%%%%%%%%%%%%%
%% subsection 12.5.4 %%%%%%%%%%%%%%%%%%%%%%%%%%%%%%%%%%%%%
%%%%%%%%%%%%%%%%%%%%%%%%%%%%%%%%%%%%%%%%%%%%%%%%%%%%%%%%%%
\subsection{ドキュメンテーションの保存}
ドキュメンテーションは関係者が容易にアクセスできる場所に保存を行うものとする。

