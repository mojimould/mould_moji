%!TEX root = ../RPA_for_Creating_Program_Note.tex


著作物をオンライン上に公表することで、その開発や保守等における生産性が大きく向上する。
実際、\DMname におけるバージョン管理やイシュー管理は、オンライン上の\index{バージョンかんりシステム@バージョン管理システム}バージョン管理システム, \index{ソースコードかんりシステム@ソースコード管理システム}ソースコード管理(\index{SCM}SCM)システム, \index{リポジトリホスティングサービス}リポジトリホスティングサービスを用いて行われている
%% footnote %%%%%%%%%%%%%%%%%%%%%
\footnote{これによりコードの共有・バージョン管理・ビルド・テスト・デプロイ・問題追跡・ドキュメンテーションなどの機能を用いることで、開発・保守の生産性が大きく向上している。}。
%%%%%%%%%%%%%%%%%%%%%%%%%%%%%%%%%
一方で、サーバ停止等によるリスクや公表による情報漏洩等の\index{セキュリティ}セキュリティリスクも考えられる。

こうしたことを踏まえ、ここでは作成されたソフトウェア関連の\index{ちょさくぶつ@著作物}著作物の取り扱いについて述べる。



%%%%%%%%%%%%%%%%%%%%%%%%%%%%%%%%%%%%%%%%%%%%%%%%%%%%%%%%%%
%% section 20.1 %%%%%%%%%%%%%%%%%%%%%%%%%%%%%%%%%%%%%%%%%%
%%%%%%%%%%%%%%%%%%%%%%%%%%%%%%%%%%%%%%%%%%%%%%%%%%%%%%%%%%
\modHeadsection{関連する著作物}
マシニングセンタによるモールドの加工に対して作成された(ソフトウェア関連の)著作物として、主に以下のものが挙げられる。
\begin{enumerate}
\item 本書
\item 位置情報等の数値計算用プログラム
\item 使用スペーサ計算用プログラム
\item バンドルのプログラムを除いたNCプログラム
\item モールドのRDB
\item モールドの3次元CADモデリング用テンプレート
\item 内面テーパの3次元CADモデリング用テンプレート
\end{enumerate}
著作権法第2条第1項(著作物の定義)より、これらは「思想又は感情を創作的に表現したもの」であり、著作権法の保護対象となる
%% footnote %%%%%%%%%%%%%%%%%%%%%
\footnote{なお、著作権法第10条第1項第9号(プログラムの著作物)より、プログラムは著作物の一種として明示的に規程されている。}。
%%%%%%%%%%%%%%%%%%%%%%%%%%%%%%%%%



%%%%%%%%%%%%%%%%%%%%%%%%%%%%%%%%%%%%%%%%%%%%%%%%%%%%%%%%%%
%% section 20.2 %%%%%%%%%%%%%%%%%%%%%%%%%%%%%%%%%%%%%%%%%%
%%%%%%%%%%%%%%%%%%%%%%%%%%%%%%%%%%%%%%%%%%%%%%%%%%%%%%%%%%
\modHeadsection{関連著作物の著作権および著作権者}


%%%%%%%%%%%%%%%%%%%%%%%%%%%%%%%%%%%%%%%%%%%%%%%%%%%%%%%%%%
%% subsection 20.2.1 %%%%%%%%%%%%%%%%%%%%%%%%%%%%%%%%%%%%%
%%%%%%%%%%%%%%%%%%%%%%%%%%%%%%%%%%%%%%%%%%%%%%%%%%%%%%%%%%
\subsection{著作人格権}
\index{ちょさくけんほう@著作権法}著作権法第59条より、すべての\index{かんれんちょさくぶつ@関連著作物}関連著作物の\index{ちょさくじんかくけん@著作人格権}著作人格権は、その\index{ちょさくしゃ@著作者}著作者に帰属する。
また、原則として著作人格権の行使の判断・決定は、著作者に委れられるものとする。


%%%%%%%%%%%%%%%%%%%%%%%%%%%%%%%%%%%%%%%%%%%%%%%%%%%%%%%%%%
%% subsection 20.2.2 %%%%%%%%%%%%%%%%%%%%%%%%%%%%%%%%%%%%%
%%%%%%%%%%%%%%%%%%%%%%%%%%%%%%%%%%%%%%%%%%%%%%%%%%%%%%%%%%
\subsection{著作財産権}
\index{ちょさくけんほう@著作権法}著作権法第15条(職務著作物)より、関連著作物が職務上作成された著作物(\index{しょくむちょさくぶつ@職務著作物}職務著作物)に該当する場合、その\index{ちょさくざいさんけん@著作財産権}著作財産権はその業務を指示した法人に帰属する。

関連著作物が職務著作物に該当しない場合、その著作財産権は著作者個人に帰属する。



\clearpage
%%%%%%%%%%%%%%%%%%%%%%%%%%%%%%%%%%%%%%%%%%%%%%%%%%%%%%%%%%
%% section 20.2 %%%%%%%%%%%%%%%%%%%%%%%%%%%%%%%%%%%%%%%%%%
%%%%%%%%%%%%%%%%%%%%%%%%%%%%%%%%%%%%%%%%%%%%%%%%%%%%%%%%%%
\modHeadsection{職務著作物}
\index{ちょさくけんほう@著作権法}著作権法第15条(職務著作物)より、作成された\index{ちょさくぶつ@著作物}著作物が以下の4つの要件をすべて満たす場合に限り、その著作物は\index{しょくむちょさくぶつ@職務著作物}職務著作物に該当する
\begin{enumerate}[label=\Roman*, ref=\Roman*]
\item 法人等の発意に基づくこと
\item 法人等の業務に従事する者が職務上作成するものであること
\item\label{item:copyrightrule3} 法人等の名義の下に公表するものであること
\item 作成の時における契約、勤務規則その他に別段の定めがないこと
\end{enumerate}
この4つの要件をもう少し具体的に述べると、
\begin{enumerate}[label=\Roman*$'$]
\item 法人等がある目的を持って構想した著作物の具体的な作成を従業員に命じることを意味する
\item その著作物が従業員の通常の業務範囲内で作成されたものであれば、それは依然として職務著作物に該当する可能性がある
\item その著作物が作成した業務従事者の名前で公表されれば職務著作物とは認められない
\item その著作物についての契約や勤務規則等に、別段の(適切な)定めがある場合は、その定めに従う
\end{enumerate}
ただし一般に、\ref{item:copyrightrule3}{}について、プログラムの著作物に関してはこの要件は不要とされる。



%\clearpage
%%%%%%%%%%%%%%%%%%%%%%%%%%%%%%%%%%%%%%%%%%%%%%%%%%%%%%%%%%
%% section 20.4 %%%%%%%%%%%%%%%%%%%%%%%%%%%%%%%%%%%%%%%%%%
%%%%%%%%%%%%%%%%%%%%%%%%%%%%%%%%%%%%%%%%%%%%%%%%%%%%%%%%%%
\modHeadsection{関連著作物の公表}
著作権法第18条(公表権)より、著作者は、その著作物でまだ公表されていないもの(その同意を得ないで公表された著作物を含む)を公衆に提供し、または提示する権利を有する(当該著作物を原著作物とする二次的著作物についても同様)。


%%%%%%%%%%%%%%%%%%%%%%%%%%%%%%%%%%%%%%%%%%%%%%%%%%%%%%%%%%
%% subsection 20.4.1 %%%%%%%%%%%%%%%%%%%%%%%%%%%%%%%%%%%%%
%%%%%%%%%%%%%%%%%%%%%%%%%%%%%%%%%%%%%%%%%%%%%%%%%%%%%%%%%%
\subsection{公表する関連著作物}
関連著作物については、開発・保守等の生産性の向上を目的に、原則としてすべてオンライン上に公表する。
ただし、著作権法第18条に基づき、公表の判断はその\index{ちょさくじんかくけん@著作人格権}著作人格権の保有者(\index{ちょさくしゃ@著作者}著作者)および\index{ちょさくざいさんけん@著作財産権}著作財産権の保有者の双方の同意の下で行われることを前提とする。
また、公表の際は個人情報保護法(データ保護・プライバシー保護)にも配慮しなければならない。


%%%%%%%%%%%%%%%%%%%%%%%%%%%%%%%%%%%%%%%%%%%%%%%%%%%%%%%%%%
%% subsection 20.4.2 %%%%%%%%%%%%%%%%%%%%%%%%%%%%%%%%%%%%%
%%%%%%%%%%%%%%%%%%%%%%%%%%%%%%%%%%%%%%%%%%%%%%%%%%%%%%%%%%
\subsection{非公表にする関連著作物}
たとえば作成したメインプログラムやモールドのデータベースについては、個々の明細の情報(機密事項)を推察できるデータを含む。
このような機密事項を含む著作物については、原則として公表しないものとする。
あるいは、公表する場合でも(個々のものすべてでなく)代表的なものに留めるものとする。



\clearpage
\begin{Column}{\DMname の関連著作物}
\DMname のソフトウェアにおける関連著作物の場合、その一切が著作者個人に(管理職・スタッフにより)一任されており、
\begin{enumerate}[label=\Roman*]
\item 法人等がある目的を持って構想した著作物が(少なくともソフトウェアに関しては)全く存在しない。
\item 著作者は、ソフトウェア開発・諸規定等の策定・技術書の作成などを行っており、通常の業務範囲から明らかに逸脱している。
\item 関連著作物は著作者個人の名前あるいはアカウントの下にオンライン上で公表されている。
\item ソフトウェア作成時において、その著作物についての別段の定め等は一切ない。
\end{enumerate}
したがって、いずれの要件も満たしていないため、その\index{ちょさくざいさんけん@著作財産権}著作財産権は著作者個人に帰属する。
\end{Column}




