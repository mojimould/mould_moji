%!TEX root = ../RPA_for_Creating_Program_Note.tex


%%%%% GRAPHICSPATH %%%%%%%%%%%%%%%%%%%%%%%%%%%%%%%
\graphicspath{{./RfCPN_a1_pictures/}}
%%%%% FONT %%%%%%%%%%%%%%%%%%%%%%%%%%%%%%%
\setmainjfont[BoldFont={HaranoAjiGothic}, BoldFeatures={FakeBold=2}]{HaranoAjiMincho}
\setsansjfont{HaranoAjiGothic}
%\setmonojfont{HaranoAjiMincho}
%\setmainfont{Latin Modern Roman}
%\setsansfont{Latin Modern Sans}
%\setmonofont{Latin Modern Mono}
\setmathfont[BoldFont=*, BoldFeatures={FakeBold=2}]{Euler Math}
\setmathfont[range={\leq, \geq},  BoldFont=*, BoldFeatures={FakeBold=2}]{Latin Modern Math}
\setmathfont[range={up/num, bfup/num}, Scale=MatchUppercase, BoldFont=*, BoldFeatures={FakeBold=2}]{Latin Modern Math}
%%%%% NEWIF %%%%%%%%%%%%%%%%%%%%%%%%%%%%%%%%%%
\newif\if@backmatter%\@backmattertrue
\newif\if@frontmatter%\@frontmattertrueko
\newif\if@appendix%\@appendixtrue
%%%%%% DIMENSION %%%%%%%%%%%%%%%%%%%%%%%%%%%%%%
\ltjsetparameter{kanjiskip=0.0pt plus 0.4pt minus 0.5pt}
\ltjsetparameter{xkanjiskip=2.40555pt plus 1.0pt minus 1.0pt}
\newcommand{\hk}{\hspace{\glueexpr\ltjgetparameter{kanjiskip}\relax}}
\newcommand{\hx}{\hspace{\glueexpr\ltjgetparameter{xkanjiskip}\relax}}

%%%%% TMPPAGE %%%%%%%%%%%%%%%%%%%%%%%%%%%%%%%%%%
\let\tmpcleardoublepage\cleardoublepage
\let\tmpclearpage\clearpage
\let\tmpnewpage\newpage
%%%%% CLEARRIGHTPAGE %%%%%%%%%%%%%%%%%%%%%%%%%%
\newcommand{\clearrightpage}{%
  \if@twoside%
    \ifodd\numexpr\thepage\relax\cleardoublepage\else\clearpage\fi
  \else\clearpage%
  \fi
}

%%%%% DECLARENEWTOC %%%%%%%%%%%%%%%%%%%%
\DeclareNewTOC[owner=\jobname, name=Part]{lop}
\DeclareNewTOC[owner=\jobname, name=Column]{loC}
%%%%% TO GET CHAPTER TITLE %%%%%%
\newcommand\Chaptername{} % initialize \Chaptername
\let\old@chapter\@chapter
\def\@chapter[#1]#2{\gdef\Chaptername{#2}\old@chapter[#1]{#2}}
%%%%% TO GET SECTION TITLE %%%%%%
\newcommand\Sectionname{} % initialize \Sectionname
\let\Sectionmark\sectionmark
\def\sectionmark#1{\def\Sectionname{#1}\Sectionmark{#1}}
%%%%% FOR REFERENCES %%%%%%%%%%%%%%%%%%%%%%%%%
\DeclareLanguageMapping{japanese}{english}
\newcommand{\Articlename}{論文}
\newcommand{\Bookname}{書籍}
\newcommand{\OnlineSourcename}{ウェブサイト}
\newcommand{\Manualname}{マニュアル}
\DeclareFieldFormat{urldate}{(urlseen~\thefield{urlyear}/\ifnum\thefield{urlmonth}<10 0\fi\thefield{urlmonth})}
