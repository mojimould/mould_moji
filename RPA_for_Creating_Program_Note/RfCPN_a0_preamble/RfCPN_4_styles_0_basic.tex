%!TEX root = ../RPA_for_Creating_Program_Note.tex


%%%%% GEOMETRY %%%%%%%%%%%%%%%%%%%%%%%%%%%%%%%
\geometry{
  a4paper, % paper size
  centering,
  textwidth={6.5in},
  includehead,  % include the head of the page
%  headheight = 13.6pt,
  includefoot,  % include the foot of the page
  top=15.0truemm,
  bottom=-0.5truemm,
}
%%%%% HEYPERSETUP %%%%%%%%%%%%%%%%%%%%%%%%%%%%
\hypersetup{
%  pdfcreationdate=date,
%  pdfcreator={luaLaTeX with hyperref}, % creator for PDF subjct field
  pdftitle={RPAに向けたRDBSの構築と応用}, % title for PDF subjct field
  pdfsubject={プログラム作成業務の自動化}, % text for PDF subjct field
  pdfauthor={Kurahashi Nobuaki},  % text for PDF Author field
  pdfkeywords={
    mold,
    mould,
    G-code,
    NC program,
    Machining center,
  },
  pdfproducer=producer,
  linktoc=all,
%  linktocpage=false,   % (if it is true) make page number, not text, be link on TOC, LOF and LOT
  pdfcenterwindow=false, % position the document window center of the screen
  pdffitwindow=true,     % resize document window to fit document size
  bookmarksnumbered=true,
  bookmarksopen=true, %bookmarks open
  pdfstartview={FitH}, % Fit, FitV, FitH, FitB
  pdfpagemode=UseThumbs, % set default mode of PDF display
  unicode=true,
  pdfencoding=unicode,   % PDFDocEncoding or Unicode
  colorlinks=true,     % color links
  linkcolor=ai,        % color of links
  urlcolor=ai,         % color of urls
  citecolor=sora,      % color of citation links
}
%%%%% DISPLAYBREAK %%%%%%%%%%%%%%%%%%%%%%%%%%%
\allowdisplaybreaks
%%%%% UNIT LENGTH %%%%%%%%%%%%%%%%%%%%%%%%%%%%
\setlength{\unitlength}{1pt}
%%%%% LINESPREAD %%%%%%%%%%%%%%%%%%%%%%%%%%%%%
\linespread{1.15}\selectfont
%%%%% PARINDENT %%%%%%%%%%%%%%%%%%%%%%%%%%%%%%
\newcommand{\indentspace}{\setlength\parindent{11pt}}
\indentspace
%%%%% EQUATION %%%%%%%%%%%%%%%%%%%%%%%%%%%%
\renewcommand{\theequation}{\thesection.\arabic{equation}}
\@addtoreset{equation}{section}
%%%%% FOR FOOTNOTE %%%%%%%%%%%%%%%%%%%%%%%%%%%
\renewcommand*{\footnoteautorefname}{脚注}
\interfootnotelinepenalty=10000
\counterwithout{footnote}{chapter}
\def\@makefnmark{\hbox{}\hbox{\@textsuperscript{\normalfont\@thefnmark}}\hbox{}}
\deffootnote[1em]{1em}{1em}{\textsuperscript{\thefootnotemark}}
\renewcommand\footnoterule{%
  \kern3pt
  \hrule\@width.75\columnwidth
  \kern2.6pt
}
\makesavenoteenv{Column}
\makesavenoteenv{hosoku}
\makesavenoteenv{tablePart}
\makesavenoteenv{marker}
\makesavenoteenv{Formula}
%%%%% SETLIST %%%%%%%%%%%%%%%%%%%%%%%%%%%%%%%%
\setlist[enumerate]{listparindent=\parindent, parsep=0pt, partopsep=0pt, topsep=3pt, itemsep=3pt, leftmargin=*}
\setlist[enumerate, 1]{leftmargin=\leftmargini}
%%%%% CAPTION STYLE %%%%%%%%%%%%%%%%%%%%%%%%%%
\captionsetup[figure]{%
  width=.9\textwidth,
  format=hang,
  labelfont={bf, sf},
  labelsep={colon},
  labelformat=simple,
  aboveskip=3pt,
}
\captionsetup[lstlisting]{
  justification=raggedright,
  singlelinecheck=false,
  position=above,
  aboveskip=5pt,
  belowskip=0.0pt,
  labelformat={empty},
  labelfont={bf, sf},
  labelsep={space},
  font={bf, large, sf},
  hypcap=false,
}
\captionsetup[table]{
  justification=raggedright,
  singlelinecheck=false,
  position=above,
  aboveskip=4pt,
  belowskip=5pt,
  labelformat={empty},
  labelfont={bf, sf},
  labelsep={space},
  font={bf, large, sf},
  hypcap=false,
}
%%%%% FONT STYLE %%%%%%%%%%%%%%%%%%%%%%%%%%
\setmathfont[range=\leq]{Latin Modern Math}
\setmathfont[range=\geq]{Latin Modern Math}
\setmathfont[range={up/num, bfup/num}, Scale=MatchUppercase]{Latin Modern Math}
