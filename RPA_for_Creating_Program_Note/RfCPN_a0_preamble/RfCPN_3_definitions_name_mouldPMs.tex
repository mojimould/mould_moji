%!TEX root = ../RPA_for_Creating_Program_Note.tex


%%%%% NEWNAMECOMMAND %%%%%%%%%%%%%%%%%%%%%%%%%%%
\newcommand{\newNamecommand}[3]{%
  \expandafter\newcommand\csname yomi#1\endcsname{#3}%
  \expandafter\newcommand\csname name#1\endcsname{#2}%
  \expandafter\newcommand\csname #1\endcsname{\expandafterindex{#3@#2}#2}%
}
%%%%% MACHINING NAME %%%%%%%%%%%%%%%%%%
\newNamecommand{DMC}{Dマシニングセンタ}{Dマシニングセンタ}
\newNamecommand{MMC}{Mマシニングセンタ}{Mマシニングセンタ}
\newNamecommand{Dimple}{内面ディンプル}{ないめんディンプル}
\newNamecommand{Keyway}{キー溝}{キーみぞ}
\newNamecommand{TanmenZaguri}{端面座ぐり}{たんめんざぐり}


%%%%% FOR TCBOX %%%%%%%%%%%%%%%%%%%%
\tcbset{%
  commonbox/.style={%
    enhanced,
    nobeforeafter,
    tcbox raise base,
    boxrule=0.4pt,
    top=0mm,
    bottom=0mm,
    right=0mm,
    left=4mm,
    arc=1pt,
    boxsep=3pt,
    left skip=2pt,
    right skip=2pt,
    before upper={\vphantom{dl}},
    fontupper=\ttfamily,
  },
}

%%%%% PMBOX %%%%%%%%%%%%%%%%%%%%%%%%%%%
\newtcbox{\PMbox}{%
  commonbox,
  enlarge top by=2pt,
  top=0pt,
  bottom=0pt,
  colframe=green!50!black,
  colback=green!10!white,
  overlay={%
    \begin{tcbclipinterior}%
      \fill[green!80!blue!50!black!50!] (frame.south west)
      rectangle node[text=white, font=\bfseries\tiny, rotate=90] {{\fontspec{Latin Modern Sans}PM}} ([xshift=4mm]frame.north west);
    \end{tcbclipinterior}%
  },
}
\robustify{\PMbox}
\pdfstringdefDisableCommands{%
  \def\PMbox#1{`#1'}%
}

%%%%% NEWPRGCOMMAND %%%%%%%%%%%%%%%%%%%%%%%%%%%
\newcommand{\newPMcommand}[3]{%
  \expandafter\newcommand\csname yomi#1\endcsname{#3}%
  \expandafter\newcommand\csname name#1\endcsname{#2}%
  \expandafter\newcommand\csname #1\endcsname{\expandafterindex{#3@#2}\csname name#1\endcsname}%
  \expandafter\newcommand\csname PM#1\endcsname{\expandafterindex{#3@#2}\PMbox{\csname name#1\endcsname}}%
}

%%%%% TO PARAMETER NAME %%%
\newPMcommand{DrawingExists}{図面の有無}{ずめんのうむ}
\newPMcommand{DrawingNumber}{図面番号}{ずめんばんごう}

\newPMcommand{CurvatureExists}{湾曲の有無}{わんきょくのうむ}
\newPMcommand{CenterCurvature}{中心湾曲半径}{ちゅうしんわんきょくはんけい}

\newPMcommand{ACOD}{AC外径}{ACがいけい}
\newPMcommand{BDOD}{BD外径}{BDがいけい}
\newPMcommand{ODCornerR}{外径コーナーR}{がいけいコーナーR}

\newPMcommand{TopEndFace}{トップ端面}{トップたんめん}
\newPMcommand{BottomEndFace}{ボトム端面}{ボトムたんめん}
\newPMcommand{TopEndACID}{トップ端AC内径}{トップたんACないけい}
\newPMcommand{TopEndBDID}{トップ端BD内径}{トップたんBDないけい}
\newPMcommand{BottomEndACID}{ボトム端AC内径}{ボトムたんACないけい}
\newPMcommand{BottomEndBDID}{ボトム端BD内径}{ボトムたんBDないけい}
\newPMcommand{TopEndIDCornerR}{トップ端内径コーナーR}{トップたんないけいコーナーR}
\newPMcommand{BottomEndIDCornerR}{ボトム端内径コーナーR}{ボトムたんないけいコーナーR}

\newPMcommand{TopOutCutExists}{トップ外削の有無}{トップがいさくのうむ}
\newPMcommand{BottomInCutExists}{ボトム外削の有無}{ボトムがいさくのうむ}
\newPMcommand{TopOutCutShape}{トップ外削の種類}{トップがいさくのしゅるい}
\newPMcommand{BottomOutCutShape}{ボトム外削の種類}{ボトムがいさくのしゅるい}

\newPMcommand{KeywayType}{\nameKeyway の種類}{\yomiKeyway のしゅるい}
\newPMcommand{KeywayPos}{\nameKeyway 位置長}{\yomiKeyway いちちょう}
\newPMcommand{KeywayWidth}{\nameKeyway 幅}{\yomiKeyway はば}
\newPMcommand{AsideKeywayDepth}{A側\nameKeyway 深さ}{Aがわ\yomiKeyway ふかさ}
\newPMcommand{KeywayACOD}{\nameKeyway AC径}{\yomiKeyway ACけい}
\newPMcommand{KeywayBDOD}{\nameKeyway BD径}{\yomiKeyway BDけい}
\newPMcommand{KeywayCornerC}{\nameKeyway コーナーC}{\yomiKeyway コーナーC}
\newPMcommand{KeywayCornerR}{\nameKeyway コーナーR}{\yomiKeyway コーナーR}

\newPMcommand{ChamferType}{面取の種類}{めんとりのしゅるい}
\newPMcommand{CChamferLength}{C面取長}{Cめんとりちょう}
\newPMcommand{CChamferAngle}{C面取角}{Cめんとりかく}
\newPMcommand{RChamferLength}{R面取長}{Rめんとりちょう}

\newPMcommand{EndFaceBoringExists}{\nameTanmenZaguri の有無}{\yomiTanmenZaguri のうむ}
\newPMcommand{EndFaceBoringCornerR}{\nameTanmenZaguri コーナーR}{\yomiTanmenZaguri コーナーR}
