%!TEX root = ../RPA_for_Creating_Program_Note.tex


%%%%% NEWNAMECOMMAND %%%%%%%%%%%%%%%%%%%%%%%%%%%
\newcommand{\newNamecommand}[3]{%
  \expandafter\newcommand\csname name#1\endcsname{\expandafterindex{#3@#2}#2}%
  \expandafter\newcommand\csname #1\endcsname{\csname name#1\endcsname}%
}
%%%%% MACHINING NAME %%%%%%%%%%%%%%%%%%
\newcommand{\DMname}{Dマシニングセンタ}
\newcommand{\DMnamekana}{Dマシニングセンタ}
\newNamecommand{DMCname}{\DMname}{\DMnamekana}
\newcommand{\MMname}{Mマシニングセンタ}
\newcommand{\MMnamekana}{Mマシニングセンタ}
\newNamecommand{MMCname}{\MMname}{\MMnamekana}
\newcommand{\dimple}{内面ディンプル}
\newcommand{\dimplekana}{ないめんディンプル}
\newNamecommand{Dimple}{\dimple}{\dimplekana}


%%%%% PMBOX %%%%%%%%%%%%%%%%%%%%%%%%%%%
\newtcbox{\PMbox}{%
  commonbox,
  enlarge top by=2pt,
  top=0pt,
  bottom=0pt,
  colframe=green!50!black,
  colback=green!10!white,
  overlay={%
    \begin{tcbclipinterior}%
      \fill[green!80!blue!50!black!50!] (frame.south west)
      rectangle node[text=white, font=\bfseries\tiny, rotate=90] {PM} ([xshift=4mm]frame.north west);
    \end{tcbclipinterior}%
  },
}
\robustify{\PMbox}
\pdfstringdefDisableCommands{%
  \def\PMbox#1{`#1'}%
}

%%%%% NEWPRGCOMMAND %%%%%%%%%%%%%%%%%%%%%%%%%%%
\newcommand{\newPMcommand}[3]{%
  \expandafter\newcommand\csname name#1\endcsname{\expandafterindex{#3@#2}#2}%
  \expandafter\newcommand\csname #1\endcsname{\PMbox{\csname name#1\endcsname}}%
}

%%%%% TO PARAMETER NAME %%%
\newPMcommand{DrawingExists}{図面の有無}{ずめんのうむ}
\newPMcommand{DrawingNumber}{図面番号}{ずめんばんごう}

\newPMcommand{CurvatureExists}{湾曲の有無}{わんきょくのうむ}
\newPMcommand{CenterCurvature}{中心湾曲}{ちゅうしんわんきょく}

\newPMcommand{ACOD}{AC外径}{ACがいけい}
\newPMcommand{BDOD}{BD外径}{BDがいけい}
\newPMcommand{ODCornerR}{外径コーナーR}{がいけいコーナーR}

\newPMcommand{TopEndFace}{トップ端面}{トップたんめん}
\newPMcommand{BottomEndFace}{ボトム端面}{ボトムたんめん}
\newPMcommand{TopEndACID}{トップ端AC内径}{トップたんACないけい}
\newPMcommand{TopEndBDID}{トップ端BD内径}{トップたんBDないけい}
\newPMcommand{BottomEndACID}{ボトム端AC内径}{ボトムたんACないけい}
\newPMcommand{BottomEndBDID}{ボトム端BD内径}{ボトムたんBDないけい}
\newPMcommand{TopEndIDCornerR}{トップ端内径コーナーR}{トップたんないけいコーナーR}
\newPMcommand{BottomEndIDCornerR}{ボトム端内径コーナーR}{ボトムたんないけいコーナーR}

\newPMcommand{TopOutCutExists}{トップ外削の有無}{トップがいさくのうむ}
\newPMcommand{BottomInCutExists}{ボトム外削の有無}{ボトムがいさくのうむ}
\newPMcommand{TopOutCutShape}{トップ外削の種類}{トップがいさくのしゅるい}
\newPMcommand{BottomOutCutShape}{ボトム外削の種類}{ボトムがいさくのしゅるい}

\newPMcommand{KeywayType}{溝の種類}{みぞのしゅるい}
\newPMcommand{KeywayPos}{溝位置}{みぞいち}
\newPMcommand{KeywayWidth}{溝幅}{みぞはば}
\newPMcommand{AsideKeywayDepth}{A側溝深さ}{Aがわみぞふかさ}
\newPMcommand{KeywayACOD}{溝AC径}{みぞACけい}
\newPMcommand{KeywayBDOD}{溝BD径}{みぞBDけい}
\newPMcommand{KeywayCornerC}{溝コーナーC}{みぞコーナーC}
\newPMcommand{KeywayCornerR}{溝コーナーR}{みぞコーナーR}

\newPMcommand{ChamferType}{面取の種類}{めんとりのしゅるい}
\newPMcommand{CChamferLength}{C面取長}{Cめんとりちょう}
\newPMcommand{CChamferAngle}{C面取角}{Cめんとりかく}
\newPMcommand{RChamferLength}{R面取長}{Rめんとりちょう}

\newPMcommand{EndFaceBoringExists}{端面座ぐりの有無}{たんめんざくりのうむ}
\newPMcommand{EndFaceBoringCornerR}{端面座ぐりコーナーR}{たんめんざくりコーナーR}
