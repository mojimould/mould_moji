%!TEX root = ../RPA_for_Creating_Program_Note.tex


%%%%% COLOR %%%%%%%%%%%%%%%%%%%%%%%%%%%%%%%%%%
\definecolor{ai}     {rgb}{0.2039, 0.3765, 0.4314}
\definecolor{hosoku}{cmyk}{0, 0, 0, .1}
\definecolor{kon}    {rgb}{0.0000, 0.2000, 0.4000}
\definecolor{konpeki}{rgb}{0.0902, 0.5098, 0.7333}
\definecolor{moegi}  {rgb}{0.3020, 0.5961, 0.1882}
\definecolor{sssec}  {rgb}{0.7333, 0.5, 0.7333}
\definecolor{sora}   {rgb}{0.1451, 0.7216, 0.8039}
\definecolor{sumire} {rgb}{0.3882, 0.2157, 0.5922}
\colorlet{unusingVariables}{gray!35!}
\newcommand{\getlinkcolor}{\@linkcolor}

%%%%%%%%%%%%%%%%%%%%%%%%%%%%%%%%%%%%%%%%%%%%%%
%%%%% NEWTCOLORBOX %%%%%%%%%%%%%%%%%%%%%%%%%%%
%%%%%%%%%%%%%%%%%%%%%%%%%%%%%%%%%%%%%%%%%%%%%%
\newcounter{GlobalFootnote}% difine a counter Global Footnote
%%%%% COLUMN %%%%%
\newcommand{\Columnname}{Column}
\newtcolorbox[auto counter, number within=chapter, list inside=loC]{Column}[2][]{%
  Columnbox,
  title={#2},
  #1,
  list entry={\numberline{\protect\hspace*{9.5pt}\Columnname~\thetcbcounter}{\protect\hspace*{55pt}#2}}
}
%\newcommand{\tcb@cnt@Columnautorefname}{Column}
%%%%% Formula %%%%%
\newcommand{\Formulaname}{Eq.}
\newtcolorbox[auto counter, number within=chapter]{Formula}[2][]{%
  Formulabox, title={#2}, #1}
\newcommand{\tcb@cnt@Formulaautorefname}{Eq.\!}
%%%%% FIGLANDSCA`EBOX %%%%%
\newtcolorbox{Figlandscape}[1][]{Figurelnadscapebox, #1}
%%%%% FIGBOX %%%%%
\newtcolorbox{Figbox}[1][]{Figurebox, #1}
%%%%% TABBOX %%%%%
\newtcolorbox{Tabbox}[1][]{Tabularbox, #1}
%\renewcommand{\tableautorefname}{表}
%%%%% TOCBOX %%%%%
\newtcolorbox{tocBox}[1]{%
  tocstyle,%
  watermark text={\sffamily\bfseries{#1}},%
}
%%%%% LOFBOX %%%%%
\newtcolorbox{lofBox}{lofstyle}
%%%%% LOTBOX %%%%%
\newtcolorbox{lotBox}{lotstyle}
%%%%% LOlBOX %%%%%
\newtcolorbox{lolBox}{lolstyle}
%%%%% LOCBOX %%%%%
\newtcolorbox{loCBox}{loCstyle}
%%%%% LOPBOX %%%%%
\newtcolorbox{lopBox}[1]{%
  lopstyle,%
  title={#1},%
}
%%%%% HOSOKU %%%%%
\newcommand{\hosokuname}{補}
\newtcolorbox[auto counter, number within=chapter]{hosoku}[1][]{hosokubox, #1}
\newcommand{\tcb@cnt@hosokuautorefname}{補足}
%%%%% ISSUE %%%%%
\newcommand{\issuename}{Issue}
\newtcolorbox{Issues}[2][]{%
  Issuebox,
  title={#2},
  #1,
}
%%%%% ISSUE %%%%%
\newcommand{\parametername}{Parameter}
\newtcolorbox{Parameter}[2][]{%
  Parameterbox,
  title={#2},
  #1,
}
%%%%% MARKER %%%%%
\newtcolorbox{marker}[1][]{
  enhanced,
  before skip balanced=2mm,
  after skip balanced=3mm,
  boxrule=0.4pt,
  left=5mm,
  right=2mm,
  top=1mm,
  bottom=1mm,
  colback=yellow!50,
  colframe=yellow!50!black,
  sharp corners,
  rounded corners=southeast,
  arc is angular,
  arc=3mm,
  overlay={%
    \path[fill=tcbcolback!80!black] ([yshift=3mm]interior.south east)--++(-0.4,-0.1)--++(0.1,-0.2);
    \path[draw=tcbcolframe, shorten <=-0.05mm,shorten >=-0.05mm] ([yshift=3mm]interior.south east)--++(-0.4,-0.1)--++(0.1,-0.2);
    \path[fill=yellow!50!black,draw=none] (interior.south west) rectangle node[white]{\Huge\bfseries !} ([xshift=4mm]interior.north west);
    },
  drop fuzzy shadow,
  #1,
}
%%%%% TABLEPART %%%%%
\newtcolorbox[auto counter, number within=part]{tablePart}[2][]{%
  Columnbox,
  title={\termblue{{\sffamily\bfseries\thepart}:#2}},
  #1,
  after title={}
}

