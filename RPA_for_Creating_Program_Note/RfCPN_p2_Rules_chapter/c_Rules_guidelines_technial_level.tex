%!TEX root = ../RPA_for_Creating_Program_Note.tex



%%%%%%%%%%%%%%%%%%%%%%%%%%%%%%%%%%%%%%%%%%%%%%%%%%%%%%%%%%
%% section 07.1 %%%%%%%%%%%%%%%%%%%%%%%%%%%%%%%%%%%%%%%%%%
%%%%%%%%%%%%%%%%%%%%%%%%%%%%%%%%%%%%%%%%%%%%%%%%%%%%%%%%%%
\modHeadsection{目的}
\begin{enumerate}
\item \index{じょうほうしょりぎじゅつしゃ@情報処理技術}情報処理技術者(利活用者を含む)として備えるべき能力についての水準を示すことにより、社内教育における水準の確保に資すること
\item 情報処理技術者(利活用者を含む)の起用を行う際に有用となる、客観的な評価の尺度を提供すること
\end{enumerate}


%%%%%%%%%%%%%%%%%%%%%%%%%%%%%%%%%%%%%%%%%%%%%%%%%%%%%%%%%%
%% section 07.2 %%%%%%%%%%%%%%%%%%%%%%%%%%%%%%%%%%%%%%%%%%
%%%%%%%%%%%%%%%%%%%%%%%%%%%%%%%%%%%%%%%%%%%%%%%%%%%%%%%%%%
\section{利用者および技術者の区分}

%%%%%%%%%%%%%%%%%%%%%%%%%%%%%%%%%%%%%%%%%%%%%%%%%%%%%%%%%%
%% subsection 07.2.1 %%%%%%%%%%%%%%%%%%%%%%%%%%%%%%%%%%%%%
%%%%%%%%%%%%%%%%%%%%%%%%%%%%%%%%%%%%%%%%%%%%%%%%%%%%%%%%%%
\subsection{ITを利活用する者}
\begin{enumerate}[label*=\Roman*., ref=\Roman*]
\item\label{item:ITseg1}
職業人として備えておくべきITに関する共通的な基礎知識をもち、担当業務に対してITを活用していこうとする者
\item\label{item:ITseg2}
担当業務の遂行に必要な情報セキュリティ対策を適切に理解し、情報および情報システムを安全に活用するために、\index{じょうほうセキュリティ@情報セキュリティ}情報セキュリティが確保された状況を実現し、維持・改善する者
\end{enumerate}

%%%%%%%%%%%%%%%%%%%%%%%%%%%%%%%%%%%%%%%%%%%%%%%%%%%%%%%%%%
%% subsection 07.2.2 %%%%%%%%%%%%%%%%%%%%%%%%%%%%%%%%%%%%%
%%%%%%%%%%%%%%%%%%%%%%%%%%%%%%%%%%%%%%%%%%%%%%%%%%%%%%%%%%
\subsection{情報処理技術者}
\begin{enumerate}[start=3, label*=\Roman*., ref=\Roman*]
\item\label{item:ITseg3}
ITを活用したサービス・製品・システムおよびソフトウェアを作る人材に必要な基本的知識・技能をもち、実践的な活用能力を身に付けた者
\item\label{item:ITseg4}
ITを活用したサービス・製品・システムおよびソフトウェアを作る人材に必要な応用的知識・技能をもち、高度IT人材としての方向性を確立した者
\end{enumerate}


\clearpage
%%%%%%%%%%%%%%%%%%%%%%%%%%%%%%%%%%%%%%%%%%%%%%%%%%%%%%%%%%
%% section 07.3 %%%%%%%%%%%%%%%%%%%%%%%%%%%%%%%%%%%%%%%%%%
%%%%%%%%%%%%%%%%%%%%%%%%%%%%%%%%%%%%%%%%%%%%%%%%%%%%%%%%%%
\section{期待される技術水準}
上記の区分\ref{item:ITseg1}, \ref{item:ITseg2}, \ref{item:ITseg3}, \ref{item:ITseg4}\hx に該当する者は、それぞれ以下のような\index{ぎじゅつすいじゅん@技術水準}技術水準を有することが期待される。

%%%%%%%%%%%%%%%%%%%%%%%%%%%%%%%%%%%%%%%%%%%%%%%%%%%%%%%%%%
%% subsection 07.3.1 %%%%%%%%%%%%%%%%%%%%%%%%%%%%%%%%%%%%%
%%%%%%%%%%%%%%%%%%%%%%%%%%%%%%%%%%%%%%%%%%%%%%%%%%%%%%%%%%
\subsection{区分\ref{item:ITseg1}}
\begin{enumerate}
\item 利用する情報機器やシステムを把握するために、コンピュータシステム・データベース・ネットワーク・情報セキュリティ・情報デザイン・情報メディアに関する知識をもち、オフィスツールを活用できる
\item 担当業務の問題把握および必要な解決を図るためにデータを利活用し、システム的な考え方や論理的な思考力(プログラミング的思考力等)を有し、かつ問題分析および問題解決手法に関する知識をもつ
\item 安全に情報を収集し、効果的に活用するために、関連法規・情報セキュリティに関する各種規程・情報倫理に従って活動できる
\item 業務の分析やシステム化の支援を行うために、情報システムの開発・運用に関する知識をもつ
\item 新しい技術や新しい手法の概要に関する知識をもつ
\end{enumerate}

%%%%%%%%%%%%%%%%%%%%%%%%%%%%%%%%%%%%%%%%%%%%%%%%%%%%%%%%%%
%% subsection 07.3.2 %%%%%%%%%%%%%%%%%%%%%%%%%%%%%%%%%%%%%
%%%%%%%%%%%%%%%%%%%%%%%%%%%%%%%%%%%%%%%%%%%%%%%%%%%%%%%%%%
\subsection{区分\ref{item:ITseg2}}
\begin{enumerate}
\item 所属部門の情報セキュリティマネジメントの一部を独力で遂行できる
\item 情報セキュリティインシデントの発生またはそのおそれがあるときに、情報セキュリティリーダーとして適切に対処できる
\item IT全般に関する基本的な用語・内容を理解できる
\item 情報セキュリティ技術や情報セキュリティ諸規程に関する基本的な知識をもち、部門の情報セキュリティ対策の一部を独力、または上位者の指導のもとに実現できる
\item 情報セキュリティ機関や他の企業等の動向や事例を収集し、部門の環境への適用の必要性を評価できる
\end{enumerate}
なお、区分\ref{item:ITseg2}\hx に該当する者は、区分\ref{item:ITseg1}\hx の技術水準も有するものとする。

%%%%%%%%%%%%%%%%%%%%%%%%%%%%%%%%%%%%%%%%%%%%%%%%%%%%%%%%%%
%% subsection 07.3.3 %%%%%%%%%%%%%%%%%%%%%%%%%%%%%%%%%%%%%
%%%%%%%%%%%%%%%%%%%%%%%%%%%%%%%%%%%%%%%%%%%%%%%%%%%%%%%%%%
\subsection{区分\ref{item:ITseg3}}
\begin{enumerate}
\item IT全般に関する基本的事項を理解し、担当する業務に活用できる
\item 上位者の指導のもとに、IT戦略に関する予測・分析・評価に参加できる
\item 上位者の指導のもとに、システムまたはサービスの提案活動に参加できる
\item 上位者の指導のもとに、システムの企画・要件定義に参加できる
\item 上位者の指導のもとに、システムの設計・開発・運用が行える
\item 上位者の指導のもとに、ソフトウェアを設計できる
\item 上位者の方針を理解し、自らプログラムを作成できる
\end{enumerate}
なお、区分\ref{item:ITseg3}\hx に該当する者は、区分\ref{item:ITseg1}\hx の技術水準も有するものとする。

\clearpage
%%%%%%%%%%%%%%%%%%%%%%%%%%%%%%%%%%%%%%%%%%%%%%%%%%%%%%%%%%
%% subsection 07.3.4 %%%%%%%%%%%%%%%%%%%%%%%%%%%%%%%%%%%%%
%%%%%%%%%%%%%%%%%%%%%%%%%%%%%%%%%%%%%%%%%%%%%%%%%%%%%%%%%%
\subsection{区分\ref{item:ITseg4}}
\begin{enumerate}
\item 経営戦略・IT戦略の策定に際して、経営者の方針を理解し、経営を取巻く外部環境を正確に捉え、動向や事例を収集できる
\item 経営戦略・IT戦略の評価に際して、定められたモニタリング指標に基づいて差異分析などを行える
\item システムまたはサービスの提案活動に際して、提案討議に参加し、提案書の一部を作成できる
\item システムの企画・要件定義, アーキテクチャの設計において、システムに対する要求を整理し適用できる技術の調査が行える
\item 運用管理チーム・オペレーションチーム等の一員として、担当分野におけるサービス提供と安定稼働の確保が行える
\item プロジェクトメンバーとして、プロジェクトマネージャまたはリーダーのもとでスコープ・予算・工程・品質等の管理ができる
\item 情報システム・ネットワーク・データベース・組込みシステム等の設計・開発・運用・保守において、上位者の方針を理解し、自ら技術的問題を解決できる
\end{enumerate}
なお、区分\ref{item:ITseg4}\hx に該当する者は、区分\ref{item:ITseg2}\hx および\ref{item:ITseg3}\hx の技術水準も有するものとする。

