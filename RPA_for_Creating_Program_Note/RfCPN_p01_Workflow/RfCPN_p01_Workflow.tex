%!TEX root = ../RPA_for_Creating_Program_Note.tex


%%%%%%%%%%%%%%%%%%%%%%%%%%%%%%%%%%%%%%%%%%%%%%%%%%%%%%%%%
%% Part Work Flow %%%%%%%%%%%%%%%%%%%%%%%%%%%%%%%%%%%%%%%
%%%%%%%%%%%%%%%%%%%%%%%%%%%%%%%%%%%%%%%%%%%%%%%%%%%%%%%%%
\addtocontents{toc}{\protect\begin{tocBox}{\tmppartnum}}%
\tPart{現状の業務フローの把握および整理}{%
\paragraph*{目標(なにがしたいか?)}
現状の\index{よこがたマシニングセンタ@横型マシニングセンタ}横型マシニングセンタにおける作業において、\textbf{ソフトウェアの視点から現在の業務の流れを明確化}する。

\tcbline*
\paragraph*{手段(どうやって?)}
作業者・担当スタッフに\index{さぎょうてじゅん@作業手順}作業手順を詳しく聞き取りを行い、ソフトウェアの観点から現在の業務の流れを整理する。

\tcbline*
\paragraph*{背景(なぜ?)}
新たなマシニングセンタについて、ハードウェア視点では大きな問題はないという形で設置に至った。

 一方、ソフトウェア視点においては、\textbf{事態は惨憺たるもの}である。
「\index{ソフトウェアにかんするかんり・ぎょうむ@ソフトウェアに関する管理・業務}ソフトウェア関する管理・業務」を行う部門はおろか、担当者(特に管理職)さえ社内に存在しない。
そもそも業務の流れすら体系的に把握されておらず、したがって関連規程・標準や開発計画も事実上皆無である
%% footnote %%%%%%%%%%%%%%%%%%%%%
\footnote{つまり、ソフトウェアの観点からすると、各部門全体のレベルで全く機能しておらず、もはや企業としての体をなしていない。
(著者個人の意見・批評ではなく)\textbf{事実に基づいた客観的な帰結として}、「モラルの著しく悪い企業」と言わざるを得ない。
このような状態が数十年にわたって改善されることもなく(2024/04現在に至るまで)放置され続けている。

なおこのような状態は、いわゆる「\index{ブラックきぎょう@ブラック企業}ブラック企業」の典型的な特徴の1つである。
}。
%%%%%%%%%%%%%%%%%%%%%%%%%%%%%%%%%

 このように、客観的事実として、長期にわたり業務の中枢となる部分が全く機能していない。
そのため、加工システムの開発において、\index{かいはつプロセス@開発プロセス}\textbf{開発プロセスの最初期の段階である、現在の\index{ぎょうむフロー@業務フロー}業務フローの把握}から着手しなければならない。
}{%
\paragraph*{結論(どうなった?)}
現状の作業手順を把握し、それに基づきソフトウェアの観点から業務の流れの整理を行った。
併せて、安全性および生産性の低下となる要因の抽出も行った。
\tcbline*
\paragraph*{次の段階(それで?)}
整理した業務フローに基づき、新たなマシニングセンタにおける\textbf{システム構築に関する\index{ようけんていぎ@要件定義}要件定義}に着手する。
}

%%%%%%%%%%%%%%%%%%%%%%%%%%%%%%%%%%%%%%%%%%%%%%%%%%%%%%%%%%
%% chapters %%%%%%%%%%%%%%%%%%%%%%%%%%%%%%%%%%%%%%%%%%%%%%
%%%%%%%%%%%%%%%%%%%%%%%%%%%%%%%%%%%%%%%%%%%%%%%%%%%%%%%%%%
%!TEX root = ../RPA_for_Creating_Program_Note.tex


\modHeadchapter{現状の業務フローの理解}
新たに導入する\index{よこがたマシニングセンタ@横型マシニングセンタ}横型マシニングセンタ(以下、\textbf{\DMname})での工程は、\dimple の測定・加工を除けば\MMname と同様である。
そこで、まずは\MMname ではどのようなフローで業務が行われているかを(ソフトウェアの観点から)みることにする。
%%%%%%%%%%%%%%%%%%%%%%%%%%%%%%%%%%%%%%%%%%%%%%%%%%%%%%%%%%
%% marker %%%%%%%%%%%%%%%%%%%%%%%%%%%%%%%%%%%%%%%%%%%%%%%%
%%%%%%%%%%%%%%%%%%%%%%%%%%%%%%%%%%%%%%%%%%%%%%%%%%%%%%%%%%
\begin{marker}
ここでは主に\MMname の\expandafterindex{No.1パレット(\MMname)@No.1パレット(\MMname)}No.1パレットで加工を行うものを対象とする
%% footnote %%%%%%%%%%%%%%%%%%%%%
\footnote{\expandafterindex{No.2パレット(\MMname)@No.2パレット(\MMname)}No.2パレットでは、\index{おおがたのモールド@大型のモールド}径の大きなものや\index{まるがたのモールド@丸型のモールド}丸形のもの等の加工が主に行われる。}。
%%%%%%%%%%%%%%%%%%%%%%%%%%%%%%%%%
\end{marker}
%%%%%%%%%%%%%%%%%%%%%%%%%%%%%%%%%%%%%%%%%%%%%%%%%%%%%%%%%%
%%%%%%%%%%%%%%%%%%%%%%%%%%%%%%%%%%%%%%%%%%%%%%%%%%%%%%%%%%
%%%%%%%%%%%%%%%%%%%%%%%%%%%%%%%%%%%%%%%%%%%%%%%%%%%%%%%%%%
%%%%%%%%%%%%%%%%%%%%%%%%%%%%%%%%%%%%%%%%%%%%%%%%%%%%%%%%%%
%% marker %%%%%%%%%%%%%%%%%%%%%%%%%%%%%%%%%%%%%%%%%%%%%%%%
%%%%%%%%%%%%%%%%%%%%%%%%%%%%%%%%%%%%%%%%%%%%%%%%%%%%%%%%%%
\begin{marker}
ここで挙げている必要なパラメータ(\index{すんぽう@寸法}寸法)には、その\index{こうさ@公差}公差も考慮されているものとする。
\end{marker}
%%%%%%%%%%%%%%%%%%%%%%%%%%%%%%%%%%%%%%%%%%%%%%%%%%%%%%%%%%
%%%%%%%%%%%%%%%%%%%%%%%%%%%%%%%%%%%%%%%%%%%%%%%%%%%%%%%%%%
%%%%%%%%%%%%%%%%%%%%%%%%%%%%%%%%%%%%%%%%%%%%%%%%%%%%%%%%%%



%%%%%%%%%%%%%%%%%%%%%%%%%%%%%%%%%%%%%%%%%%%%%%%%%%%%%%%%%%
%% section 1.1 %%%%%%%%%%%%%%%%%%%%%%%%%%%%%%%%%%%%%%%%%%%
%%%%%%%%%%%%%%%%%%%%%%%%%%%%%%%%%%%%%%%%%%%%%%%%%%%%%%%%%%
\modHeadsection{現在のマシニングセンタの操作方法\TBW}



%%%%%%%%%%%%%%%%%%%%%%%%%%%%%%%%%%%%%%%%%%%%%%%%%%%%%%%%%%
%% section 1.2 %%%%%%%%%%%%%%%%%%%%%%%%%%%%%%%%%%%%%%%%%%%
%%%%%%%%%%%%%%%%%%%%%%%%%%%%%%%%%%%%%%%%%%%%%%%%%%%%%%%%%%
\modHeadsection{使用ソフトウェアおよびツール\TBW}
\index{モールド}モールドの\index{たんめんかこう@端面加工}端面加工・\index{がいさくかこう@外削加工}外削加工・\index{ないさくかこう@内削加工}内削加工・\index{みぞかこう@溝加工}溝加工・\index{ざぐりかこう(たんめん)@座ぐり加工(端面)}端面の座ぐり加工・\index{そとがわCめんとりかこう(たんめん)@外側C面取加工(端面)}端面の外側C面取加工・\index{うちがわCめんとりかこう(たんめん)@内側C面取加工(端面)}端面の内側C面取等は主に三菱製横型マシニングセンタ(以下、\textbf{\MMname})にて行われている。



%%%%%%%%%%%%%%%%%%%%%%%%%%%%%%%%%%%%%%%%%%%%%%%%%%%%%%%%%%
%% section 1.3 %%%%%%%%%%%%%%%%%%%%%%%%%%%%%%%%%%%%%%%%%%%
%%%%%%%%%%%%%%%%%%%%%%%%%%%%%%%%%%%%%%%%%%%%%%%%%%%%%%%%%%
\modHeadsection{生産ラインの流れ\TBW}
\MMname において、ある\index{めいさい(モールド)@明細(モールド)}明細のモールドを加工をする際に、以下のような流れで作業が行われる。


%%%%%%%%%%%%%%%%%%%%%%%%%%%%%%%%%%%%%%%%%%%%%%%%%%%%%%%%%%
%% subsection 01.1.1 %%%%%%%%%%%%%%%%%%%%%%%%%%%%%%%%%%%%%
%%%%%%%%%%%%%%%%%%%%%%%%%%%%%%%%%%%%%%%%%%%%%%%%%%%%%%%%%%
\subsection{図面の確認}
\begin{enumerate}
\item 対象となる明細の\index{ずめん(モールド)@図面(モールド)}図面を用意する
\item 他に内容が類似する明細の図面があれば、それも併せて用意する
\end{enumerate}
%%%%%%%%%%%%%%%%%%%%%%%%%%%%%%%%%%%%%%%%%%%%%%%%%%%%%%%%%%
%% PARAMETER %%%%%%%%%%%%%%%%%%%%%%%%%%%%%%%%%%%%%%%%%%%%%
%%%%%%%%%%%%%%%%%%%%%%%%%%%%%%%%%%%%%%%%%%%%%%%%%%%%%%%%%%
\begin{Parameter}{必要なパラメータ}
\PMbox{図面の有無}%
\PMbox{図面番号}%
\end{Parameter}
%%%%%%%%%%%%%%%%%%%%%%%%%%%%%%%%%%%%%%%%%%%%%%%%%%%%%%%%%%
%%%%%%%%%%%%%%%%%%%%%%%%%%%%%%%%%%%%%%%%%%%%%%%%%%%%%%%%%%
%%%%%%%%%%%%%%%%%%%%%%%%%%%%%%%%%%%%%%%%%%%%%%%%%%%%%%%%%%


%%%%%%%%%%%%%%%%%%%%%%%%%%%%%%%%%%%%%%%%%%%%%%%%%%%%%%%%%%
%% subsection 01.1.2 %%%%%%%%%%%%%%%%%%%%%%%%%%%%%%%%%%%%%
%%%%%%%%%%%%%%%%%%%%%%%%%%%%%%%%%%%%%%%%%%%%%%%%%%%%%%%%%%
\subsection{加工部分の有無の確認}

%%%%%%%%%%%%%%%%%%%%%%%%%%%%%%%%%%%%%%%%%%%%%%%%%%%%%%%%%%
%% subsubsection 01.1.2.2 %%%%%%%%%%%%%%%%%%%%%%%%%%%%%%%%
%%%%%%%%%%%%%%%%%%%%%%%%%%%%%%%%%%%%%%%%%%%%%%%%%%%%%%%%%%
\subsubsection{端面部分}
\index{たんめんかこう@端面加工}端面の加工については、全明細に共通の形で存在する。

%%%%%%%%%%%%%%%%%%%%%%%%%%%%%%%%%%%%%%%%%%%%%%%%%%%%%%%%%%
%% subsubsection 01.1.2.2 %%%%%%%%%%%%%%%%%%%%%%%%%%%%%%%%
%%%%%%%%%%%%%%%%%%%%%%%%%%%%%%%%%%%%%%%%%%%%%%%%%%%%%%%%%%
\subsubsection{外削部分}
\index{がいさくかこう@外削加工}外削の加工については、明細により\index{がいさくのうむ@外削の有無}外削の有無または\index{がいさくのけいじょう@外削の形状}形状の違いが存在する。
\begin{enumerate}
\item トップ側またはボトム側の外削の有無を確認する
\item 外削の形状を確認し、使用する\index{こうぐ(がいさく)@工具(外削)}工具を決定する
\item \index{わんきょくにそったがいさく@湾曲に沿った外削}外削が湾曲に沿ったものかどうかも確認する
\end{enumerate}
%\clearpage
%%%%%%%%%%%%%%%%%%%%%%%%%%%%%%%%%%%%%%%%%%%%%%%%%%%%%%%%%%
%% PARAMETER %%%%%%%%%%%%%%%%%%%%%%%%%%%%%%%%%%%%%%%%%%%%%
%%%%%%%%%%%%%%%%%%%%%%%%%%%%%%%%%%%%%%%%%%%%%%%%%%%%%%%%%%
\begin{Parameter}{必要なパラメータ}
\PMbox{トップ外削の有無}%
\PMbox{ボトム外削の有無}%
\PMbox{トップ外削の形状}%
\PMbox{ボトム外削の形状}%
\end{Parameter}
%%%%%%%%%%%%%%%%%%%%%%%%%%%%%%%%%%%%%%%%%%%%%%%%%%%%%%%%%%
%%%%%%%%%%%%%%%%%%%%%%%%%%%%%%%%%%%%%%%%%%%%%%%%%%%%%%%%%%
%%%%%%%%%%%%%%%%%%%%%%%%%%%%%%%%%%%%%%%%%%%%%%%%%%%%%%%%%%

%\clearpage
%%%%%%%%%%%%%%%%%%%%%%%%%%%%%%%%%%%%%%%%%%%%%%%%%%%%%%%%%%
%% subsubsection 01.1.2.3 %%%%%%%%%%%%%%%%%%%%%%%%%%%%%%%%
%%%%%%%%%%%%%%%%%%%%%%%%%%%%%%%%%%%%%%%%%%%%%%%%%%%%%%%%%%
\subsubsection{溝部分}
\index{みぞかこう@溝加工}溝の加工については、全明細のトップ側に存在し、明細により形状の違いが存在する。
\begin{enumerate}
\item \index{みぞのけいじょう@溝の形状}溝の形状を確認し、使用する\index{サブプログラム(みぞ)@サブプログラム(溝)}サブプログラムの判断を行う
\item \index{みぞはば@溝幅}溝幅を確認し、使用する\index{こうぐ(みぞ)@工具(溝)}工具の判断を行う
\end{enumerate}
%%%%%%%%%%%%%%%%%%%%%%%%%%%%%%%%%%%%%%%%%%%%%%%%%%%%%%%%%%
%% PARAMETER %%%%%%%%%%%%%%%%%%%%%%%%%%%%%%%%%%%%%%%%%%%%%
%%%%%%%%%%%%%%%%%%%%%%%%%%%%%%%%%%%%%%%%%%%%%%%%%%%%%%%%%%
\begin{Parameter}{必要なパラメータ}
\PMbox{溝の形状}\PMbox{トップ外削の有無}\PMbox{溝幅}
\end{Parameter}
%%%%%%%%%%%%%%%%%%%%%%%%%%%%%%%%%%%%%%%%%%%%%%%%%%%%%%%%%%
%%%%%%%%%%%%%%%%%%%%%%%%%%%%%%%%%%%%%%%%%%%%%%%%%%%%%%%%%%
%%%%%%%%%%%%%%%%%%%%%%%%%%%%%%%%%%%%%%%%%%%%%%%%%%%%%%%%%%

%\clearpage
%%%%%%%%%%%%%%%%%%%%%%%%%%%%%%%%%%%%%%%%%%%%%%%%%%%%%%%%%%
%% subsubsection 01.1.2.4 %%%%%%%%%%%%%%%%%%%%%%%%%%%%%%%%
%%%%%%%%%%%%%%%%%%%%%%%%%%%%%%%%%%%%%%%%%%%%%%%%%%%%%%%%%%
\subsubsection{端面の面取部分}
\index{めんとりかこう(たんめん)@面取加工(端面)}端面の面取の加工については、全明細に存在し、明細により形状の違いが存在する。
\begin{enumerate}
\item 面取がC面取であれば、マシニングセンタによる加工を行うか判断を行う
\item \index{Cめんとり(たんめん)@C面取(端面)}C面取の角度を確認し、使用する\index{こうぐ(Cめんとり)@工具(C面取)}工具を決定する
\end{enumerate}
%%%%%%%%%%%%%%%%%%%%%%%%%%%%%%%%%%%%%%%%%%%%%%%%%%%%%%%%%%
%% PARAMETER %%%%%%%%%%%%%%%%%%%%%%%%%%%%%%%%%%%%%%%%%%%%%
%%%%%%%%%%%%%%%%%%%%%%%%%%%%%%%%%%%%%%%%%%%%%%%%%%%%%%%%%%
\begin{Parameter}{必要なパラメータ}
\PMbox{面取の形状}\PMbox{C面取長}\PMbox{C面取の角度}\PMbox{トップ外削の有無}
\end{Parameter}
%%%%%%%%%%%%%%%%%%%%%%%%%%%%%%%%%%%%%%%%%%%%%%%%%%%%%%%%%%
%%%%%%%%%%%%%%%%%%%%%%%%%%%%%%%%%%%%%%%%%%%%%%%%%%%%%%%%%%
%%%%%%%%%%%%%%%%%%%%%%%%%%%%%%%%%%%%%%%%%%%%%%%%%%%%%%%%%%

%\clearpage
%%%%%%%%%%%%%%%%%%%%%%%%%%%%%%%%%%%%%%%%%%%%%%%%%%%%%%%%%%
%% subsubsection 01.1.2.4 %%%%%%%%%%%%%%%%%%%%%%%%%%%%%%%%
%%%%%%%%%%%%%%%%%%%%%%%%%%%%%%%%%%%%%%%%%%%%%%%%%%%%%%%%%%
\subsubsection{端面の座ぐり部分\TBW}
(to be written...)
%%%%%%%%%%%%%%%%%%%%%%%%%%%%%%%%%%%%%%%%%%%%%%%%%%%%%%%%%%
%% PARAMETER %%%%%%%%%%%%%%%%%%%%%%%%%%%%%%%%%%%%%%%%%%%%%
%%%%%%%%%%%%%%%%%%%%%%%%%%%%%%%%%%%%%%%%%%%%%%%%%%%%%%%%%%
\begin{Parameter}{必要なパラメータ}
\PMbox{座ぐりの有無}
\end{Parameter}
%%%%%%%%%%%%%%%%%%%%%%%%%%%%%%%%%%%%%%%%%%%%%%%%%%%%%%%%%%
%%%%%%%%%%%%%%%%%%%%%%%%%%%%%%%%%%%%%%%%%%%%%%%%%%%%%%%%%%
%%%%%%%%%%%%%%%%%%%%%%%%%%%%%%%%%%%%%%%%%%%%%%%%%%%%%%%%%%


\clearpage
%%%%%%%%%%%%%%%%%%%%%%%%%%%%%%%%%%%%%%%%%%%%%%%%%%%%%%%%%%
%% subsection 01.1.3 %%%%%%%%%%%%%%%%%%%%%%%%%%%%%%%%%%%%%
%%%%%%%%%%%%%%%%%%%%%%%%%%%%%%%%%%%%%%%%%%%%%%%%%%%%%%%%%%
\subsection{加工部分の寸法の確認}

%%%%%%%%%%%%%%%%%%%%%%%%%%%%%%%%%%%%%%%%%%%%%%%%%%%%%%%%%%
%% subsubsection 01.1.3.1 %%%%%%%%%%%%%%%%%%%%%%%%%%%%%%%%
\subsubsection{端面における寸法}
\begin{enumerate}
\item \index{こうさ(ぜんちょう)@公差(全長)}全長の公差を確認し、\index{トップふりわけちょう@トップ振分長}トップ振分長および\index{ボトムふりわけちょう@ボトム振分長}ボトム振分長の\index{こうさ(ふりわけちょう)@公差(振分長)}公差の判断を行う
\item トップ側・ボトム側の\index{ふりわけちょう@振分長}振分長を確認し、\index{スペーサ}スペーサによる調整が必要か判断を行う
\item 使用するスペーサおよび\index{さいふりわけちょう@再振分長}再振分長は、専用の計算\index{プログラム(Excel VBA)}プログラム(\index{Excel VBA}Excel VBA)を用いて決定する
\item \index{がいけい@外径}外径・\index{にくあつ(たんめん)@肉厚(端面)}端面部の肉厚・\index{コーナーR(たんめん)@コーナーR(端面)}コーナーRの大きさを確認し、それに応じて\index{こうぐけいほせいち@工具径補正値}工具径補正値を決定する
\item 振分長に応じて、$Z$方向の\index{クリアランスへいめん(Zほうこう)@クリアランス平面($Z$方向)}クリアランス平面の位置を決定する
\end{enumerate}
%%%%%%%%%%%%%%%%%%%%%%%%%%%%%%%%%%%%%%%%%%%%%%%%%%%%%%%%%%
%% PARAMETER %%%%%%%%%%%%%%%%%%%%%%%%%%%%%%%%%%%%%%%%%%%%%
%%%%%%%%%%%%%%%%%%%%%%%%%%%%%%%%%%%%%%%%%%%%%%%%%%%%%%%%%%
\begin{Parameter}{必要なパラメータ}
\paragraph*{再振分長}
\PMbox{全長}\PMbox{トップ振分長}\PMbox{ボトム振分長}\PMbox{AC外径}\PMbox{ジグの長さ}\PMbox{受板の幅}
\tcbline*
\paragraph*{トップ端面}
\PMbox{トップ再振分長}\PMbox{AC外径}\PMbox{BD外径}\PMbox{外径コーナーR}\\
\PMbox{トップ端AC内径}\PMbox{トップ端BD内径}\\
\PMbox{トップ側$Z$方向クリアランス平面距離}
\tcbline*
\paragraph*{ボトム端面}
\PMbox{ボトム再振分長}\PMbox{AC外径}\PMbox{BD外径}\PMbox{外径コーナーR}\\
\PMbox{ボトム端AC内径}\PMbox{ボトム端BD内径}\\
\PMbox{ボトム側$Z$方向クリアランス平面距離}
\end{Parameter}
%%%%%%%%%%%%%%%%%%%%%%%%%%%%%%%%%%%%%%%%%%%%%%%%%%%%%%%%%%
%%%%%%%%%%%%%%%%%%%%%%%%%%%%%%%%%%%%%%%%%%%%%%%%%%%%%%%%%%
%%%%%%%%%%%%%%%%%%%%%%%%%%%%%%%%%%%%%%%%%%%%%%%%%%%%%%%%%%

%\clearpage
%%%%%%%%%%%%%%%%%%%%%%%%%%%%%%%%%%%%%%%%%%%%%%%%%%%%%%%%%%
%% subsubsection 01.1.3.2 %%%%%%%%%%%%%%%%%%%%%%%%%%%%%%%%
%%%%%%%%%%%%%%%%%%%%%%%%%%%%%%%%%%%%%%%%%%%%%%%%%%%%%%%%%%
\subsubsection{外削における寸法}
\begin{enumerate}
\item \index{がいさくちょう@外削長}外削長と\index{みぞ@溝}溝の位置を確認し、実際に加工する外削の長さの判断を行う
\item トップ側・ボトム側の両方に外削のある場合は、どちら側が\index{きじゅん(がいさくちゅうしん)@基準(外削中心)}基準であるのかを確認する
\item \index{ないけい(たんめん)@内径(端面)}端面の内径・\index{Aがわにくあつ(がいさく)@A側肉厚(外削)}外削部のA側肉厚・内面の\index{めっきまくあつ@めっき膜厚}めっき膜厚から、\index{がいさくちゅうしん@外削中心}外削中心$X$位置用のパラメータを手動による計算で決定する
\item \index{わんきょくにそったがいさく@湾曲に沿った外削}湾曲に沿った外削の場合は、\index{かたむきかく(がいさく)@傾き角(外削)}傾き角を手動による計算で決定する
\end{enumerate}
%%%%%%%%%%%%%%%%%%%%%%%%%%%%%%%%%%%%%%%%%%%%%%%%%%%%%%%%%%
%% PARAMETER %%%%%%%%%%%%%%%%%%%%%%%%%%%%%%%%%%%%%%%%%%%%%
%%%%%%%%%%%%%%%%%%%%%%%%%%%%%%%%%%%%%%%%%%%%%%%%%%%%%%%%%%
\begin{Parameter}{必要なパラメータ}
\paragraph*{ボトム側の外削のみの場合}
\PMbox{ボトム外削AC径}\PMbox{ボトム外削BD径}\PMbox{ボトム外削コーナーR}\\
\PMbox{ボトム外削長}\\
\PMbox{ボトム端AC内径}\PMbox{ボトム側A側肉厚}\PMbox{めっき膜厚}
\tcbline*
\paragraph*{トップ側の外削のみの場合}
\PMbox{トップ外削AC径}\PMbox{トップ外削BD径}\PMbox{トップ外削コーナーR}\\
\PMbox{トップ外削長}\PMbox{溝位置}\PMbox{溝幅}\\
\PMbox{トップ端AC内径}\PMbox{トップ外削A側肉厚}\PMbox{めっき膜厚}
\tcbline*
\paragraph*{両方に外削があり、ボトム側が基準の場合}
\PMbox{ボトム外削AC径}\PMbox{ボトム外削BD径}\PMbox{ボトム外削コーナーR}\\
\PMbox{ボトム外削長}\\
\PMbox{ボトム端AC内径}\PMbox{ボトム外削A側肉厚}\PMbox{めっき膜厚}\\
\PMbox{トップ外削AC径}\PMbox{トップ外削BD径}\PMbox{トップ外削コーナーR}\\
\PMbox{トップ外削長}\PMbox{溝位置}\PMbox{溝幅}\PMbox{通り芯}
\tcbline*
\paragraph*{両方に外削があり、トップ側が基準の場合}
\PMbox{トップ外削AC径}\PMbox{トップ外削BD径}\PMbox{トップ外削コーナーR}\\
\PMbox{トップ外削長}\PMbox{溝位置}\PMbox{溝幅}\\
\PMbox{トップ端AC内径}\PMbox{トップ外削A側肉厚}\PMbox{めっき膜厚}\\
\PMbox{ボトム外削AC径}\PMbox{ボトム外削BD径}\PMbox{ボトム外削コーナーR}\\
\PMbox{ボトム外削長}\PMbox{通り芯}
\tcbline*
\paragraph*{湾曲に沿った外削の場合}
(以上に加えて)\PMbox{中心湾曲}
\end{Parameter}
%%%%%%%%%%%%%%%%%%%%%%%%%%%%%%%%%%%%%%%%%%%%%%%%%%%%%%%%%%
%%%%%%%%%%%%%%%%%%%%%%%%%%%%%%%%%%%%%%%%%%%%%%%%%%%%%%%%%%
%%%%%%%%%%%%%%%%%%%%%%%%%%%%%%%%%%%%%%%%%%%%%%%%%%%%%%%%%%

%\clearpage
%%%%%%%%%%%%%%%%%%%%%%%%%%%%%%%%%%%%%%%%%%%%%%%%%%%%%%%%%%
%% subsubsection 01.1.3.3 %%%%%%%%%%%%%%%%%%%%%%%%%%%%%%%%
%%%%%%%%%%%%%%%%%%%%%%%%%%%%%%%%%%%%%%%%%%%%%%%%%%%%%%%%%%
\subsubsection{溝における寸法}
\begin{enumerate}
\item \index{みぞのけいじょう@溝の形状}溝の形状を確認し、必要に応じて加工における径の決定する
\item \index{きじゅん(みぞちゅうしん)@基準(溝中心)}溝中心の基準を確認し、\index{みぞちゅうしん@溝中心}溝中心の$X$位置を手動で計算し、決定する
\item \index{みぞはば@溝幅}溝幅を確認し、\index{かこうかいすう(みぞはば)@加工回数(溝幅)}加工の回数を決定する
\end{enumerate}
%%%%%%%%%%%%%%%%%%%%%%%%%%%%%%%%%%%%%%%%%%%%%%%%%%%%%%%%%%
%% PARAMETER %%%%%%%%%%%%%%%%%%%%%%%%%%%%%%%%%%%%%%%%%%%%%
%%%%%%%%%%%%%%%%%%%%%%%%%%%%%%%%%%%%%%%%%%%%%%%%%%%%%%%%%%
\begin{Parameter}{必要なパラメータ}
\paragraph*{湾曲中心が基準の場合}
\PMbox{溝AC径}\PMbox{溝BD径}\PMbox{溝位置}\PMbox{溝幅}\PMbox{中心湾曲}\\
\PMbox{溝コーナーR}または\PMbox{溝コーナーC}
\tcbline*
\paragraph*{外削中心が基準の場合}
\PMbox{溝AC径}\PMbox{溝BD径}\PMbox{溝位置}\PMbox{溝幅}\\
\PMbox{溝コーナーR}または\PMbox{溝コーナーC}
\tcbline*
\paragraph*{A側溝深さが基準の場合}
(以上に加えて)\PMbox{A側溝深さ}
\end{Parameter}
%%%%%%%%%%%%%%%%%%%%%%%%%%%%%%%%%%%%%%%%%%%%%%%%%%%%%%%%%%
%%%%%%%%%%%%%%%%%%%%%%%%%%%%%%%%%%%%%%%%%%%%%%%%%%%%%%%%%%
%%%%%%%%%%%%%%%%%%%%%%%%%%%%%%%%%%%%%%%%%%%%%%%%%%%%%%%%%%

\clearpage
%%%%%%%%%%%%%%%%%%%%%%%%%%%%%%%%%%%%%%%%%%%%%%%%%%%%%%%%%%
%% subsubsection 01.1.3.4 %%%%%%%%%%%%%%%%%%%%%%%%%%%%%%%%
%%%%%%%%%%%%%%%%%%%%%%%%%%%%%%%%%%%%%%%%%%%%%%%%%%%%%%%%%%
\subsubsection{端面の面取における寸法}
\begin{enumerate}
\item \index{そとがわCめんとり(たんめん)@外側C面取}端面の外側C面取の場合は、\index{がいさくのうむ@外削の有無}外削の有無を確認し、加工の径を決定する
\item \index{Cめんとり(がいさくせんたん)@C面取(外削先端)}外削先端部のC面取の場合は、\index{がいさくのけいじょう@外削の形状}外削の形状を確認し、\index{こうぐ(がいさく)@工具(外削)}工具を決定する
\end{enumerate}
%%%%%%%%%%%%%%%%%%%%%%%%%%%%%%%%%%%%%%%%%%%%%%%%%%%%%%%%%%
%% PARAMETER %%%%%%%%%%%%%%%%%%%%%%%%%%%%%%%%%%%%%%%%%%%%%
%%%%%%%%%%%%%%%%%%%%%%%%%%%%%%%%%%%%%%%%%%%%%%%%%%%%%%%%%%
\begin{Parameter}{必要なパラメータ}
\paragraph*{トップ端外側C面取:外削のない場合}
\PMbox{AC外径}\PMbox{BD外径}\PMbox{トップ端外側C面取長}\PMbox{外径コーナーR}
\tcbline*
\paragraph*{トップ端外側C面取:外削のある場合}
\PMbox{トップ外削AC径}\PMbox{トップ外削BD径}\PMbox{トップ端外削コーナーR}\\
\PMbox{トップ端外側C面取長}
\tcbline*
\paragraph*{ボトム端外側C面取:外削のない場合}
\PMbox{AC外径}\PMbox{BD外径}\PMbox{ボトム端外側C面取長}\PMbox{外径コーナーR}
\tcbline*
\paragraph*{ボトム端外側C面取:外削のある場合}
\PMbox{ボトム外削AC径}\PMbox{ボトム外削BD径}\PMbox{ボトム端外削コーナーR}\\
\PMbox{ボトム端外側C面取長}
\tcbline*
\paragraph*{トップ端内側C面取}
\PMbox{トップ端AC内径}\PMbox{トップ端BD内径}\PMbox{トップ端内径コーナーR}\\
\PMbox{トップ端内側C面取長}\PMbox{めっき膜厚}
\tcbline*
\paragraph*{ボトム端内側C面取}
\PMbox{ボトム端AC内径}\PMbox{ボトム端BD内径}\PMbox{ボトム端内径コーナーR}\\
\PMbox{ボトム端内側C面取長}\PMbox{めっき膜厚}
\end{Parameter}
%%%%%%%%%%%%%%%%%%%%%%%%%%%%%%%%%%%%%%%%%%%%%%%%%%%%%%%%%%
%%%%%%%%%%%%%%%%%%%%%%%%%%%%%%%%%%%%%%%%%%%%%%%%%%%%%%%%%%
%%%%%%%%%%%%%%%%%%%%%%%%%%%%%%%%%%%%%%%%%%%%%%%%%%%%%%%%%%

%\clearpage
%%%%%%%%%%%%%%%%%%%%%%%%%%%%%%%%%%%%%%%%%%%%%%%%%%%%%%%%%%
%% subsubsection 01.1.3.4 %%%%%%%%%%%%%%%%%%%%%%%%%%%%%%%%
%%%%%%%%%%%%%%%%%%%%%%%%%%%%%%%%%%%%%%%%%%%%%%%%%%%%%%%%%%
\subsubsection{座ぐりにおける寸法\TBW}
(to be written...)
%%%%%%%%%%%%%%%%%%%%%%%%%%%%%%%%%%%%%%%%%%%%%%%%%%%%%%%%%%
%% PARAMETER %%%%%%%%%%%%%%%%%%%%%%%%%%%%%%%%%%%%%%%%%%%%%
%%%%%%%%%%%%%%%%%%%%%%%%%%%%%%%%%%%%%%%%%%%%%%%%%%%%%%%%%%
\begin{Parameter}{必要なパラメータ}
\PMbox{座ぐりの位置}\PMbox{座ぐりの長さ}\PMbox{座ぐりコーナーR}\PMbox{座ぐり深さ}\PMbox{トップAC外径}
\end{Parameter}
%%%%%%%%%%%%%%%%%%%%%%%%%%%%%%%%%%%%%%%%%%%%%%%%%%%%%%%%%%
%%%%%%%%%%%%%%%%%%%%%%%%%%%%%%%%%%%%%%%%%%%%%%%%%%%%%%%%%%
%%%%%%%%%%%%%%%%%%%%%%%%%%%%%%%%%%%%%%%%%%%%%%%%%%%%%%%%%%


\clearpage
%%%%%%%%%%%%%%%%%%%%%%%%%%%%%%%%%%%%%%%%%%%%%%%%%%%%%%%%%%
%% subsection 01.1.4 %%%%%%%%%%%%%%%%%%%%%%%%%%%%%%%%%%%%%
%%%%%%%%%%%%%%%%%%%%%%%%%%%%%%%%%%%%%%%%%%%%%%%%%%%%%%%%%%
\subsection{プログラムの入力}
\begin{enumerate}
\item 原則として、\index{プログラムばんごう@プログラム番号}プログラム番号は\index{せいひんばんごう@製品番号}製品番号と一致させる\\
ただし、加工内容が同一のものである場合は、既存のプログラムをそのまま流用する
\item 各々の加工部分およびその形状に対する\index{サブプログラム}サブプログラムを決定する
\item 各々のサブプログラムに対し、適切な寸法値を手動で計算する
\item 各々のサブプログラムに対し、計算した寸法値・\index{こうぐばんごう@工具番号}工具番号・\index{おくりはやさ@送り速さ}送り速さ・\index{しゅじくかいてんすう@主軸回転数}主軸回転数を格納する
\item \index{さいふりわけちょう@再振分長}再振分長の寸法に応じて、\index{クリアランスへいめん(Zほうこう)@クリアランス平面($Z$方向)}$Z$方向クリアランス平面の位置を決定する
\item マシニングセンタの操作画面にて\index{メインプログラム}メインプログラムを直接編集し、必要なコードまたは数値を記入する
\item 必要に応じて、\index{いちじていし(プログラム)@一時停止(プログラム)}一時停止用のコードを代入する
\item \index{こうぐけいほせい@工具径補正}工具径または\index{こうぐちょうほせい@工具長補正}工具長の補正が必要な場合は、別途専用画面にて手動で編集を行う
\end{enumerate}


%\clearpage
%%%%%%%%%%%%%%%%%%%%%%%%%%%%%%%%%%%%%%%%%%%%%%%%%%%%%%%%%%
%% subsection 01.1.4 %%%%%%%%%%%%%%%%%%%%%%%%%%%%%%%%%%%%%
%%%%%%%%%%%%%%%%%%%%%%%%%%%%%%%%%%%%%%%%%%%%%%%%%%%%%%%%%%
\subsection{ワークの設置}
\begin{enumerate}
\item \index{スペーサ}スペーサが必要な場合は、適切なスペーサを\index{ジグ}ジグの\index{うけいた@受板}受板に設置する
\item \index{ワーク}ワークの大きさを考慮して、\index{ワークこていようボルト@ワーク固定用ボルト}ワーク固定用ボルトの長さを目分量で適宜決定し、ジグに設置する
\item \index{さいふりわけちょう@再振分長}再振分長に応じた位置に\index{ワーク}ワークを設置し、固定する
\item トップ側およびボトム側の、ジグからの\index{はりだしちょう@張出長}張出長を\index{メジャー}メジャーを用いて測定する
\item 測定した張出長から、\index{たんめんかこう@端面加工}端面加工における\index{ぜんけずりしろ(たんめん)@全削り代(端面)}全削り代を手動でおおまかに計算する
\end{enumerate}
%%%%%%%%%%%%%%%%%%%%%%%%%%%%%%%%%%%%%%%%%%%%%%%%%%%%%%%%%%
%% PARAMETER %%%%%%%%%%%%%%%%%%%%%%%%%%%%%%%%%%%%%%%%%%%%%
%%%%%%%%%%%%%%%%%%%%%%%%%%%%%%%%%%%%%%%%%%%%%%%%%%%%%%%%%%
\begin{Parameter}{必要なパラメータ}
\paragraph*{ワーク固定用ボルト}
\PMbox{AC外径}\PMbox{BD外径}\\
\PMbox{ジグ床面とボルト取付具(上)間の距離}\PMbox{受板とボルト取付具(横)間の距離}
\tcbline*
\paragraph*{端面の削り代}
\PMbox{ジグの長さ}\PMbox{トップ再振分長}\PMbox{ボトム再振分長}\PMbox{端面加工1回あたりの削り代}\\
\PMbox{トップ側張出長実測値}\PMbox{ボトム側張出長実測値}
\end{Parameter}
%%%%%%%%%%%%%%%%%%%%%%%%%%%%%%%%%%%%%%%%%%%%%%%%%%%%%%%%%%
%%%%%%%%%%%%%%%%%%%%%%%%%%%%%%%%%%%%%%%%%%%%%%%%%%%%%%%%%%
%%%%%%%%%%%%%%%%%%%%%%%%%%%%%%%%%%%%%%%%%%%%%%%%%%%%%%%%%%


\clearpage
%%%%%%%%%%%%%%%%%%%%%%%%%%%%%%%%%%%%%%%%%%%%%%%%%%%%%%%%%%
%% subsection 01.1.5 %%%%%%%%%%%%%%%%%%%%%%%%%%%%%%%%%%%%%
%%%%%%%%%%%%%%%%%%%%%%%%%%%%%%%%%%%%%%%%%%%%%%%%%%%%%%%%%%
\subsection{ワーク設置後の調整}
\begin{enumerate}
\item トップ側およびボトム側の\index{ぜんけずりしろ(たんめん)@全削り代(端面)}全削り代に応じて、\index{かこうかいすう(たんめんかこう)@加工回数(端面加工)}端面加工の回数を設定する
\item トップ端およびボトム端の\index{がいけいちゅうしん@外径中心}外径中心の位置を\index{メジャー}メジャーで測定する
\item 測定した中心位置を用いて、\index{ワークざひょうけいげんてん@ワーク座標系原点}ワーク座標系原点の設定を行う
\item \expandafterindex{テーブルのかいてんちゅうしん(\MMname)@テーブルの回転中心(\MMname)}テーブルの回転中心とのずれを考慮して、端面の$Z$方向の長さを調整する
\item \index{とおりしん@通り芯}通り芯がある場合\expandafterindex{テーブルのかいてんちゅうしん(\MMname)@テーブルの回転中心(\MMname)}テーブルの回転中心とのずれを考慮して、\index{がいさくけいのちゅうしん@外削径の中心}の$X$方向の長さを調整する
\end{enumerate}
これらの設定は、マシニングセンタの操作画面から\index{メインプログラム}メインプログラムを直接手動で編集する形で行われる。



\clearpage
%%%%%%%%%%%%%%%%%%%%%%%%%%%%%%%%%%%%%%%%%%%%%%%%%%%%%%%%%%
%% section 1.2 %%%%%%%%%%%%%%%%%%%%%%%%%%%%%%%%%%%%%%%%%%%
%%%%%%%%%%%%%%%%%%%%%%%%%%%%%%%%%%%%%%%%%%%%%%%%%%%%%%%%%%
\modHeadsection{\MMname における工程(加工中)}


%%%%%%%%%%%%%%%%%%%%%%%%%%%%%%%%%%%%%%%%%%%%%%%%%%%%%%%%%%
%% subsection 01.2.1 %%%%%%%%%%%%%%%%%%%%%%%%%%%%%%%%%%%%%
%%%%%%%%%%%%%%%%%%%%%%%%%%%%%%%%%%%%%%%%%%%%%%%%%%%%%%%%%%
\subsection{芯出し測定後}
\begin{enumerate}
\item 各々の\index{ワークざひょうけいげんてん@ワーク座標系原点}ワーク座標系原点の測定後、必要に応じてワーク座標系原点の設定変更を行う
\item 各々の測定箇所の$Z$位置の変更を、必要に応じて行う
\end{enumerate}
これらの設定は、\index{マシニングセンタ}マシニングセンタの操作画面から\index{メインプログラム}メインプログラムを直接手動で編集する形で行われる。


%%%%%%%%%%%%%%%%%%%%%%%%%%%%%%%%%%%%%%%%%%%%%%%%%%%%%%%%%%
%% subsection 01.2.1 %%%%%%%%%%%%%%%%%%%%%%%%%%%%%%%%%%%%%
%%%%%%%%%%%%%%%%%%%%%%%%%%%%%%%%%%%%%%%%%%%%%%%%%%%%%%%%%%
\subsection{端面加工中}
\begin{enumerate}
\item 必要に応じて、\index{1かいあたりのけずりしろ(たんめん)@1回あたりの削り代(端面)}1回あたりの削り代を調整する
\end{enumerate}


%%%%%%%%%%%%%%%%%%%%%%%%%%%%%%%%%%%%%%%%%%%%%%%%%%%%%%%%%%
%% subsection 01.2.1 %%%%%%%%%%%%%%%%%%%%%%%%%%%%%%%%%%%%%
%%%%%%%%%%%%%%%%%%%%%%%%%%%%%%%%%%%%%%%%%%%%%%%%%%%%%%%%%%
\subsection{外削加工中}
\begin{enumerate}
\item 必要に応じて\index{しあげかこう(がいさく)@仕上げ加工(外削)}仕上加工前に\index{いちじていし(プログラム)@一時停止(プログラム)}一時停止を行い、\index{Aがわにくあつ@A側肉厚}A側肉厚および\index{がいさくけい@外削径}外削径の測定を行う
\item A側肉厚を調整する場合は、該当する\index{しんだしそくてい(がいさくちゅうしん@芯出し測定(外削中心)}芯出し測定部分のパラメータをメインプログラムから直接手動で編集する
\item \index{がいさくけい@外削径}外削径を調整する場合は、該当する加工部分のパラメータをマシニングセンタの操作画面から\index{メインプログラム}メインプログラムを直接手動で編集する
\item \index{かこうかいすう(がいさく)@加工回数(外削)}加工の回数を変更する場合は、該当する加工部分をマシニングセンタの操作画面からメインプログラムを直接手動で編集する
\end{enumerate}


%%%%%%%%%%%%%%%%%%%%%%%%%%%%%%%%%%%%%%%%%%%%%%%%%%%%%%%%%%
%% subsection 01.2.1 %%%%%%%%%%%%%%%%%%%%%%%%%%%%%%%%%%%%%
%%%%%%%%%%%%%%%%%%%%%%%%%%%%%%%%%%%%%%%%%%%%%%%%%%%%%%%%%%
\subsection{溝加工中}
\begin{enumerate}
\item 必要に応じて\index{しあげかこう(みぞ)@仕上げ加工(溝)}仕上加工前に\index{いちじていし(プログラム)@一時停止(プログラム)}一時停止を行い、\index{Aがわみぞふかさ@A側溝深さ}A側溝深さおよび\index{みぞけい@溝径}溝径の測定を行う
\item A側溝深さを調整する場合は、該当する\index{しんだしそくてい(みぞちゅうしん)@芯出し測定(溝)}芯出し測定部分のパラメータをマシニングセンタの操作画面からメインプログラムを直接手動で編集する
\item 溝径を調整する場合は、該当する加工部分のパラメータをマシニングセンタの操作画面からメインプログラムを直接手動で編集する
\item \index{かこうかいすう(みぞ)@加工回数(溝)}加工の回数を変更する場合は、該当する加工部分をマシニングセンタの操作画面からメインプログラムを直接手動で編集する
\item 必要に応じて、\index{ブロックゲージ}ブロックゲージによる\index{みぞはば@溝幅}溝幅の測定を行う
\end{enumerate}


%%%%%%%%%%%%%%%%%%%%%%%%%%%%%%%%%%%%%%%%%%%%%%%%%%%%%%%%%%
%% subsection 01.2.1 %%%%%%%%%%%%%%%%%%%%%%%%%%%%%%%%%%%%%
%%%%%%%%%%%%%%%%%%%%%%%%%%%%%%%%%%%%%%%%%%%%%%%%%%%%%%%%%%
\subsection{端面の外側C面取加工中}
\begin{enumerate}
\item 必要に応じて\index{しあげかこう(そとがわCめんとり)@仕上げ加工(外側C面取)}仕上加工前に\index{いちじていし(プログラム)@一時停止(プログラム)}一時停止を行い、\index{Cめんとり@C面取}C面取の測定・位置の確認を行う
\item C面取の位置を調整する場合は、該当する加工部分のパラメータをマシニングセンタの操作画面からメインプログラムを直接手動で編集する
\item \index{かこうかいすう(たんめんそとがわCめんとり)@加工回数(端面外側C面取)}加工の回数を変更する場合は、該当する加工部分をマシニングセンタの操作画面からメインプログラムを直接手動で編集する
\end{enumerate}


\clearpage
%%%%%%%%%%%%%%%%%%%%%%%%%%%%%%%%%%%%%%%%%%%%%%%%%%%%%%%%%%
%% subsection 01.2.1 %%%%%%%%%%%%%%%%%%%%%%%%%%%%%%%%%%%%%
%%%%%%%%%%%%%%%%%%%%%%%%%%%%%%%%%%%%%%%%%%%%%%%%%%%%%%%%%%
\subsection{端面の内側C面取加工中}
\begin{enumerate}
\item 必要に応じて\index{しあげかこう(うちがわCめんとり)@仕上げ加工(内側C面取)}仕上加工前に\index{いちじていし@一時停止}一時停止を行い、C面取の測定・位置の確認を行う
\item C面取の位置を調整する場合は、該当する加工部分のパラメータをマシニングセンタの操作画面からメインプログラムを直接手動で編集する
\item \index{かこうかいすう(たんめんうちがわCめんとり)@加工回数(端面内側C面取)}加工の回数を変更する場合は、該当する加工部分をマシニングセンタの操作画面からメインプログラムを直接手動で編集する
\end{enumerate}


%\clearpage
%%%%%%%%%%%%%%%%%%%%%%%%%%%%%%%%%%%%%%%%%%%%%%%%%%%%%%%%%%
%% subsection 01.2.1 %%%%%%%%%%%%%%%%%%%%%%%%%%%%%%%%%%%%%
%%%%%%%%%%%%%%%%%%%%%%%%%%%%%%%%%%%%%%%%%%%%%%%%%%%%%%%%%%
\subsection{座ぐり加工中\TBW}
(to be written...)



\clearpage
%%%%%%%%%%%%%%%%%%%%%%%%%%%%%%%%%%%%%%%%%%%%%%%%%%%%%%%%%%
%% section 01.3 %%%%%%%%%%%%%%%%%%%%%%%%%%%%%%%%%%%%%%%%%%
%%%%%%%%%%%%%%%%%%%%%%%%%%%%%%%%%%%%%%%%%%%%%%%%%%%%%%%%%%
\modHeadsection{\MMname における工程(加工後)}


%%%%%%%%%%%%%%%%%%%%%%%%%%%%%%%%%%%%%%%%%%%%%%%%%%%%%%%%%%
%% subsection 01.3.1 %%%%%%%%%%%%%%%%%%%%%%%%%%%%%%%%%%%%%
%%%%%%%%%%%%%%%%%%%%%%%%%%%%%%%%%%%%%%%%%%%%%%%%%%%%%%%%%%
\subsection{ワークの取外し}
\begin{enumerate}
\item 必要に応じて、\index{ワークこていようボルト@ワーク固定用ボルト}ワーク固定用ボルトを緩める前に、各種\index{そくていき@測定器}測定器で\index{すんぽう@寸法}寸法を確認する
\item クーラント用の液およびエアーブローを用いて軽く洗浄を行い、固定用ボルトを緩めて\index{ワーク}ワークを取り出し、軽く拭取りを行う
\item \index{リフター}リフターまたは\index{クレーン}クレーンを用いて、\index{めんとりようさぎょうだい@面取用作業台}面取用作業台にワークを移動する
\end{enumerate}


%%%%%%%%%%%%%%%%%%%%%%%%%%%%%%%%%%%%%%%%%%%%%%%%%%%%%%%%%%
%% subsection 01.3.2 %%%%%%%%%%%%%%%%%%%%%%%%%%%%%%%%%%%%%
%%%%%%%%%%%%%%%%%%%%%%%%%%%%%%%%%%%%%%%%%%%%%%%%%%%%%%%%%%
\subsection{外観の確認・寸法の検査}
\begin{enumerate}
\item \index{がいかん(ワーク)@外観(ワーク)}外観に異常がないか確認を行う
\item \index{そくていき@測定器}測定器を用いて\index{すんぽう@寸法}寸法の確認を行う
\item 所定の用紙に、指定箇所の\index{こうさ@公差}公差を考慮した寸法値を、手動で計算を行い手動で記入する
\item 必要に応じて、所定の用紙に測定値の記入を行う
\end{enumerate}


%%%%%%%%%%%%%%%%%%%%%%%%%%%%%%%%%%%%%%%%%%%%%%%%%%%%%%%%%%
%% subsection 01.3.3 %%%%%%%%%%%%%%%%%%%%%%%%%%%%%%%%%%%%%
%%%%%%%%%%%%%%%%%%%%%%%%%%%%%%%%%%%%%%%%%%%%%%%%%%%%%%%%%%
\subsection{手動による仕上げ加工}
\begin{enumerate}
\item 所定の寸法の\index{めんとり(たんめん)@面取(端面)}端面の面取を、\index{てもちけんまき@手持ち研磨機}手持ち研磨機を用いて手動で行う
\item \index{ばり}ばり等を除去を、\index{やすり}やすりを用いて全体的に手動で行う
\item \index{にくあつ(たんめん)@肉厚(端面)}端面の肉厚に応じて\index{こくいん@刻印}刻印の大きさを決定する
\item 明細のによる指定に応じて、刻印の位置を調整する
\item リフターまたはクレーンを用いて、所定の置き場に移動する
\end{enumerate}


%!TEX root = ../RfCPN.tex


\modHeadchapter{イシュー・問題の特定}
先に述べた\expandafterindex{ぎょうむフロー(\yomiMMC)@業務フロー(\nameMMC)}業務フローを通して、\MMC に関する\index{イシュー}イシュー(issue)および\index{もんだい(problem)@問題(problem)}問題(problem)の特定を試みる。
なお{\color{red}赤色}(\,\sarrow[red]\!)の項目は、ソフトウェア側(\index{NCプログラム}NCプログラムの内容)で対処・改善できると考えられるものを示す
%% footnote %%%%%%%%%%%%%%%%%%%%%
\footnote{つまり、システムの管理・監督者の能力および責任によるところが大きなものである。}。
%%%%%%%%%%%%%%%%%%%%%%%%%%%%%%%%%



%%%%%%%%%%%%%%%%%%%%%%%%%%%%%%%%%%%%%%%%%%%%%%%%%%%%%%%%%%
%% section 02.01 %%%%%%%%%%%%%%%%%%%%%%%%%%%%%%%%%%%%%%%%%
%%%%%%%%%%%%%%%%%%%%%%%%%%%%%%%%%%%%%%%%%%%%%%%%%%%%%%%%%%
\modHeadsection{安全(safety)に関するイシュー・問題}

\begin{Issues}{オペレータの精神的高ストレス下での作業}
通常、\index{NCプログラム}NCプログラムの編集はシステム管理者・設計者・\index{プログラマ}プログラマ等が行う高度な業務である
\begin{enumerate}[label=\sarrow]
\item[{\sarrow[red]}]作業者に対してNCメインプログラムの直接編集を強いている状態になっている
\item[{\sarrow[red]}]プログラマが存在しない
\end{enumerate}
\end{Issues}

\begin{Issues}{\index{オペレータ}オペレータの機内の侵入に伴うリスク}
一般に、\index{きないへのしんにゅう@機内への侵入}機内への侵入は、転倒・巻き込まれ等のリスクを伴う
\begin{enumerate}[label=\sarrow]
\item[{\sarrow[red]}]機内に侵入し直接測定をしないと、ワークの\index{かこうげんてん(がいさんち)@加工原点(概算値)}加工原点の概算値が見出だせない状態にある
\item[{\sarrow[red]}]機内に侵入しないと、\index{はのこうかん(フェイスミル)@刃の交換(フェイスミル)}フェイスミルの刃の交換ができない状態にある
\item 機内に侵入しないと、内部の掃除ができない状態にある
\end{enumerate}
\end{Issues}

\begin{Issues}{\index{てもちけんまき@手持ち研磨機}手持ち研磨機の使用に伴うリスク}
一般に、\index{てもちけんまき@手持ち研磨機}手持ち研磨機による加工は、巻き込まれや粉塵の付着・吸引等のリスクを伴う
\begin{enumerate}[label=\sarrow]
\item[{\sarrow[red]}]寸法の小さな\EndFaceChamferMilling が\MMC で行われず、\index{てもちけんまき@手持ち研磨機}手持ち研磨機により手作業で行われている状態にある
\item[{\sarrow[red]}]\EndFaceChamferMilling に関する解析的な幾何情報を導出しないまま放置され続けている
\end{enumerate}
\end{Issues}

\begin{Issues}{柵への衝突に伴うリスク}
\begin{enumerate}[label=\sarrow]
\item 後付けされた\expandafterindex{あんぜんさく(\yomiMMC)@安全柵(\nameMMC)}安全柵(と称されている柵)が、衝突の\index{リスク(しょうとつ)@リスク(衝突)}リスクを生み出している
\item 一部のみ(出入口のみ)を着目し、周囲(柵の周辺)が軽視され、\index{あんぜんたいさく@安全対策}安全対策が機能せず反対に危険リスクを生み出している
\end{enumerate}
\end{Issues}

\clearpage
\begin{Issues}{\index{リフター}リフターへの衝突に伴うリスク}
\begin{enumerate}[label=\sarrow]
\item ワークの\index{うけいれけんさ(ワーク)@受入検査(ワーク)}受入検査の際に\index{リフター}リフターや\index{クレーン}クレーンが必ず\index{あんぜんつうろ@安全通路}安全通路を通る構造にある
\item 安全性より生産性を優先する\index{けいえいほうしん(とうしゃ)@経営方針(当社)}経営方針となっている
\end{enumerate}
\end{Issues}


\clearpage
%%%%%%%%%%%%%%%%%%%%%%%%%%%%%%%%%%%%%%%%%%%%%%%%%%%%%%%%%%
%% section 02.02 %%%%%%%%%%%%%%%%%%%%%%%%%%%%%%%%%%%%%%%%%
%%%%%%%%%%%%%%%%%%%%%%%%%%%%%%%%%%%%%%%%%%%%%%%%%%%%%%%%%%
\modHeadsection{\index{ひんしつ@品質}品質に関するイシュー・問題}


%%%%%%%%%%%%%%%%%%%%%%%%%%%%%%%%%%%%%%%%%%%%%%%%%%%%%%%%%%
%% subsection 02.02.01 %%%%%%%%%%%%%%%%%%%%%%%%%%%%%%%%%%%
%%%%%%%%%%%%%%%%%%%%%%%%%%%%%%%%%%%%%%%%%%%%%%%%%%%%%%%%%%
\subsection{測定における品質}

\begin{Issues}{\KeywayCenterMeasurement(AC方向)の不安定性}
\begin{enumerate}[label=\sarrow]
\item[{\sarrow[red]}]\KeywayCenter(AC方向)の測定が、手計算による関節的な方法で導出されている
\item[{\sarrow[red]}]自動化が十分可能にもかかわらず、事態が放置され続けている
\item[{\sarrow[red]}]\CenterCurvature に伴う誤差を含み、位置(特に\AsideKeywayDepth)が安定しない
\end{enumerate}
\end{Issues}

\begin{Issues}{\KeywayCenterMeasurement(BD方向)の不安定性}
\begin{enumerate}[label=\sarrow]
\item[{\sarrow[red]}]\KeywayCenter(BD方向)が、端面におけるそれとして与えられている
\item[{\sarrow[red]}]BD方向の真直度に伴う誤差を含み、位置が安定しない
\end{enumerate}
\end{Issues}

\begin{Issues}{\CenterlineEndFaceDifMeasurement の不安定性}
\begin{enumerate}[label=\sarrow]
\item[{\sarrow[red]}]\CenterlineEndFaceDifMeasurement が、手動による\index{ハンドルそうさ@ハンドル操作}ハンドル操作で行われている
\item[{\sarrow[red]}]手作業によるため、測定位置や送り速さが安定しない
\item[{\sarrow[red]}]自動化が十分可能にもかかわらず、事態が放置され続けている
\end{enumerate}
\end{Issues}

\begin{Issues}{\TLMeasurement の故障}
\begin{enumerate}[label=\sarrow]
\item\TLMeasurement 用装置が、物理的に壊れている
\item 現物合わせで\TLMeasurement を行っている
\end{enumerate}
\end{Issues}

\begin{Issues}{\TopOutcutCenter と\BottomOutcutCenter との差の未検出}
\begin{enumerate}[label=\sarrow]
\item[{\sarrow[red]}]測定した\TopOutcutCenter と\BottomOutcutCenter に大きな差があっても検出できない
\item[{\sarrow[red]}]\index{へんにく@偏肉}偏肉等による\OutcutCenter の偏りが検出ができない
\end{enumerate}
\end{Issues}


%%%%%%%%%%%%%%%%%%%%%%%%%%%%%%%%%%%%%%%%%%%%%%%%%%%%%%%%%%
%% subsection 02.02.02 %%%%%%%%%%%%%%%%%%%%%%%%%%%%%%%%%%%
%%%%%%%%%%%%%%%%%%%%%%%%%%%%%%%%%%%%%%%%%%%%%%%%%%%%%%%%%%
\subsection{\OutcutMilling における品質}

\begin{Issues}{\CurvedOutcutMilling の不安定性}
\begin{enumerate}[label=\sarrow]
\item[{\sarrow[red]}]手作業による\index{ハンドルそうさ@ハンドル操作}ハンドル操作により測定が行われている
\item[{\sarrow[red]}]\index{NCプログラム}NCプログラム内の\OutcutLength の寸法に誤りが存在している
\item[{\sarrow[red]}]自動化が十分可能にもかかわらず、事態が放置され続けている
\end{enumerate}
\end{Issues}


\clearpage
%%%%%%%%%%%%%%%%%%%%%%%%%%%%%%%%%%%%%%%%%%%%%%%%%%%%%%%%%%
%% subsection 02.02.03 %%%%%%%%%%%%%%%%%%%%%%%%%%%%%%%%%%%
%%%%%%%%%%%%%%%%%%%%%%%%%%%%%%%%%%%%%%%%%%%%%%%%%%%%%%%%%%
\subsection{\KeywayMilling における品質}

\begin{Issues}{\KeywayMilling のかえりの\index{てさぎょう@手作業}手作業による除去}
\begin{enumerate}[label=\sarrow]
\item[{\sarrow[red]}]長方形・8角形(C面取)の\KeywayMilling の一部の頂点にかえりが残る
\item[{\sarrow[red]}]手作業でかえりを除去するため、頂点部が歪な形状になる
\end{enumerate}
\end{Issues}


%\clearpage
%%%%%%%%%%%%%%%%%%%%%%%%%%%%%%%%%%%%%%%%%%%%%%%%%%%%%%%%%%
%% subsection 02.02.04 %%%%%%%%%%%%%%%%%%%%%%%%%%%%%%%%%%%
%%%%%%%%%%%%%%%%%%%%%%%%%%%%%%%%%%%%%%%%%%%%%%%%%%%%%%%%%%
\subsection{\EndFaceChamferMilling における品質}

\begin{Issues}{\index{てさぎょう@手作業}手作業による\EndFaceChamferMilling}
\begin{enumerate}[label=\sarrow]
\item[{\sarrow[red]}]
寸法の小さな\EndFaceChamferMilling が\index{てもちけんまき@手持ち研磨機}手持ち研磨機により\index{てさぎょう@手作業}手作業で行われている
\item[{\sarrow[red]}]大半の場合は自動化が十分可能にもかかわらず、事態が放置され続けている
\end{enumerate}
\end{Issues}

%%%%%%%%%%%%%%%%%%%%%%%%%%%%%%%%%%%%%%%%%%%%%%%%%%%%%%%%%%
%% subsection 02.02.04.1 %%%%%%%%%%%%%%%%%%%%%%%%%%%%%%%%%
%%%%%%%%%%%%%%%%%%%%%%%%%%%%%%%%%%%%%%%%%%%%%%%%%%%%%%%%%%
\subsubsection{\EndFaceOutCChamferMilling における品質}

\begin{Issues}{\EndFaceOutCChamferMilling の位置調整}
\begin{enumerate}[label=\sarrow]
\item[{\sarrow[red]}]\EndFaceOutCChamferMilling のAC方向の位置調整が手作業により行われている
\item[{\sarrow[red]}]目分量により位置調整がなされているため、位置が安定しない
\item[{\sarrow[red]}]自動化が十分可能にもかかわらず、事態が放置され続けている
\end{enumerate}
\end{Issues}

%%%%%%%%%%%%%%%%%%%%%%%%%%%%%%%%%%%%%%%%%%%%%%%%%%%%%%%%%%
%% subsection 02.02.04.2 %%%%%%%%%%%%%%%%%%%%%%%%%%%%%%%%%
%%%%%%%%%%%%%%%%%%%%%%%%%%%%%%%%%%%%%%%%%%%%%%%%%%%%%%%%%%
\subsubsection{\EndFaceInCChamferMilling における品質}

\begin{Issues}{\EndFaceInCChamferMilling の位置調整}
\begin{enumerate}[label=\sarrow]
\item[{\sarrow[red]}]\EndFaceInCChamferMilling のAC方向の位置調整が手作業により行われている
\item[{\sarrow[red]}]目分量により位置調整がなされているため、位置が安定しない
\item[{\sarrow[red]}]自動化が十分可能にもかかわらず、事態が放置され続けている
\end{enumerate}
\end{Issues}

\begin{Issues}{\EndFaceInCChamferMilling の始点・終点の位置}
\begin{enumerate}[label=\sarrow]
\item[{\sarrow[red]}]\EndFaceInCChamferMilling の始点および終点が直線部分にあり、加工の跡が残りやすい
\item[{\sarrow[red]}]始点および終点が同じ点になっているため、工具が少し摩耗しただけでも跡が残る
\end{enumerate}
\end{Issues}

\begin{Issues}{\EndFaceInCChamferMilling のコーナーRの補正}
\begin{enumerate}[label=\sarrow]
\item\index{しんがね@芯金}芯金の摩耗等により、\EndFaceInCChamfer のコーナーが変化しうる
\item[{\sarrow[red]}]\EndFaceInCChamferMilling のコーナーの値が一定なため、変化に対応ができない
\end{enumerate}
\end{Issues}


\clearpage
%%%%%%%%%%%%%%%%%%%%%%%%%%%%%%%%%%%%%%%%%%%%%%%%%%%%%%%%%%
%% subsection 02.02.02 %%%%%%%%%%%%%%%%%%%%%%%%%%%%%%%%%%%
%%%%%%%%%%%%%%%%%%%%%%%%%%%%%%%%%%%%%%%%%%%%%%%%%%%%%%%%%%
\subsection{\EndFaceBoringMilling における品質}

\begin{Issues}{\EndFaceBoringMilling の寸法の誤り}
\begin{enumerate}[label=\sarrow]
\item[{\sarrow[red]}]\EndFaceBoringMilling のAC方向の寸法が、図面のものと異なっている
\item 後工程で用いる端面座ぐり部の穴あけ用ジグを、誤った寸法のものに手作業で修正する必要がある
\end{enumerate}
\end{Issues}


%\clearpage
%%%%%%%%%%%%%%%%%%%%%%%%%%%%%%%%%%%%%%%%%%%%%%%%%%%%%%%%%%
%% subsection 02.02.02 %%%%%%%%%%%%%%%%%%%%%%%%%%%%%%%%%%%
%%%%%%%%%%%%%%%%%%%%%%%%%%%%%%%%%%%%%%%%%%%%%%%%%%%%%%%%%%
\subsection{加工全般における品質}

\begin{Issues}{仕上げ前加工・仕上げ加工の分離の非統一}
一般にどの加工においても、削り代がより少ないほど仕上がりがきれいなものになる
\begin{enumerate}[label=\sarrow]
\item[{\sarrow[red]}]仕上げ前の加工と仕上げの加工とで、削り代が統一されていない
\end{enumerate}
\end{Issues}

\begin{Issues}{手書きによる\index{きかいかこうすんぽううけいれチェックひょう@機械加工寸法受入チェック表}機械加工寸法受入チェック表}
\begin{enumerate}[label=\sarrow]
\item[{\sarrow[red]}]\DrawingNumber・向先・公称寸法・寸法公差許容範囲を、作業員が手書きで記述しなければならない状態にある
\item[{\sarrow[red]}]必要な項目が記載されていないことが頻繁にある
\item[{\sarrow[red]}]不必要な項目が記載されていることが頻繁にある
\item[{\sarrow[red]}]全明細の分を1つの形式で無理にまとめようとしており、かつできていない
\item[{\sarrow[red]}]自動化が十分可能にもかかわらず、事態が放置され続けている
\item[{\sarrow[red]}]しわ寄せがすべて作業員に押し付けられており、記載ミスを誘発・頻発させている
\end{enumerate}
\end{Issues}


%\clearpage
%%%%%%%%%%%%%%%%%%%%%%%%%%%%%%%%%%%%%%%%%%%%%%%%%%%%%%%%%%
%% subsection 02.02.07 %%%%%%%%%%%%%%%%%%%%%%%%%%%%%%%%%%%
%%%%%%%%%%%%%%%%%%%%%%%%%%%%%%%%%%%%%%%%%%%%%%%%%%%%%%%%%%
\subsection{工具における品質}

\begin{Issues}{工具の不整理・不整頓}
\begin{enumerate}[label=\sarrow]
\item 工具が整理されておらず、使用可能かも不明な工具が設置されている
\item 工具が整頓されておらず、無分別に工具が設置されている
\end{enumerate}
\end{Issues}



\clearpage
%%%%%%%%%%%%%%%%%%%%%%%%%%%%%%%%%%%%%%%%%%%%%%%%%%%%%%%%%%
%% section 02.03 %%%%%%%%%%%%%%%%%%%%%%%%%%%%%%%%%%%%%%%%%
%%%%%%%%%%%%%%%%%%%%%%%%%%%%%%%%%%%%%%%%%%%%%%%%%%%%%%%%%%
\modHeadsection{\index{さぎょうこうりつ@作業効率}作業効率に関するイシュー・問題}


%%%%%%%%%%%%%%%%%%%%%%%%%%%%%%%%%%%%%%%%%%%%%%%%%%%%%%%%%%
%% subsection 02.03.01 %%%%%%%%%%%%%%%%%%%%%%%%%%%%%%%%%%%
%%%%%%%%%%%%%%%%%%%%%%%%%%%%%%%%%%%%%%%%%%%%%%%%%%%%%%%%%%
\subsection{\index{NCプログラム}NCプログラムの作成における作業効率}

\begin{Issues}{\KeywayCenter 位置の手動計算による導出}
\begin{enumerate}[label=\sarrow]
\item[{\sarrow[red]}]\KeywayCenter の$X$座標を、電卓により手作業で計算して導出している
\item[{\sarrow[red]}]自動化が十分可能にもかかわらず、事態が放置され続けている
\end{enumerate}
\end{Issues}

\begin{Issues}{\AsideKeywayDepth 指定時の\KeywayCenter 位置の関節的導出}
\begin{enumerate}[label=\sarrow]
\item[{\sarrow[red]}]\AsideKeywayDepth 指定時の\KeywayCenter 位置を、直接的な測定ではなく、端面部の外側中心を基準に関節的に導出している
\item[{\sarrow[red]}]\CenterCurvature に伴う誤差を含み、\AsideKeywayDepth が安定しない
\item[{\sarrow[red]}]手動による位置調整の頻度が必然的に多くなっている
\end{enumerate}
\end{Issues}

\begin{Issues}{\index{NCメインプログラム}NCメインプログラムの手動作成}
\begin{enumerate}[label=\sarrow]
\item[{\sarrow[red]}]\index{NCメインプログラム}NCメインプログラムが人手により手動で作成されている
\item[{\sarrow[red]}]\index{NCメインプログラム}NCメインプログラムのパターン化がなされていない
\item[{\sarrow[red]}]\index{プログラマ}プログラマとしての能力が必要な\index{NCメインプログラム}NCメインプログラムの作成業務が、(スタッフや管理職でなく)作業者に課されている
\item[{\sarrow[red]}]作業者に\index{プログラマ}プログラマとしての能力が問われ、\index{きょういくコスト@教育コスト}教育コストを多くかかる状態になっている
\item[{\sarrow[red]}]自動化が十分可能にもかかわらず、事態が放置され続けている
\end{enumerate}
\end{Issues}


%%%%%%%%%%%%%%%%%%%%%%%%%%%%%%%%%%%%%%%%%%%%%%%%%%%%%%%%%%
%% subsection 02.03.02 %%%%%%%%%%%%%%%%%%%%%%%%%%%%%%%%%%%
%%%%%%%%%%%%%%%%%%%%%%%%%%%%%%%%%%%%%%%%%%%%%%%%%%%%%%%%%%
\subsection{測定における作業効率}

\begin{Issues}{測定基準値の手動による実測}
\begin{enumerate}[label=\sarrow]
\item[{\sarrow[red]}]\index{ワークざひょうけいげんてん@ワーク座標系原点}ワーク座標系原点の概算値を見出すために、オペレータが直接機内で測定している
\item[{\sarrow[red]}]機械的に見出だせるにもかかわらず、放置され続けている
\end{enumerate}
\end{Issues}

\begin{Issues}{測定時の送り速さ}
\begin{enumerate}[label=\sarrow]
\item[{\sarrow[red]}]すべての測定時の送り速さが1つの値で制御されている状態にある
\item[{\sarrow[red]}]すべての測定に対し、最も小さい送り速さに合わせる必要があり、必然的に測定に時間がかかってしまう
\end{enumerate}
\end{Issues}

\begin{Issues}{\CenterlineEndFaceDifMeasurement の作業効率}
\begin{enumerate}[label=\sarrow]
\item[{\sarrow[red]}]\CenterlineEndFaceDifMeasurement が、手動による\index{ハンドルそうさ@ハンドル操作}ハンドル操作で行われている
\item[{\sarrow[red]}]自動化が可能にもかかわらず、事態が放置され続けている
\end{enumerate}
\end{Issues}


\clearpage
%%%%%%%%%%%%%%%%%%%%%%%%%%%%%%%%%%%%%%%%%%%%%%%%%%%%%%%%%%
%% subsection 02.03.03 %%%%%%%%%%%%%%%%%%%%%%%%%%%%%%%%%%%
%%%%%%%%%%%%%%%%%%%%%%%%%%%%%%%%%%%%%%%%%%%%%%%%%%%%%%%%%%
\subsection{\EndFacecutMilling における作業効率}

\begin{Issues}{\EndFacecutMilling の手動による\TDCorrection}
\begin{enumerate}[label=\sarrow]
\item[{\sarrow[red]}]\EndFacecutMilling の基準点が外側中心として与えられており、\index{めいさい@明細}明細に依存する\indexTDFaceMill\nameTDCorrection を行わなければならない
\item[{\sarrow[red]}]作業者が手動で\indexTDFaceMill\nameTDCorrection 値を編集しなければならない状態にある
\item[{\sarrow[red]}]自動補正が可能にもかかわらず、事態が放置され続けている
\end{enumerate}
\end{Issues}

\begin{Issues}{\EndFacecutMilling の手動による加工回数の変更}
\begin{enumerate}[label=\sarrow]
\item[{\sarrow[red]}]\EndFacecutMilling の回数を変更する際、作業者が手動により直接メインプログラムを編集しなければならない状態にある
\end{enumerate}
\end{Issues}


%%%%%%%%%%%%%%%%%%%%%%%%%%%%%%%%%%%%%%%%%%%%%%%%%%%%%%%%%%
%% subsection 02.03.05 %%%%%%%%%%%%%%%%%%%%%%%%%%%%%%%%%%%
%%%%%%%%%%%%%%%%%%%%%%%%%%%%%%%%%%%%%%%%%%%%%%%%%%%%%%%%%%
\subsection{\KeywayMilling における作業効率}

\begin{Issues}{\KeywayMilling 回数($Z$方向)の手動による設定}
\begin{enumerate}[label=\sarrow]
\item[{\sarrow[red]}]\KeywayWidth・工具幅に応じた\KeywayMilling の回数を、作業者が決定しなければならない状態にある
\item[{\sarrow[red]}]\KeywayMilling の回数の変更は、作業者が\index{NCメインプログラム}NCメインプログラムを直接編集しなければならない状態にある
\item[{\sarrow[red]}]自動化が十分可能にもかかわらず、事態が放置され続けている
\end{enumerate}
\end{Issues}


%%%%%%%%%%%%%%%%%%%%%%%%%%%%%%%%%%%%%%%%%%%%%%%%%%%%%%%%%%
%% subsection 02.03.06 %%%%%%%%%%%%%%%%%%%%%%%%%%%%%%%%%%%
%%%%%%%%%%%%%%%%%%%%%%%%%%%%%%%%%%%%%%%%%%%%%%%%%%%%%%%%%%
\subsection{\EndFaceChamferMilling における作業効率}

\begin{Issues}{手作業による\EndFaceChamferMilling の加工}
\begin{enumerate}[label=\sarrow]
\item[{\sarrow[red]}]寸法の小さな\EndFaceChamferMilling が、手持ち研磨機を用いて手作業で加工されている
\item[{\sarrow[red]}]大半の場合は自動化が十分可能にもかかわらず、事態が放置され続けている
\end{enumerate}
\end{Issues}


%%%%%%%%%%%%%%%%%%%%%%%%%%%%%%%%%%%%%%%%%%%%%%%%%%%%%%%%%%
%% subsection 02.03.07 %%%%%%%%%%%%%%%%%%%%%%%%%%%%%%%%%%%
%%%%%%%%%%%%%%%%%%%%%%%%%%%%%%%%%%%%%%%%%%%%%%%%%%%%%%%%%%
\subsection{\index{とくしゅなかこう@特殊な加工}特殊な加工における作業効率}

\begin{Issues}{\CurvedOutcutMilling の\index{ハンドルそうさ@ハンドル操作}ハンドル操作による測定}
\begin{enumerate}[label=\sarrow]
\item[{\sarrow[red]}]手作業による\index{ハンドルそうさ@ハンドル操作}ハンドル操作により測定が行われている
\item[{\sarrow[red]}]自動化が十分可能にもかかわらず、事態が放置され続けている
\end{enumerate}
\end{Issues}

\begin{Issues}{\Keyway 8角形コーナーRの加工の未対応}
\begin{enumerate}[label=\sarrow]
\item[{\sarrow[red]}]8角形コーナーRの形状をした\KeywayMilling に対応できていない
\end{enumerate}
\end{Issues}


\clearpage
%%%%%%%%%%%%%%%%%%%%%%%%%%%%%%%%%%%%%%%%%%%%%%%%%%%%%%%%%%
%% subsection 02.03.08 %%%%%%%%%%%%%%%%%%%%%%%%%%%%%%%%%%%
%%%%%%%%%%%%%%%%%%%%%%%%%%%%%%%%%%%%%%%%%%%%%%%%%%%%%%%%%%
\subsection{加工全般における作業効率}

\begin{Issues}{\Table 回転による振分調整の未対応}
\begin{enumerate}[label=\sarrow]
\item[{\sarrow[red]}]振分調整が\Spacer を用いた方法のみでしかできない状態にある
\item[{\sarrow[red]}]再振分けを行う際に、必ず\Spacer の取付けまたは取外し作業を行わなければならない
\end{enumerate}
\end{Issues}

\begin{Issues}{メインプログラム編集による\index{おくりそくど@送り速度}送り速度の変更}
\begin{enumerate}[label=\sarrow]
\item[{\sarrow[red]}]\index{おくりそくど@送り速度}送り速度を変更するために、\index{NCメインプログラム}NCメインプログラムを直接編集しなければならない
\end{enumerate}
\end{Issues}

\begin{Issues}{メインプログラム編集による\index{しゅじくかいてんすう@主軸回転数}主軸回転数の変更}
\begin{enumerate}[label=\sarrow]
\item[{\sarrow[red]}]\index{しゅじくかいてんすう@主軸回転数}主軸回転数を変更するために、\index{NCメインプログラム}NCメインプログラムを直接編集しなければならない
\end{enumerate}
\end{Issues}


\clearpage
%%%%%%%%%%%%%%%%%%%%%%%%%%%%%%%%%%%%%%%%%%%%%%%%%%%%%%%%%%
%% section 02.03 %%%%%%%%%%%%%%%%%%%%%%%%%%%%%%%%%%%%%%%%%
%%%%%%%%%%%%%%%%%%%%%%%%%%%%%%%%%%%%%%%%%%%%%%%%%%%%%%%%%%
\modHeadsection{\index{しんらいせい@信頼性}信頼性に関するイシュー・問題}

\begin{Issues}{\index{だんきうんてん@暖機運転}暖機運転}
\begin{enumerate}[label=\sarrow]
\item[{\sarrow[red]}]十分な\index{だんきうんてん@暖機運転}暖機運転がなされていない
\end{enumerate}
\end{Issues}

\begin{Issues}{パラメタ誤入力の予防策}
\begin{enumerate}[label=\sarrow]
\item[{\sarrow[red]}]\index{NCメインプログラム}NCメインプログラム・\index{NCサブプログラム}NCサブプログラムともに、\index{ひきすう@引数}引数に対する誤入力の予防措置がなされていない
\item[{\sarrow[red]}]\index{NCメインプログラム}NCメインプログラム・\index{NCサブプログラム}NCサブプログラムともに、\index{コモンへんすう@コモン変数}コモン変数に対する誤入力の予防措置がなされていない
\end{enumerate}
\end{Issues}

\begin{Issues}{ワーク設置ミスによる衝突の防止策}
\begin{enumerate}[label=\sarrow]
\item[{\sarrow[red]}]ワークの設置にミスがあった場合に起因する衝突の防止策がなにも取られていない
\end{enumerate}
\end{Issues}

\begin{Issues}{\KeywayMilling 時 \index{がいぶワークざひょうけい@外部ワーク座標系}外部ワーク座標系による衝突の予防策}
\begin{enumerate}[label=\sarrow]
\item[{\sarrow[red]}]\index{がいぶワークざひょうけい@外部ワーク座標系}外部ワーク座標系による衝突の予防措置がなされていない
\item[{\sarrow[red]}]特に影響の大きい\KeywayMilling にさえなされていない
\end{enumerate}
\end{Issues}



\clearpage
%%%%%%%%%%%%%%%%%%%%%%%%%%%%%%%%%%%%%%%%%%%%%%%%%%%%%%%%%%
%% section 02.04 %%%%%%%%%%%%%%%%%%%%%%%%%%%%%%%%%%%%%%%%%
%%%%%%%%%%%%%%%%%%%%%%%%%%%%%%%%%%%%%%%%%%%%%%%%%%%%%%%%%%
\modHeadsection{全般的な問題点}
そもそも、以下のような非常に大きな問題が存在し続けている。

\begin{Issues}{根本的な問題}
\begin{enumerate}[label=\sarrow]
\item[{\sarrow[red]}] \expandafterindex{\yomiDrawing(モールド)@\nameDrawing(モールド)}\nameDrawing 作成と\index{NCプログラム}NCプログラム作成作業が直列の工程となっている
\item[{\sarrow[red]}] \expandafterindex{\yomiDrawing のさくせい)@\nameDrawing の作成}\nameDrawing の作成が、\index{CADソフトウェア}CADソフトウェアを用いてすべて手動でなされている
\item[{\sarrow[red]}] \index{モールド}モールドに関する\index{データベース@データベース}データベースに相当するものが存在しない
\end{enumerate}
\end{Issues}
これらについては、\pageautoref{part:XI}以降で取り扱う。


%!TEX root = ../RfCPN.tex


\modHeadchapter{改善の余地の特定\TBW}
\MMC における改善項目は、そのほとんどが\DMC にも適用できることが想定される。


%%%%%%%%%%%%%%%%%%%%%%%%%%%%%%%%%%%%%%%%%%%%%%%%%%%%%%%%%%
%% section 3.2 %%%%%%%%%%%%%%%%%%%%%%%%%%%%%%%%%%%%%%%%%%%
%%%%%%%%%%%%%%%%%%%%%%%%%%%%%%%%%%%%%%%%%%%%%%%%%%%%%%%%%%
\modHeadsection{新たな技術の導入\TBW}
(to be written...)



%%%%%%%%%%%%%%%%%%%%%%%%%%%%%%%%%%%%%%%%%%%%%%%%%%%%%%%%%%
%% section 3.2 %%%%%%%%%%%%%%%%%%%%%%%%%%%%%%%%%%%%%%%%%%%
%%%%%%%%%%%%%%%%%%%%%%%%%%%%%%%%%%%%%%%%%%%%%%%%%%%%%%%%%%
\modHeadsection{業務プロセスの最適化\TBW}
(to be written...)



%%%%%%%%%%%%%%%%%%%%%%%%%%%%%%%%%%%%%%%%%%%%%%%%%%%%%%%%%%
%% section 3.2 %%%%%%%%%%%%%%%%%%%%%%%%%%%%%%%%%%%%%%%%%%%
%%%%%%%%%%%%%%%%%%%%%%%%%%%%%%%%%%%%%%%%%%%%%%%%%%%%%%%%%%
\modHeadsection{教育・トレーニングの強化}
(to be written...)


\clearrightpage
%%%%%%%%%%%%%%%%%%%%%%%%%%%%%%%%%%%%%%%%%%%%%%%%%%%%%%%%%
%% Appendices %%%%%%%%%%%%%%%%%%%%%%%%%%%%%%%%%%%%%%%%%%%
%%%%%%%%%%%%%%%%%%%%%%%%%%%%%%%%%%%%%%%%%%%%%%%%%%%%%%%%%
\begin{appendices}
%\Appendixpart
\end{appendices}

\addtocontents{toc}{\protect\end{tocBox}}
\clearrightpage
