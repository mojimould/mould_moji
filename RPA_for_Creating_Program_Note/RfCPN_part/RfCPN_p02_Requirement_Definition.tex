%!TEX root = ./RPA_for_Creating_Program_Note.tex


\addtocontents{toc}{\protect\cleardoublepage}
%%%%%%%%%%%%%%%%%%%%%%%%%%%%%%%%%%%%%%%%%%%%%%%%%%%%%%%%%
%% Part Requirement Definition %%%%%%%%%%%%%%%%%%%%%%%%%%
%%%%%%%%%%%%%%%%%%%%%%%%%%%%%%%%%%%%%%%%%%%%%%%%%%%%%%%%%
\addtocontents{toc}{\protect\begin{tocBox}{\tmppartnum}}%
\tPart{\DMname における要件定義\TBW}{概要}{%
\paragraph*{目標(なにがしたいか?)}
(to be written...)

\tcbline*
\paragraph*{手段(どうやって?)}
(to be written...)

\tcbline*
\paragraph*{背景(なぜ?)}
(to be written...)\\
外注で作成した\index{プログラム(がいちゅう)@プログラム(外注)}プログラムも、当社が具体的な\index{ようけんていぎ@要件定義}要件定義すら行えず、実用に至っていない
%% footnote %%%%%%%%%%%%%%%%%%%%%
\footnote{プログラム自体は尤もな内容であり、レベルも十分なものである。
(第二製造室に関わらず)当社の\index{ソフトウェアエンジニアリング}ソフトウェアエンジニアリングに関する致命的なほどの無関心が顕わに露呈したものであり、自明な帰結である。}。
%%%%%%%%%%%%%%%%%%%%%%%%%%%%%%%%%

 したがって、\index{システムかいはつプロセス@システム開発プロセス}システム開発プロセスの計画の策定を行うことが喫緊の課題である。

\tcbline*
\paragraph*{結論(どうなった?)}
\MMname における業務の流れを把握し、それを基に\DMname の稼働に向けたソフトウェア視点によるシステム開発プロセスの計画を策定した。
}

%%%%%%%%%%%%%%%%%%%%%%%%%%%%%%%%%%%%%%%%%%%%%%%%%%%%%%%%%
%% chapters %%%%%%%%%%%%%%%%%%%%%%%%%%%%%%%%%%%%%%%%%%%%%%
%%%%%%%%%%%%%%%%%%%%%%%%%%%%%%%%%%%%%%%%%%%%%%%%%%%%%%%%%%
%!TEX root = ../RPA_for_Creating_Program_Note.tex


\modHeadchapter{はじめに\TBW}
% 本文書の目的と範囲について説明



%%%%%%%%%%%%%%%%%%%%%%%%%%%%%%%%%%%%%%%%%%%%%%%%%%%%%%%%%%
%% section 8.1 %%%%%%%%%%%%%%%%%%%%%%%%%%%%%%%%%%%%%%%%%%%
%%%%%%%%%%%%%%%%%%%%%%%%%%%%%%%%%%%%%%%%%%%%%%%%%%%%%%%%%%
\modHeadsection{文書の目的\TBW}



%%%%%%%%%%%%%%%%%%%%%%%%%%%%%%%%%%%%%%%%%%%%%%%%%%%%%%%%%%
%% section 8.2 %%%%%%%%%%%%%%%%%%%%%%%%%%%%%%%%%%%%%%%%%%%
%%%%%%%%%%%%%%%%%%%%%%%%%%%%%%%%%%%%%%%%%%%%%%%%%%%%%%%%%%
\modHeadsection{文書の範囲\TBW}


\modHeadchapter{目標の明確化\TBW}
% 新たなマシニングセンタで何を達成したいのか、具体的な目標を設定
\modHeadsection{達成したい目標\TBW}
\modHeadsection{目標の優先順位\TBW}


\modHeadchapter{機能要件の洗い出し\TBW}
% 新たなマシニングセンタがどのような機能を持つべきかをリストアップ
\modHeadsection{必要な加工能力\TBW}
\modHeadsection{必要な精度\TBW}
\modHeadsection{必要な速度\TBW}

\modHeadchapter{ソフトウェア要件の定義\TBW}
% 新たなマシニングセンタの操作や管理に必要なソフトウェアの要件を定義
\modHeadsection{ユーザーインターフェース\TBW}
\modHeadsection{データ管理\TBW}
\modHeadsection{セキュリティ\TBW}
\modHeadsection{互換性\TBW}

\modHeadchapter{非機能要件の考慮\TBW}
% システムの性能、信頼性、拡張性など、ソフトウェアの「どのように動作するべきか」に関する要件を定義
\modHeadsection{システムの性能\TBW}
\modHeadsection{システムの信頼性\TBW}
\modHeadsection{システムの拡張性\TBW}

\modHeadchapter{制約の特定\TBW}
% 予算、時間、既存のシステムとの互換性など、プロジェクトに影響を与える可能性のある制約を特定
\modHeadsection{予算\TBW}
\modHeadsection{時間\TBW}
\modHeadsection{既存のシステムとの互換性\TBW}

\modHeadchapter{要件の文書化と検証\TBW}
% すべての要件を文書化し、関係者全員が理解し、合意できることを確認
\modHeadsection{要件の文書化\TBW}
\modHeadsection{要件の検証\TBW}

\modHeadchapter{まとめ\TBW}
% 本文書の結論と次のステップについて説明
\modHeadsection{結論\TBW}
\modHeadsection{次のステップ\TBW}

\clearrightpage
%%%%%%%%%%%%%%%%%%%%%%%%%%%%%%%%%%%%%%%%%%%%%%%%%%%%%%%%%
%% Appendices %%%%%%%%%%%%%%%%%%%%%%%%%%%%%%%%%%%%%%%%%%%
%%%%%%%%%%%%%%%%%%%%%%%%%%%%%%%%%%%%%%%%%%%%%%%%%%%%%%%%%
\begin{appendices}
%\Appendixpart
\end{appendices}

\addtocontents{toc}{\protect\end{tocBox}}
\clearrightpage
