%!TEX root = ../RPA_for_Creating_Program_Note.tex


%%%%%%%%%%%%%%%%%%%%%%%%%%%%%%%%%%%%%%%%%%%%%%%%%%%%%%%%%
%% Part Work Flow     %%%%%%%%%%%%%%%%%%%%%%%%%%%%%%%%%%%
%%%%%%%%%%%%%%%%%%%%%%%%%%%%%%%%%%%%%%%%%%%%%%%%%%%%%%%%%
\addtocontents{toc}{\protect\begin{tocBox}{\tmppartnum}}%
\tPart{ソフトウェア視点における業務フロー}{概要}{%
\paragraph*{目標(なにがしたいか?)}
\index{よこがたマシニングセンタ@横型マシニングセンタ}横型マシニングセンタにおける作業において、\textbf{ソフトウェアの視点から現在の業務の流れを明確化}する。
それに基づき、新たなマシニングセンタ\textbf{導入後の業務の流れを計画}する。

\tcbline*
\paragraph*{手段(どうやって?)}
関係者に詳しく聞き取りを行い、ソフトウェアの観点における現在の業務の流れを把握する。
それに基づいて解決可能な課題を抽出し、改善を施した上で導入後の業務の流れを取り決める。

\tcbline*
\paragraph*{背景(なぜ?)}
新たなマシニングセンタについて、ハードウェア視点では大きな問題はないという形で設置に至った。

 一方、ソフトウェア視点においては、事態は惨憺たるものである。
「\index{ソフトウェアにかんするかんり・ぎょうむ@ソフトウェアに関する管理・業務}」ソフトウェア関する管理・業務」を行う部門はおろか、担当者(特に管理職)さえ社内に存在しない。
そのため業務の流れすら体系的に把握されておらず、したがって\index{ぎょうむきてい@業務規程}業務規程や\index{かいはつプロセス@開発プロセス}開発プロセスの計画も事実上皆無である
%% footnote %%%%%%%%%%%%%%%%%%%%%
\footnote{つまり、ソフトウェアの観点からすると、もはや企業としての体をなしていない。
事実に基づいて客観的に判断すると、著しくモラルが低いと言わざるを得ない状態にある。
なお、これは2024/03現在、全く事態は変わっていない。}。
%%%%%%%%%%%%%%%%%%%%%%%%%%%%%%%%%

 このように、客観的事実として、長期にわたり業務の中枢となる部分が全く機能していない。
そのため、システムの\index{かいはつプロセス@開発プロセス}開発プロセスの最初期の段階である\textbf{現在の\index{ぎょうむフロー@業務フロー}業務フローを把握}する段階から着手しなければならない。

\tcbline*
\paragraph*{結論(どうなった?)}
ソフトウェアの視点に立って現状の横型マシニングセンタにおける業務の流れを把握し、それに基づいて新たなマシニングセンタ導入後の業務の流れを取り決めた。
}

%%%%%%%%%%%%%%%%%%%%%%%%%%%%%%%%%%%%%%%%%%%%%%%%%%%%%%%%%%
%% chapters %%%%%%%%%%%%%%%%%%%%%%%%%%%%%%%%%%%%%%%%%%%%%%
%%%%%%%%%%%%%%%%%%%%%%%%%%%%%%%%%%%%%%%%%%%%%%%%%%%%%%%%%%
%!TEX root = ../RfCPN.tex


\modHeadchapter[lot]{現状の\index{よこがたマシニングセンタ@横型マシニングセンタ}横型マシニングセンタの\index{ぎょうむフロー@業務フロー}業務フロー}
新たに導入する\index{よこがたマシニングセンタ@横型マシニングセンタ}横型マシニングセンタ(以下、\textbf{\DMC})での\expandafterindex{こうてい(\yomiDMC)@工程(\yomiDMC)}工程は、\Dimple や\ReliefGroove の測定・加工を除けば三菱製横型マシニングセンタ(以下、\textbf{\MMC})とほぼ同様である。
そこで、まずは\MMC ではどのようなフローで業務が行われているかを(ソフトウェアの観点から)みることにする。
%%%%%%%%%%%%%%%%%%%%%%%%%%%%%%%%%%%%%%%%%%%%%%%%%%%%%%%%%%
%% marker %%%%%%%%%%%%%%%%%%%%%%%%%%%%%%%%%%%%%%%%%%%%%%%%
%%%%%%%%%%%%%%%%%%%%%%%%%%%%%%%%%%%%%%%%%%%%%%%%%%%%%%%%%%
\begin{marker}
ここでは主に\MMC の\expandafterindex{\yomiNoOnePalette(\yomiMMC)@\nameNoOnePalette(\nameMMC)}\nameNoOnePalette で加工を行うものを対象とする
%% footnote %%%%%%%%%%%%%%%%%%%%%
\footnote{\expandafterindex{\yomiNoTwoPalette(\yomiMMC)@\nameNoTwoPalette(\nameMMC)}\nameNoTwoPalette では、\index{おおがたのモールド@大型のモールド}大型なものや\index{まるがたのモールド@丸型のモールド}丸形のもの等の加工が主に行われる。}。
%%%%%%%%%%%%%%%%%%%%%%%%%%%%%%%%%
\end{marker}
%%%%%%%%%%%%%%%%%%%%%%%%%%%%%%%%%%%%%%%%%%%%%%%%%%%%%%%%%%
%%%%%%%%%%%%%%%%%%%%%%%%%%%%%%%%%%%%%%%%%%%%%%%%%%%%%%%%%%
%%%%%%%%%%%%%%%%%%%%%%%%%%%%%%%%%%%%%%%%%%%%%%%%%%%%%%%%%%



%%%%%%%%%%%%%%%%%%%%%%%%%%%%%%%%%%%%%%%%%%%%%%%%%%%%%%%%%%
%% section 01.01 %%%%%%%%%%%%%%%%%%%%%%%%%%%%%%%%%%%%%%%%%
%%%%%%%%%%%%%%%%%%%%%%%%%%%%%%%%%%%%%%%%%%%%%%%%%%%%%%%%%%
\modHeadsection{\MMC における\expandafterindex{こうてい(\yomiMMC)@工程(\nameMMC)}工程および\expandafterindex{しようツール(\yomiMMC)@使用ツール(\nameMMC)}使用ツール}


%%%%%%%%%%%%%%%%%%%%%%%%%%%%%%%%%%%%%%%%%%%%%%%%%%%%%%%%%%
%% subsection 01.01.1 %%%%%%%%%%%%%%%%%%%%%%%%%%%%%%%%%%%%
%%%%%%%%%%%%%%%%%%%%%%%%%%%%%%%%%%%%%%%%%%%%%%%%%%%%%%%%%%
\subsection{\expandafterindex{こうていのしゅるい(\yomiMMC)@工程の種類(\nameMMC)}工程の種類(\yomiMMC)}
\MMC について、直接的な\index{ワーク}ワークに対する測定・加工に関するものに着目すると、主に以下のような\expandafterindex{こうてい(\yomiMMC)@工程(\nameMMC)}工程が行われる。\\

\begin{multicollongtblr}{\index{ワーク}ワークワークに直接関わる主な\expandafterindex{こうていのしゅるい(\yomiMMC)@工程の種類(\nameMMC)}工程の種類(\yomiMMC)}{X[l] X[l]}
測定 & 加工\\
\EndFace 外側AC方向 芯出し(外側 両側$X$測定) & \EndFacecutMilling\\
\EndFace 外側BD方向 芯出し(外側 両側$Y$測定) & \OutcutMilling\\
\OutcutWidth AC方向 芯出し(内側 片側$X$測定) & \KeywayMilling\\
\EndFace 内側AC方向 芯出し(内側 両側$X$測定) & \EndFaceOutCChamferMilling\\
\EndFace 内側BD方向 芯出し(内側 両側$Y$測定) & \EndFaceInCChamferMilling\\
\CenterlineEndFaceDifAC 測定(外側 片側$Z$測定) & \EndFaceBoringMilling\\
\CenterlineEndFaceDifBD 測定(外側 片側$Y$測定) & \IncutBoringMilling\\
\end{multicollongtblr}


\clearpage
%%%%%%%%%%%%%%%%%%%%%%%%%%%%%%%%%%%%%%%%%%%%%%%%%%%%%%%%%%
%% subsection 01.01.2 %%%%%%%%%%%%%%%%%%%%%%%%%%%%%%%%%%%%
%%%%%%%%%%%%%%%%%%%%%%%%%%%%%%%%%%%%%%%%%%%%%%%%%%%%%%%%%%
\subsection{\expandafterindex{しようツール(\yomiMMC)@使用ツール(\nameMMC)}使用ツール(\yomiMMC)\label{subsec:using_tool}}
ソフトウェアに着目すると、\MMC による加工では主に以下のようなツールが使用されている。\\

\begin{multicollongtblr}{\expandafterindex{しようソフトウェア(\yomiMMC)@使用ソフトウェア(\nameMMC)}使用ソフトウェアおよび\expandafterindex{しようツール(\yomiMMC)@使用ツール(\nameMMC)}ツール(\yomiMMC)}{l X[l]}
ツール & 主な用途\\
\MMC 操作盤 & \index{NCプログラム}NCプログラムや各種\index{へんすう(NCプログラム)@変数(NCプログラム)}変数の編集\\
\index{みつびしマシニングセンタむじんかシステム@三菱マシニングセンタ無人化システム}三菱マシニングセンタ無人化システム & \index{タッチセンサープローブ}タッチセンサープローブ測定\\
\expandafterindex{さくせいしたNCプログラム(\yomiMMC)@作成したNCプログラム(\nameMMC)}作成したNCプログラム & 各々の加工\\
振分調整用スペーサ計算プログラム & 振分調整用スペーサの選定 および \ReAlocationLength の決定\\
\index{かんすうでんたく@関数電卓}関数電卓 & 四則演算・根号の計算\\
\end{multicollongtblr}

\noindent
\MMC では、各々の\index{こうてい(しんだしそくてい)@工程(芯出し測定)}芯出し測定の工程では「\index{みつびしマシニングセンタむじんかシステム@三菱マシニングセンタ無人化システム}三菱マシニングセンタ無人化システム」を、各々の加工工程では当社の従業員(一般職)により作成された\index{サブプログラム}サブプログラムを用いている。
これらの\index{サブプログラム}サブプログラムを用いることにより、各々の\index{めいさい(モールド)@明細(モールド)}明細における必要な\index{すんぽう@寸法}寸法値等を\index{ひきすう(NCプログラム)@引数(NCプログラム)}引数に格納して用いればよい状態になっている。
%%%%%%%%%%%%%%%%%%%%%%%%%%%%%%%%%%%%%%%%%%%%%%%%%%%%%%%%%%
%% marker %%%%%%%%%%%%%%%%%%%%%%%%%%%%%%%%%%%%%%%%%%%%%%%%
%%%%%%%%%%%%%%%%%%%%%%%%%%%%%%%%%%%%%%%%%%%%%%%%%%%%%%%%%%
\begin{marker}
つまり、\MMC において「NCプログラムの作成」とは、主にこの具体的な\index{ひきすう(NCプログラム)@引数(NCプログラム)}引数等を計算することを意味している。
\end{marker}
%%%%%%%%%%%%%%%%%%%%%%%%%%%%%%%%%%%%%%%%%%%%%%%%%%%%%%%%%%
%%%%%%%%%%%%%%%%%%%%%%%%%%%%%%%%%%%%%%%%%%%%%%%%%%%%%%%%%%
%%%%%%%%%%%%%%%%%%%%%%%%%%%%%%%%%%%%%%%%%%%%%%%%%%%%%%%%%%



\clearpage
%%%%%%%%%%%%%%%%%%%%%%%%%%%%%%%%%%%%%%%%%%%%%%%%%%%%%%%%%%
%% section 01.02 %%%%%%%%%%%%%%%%%%%%%%%%%%%%%%%%%%%%%%%%%
%%%%%%%%%%%%%%%%%%%%%%%%%%%%%%%%%%%%%%%%%%%%%%%%%%%%%%%%%%
\modHeadsection{加工の流れ(加工前)}
\MMC において、ある\index{めいさい@明細}明細のワークを加工をする際に、以下のような流れで作業が行われる。
%%%%%%%%%%%%%%%%%%%%%%%%%%%%%%%%%%%%%%%%%%%%%%%%%%%%%%%%%%
%% marker %%%%%%%%%%%%%%%%%%%%%%%%%%%%%%%%%%%%%%%%%%%%%%%%
%%%%%%%%%%%%%%%%%%%%%%%%%%%%%%%%%%%%%%%%%%%%%%%%%%%%%%%%%%
\begin{marker}
ここで挙げている必要なパラメタ(\index{すんぽう@寸法}寸法)には、その\index{こうさ@公差}公差も考慮されているものとする。
\end{marker}
%%%%%%%%%%%%%%%%%%%%%%%%%%%%%%%%%%%%%%%%%%%%%%%%%%%%%%%%%%
%%%%%%%%%%%%%%%%%%%%%%%%%%%%%%%%%%%%%%%%%%%%%%%%%%%%%%%%%%
%%%%%%%%%%%%%%%%%%%%%%%%%%%%%%%%%%%%%%%%%%%%%%%%%%%%%%%%%%


%%%%%%%%%%%%%%%%%%%%%%%%%%%%%%%%%%%%%%%%%%%%%%%%%%%%%%%%%%
%% subsection 01.1.1 %%%%%%%%%%%%%%%%%%%%%%%%%%%%%%%%%%%%%
%%%%%%%%%%%%%%%%%%%%%%%%%%%%%%%%%%%%%%%%%%%%%%%%%%%%%%%%%%
\subsection{\expandafterindex{\yomiDrawing(モールド)@\nameDrawing(モールド)}\nameDrawing の確認}
\begin{enumerate}[label=\sarrow]
\item 対象となる\index{めいさい@明細}明細の\expandafterindex{\yomiDrawing(モールド)@\nameDrawing(モールド)}\nameDrawing を用意する
\item 他に内容が類似する\index{めいさい@明細}明細の\expandafterindex{\yomiDrawing(モールド)@\nameDrawing(モールド)}\nameDrawing があれば、それも併せて用意する
\end{enumerate}
%%%%%%%%%%%%%%%%%%%%%%%%%%%%%%%%%%%%%%%%%%%%%%%%%%%%%%%%%%
%% PARAMETER %%%%%%%%%%%%%%%%%%%%%%%%%%%%%%%%%%%%%%%%%%%%%
%%%%%%%%%%%%%%%%%%%%%%%%%%%%%%%%%%%%%%%%%%%%%%%%%%%%%%%%%%
\begin{Parameter}{必要なパラメタ}
\PMDrawingExists%
\PMDrawingNumber%
\end{Parameter}
%%%%%%%%%%%%%%%%%%%%%%%%%%%%%%%%%%%%%%%%%%%%%%%%%%%%%%%%%%
%%%%%%%%%%%%%%%%%%%%%%%%%%%%%%%%%%%%%%%%%%%%%%%%%%%%%%%%%%
%%%%%%%%%%%%%%%%%%%%%%%%%%%%%%%%%%%%%%%%%%%%%%%%%%%%%%%%%%


%%%%%%%%%%%%%%%%%%%%%%%%%%%%%%%%%%%%%%%%%%%%%%%%%%%%%%%%%%
%% subsection 01.1.2 %%%%%%%%%%%%%%%%%%%%%%%%%%%%%%%%%%%%%
%%%%%%%%%%%%%%%%%%%%%%%%%%%%%%%%%%%%%%%%%%%%%%%%%%%%%%%%%%
\subsection{加工部分の有無の確認}

%%%%%%%%%%%%%%%%%%%%%%%%%%%%%%%%%%%%%%%%%%%%%%%%%%%%%%%%%%
%% subsubsection 01.1.2.2 %%%%%%%%%%%%%%%%%%%%%%%%%%%%%%%%
%%%%%%%%%%%%%%%%%%%%%%%%%%%%%%%%%%%%%%%%%%%%%%%%%%%%%%%%%%
\subsubsection{\EndFacecutMilling 部分}
\EndFacecutMilling については、全明細に共通の形で存在する。

%%%%%%%%%%%%%%%%%%%%%%%%%%%%%%%%%%%%%%%%%%%%%%%%%%%%%%%%%%
%% subsubsection 01.1.2.2 %%%%%%%%%%%%%%%%%%%%%%%%%%%%%%%%
%%%%%%%%%%%%%%%%%%%%%%%%%%%%%%%%%%%%%%%%%%%%%%%%%%%%%%%%%%
\subsubsection{\OutcutMilling 部分}
\OutcutMilling については、\index{めいさい@明細}明細により\OutcutExists または\OutcutTaperExists の違いが存在する。
\begin{enumerate}[label=\sarrow]
\item \TopOutcutExists または\BottomOutcutExists を確認する
\item \TopOutcutTaperExists および\BottomOutcutTaperExists を確認し、使用する\expandafterindex{こうぐ(\yomiOutcut)@工具(\nameOutcut)}工具を決定する
\item \nameOutcut が\CurvedOutcut かどうかも確認する
\end{enumerate}
%\clearpage
%%%%%%%%%%%%%%%%%%%%%%%%%%%%%%%%%%%%%%%%%%%%%%%%%%%%%%%%%%
%% PARAMETER %%%%%%%%%%%%%%%%%%%%%%%%%%%%%%%%%%%%%%%%%%%%%
%%%%%%%%%%%%%%%%%%%%%%%%%%%%%%%%%%%%%%%%%%%%%%%%%%%%%%%%%%
\begin{Parameter}{必要なパラメタ}
\PMTopOutcutExists%
\PMBottomOutcutExists\\
\PMTopOutcutTaperExists%
\PMBottomOutcutTaperExists%
\end{Parameter}
%%%%%%%%%%%%%%%%%%%%%%%%%%%%%%%%%%%%%%%%%%%%%%%%%%%%%%%%%%
%%%%%%%%%%%%%%%%%%%%%%%%%%%%%%%%%%%%%%%%%%%%%%%%%%%%%%%%%%
%%%%%%%%%%%%%%%%%%%%%%%%%%%%%%%%%%%%%%%%%%%%%%%%%%%%%%%%%%

%\clearpage
%%%%%%%%%%%%%%%%%%%%%%%%%%%%%%%%%%%%%%%%%%%%%%%%%%%%%%%%%%
%% subsubsection 01.02.02.3 %%%%%%%%%%%%%%%%%%%%%%%%%%%%%%
%%%%%%%%%%%%%%%%%%%%%%%%%%%%%%%%%%%%%%%%%%%%%%%%%%%%%%%%%%
\subsubsection{\KeywayMilling 部分}
\KeywayMilling については、全明細のトップ側に存在し、明細により種類の違いが存在する。
\begin{enumerate}[label=\sarrow]
\item \nameKeywayCornerType を確認し、使用する\expandafterindex{サブプログラム(\yomiKeyway)@サブプログラム(\nameKeyway)}サブプログラムの判断を行う
\item \nameKeywayWidth を確認し、使用する\expandafterindex{こうぐ(\yomiKeyway)@工具(\nameKeyway)}工具の判断を行う
\end{enumerate}
%%%%%%%%%%%%%%%%%%%%%%%%%%%%%%%%%%%%%%%%%%%%%%%%%%%%%%%%%%
%% PARAMETER %%%%%%%%%%%%%%%%%%%%%%%%%%%%%%%%%%%%%%%%%%%%%
%%%%%%%%%%%%%%%%%%%%%%%%%%%%%%%%%%%%%%%%%%%%%%%%%%%%%%%%%%
\begin{Parameter}{必要なパラメタ}
\PMKeywayCornerType%
\PMTopOutcutExists%
\PMKeywayWidth%
\end{Parameter}
%%%%%%%%%%%%%%%%%%%%%%%%%%%%%%%%%%%%%%%%%%%%%%%%%%%%%%%%%%
%%%%%%%%%%%%%%%%%%%%%%%%%%%%%%%%%%%%%%%%%%%%%%%%%%%%%%%%%%
%%%%%%%%%%%%%%%%%%%%%%%%%%%%%%%%%%%%%%%%%%%%%%%%%%%%%%%%%%

\clearpage
%%%%%%%%%%%%%%%%%%%%%%%%%%%%%%%%%%%%%%%%%%%%%%%%%%%%%%%%%%
%% subsubsection 01.02.02.04 %%%%%%%%%%%%%%%%%%%%%%%%%%%%%
%%%%%%%%%%%%%%%%%%%%%%%%%%%%%%%%%%%%%%%%%%%%%%%%%%%%%%%%%%
\subsubsection{\EndFaceChamferMilling 部分}
\EndFaceChamferMilling については、全明細に存在し、明細により種類の違いが存在する。
\begin{enumerate}[label=\sarrow]
\item 種類が\EndFaceCChamfer であれば、その\EndFaceCChamferLength により\MMC による加工を行うか判断を行う
\item \EndFaceCChamferAngle を確認し、使用する\expandafterindex{こうぐ(\yomiEndFaceCChamfer)@工具(\nameEndFaceCChamfer)}工具を決定する
\end{enumerate}
%%%%%%%%%%%%%%%%%%%%%%%%%%%%%%%%%%%%%%%%%%%%%%%%%%%%%%%%%%
%% PARAMETER %%%%%%%%%%%%%%%%%%%%%%%%%%%%%%%%%%%%%%%%%%%%%
%%%%%%%%%%%%%%%%%%%%%%%%%%%%%%%%%%%%%%%%%%%%%%%%%%%%%%%%%%
\begin{Parameter}{必要なパラメタ}
\PMChamferType%
\PMTopEndFaceOutCChamferLength
\PMTopEndFaceOutCChamferAngle%
\PMTopOutcutExists\\
\PMBottomEndFaceOutCChamferLength%
\PMBottomEndFaceOutCChamferAngle%
\PMBottomOutcutExists%
\end{Parameter}
%%%%%%%%%%%%%%%%%%%%%%%%%%%%%%%%%%%%%%%%%%%%%%%%%%%%%%%%%%
%%%%%%%%%%%%%%%%%%%%%%%%%%%%%%%%%%%%%%%%%%%%%%%%%%%%%%%%%%
%%%%%%%%%%%%%%%%%%%%%%%%%%%%%%%%%%%%%%%%%%%%%%%%%%%%%%%%%%

%\clearpage
%%%%%%%%%%%%%%%%%%%%%%%%%%%%%%%%%%%%%%%%%%%%%%%%%%%%%%%%%%
%% subsubsection 01.02.02.5 %%%%%%%%%%%%%%%%%%%%%%%%%%%%%%
%%%%%%%%%%%%%%%%%%%%%%%%%%%%%%%%%%%%%%%%%%%%%%%%%%%%%%%%%%
\subsubsection{\EndFaceBoringMilling 部分\TBW}
(to be written...)
%%%%%%%%%%%%%%%%%%%%%%%%%%%%%%%%%%%%%%%%%%%%%%%%%%%%%%%%%%
%% PARAMETER %%%%%%%%%%%%%%%%%%%%%%%%%%%%%%%%%%%%%%%%%%%%%
%%%%%%%%%%%%%%%%%%%%%%%%%%%%%%%%%%%%%%%%%%%%%%%%%%%%%%%%%%
\begin{Parameter}{必要なパラメタ}
\PMEndFaceBoringExists%
\PMEndFaceBoringCornerR%
\end{Parameter}
%%%%%%%%%%%%%%%%%%%%%%%%%%%%%%%%%%%%%%%%%%%%%%%%%%%%%%%%%%
%%%%%%%%%%%%%%%%%%%%%%%%%%%%%%%%%%%%%%%%%%%%%%%%%%%%%%%%%%
%%%%%%%%%%%%%%%%%%%%%%%%%%%%%%%%%%%%%%%%%%%%%%%%%%%%%%%%%%

%\clearpage
%%%%%%%%%%%%%%%%%%%%%%%%%%%%%%%%%%%%%%%%%%%%%%%%%%%%%%%%%%
%% subsubsection 01.02.02.06 %%%%%%%%%%%%%%%%%%%%%%%%%%%%%
%%%%%%%%%%%%%%%%%%%%%%%%%%%%%%%%%%%%%%%%%%%%%%%%%%%%%%%%%%
\subsubsection{\IncutBoringMilling 部分\TBW}
(to be written...)
%%%%%%%%%%%%%%%%%%%%%%%%%%%%%%%%%%%%%%%%%%%%%%%%%%%%%%%%%%
%% PARAMETER %%%%%%%%%%%%%%%%%%%%%%%%%%%%%%%%%%%%%%%%%%%%%
%%%%%%%%%%%%%%%%%%%%%%%%%%%%%%%%%%%%%%%%%%%%%%%%%%%%%%%%%%
\begin{Parameter}{必要なパラメタ}
\PMIncutBoringExists%
\end{Parameter}
%%%%%%%%%%%%%%%%%%%%%%%%%%%%%%%%%%%%%%%%%%%%%%%%%%%%%%%%%%
%%%%%%%%%%%%%%%%%%%%%%%%%%%%%%%%%%%%%%%%%%%%%%%%%%%%%%%%%%
%%%%%%%%%%%%%%%%%%%%%%%%%%%%%%%%%%%%%%%%%%%%%%%%%%%%%%%%%%


\clearpage
%%%%%%%%%%%%%%%%%%%%%%%%%%%%%%%%%%%%%%%%%%%%%%%%%%%%%%%%%%
%% subsection 01.02.01 %%%%%%%%%%%%%%%%%%%%%%%%%%%%%%%%%%%
%%%%%%%%%%%%%%%%%%%%%%%%%%%%%%%%%%%%%%%%%%%%%%%%%%%%%%%%%%
\subsection{加工部分の寸法の確認}

%%%%%%%%%%%%%%%%%%%%%%%%%%%%%%%%%%%%%%%%%%%%%%%%%%%%%%%%%%
%% subsubsection 01.02.01.1 %%%%%%%%%%%%%%%%%%%%%%%%%%%%%%
%%%%%%%%%%%%%%%%%%%%%%%%%%%%%%%%%%%%%%%%%%%%%%%%%%%%%%%%%%
\subsubsection{\EndFacecutMilling における寸法}
\begin{enumerate}[label=\sarrow]
\item \expandafterindex{こうさ(\yomiWorkTotalLength)@公差(\nameWorkTotalLength)}\nameWorkTotalLength の公差を確認し、\TopAlocationLength および\BottomAlocationLength の\expandafterindex{こうさ(\yomiAlocationLength)@公差(\nameAlocationLength)}公差の判断を行う
\item \TopAlocationLength・\BottomAlocationLength を確認し、\Spacer による調整が必要か判断を行う
\item 使用する\Spacer および\ReAlocationLength は、専用の計算プログラム(\linkExcelVBA)を用いて決定する
\item \OuterDiameter・\expandafterindex{\yomiThickness(\yomiEndFace)@\nameThickness(\nameEndFace)}\nameEndFace 部\nameThickness・\EndFaceOuterCornerR を確認し、それに応じて\TDCValue を決定する
\item \ReAlocationLength に応じて、$Z$方向の\index{クリアランスへいめん(Zほうこう)@クリアランス平面($Z$方向)}クリアランス平面の位置を決定する
\end{enumerate}
%%%%%%%%%%%%%%%%%%%%%%%%%%%%%%%%%%%%%%%%%%%%%%%%%%%%%%%%%%
%% PARAMETER %%%%%%%%%%%%%%%%%%%%%%%%%%%%%%%%%%%%%%%%%%%%%
%%%%%%%%%%%%%%%%%%%%%%%%%%%%%%%%%%%%%%%%%%%%%%%%%%%%%%%%%%
\begin{Parameter}{必要なパラメタ}
\paragraph*{\ReAlocationLength}
\PMWorkTotalLength
\PMTopAlocationLength
\PMBottomAlocationLength
\PMACOD
\PMJigLength
\GPMbox{受板の幅}
\tcbline*
\paragraph*{\TopEndFacecut}
\PMTopReAlocationLength
\PMACOD
\PMBDOD
\PMODCornerR\\
\PMTopEndACID
\PMTopEndBDID
\GPMbox{トップ側$Z$方向クリアランス平面距離}
\tcbline*
\paragraph*{\BottomEndFacecut}
\PMBottomReAlocationLength
\PMACOD
\PMBDOD
\PMODCornerR\\
\PMBottomEndACID
\PMBottomEndBDID
\GPMbox{ボトム側$Z$方向クリアランス平面距離}
\tcbline*
\paragraph*{\TopEndFacecut・\BottomEndFacecut~共通}
\GPMbox{\EndFacecutMilling 用\TDCValue}
\end{Parameter}
%%%%%%%%%%%%%%%%%%%%%%%%%%%%%%%%%%%%%%%%%%%%%%%%%%%%%%%%%%
%%%%%%%%%%%%%%%%%%%%%%%%%%%%%%%%%%%%%%%%%%%%%%%%%%%%%%%%%%
%%%%%%%%%%%%%%%%%%%%%%%%%%%%%%%%%%%%%%%%%%%%%%%%%%%%%%%%%%

\clearpage
%%%%%%%%%%%%%%%%%%%%%%%%%%%%%%%%%%%%%%%%%%%%%%%%%%%%%%%%%%
%% subsubsection 01.01.03.2 %%%%%%%%%%%%%%%%%%%%%%%%%%%%%%
%%%%%%%%%%%%%%%%%%%%%%%%%%%%%%%%%%%%%%%%%%%%%%%%%%%%%%%%%%
\subsubsection{\OutcutMilling における寸法}
\begin{enumerate}[label=\sarrow]
\item \OutcutLength と\Keyway の位置を確認し、実際に加工する\OutcutLength の判断を行う
\item トップ側・ボトム側の両方に\Outcut のある場合は、どちら側が\expandafterindex{きじゅん(\yomiOutcutCenter)@基準(\nameOutcutCenter)}基準であるのかを確認する
\item \EndFaceID・\OutcutAsideThickness・内面の\PlatingThk から、\OutcutCenter$X$位置用のパラメタを手動による計算で決定する
\item \CurvedOutcut の場合は、\CurvedOutcutAngle を手動による計算で決定する
\end{enumerate}
%%%%%%%%%%%%%%%%%%%%%%%%%%%%%%%%%%%%%%%%%%%%%%%%%%%%%%%%%%
%% PARAMETER %%%%%%%%%%%%%%%%%%%%%%%%%%%%%%%%%%%%%%%%%%%%%
%%%%%%%%%%%%%%%%%%%%%%%%%%%%%%%%%%%%%%%%%%%%%%%%%%%%%%%%%%
\begin{Parameter}{必要なパラメタ}
\paragraph*{ボトム側の\Outcut のみの場合}
\PMBottomOutcutACWidth
\PMBottomOutcutBDWidth
\PMBottomOutcutConerR
\PMBottomOutcutLength\\
\PMBottomEndACID
\PMBottomOutcutAsideThickness
\PMPlatingThk
\tcbline*
\paragraph*{トップ側の\Outcut のみの場合}
\PMTopOutcutACWidth
\PMTopOutcutBDWidth
\PMTopOutcutCornerR\\
\PMTopOutcutLength
\PMKeywayPos
\PMKeywayWidth\\
\PMTopEndACID
\PMTopOutcutAsideThickness
\PMPlatingThk
\tcbline*
\paragraph*{両方に\Outcut があり、ボトム側が基準の場合}
\PMBottomOutcutACWidth
\PMBottomOutcutBDWidth
\PMBottomOutcutConerR
\PMBottomOutcutLength\\
\PMBottomEndACID
\PMBottomOutcutAsideThickness
\PMPlatingThk\\
\PMTopOutcutACWidth
\PMTopOutcutBDWidth
\PMTopOutcutCornerR
\PMTopOutcutLength\\
\PMKeywayPos
\PMKeywayWidth
\PMCenterlineEndFaceDifAC
\tcbline*
\paragraph*{両方に\Outcut があり、トップ側が基準の場合}
\PMTopOutcutACWidth
\PMTopOutcutBDWidth
\PMTopOutcutCornerR\\
\PMTopOutcutLength
\PMKeywayPos
\PMKeywayWidth\\
\PMTopEndACID
\PMTopOutcutAsideThickness
\PMPlatingThk\\
\PMBottomOutcutACWidth
\PMBottomOutcutBDWidth
\PMBottomOutcutConerR
\PMBottomOutcutLength\\
\PMCenterlineEndFaceDifAC
\tcbline*
\paragraph*{\CurvedOutcut の場合}
(以上に加えて)\PMCenterCurvatureRadius
\end{Parameter}
%%%%%%%%%%%%%%%%%%%%%%%%%%%%%%%%%%%%%%%%%%%%%%%%%%%%%%%%%%
%%%%%%%%%%%%%%%%%%%%%%%%%%%%%%%%%%%%%%%%%%%%%%%%%%%%%%%%%%
%%%%%%%%%%%%%%%%%%%%%%%%%%%%%%%%%%%%%%%%%%%%%%%%%%%%%%%%%%

\clearpage
%%%%%%%%%%%%%%%%%%%%%%%%%%%%%%%%%%%%%%%%%%%%%%%%%%%%%%%%%%
%% subsubsection 01.02.03.3 %%%%%%%%%%%%%%%%%%%%%%%%%%%%%%
%%%%%%%%%%%%%%%%%%%%%%%%%%%%%%%%%%%%%%%%%%%%%%%%%%%%%%%%%%
\subsubsection{\KeywayMilling における寸法}
\begin{enumerate}[label=\sarrow]
\item \KeywayCornerType を確認し、必要に応じて加工における径の決定する
\item \expandafterindex{きじゅん(\yomiKeywayCenter)@基準(\nameKeywayCenter)}\nameKeywayCenter の基準を確認し、\KeywayCenter の$X$位置を手動で計算し、決定する
\item \KeywayWidth を確認し、\expandafterindex{かこうかいすう(\yomiKeywayWidth)@加工回数(\nameKeywayWidth)}加工の回数を決定する
\end{enumerate}
%%%%%%%%%%%%%%%%%%%%%%%%%%%%%%%%%%%%%%%%%%%%%%%%%%%%%%%%%%
%% PARAMETER %%%%%%%%%%%%%%%%%%%%%%%%%%%%%%%%%%%%%%%%%%%%%
%%%%%%%%%%%%%%%%%%%%%%%%%%%%%%%%%%%%%%%%%%%%%%%%%%%%%%%%%%
\begin{Parameter}{必要なパラメタ}
\paragraph*{\CenterCurvatureLine が基準の場合}
\PMKeywayACOD
\PMKeywayBDOD
\PMKeywayPos
\PMKeywayWidth
\PMCenterCurvatureRadius\\
\PMKeywayCornerR または\PMKeywayCornerC
\tcbline*
\paragraph*{\OutcutCenter が基準の場合}
\PMKeywayACOD
\PMKeywayBDOD
\PMKeywayPos
\PMKeywayWidth\\
\PMKeywayCornerR または\PMKeywayCornerC
\tcbline*
\paragraph*{\AsideKeywayDepth が基準の場合}
(以上に加えて)\PMAsideKeywayDepth
\end{Parameter}
%%%%%%%%%%%%%%%%%%%%%%%%%%%%%%%%%%%%%%%%%%%%%%%%%%%%%%%%%%
%%%%%%%%%%%%%%%%%%%%%%%%%%%%%%%%%%%%%%%%%%%%%%%%%%%%%%%%%%
%%%%%%%%%%%%%%%%%%%%%%%%%%%%%%%%%%%%%%%%%%%%%%%%%%%%%%%%%%

\clearpage
%%%%%%%%%%%%%%%%%%%%%%%%%%%%%%%%%%%%%%%%%%%%%%%%%%%%%%%%%%
%% subsubsection 01.02.03.4 %%%%%%%%%%%%%%%%%%%%%%%%%%%%%%
%%%%%%%%%%%%%%%%%%%%%%%%%%%%%%%%%%%%%%%%%%%%%%%%%%%%%%%%%%
\subsubsection{\EndFaceChamferMilling における寸法}
\begin{enumerate}[label=\sarrow]
\item \EndFaceOutCChamfer の場合は、\OutcutExists を確認し、加工の径を決定する
\item \expandafterindex{Cめんとり(\yomiOutcut ようこうぐせんたん)@C面取(\nameOutcut 用工具先端)}\nameOutcut 用工具先端部がC面取の場合は、\expandafterindex{\yomiOutcut のけいじょう@\nameOutcut の形状}\nameOutcut の形状を確認し、\expandafterindex{こうぐ(\yomiOutcut)@工具(\nameOutcut)}工具を決定する
\end{enumerate}
%%%%%%%%%%%%%%%%%%%%%%%%%%%%%%%%%%%%%%%%%%%%%%%%%%%%%%%%%%
%% PARAMETER %%%%%%%%%%%%%%%%%%%%%%%%%%%%%%%%%%%%%%%%%%%%%
%%%%%%%%%%%%%%%%%%%%%%%%%%%%%%%%%%%%%%%%%%%%%%%%%%%%%%%%%%
\begin{Parameter}{必要なパラメタ}
\paragraph*{トップ\EndFaceOutCChamfer:\Outcut のない場合}
\PMACOD
\PMBDOD
\PMTopEndFaceOutCChamferLength
\PMODCornerR
\tcbline*
\paragraph*{トップ\EndFaceOutCChamfer:\Outcut のある場合}
\PMTopOutcutACWidth
\PMTopOutcutBDWidth
\PMTopOutcutCornerR
\PMTopEndFaceOutCChamferLength
\tcbline*
\paragraph*{ボトム\EndFaceOutCChamfer:\Outcut のない場合}
\PMACOD
\PMBDOD
\PMBottomEndFaceOutCChamferLength
\PMODCornerR
\tcbline*
\paragraph*{ボトム\EndFaceOutCChamfer:\Outcut のある場合}
\PMBottomOutcutACWidth
\PMBottomOutcutBDWidth
\PMBottomOutcutConerR
\PMBottomEndFaceOutCChamferLength
\tcbline*
\paragraph*{トップ\EndFaceInCChamfer}
\PMTopEndACID
\PMTopEndBDID
\PMTopEndIDCornerR\\
\PMTopEndFaceInCChamferLength
\PMPlatingThk
\tcbline*
\paragraph*{ボトム\EndFaceInCChamfer}
\PMBottomEndACID
\PMBottomEndBDID
\PMBottomEndIDCornerR\\
\PMBottomEndFaceInCChamferLength
\PMPlatingThk
\end{Parameter}
%%%%%%%%%%%%%%%%%%%%%%%%%%%%%%%%%%%%%%%%%%%%%%%%%%%%%%%%%%
%%%%%%%%%%%%%%%%%%%%%%%%%%%%%%%%%%%%%%%%%%%%%%%%%%%%%%%%%%
%%%%%%%%%%%%%%%%%%%%%%%%%%%%%%%%%%%%%%%%%%%%%%%%%%%%%%%%%%

%\clearpage
%%%%%%%%%%%%%%%%%%%%%%%%%%%%%%%%%%%%%%%%%%%%%%%%%%%%%%%%%%
%% subsubsection 01.02.03.5 %%%%%%%%%%%%%%%%%%%%%%%%%%%%%%
%%%%%%%%%%%%%%%%%%%%%%%%%%%%%%%%%%%%%%%%%%%%%%%%%%%%%%%%%%
\subsubsection{\EndFaceBoringMilling における寸法\TBW}
(to be written...)
%%%%%%%%%%%%%%%%%%%%%%%%%%%%%%%%%%%%%%%%%%%%%%%%%%%%%%%%%%
%% PARAMETER %%%%%%%%%%%%%%%%%%%%%%%%%%%%%%%%%%%%%%%%%%%%%
%%%%%%%%%%%%%%%%%%%%%%%%%%%%%%%%%%%%%%%%%%%%%%%%%%%%%%%%%%
\begin{Parameter}{必要なパラメタ}
\PMEndFaceBoringAsideDistance%
\PMEndFaceBoringWidth%
\PMEndFaceBoringDepth%
\PMEndFaceBoringCornerR\\
\PMACOD%
\PMBDOD%
\PMEndFaceBoringLength
\end{Parameter}
%%%%%%%%%%%%%%%%%%%%%%%%%%%%%%%%%%%%%%%%%%%%%%%%%%%%%%%%%%
%%%%%%%%%%%%%%%%%%%%%%%%%%%%%%%%%%%%%%%%%%%%%%%%%%%%%%%%%%
%%%%%%%%%%%%%%%%%%%%%%%%%%%%%%%%%%%%%%%%%%%%%%%%%%%%%%%%%%

%\clearpage
%%%%%%%%%%%%%%%%%%%%%%%%%%%%%%%%%%%%%%%%%%%%%%%%%%%%%%%%%%
%% subsubsection 01.02.03.6 %%%%%%%%%%%%%%%%%%%%%%%%%%%%%%
%%%%%%%%%%%%%%%%%%%%%%%%%%%%%%%%%%%%%%%%%%%%%%%%%%%%%%%%%%
\subsubsection{\IncutBoringMilling における寸法\TBW}
(to be written...)
%%%%%%%%%%%%%%%%%%%%%%%%%%%%%%%%%%%%%%%%%%%%%%%%%%%%%%%%%%
%% PARAMETER %%%%%%%%%%%%%%%%%%%%%%%%%%%%%%%%%%%%%%%%%%%%%
%%%%%%%%%%%%%%%%%%%%%%%%%%%%%%%%%%%%%%%%%%%%%%%%%%%%%%%%%%
\begin{Parameter}{必要なパラメタ}
\PMIncutBoringACWidth%
\PMIncutBoringBDWidth%
\PMIncutBoringCornerR%
\PMIncutBoringLength%
\end{Parameter}
%%%%%%%%%%%%%%%%%%%%%%%%%%%%%%%%%%%%%%%%%%%%%%%%%%%%%%%%%%
%%%%%%%%%%%%%%%%%%%%%%%%%%%%%%%%%%%%%%%%%%%%%%%%%%%%%%%%%%
%%%%%%%%%%%%%%%%%%%%%%%%%%%%%%%%%%%%%%%%%%%%%%%%%%%%%%%%%%


\clearpage
%%%%%%%%%%%%%%%%%%%%%%%%%%%%%%%%%%%%%%%%%%%%%%%%%%%%%%%%%%
%% subsection 01.01.04 %%%%%%%%%%%%%%%%%%%%%%%%%%%%%%%%%%%
%%%%%%%%%%%%%%%%%%%%%%%%%%%%%%%%%%%%%%%%%%%%%%%%%%%%%%%%%%
\subsection{NCプログラムの入力}
\begin{enumerate}[label=\sarrow]
\item 原則として、\index{プログラムばんごう@プログラム番号}プログラム番号は\DrawingExists と一致させる\\
ただし、加工内容が同一のものである場合は、既存のNCプログラムをそのまま流用する
\item 各々の加工部分およびその形状に対する\index{サブプログラム}サブプログラムを決定する
\item 各々のサブプログラムに対し、適切な寸法値を手動で計算する
\item 各々のサブプログラムに対し、計算した寸法値・\index{こうぐばんごう@工具番号}工具番号・\index{おくりはやさ@送り速さ}送り速さ・\index{しゅじくかいてんすう@主軸回転数}主軸回転数を格納する
\item \ReAlocationLength の寸法に応じて、\index{クリアランスへいめん(Zほうこう)@クリアランス平面($Z$方向)}$Z$方向クリアランス平面の位置を決定する
\item マシニングセンタの操作画面にて\index{メインプログラム}メインプログラムを直接編集し、必要なコードまたは数値を記入する
\item 必要に応じて、\index{いちじていし(NCプログラム)@一時停止(NCプログラム)}一時停止用のコードを代入する
\item \TDCorrection または\TLCorrection が必要な場合は、別途専用画面にて手動で編集を行う
\end{enumerate}


%\clearpage
%%%%%%%%%%%%%%%%%%%%%%%%%%%%%%%%%%%%%%%%%%%%%%%%%%%%%%%%%%
%% subsection 01.1.4 %%%%%%%%%%%%%%%%%%%%%%%%%%%%%%%%%%%%%
%%%%%%%%%%%%%%%%%%%%%%%%%%%%%%%%%%%%%%%%%%%%%%%%%%%%%%%%%%
\subsection{\index{ワーク}ワークの設置}
\begin{enumerate}[label=\sarrow]
\item \Spacer が必要な場合は、適切なスペーサを\Jig の\ReceiverPlate に設置する
\item \index{ワーク}ワークの大きさを考慮して、\FixtureBoltLength を目分量で適宜決定し、\Jig に設置する
\item \ReAlocationLength に応じた位置に\index{ワーク}ワークを設置し、固定する
\item トップ側およびボトム側の、\Jig からの\index{はりだしちょう@張出長}張出長を\index{メジャー}メジャーを用いて測定する
\item 測定した\index{はりだしちょう@張出長}張出長から、\EndFacecutMilling における\index{ぜんけずりしろ(\yomiEndFacecut)@全削り代(\nameEndFacecut)}全削り代を手動でおおまかに計算する
\end{enumerate}
%%%%%%%%%%%%%%%%%%%%%%%%%%%%%%%%%%%%%%%%%%%%%%%%%%%%%%%%%%
%% PARAMETER %%%%%%%%%%%%%%%%%%%%%%%%%%%%%%%%%%%%%%%%%%%%%
%%%%%%%%%%%%%%%%%%%%%%%%%%%%%%%%%%%%%%%%%%%%%%%%%%%%%%%%%%
\begin{Parameter}{必要なパラメタ}
\paragraph*{\FixtureBolt}
\PMACOD
\PMBDOD\\
\GPMbox{\nameJig 床面とボルト取付具(上)間の距離}
\GPMbox{\ReceiverPlate とボルト取付具(横)間の距離}
\tcbline*
\paragraph*{\EndFacecut の削り代}
\PMJigLength
\PMTopReAlocationLength
\PMBottomReAlocationLength
\GPMbox{\nameEndFacecutMilling 1回あたりの削り代}\\
\GPMbox{トップ側張出長実測値}
\GPMbox{ボトム側張出長実測値}
\end{Parameter}
%%%%%%%%%%%%%%%%%%%%%%%%%%%%%%%%%%%%%%%%%%%%%%%%%%%%%%%%%%
%%%%%%%%%%%%%%%%%%%%%%%%%%%%%%%%%%%%%%%%%%%%%%%%%%%%%%%%%%
%%%%%%%%%%%%%%%%%%%%%%%%%%%%%%%%%%%%%%%%%%%%%%%%%%%%%%%%%%


\clearpage
%%%%%%%%%%%%%%%%%%%%%%%%%%%%%%%%%%%%%%%%%%%%%%%%%%%%%%%%%%
%% subsection 01.1.5 %%%%%%%%%%%%%%%%%%%%%%%%%%%%%%%%%%%%%
%%%%%%%%%%%%%%%%%%%%%%%%%%%%%%%%%%%%%%%%%%%%%%%%%%%%%%%%%%
\subsection{\index{ワーク}ワーク設置後の調整}
\begin{enumerate}[label=\sarrow]
\item トップ側およびボトム側の\expandafterindex{ぜんけずりしろ(\yomiEndFacecut)@全削り代(\nameEndFacecut)}全削り代に応じて、\index{かこうかいすう(\yomiEndFacecutMilling)@加工回数(\nameEndFacecutMilling)}\nameEndFacecutMilling の回数を設定する
\item \TopODCenter および\BottomODCenter の位置を\index{メジャー}メジャーで測定する
\item 測定した中心位置を用いて、\index{ワークざひょうけいげんてん@ワーク座標系原点}ワーク座標系原点の設定を行う
\item \expandafterindex{\yomiTableCenter(\yomiMMC)@\nameTableCenter(\nameMMC)}\nameTableCenter とのずれを考慮して、\EndFace の$Z$方向の長さを調整する
\item \CenterlineEndFaceDif がある場合\expandafterindex{\yomiTableCenter(\yomiMMC)@\nameTableCenter(\nameMMC)}\nameTableCenter とのずれを考慮して、\OutcutCenter の$X$方向の位置を調整する
\end{enumerate}
これらの設定は、\MMC の操作盤から\index{メインプログラム}メインプログラムを直接手動で編集する形で行われる。



\clearpage
%%%%%%%%%%%%%%%%%%%%%%%%%%%%%%%%%%%%%%%%%%%%%%%%%%%%%%%%%%
%% section 1.2 %%%%%%%%%%%%%%%%%%%%%%%%%%%%%%%%%%%%%%%%%%%
%%%%%%%%%%%%%%%%%%%%%%%%%%%%%%%%%%%%%%%%%%%%%%%%%%%%%%%%%%
\modHeadsection{\expandafterindex{こうてい(\yomiMMC)@工程(\nameMMC)}\nameMMC における工程(加工中)}


%%%%%%%%%%%%%%%%%%%%%%%%%%%%%%%%%%%%%%%%%%%%%%%%%%%%%%%%%%
%% subsection 01.2.1 %%%%%%%%%%%%%%%%%%%%%%%%%%%%%%%%%%%%%
%%%%%%%%%%%%%%%%%%%%%%%%%%%%%%%%%%%%%%%%%%%%%%%%%%%%%%%%%%
\subsection{芯出し測定後}
\begin{enumerate}[label=\sarrow]
\item 各々の\index{ワークざひょうけいげんてん@ワーク座標系原点}ワーク座標系原点の測定後、必要に応じて\index{ワークざひょうけいげんてん@ワーク座標系原点}ワーク座標系原点の設定変更を行う
\item 各々の測定箇所の$Z$位置の変更を、必要に応じて行う
\end{enumerate}
これらの設定は、\MMC の操作盤から\index{メインプログラム}メインプログラムを直接手動で編集する形で行われる。


%%%%%%%%%%%%%%%%%%%%%%%%%%%%%%%%%%%%%%%%%%%%%%%%%%%%%%%%%%
%% subsection 01.2.1 %%%%%%%%%%%%%%%%%%%%%%%%%%%%%%%%%%%%%
%%%%%%%%%%%%%%%%%%%%%%%%%%%%%%%%%%%%%%%%%%%%%%%%%%%%%%%%%%
\subsection{\EndFacecutMilling 中}
\begin{enumerate}[label=\sarrow]
\item 必要に応じて、\expandafterindex{1かいあたりのけずりしろ(\yomiEndFacecut)@1回あたりの削り代(\nameEndFacecut)}1回あたりの削り代を調整する
\end{enumerate}


%%%%%%%%%%%%%%%%%%%%%%%%%%%%%%%%%%%%%%%%%%%%%%%%%%%%%%%%%%
%% subsection 01.2.1 %%%%%%%%%%%%%%%%%%%%%%%%%%%%%%%%%%%%%
%%%%%%%%%%%%%%%%%%%%%%%%%%%%%%%%%%%%%%%%%%%%%%%%%%%%%%%%%%
\subsection{\OutcutMilling 中}
\begin{enumerate}[label=\sarrow]
\item 必要に応じて\expandafterindex{しあげかこう(\yomiOutcut)@仕上げ加工(\nameOutcut)}仕上加工前に\index{いちじていし(NCプログラム)@一時停止(NCプログラム)}一時停止を行い、\OutcutAsideThickness および\OutcutWidth の測定を行う
\item \OutcutAsideThickness を調整する場合は、該当する\expandafterindex{しんだしそくてい(\yomiOutcutCenter)@芯出し測定(\nameOutcutCenter)}芯出し測定部分のパラメタを\index{メインプログラム}メインプログラムから直接手動で編集する
\item \OutcutWidth を調整する場合は、該当する加工部分のパラメタを\MMC の操作盤から\index{メインプログラム}メインプログラムを直接手動で編集する
\item \expandafterindex{かこうかいすう(\yomiOutcut)@加工回数(\nameOutcut)}加工の回数を変更する場合は、該当する加工部分を\MMC の操作盤から\index{メインプログラム}メインプログラムを直接手動で編集する
\end{enumerate}


%%%%%%%%%%%%%%%%%%%%%%%%%%%%%%%%%%%%%%%%%%%%%%%%%%%%%%%%%%
%% subsection 01.2.1 %%%%%%%%%%%%%%%%%%%%%%%%%%%%%%%%%%%%%
%%%%%%%%%%%%%%%%%%%%%%%%%%%%%%%%%%%%%%%%%%%%%%%%%%%%%%%%%%
\subsection{\KeywayMilling 中}
\begin{enumerate}[label=\sarrow]
\item 必要に応じて\expandafterindex{しあげかこう(\yomiKeyway)@仕上げ加工(\nameKeyway)}仕上加工前に\index{いちじていし(NCプログラム)@一時停止(NCプログラム)}一時停止を行い、\AsideKeywayDepth および\KeywayDiameter の測定を行う
\item \AsideKeywayDepth を調整する場合は、該当する\expandafterindex{しんだしそくてい(\yomiKeywayCenter)@芯出し測定(\nameKeywayCenter)}芯出し測定部分のパラメタを\MMC の操作画面から\index{メインプログラム}メインプログラムを直接手動で編集する
\item \KeywayDiameter を調整する場合は、該当する加工部分のパラメタを\MMC の操作画面から\index{メインプログラム}メインプログラムを直接手動で編集する
\item \expandafterindex{かこうかいすう(\yomiKeyway)@加工回数(\nameKeyway)}加工の回数を変更する場合は、該当する加工部分を\MMC の操作画面から\index{メインプログラム}メインプログラムを直接手動で編集する
\item 必要に応じて、\index{ブロックゲージ}ブロックゲージによる\KeywayWidth の測定を行う
\end{enumerate}


%%%%%%%%%%%%%%%%%%%%%%%%%%%%%%%%%%%%%%%%%%%%%%%%%%%%%%%%%%
%% subsection 01.2.1 %%%%%%%%%%%%%%%%%%%%%%%%%%%%%%%%%%%%%
%%%%%%%%%%%%%%%%%%%%%%%%%%%%%%%%%%%%%%%%%%%%%%%%%%%%%%%%%%
\subsection{\EndFaceCChamferMilling 中}
\begin{enumerate}[label=\sarrow]
\item 必要に応じて\expandafterindex{しあげかこう(\yomiEndFaceOutCChamfer)@仕上げ加工(\nameEndFaceOutCChamfer)}仕上加工前に\index{いちじていし(NCプログラム)@一時停止(NCプログラム)}一時停止を行い、\EndFaceOutCChamfer の測定・位置の確認を行う
\item \EndFaceOutCChamfer の位置を調整する場合は、該当する加工部分のパラメタを\MMC の操作画面から\index{メインプログラム}メインプログラムを直接手動で編集する
\item \expandafterindex{かこうかいすう(\yomiEndFaceOutCChamfer)@加工回数(\nameEndFaceOutCChamfer)}加工の回数を変更する場合は、該当する加工部分を\MMC の操作画面からメインプログラムを直接手動で編集する
\end{enumerate}


\clearpage
%%%%%%%%%%%%%%%%%%%%%%%%%%%%%%%%%%%%%%%%%%%%%%%%%%%%%%%%%%
%% subsection 01.2.1 %%%%%%%%%%%%%%%%%%%%%%%%%%%%%%%%%%%%%
%%%%%%%%%%%%%%%%%%%%%%%%%%%%%%%%%%%%%%%%%%%%%%%%%%%%%%%%%%
\subsection{\EndFaceInCChamferMilling 中}
\begin{enumerate}[label=\sarrow]
\item 必要に応じて\expandafterindex{しあげかこう(\yomiEndFaceInCChamfer)@仕上げ加工(\nameEndFaceInCChamfer)}仕上加工前に\index{いちじていし(NCプログラム)@一時停止(NCプログラム)}一時停止を行い、\EndFaceInCChamfer の状態の確認を行う
\item \EndFaceInCChamfer の位置を調整する場合は、該当する加工部分のパラメタを\MMC の操作画面から\index{メインプログラム}メインプログラムを直接手動で編集する
\item \expandafterindex{かこうかいすう(\yomiEndFaceInCChamfer)@加工回数(\nameEndFaceInCChamfer)}加工の回数を変更する場合は、該当する加工部分を\MMC の操作画面から\index{メインプログラム}メインプログラムを直接手動で編集する
\end{enumerate}


%\clearpage
%%%%%%%%%%%%%%%%%%%%%%%%%%%%%%%%%%%%%%%%%%%%%%%%%%%%%%%%%%
%% subsection 01.2.1 %%%%%%%%%%%%%%%%%%%%%%%%%%%%%%%%%%%%%
%%%%%%%%%%%%%%%%%%%%%%%%%%%%%%%%%%%%%%%%%%%%%%%%%%%%%%%%%%
\subsection{\EndFaceBoringMilling 中\TBW}
(to be written...)


%\clearpage
%%%%%%%%%%%%%%%%%%%%%%%%%%%%%%%%%%%%%%%%%%%%%%%%%%%%%%%%%%
%% subsection 01.03.7 %%%%%%%%%%%%%%%%%%%%%%%%%%%%%%%%%%%%
%%%%%%%%%%%%%%%%%%%%%%%%%%%%%%%%%%%%%%%%%%%%%%%%%%%%%%%%%%
\subsection{\IncutBoringMilling 中\TBW}
(to be written...)



\clearpage
%%%%%%%%%%%%%%%%%%%%%%%%%%%%%%%%%%%%%%%%%%%%%%%%%%%%%%%%%%
%% section 01.3 %%%%%%%%%%%%%%%%%%%%%%%%%%%%%%%%%%%%%%%%%%
%%%%%%%%%%%%%%%%%%%%%%%%%%%%%%%%%%%%%%%%%%%%%%%%%%%%%%%%%%
\modHeadsection{\expandafterindex{こうてい(\yomiMMC)@工程(\nameMMC)}\nameMMC における工程(加工後)}


%%%%%%%%%%%%%%%%%%%%%%%%%%%%%%%%%%%%%%%%%%%%%%%%%%%%%%%%%%
%% subsection 01.3.1 %%%%%%%%%%%%%%%%%%%%%%%%%%%%%%%%%%%%%
%%%%%%%%%%%%%%%%%%%%%%%%%%%%%%%%%%%%%%%%%%%%%%%%%%%%%%%%%%
\subsection{\index{ワークのとりはずし@ワークの取外し}ワークの取外し}
\begin{enumerate}[label=\sarrow]
\item 必要に応じて、\FixtureBolt を緩める前に、各種\index{そくていき@測定器}測定器で各々の\index{すんぽう@寸法}寸法を確認する
\item \index{クーラント}クーラントおよび\index{エアーブロー}エアーブローを用いて軽く洗浄を行い、\FixtureBolt を緩めて\index{ワーク}ワークを取り出し、軽く乾拭きを行う
\item \index{リフター}リフターまたは\index{クレーン}クレーンを用いて、\index{めんとりようさぎょうだい@面取用作業台}面取用作業台にワークを移動する
\end{enumerate}


%%%%%%%%%%%%%%%%%%%%%%%%%%%%%%%%%%%%%%%%%%%%%%%%%%%%%%%%%%
%% subsection 01.3.2 %%%%%%%%%%%%%%%%%%%%%%%%%%%%%%%%%%%%%
%%%%%%%%%%%%%%%%%%%%%%%%%%%%%%%%%%%%%%%%%%%%%%%%%%%%%%%%%%
\subsection{外観の確認・寸法の測定}
\begin{enumerate}[label=\sarrow]
\item \index{がいかん(ワーク)@外観(ワーク)}外観に異常がないか確認を行う
\item \index{そくていき@測定器}測定器を用いて\index{すんぽう@寸法}寸法の確認を行う
\item 所定の用紙(\index{きかいかこうすんぽううけいれチェックひょう@機械加工寸法受入チェック表}機械加工寸法受入チェック表)に、指定箇所の\index{こうさ@公差}公差を考慮した寸法値を、手動で計算を行い手動で記入する
\item 必要に応じて、所定の用紙に測定値の記入を行う
\end{enumerate}


%%%%%%%%%%%%%%%%%%%%%%%%%%%%%%%%%%%%%%%%%%%%%%%%%%%%%%%%%%
%% subsection 01.3.3 %%%%%%%%%%%%%%%%%%%%%%%%%%%%%%%%%%%%%
%%%%%%%%%%%%%%%%%%%%%%%%%%%%%%%%%%%%%%%%%%%%%%%%%%%%%%%%%%
\subsection{手動による\EndFaceChamferMilling}
\begin{enumerate}[label=\sarrow]
\item 所定の\index{すんぽう@寸法}寸法の\EndFaceChamfer を、\index{てもちけんまき@手持ち研磨機}手持ち研磨機を用いて手動で行う
\item \index{かえり}かえり等の除去を、\index{やすり}やすりを用いて全体的に手動で行う
\end{enumerate}


%%%%%%%%%%%%%%%%%%%%%%%%%%%%%%%%%%%%%%%%%%%%%%%%%%%%%%%%%%
%% subsection 01.3.4 %%%%%%%%%%%%%%%%%%%%%%%%%%%%%%%%%%%%%
%%%%%%%%%%%%%%%%%%%%%%%%%%%%%%%%%%%%%%%%%%%%%%%%%%%%%%%%%%
\subsection{手動による\index{こくいん@刻印}刻印加工}
\begin{enumerate}[label=\sarrow]
\item \expandafterindex{Cがわにくあつ(\yomiBottomEndFace)@C側肉厚(\nameBottomEndFace)}\nameBottomEndFace のC側肉厚に応じて\index{こくいん@刻印}刻印の大きさを決定する
\item \index{めいさい@明細}明細のによる指定に応じて、\index{こくいん@刻印}刻印の位置を調整する
\item \index{リフター}リフターまたは\index{クレーン}クレーンを用いて、所定の置き場に移動する
\end{enumerate}


%!TEX root = ../RfCPN.tex


\modHeadchapter{イシュー・問題の特定}
先に述べた\expandafterindex{ぎょうむフロー(\yomiMMC)@業務フロー(\nameMMC)}業務フローを通して、\MMC に関する\index{イシュー}イシュー(issue)および\index{もんだい(problem)@問題(problem)}問題(problem)の特定を試みる。



%%%%%%%%%%%%%%%%%%%%%%%%%%%%%%%%%%%%%%%%%%%%%%%%%%%%%%%%%%
%% section 02.01 %%%%%%%%%%%%%%%%%%%%%%%%%%%%%%%%%%%%%%%%%
%%%%%%%%%%%%%%%%%%%%%%%%%%%%%%%%%%%%%%%%%%%%%%%%%%%%%%%%%%
\modHeadsection{安全に関するイシュー・問題}

%%%%%%%%%%%%%%%%%%%%%%%%%%%%%%%%%%%%%%%%%%%%%%%%%%%%%%%%%%
%% Issues %%%%%%%%%%%%%%%%%%%%%%%%%%%%%%%%%%%%%%%%%%%%%%%%
%%%%%%%%%%%%%%%%%%%%%%%%%%%%%%%%%%%%%%%%%%%%%%%%%%%%%%%%%%
\begin{Issues}{作業員の機内の侵入に伴うリスク}
一般に、\index{きないへのしんにゅう@機内への侵入}機内への侵入は、転倒・巻き込まれ等のリスクを伴う
\begin{enumerate}[label=\sarrow]
\item[{\sarrow[red]}]
機内に侵入し直接測定をしないと、ワークの\index{かこうげんてん(がいさんち)@加工原点(概算値)}加工原点の概算値が見出だせない状態にある
\item[{\sarrow[red]}] 機内に侵入しないと、\index{はのこうかん(フェイスミル)@刃の交換(フェイスミル)}フェイスミルの刃の交換ができない状態にある
\item 機内に侵入しないと、内部の掃除ができない状態にある
\end{enumerate}
\end{Issues}
%%%%%%%%%%%%%%%%%%%%%%%%%%%%%%%%%%%%%%%%%%%%%%%%%%%%%%%%%%
%%%%%%%%%%%%%%%%%%%%%%%%%%%%%%%%%%%%%%%%%%%%%%%%%%%%%%%%%%
%%%%%%%%%%%%%%%%%%%%%%%%%%%%%%%%%%%%%%%%%%%%%%%%%%%%%%%%%%
%%%%%%%%%%%%%%%%%%%%%%%%%%%%%%%%%%%%%%%%%%%%%%%%%%%%%%%%%%
%% Issues %%%%%%%%%%%%%%%%%%%%%%%%%%%%%%%%%%%%%%%%%%%%%%%%
%%%%%%%%%%%%%%%%%%%%%%%%%%%%%%%%%%%%%%%%%%%%%%%%%%%%%%%%%%
\begin{Issues}{\index{てもちけんまき@手持ち研磨機}手持ち研磨機の使用に伴うリスク}
一般に、\index{てもちけんまき@手持ち研磨機}手持ち研磨機による加工は、巻き込まれや粉塵の付着・吸引等のリスクを伴う
\begin{enumerate}[label=\sarrow]
\item[{\sarrow[red]}]
寸法の小さな\EndFaceCChamferMilling が\MMC で行われず、\index{てもちけんまき@手持ち研磨機}手持ち研磨機により人手で行われている状態にある
\item[{\sarrow[red]}] \EndFaceCChamferMilling に関する解析的な幾何情報を導出しないまま放置され続けている
\end{enumerate}
\end{Issues}
%%%%%%%%%%%%%%%%%%%%%%%%%%%%%%%%%%%%%%%%%%%%%%%%%%%%%%%%%%
%%%%%%%%%%%%%%%%%%%%%%%%%%%%%%%%%%%%%%%%%%%%%%%%%%%%%%%%%%
%%%%%%%%%%%%%%%%%%%%%%%%%%%%%%%%%%%%%%%%%%%%%%%%%%%%%%%%%%
%%%%%%%%%%%%%%%%%%%%%%%%%%%%%%%%%%%%%%%%%%%%%%%%%%%%%%%%%%
%% Issues %%%%%%%%%%%%%%%%%%%%%%%%%%%%%%%%%%%%%%%%%%%%%%%%
%%%%%%%%%%%%%%%%%%%%%%%%%%%%%%%%%%%%%%%%%%%%%%%%%%%%%%%%%%
\begin{Issues}{柵への衝突に伴うリスク}
\begin{enumerate}[label=\sarrow]
\item 後付けされた\expandafterindex{あんぜんさく(\yomiMMC)@安全柵(\nameMMC)}安全柵(と称されている柵)が、衝突の\index{リスク(しょうとつ)@リスク(衝突)}リスクを生み出している
\item 一部のみ(出入口のみ)を着目し、周囲(柵の周辺)が軽視され、\index{あんぜんたいさく@安全対策}安全対策が機能せず反対に危険リスクを生み出している
\end{enumerate}
\end{Issues}
%%%%%%%%%%%%%%%%%%%%%%%%%%%%%%%%%%%%%%%%%%%%%%%%%%%%%%%%%%
%%%%%%%%%%%%%%%%%%%%%%%%%%%%%%%%%%%%%%%%%%%%%%%%%%%%%%%%%%
%%%%%%%%%%%%%%%%%%%%%%%%%%%%%%%%%%%%%%%%%%%%%%%%%%%%%%%%%%
%%%%%%%%%%%%%%%%%%%%%%%%%%%%%%%%%%%%%%%%%%%%%%%%%%%%%%%%%%
%% Issues %%%%%%%%%%%%%%%%%%%%%%%%%%%%%%%%%%%%%%%%%%%%%%%%
%%%%%%%%%%%%%%%%%%%%%%%%%%%%%%%%%%%%%%%%%%%%%%%%%%%%%%%%%%
\begin{Issues}{\index{リフター}リフターへの衝突に伴うリスク}
\begin{enumerate}[label=\sarrow]
\item ワークの\index{うけいれけんさ(ワーク)@受入検査(ワーク)}受入検査の際に\index{リフター}リフターや\index{クレーン}クレーンが必ず\index{あんぜんつうろ@安全通路}安全通路を通る構造にある
\item 当社の\index{けいえいほうしん(とうしゃ)@経営方針(当社)}経営方針が、\index{じぎょうぶりねん@事業部理念}事業部理念に相反し、安全性より生産性を優先している
\end{enumerate}
\end{Issues}
%%%%%%%%%%%%%%%%%%%%%%%%%%%%%%%%%%%%%%%%%%%%%%%%%%%%%%%%%%
%%%%%%%%%%%%%%%%%%%%%%%%%%%%%%%%%%%%%%%%%%%%%%%%%%%%%%%%%%
%%%%%%%%%%%%%%%%%%%%%%%%%%%%%%%%%%%%%%%%%%%%%%%%%%%%%%%%%%


\clearpage
%%%%%%%%%%%%%%%%%%%%%%%%%%%%%%%%%%%%%%%%%%%%%%%%%%%%%%%%%%
%% section 02.02 %%%%%%%%%%%%%%%%%%%%%%%%%%%%%%%%%%%%%%%%%
%%%%%%%%%%%%%%%%%%%%%%%%%%%%%%%%%%%%%%%%%%%%%%%%%%%%%%%%%%
\modHeadsection{品質に関するイシュー・問題\TBW}
(to be written...)


\clearpage
%%%%%%%%%%%%%%%%%%%%%%%%%%%%%%%%%%%%%%%%%%%%%%%%%%%%%%%%%%
%% section 02.03 %%%%%%%%%%%%%%%%%%%%%%%%%%%%%%%%%%%%%%%%%
%%%%%%%%%%%%%%%%%%%%%%%%%%%%%%%%%%%%%%%%%%%%%%%%%%%%%%%%%%
\modHeadsection{生産効率に関するイシュー・問題\TBW}
(to be written...)



\clearpage
%%%%%%%%%%%%%%%%%%%%%%%%%%%%%%%%%%%%%%%%%%%%%%%%%%%%%%%%%%
%% section 02.04 %%%%%%%%%%%%%%%%%%%%%%%%%%%%%%%%%%%%%%%%%
%%%%%%%%%%%%%%%%%%%%%%%%%%%%%%%%%%%%%%%%%%%%%%%%%%%%%%%%%%
\modHeadsection{根源的な問題点\TBW}
そもそも、「\expandafterindex{\yomiDrawing(モールド)@\nameDrawing(モールド)}\nameDrawing を見て\index{NCプログラム}NCプログラムを作成」するという時点で、明らかに尋常な状態でないことは言うまでもない
%% footnote %%%%%%%%%%%%%%%%%%%%%
\footnote{さらに言及すると、これが問題点(改善可能な点)だということがこれまで一切(スタッフ・管理職を含めて)認識されなかったことが(非常に大きな・根本的な)問題点として挙げられる。
つまり、(\index{せいぞうぎょう@製造業}製造業にも関わらず)当社が\index{ソフトウェアエンジニアリング}ソフトウェアエンジニアリングをあからさまに蔑ろにし続けていることが根本に存在する。
ソフトウェアエンジニアリングに対する意識の低さ、ならびにモラル・マナーの欠如が顕著に露呈している。
このような事態は、この業務に限らず社内のほぼ全ての業務において同様である。

なお、これは客観的な事実を述べたものであるということを、念のために付記しておく。}。
%%%%%%%%%%%%%%%%%%%%%%%%%%%%%%%%%
%%%%%%%%%%%%%%%%%%%%%%%%%%%%%%%%%%%%%%%%%%%%%%%%%%%%%%%%%%
%% Issues %%%%%%%%%%%%%%%%%%%%%%%%%%%%%%%%%%%%%%%%%%%%%%%%
%%%%%%%%%%%%%%%%%%%%%%%%%%%%%%%%%%%%%%%%%%%%%%%%%%%%%%%%%%
\begin{Issues}{NCプログラムの作成}
NCプログラム作成より圧倒的に情報量の多い\expandafterindex{\yomiDrawing のさくせい)@\nameDrawing の作成}\nameDrawing の作成がスタッフまたは管理職によりなされているにも関わらず、NCプログラム作成はその後工程としてなされている
\tcbline*
NCプログラム作成が、現場作業員(一般職)により行われている
\end{Issues}
%%%%%%%%%%%%%%%%%%%%%%%%%%%%%%%%%%%%%%%%%%%%%%%%%%%%%%%%%%
%%%%%%%%%%%%%%%%%%%%%%%%%%%%%%%%%%%%%%%%%%%%%%%%%%%%%%%%%%
%%%%%%%%%%%%%%%%%%%%%%%%%%%%%%%%%%%%%%%%%%%%%%%%%%%%%%%%%%
その\expandafterindex{\yomiDrawing のさくせい)@\nameDrawing の作成}\nameDrawing の作成についても以下のような状態にある。
%%%%%%%%%%%%%%%%%%%%%%%%%%%%%%%%%%%%%%%%%%%%%%%%%%%%%%%%%%
%% Issues %%%%%%%%%%%%%%%%%%%%%%%%%%%%%%%%%%%%%%%%%%%%%%%%
%%%%%%%%%%%%%%%%%%%%%%%%%%%%%%%%%%%%%%%%%%%%%%%%%%%%%%%%%%
\begin{Issues}{\expandafterindex{\yomiDrawing のさくせい)@\nameDrawing の作成}\nameDrawing の作成}
\expandafterindex{\yomiDrawing のさくせい)@\nameDrawing の作成}\nameDrawing の作成が、\index{CADソフトウェア}CADソフトウェアを用いて明細ごとに手動で描かれている
\end{Issues}
%%%%%%%%%%%%%%%%%%%%%%%%%%%%%%%%%%%%%%%%%%%%%%%%%%%%%%%%%%
%%%%%%%%%%%%%%%%%%%%%%%%%%%%%%%%%%%%%%%%%%%%%%%%%%%%%%%%%%
%%%%%%%%%%%%%%%%%%%%%%%%%%%%%%%%%%%%%%%%%%%%%%%%%%%%%%%%%%
これらは単に、モールドに関するデータが管理状態にないことを示している。
%%%%%%%%%%%%%%%%%%%%%%%%%%%%%%%%%%%%%%%%%%%%%%%%%%%%%%%%%%
%% Issues %%%%%%%%%%%%%%%%%%%%%%%%%%%%%%%%%%%%%%%%%%%%%%%%
%%%%%%%%%%%%%%%%%%%%%%%%%%%%%%%%%%%%%%%%%%%%%%%%%%%%%%%%%%
\begin{Issues}{データベースの非存在}
\index{モールド}モールドに関する\index{かんけいデータベース@関係データベース}関係データベース(\index{RDB}RDB)に相当するものが存在しない
\end{Issues}
%%%%%%%%%%%%%%%%%%%%%%%%%%%%%%%%%%%%%%%%%%%%%%%%%%%%%%%%%%
%%%%%%%%%%%%%%%%%%%%%%%%%%%%%%%%%%%%%%%%%%%%%%%%%%%%%%%%%%
%%%%%%%%%%%%%%%%%%%%%%%%%%%%%%%%%%%%%%%%%%%%%%%%%%%%%%%%%%

%!TEX root = ../RPA_for_Creating_Program_Note.tex


\modHeadchapter{改善の余地の特定\TBW}



%%%%%%%%%%%%%%%%%%%%%%%%%%%%%%%%%%%%%%%%%%%%%%%%%%%%%%%%%%
%% section 3.2 %%%%%%%%%%%%%%%%%%%%%%%%%%%%%%%%%%%%%%%%%%%
%%%%%%%%%%%%%%%%%%%%%%%%%%%%%%%%%%%%%%%%%%%%%%%%%%%%%%%%%%
\modHeadsection{新たな技術の導入\TBW}
(to be written...)



%%%%%%%%%%%%%%%%%%%%%%%%%%%%%%%%%%%%%%%%%%%%%%%%%%%%%%%%%%
%% section 3.2 %%%%%%%%%%%%%%%%%%%%%%%%%%%%%%%%%%%%%%%%%%%
%%%%%%%%%%%%%%%%%%%%%%%%%%%%%%%%%%%%%%%%%%%%%%%%%%%%%%%%%%
\modHeadsection{業務プロセスの最適化\TBW}
(to be written...)



%%%%%%%%%%%%%%%%%%%%%%%%%%%%%%%%%%%%%%%%%%%%%%%%%%%%%%%%%%
%% section 3.2 %%%%%%%%%%%%%%%%%%%%%%%%%%%%%%%%%%%%%%%%%%%
%%%%%%%%%%%%%%%%%%%%%%%%%%%%%%%%%%%%%%%%%%%%%%%%%%%%%%%%%%
\modHeadsection{教育・トレーニングの強化}
(to be written...)


\clearrightpage
%%%%%%%%%%%%%%%%%%%%%%%%%%%%%%%%%%%%%%%%%%%%%%%%%%%%%%%%%
%% Appendices %%%%%%%%%%%%%%%%%%%%%%%%%%%%%%%%%%%%%%%%%%%
%%%%%%%%%%%%%%%%%%%%%%%%%%%%%%%%%%%%%%%%%%%%%%%%%%%%%%%%%
\begin{appendices}
%\Appendixpart
\end{appendices}

\addtocontents{toc}{\protect\end{tocBox}}
\clearrightpage
