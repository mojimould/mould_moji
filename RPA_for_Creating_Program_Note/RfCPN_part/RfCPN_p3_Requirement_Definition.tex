%!TEX root = ./RPA_for_Creating_Program_Note.tex


\addtocontents{toc}{\protect\cleardoublepage}
%%%%%%%%%%%%%%%%%%%%%%%%%%%%%%%%%%%%%%%%%%%%%%%%%%%%%%%%%
%% Part Requirement Definition %%%%%%%%%%%%%%%%%%%%%%%%%%
%%%%%%%%%%%%%%%%%%%%%%%%%%%%%%%%%%%%%%%%%%%%%%%%%%%%%%%%%
\addtocontents{toc}{\protect\begin{tocBox}{\tmppartnum}}%
\tPart{要件定義\TBW}{概要}{%
\paragraph*{目標(なにがしたいか?)}
(to be written...)

\tcbline*
\paragraph*{手段(どうやって?)}
(to be written...)

\tcbline*
\paragraph*{背景(なぜ?)}
(to be written...)\\
外注で作成した\index{プログラム(がいちゅう)@プログラム(外注)}プログラムも、当社が具体的な\index{ようけんていぎ@要件定義}要件定義すら行えず、実用に至っていない
%% footnote %%%%%%%%%%%%%%%%%%%%%
\footnote{プログラム自体は尤もな内容であり、レベルも十分なものである。
(第二製造室に関わらず)当社の\index{ソフトウェアエンジニアリング}ソフトウェアエンジニアリングに関する致命的なほどの無関心が顕わに露呈したものであり、自明な帰結である。}。
%%%%%%%%%%%%%%%%%%%%%%%%%%%%%%%%%

 したがって、\index{システムかいはつプロセス@システム開発プロセス}システム開発プロセスの計画の策定を行うことが喫緊の課題である。

\tcbline*
\paragraph*{結論(どうなった?)}
\MMname における業務の流れを把握し、それを基に\DMname の稼働に向けたソフトウェア視点によるシステム開発プロセスの計画を策定した。
}

%%%%%%%%%%%%%%%%%%%%%%%%%%%%%%%%%%%%%%%%%%%%%%%%%%%%%%%%%
%% chapters %%%%%%%%%%%%%%%%%%%%%%%%%%%%%%%%%%%%%%%%%%%%%%
%%%%%%%%%%%%%%%%%%%%%%%%%%%%%%%%%%%%%%%%%%%%%%%%%%%%%%%%%%
%!TEX root = ../RPA_for_Creating_Program_Note.tex


\modHeadchapter{はじめに\TBW}
% 本文書の目的と範囲について説明



%%%%%%%%%%%%%%%%%%%%%%%%%%%%%%%%%%%%%%%%%%%%%%%%%%%%%%%%%%
%% section 8.1 %%%%%%%%%%%%%%%%%%%%%%%%%%%%%%%%%%%%%%%%%%%
%%%%%%%%%%%%%%%%%%%%%%%%%%%%%%%%%%%%%%%%%%%%%%%%%%%%%%%%%%
\modHeadsection{文書の目的\TBW}



%%%%%%%%%%%%%%%%%%%%%%%%%%%%%%%%%%%%%%%%%%%%%%%%%%%%%%%%%%
%% section 8.2 %%%%%%%%%%%%%%%%%%%%%%%%%%%%%%%%%%%%%%%%%%%
%%%%%%%%%%%%%%%%%%%%%%%%%%%%%%%%%%%%%%%%%%%%%%%%%%%%%%%%%%
\modHeadsection{文書の範囲\TBW}


%%%%%%%%%%%%%%%%%%%%%%%%%%%%%%%%%%%%%%%%%%%%%%%%%%%%%%%%%
%% Appendices %%%%%%%%%%%%%%%%%%%%%%%%%%%%%%%%%%%%%%%%%%%
%%%%%%%%%%%%%%%%%%%%%%%%%%%%%%%%%%%%%%%%%%%%%%%%%%%%%%%%%
\begin{appendices}
%\Appendixpart
\end{appendices}

\addtocontents{toc}{\protect\end{tocBox}}
\clearrightpage
