%!TEX root = ./RPA_for_Creating_Program_Note.tex


\addtocontents{toc}{\protect\cleardoublepage}
%%%%%%%%%%%%%%%%%%%%%%%%%%%%%%%%%%%%%%%%%%%%%%%%%%%%%%%%%
%% Part Standards %%%%%%%%%%%%%%%%%%%%%%%%%%%%%%%%%%%%%%%
%%%%%%%%%%%%%%%%%%%%%%%%%%%%%%%%%%%%%%%%%%%%%%%%%%%%%%%%%
\addtocontents{toc}{\protect\begin{tocBox}{\tmppartnum}}%
\tPart[lot,loC]{諸標準の策定}{概要}{%
\paragraph*{目標(なにがしたいか?)}
プログラムの記述に関して一定の基準・規則を設けることで、一貫性および効率性を向上する
\tcbline*
\paragraph*{手段(どうやって?)}
\DMname についての(特にソフトウェアの観点による)\textbf{諸標準の策定}を試みる。
\tcbline*
\paragraph*{背景(なぜ?)}
プログラムの記述には、一般にG-codeが用いられる。
一方で、G-codeの記述に関してはその規格が多岐にわたり統一された標準が存在しない。
そのため、記述に際して社内の標準を参照する必要がある。

しかし、現時点(\DMname 設置時点)において、マシニングセンタに関連した(ソフトウェア視点による)\textbf{社内の標準はほとんど存在しない}
%% footnote %%%%%%%%%%%%%%%%%%%%%
\footnote{つまり、マシニングセンタについては長年にわたり管理業務が事実上放棄されている。
またその皺寄せの大部分が、作業者(一般職)に押し付けられている。}。
%%%%%%%%%%%%%%%%%%%%%%%%%%%%%%%%%

 したがって、プログラムの記述に関する標準の策定が急務である。
これにより、プログラムの記述における一貫性と効率性が向上し、業務の効率化(外注を含む)とコスト削減が期待される。
また標準を設けることにより、G-codeの記述に関する混乱を解消し、より高品質なソフトウェア開発を実現するための重要なステップとなる。
\tcbline*
\paragraph*{結論(どうなった?)}
\DMname 関連のソフトウェア開発における、(ソフトウェアの観点による)寸法・コードの記述法・工具・保守等の標準を策定した。
}

%%%%%%%%%%%%%%%%%%%%%%%%%%%%%%%%%%%%%%%%%%%%%%%%%%%%%%%%%%
%% chapters %%%%%%%%%%%%%%%%%%%%%%%%%%%%%%%%%%%%%%%%%%%%%%
%%%%%%%%%%%%%%%%%%%%%%%%%%%%%%%%%%%%%%%%%%%%%%%%%%%%%%%%%%
%!TEX root = ../RPA_for_Creating_Program_Note.tex


\modHeadchapter[lot]{\DMname の設置環境}
マシニングセンタの精度を保つためには、\index{せっちかんきょう(きかいほんたい)@設置環境(機械本体)}設置環境を整える必要がある。
ここではその目安を与える。



%%%%%%%%%%%%%%%%%%%%%%%%%%%%%%%%%%%%%%%%%%%%%%%%%%%%%%%%%%
%% section 09.1 %%%%%%%%%%%%%%%%%%%%%%%%%%%%%%%%%%%%%%%%%%
%%%%%%%%%%%%%%%%%%%%%%%%%%%%%%%%%%%%%%%%%%%%%%%%%%%%%%%%%%
\modHeadsection{設置箇所における基本事項}
\index{マシニングセンタ}マシニングセンタを設置する場所としては以下のような場所を選ぶものとする。
\begin{enumerate}
\item 日光が直接当たらない
\item 温風や冷風が直接当たらない
\item 基礎の\index{ちたいりょく@地耐力}地耐力が10t/m$^2$以上である
\end{enumerate}



%%%%%%%%%%%%%%%%%%%%%%%%%%%%%%%%%%%%%%%%%%%%%%%%%%%%%%%%%%
%% section 09.2 %%%%%%%%%%%%%%%%%%%%%%%%%%%%%%%%%%%%%%%%%%
%%%%%%%%%%%%%%%%%%%%%%%%%%%%%%%%%%%%%%%%%%%%%%%%%%%%%%%%%%
\modHeadsection{設置の条件}
\DMname についてメーカーが指定する主な\index{せっちじょうけん(きかいほんたい)@設置条件(機械本体)}設置条件は以下のとおりである。
詳細については\expandafterindex{オペレーションマニュアル(\DMname)@オペレーションマニュアル(\DMname)}オペレーションマニュアルを参照されたし。\\

\begin{multicollongtblr}{機械据付要件}{l X[l]}
項目 & 内容\\
\index{しつおん@室温}室温 & 20$\pm$1$^\circ$C\\
室温の温度変化 & 0.5$^\circ$C/day以内\\
温度勾配 & 0.2$^\circ$C/h以内\\
床面より5m間での上下間の室温差 & 0.7$^\circ$C以内\\
基礎床面温度と室温の差 & 0.5$^\circ$C以内\\
\index{しつど@湿度}湿度 & 60$\pm$5\%\\
\index{せっさくゆ@切削油}切削油の温度 & $\pm$2$^\circ$C以内\\
周囲の振動による機械への影響 & 0.1$\mu$m以内\\
機械と天井との間隔 & 1.5m以上\\
床面の\index{へいめんど(とこめん)@平面度(床面)}平面度 & $\pm$5mm(推奨値)\\
\end{multicollongtblr}


%!TEX root = ../RPA_for_Creating_Program_Note.tex


\modHeadchapter{マシニングセンタにおける寸法}
ここでは\index{プログラム}プログラムを記述する際や、\index{ずめん(モールド)@図面(モールド)}図面・\index{3Dモデル(モールド)}3Dモデルの描画をする際に必要となる、\index{すんぽう@寸法}寸法や\index{こうさ@公差}公差等の取り扱いについて触れる。

なお、以降で述べる水平方向とは、端面のAC方向のことを指す。



%%%%%%%%%%%%%%%%%%%%%%%%%%%%%%%%%%%%%%%%%%%%%%%%%%%%%%%%%%
%% section 13.1 %%%%%%%%%%%%%%%%%%%%%%%%%%%%%%%%%%%%%%%%%%
%%%%%%%%%%%%%%%%%%%%%%%%%%%%%%%%%%%%%%%%%%%%%%%%%%%%%%%%%%
\modHeadsection{寸法における基本事項}


%%%%%%%%%%%%%%%%%%%%%%%%%%%%%%%%%%%%%%%%%%%%%%%%%%%%%%%%%%
%% subsection 13.1.1 %%%%%%%%%%%%%%%%%%%%%%%%%%%%%%%%%%%%%
%%%%%%%%%%%%%%%%%%%%%%%%%%%%%%%%%%%%%%%%%%%%%%%%%%%%%%%%%%
\subsection{寸法公差の取扱い}
全般的に、\index{すんぽうこうさ@寸法公差}寸法公差がある場合、\index{+こうさ@$+$公差}$+$公差と\index{-こうさ@$-$公差}$-$公差の中央(算術平均)を見るものとする。
ただし、\index{ないけいテーパひょう@内径テーパ表}内径テーパ表の数値については、この限りではない。
%%%%%%%%%%%%%%%%%%%%%%%%%%%%%%%%%%%%%%%%%%%%%%%%%%%%%%%%%%
%% hosoku %%%%%%%%%%%%%%%%%%%%%%%%%%%%%%%%%%%%%%%%%%%%%%%%
%%%%%%%%%%%%%%%%%%%%%%%%%%%%%%%%%%%%%%%%%%%%%%%%%%%%%%%%%%
\begin{hosoku}
たとえば、$100^{+0.5}_{\phantom -0}$であれば、100.25とみなす。
\end{hosoku}
%%%%%%%%%%%%%%%%%%%%%%%%%%%%%%%%%%%%%%%%%%%%%%%%%%%%%%%%%%
%%%%%%%%%%%%%%%%%%%%%%%%%%%%%%%%%%%%%%%%%%%%%%%%%%%%%%%%%%
%%%%%%%%%%%%%%%%%%%%%%%%%%%%%%%%%%%%%%%%%%%%%%%%%%%%%%%%%%


%%%%%%%%%%%%%%%%%%%%%%%%%%%%%%%%%%%%%%%%%%%%%%%%%%%%%%%%%%
%% subsection 13.1.2 %%%%%%%%%%%%%%%%%%%%%%%%%%%%%%%%%%%%%
%%%%%%%%%%%%%%%%%%%%%%%%%%%%%%%%%%%%%%%%%%%%%%%%%%%%%%%%%%
\subsection{寸法の優先度}
\index{こうさのあるすんぽう@公差のある寸法}公差のある寸法と\index{こうさのないすんぽう@公差のない寸法}公差のない寸法(\index{かっこすんぽう@括弧寸法}括弧寸法含む)とが共存して記載されている場合、公差のある寸法を優先する。

ただし、\index{とっきじこう(すんぽう)@特記事項(寸法)}特記事項等がある場合は、それを優先するものとする。
%%%%%%%%%%%%%%%%%%%%%%%%%%%%%%%%%%%%%%%%%%%%%%%%%%%%%%%%%%
%% hosoku %%%%%%%%%%%%%%%%%%%%%%%%%%%%%%%%%%%%%%%%%%%%%%%%
%%%%%%%%%%%%%%%%%%%%%%%%%%%%%%%%%%%%%%%%%%%%%%%%%%%%%%%%%%
\begin{hosoku}
たとえば、2つの線の寸法がそれぞれ$12^{+0.1}_{\phantom -0}$, 4.05と記述されていて、かつその和に相当する部分の寸法が16と記述されている場合は、16.10とみなす。
\end{hosoku}
%%%%%%%%%%%%%%%%%%%%%%%%%%%%%%%%%%%%%%%%%%%%%%%%%%%%%%%%%%
%%%%%%%%%%%%%%%%%%%%%%%%%%%%%%%%%%%%%%%%%%%%%%%%%%%%%%%%%%
%%%%%%%%%%%%%%%%%%%%%%%%%%%%%%%%%%%%%%%%%%%%%%%%%%%%%%%%%%



\clearpage
%%%%%%%%%%%%%%%%%%%%%%%%%%%%%%%%%%%%%%%%%%%%%%%%%%%%%%%%%%
%% section 10.2 %%%%%%%%%%%%%%%%%%%%%%%%%%%%%%%%%%%%%%%%%%
%%%%%%%%%%%%%%%%%%%%%%%%%%%%%%%%%%%%%%%%%%%%%%%%%%%%%%%%%%
\modHeadsection{全長および振分長に関する寸法}


%%%%%%%%%%%%%%%%%%%%%%%%%%%%%%%%%%%%%%%%%%%%%%%%%%%%%%%%%%
%% subsection 10.2.1 %%%%%%%%%%%%%%%%%%%%%%%%%%%%%%%%%%%%%
%%%%%%%%%%%%%%%%%%%%%%%%%%%%%%%%%%%%%%%%%%%%%%%%%%%%%%%%%%
\subsection{全長と振分長の公差の関係}
\index{ふりわけちょう@振分長}振分長の\index{こうさ(ふりわけちょう)@公差(振分長)}公差については、\index{ぜんちょう(モールド)@全長(モールド)}全長の\index{こうさ(ぜんちょう)@公差(全長)}公差を\index{トップふりわけちょう@トップ振分長}トップ振分長と\index{ボトムふりわけちょう@ボトム振分長}ボトム振分長とで等分配する。
%%%%%%%%%%%%%%%%%%%%%%%%%%%%%%%%%%%%%%%%%%%%%%%%%%%%%%%%%%
%% hosoku %%%%%%%%%%%%%%%%%%%%%%%%%%%%%%%%%%%%%%%%%%%%%%%%
%%%%%%%%%%%%%%%%%%%%%%%%%%%%%%%%%%%%%%%%%%%%%%%%%%%%%%%%%%
\begin{hosoku}
たとえば、全長が$1000^{\phantom +0}_{-1.0}$でトップ振分長が200であれば、全長の公差分$-0.5$を等分配し、それぞれ$-0.25$, $-0.25$とする。
つまり、トップ振分長は199.75, ボトム振分長は799.75とする
%% footnote %%%%%%%%%%%%%%%%%%%%%
\footnote{\index{ふりわけちゅうしん@振分中心}振分中心からのずれとして考えると、振分長に依らず等分配するのが自然、と捉えることができる。}。
%%%%%%%%%%%%%%%%%%%%%%%%%%%%%%%%%
\end{hosoku}
%%%%%%%%%%%%%%%%%%%%%%%%%%%%%%%%%%%%%%%%%%%%%%%%%%%%%%%%%%
%%%%%%%%%%%%%%%%%%%%%%%%%%%%%%%%%%%%%%%%%%%%%%%%%%%%%%%%%%
%%%%%%%%%%%%%%%%%%%%%%%%%%%%%%%%%%%%%%%%%%%%%%%%%%%%%%%%%%


%%%%%%%%%%%%%%%%%%%%%%%%%%%%%%%%%%%%%%%%%%%%%%%%%%%%%%%%%%
%% subsection 10.2.2 %%%%%%%%%%%%%%%%%%%%%%%%%%%%%%%%%%%%%
%%%%%%%%%%%%%%%%%%%%%%%%%%%%%%%%%%%%%%%%%%%%%%%%%%%%%%%%%%
\subsection{振分長が括弧寸法の場合}
片方の振分長が\index{かっこすんぽう@括弧寸法}括弧寸法の場合は、全長の公差をそのまま括弧寸法に割り当てる。
%%%%%%%%%%%%%%%%%%%%%%%%%%%%%%%%%%%%%%%%%%%%%%%%%%%%%%%%%%
%% hosoku %%%%%%%%%%%%%%%%%%%%%%%%%%%%%%%%%%%%%%%%%%%%%%%%
%%%%%%%%%%%%%%%%%%%%%%%%%%%%%%%%%%%%%%%%%%%%%%%%%%%%%%%%%%
\begin{hosoku}
たとえば、全長が$1000^{\phantom +0}_{-1.0}$でトップ振分長が200, ボトム振分長が(800)であれば、トップ振分長は200, ボトム振分長は799.5とする。
\end{hosoku}
%%%%%%%%%%%%%%%%%%%%%%%%%%%%%%%%%%%%%%%%%%%%%%%%%%%%%%%%%%
%%%%%%%%%%%%%%%%%%%%%%%%%%%%%%%%%%%%%%%%%%%%%%%%%%%%%%%%%%
%%%%%%%%%%%%%%%%%%%%%%%%%%%%%%%%%%%%%%%%%%%%%%%%%%%%%%%%%%


%%%%%%%%%%%%%%%%%%%%%%%%%%%%%%%%%%%%%%%%%%%%%%%%%%%%%%%%%%
%% subsection 11.2.3 %%%%%%%%%%%%%%%%%%%%%%%%%%%%%%%%%%%%%
%%%%%%%%%%%%%%%%%%%%%%%%%%%%%%%%%%%%%%%%%%%%%%%%%%%%%%%%%%
\subsection{振分の調整}
\index{ふりわけちょう@振分長}振分長の調整を行う場合は、\index{スペーサ}スペーサまたは\index{テーブルかいてん(ふりわけちょうせい)@テーブル回転(振分調整)}テーブル回転のどちらかを用いて行うものとする。

%%%%%%%%%%%%%%%%%%%%%%%%%%%%%%%%%%%%%%%%%%%%%%%%%%%%%%%%%%
%% subsubsection 01.1.2.3 %%%%%%%%%%%%%%%%%%%%%%%%%%%%%%%%
%%%%%%%%%%%%%%%%%%%%%%%%%%%%%%%%%%%%%%%%%%%%%%%%%%%%%%%%%%
\subsubsection{スペーサによる調整}
スペーサは原則として\index{ジグ}ジグのトップ側の\index{うけいた@受板}受板に設置する。
厚さ$\delta_\mathrm s$の\index{スペーサ}スペーサによる調整を行う場合、もともとのトップ振分長$f_\mathrm T$に対し、\index{トップふりわけちょう@トップ振分長}トップ側の振分長(\index{トップさいふりわけちょう@トップ再振分長}トップ再振分長)$f'_\mathrm T$を以下のように調整する。
\begin{align*}
  f'_\mathrm T
  = f_\mathrm T
    +\sqrt{R_\mathrm i'^2-\frac{\delta_\mathrm s^2+(2\bar l)^2}4}\frac{\delta_\mathrm s}{\sqrt{\delta_\mathrm s^2+(2\bar l)^2}}\qquad
    \left(R_\mathrm i' = R_\mathrm c-\frac{W_x}2-\rho~,~~\bar l = l-\frac\sigma2\right).
\end{align*}
$R_\textrm c$, $W_x$, $l$, $\rho$, $\sigma$はそれぞれ\CenterCurvatureRadius, AC方向の外径, ジグ幅の半分, 受板の半径, 受板の幅を示す。

%%%%%%%%%%%%%%%%%%%%%%%%%%%%%%%%%%%%%%%%%%%%%%%%%%%%%%%%%%
%% subsubsection 01.1.2.3 %%%%%%%%%%%%%%%%%%%%%%%%%%%%%%%%
%%%%%%%%%%%%%%%%%%%%%%%%%%%%%%%%%%%%%%%%%%%%%%%%%%%%%%%%%%
\subsubsection{テーブル回転による調整}
\index{かたむきかく(ふりわけちょうせい)@傾き角(振分調整)}角度$\theta$だけテーブル回転をして調整を行う場合は、もともとの\index{トップふりわけちょう@トップ振分長}トップ振分長$f_\mathrm T$に対し、トップ側の振分長(\index{トップさいふりわけちょう@トップ再振分長}再振分長)$f'_\mathrm T$を以下のように調整する。
\begin{align*}
  f_\mathrm T'
  = f_\mathrm T+\left(\Delta+\sqrt{R_\mathrm i'-\bar l^2}\right)\sin\theta\qquad
    \left(R_\mathrm i' = R_\mathrm c-\frac{W_x}2-\rho~,~~\bar l = l-\frac\sigma2\right).
\end{align*}
$R_\textrm c$, $W_x$, $l$, $\rho$, $\sigma$, $\Delta$はそれぞれ\CenterCurvatureRadius, AC方向の外径, ジグ幅の半分, 受板の半径, 受板の幅, 受板中心とテーブル中心との水平距離を示す。



\clearpage
%%%%%%%%%%%%%%%%%%%%%%%%%%%%%%%%%%%%%%%%%%%%%%%%%%%%%%%%%%
%% section 13.3 %%%%%%%%%%%%%%%%%%%%%%%%%%%%%%%%%%%%%%%%%%
%%%%%%%%%%%%%%%%%%%%%%%%%%%%%%%%%%%%%%%%%%%%%%%%%%%%%%%%%%
\modHeadsection{外径に関する寸法}
\nameCenterCurvatureRadius を$R_\mathrm c$, トップ振分長を$f_\mathrm T$, 外径を$W_x$とすると、\nameTopEndFace 部の水平方向の長さ$W_\mathrm T$は以下で与えられる。(\nameBottomEndFace 部も同様)
\begin{align*}
  W_\mathrm T
  = \sqrt{\left(R_\mathrm c+\frac{W_x}2\right)^2-f_\mathrm T^2}
    -\sqrt{\left(R_\mathrm c-\frac{W_x}2\right)^2-f_\mathrm T^2}\ .
\end{align*}
なお、$(\nicefrac{f_\mathrm T}{R_\mathrm c})^2$が十分小さい場合は、$W_\mathrm T$は
\begin{align*}
  W_\mathrm T \simeq W_x\left(1+\frac{f_\mathrm T^2}{2R^2}\right)
\end{align*}
とみなしてもよいものとする。



%%%%%%%%%%%%%%%%%%%%%%%%%%%%%%%%%%%%%%%%%%%%%%%%%%%%%%%%%%
%% section 10.4 %%%%%%%%%%%%%%%%%%%%%%%%%%%%%%%%%%%%%%%%%%
%%%%%%%%%%%%%%%%%%%%%%%%%%%%%%%%%%%%%%%%%%%%%%%%%%%%%%%%%%
\modHeadsection{内径に関する寸法}

%%%%%%%%%%%%%%%%%%%%%%%%%%%%%%%%%%%%%%%%%%%%%%%%%%%%%%%%%%
%% subsection 04.4.1 %%%%%%%%%%%%%%%%%%%%%%%%%%%%%%%%%%%%%
%%%%%%%%%%%%%%%%%%%%%%%%%%%%%%%%%%%%%%%%%%%%%%%%%%%%%%%%%%
\subsection{内径テーパ表の公差}
\index{ないけいテーパひょう@内径テーパ表}内径テーパ表を参照する際は、\index{ぜんちょう@全長}全長の\index{こうさ@公差}公差は考慮しないものとする。
また、トップ端からの距離の\index{ピッチ(ないけいテーパひょう)@ピッチ(内径テーパ表)}ピッチも、同様に公差は考慮しないものとする。
%%%%%%%%%%%%%%%%%%%%%%%%%%%%%%%%%%%%%%%%%%%%%%%%%%%%%%%%%%
%% hosoku %%%%%%%%%%%%%%%%%%%%%%%%%%%%%%%%%%%%%%%%%%%%%%%%
%%%%%%%%%%%%%%%%%%%%%%%%%%%%%%%%%%%%%%%%%%%%%%%%%%%%%%%%%%
\begin{hosoku}
たとえば、全長が$800^{+0.5}_{\phantom -0}$, トップ振分長が400, ピッチが25である場合を考える。
このとき、トップ端は\index{ふりわけちゅうしん@振分中心}振分中心から400の位置にあり、ピッチは25であるものとし、両端についてはそれを適宜延長して調整する。
\end{hosoku}
%%%%%%%%%%%%%%%%%%%%%%%%%%%%%%%%%%%%%%%%%%%%%%%%%%%%%%%%%%
%%%%%%%%%%%%%%%%%%%%%%%%%%%%%%%%%%%%%%%%%%%%%%%%%%%%%%%%%%
%%%%%%%%%%%%%%%%%%%%%%%%%%%%%%%%%%%%%%%%%%%%%%%%%%%%%%%%%%

%%%%%%%%%%%%%%%%%%%%%%%%%%%%%%%%%%%%%%%%%%%%%%%%%%%%%%%%%%
%% subsection 04.4.1 %%%%%%%%%%%%%%%%%%%%%%%%%%%%%%%%%%%%%
%%%%%%%%%%%%%%%%%%%%%%%%%%%%%%%%%%%%%%%%%%%%%%%%%%%%%%%%%%
\subsection{内径テーパ表にない内径}
内径テーパ表におけるトップ端からの距離$\lambda_i$ ($i = 0, 1, 2, \cdots$), それに対するAC方向の内径$w_{\mathrm Ai}$に対し、トップ端から$\lambda$の位置にある\index{ないけい(ACほうこう)@内径(AC方向)}内径$w_{\mathrm A\lambda}$は、
\begin{align*}
  w_{\mathrm A\lambda}
  = \frac{(\lambda-\lambda_i)w_{\mathrm Ai+1}+(\lambda_{i+1}-\lambda)w_{\mathrm Ai}}{\lambda_{i+1}-\lambda_i}
  \qquad
  \Big(\lambda_i \leq \lambda < \lambda_{i+1}\Big)
\end{align*}
とみなしてもよいものとする。
\index{ないけい(BDほうこう)@内径(BD方向)}BD方向の内径$w_{\mathrm B\lambda}$についても同様である。

%%%%%%%%%%%%%%%%%%%%%%%%%%%%%%%%%%%%%%%%%%%%%%%%%%%%%%%%%%
%% subsection 11.4.3 %%%%%%%%%%%%%%%%%%%%%%%%%%%%%%%%%%%%%
%%%%%%%%%%%%%%%%%%%%%%%%%%%%%%%%%%%%%%%%%%%%%%%%%%%%%%%%%%
\subsection{水平方向の内径}
中心湾曲線上のトップ端から$\lambda$の位置における水平方向の\index{ないけい(すいへいほうこう)@内径(水平方向)}内径は、トップ端から$\lambda$の位置におけるAC方向の内径$w_{A\lambda}$とみなしてもよいものとする。

%%%%%%%%%%%%%%%%%%%%%%%%%%%%%%%%%%%%%%%%%%%%%%%%%%%%%%%%%%
%% subsection 11.4.4 %%%%%%%%%%%%%%%%%%%%%%%%%%%%%%%%%%%%%
%%%%%%%%%%%%%%%%%%%%%%%%%%%%%%%%%%%%%%%%%%%%%%%%%%%%%%%%%%
\subsection{\PlatingThk の考慮}
後工程にて内面に\index{めっき}めっきを施す場合は、\PlatingThk$\mu$を考慮した上で内径の調整を行うものとする。
このとき、AC方向の内径を
\begin{align*}
  w_{\mathrm A\lambda}' = w_{\mathrm A\lambda}+2\mu
\end{align*}
とみなすものとする。
BD方向の内径$w_{\mathrm B\lambda}'$についても同様である。



\clearpage
%%%%%%%%%%%%%%%%%%%%%%%%%%%%%%%%%%%%%%%%%%%%%%%%%%%%%%%%%%
%% section 13.5 %%%%%%%%%%%%%%%%%%%%%%%%%%%%%%%%%%%%%%%%%%
%%%%%%%%%%%%%%%%%%%%%%%%%%%%%%%%%%%%%%%%%%%%%%%%%%%%%%%%%%
\modHeadsection{端面加工に関する寸法}


%%%%%%%%%%%%%%%%%%%%%%%%%%%%%%%%%%%%%%%%%%%%%%%%%%%%%%%%%%
%% subsection 10.5.1 %%%%%%%%%%%%%%%%%%%%%%%%%%%%%%%%%%%%%
%%%%%%%%%%%%%%%%%%%%%%%%%%%%%%%%%%%%%%%%%%%%%%%%%%%%%%%%%%
\subsection{端面加工の基準点}
\index{たんめんかこう@端面加工}端面加工の\index{きじゅん(たんめんかこう)@基準(端面加工)}基準は、\index{ないけいちゅうしん(たんめん)@内径中心(端面)}端面における内径中心を基準として行うものとする。


%%%%%%%%%%%%%%%%%%%%%%%%%%%%%%%%%%%%%%%%%%%%%%%%%%%%%%%%%%
%% subsection 11.5.1 %%%%%%%%%%%%%%%%%%%%%%%%%%%%%%%%%%%%%
%%%%%%%%%%%%%%%%%%%%%%%%%%%%%%%%%%%%%%%%%%%%%%%%%%%%%%%%%%
\subsection{工具補正:端面加工}

%%%%%%%%%%%%%%%%%%%%%%%%%%%%%%%%%%%%%%%%%%%%%%%%%%%%%%%%%%
%% subsubsection 10.7.3.1 %%%%%%%%%%%%%%%%%%%%%%%%%%%%%%%%
%%%%%%%%%%%%%%%%%%%%%%%%%%%%%%%%%%%%%%%%%%%%%%%%%%%%%%%%%%
\subsubsection{工具長補正:端面加工}
端面加工に使用する\index{フェイスミル}フェイスミルの\index{こうぐちょう(フェイスミル)@工具長(フェイスミル)}工具長は、その刃の\index{せんたんぶ(フェイスミル)@先端部(フェイスミル)}先端部を工具長として\index{オフセット(こうぐちょうほせい)@オフセット(工具長補正)}オフセット量の設定を行うものとする。

なお、このとき工具長の\index{まもうりょう(こうぐちょうほせい)@摩耗量(工具長補正)}摩耗量は0とする。

%%%%%%%%%%%%%%%%%%%%%%%%%%%%%%%%%%%%%%%%%%%%%%%%%%%%%%%%%%
%% subsubsection 10.7.3.1 %%%%%%%%%%%%%%%%%%%%%%%%%%%%%%%%
%%%%%%%%%%%%%%%%%%%%%%%%%%%%%%%%%%%%%%%%%%%%%%%%%%%%%%%%%%
\subsubsection{工具径補正:端面加工}
端面加工に使用するフェイスミルの\index{こうぐけい(フェイスミル)@工具径(フェイスミル)}工具長は、その刃の径方向の先端部を工具径として\index{オフセット(こうぐけいほせい)@オフセット(工具径補正)}オフセット量の設定を行うものとする。

なお、このとき工具径の\index{まもうりょう(こうぐけいほせい)@摩耗量(工具径補正)}摩耗量は0とする。


%%%%%%%%%%%%%%%%%%%%%%%%%%%%%%%%%%%%%%%%%%%%%%%%%%%%%%%%%%
%% subsection 11.5.1 %%%%%%%%%%%%%%%%%%%%%%%%%%%%%%%%%%%%%
%%%%%%%%%%%%%%%%%%%%%%%%%%%%%%%%%%%%%%%%%%%%%%%%%%%%%%%%%%
\subsection{端面加工のコーナーR}
端面加工の際の\index{コーナーR(たんめんかこう)@コーナーR(端面加工)}コーナーRは、\index{ないけいコーナーR(たんめん)@内径コーナーR(端面)}端面における内径のコーナーRとする。



\clearpage
%%%%%%%%%%%%%%%%%%%%%%%%%%%%%%%%%%%%%%%%%%%%%%%%%%%%%%%%%%
%% section 10.06 %%%%%%%%%%%%%%%%%%%%%%%%%%%%%%%%%%%%%%%%%
%%%%%%%%%%%%%%%%%%%%%%%%%%%%%%%%%%%%%%%%%%%%%%%%%%%%%%%%%%
\modHeadsection{外削加工に関する寸法}


%%%%%%%%%%%%%%%%%%%%%%%%%%%%%%%%%%%%%%%%%%%%%%%%%%%%%%%%%%
%% subsection 10.5.1 %%%%%%%%%%%%%%%%%%%%%%%%%%%%%%%%%%%%%
%%%%%%%%%%%%%%%%%%%%%%%%%%%%%%%%%%%%%%%%%%%%%%%%%%%%%%%%%%
\subsection{外削加工の基準点}
\index{がいさくかこう@外削加工}外削加工の\index{きじゅん(がいさくかこう)@基準(外削加工)}基準は、\index{がいさくちゅうしん@外削中心}外削中心を基準として行うものとする。


%%%%%%%%%%%%%%%%%%%%%%%%%%%%%%%%%%%%%%%%%%%%%%%%%%%%%%%%%%
%% subsection 10.06.1 %%%%%%%%%%%%%%%%%%%%%%%%%%%%%%%%%%%%
%%%%%%%%%%%%%%%%%%%%%%%%%%%%%%%%%%%%%%%%%%%%%%%%%%%%%%%%%%
\subsection{工具補正:外削加工}


%%%%%%%%%%%%%%%%%%%%%%%%%%%%%%%%%%%%%%%%%%%%%%%%%%%%%%%%%%
%% subsubsection 10.06.1.1 %%%%%%%%%%%%%%%%%%%%%%%%%%%%%%%
%%%%%%%%%%%%%%%%%%%%%%%%%%%%%%%%%%%%%%%%%%%%%%%%%%%%%%%%%%
\subsubsection{工具長補正:外削加工}
外削加工に使用する\index{スクエアエンドミル}スクエアエンドミルの\index{こうぐちょう(スクエアエンドミル)@工具長(スクエアエンドミル)}工具長は、その刃の\index{せんたんぶ(スクエアエンドミル)@先端部(スクエアエンドミル)}先端部を工具長として\index{オフセット(こうぐちょうほせい)@オフセット(工具長補正)}オフセット量の設定を行うものとする。

なお、このとき工具長の\index{まもうりょう(こうぐちょうほせい)@摩耗量(工具長補正)}摩耗量は0とする。


%%%%%%%%%%%%%%%%%%%%%%%%%%%%%%%%%%%%%%%%%%%%%%%%%%%%%%%%%%
%% subsubsection 10.06.1.2 %%%%%%%%%%%%%%%%%%%%%%%%%%%%%%%
%%%%%%%%%%%%%%%%%%%%%%%%%%%%%%%%%%%%%%%%%%%%%%%%%%%%%%%%%%
\subsubsection{工具径補正:外削加工}
外削加工に使用するスクエアエンドミルの\index{こうぐけい(スクエアエンドミル)@工具径(スクエアエンドミル)}工具長は、その刃の径方向の先端部を工具径として\index{オフセット(こうぐけいほせい)@オフセット(工具径補正)}オフセット量の設定を行うものとする。

なお、このとき工具径の\index{まもうりょう(こうぐけいほせい)@摩耗量(工具径補正)}摩耗量は0とする。


%%%%%%%%%%%%%%%%%%%%%%%%%%%%%%%%%%%%%%%%%%%%%%%%%%%%%%%%%%
%% subsection 10.06.2 %%%%%%%%%%%%%%%%%%%%%%%%%%%%%%%%%%%%
%%%%%%%%%%%%%%%%%%%%%%%%%%%%%%%%%%%%%%%%%%%%%%%%%%%%%%%%%%
\subsection{\OutcutLength}
\OutcutLength の\expandafterindex{すんぽう(\yomiOutcutLength)@寸法(\nameOutcutLength)}寸法は、\index{たんめん@端面}端面に垂直な方向の値とする。
また\TopOutcutLength については、\KeywayWidth の\expandafterindex{すんぽう(\yomiKeywayWidth)@寸法(\nameKeywayWidth)}寸法も含むものとする。
このとき、\TopOutcutLength$h_\mathrm T$が、\KeywayPos$\kappa_p$と\KeywayWidth$\kappa_w$の和に等しい場合は、
\begin{align*}
  h_\mathrm T = \kappa_p+1[\text{mm}]
\end{align*}
とみなして加工を行うものとする。


%%%%%%%%%%%%%%%%%%%%%%%%%%%%%%%%%%%%%%%%%%%%%%%%%%%%%%%%%%
%% subsection 10.06.3 %%%%%%%%%%%%%%%%%%%%%%%%%%%%%%%%%%%%
%%%%%%%%%%%%%%%%%%%%%%%%%%%%%%%%%%%%%%%%%%%%%%%%%%%%%%%%%%
\subsection{湾曲に沿った外削\TBW}
(to be written ...)


\clearpage
%%%%%%%%%%%%%%%%%%%%%%%%%%%%%%%%%%%%%%%%%%%%%%%%%%%%%%%%%%
%% section 11.6 %%%%%%%%%%%%%%%%%%%%%%%%%%%%%%%%%%%%%%%%%%
%%%%%%%%%%%%%%%%%%%%%%%%%%%%%%%%%%%%%%%%%%%%%%%%%%%%%%%%%%
\modHeadsection{\Keyway 加工に関する寸法}


%%%%%%%%%%%%%%%%%%%%%%%%%%%%%%%%%%%%%%%%%%%%%%%%%%%%%%%%%%
%% subsection 10.7.1 %%%%%%%%%%%%%%%%%%%%%%%%%%%%%%%%%%%%%
%%%%%%%%%%%%%%%%%%%%%%%%%%%%%%%%%%%%%%%%%%%%%%%%%%%%%%%%%%
\subsection{\Keyway 加工の基準点\TBW}
(to be written ...)


%%%%%%%%%%%%%%%%%%%%%%%%%%%%%%%%%%%%%%%%%%%%%%%%%%%%%%%%%%
%% subsection 10.06.1 %%%%%%%%%%%%%%%%%%%%%%%%%%%%%%%%%%%%
%%%%%%%%%%%%%%%%%%%%%%%%%%%%%%%%%%%%%%%%%%%%%%%%%%%%%%%%%%
\subsection{工具補正:\Keyway 加工}


%%%%%%%%%%%%%%%%%%%%%%%%%%%%%%%%%%%%%%%%%%%%%%%%%%%%%%%%%%
%% subsubsection 10.06.1.1 %%%%%%%%%%%%%%%%%%%%%%%%%%%%%%%
%%%%%%%%%%%%%%%%%%%%%%%%%%%%%%%%%%%%%%%%%%%%%%%%%%%%%%%%%%
\subsubsection{工具長補正:\Keyway 加工}
\expandafterindex{\yomiKeyway かこう@\nameKeyway 加工}\Keyway 加工に使用する\index{サイドカッター}サイドカッターの\index{こうぐちょう(サイドカッター)@工具長(サイドカッター)}工具長は、その切削部(刃の部分)の\index{せんたんぶ(サイドカッター)@先端部(サイドカッター)}先端部を工具長として\index{オフセット(こうぐちょうほせい)@オフセット(工具長補正)}オフセット量の設定を行うものとする。

なお、このとき工具長の\index{まもうりょう(こうぐちょうほせい)@摩耗量(工具長補正)}摩耗量は0とする。


%%%%%%%%%%%%%%%%%%%%%%%%%%%%%%%%%%%%%%%%%%%%%%%%%%%%%%%%%%
%% subsubsection 10.06.1.2 %%%%%%%%%%%%%%%%%%%%%%%%%%%%%%%
%%%%%%%%%%%%%%%%%%%%%%%%%%%%%%%%%%%%%%%%%%%%%%%%%%%%%%%%%%
\subsubsection{工具径補正:\Keyway 加工}
\Keyway 加工に使用するサイドカッターの\index{こうぐけい(サイドカッター)@工具径(サイドカッター)}工具長は、その刃の径方向の先端部を工具径として\index{オフセット(こうぐけいほせい)@オフセット(工具径補正)}オフセット量の設定を行うものとする。

なお、このとき工具径の\index{まもうりょう(こうぐけいほせい)@摩耗量(工具径補正)}摩耗量は0とする。


%%%%%%%%%%%%%%%%%%%%%%%%%%%%%%%%%%%%%%%%%%%%%%%%%%%%%%%%%%
%% subsection 11.6.2 %%%%%%%%%%%%%%%%%%%%%%%%%%%%%%%%%%%%%
%%%%%%%%%%%%%%%%%%%%%%%%%%%%%%%%%%%%%%%%%%%%%%%%%%%%%%%%%%
\subsection{\KeywayPos および\KeywayWidth}
トップ端から垂直方向に、\Keyway のトップ側の端までの距離を\KeywayPos とする。
また、同様の方向に、\Keyway のトップ側の端から\Keyway のボトム側の端までの距離を\KeywayWidth とする。


%%%%%%%%%%%%%%%%%%%%%%%%%%%%%%%%%%%%%%%%%%%%%%%%%%%%%%%%%%
%% subsection 11.6.2 %%%%%%%%%%%%%%%%%%%%%%%%%%%%%%%%%%%%%
%%%%%%%%%%%%%%%%%%%%%%%%%%%%%%%%%%%%%%%%%%%%%%%%%%%%%%%%%%
\subsection{\KeywayDepth}
トップ側に外削がなく、かつ\AsideKeywayDepth が\index{こうさ@公差}公差のある寸法$\kappa_d'$を持つ場合、\KeywayCenter における\Keyway A側面とA側外面との距離$\kappa_d$は以下のものとみなす。
\begin{gather*}
  \kappa_d
  = \frac{2\kappa_d'-\kappa_w\sin\zeta}{1+\cos^2\zeta}\cos\zeta
    +\sqrt{R_\mathrm o^2-\left(f_\mathrm T-\kappa_p-\frac{\kappa_w}2\right)^2}
    -\sqrt{R_\mathrm o^2-\left(f_\mathrm T-\kappa_p\right)^2}\\[3pt]
  \left(
  \tan\zeta
  = \frac{\sqrt{R_\mathrm o^2-\left(f_\mathrm T-\kappa_p-\kappa_w\right)^2}
          -\sqrt{R_\mathrm o^2-\left(f_\mathrm T-\kappa_p\right)^2}}
         {\kappa_w}
    ~~, \quad
    R_\mathrm o = R_\mathrm c+\frac{W_x}2
  \right).
\end{gather*}
$R_\mathrm c$, $W_x$, $f_\mathrm T$, $\kappa_p$, $\kappa_w$はそれぞれ\CenterCurvatureRadius, 外径, トップ振分長, \KeywayPos, \KeywayWidth を示す。

なお、$(\nicefrac{f_\mathrm T}{R_\mathrm c})^2$が十分小さい場合は、
\begin{align*}
  \kappa_d \simeq \kappa_d'+\frac{\kappa_w^2}{8R_\mathrm o}
\end{align*}
とみなしてもよいものとする。



\clearpage
%%%%%%%%%%%%%%%%%%%%%%%%%%%%%%%%%%%%%%%%%%%%%%%%%%%%%%%%%%
%% section 12.7 %%%%%%%%%%%%%%%%%%%%%%%%%%%%%%%%%%%%%%%%%%
%%%%%%%%%%%%%%%%%%%%%%%%%%%%%%%%%%%%%%%%%%%%%%%%%%%%%%%%%%
\modHeadsection{端面C面取に関する寸法}


%%%%%%%%%%%%%%%%%%%%%%%%%%%%%%%%%%%%%%%%%%%%%%%%%%%%%%%%%%
%% subsection 10.8.1 %%%%%%%%%%%%%%%%%%%%%%%%%%%%%%%%%%%%%
%%%%%%%%%%%%%%%%%%%%%%%%%%%%%%%%%%%%%%%%%%%%%%%%%%%%%%%%%%
\subsection{端面C面取加工の基準点}

%%%%%%%%%%%%%%%%%%%%%%%%%%%%%%%%%%%%%%%%%%%%%%%%%%%%%%%%%%
%% subsubsection 10.8.1.1 %%%%%%%%%%%%%%%%%%%%%%%%%%%%%%%%
%%%%%%%%%%%%%%%%%%%%%%%%%%%%%%%%%%%%%%%%%%%%%%%%%%%%%%%%%%
\subsubsection{端面外C面取加工の基準点}
\index{たんめんそとCめんとりかこう@端面外C面取加工}端面外C面取加工の\index{きじゅん(たんめんそとCめんとりかこう)@基準(端面外C面取加工)}基準は、外削のある場合は\index{がいさくちゅうしん@外削中心}外削中心を基準とし、外削のない場合は\index{がいけいちゅうしん(たんめん)@外径中心(端面)}端面における外径中心を基準として行うものとする。

%%%%%%%%%%%%%%%%%%%%%%%%%%%%%%%%%%%%%%%%%%%%%%%%%%%%%%%%%%
%% subsubsection 10.8.1.1 %%%%%%%%%%%%%%%%%%%%%%%%%%%%%%%%
%%%%%%%%%%%%%%%%%%%%%%%%%%%%%%%%%%%%%%%%%%%%%%%%%%%%%%%%%%
\subsubsection{端面内C面取加工の基準点}
\index{たんめんうちCめんとりかこう@端面内C面取加工}端面内C面取加工の\index{きじゅん(たんめんうちCめんとりかこう)@基準(端面内C面取加工)}基準は、\index{ないけいちゅうしん(たんめん)@内径中心(端面)}端面における内径中心を基準として行うものとする。


%%%%%%%%%%%%%%%%%%%%%%%%%%%%%%%%%%%%%%%%%%%%%%%%%%%%%%%%%%
%% subsection 11.6.2 %%%%%%%%%%%%%%%%%%%%%%%%%%%%%%%%%%%%%
%%%%%%%%%%%%%%%%%%%%%%%%%%%%%%%%%%%%%%%%%%%%%%%%%%%%%%%%%%
\subsection{端面C面取の寸法}
\index{たんめんそとCめんとり@端面外C面取}端面外C面取の寸法$c_\mathrm o$ならびに\index{たんめんうちCめんとり@端面内C面取}端面内C面取の寸法$c_\mathrm i$は、\index{たんめん@端面}端面に垂直な方向の距離とみなす。
このとき、\index{かたかく(テーパエンドミル)@片角(テーパエンドミル)}片角が$\xi_\mathrm e$の\index{テーパエンドミル}テーパエンドミルに対して、\index{Cめんとり(たんめん)@C面取(端面)}C面取の$XY$方向の寸法は、$c_\mathrm o\tan\xi_\mathrm e$および$c_\mathrm i\tan\xi_\mathrm e$で与えられる。


%%%%%%%%%%%%%%%%%%%%%%%%%%%%%%%%%%%%%%%%%%%%%%%%%%%%%%%%%%
%% subsection 11.6.2 %%%%%%%%%%%%%%%%%%%%%%%%%%%%%%%%%%%%%
%%%%%%%%%%%%%%%%%%%%%%%%%%%%%%%%%%%%%%%%%%%%%%%%%%%%%%%%%%
\subsection{工具補正}

%%%%%%%%%%%%%%%%%%%%%%%%%%%%%%%%%%%%%%%%%%%%%%%%%%%%%%%%%%
%% subsubsection 10.7.3.1 %%%%%%%%%%%%%%%%%%%%%%%%%%%%%%%%
%%%%%%%%%%%%%%%%%%%%%%%%%%%%%%%%%%%%%%%%%%%%%%%%%%%%%%%%%%
\subsubsection{工具長補正}
\index{Cめんとり@C面取}C面取に使用する\index{テーパエンドミル}テーパエンドミルの\index{こうぐちょう(テーパエンドミル)@工具長(テーパエンドミル)}工具長は、その\index{せんたんぶ(テーパエンドミル)@先端部(テーパエンドミル)}先端部から軸方向に適当な距離$d_\mathrm e$を差し引いた長さを工具長として\index{オフセット(こうぐちょうほせい)@オフセット(工具長補正)}オフセット量の設定を行うものとする。

したがって、工具の先端部を測定し、先端から$d_\mathrm e$を引いた値を\index{こうぐちょうほせいち@工具長補正値}工具長補正値とする。
なお、このとき工具長の\index{まもうりょう(こうぐちょうほせい)@摩耗量(工具長補正)}摩耗量は0とする。

%%%%%%%%%%%%%%%%%%%%%%%%%%%%%%%%%%%%%%%%%%%%%%%%%%%%%%%%%%
%% subsubsection 10.7.3.1 %%%%%%%%%%%%%%%%%%%%%%%%%%%%%%%%
%%%%%%%%%%%%%%%%%%%%%%%%%%%%%%%%%%%%%%%%%%%%%%%%%%%%%%%%%%
\subsubsection{工具径補正}
\index{せんたんけい(テーパエンドミル)@先端径(テーパエンドミル)}先端径(直径)および先端の\index{かたかく(テーパエンドミル)@片角(テーパエンドミル)}片角がそれぞれ$D_\mathrm e$, $\xi_\mathrm e$の\index{テーパエンドミル}テーパエンドミルに対し、その\index{こうぐけいほせいち@工具径補正値}工具径補正値を$\nicefrac{D_\mathrm e}2$として設定を行い、さらに\index{こうぐけいまもうりょう@工具径摩耗量}工具径摩耗量を$d_\mathrm e\tan\xi_\mathrm e$として設定を行うものとする。
ここで$d_\mathrm e$は、\index{こうぐちょうほせい@工具長補正}工具長補正において用いたものとする。


%%%%%%%%%%%%%%%%%%%%%%%%%%%%%%%%%%%%%%%%%%%%%%%%%%%%%%%%%%
%% subsection 10.7.3 %%%%%%%%%%%%%%%%%%%%%%%%%%%%%%%%%%%%%
%%%%%%%%%%%%%%%%%%%%%%%%%%%%%%%%%%%%%%%%%%%%%%%%%%%%%%%%%%
\subsection{端面外C面取加工}

%%%%%%%%%%%%%%%%%%%%%%%%%%%%%%%%%%%%%%%%%%%%%%%%%%%%%%%%%%
%% subsubsection 10.7.3.1 %%%%%%%%%%%%%%%%%%%%%%%%%%%%%%%%
%%%%%%%%%%%%%%%%%%%%%%%%%%%%%%%%%%%%%%%%%%%%%%%%%%%%%%%%%%
\subsubsection{面取Cが小さい場合}
\index{たんめんそとCめんとり@端面外C面取}端面外C面取について、そのC面取長が小さい場合は、\index{てもちけんまき@手持ち研磨機}手持ち研磨機を用いた手動による加工で行ってもよいものとする。

%%%%%%%%%%%%%%%%%%%%%%%%%%%%%%%%%%%%%%%%%%%%%%%%%%%%%%%%%%
%% subsubsection 10.7.3.2 %%%%%%%%%%%%%%%%%%%%%%%%%%%%%%%%
%%%%%%%%%%%%%%%%%%%%%%%%%%%%%%%%%%%%%%%%%%%%%%%%%%%%%%%%%%
\subsubsection{面取Cが大きい場合}
マシニングセンタを用いて加工する場合、外削のない場合は、加工径の中心座標$X$をトップ側・ボトム側のそれぞれに対して以下だけ補正する。
\begin{align*}
  \text{トップ側:}&~~
  \sqrt{R_\mathrm c^2-\left(f_\mathrm T-c_\mathrm{To}\right)^2}-\sqrt{R_\mathrm c^2-f_\mathrm T^2}\ ,\\
  \text{ボトム側:}&~~
  \sqrt{R_\mathrm c^2-f_\mathrm B^2}-\sqrt{R_\mathrm c^2-\left(f_\mathrm B-c_\mathrm{Bo}\right)^2}\ .
\end{align*}
ここで$c_\mathrm{To}$, $c_\mathrm{Bo}$, $R_\mathrm c$, $f_\mathrm T$, $f_\mathrm B$はそれぞれトップ端面外C面取長, ボトム端面外C面取長, \CenterCurvatureRadius, トップ振分長, ボトム振分長を示す。


\clearpage
%%%%%%%%%%%%%%%%%%%%%%%%%%%%%%%%%%%%%%%%%%%%%%%%%%%%%%%%%%
%% subsection 10.7.4 %%%%%%%%%%%%%%%%%%%%%%%%%%%%%%%%%%%%%
%%%%%%%%%%%%%%%%%%%%%%%%%%%%%%%%%%%%%%%%%%%%%%%%%%%%%%%%%%
\subsection{端面内C面取加工}

%%%%%%%%%%%%%%%%%%%%%%%%%%%%%%%%%%%%%%%%%%%%%%%%%%%%%%%%%%
%% subsubsection 10.7.4.1 %%%%%%%%%%%%%%%%%%%%%%%%%%%%%%%%
%%%%%%%%%%%%%%%%%%%%%%%%%%%%%%%%%%%%%%%%%%%%%%%%%%%%%%%%%%
\subsubsection{面取Cが小さい場合}
\index{たんめんうちCめんとり@端面内C面取}端面内C面取について、そのC面取長が小さい場合は、\index{てもちけんまき@手持ち研磨機}手持ち研磨機を用いた手動による加工で行ってもよいものとする。

%%%%%%%%%%%%%%%%%%%%%%%%%%%%%%%%%%%%%%%%%%%%%%%%%%%%%%%%%%
%% subsubsection 10.7.4.2 %%%%%%%%%%%%%%%%%%%%%%%%%%%%%%%%
%%%%%%%%%%%%%%%%%%%%%%%%%%%%%%%%%%%%%%%%%%%%%%%%%%%%%%%%%%
\subsubsection{面取Cが大きい場合}
マシニングセンタを用いて加工する場合、加工径の中心座標$X$をトップ側・ボトム側のそれぞれに対して以下だけ補正する。
\begin{align*}
  \text{トップ側:}&~~
  \sqrt{R_\mathrm c^2-\left(f_\mathrm T-c_\mathrm{Ti}\right)^2}-\sqrt{R_\mathrm c^2-f_\mathrm T^2}\ ,\\
  \text{ボトム側:}&~~
  \sqrt{R_\mathrm c^2-f_\mathrm B^2}-\sqrt{R_\mathrm c^2-\left(f_\mathrm B-c_\mathrm{Bi}\right)^2}\ .
\end{align*}
ここで$c_\mathrm{Ti}$, $c_\mathrm{Bi}$, $R_\mathrm c$, $f_\mathrm T$, $f_\mathrm B$はそれぞれトップ端面内C面取長, ボトム端面内C面取長, \CenterCurvatureRadius, トップ振分長, ボトム振分長を示す。



\clearpage
%%%%%%%%%%%%%%%%%%%%%%%%%%%%%%%%%%%%%%%%%%%%%%%%%%%%%%%%%%
%% section 10.8 %%%%%%%%%%%%%%%%%%%%%%%%%%%%%%%%%%%%%%%%%%
%%%%%%%%%%%%%%%%%%%%%%%%%%%%%%%%%%%%%%%%%%%%%%%%%%%%%%%%%%
\modHeadsection{端面R面取に関する寸法}


%%%%%%%%%%%%%%%%%%%%%%%%%%%%%%%%%%%%%%%%%%%%%%%%%%%%%%%%%%
%% subsection 10.9.1 %%%%%%%%%%%%%%%%%%%%%%%%%%%%%%%%%%%%%
%%%%%%%%%%%%%%%%%%%%%%%%%%%%%%%%%%%%%%%%%%%%%%%%%%%%%%%%%%
\subsection{端面R面取加工の基準点}

%%%%%%%%%%%%%%%%%%%%%%%%%%%%%%%%%%%%%%%%%%%%%%%%%%%%%%%%%%
%% subsubsection 10.8.1.1 %%%%%%%%%%%%%%%%%%%%%%%%%%%%%%%%
%%%%%%%%%%%%%%%%%%%%%%%%%%%%%%%%%%%%%%%%%%%%%%%%%%%%%%%%%%
\subsubsection{端面外R面取加工の基準点}
\index{たんめんそとRめんとりかこう@端面外R面取加工}端面外R面取加工の\index{きじゅん(たんめんそとRめんとりかこう)@基準(端面外R面取加工)}基準は、外削のある場合は\index{がいさくちゅうしん@外削中心}外削中心を基準とし、外削のない場合は\index{がいけいちゅうしん(たんめん)@外径中心(端面)}端面における外径中心を基準として行うものとする。

%%%%%%%%%%%%%%%%%%%%%%%%%%%%%%%%%%%%%%%%%%%%%%%%%%%%%%%%%%
%% subsubsection 10.8.1.1 %%%%%%%%%%%%%%%%%%%%%%%%%%%%%%%%
%%%%%%%%%%%%%%%%%%%%%%%%%%%%%%%%%%%%%%%%%%%%%%%%%%%%%%%%%%
\subsubsection{端面内R面取加工}
\index{たんめんうちRめんとりかこう@端面内R面取加工}端面内R面取加工の\index{きじゅん(たんめんうちRめんとりかこう)@基準(端面内R面取加工)}基準は、\index{ないけいちゅうしん(たんめん)@内径中心(端面)}端面における内径中心を基準として行うものとする。


%%%%%%%%%%%%%%%%%%%%%%%%%%%%%%%%%%%%%%%%%%%%%%%%%%%%%%%%%%
%% subsection 10.8.1 %%%%%%%%%%%%%%%%%%%%%%%%%%%%%%%%%%%%%
%%%%%%%%%%%%%%%%%%%%%%%%%%%%%%%%%%%%%%%%%%%%%%%%%%%%%%%%%%
\subsection{端面外R面取加工}

%%%%%%%%%%%%%%%%%%%%%%%%%%%%%%%%%%%%%%%%%%%%%%%%%%%%%%%%%%
%% subsubsection 10.8.1.1 %%%%%%%%%%%%%%%%%%%%%%%%%%%%%%%%
%%%%%%%%%%%%%%%%%%%%%%%%%%%%%%%%%%%%%%%%%%%%%%%%%%%%%%%%%%
\subsubsection{面取Rが小さい場合}
\index{たんめんそとRめんとり@端面外R面取}端面外R面取については、基本的にはマシニングセンタでは行わず、\index{てもちけんまき@手持ち研磨機}手持ち研磨機を用いて手動で行うものとする。

%%%%%%%%%%%%%%%%%%%%%%%%%%%%%%%%%%%%%%%%%%%%%%%%%%%%%%%%%%
%% subsubsection 10.8.1.2 %%%%%%%%%%%%%%%%%%%%%%%%%%%%%%%%
%%%%%%%%%%%%%%%%%%%%%%%%%%%%%%%%%%%%%%%%%%%%%%%%%%%%%%%%%%
\subsubsection{面取Rが大きい場合}
\index{めんとりR@面取R}面取Rが大きい場合、\index{さぎょうしゃ@作業者}作業者の負担や\index{さぎょうじかん@作業時間}作業時間を考慮して、その一部を\index{テーパエンドミル}テーパエンドミルを用いて加工してもよいものとする。

このとき、トップ端およびボトム端における外R面取長$r_\mathrm{To}$, $r_\mathrm{Bo}$, および片角$\xi_\mathrm e$のテーパエンドミルに対して、
\begin{align*}
  c_\mathrm{To} &= r_\mathrm{To}\left(1+\cot\xi_\mathrm e-\csc\xi_\mathrm e\right)\ ,\\
  c_\mathrm{Bo} &= r_\mathrm{Bo}\left(1+\cot\xi_\mathrm e-\csc\xi_\mathrm e\right)
\end{align*}
の\index{たんめんそとCめんとりちょう@端面C面取長}端面外C面取長とみなして、\index{たんめんそとCめんとりかこう@端面外C面取加工}端面外C面取加工を行うものとする。


%%%%%%%%%%%%%%%%%%%%%%%%%%%%%%%%%%%%%%%%%%%%%%%%%%%%%%%%%%
%% subsection 10.8.2 %%%%%%%%%%%%%%%%%%%%%%%%%%%%%%%%%%%%%
%%%%%%%%%%%%%%%%%%%%%%%%%%%%%%%%%%%%%%%%%%%%%%%%%%%%%%%%%%
\subsection{端面内R面取加工}

%%%%%%%%%%%%%%%%%%%%%%%%%%%%%%%%%%%%%%%%%%%%%%%%%%%%%%%%%%
%% subsubsection 10.8.2.1 %%%%%%%%%%%%%%%%%%%%%%%%%%%%%%%%
%%%%%%%%%%%%%%%%%%%%%%%%%%%%%%%%%%%%%%%%%%%%%%%%%%%%%%%%%%
\subsubsection{面取Rが小さい場合}
\index{たんめんうちRめんとり@端面内R面取}トップ端面およびボトム端面の内R面取については、基本的にはマシニングセンタでは行わず、\index{てもちけんまき@手持ち研磨機}手持ち研磨機を用いて手動で行うものとする。

%%%%%%%%%%%%%%%%%%%%%%%%%%%%%%%%%%%%%%%%%%%%%%%%%%%%%%%%%%
%% subsubsection 10.8.2.2 %%%%%%%%%%%%%%%%%%%%%%%%%%%%%%%%
%%%%%%%%%%%%%%%%%%%%%%%%%%%%%%%%%%%%%%%%%%%%%%%%%%%%%%%%%%
\subsubsection{面取Rが大きい場合}
面取Rが大きい場合、作業者の負担や作業時間を考慮して、その一部を\index{テーパエンドミル}テーパエンドミルを用いて加工してもよいものとする。

このとき、トップ端面およびボトム端面における内R面取長$r_\mathrm{Ti}$, $r_\mathrm{Bi}$, および片角$\xi_\mathrm e$のテーパエンドミルに対して、
\begin{align*}
  c_\mathrm{Ti} &= r_\mathrm{Ti}\left(1+\cot\xi_\mathrm e-\csc\xi_\mathrm e\right)\ ,\\
  c_\mathrm{Bi} &= r_\mathrm{Bi}\left(1+\cot\xi_\mathrm e-\csc\xi_\mathrm e\right)
\end{align*}
の\index{たんめんうちCめんとりちょう@端面内C面取長}端面内C面取長とみなして、\index{たんめんうちCめんとりかこう@端面内C面取加工}端面内C面取加工を行うものとする。



\clearpage
%%%%%%%%%%%%%%%%%%%%%%%%%%%%%%%%%%%%%%%%%%%%%%%%%%%%%%%%%%
%% section 10.9 %%%%%%%%%%%%%%%%%%%%%%%%%%%%%%%%%%%%%%%%%%
%%%%%%%%%%%%%%%%%%%%%%%%%%%%%%%%%%%%%%%%%%%%%%%%%%%%%%%%%%
\modHeadsection{\EndFaceBoring に関する寸法\TBW}
(to be written...)


\clearpage
%%%%%%%%%%%%%%%%%%%%%%%%%%%%%%%%%%%%%%%%%%%%%%%%%%%%%%%%%%
%% section 10.9 %%%%%%%%%%%%%%%%%%%%%%%%%%%%%%%%%%%%%%%%%%
%%%%%%%%%%%%%%%%%%%%%%%%%%%%%%%%%%%%%%%%%%%%%%%%%%%%%%%%%%
\modHeadsection{\Dimple に関する寸法}


%%%%%%%%%%%%%%%%%%%%%%%%%%%%%%%%%%%%%%%%%%%%%%%%%%%%%%%%%%
%% subsection 10.06.1 %%%%%%%%%%%%%%%%%%%%%%%%%%%%%%%%%%%%
%%%%%%%%%%%%%%%%%%%%%%%%%%%%%%%%%%%%%%%%%%%%%%%%%%%%%%%%%%
\subsection{工具補正:\Dimple 加工}

%%%%%%%%%%%%%%%%%%%%%%%%%%%%%%%%%%%%%%%%%%%%%%%%%%%%%%%%%%
%% subsubsection 10.06.1.1 %%%%%%%%%%%%%%%%%%%%%%%%%%%%%%%
%%%%%%%%%%%%%%%%%%%%%%%%%%%%%%%%%%%%%%%%%%%%%%%%%%%%%%%%%%
\subsubsection{工具長補正:\Dimple 加工}
\Dimple 加工に用いる\index{Tスロットカッター}Tスロットカッターの\index{こうぐちょう(Tスロットカッター)@工具長(Tスロットカッター)}工具長については、切削部(刃の部分)の中央を工具長として\index{オフセット(こうぐちょうほせい)@オフセット(工具長補正)}オフセット量の設定を行うものとする。

したがって、工具の切削部の先端、および切削部の厚みを測定し、切削部の先端から厚みの半分の値を引いた値を\index{こうぐちょうほせいち@工具長補正値}工具長補正値とする。
なお、このとき工具長の\index{まもうりょう(こうぐちょうほせい)@摩耗量(工具長補正)}摩耗量は0とする。

%%%%%%%%%%%%%%%%%%%%%%%%%%%%%%%%%%%%%%%%%%%%%%%%%%%%%%%%%%
%% subsubsection 10.06.1.2 %%%%%%%%%%%%%%%%%%%%%%%%%%%%%%%
%%%%%%%%%%%%%%%%%%%%%%%%%%%%%%%%%%%%%%%%%%%%%%%%%%%%%%%%%%
\subsubsection{工具径補正:\Dimple 加工}
\Dimple 加工に用いる\index{Tスロットカッター}Tスロットカッターの\index{こうぐけい(Tスロットカッター)@工具径(Tスロットカッター)}工具長は、その刃の径方向の先端部を工具径として\index{オフセット(こうぐけいほせい)@オフセット(工具径補正)}オフセット量の設定を行うものとする。

なお、このとき工具径の\index{まもうりょう(こうぐけいほせい)@摩耗量(工具径補正)}摩耗量は0とする。


%%%%%%%%%%%%%%%%%%%%%%%%%%%%%%%%%%%%%%%%%%%%%%%%%%%%%%%%%%
%% subsection 10.9.2 %%%%%%%%%%%%%%%%%%%%%%%%%%%%%%%%%%%%%
%%%%%%%%%%%%%%%%%%%%%%%%%%%%%%%%%%%%%%%%%%%%%%%%%%%%%%%%%%
\subsection{\Dimple 加工の基準点}
\Dimple を加工する際は、トップ側の端面における内側の中心座標の\index{じっそくち@実測値}実測値を基準にして行うものとする。


%%%%%%%%%%%%%%%%%%%%%%%%%%%%%%%%%%%%%%%%%%%%%%%%%%%%%%%%%%
%% subsection 10.9.3 %%%%%%%%%%%%%%%%%%%%%%%%%%%%%%%%%%%%%
%%%%%%%%%%%%%%%%%%%%%%%%%%%%%%%%%%%%%%%%%%%%%%%%%%%%%%%%%%
\subsection{\Dimple 加工の傾き角}
\Dimple の測定および加工は、\index{アンダーカット}アンダーカットが生じないように適切にワークを$B$軸方向に傾けるものとする。
このとき\expandafterindex{かたむきかく(\yomiDimple)@傾き角(\nameDimple)}傾ける角度$\phi$は、\expandafterindex{\yomiDimple1れつめ@\nameDimple1列目}トップ端から1列目の\nameDimple までの距離$q$に対して、
\begin{align*}
  \tan\phi
  &= \frac{\displaystyle
           \sqrt{\left(R_\mathrm c+\frac{w'_{\mathrm Aq}}2\right)^2-(f_\mathrm T-q)^2}
           -\sqrt{\left(R_\mathrm c+\frac{w'_{\mathrm A0}}2\right)^2-f_\mathrm T^2}}q
\end{align*}
とする。
ここで$w'_{\mathrm A0}$, $w'_{\mathrm Aq}$は(\PlatingThk$\mu$を考慮した)トップ端におけるAC方向の内径およびトップ端から距離$q$におけるAC方向の内径を示し、$R_\mathrm c$, $f_\mathrm T$はそれぞれ\CenterCurvatureRadius, トップ振分長を示す。

ただし、$\phi < 0$となる場合は、$\phi = 0$とみなすものとする。



%!TEX root = ../RPA_for_Creating_Program_Note.tex


\modHeadchapter{プログラムの番号付け}
ここでは\DMname で使用する\index{プログラムばんごう@プログラム番号}プログラム番号(\index{Oコード}Oコード)についての規則を与える
%% footnote %%%%%%%%%%%%%%%%%%%%%
\footnote{機械設置時に付属の\index{バンドルのプログラム}バンドルのプログラムについてはこの限りではない。}。
%%%%%%%%%%%%%%%%%%%%%%%%%%%%%%%%%



%%%%%%%%%%%%%%%%%%%%%%%%%%%%%%%%%%%%%%%%%%%%%%%%%%%%%%%%%%
%% section 12.1 %%%%%%%%%%%%%%%%%%%%%%%%%%%%%%%%%%%%%%%%%%
%%%%%%%%%%%%%%%%%%%%%%%%%%%%%%%%%%%%%%%%%%%%%%%%%%%%%%%%%%
\modHeadsection{プログラム番号の基本事項}
\begin{enumerate}[label=\Roman*., ref=\Roman*]
\item プログラム番号は8桁の半角英数字で表される({\ttfamily(a-zA-Z|\textbackslash d){8}})
\item 原則として、プログラム番号には半角数字のみを用いる({\ttfamily\textbackslash d{8}})
\item プログラム番号には右から順に1桁目, 2桁目, ..., 8桁目と数えるものとする
\item\label{item:PNbasicGE4}プログラム番号は4桁までは左側0埋めを行い、5桁目以上の左側0埋めの有無は問わない
\end{enumerate}
%%%%%%%%%%%%%%%%%%%%%%%%%%%%%%%%%%%%%%%%%%%%%%%%%%%%%%%%%%
%% hosoku %%%%%%%%%%%%%%%%%%%%%%%%%%%%%%%%%%%%%%%%%%%%%%%%
%%%%%%%%%%%%%%%%%%%%%%%%%%%%%%%%%%%%%%%%%%%%%%%%%%%%%%%%%%
\begin{hosoku}
なおこの規則だと、\index{バンドルのプログラム}バンドルのプログラム(O7xxx, O8xxx, O9xxx)と重複する恐れがある。
これについては、実際にそうした問題に直面したときにその都度に対応するものとする。
基本的には、バンドルのプログラムを(可能であれば)変更する方針とする。
\end{hosoku}
%%%%%%%%%%%%%%%%%%%%%%%%%%%%%%%%%%%%%%%%%%%%%%%%%%%%%%%%%%
%%%%%%%%%%%%%%%%%%%%%%%%%%%%%%%%%%%%%%%%%%%%%%%%%%%%%%%%%%
%%%%%%%%%%%%%%%%%%%%%%%%%%%%%%%%%%%%%%%%%%%%%%%%%%%%%%%%%%


%%%%%%%%%%%%%%%%%%%%%%%%%%%%%%%%%%%%%%%%%%%%%%%%%%%%%%%%%%
%% section 12.2 %%%%%%%%%%%%%%%%%%%%%%%%%%%%%%%%%%%%%%%%%%
%%%%%%%%%%%%%%%%%%%%%%%%%%%%%%%%%%%%%%%%%%%%%%%%%%%%%%%%%%
\modHeadsection{番号付け:8, 7桁目}
\index{7けため(プログラムばんごう)@7桁目(プログラム番号)}7桁目および\index{8けため(プログラムばんごう)@8桁目(プログラム番号)}8桁目はともに0とする。

これに伴い、\ref{item:PNbasicGE4}に基づき、以下では0埋めを省略してすべてのプログラム番号を6桁の数字 ({\ttfamily\textbackslash d{6}})で表す。


\clearpage
%%%%%%%%%%%%%%%%%%%%%%%%%%%%%%%%%%%%%%%%%%%%%%%%%%%%%%%%%%
%% section 12.3 %%%%%%%%%%%%%%%%%%%%%%%%%%%%%%%%%%%%%%%%%%
%%%%%%%%%%%%%%%%%%%%%%%%%%%%%%%%%%%%%%%%%%%%%%%%%%%%%%%%%%
\modHeadsection{番号付け:6桁目}
\index{6けため(プログラムばんごう)@6桁目(プログラム番号)}6桁目は主にプログラムの種類を表すものとし、以下のように分類する。

%%%%%%%%%%%%%%%%%%%%%%%%%%%%%%%%%%%%%%%%%%%%%%%%%%%%%%%%%%
%% subsection 9.3.1 %%%%%%%%%%%%%%%%%%%%%%%%%%%%%%%%%%%%%%
%%%%%%%%%%%%%%%%%%%%%%%%%%%%%%%%%%%%%%%%%%%%%%%%%%%%%%%%%%
\subsection{6桁目:0}
6桁目が0のものは、原則として\index{メインプログラム}メインプログラムとする。
このとき、下5桁は製品の\index{ずめんばんごう@図面番号}図面番号(番号部分・右詰め, {\ttfamily\textbackslash d{5}})とする
%% footnote %%%%%%%%%%%%%%%%%%%%%
\footnote{稀に、図面番号にアルファベットが含まれるものが存在する。
その場合は、その都度に別途対応する。}。
%%%%%%%%%%%%%%%%%%%%%%%%%%%%%%%%%

%%%%%%%%%%%%%%%%%%%%%%%%%%%%%%%%%%%%%%%%%%%%%%%%%%%%%%%%%%
%% subsection 9.3.2 %%%%%%%%%%%%%%%%%%%%%%%%%%%%%%%%%%%%%%
%%%%%%%%%%%%%%%%%%%%%%%%%%%%%%%%%%%%%%%%%%%%%%%%%%%%%%%%%%
\subsection{6桁目:0, 9以外}
6桁目が0, 9以外の場合は、以下のように分類する。
\begin{enumerate}[label=\arabic*., ref=\arabic*, start=1]
\item\label{item:6Mmain} 測定(\dimple 以外)を行うプログラム({\ttfamily1\textbackslash d{5}})
\item\label{item:6MD} 測定(\dimple)を行うプログラム({\ttfamily2\textbackslash d{5}})
%\item\label{item:6MN} 測定(逃し溝)を行うプログラム
\setcounter{enumi}{3}
\item\label{item:6Kmain} 加工(\dimple 以外)を行うプログラム({\ttfamily4\textbackslash d{5}})
\item\label{item:6KD} 加工(\dimple)を行うプログラム({\ttfamily5\textbackslash d{5}})
%\item\label{item:6KN} 加工(逃し溝)を行うプログラム
\end{enumerate}
複数の用途での使用が想定されるものに対しては、番号の若いほうに合わせる。


%%%%%%%%%%%%%%%%%%%%%%%%%%%%%%%%%%%%%%%%%%%%%%%%%%%%%%%%%%
%% subsection 9.3.1 %%%%%%%%%%%%%%%%%%%%%%%%%%%%%%%%%%%%%%
%%%%%%%%%%%%%%%%%%%%%%%%%%%%%%%%%%%%%%%%%%%%%%%%%%%%%%%%%%
\subsection{6桁目:9}
6桁目が9のものは、製品の測定・加工に直接関係しないプログラムとする。
このとき、下5桁はその都度に別途考慮し番号付けを行う。({\ttfamily9\textbackslash d{5}})
%%%%%%%%%%%%%%%%%%%%%%%%%%%%%%%%%%%%%%%%%%%%%%%%%%%%%%%%%%
%% hosoku %%%%%%%%%%%%%%%%%%%%%%%%%%%%%%%%%%%%%%%%%%%%%%%%
%%%%%%%%%%%%%%%%%%%%%%%%%%%%%%%%%%%%%%%%%%%%%%%%%%%%%%%%%%
\begin{hosoku}
たとえば、\index{タッチセンサーでんげん@タッチセンサー電源}タッチセンサー電源のON・OFF, \index{だんきうんてん@暖機運転}暖機運転, \index{こうぐそくてい@工具測定}工具測定などのプログラムはこれに属するものとする。
\end{hosoku}
%%%%%%%%%%%%%%%%%%%%%%%%%%%%%%%%%%%%%%%%%%%%%%%%%%%%%%%%%%
%%%%%%%%%%%%%%%%%%%%%%%%%%%%%%%%%%%%%%%%%%%%%%%%%%%%%%%%%%
%%%%%%%%%%%%%%%%%%%%%%%%%%%%%%%%%%%%%%%%%%%%%%%%%%%%%%%%%%


\clearpage
%%%%%%%%%%%%%%%%%%%%%%%%%%%%%%%%%%%%%%%%%%%%%%%%%%%%%%%%%%
%% section 12.4 %%%%%%%%%%%%%%%%%%%%%%%%%%%%%%%%%%%%%%%%%%
%%%%%%%%%%%%%%%%%%%%%%%%%%%%%%%%%%%%%%%%%%%%%%%%%%%%%%%%%%
\modHeadsection{番号付け:5桁目}
\index{5けため(プログラムばんごう)@5桁目(プログラム番号)}5桁目は、以下のように分類する。
なお、以降では6桁目が\ref{item:6Mmain}, \ref{item:6MD}, \ref{item:6Kmain}, \ref{item:6KD}の場合のみについて記述する
\begin{enumerate}[label=\alph*)]
\item 6桁目が\ref{item:6Mmain}(\dimple 以外の測定)の場合、5桁目を以下にように分類する
  \begin{enumerate}[label=\arabic*., ref=\arabic*, start=1]
  \item\label{item:5MCOBsZ} 芯出しにおいて、$Z$一定で($X$または$Y$の)外側両面を測定するプログラム({\ttfamily 11\textbackslash d{4}})
  \item\label{item:5MCOO} \index{しんだし@芯出し}芯出しにおいて、($XY$)外側片面を測定するプログラム({\ttfamily 12\textbackslash d{4}})
  \item\label{item:5MCIB} 芯出しにおいて、$Z$一定で($X$または$Y$の)内側両面を測定するプログラム({\ttfamily 13\textbackslash d{4}})
  \item\label{item:5MCIO} 芯出しにおいて、($XY$)内側片面を測定するプログラム({\ttfamily 14\textbackslash d{4}})
  \item\label{item:5MCL} \index{とおりしん@通り芯}通り芯を測定するプログラム({\ttfamily 15\textbackslash d{5}})
  \end{enumerate}
\item 6桁目が\ref{item:6MD}, \ref{item:6KD}(\dimple の測定, 加工)の場合、5桁目を以下のように分類する
  \begin{enumerate}[label=\arabic*., ref=\arabic*]
  \item 主に、\expandafterindex{\dimplekana@\dimple}\dimple の行と中心湾曲線上の交点への移動を繰返すプログラム({\ttfamily 21\textbackslash d{4}})
%  \item \dimple において、各内面方向への($X$または$Y$方向)の移動を繰返すプログラム({\ttfamily 22\textbackslash d{4}})
  \item 主に、\dimple の個々の行方向の移動を繰返すプログラム({\ttfamily 22\textbackslash d{4}})
  \item 主に、\dimple の個々の深さ方向に測定または加工するプログラム({\ttfamily [25]3\textbackslash d{4}})
%  \item 主にレベル4の階層で用いるプログラム(c)
  \end{enumerate}
%  複数の用途での使用が想定されるものに対しては、番号の若いほうに合わせる。
\item 6桁目が\ref{item:6Kmain}(\dimple 以外の加工)の場合、5桁目を以下にように分類する
  \begin{enumerate}[label=\arabic*., ref=\arabic*, start=1]
%  \item\label{item:5Kaux} 加工の種類に依存しない加工のプログラム(\ttfamily{40\textbackslash d{4}})
  \item\label{item:5KF} 主に、\index{たんめんかこう@端面加工}端面加工の位置決めを行うプログラム({\ttfamily 41\textbackslash d{4}})
  \item\label{item:5KO} 主に、\index{がいさくかこう@外削加工}外削加工の位置決めを行うプログラム({\ttfamily 42\textbackslash d{4}})
  \item\label{item:5KK} 主に、\index{みぞかこう@溝加工}溝加工の位置決めを行うプログラム({\ttfamily 43\textbackslash d{4}})
  \item\label{item:5KCO} 主に、(端面部)\index{そとがわCめんとりかこう(たんめん)@外側C面取加工(端面)}外側\index{Cめんとりかこう@C面取加工}C面取加工の位置決めを行うプログラム({\ttfamily 44\textbackslash d{4}})
  \item\label{item:5KCI} 主に、(端面部)\index{うちがわCめんとりかこう(たんめん)@内側C面取加工(端面)}内側C面取加工の位置決めを行うプログラム({\ttfamily 45\textbackslash d{4}})
%  \item\label{item:5KZ} 座ぐりの加工のプログラム(\ttfamily{46\textbackslash d{4}})
  \setcounter{enumii}{8}
  \item 主に、位置決め後、実際に加工を行うプログラム({\ttfamily 49\textbackslash d{4}})
  \end{enumerate}
\end{enumerate}


%%%%%%%%%%%%%%%%%%%%%%%%%%%%%%%%%%%%%%%%%%%%%%%%%%%%%%%%%%
%% section 12.4 %%%%%%%%%%%%%%%%%%%%%%%%%%%%%%%%%%%%%%%%%%
%%%%%%%%%%%%%%%%%%%%%%%%%%%%%%%%%%%%%%%%%%%%%%%%%%%%%%%%%%
\modHeadsection{番号付け:4桁目}
(6桁目が\ref{item:6Mmain}, \ref{item:6MD}, \ref{item:6Kmain}, \ref{item:6KD}の場合)\index{4けため(プログラムばんごう)@4桁目(プログラム番号)}4桁目は、以下のように分類する。
\begin{enumerate}[label=\alph*), ref=\alph*)]
\item 6桁目が\ref{item:6Kmain}\hx 以外の場合、4桁目は0とする({\ttfamily{[126][1-5]0\textbackslash d{3}}})
\item 6桁目が\ref{item:6Kmain}(\dimple 以外の加工)の場合、4桁目を以下にように分類する
  \begin{enumerate}[label=\alph{enumi}\,-\arabic*), leftmargin=\leftmargini]
  \item 5桁目が\ref{item:5KO}(外削加工)以外の場合、4桁目は0とする({\ttfamily{4[13-59]0\textbackslash d{3}}})
  \item 5桁目が\ref{item:5KO}(外削加工)の場合、4桁目を以下のように分類する
    \begin{enumerate}[label=\arabic*., ref=\arabic*, start=0, leftmargin=*]
    \item 端面に垂直方向の外削加工のプログラム({\ttfamily{420\textbackslash d{3}}})
    \item 湾曲に沿った方向の外削加工のプログラム({\ttfamily{421\textbackslash d{3}}})
    \end{enumerate}
  \end{enumerate}
\end{enumerate}



%%%%%%%%%%%%%%%%%%%%%%%%%%%%%%%%%%%%%%%%%%%%%%%%%%%%%%%%%%
%% section 12.6 %%%%%%%%%%%%%%%%%%%%%%%%%%%%%%%%%%%%%%%%%%
%%%%%%%%%%%%%%%%%%%%%%%%%%%%%%%%%%%%%%%%%%%%%%%%%%%%%%%%%%
\modHeadsection{番号付け:3, 2桁目}
(6桁目が\ref{item:6Mmain}, \ref{item:6MD}, \ref{item:6Kmain}, \ref{item:6KD}\hx の場合の)\index{3けため(プログラムばんごう)@3桁目(プログラム番号)}3桁目および\index{2けため(プログラムばんごう)@2桁目(プログラム番号)}2桁目はともに0とする。({\ttfamily{[1245][1-69][01]0{2}\textbackslash d}})


\clearpage
%%%%%%%%%%%%%%%%%%%%%%%%%%%%%%%%%%%%%%%%%%%%%%%%%%%%%%%%%%
%% section 12.7 %%%%%%%%%%%%%%%%%%%%%%%%%%%%%%%%%%%%%%%%%%
%%%%%%%%%%%%%%%%%%%%%%%%%%%%%%%%%%%%%%%%%%%%%%%%%%%%%%%%%%
\modHeadsection{番号付け:1桁目}
(6桁目が\ref{item:6Mmain}, \ref{item:6MD}, \ref{item:6Kmain}, \ref{item:6KD}\hx の場合の)\index{1けため(プログラムばんごう)@1桁目(プログラム番号)}1桁目は、以下のように分類する。
\begin{enumerate}[label=\arabic*.]
\item 主に$X$方向に測定または加工するプログラム({\ttfamily{[125][1-69]0{3}1}})
\item 主に$Y$方向に測定または加工するプログラム({\ttfamily{[125][1-69]0{3}2}})
\item 主に$Z$方向に測定または加工するプログラム({\ttfamily{[125][1-69]0{3}3}})
\item 工具から見て右回り($XY$面)に、主に外側から加工するプログラム({\ttfamily{[125][1-69]0{3}4}})
\item 工具から見て左回り($XY$面)に、主に外側から加工するプログラム({\ttfamily{[125][1-69]0{3}5}})
\item 工具から見て右回り($XY$面)に、主に内側から加工するプログラム({\ttfamily{[125][1-69]0{3}6}})
\item 工具から見て左回り($XY$面)に、主に内側から加工するプログラム({\ttfamily{[125][1-69]0{3}7}})
\end{enumerate}

%!TEX root = ../RPA_for_Creating_Program_Note.tex



ここでは\DMname で使用する\index{こうぐばんごう@工具番号}工具番号(\index{Tコード}Tコード値)について記述する。
%%%%%%%%%%%%%%%%%%%%%%%%%%%%%%%%%%%%%%%%%%%%%%%%%%%%%%%%%%
%% section 13.1 %%%%%%%%%%%%%%%%%%%%%%%%%%%%%%%%%%%%%%%%%%
%%%%%%%%%%%%%%%%%%%%%%%%%%%%%%%%%%%%%%%%%%%%%%%%%%%%%%%%%%
\modHeadsection{基本事項}
\begin{enumerate}[label=\Roman*), ref=\Roman*)]
\item 工具番号01は空とする
\item 工具番号02-05は\index{たんめんかこう@端面加工}端面加工用工具(\index{フェイスミル}フェイスミル)とする
\item 工具番号06-09は\index{みぞかこう@溝加工}溝加工用工具(\index{サイドカッター}サイドカッター)とする
\item 工具番号11-15は\index{めんとりかこう@面取加工}面取加工用工具(\index{テーパーエンドミル}テーパーエンドミル)とする
\item 工具番号16-20は\index{がいさくかこう@外削加工}外削加工用工具(\index{スクエアエンドミル}スクエアエンドミル)とする
\item 工具番号31-35は\index{ないめんみぞかこう@内面溝加工}内面溝加工用工具(\index{Tスロットカッター}Tスロットカッター)とする
\item 工具番号36-40は内面溝加工用工具(\index{アングルヘッド}アングルヘッド)とする
\item 工具番号41-45は\index{にがしみぞかこう@逃し溝加工}逃し溝加工用工具(アングルヘッド)とする
\item 工具番号49, 50は測定用工具(\index{タッチセンサープローブ}タッチセンサープローブ)とする
\end{enumerate}



%%%%%%%%%%%%%%%%%%%%%%%%%%%%%%%%%%%%%%%%%%%%%%%%%%%%%%%%%%
%% section 13.2 %%%%%%%%%%%%%%%%%%%%%%%%%%%%%%%%%%%%%%%%%%
%%%%%%%%%%%%%%%%%%%%%%%%%%%%%%%%%%%%%%%%%%%%%%%%%%%%%%%%%%
\modHeadsection{登録工具}
\addtocontents{lot}{\protect\addvspace{3pt}}{}{}
\addcontentsline{lot}{section}{\numberline{\thesection}\Sectionname}
\customtodayap 時点において\index{とうろくこうぐ@登録工具}登録されている工具は以下の通りである\\
%%%%%%%%%%%%%%%%%%%%%%%%%%%%%%%%%%%%%%%%%%%%%%%%%%%%%%%%%%
%% common variables %%%%%%%%%%%%%%%%%%%%%%%%%%%%%%%%%%%%%%
%%%%%%%%%%%%%%%%%%%%%%%%%%%%%%%%%%%%%%%%%%%%%%%%%%%%%%%%%%
\modcaptionof{table}{\DMname の登録工具}
\begin{twoCtable}{}
\verb|T01| & 空\\\hline
\hline
\verb|T02| & $\phi100$端面加工用フェイスミル\\\hline
\hline
\verb|T06| & $\phi100\times t7$溝加工用サイドカッター\\\hline
\verb|T07| & $\phi100\times t5$溝加工用サイドカッター\\\hline
\hline
\verb|T11| & $\phi4.0\times 15^\circ$面取加工用テーパーエンドミル\\\hline
\verb|T12| & $\phi0.8\times 30^\circ$面取加工用テーパーエンドミル\\\hline
\verb|T13| & $\phi2.0\times 45^\circ$面取加工用テーパーエンドミル\\\hline
\hline
\verb|T16| & $\phi20$外削加工用スクエアエンドミル\\\hline
\hline
\verb|T31| & $\phi40\times R20$内面溝加工用Tスロットカッター\\\hline
\verb|T32| & $\phi40\times R6.6$内面溝加工用Tスロットカッター\\\hline
\verb|T33| & $\phi30\times R15$内面溝加工用Tスロットカッター\\\hline
\hline
\verb|T36| & 10.0in内面溝加工用アングルヘッド\\\hline
\hline
\verb|T41| & 15.5in逃し溝加工用アングルヘッド\\\hline
\hline
\verb|T50| & $\phi10\times200$(延長100)測定用タッチセンサープローブ
\end{twoCtable}

%!TEX root = ../RPA_for_Creating_Program_Note.tex



工具の移動には、主に\verb|G00|, \verb|G01|, \verb|G02|, \verb|G03|, \verb|G31|が用いられる。
一般に、
\begin{enumerate}
\item \verb|G00|は主に\index{はやおくり@早送り}早送りに使用されることが多い
\item \verb|G01|は直線状に移動し、主に切削の際の送りに使用されることが多い
\item \verb|G02|, \verb|G03|は円弧状に移動し、主に切削の際の送りに使用されることが多い
\item \verb|G31|は主に測定の際の\index{スキップきのう@スキップ機能}スキップ機能に伴って使用されることが多い
\end{enumerate}
\verb|G00|は\index{さいだいおくりはやさ@最大送り速さ}最大送り速さで移動し、その速さは(Fコード値では)指定することはできない。
\verb|G01|, \verb|G02|, \verb|G03|, \verb|G31|はその送りの速さを\index{Fコードち@Fコード値}Fコード値で指定することができる。

なお、\verb|G28|, \verb|G30|では\verb|G00|によって移動する。
また、\verb|G31|はすべての工具で指定はすることができるが、スキップ機能を有するものは\index{タッチセンサープローブ}タッチセンサープローブのみである。
%%%%%%%%%%%%%%%%%%%%%%%%%%%%%%%%%%%%%%%%%%%%%%%%%%%%%%%%%%
%% hosoku %%%%%%%%%%%%%%%%%%%%%%%%%%%%%%%%%%%%%%%%%%%%%%%%
%%%%%%%%%%%%%%%%%%%%%%%%%%%%%%%%%%%%%%%%%%%%%%%%%%%%%%%%%%
\begin{hosoku}
この章では工具の具体的な送り速さ値を記述している。
しかし、工具の送り速さ値の具体的な数値は、(ソフトウェアでなく)ハードウェアの標準に記載するほうが望ましい。
\end{hosoku}
%%%%%%%%%%%%%%%%%%%%%%%%%%%%%%%%%%%%%%%%%%%%%%%%%%%%%%%%%%
%%%%%%%%%%%%%%%%%%%%%%%%%%%%%%%%%%%%%%%%%%%%%%%%%%%%%%%%%%
%%%%%%%%%%%%%%%%%%%%%%%%%%%%%%%%%%%%%%%%%%%%%%%%%%%%%%%%%%



%%%%%%%%%%%%%%%%%%%%%%%%%%%%%%%%%%%%%%%%%%%%%%%%%%%%%%%%%%
%% section 07.1 %%%%%%%%%%%%%%%%%%%%%%%%%%%%%%%%%%%%%%%%%%
%%%%%%%%%%%%%%%%%%%%%%%%%%%%%%%%%%%%%%%%%%%%%%%%%%%%%%%%%%
\modHeadsection{送り速さの基本事項}
\begin{enumerate}
\item $X$, $Y$, $Z$方向の\index{いちぎめ@位置決め}位置決め(\index{はやおくり@早送り}早送り)\verb|G00|の\index{おくりはやさ@送り速さ}送り速さデフォルト値:60000mm/min
\item $B$方向の\index{はやおくり(Bじく)@早送り(B軸)}早送り\verb|G00|の送り速さデフォルト値:12000$^\circ$/min ($\sim 33.33$回転/min)
\item $X$, $Y$, $Z$方向の指定できる送り速さ値の範囲:0~60000mm/min
\item 設定可能な最小単位:0.1mm/min
\end{enumerate}



\clearpage
%%%%%%%%%%%%%%%%%%%%%%%%%%%%%%%%%%%%%%%%%%%%%%%%%%%%%%%%%%
%% section 07.2 %%%%%%%%%%%%%%%%%%%%%%%%%%%%%%%%%%%%%%%%%%
%%%%%%%%%%%%%%%%%%%%%%%%%%%%%%%%%%%%%%%%%%%%%%%%%%%%%%%%%%
\modHeadsection{タッチセンサープローブ}
\DMname では、全長の長い\index{タッチセンサープローブ}タッチセンサープローブを用いる。
したがって、\index{おくりはやさ@送り速さ}送り速さを大きくして移動をすると、その加速度によって\index{タッチセンサー}タッチセンサーが反応してしまったり、タッチセンサープローブそのものに大きな負担がかかる。
そのため、タッチセンサーの送り速さに関しては他の工具よりも低めに設定する。
\begin{enumerate}[label=\Roman*., ref=\Roman*]
\item \label{item:FDTS} 原則として、\verb|G00|を用いた移動はしない
\item \ref{item:FDTS}に伴い、\verb|G28|, \verb|G30|を(直接的に)用いた移動はしない
\item \verb|G01|を位置決め(早送り)として用いるものとし、送り速さはF5400以下とする
\item ワークへの\index{アプローチ}アプローチの際は、\verb|G31|を用いるものとし、送り速さはF1500以下とする
\item 計測の際の\index{スキップ}スキップ(\verb|G31|)の送り速さは、計測の仕方に応じて以下のものとする
  \begin{enumerate}
  \item \index{しんごうおくれほせい@信号遅れ補正}信号遅れ補正を考慮する必要があるような場合は、送り速さはF50とする
  \item 信号遅れ補正を考慮する必要がない場合は、送り速さはF50以上300以下とする
  \end{enumerate}
\item $XY$方向におけるワークからの\index{リトラクト}リトラクトの際は\verb|G01|を用いるものとする。
\end{enumerate}
なお、$Z$方向の移動は工具ではなくテーブルが移動するため、アプローチを除いて次節(タッチセンサー以外の工具)にしたがうものとする。

なお、\verb|G31|で送る場合は、タッチセンサープローブの\index{でんげん(タッチセンサープローブ)@電源(タッチセンサープローブ)}電源を入れて行うこと。
電源が入っていなければ移動せず、その点が\index{ブロックエンド}ブロックエンドとなる。



%\clearpage
%%%%%%%%%%%%%%%%%%%%%%%%%%%%%%%%%%%%%%%%%%%%%%%%%%%%%%%%%%
%% section 10.2 %%%%%%%%%%%%%%%%%%%%%%%%%%%%%%%%%%%%%%%%%%
%%%%%%%%%%%%%%%%%%%%%%%%%%%%%%%%%%%%%%%%%%%%%%%%%%%%%%%%%%
\modHeadsection{タッチセンサー以外の工具}
\DMname では、$Z$方向については\index{テーブル}テーブル(\index{パレット(\DMname)@パレット(\DMname)}パレット)が移動する。
したがって、$Z$方向の送り速さを大きくして移動すると、その加速度によってテーブル上の\index{ジグ}ジグが動いてしまう恐れがある。
そのため、$Z$方向の送り速さに関しては$X$, $Y$方向の送り速さより低めに設定する。
\begin{enumerate}[label=\Roman*., ref=\Roman*]
\item $X$, $Y$方向の移動は、\verb|G00|を用いてもよい
\item $Z$方向の移動は、原則として、\verb|G00|を用いた移動はしない
\item $Z$方向の送り速さは、F12000以下とする
\item ワークへの\index{アプローチ}アプローチの際は、\verb|G01|を用いるものとし、速さはF6000以下とする
\item 加工の際は、それぞれの加工や状況に応じて適切なFコード値を設定する
\item ワークからの\index{リトラクト}リトラクトの際は\verb|G01|を用いるものとし、速さはF12000以下とする。
\end{enumerate}


\clearpage
\noindent
\dateKouguSpeed における送り速さの\index{せっていち(おくりはやさ)@設定値(送り速さ)}設定値は、以下のとおりである。\\

\begin{3columnstable}{工具の送り速さ値 一覧}{|Sl|Sc|Sc|}{内容}{速さ値}{\ttfamily G\ttNum   }
早送り$XY$:\verb|T50|以外 & 60000 & \verb|G00|\\\hline
早送り$XY$:\verb|T50|    & 5400 & \\\hline
早送り$Z$:\verb|T50|アプローチ以外 & 12000 &\\\hline
早送り$Z$:\verb|T50|アプローチ    & 5400 & \\\hline
アプローチ:\verb|T50|以外 & 6000 & \\\hline
アプローチ:\verb|T50|    & 1500 & \verb|G31|\\\hline
アプローチ:工具長測定 & 1000 & \verb|G31|\\\hline
測定時アプローチ:\verb|T50|片側測定 & 50 & \verb|G31|\\\hline
測定時アプローチ:\verb|T50|両側測定 & 200 & \verb|G31|\\\hline
端面加工:直線             & 900 &\\\hline
端面加工:コーナー          & 900 &\\\hline
外削加工:直線             & 400 &\\\hline
外削加工:コーナー          & 400 &\\\hline
溝加工:直線               & 500 &\\\hline
溝加工:コーナー            & 400 &\\\hline
外面取加工:直線            & 500 &\\\hline
外面取加工:コーナー         & 400 &\\\hline
内面取加工:直線            & 400 &\\\hline
内面取加工:コーナー         & 400 &\\\hline
座ぐり加工:直線            & 40 &\\\hline
座ぐり加工:コーナー         & 40 &\\\hline
\dimple 加工:表面アプローチ & 540 &\\\hline
\dimple 加工:加工         & 100 &\\
\end{3columnstable}



%!TEX root = ../RfCPN.tex


\modHeadchapter[lot]{工具の\index{しゅじくかいてんすう@主軸回転数}主軸回転数(\index{Sコードち@Sコード値}Sコード値)\TBW}
%%%%%%%%%%%%%%%%%%%%%%%%%%%%%%%%%%%%%%%%%%%%%%%%%%%%%%%%%%
%% marker %%%%%%%%%%%%%%%%%%%%%%%%%%%%%%%%%%%%%%%%%%%%%%%%
%%%%%%%%%%%%%%%%%%%%%%%%%%%%%%%%%%%%%%%%%%%%%%%%%%%%%%%%%%
\begin{marker}
この章では工具の具体的な\index{しゅじくかいてんすう@主軸回転数}主軸回転数(\index{スピンドルかいてんすう@スピンドル回転数}スピンドル回転数)を記述している。
しかし、工具の主軸回転数の具体的な数値は、(ソフトウェアでなく)ハードウェアの標準に記載するほうが望ましい。
\end{marker}
%%%%%%%%%%%%%%%%%%%%%%%%%%%%%%%%%%%%%%%%%%%%%%%%%%%%%%%%%%
%%%%%%%%%%%%%%%%%%%%%%%%%%%%%%%%%%%%%%%%%%%%%%%%%%%%%%%%%%
%%%%%%%%%%%%%%%%%%%%%%%%%%%%%%%%%%%%%%%%%%%%%%%%%%%%%%%%%%



%%%%%%%%%%%%%%%%%%%%%%%%%%%%%%%%%%%%%%%%%%%%%%%%%%%%%%%%%%
%% section 13.1 %%%%%%%%%%%%%%%%%%%%%%%%%%%%%%%%%%%%%%%%%%
%%%%%%%%%%%%%%%%%%%%%%%%%%%%%%%%%%%%%%%%%%%%%%%%%%%%%%%%%%
\modHeadsection{\index{しゅじくかいてんすう@主軸回転数}主軸回転数の基本事項\TBW}
\begin{enumerate}[label=\sarrow]
\item \index{でんげんとうにゅうじ(きかいほんたい)@電源投入時(機械本体)}電源投入時の\index{しゅじくかいてんすう@主軸回転数}主軸回転数(設定値):0
\item 設定可能な最小単位:1回転/min
\item \index{ハイギア(しゅじくかいてん)@ハイギア(主軸回転)}ハイギア(high gear)と\index{ローギア(しゅじくかいてん)@ローギア(主軸回転)}ローギア(low gear)の2種類がある%
%% footnote %%%%%%%%%%%%%%%%%%%%%
\footnote{機能はしないが、操作盤には\index{ミドルギア(しゅじくかいてん)@ミドルギア(主軸回転)}ミドルギアのボタンがある。}
%%%%%%%%%%%%%%%%%%%%%%%%%%%%%%%%%

\item 回転数(設定値)2600以上でハイギアに、2599以下でローギアとなる
\item 設定可能最低値は35
\end{enumerate}
~\newline\noindent
\dateKouguRotation 時点における主軸回転数の\index{せっていち(しゅじくかいてんすう)@設定値(主軸回転数)}設定値は、以下のとおりである。\\

\begin{multicollongtblr}{工具の\index{しゅじくかいてんすう@主軸回転数}主軸回転数設定値 一覧}{X[l]c}
加工の種類 & 回転数\\
\EndFacecutMilling & \EndFaceSpindleSpeed\\
\OutcutMilling & \OutcutSpindleSpeed\\
\KeywayMilling & \KeywaySpindleSpeed\\
\EndFaceOutCChamferMilling & \OutCChamferSpindleSpeed\\
\EndFaceInCChamferMilling & \InCChamferSpindleSpeed\\
\EndFaceBoringMilling & \EndFaceBoringSpindleSpeed\\
\DimpleMilling(\TSlotCutter) & \DimpleTSlotSpindleSpeed\\
\DimpleMilling(\BallEndMill) & \DimpleBallendSpindleSpeed\\
\end{multicollongtblr}


%!TEX root = ../RPA_for_Creating_Program_Note.tex



一般に、\index{シーケンスばんごう@シーケンス番号}シーケンス番号は(重複していなければ)自由に付けて問題ない。
しかしこれに一定のルールを与えておくことで、プログラムの
\begin{enumerate}
\item どの部分で何が行われているか
\item どの部分でエラーが起きているか
\item 途中から稼働する場合、どのブロックから始めればよいか
\end{enumerate}
など、作業や管理をする際に効率よく制御することが可能になる。

そこで、ここでは\index{シーケンスばんごう@シーケンス番号}シーケンス番号(\index{Nコード}Nコード)についての規則を与える。



%%%%%%%%%%%%%%%%%%%%%%%%%%%%%%%%%%%%%%%%%%%%%%%%%%%%%%%%%%
%% section 15.1 %%%%%%%%%%%%%%%%%%%%%%%%%%%%%%%%%%%%%%%%%%
%%%%%%%%%%%%%%%%%%%%%%%%%%%%%%%%%%%%%%%%%%%%%%%%%%%%%%%%%%
\modHeadsection{シーケンス番号の基本事項}
\begin{enumerate}[label=\Roman*), ref=\Roman*)]
\item シーケンス番号は3桁とし、0埋めする
\item プログラムの始まりのシーケンス番号は\verb|N001|とする
\item 原則として、シーケンス番号は昇順とし、特に1桁目は連番とする
\end{enumerate}


%%%%%%%%%%%%%%%%%%%%%%%%%%%%%%%%%%%%%%%%%%%%%%%%%%%%%%%%%%
%% section 15.2 %%%%%%%%%%%%%%%%%%%%%%%%%%%%%%%%%%%%%%%%%%
%%%%%%%%%%%%%%%%%%%%%%%%%%%%%%%%%%%%%%%%%%%%%%%%%%%%%%%%%%
\modHeadsection{サブプログラム}
\DMname においては、原則としてサブプログラムは始めから実行されるものであり、途中の部分から実行されることはない。
そのため、\index{サブプログラム}サブプログラムについてはシーケンス番号は記述の順に(概ね\index{ブロック}ブロックごとに)連番とする。

なお、エラー検出時に関するシーケンス番号、およびプログラム終了に関するシーケンス番号については、以降で述べるメインプログラムのもの(\autoref{subsec:sequenceNerror}, \pageautoref{subsec:sequenceNprgEnd})と同様とする。



\clearpage
%%%%%%%%%%%%%%%%%%%%%%%%%%%%%%%%%%%%%%%%%%%%%%%%%%%%%%%%%%
%% section 15.3 %%%%%%%%%%%%%%%%%%%%%%%%%%%%%%%%%%%%%%%%%%
%%%%%%%%%%%%%%%%%%%%%%%%%%%%%%%%%%%%%%%%%%%%%%%%%%%%%%%%%%
\modHeadsection{メインプログラム}
\DMname において、\index{メインプログラム}メインプログラム
%% footnote %%%%%%%%%%%%%%%%%%%%%
\footnote{ここでいうメインプログラムとは、下5桁が\index{ずめんばんごう@図面番号}図面番号と一致するものを指す。}
%%%%%%%%%%%%%%%%%%%%%%%%%%%%%%%%%
は\index{さぎょうしゃ@作業者}作業者が実際に設定を変更したり、途中の箇所から始めたりし得る。
そのため、メインプログラムについては各作業(計測・加工)ごとにシーケンス番号を割り振ることにする。


%%%%%%%%%%%%%%%%%%%%%%%%%%%%%%%%%%%%%%%%%%%%%%%%%%%%%%%%%%
%% subsection 15.3.1 %%%%%%%%%%%%%%%%%%%%%%%%%%%%%%%%%%%%%
%%%%%%%%%%%%%%%%%%%%%%%%%%%%%%%%%%%%%%%%%%%%%%%%%%%%%%%%%%
\subsection{N100:計測(\dimple ・逃し溝以外)}
\index{タッチセンサー}タッチセンサーを用いた計測(\dimple ・逃し溝を除く)を行う工程のシーケンス番号は100番台とする。
これには以下の\index{こうてい@工程}工程が含まれ、これらは2桁目の番号で区別される。
\begin{enumerate}
\item[100:] 芯出し計測
\item[650:] 通り芯$X$計測
\item[660:] 通り芯$Y$計測
\end{enumerate}


%\clearpage
%%%%%%%%%%%%%%%%%%%%%%%%%%%%%%%%%%%%%%%%%%%%%%%%%%%%%%%%%%
%% subsection 15.3.2 %%%%%%%%%%%%%%%%%%%%%%%%%%%%%%%%%%%%%
%%%%%%%%%%%%%%%%%%%%%%%%%%%%%%%%%%%%%%%%%%%%%%%%%%%%%%%%%%
\subsection{N200:計測(\dimple)}
\expandafterindex{\dimplekana@\dimple}\dimple および\index{にがしみぞ@逃し溝}逃し溝に関する\index{タッチセンサープローブ}タッチセンサープローブを用いた計測を行う工程のシーケンス番号は200番台とする。
これには以下の工程が含まれ、これらは2桁目の番号で区別される。
\begin{enumerate}
\item[200:] \dimple 計測
\item[250:] 逃し溝計測
\end{enumerate}


%%%%%%%%%%%%%%%%%%%%%%%%%%%%%%%%%%%%%%%%%%%%%%%%%%%%%%%%%%
%% subsection 14.2.1 %%%%%%%%%%%%%%%%%%%%%%%%%%%%%%%%%%%%%
%%%%%%%%%%%%%%%%%%%%%%%%%%%%%%%%%%%%%%%%%%%%%%%%%%%%%%%%%%
\subsection{N300:\dimple ・逃し溝加工}
\dimple および逃し溝加工を行う工程のシーケンス番号は300番台とする。
これには以下の工程が含まれ、これらは2桁目の番号で区別される。
\begin{enumerate}
\item[300:] \dimple 加工
\item[350:] 逃し溝加工
\end{enumerate}


%%%%%%%%%%%%%%%%%%%%%%%%%%%%%%%%%%%%%%%%%%%%%%%%%%%%%%%%%%
%% subsection 14.2.1 %%%%%%%%%%%%%%%%%%%%%%%%%%%%%%%%%%%%%
%%%%%%%%%%%%%%%%%%%%%%%%%%%%%%%%%%%%%%%%%%%%%%%%%%%%%%%%%%
\subsection{N400:トップ側の加工}
トップ側の加工を行う工程のシーケンス番号は400番台とする。
これには以下の工程が含まれ、これらは2桁目の番号で区別される。
\begin{enumerate}
\item[400:] トップ端面加工
\item[410:] トップ外削加工
\item[420:] 溝加工
\item[430:] トップ外面取加工
\item[440:] トップ内面取加工
\item[450:] 座ぐり加工
\end{enumerate}


\clearpage
%%%%%%%%%%%%%%%%%%%%%%%%%%%%%%%%%%%%%%%%%%%%%%%%%%%%%%%%%%
%% subsection 14.2.1 %%%%%%%%%%%%%%%%%%%%%%%%%%%%%%%%%%%%%
%%%%%%%%%%%%%%%%%%%%%%%%%%%%%%%%%%%%%%%%%%%%%%%%%%%%%%%%%%
\subsection{N500:ボトム側の加工}
ボトム側の加工を行う工程のシーケンス番号は500番台とする。
これには以下の工程が含まれ、これらは2桁目の番号で区別される。
\begin{enumerate}
\item[500:] ボトム端面加工
\item[510:] ボトム外削加工
\item[530:] ボトム外面取加工
\item[540:] ボトム内面取加工
\end{enumerate}


%\clearpage
%%%%%%%%%%%%%%%%%%%%%%%%%%%%%%%%%%%%%%%%%%%%%%%%%%%%%%%%%%
%% subsection 14.2.1 %%%%%%%%%%%%%%%%%%%%%%%%%%%%%%%%%%%%%
%%%%%%%%%%%%%%%%%%%%%%%%%%%%%%%%%%%%%%%%%%%%%%%%%%%%%%%%%%
\subsection{N800:エラー\label{subsec:sequenceNerror}\TBW}
\index{エラー}エラー検出時に\index{ジャンプ}ジャンプするシーケンス番号は800番台とする。
エラーの種類(システム変数\ttNum3000の値)に応じて以下のように分類し、(概ね)\index{ブロック}プロックごとに連番とする。
\begin{enumerate}
\item[800:] \verb|#3000=121 (Argument is not assigned)|
\item[810:] \verb|#3000=...|
\item[820:] \verb|#3000=1 (Pallet Alarm)|,\\
            \verb|#3000=145 (Sensor-Low-Battery)|, \verb|#3000=146 (Sensor-Alarm)|
\end{enumerate}


%%%%%%%%%%%%%%%%%%%%%%%%%%%%%%%%%%%%%%%%%%%%%%%%%%%%%%%%%%
%% subsection 14.2.1 %%%%%%%%%%%%%%%%%%%%%%%%%%%%%%%%%%%%%
%%%%%%%%%%%%%%%%%%%%%%%%%%%%%%%%%%%%%%%%%%%%%%%%%%%%%%%%%%
\subsection{N990:プログラムの終了\label{subsec:sequenceNprgEnd}}
\index{プログラムしゅうりょう@プログラム終了}プログラムを終了する工程のシーケンス番号は990番台とする。
特に、プログラムの終了はN999とする。



\clearpage
\noindent
改めて上記の\index{シーケンスばんごういちらん@シーケンス番号一覧}シーケンス番号を一覧にしておく。\\

%%%%%%%%%%%%%%%%%%%%%%%%%%%%%%%%%%%%%%%%%%%%%%%%%%%%%%%%%%
%% sequence numbers %%%%%%%%%%%%%%%%%%%%%%%%%%%%%%%%%%%%%%
%%%%%%%%%%%%%%%%%%%%%%%%%%%%%%%%%%%%%%%%%%%%%%%%%%%%%%%%%%
\begin{2columnstable}{シーケンス番号 一覧(メインプログラム)\TBW}{|Sc|Sl|}{N番号}{内容}
\verb|N001| & プログラムの始まり\\\hline
\hline
\verb|N10x| & タッチセンサー計測(芯出し)\\\hline
\hline
\verb|N20x| & タッチセンサー計測(\dimple)\\\hline
\verb|N25x| & タッチセンサー計測(逃し溝)\\\hline
\hline
\verb|N30x| & \dimple 加工\\\hline
\verb|N35x| & 逃し溝加工\\\hline
\hline
\verb|N40x| & トップ端面加工\\\hline
\verb|N41x| & トップ外削加工\\\hline
\verb|N42x| & 溝加工\\\hline
\verb|N43x| & トップ外面取加工\\\hline
\verb|N44x| & トップ内面取加工\\\hline
\verb|N45x| & 座ぐり加工\\\hline
\hline
\verb|N500| & ボトム端面加工\\\hline
\verb|N51x| & ボトム外削加工\\\hline
\verb|N52x| & ボトム外面取加工\\\hline
\verb|N53x| & ボトム内面取加工\\\hline
\hline
\verb|N65x| & タッチセンサー計測(通り芯$X$)\\\hline
\verb|N66x| & タッチセンサー計測(通り芯$Y$)\\\hline
\hline
\verb|N80x| & 引数によるエラー\\\hline
\verb|N81x|\TBW & \\\hline
\verb|N82x| & パレットまたはタッチセンサーによるエラー\\\hline
\hline
\verb|N99x| & プログラム終了の工程\\\hline
\verb|N999| & プログラム終了(\verb|M02|または\verb|M30|または\verb|M99|)
\end{2columnstable}



%!TEX root = ../RPA_for_Creating_Program_Note.tex



ここでは\DMname におけるアラーム発生システム変数\hx\ttNum3000について記述する。


%%%%%%%%%%%%%%%%%%%%%%%%%%%%%%%%%%%%%%%%%%%%%%%%%%%%%%%%%%
%% section 17.1 %%%%%%%%%%%%%%%%%%%%%%%%%%%%%%%%%%%%%%%%%%
%%%%%%%%%%%%%%%%%%%%%%%%%%%%%%%%%%%%%%%%%%%%%%%%%%%%%%%%%%
\modHeadsection{アラームの分類:\DMname}
\DMname で使用される\index{アラームへんすう@アラーム変数}アラーム変数\hx\ttNum3000の値は、以下のように分類される。\\

%%%%%%%%%%%%%%%%%%%%%%%%%%%%%%%%%%%%%%%%%%%%%%%%%%%%%%%%%%
%% Alerm Classification %%%%%%%%%%%%%%%%%%%%%%%%%%%%%%%%%%
%%%%%%%%%%%%%%%%%%%%%%%%%%%%%%%%%%%%%%%%%%%%%%%%%%%%%%%%%%
\begin{multicollongtblr}{\DMname のアラーム番号の分類}{cX[l]X[l]}
番号 & メッセージ & 異常の内容\\
001 & pallet alarm & パレット\ttNum\\
002 & touch sensor alarm & {\ttfamily T50}タッチセンサープローブ\\
003 & tool measurement alarm & 工具長補正計測\\
100 & work coordinate is not assigned & ワーク座標系指定なし\\
200 & argument is not assigned & 引数指定なし\\
201 & are the arguments OK? & 引数の値\\
202 & are the arguments OK? & 引数の値の関係\\
203 & are the common variables OK? & コモン変数の値\\
204 & is the measurement value OK? & 測定値の値\\
\end{multicollongtblr}

\clearpage
%%%%%%%%%%%%%%%%%%%%%%%%%%%%%%%%%%%%%%%%%%%%%%%%%%%%%%%%%%
%% Alerm Classification %%%%%%%%%%%%%%%%%%%%%%%%%%%%%%%%%%
%%%%%%%%%%%%%%%%%%%%%%%%%%%%%%%%%%%%%%%%%%%%%%%%%%%%%%%%%%
\begin{multicollongtblr}{\DMname のアラーム番号の分類(バンドルのプログラム)}{cX[l]}
番号 & メッセージ\\
001 & Pallet Alarm\\
002 & Pallet Alarm\\
003 & Pallet Alarm\\
007 & Tool-Life-Check\\
010 & Macro Pallet Check Alarm\\
011 & No Program Selected\\
012 & Wrong Pallet Alarm\\
121 & Argument Is Not Assigned\\
140 & A-Axis-Is-Not-Command\\
141 & Tool-Measure-Alarm, Tool-Check, Measured-Value-Is-Over\\
142 & Tool-Brake\\
143 & Check-X,Y-Axis-Command-Value, Work-Coordinate-Setting-Error\\
145 & MP10/MP12/MP60-Low-Battery\\
146 & MP10/MP12/MP60-Alarm\\
\end{multicollongtblr}

%!TEX root = ../RfCPN.tex


\modHeadchapter[lot]{コモン変数(\DMC)\label{chap:commonvariablesDM}\vphantom{\ref{chap:commonvariablesDM}}}
ここでは\DMC の加工システムで使用している\expandafterindex{コモンへんすう(\yomiDMC)@コモン変数(\nameDMC)}コモン変数について述べる。


%%%%%%%%%%%%%%%%%%%%%%%%%%%%%%%%%%%%%%%%%%%%%%%%%%%%%%%%%%
%% section 11.1 %%%%%%%%%%%%%%%%%%%%%%%%%%%%%%%%%%%%%%%%%%
%%%%%%%%%%%%%%%%%%%%%%%%%%%%%%%%%%%%%%%%%%%%%%%%%%%%%%%%%%
\modHeadsection{コモン変数の範囲}
\DMC で使用可能なコモン変数は以下のとおりである。
\begin{enumerate}[label=\sarrow]
\item \ttNum100\,-\ttNum199
\item \ttNum400\,-\ttNum999
\item \ttNum900000\,-\ttNum907399
\end{enumerate}
%%%%%%%%%%%%%%%%%%%%%%%%%%%%%%%%%%%%%%%%%%%%%%%%%%%%%%%%%%
%% marker %%%%%%%%%%%%%%%%%%%%%%%%%%%%%%%%%%%%%%%%%%%%%%%%
%%%%%%%%%%%%%%%%%%%%%%%%%%%%%%%%%%%%%%%%%%%%%%%%%%%%%%%%%%
\begin{marker}
メーカーのマニュアルには\ttNum400\,-\ttNum999でなく\ttNum500\,-\ttNum999と記載されている誤植があるので注意
\end{marker}
%%%%%%%%%%%%%%%%%%%%%%%%%%%%%%%%%%%%%%%%%%%%%%%%%%%%%%%%%%
%%%%%%%%%%%%%%%%%%%%%%%%%%%%%%%%%%%%%%%%%%%%%%%%%%%%%%%%%%
%%%%%%%%%%%%%%%%%%%%%%%%%%%%%%%%%%%%%%%%%%%%%%%%%%%%%%%%%%



%%%%%%%%%%%%%%%%%%%%%%%%%%%%%%%%%%%%%%%%%%%%%%%%%%%%%%%%%%
%% section 18.2 %%%%%%%%%%%%%%%%%%%%%%%%%%%%%%%%%%%%%%%%%%
%%%%%%%%%%%%%%%%%%%%%%%%%%%%%%%%%%%%%%%%%%%%%%%%%%%%%%%%%%
\modHeadsection{\ttNum100\,-\ttNum199}
\ttNum100\,-\ttNum174については、(機械設置時の)\index{バンドルのNCプログラム}バンドルのNCプログラムやカスタマイズされた\index{Mコード}Mコードで既に使用されているものが多いため、基本的にはメインとしては使用しないものとし、一時的な\index{LHS(コモンへんすう)@LHS(コモン変数)}LHS, あるいは\index{RHS(コモンへんすう)@RHS(コモン変数)}RHSとして扱うものとする。


%%%%%%%%%%%%%%%%%%%%%%%%%%%%%%%%%%%%%%%%%%%%%%%%%%%%%%%%%%
%% subsection 18.2.1 %%%%%%%%%%%%%%%%%%%%%%%%%%%%%%%%%%%%%
%%%%%%%%%%%%%%%%%%%%%%%%%%%%%%%%%%%%%%%%%%%%%%%%%%%%%%%%%%
\subsection{\ttNum100\,-\ttNum174:一時保存値}
\noindent\ttNum100\,-\ttNum109については、主に一時的な保存に用いるものとする。\\

\begin{multicollongtblr}[white]{\ttNum100\,-\ttNum109:一時保存値}{cX[l]}
変数 & 内容\\
\ttNum100 & 各工程 切削回数用 一時保存値(仕上げ前 全削り代$X$ or $Z$)\\
\ttNum101 & 各工程 切削回数用 一時保存値(仕上げ前 全削り代$Y$)\\
\ttNum102 & 各工程 切削回数用 一時保存値 (max[\ttNum100, \ttNum101])\\
\ttNum103 & 各工程 切削回数用 一時保存値(仕上げ前 切削回数)\\
\ttNum104 & 各工程 切削回数用 一時保存値(加工時 径$X$)\\
\ttNum105 & 各工程 切削回数用 一時保存値(加工時 径$Y$)\\
\ttNum106 & 各工程 切削回数用 一時保存値(仕上げ 切削回数)\\
\SetRow{unusingVariables}
\ttNum107 & (予備)\\
\SetRow{unusingVariables}
\ttNum108 & (予備)\\
\SetRow{unusingVariables}
\ttNum109 & (予備)\\
\end{multicollongtblr}


\clearpage
%%%%%%%%%%%%%%%%%%%%%%%%%%%%%%%%%%%%%%%%%%%%%%%%%%%%%%%%%%
%% subsection 18.2.2 %%%%%%%%%%%%%%%%%%%%%%%%%%%%%%%%%%%%%
%%%%%%%%%%%%%%%%%%%%%%%%%%%%%%%%%%%%%%%%%%%%%%%%%%%%%%%%%%
\subsection{\ttNum175\,-\ttNum199:各工程用 補助機能(使用頻度 低)}
\noindent\ttNum175\,-\ttNum199については、使用頻度が低いと思われる\index{ほじょきのう(コモンへんすう)@補助機能(コモン変数)}補助機能のものとする。\\

\begin{multicollongtblr}[white]{\ttNum175\,-\ttNum199:補助機能(使用頻度 低)}{cX[l]c}
変数 & 内容 & 設定例\\
\ttNum175 & 測定・加工 開始N番号 & 0\\
\SetRow{unusingVariables}
\ttNum176 & (不使用) &\\
\ttNum177 & 芯出し測定後 一時停止 (0:{\ttfamily M00}, 1:non-stop) & 1\\
\ttNum178 & \Dimple 測定後 一時停止 (0:{\ttfamily M00}, 1:non-stop) & 1\\
\ttNum179 & \TopEndFacecutMilling 後 一時停止 (0:{\ttfamily M00}, 1:non-stop, 2:扉前{\ttfamily M00}) & 1\\
\ttNum180 & \BottomEndFacecutMilling 後 一時停止 (0:{\ttfamily M00}, 1:non-stop, 2:扉前{\ttfamily M00}) & 1\\
\SetRow{unusingVariables}
\ttNum181 & (不使用) &\\
\ttNum182 & \TopOutcut{} 仕上げ加工 追加回数 (上限3) & 0\\
\ttNum183 & \Keyway{} 仕上げ加工 追加回数 (上限3) & 0\\
\ttNum184 & \TopEndFaceOutCChamfer{} 仕上げ加工 追加回数 (上限3) & 0\\
\ttNum185 & \TopEndFaceInCChamfer{} 仕上げ加工 追加回数 (上限3) & 0\\
\ttNum186 & \EndFaceBoring{} 仕上げ加工 追加回数 (上限3) & 0\\
\SetRow{unusingVariables}
\ttNum187 & (不使用) &\\
\ttNum188 & \BottomOutcut{} 仕上げ加工 追加回数 (上限3) & 0\\
\ttNum189 & \BottomEndFaceOutCChamfer{} 仕上げ加工 追加回数 (上限3) & 0\\
\ttNum190 & \BottomEndFaceInCChamfer{} 仕上げ加工 追加回数 (上限3) & 0\\
\SetRow{unusingVariables}
\ttNum191 & (不使用) &\\
\SetRow{unusingVariables}
\ttNum192 & (予備) &\\
\SetRow{unusingVariables}
\ttNum193 & (予備) &\\
\SetRow{unusingVariables}
\ttNum194 & (予備) &\\
\SetRow{unusingVariables}
\ttNum195 & (予備) &\\
\SetRow{unusingVariables}
\ttNum196 & (予備) &\\
\SetRow{unusingVariables}
\ttNum197 & (予備) &\\
\SetRow{unusingVariables}
\ttNum198 & (予備) &\\
\SetRow{unusingVariables}
\ttNum199 & (予備) &\\
\end{multicollongtblr}



\clearpage
%%%%%%%%%%%%%%%%%%%%%%%%%%%%%%%%%%%%%%%%%%%%%%%%%%%%%%%%%%
%% section 18.3 %%%%%%%%%%%%%%%%%%%%%%%%%%%%%%%%%%%%%%%%%%
%%%%%%%%%%%%%%%%%%%%%%%%%%%%%%%%%%%%%%%%%%%%%%%%%%%%%%%%%%
\modHeadsection{\ttNum400\,-\ttNum499:各工程用 補助機能(使用頻度 高)}
\ttNum400\,-\ttNum499については、使用頻度が高いと思われる\index{ほじょきのう(コモンへんすう)@補助機能(コモン変数)}補助機能のものとする。\\
ただし、\ttNum475\,-\ttNum499については、該当する\index{めいさい(モールド)@明細(モールド)}明細が少ないと思われる補助機能のものとする。


%\clearpage
%%%%%%%%%%%%%%%%%%%%%%%%%%%%%%%%%%%%%%%%%%%%%%%%%%%%%%%%%%
%% subsection 17.3.1 %%%%%%%%%%%%%%%%%%%%%%%%%%%%%%%%%%%%%
%%%%%%%%%%%%%%%%%%%%%%%%%%%%%%%%%%%%%%%%%%%%%%%%%%%%%%%%%%
\subsection{\ttNum400\,-\ttNum424:初期設定および調整}

\begin{multicollongtblr}[white]{\ttNum400\,-\ttNum424:補助機能(使用頻度 高)}{cX[l]c}
変数 & 内容 & 設定例\\
\ttNum400 & \EndFacecut{} 全削り代 & 8.0\\
\ttNum401 & \expandafterindex{\yomiDimple しかこう@\nameDimple 試加工}\Dimple 試加工 (0:off, 1:on) & 0\\
\ttNum402 & \expandafterindex{\yomiCenterlineEndFaceDif そくてい@\nameCenterlineEndFaceDif 測定}\nameCenterlineEndFaceDif 測定 (0:off, 1:on) & 0\\
\SetRow{unusingVariables}
\ttNum403 & (不使用) &\\
\ttNum404 & {\ttfamily G54}$X$:ボトム外$X$芯出し(両側・片側測定)測定位置$Z-$補正($X$自動補正) & 0\\
\ttNum405 & {\ttfamily G54}$Y$:ボトム外$Y$芯出し(両側測定)測定位置$Z-$補正 & 0\\
\ttNum406 & {\ttfamily G55}$X$:ボトム内$X$芯出し(両側測定)測定位置$Z-$補正($X$自動補正) & 0\\
\ttNum407 & {\ttfamily G55}$Y$:ボトム内$Y$芯出し(両側測定)測定位置$Z-$補正 & 0\\
\SetRow{unusingVariables}
\ttNum408 & (不使用) &\\
\ttNum409 & {\ttfamily G56}$X$:トップ外$X$芯出し(両側・片側測定)測定位置$Z-$補正($X$自動補正) & 0\\
\ttNum410 & {\ttfamily G56}$Y$:トップ外$Y$芯出し(両側測定)測定位置$Z-$補正 & 0\\
\ttNum411 & {\ttfamily G57}$X$:トップ内$X$芯出し(両側測定)測定位置$Z-$補正($X$自動補正) & 0\\
\ttNum412 & {\ttfamily G57}$Y$:トップ内$Y$芯出し(両側測定)測定位置$Z-$補正 & 0\\
\SetRow{unusingVariables}
\ttNum413 & (不使用) &\\
\ttNum414 & \TopOutcutAsideThickness$+$補正(\OutcutCenter$X+$補正) & 0\\
\ttNum415 & \TopOutcutWidth(直径)$+$補正 & 0\\
\ttNum416 & \TopOutcut{} 仕上げ前 一時停止 (0:{\ttfamily M00}, 1:non-stop, 2:扉前{\ttfamily M00}) & 1\\
\ttNum417 & \TopOutcut{} 仕上げ後 一時停止 (0:{\ttfamily M00}, 1:non-stop, 2:扉前{\ttfamily M00}) & 1\\
\SetRow{unusingVariables}
\ttNum418 & (不使用) &\\
\ttNum419 & \KeywayPos$+$補正(\KeywayWidth 不変) & 0\\
\ttNum420 & \KeywayWidth$+$補正 & 0\\
\ttNum421 & \AsideKeywayDepth$+$補正(\KeywayCenter$X-$補正) & 0\\
\ttNum422 & \KeywayDiameter(直径)$+$補正 & 0\\
\ttNum423 & \Keyway 仕上げ前 一時停止 (0:{\ttfamily M00}, 1:non-stop, 2:扉前{\ttfamily M00}) & 1\\
\ttNum424 & \Keyway 仕上げ後 一時停止 (0:{\ttfamily M00}, 1:non-stop, 2:扉前{\ttfamily M00}) & 1\\
\end{multicollongtblr}
%%%%%%%%%%%%%%%%%%%%%%%%%%%%%%%%%%%%%%%%%%%%%%%%%%%%%%%%%%
%% marker %%%%%%%%%%%%%%%%%%%%%%%%%%%%%%%%%%%%%%%%%%%%%%%%
%%%%%%%%%%%%%%%%%%%%%%%%%%%%%%%%%%%%%%%%%%%%%%%%%%%%%%%%%%
\begin{marker}
\expandafterindex{\yomiOutcutCenter そくてい@\nameOutcutCenter 測定}\nameOutcutCenter 測定\MXIface で\EndFaceInChamfer がある場合、測定位置$Z$は\EndFace の$Z$位置でないことに注意
\end{marker}
%%%%%%%%%%%%%%%%%%%%%%%%%%%%%%%%%%%%%%%%%%%%%%%%%%%%%%%%%%
%%%%%%%%%%%%%%%%%%%%%%%%%%%%%%%%%%%%%%%%%%%%%%%%%%%%%%%%%%
%%%%%%%%%%%%%%%%%%%%%%%%%%%%%%%%%%%%%%%%%%%%%%%%%%%%%%%%%%


\clearpage
%%%%%%%%%%%%%%%%%%%%%%%%%%%%%%%%%%%%%%%%%%%%%%%%%%%%%%%%%%
%% subsection 18.3.3 %%%%%%%%%%%%%%%%%%%%%%%%%%%%%%%%%%%%%
%%%%%%%%%%%%%%%%%%%%%%%%%%%%%%%%%%%%%%%%%%%%%%%%%%%%%%%%%%
\subsection{\ttNum425\,-\ttNum449:初期設定および調整(続き)}

\begin{multicollongtblr}[white]{\ttNum425\,-\ttNum449:補助機能(使用頻度 高)続き}{cX[l]c}
変数 & 内容 & 設定例\\
\ttNum425 & \TopEndFaceOutCChamfer$X+$補正 & 0\\
\ttNum426 & \TopEndFaceOutCChamferWidth(直径)$+$補正 & 0\\
\ttNum427 & \TopEndFaceOutCChamfer{} 仕上げ前 一時停止 (0:{\ttfamily M00}, 1:non-stop, 2:扉前{\ttfamily M00}) & 1\\
\ttNum428 & \TopEndFaceOutCChamfer{} 仕上げ後 一時停止 (0:{\ttfamily M00}, 1:non-stop, 2:扉前{\ttfamily M00}) & 1\\
\SetRow{unusingVariables}
\ttNum429 & (不使用) &\\
\ttNum430 & \TopEndFaceInCChamfer$X+$補正 & 0\\
\ttNum431 & \TopEndFaceInCChamferWidth(直径)$+$補正 & 0\\
\ttNum432 & \TopEndFaceInCChamfer{} 仕上げ前 一時停止 (0:{\ttfamily M00}, 1:non-stop, 2:扉前{\ttfamily M00}) & 1\\
\ttNum433 & \TopEndFaceInCChamfer{} 仕上げ後 一時停止 (0:{\ttfamily M00}, 1:non-stop, 2:扉前{\ttfamily M00}) & 1\\
\SetRow{unusingVariables}
\ttNum434 & (不使用) &\\
\SetRow{unusingVariables}
\ttNum435 & (不使用) &\\
\ttNum436 & \BottomOutcutAsideThickness$+$補正(\OutcutCenter$X-$補正) & 0\\
\ttNum437 & \BottomOutcutWidth(直径)$+$補正 & 0\\
\ttNum438 & \BottomOutcut{} 仕上げ前 一時停止 (0:{\ttfamily M00}, 1:non-stop, 2:扉前{\ttfamily M00}) & 1\\
\ttNum439 & \BottomOutcut{} 仕上げ後 一時停止 (0:{\ttfamily M00}, 1:non-stop, 2:扉前{\ttfamily M00}) & 1\\
\SetRow{unusingVariables}
\ttNum440 & (不使用) &\\
\ttNum441 & \BottomEndFaceOutCChamfer$X+$補正 & 0\\
\ttNum442 & \BottomEndFaceOutCChamferWidth(直径)$+$補正 & 0\\
\ttNum443 & \BottomEndFaceOutCChamfer{} 仕上げ前 一時停止 (0:{\ttfamily M00}, 1:non-stop, 2:扉前{\ttfamily M00}) & 1\\
\ttNum444 & \BottomEndFaceOutCChamfer{} 仕上げ後 一時停止 (0:{\ttfamily M00}, 1:non-stop, 2:扉前{\ttfamily M00}) & 1\\
\SetRow{unusingVariables}
\ttNum445 & (不使用) &\\
\ttNum446 & \BottomEndFaceInCChamfer$X+$補正 & 0\\
\ttNum447 & \BottomEndFaceInCChamferWidth(直径)$+$補正 & 0\\
\ttNum448 & \BottomEndFaceInCChamfer{} 仕上げ前 一時停止 (0:{\ttfamily M00}, 1:non-stop, 2:扉前{\ttfamily M00}) & 1\\
\ttNum449 & \BottomEndFaceInCChamfer{} 仕上げ後 一時停止 (0:{\ttfamily M00}, 1:non-stop, 2:扉前{\ttfamily M00}) & 1\\
\end{multicollongtblr}


\clearpage
%%%%%%%%%%%%%%%%%%%%%%%%%%%%%%%%%%%%%%%%%%%%%%%%%%%%%%%%%%
%% subsection 18.3.2 %%%%%%%%%%%%%%%%%%%%%%%%%%%%%%%%%%%%%
%%%%%%%%%%%%%%%%%%%%%%%%%%%%%%%%%%%%%%%%%%%%%%%%%%%%%%%%%%
\subsection{\ttNum450\,-\ttNum474:\DimpleDepth 調整}
%%%%%%%%%%%%%%%%%%%%%%%%%%%%%%%%%%%%%%%%%%%%%%%%%%%%%%%%%%
%% marker %%%%%%%%%%%%%%%%%%%%%%%%%%%%%%%%%%%%%%%%%%%%%%%%
%%%%%%%%%%%%%%%%%%%%%%%%%%%%%%%%%%%%%%%%%%%%%%%%%%%%%%%%%%
%\begin{marker}
%\ttNum461-\ttNum464の値は2024/02/29時点のもの
%\end{marker}
%%%%%%%%%%%%%%%%%%%%%%%%%%%%%%%%%%%%%%%%%%%%%%%%%%%%%%%%%%
%%%%%%%%%%%%%%%%%%%%%%%%%%%%%%%%%%%%%%%%%%%%%%%%%%%%%%%%%%
%%%%%%%%%%%%%%%%%%%%%%%%%%%%%%%%%%%%%%%%%%%%%%%%%%%%%%%%%%

\begin{multicollongtblr}[white]{\ttNum450\,-\ttNum474:\DimpleDepth 調整}{cX[l]c}
変数 & 内容 & 設定例\\
\SetRow{unusingVariables}
\ttNum450 & (予備) &\\
\SetRow{unusingVariables}
\ttNum451 & (予備) &\\
\SetRow{unusingVariables}
\ttNum452 & (予備) &\\
\SetRow{unusingVariables}
\ttNum453 & (予備) &\\
\SetRow{unusingVariables}
\ttNum454 & (予備) &\\
\SetRow{unusingVariables}
\ttNum455 & (予備) &\\
\SetRow{unusingVariables}
\ttNum456 & (予備) &\\
\SetRow{unusingVariables}
\ttNum457 & (予備) &\\
\SetRow{unusingVariables}
\ttNum458 & (予備) &\\
\ttNum459 & \DimpleMilling 後 一時停止 (0:{\ttfamily M00}, 1:non-stop, 2:扉前{\ttfamily M00}) & 1\\
\SetRow{unusingVariables}
\ttNum460 & (不使用) &\\
\ttNum461 & 工具{\ttfamily T31}(Tスロット)\AfaceDimple~深さ補正値(深さに$+$補正) & -0.055\\
\ttNum462 & 工具{\ttfamily T31}(Tスロット)\CfaceDimple~深さ補正値(深さに$+$補正) & -0.055\\
\ttNum463 & 工具{\ttfamily T31}(Tスロット)\BfaceDimple~深さ補正値(深さに$+$補正) & -0.025\\
\ttNum464 & 工具{\ttfamily T31}(Tスロット)\DfaceDimple~深さ補正値(深さに$+$補正) & -0.085\\
\SetRow{unusingVariables}
\ttNum465 & (不使用) &\\
\ttNum466 & 工具{\ttfamily T32}(Tスロット)\AfaceDimple~深さ補正値(深さに$+$補正) & -0.110\\
\ttNum467 & 工具{\ttfamily T32}(Tスロット)\CfaceDimple~深さ補正値(深さに$+$補正) & -0.080\\
\ttNum468 & 工具{\ttfamily T32}(Tスロット)\BfaceDimple~深さ補正値(深さに$+$補正) & -0.060\\
\ttNum469 & 工具{\ttfamily T32}(Tスロット)\DfaceDimple~深さ補正値(深さに$+$補正) & -0.120\\
\SetRow{unusingVariables}
\ttNum470 & (不使用) &\\
\ttNum471 & 工具{\ttfamily T33}(Tスロット)\AfaceDimple~深さ補正値(深さに$+$補正) & 0.00\\
\ttNum472 & 工具{\ttfamily T33}(Tスロット)\CfaceDimple~深さ補正値(深さに$+$補正) & 0.00\\
\ttNum473 & 工具{\ttfamily T33}(Tスロット)\BfaceDimple~深さ補正値(深さに$+$補正) & 0.00\\
\ttNum474 & 工具{\ttfamily T33}(Tスロット)\DfaceDimple~深さ補正値(深さに$+$補正) & 0.00\\
\end{multicollongtblr}


\clearpage
%%%%%%%%%%%%%%%%%%%%%%%%%%%%%%%%%%%%%%%%%%%%%%%%%%%%%%%%%%
%% subsection 17.3.4 %%%%%%%%%%%%%%%%%%%%%%%%%%%%%%%%%%%%%
%%%%%%%%%%%%%%%%%%%%%%%%%%%%%%%%%%%%%%%%%%%%%%%%%%%%%%%%%%
\subsection{\ttNum475\,-\ttNum499:各工程用 補助機能(該当明細 少)}

\begin{multicollongtblr}[white]{\ttNum475\,-\ttNum499:補助機能(該当明細 少)}{cX[l]c}
変数 & 内容 & 設定例\\
\ttNum475 & \KeywayWidth$Z$方向中央切削(3回加工)(0:off, 3:on) & 0\\
\SetRow{unusingVariables}
\ttNum476 & (不使用) &\\
\ttNum477 & \EndFaceBoring$X+$補正 & 0\\
\ttNum478 & \EndFaceBoring$Y+$補正 & 0\\
\ttNum479 & \EndFaceBoringWidth $+$補正 & 0\\
\ttNum480 & \EndFaceBoring{} 仕上げ前 一時停止 (0:{\ttfamily M00}, 1:non-stop, 2:扉前{\ttfamily M00}) & 1\\
\ttNum481 & \EndFaceBoring{} 仕上げ後 一時停止 (0:{\ttfamily M00}, 1:non-stop, 2:扉前{\ttfamily M00}) & 1\\
\SetRow{unusingVariables}
\ttNum482 & (予備) &\\
\SetRow{unusingVariables}
\ttNum483 & (予備) &\\
\SetRow{unusingVariables}
\ttNum484 & (予備) &\\
\SetRow{unusingVariables}
\ttNum485 & (予備) &\\
\SetRow{unusingVariables}
\ttNum486 & (予備) &\\
\SetRow{unusingVariables}
\ttNum487 & (予備) &\\
\SetRow{unusingVariables}
\ttNum488 & (予備) &\\
\SetRow{unusingVariables}
\ttNum489 & (予備) &\\
\SetRow{unusingVariables}
\ttNum490 & (予備) &\\
\SetRow{unusingVariables}
\ttNum491 & (予備) &\\
\SetRow{unusingVariables}
\ttNum492 & (予備) &\\
\SetRow{unusingVariables}
\ttNum493 & (予備) &\\
\SetRow{unusingVariables}
\ttNum494 & (予備) &\\
\SetRow{unusingVariables}
\ttNum495 & (予備) &\\
\SetRow{unusingVariables}
\ttNum496 & (予備) &\\
\SetRow{unusingVariables}
\ttNum497 & (予備) &\\
\SetRow{unusingVariables}
\ttNum498 & (予備) &\\
\SetRow{unusingVariables}
\ttNum499 & (予備) &\\
\end{multicollongtblr}



\clearpage
%%%%%%%%%%%%%%%%%%%%%%%%%%%%%%%%%%%%%%%%%%%%%%%%%%%%%%%%%%
%% section 18.4 %%%%%%%%%%%%%%%%%%%%%%%%%%%%%%%%%%%%%%%%%%
%%%%%%%%%%%%%%%%%%%%%%%%%%%%%%%%%%%%%%%%%%%%%%%%%%%%%%%%%%
\modHeadsection{\ttNum500\,-\ttNum574:バンドルのNCプログラムの使用コモン変数}
\ttNum500\,-\ttNum574については、主に\index{バンドルのNCプログラム}バンドルのNCプログラム\Gprgbox{O910x}\Gprgbox{O93xx}で使用されているものである。
%%%%%%%%%%%%%%%%%%%%%%%%%%%%%%%%%%%%%%%%%%%%%%%%%%%%%%%%%%
%% marker %%%%%%%%%%%%%%%%%%%%%%%%%%%%%%%%%%%%%%%%%%%%%%%%
%%%%%%%%%%%%%%%%%%%%%%%%%%%%%%%%%%%%%%%%%%%%%%%%%%%%%%%%%%
\begin{marker}
これらの値は2023/09/26機械設置時に入力されていたもの
\end{marker}
%%%%%%%%%%%%%%%%%%%%%%%%%%%%%%%%%%%%%%%%%%%%%%%%%%%%%%%%%%
%%%%%%%%%%%%%%%%%%%%%%%%%%%%%%%%%%%%%%%%%%%%%%%%%%%%%%%%%%
%%%%%%%%%%%%%%%%%%%%%%%%%%%%%%%%%%%%%%%%%%%%%%%%%%%%%%%%%%
%%%%%%%%%%%%%%%%%%%%%%%%%%%%%%%%%%%%%%%%%%%%%%%%%%%%%%%%%%
%% marker %%%%%%%%%%%%%%%%%%%%%%%%%%%%%%%%%%%%%%%%%%%%%%%%
%%%%%%%%%%%%%%%%%%%%%%%%%%%%%%%%%%%%%%%%%%%%%%%%%%%%%%%%%%
\begin{marker}
\ttNum500-\ttNum574は作成したNCプログラムでは使用していない
\end{marker}
%%%%%%%%%%%%%%%%%%%%%%%%%%%%%%%%%%%%%%%%%%%%%%%%%%%%%%%%%%
%%%%%%%%%%%%%%%%%%%%%%%%%%%%%%%%%%%%%%%%%%%%%%%%%%%%%%%%%%
%%%%%%%%%%%%%%%%%%%%%%%%%%%%%%%%%%%%%%%%%%%%%%%%%%%%%%%%%%

\begin{multicollongtblr}[white]{\ttNum500\,-\ttNum574:\Gprgbox{O910x}\Gprgbox{O93xx}用}{cX[l]c}
変数 & 内容 & 設定値\\
\ttNum500 & 芯ずれ許容差 \Gprgbox{O93xx} & 5\\
\ttNum501 & タッチセンサー信号遅れ補正 \Gprgbox{O93xx} & 0.040\\
\ttNum502 & タッチセンサープローブ中心$X$補正 \Gprgbox{O93xx} & -0.016507\\
\ttNum503 & タッチセンサープローブ中心$Y$補正 \Gprgbox{O93xx} & -0.068371\\
\ttNum504 & 測定距離 \Gprgbox{O910x} & 5\\
\ttNum505 & プローブ表面からプログラムの加工原点($Z$0)までの距離 \Gprgbox{O910x} & 785.529\\
\ttNum506 & 工具長の変化の許容差 \Gprgbox{O910x} & 1.0\\
\ttNum507 & 工具破損検出の許容差 \Gprgbox{O910x} & 1.0\\
\ttNum508 & (不明) & \ttNum0\\
\ttNum509 & $Z$座標系設定 \Gprgbox{O93xx} & 441.432\\
\ttNum510 & (不明) & 270\\
\ttNum511 & インチ/ミリ切替 \Gprgbox{O910x} & \ttNum0\\
\ttNum512 & タッチセンサープローブ半径$\mathrm{mm}$値 \Gprgbox{O93xx} & 5.0\\
\ttNum513 & 移動時用の送り速さ値 \Gprgbox{O910x} & 1000\\
\ttNum514 & スキップ({\ttfamily G31})測定時用 送り速さ値 \Gprgbox{O910x}\Gprgbox{O93xx} & 50\\
\ttNum515 & (不明) & \ttNum0\\
\ttNum516 & センサーの位置$X$座標 \Gprgbox{O910x} & -30.374\\
\ttNum517 & センサーの位置$Y$座標 \Gprgbox{O910x} & -913.761\\
\ttNum518 & センサーの位置$Z$座標 & -785.529\\
\ttNum519 & (不明) & 6\\
\ttNum520 & 拡張ワーク座標系 \Gprgbox{O910x} & 1861\\
\ttNum521 & (不明) & 0\\
\ttNum522 & (不明) & 0\\
\ttNum523 & アプローチ時用の送り速さ値 \Gprgbox{O910x} & 30\\
\ttNum524 & 測定時用の送り速さ値 \Gprgbox{O910x} & 3\\
$\cdots$ & (以下不明) & \ttNum0\\
\ttNum533 & (不明) & 0\\
\ttNum534 & (不明) & 0\\
$\cdots$ & (以下不明) & \ttNum0
\end{multicollongtblr}



\clearpage
%%%%%%%%%%%%%%%%%%%%%%%%%%%%%%%%%%%%%%%%%%%%%%%%%%%%%%%%%%
%% section 11.5 %%%%%%%%%%%%%%%%%%%%%%%%%%%%%%%%%%%%%%%%%%
%%%%%%%%%%%%%%%%%%%%%%%%%%%%%%%%%%%%%%%%%%%%%%%%%%%%%%%%%%
\modHeadsection{\ttNum600\,-\ttNum699}


%%%%%%%%%%%%%%%%%%%%%%%%%%%%%%%%%%%%%%%%%%%%%%%%%%%%%%%%%%
%% subsection 11.5.1 %%%%%%%%%%%%%%%%%%%%%%%%%%%%%%%%%%%%%
%%%%%%%%%%%%%%%%%%%%%%%%%%%%%%%%%%%%%%%%%%%%%%%%%%%%%%%%%%
\subsection{\ttNum600\,-\ttNum624:ワークと工具間の距離の調整}
\ttNum600\,-\ttNum624については、主に\index{ワーク}ワークと工具間の距離に関するものとする。\\

\begin{multicollongtblr}[white]{\ttNum600\,-\ttNum624:ワークと工具間の距離の調整}{cX[l]c}
変数 & 内容 & 設定例\\
\ttNum600 & 工具 - \EndFace 間 $Z$方向クリアランス平面間距離 & 100.0\\
\SetRow{unusingVariables}
\ttNum601 & (予備) &\\
\ttNum602 & タッチセンサープローブ測定時(\Dimple 除く)の近付き量 & 8.0\\
\ttNum603 & タッチセンサープローブ測定時の行過ぎ量 & 3.0\\
\SetRow{unusingVariables}
\ttNum604 & (予備) &\\
\ttNum605 & 外側加工面 法線方向クリアランス平面間距離 最小値 & 30.0\\
\ttNum606 & 内側加工面 法線方向クリアランス平面間距離 最小値 & 15.0\\
\SetRow{unusingVariables}
\ttNum607 & (予備) &\\
\ttNum608 & \EndFacecutMilling 用 内径輪郭径$-$補正量 & 5.0\\
\SetRow{unusingVariables}
\ttNum609 & (予備) &\\
\SetRow{unusingVariables}
\ttNum610 & (\OutcutMilling 用予備) & \\
\SetRow{unusingVariables}
\ttNum611 & (予備) &\\
\SetRow{unusingVariables}
\ttNum612 & (\KeywayMilling 用予備) & \\
\SetRow{unusingVariables}
\ttNum613 & (予備) &\\
\SetRow{unusingVariables}
\ttNum614 & (\EndFaceOutCChamferMilling 用予備) & \\
\SetRow{unusingVariables}
\ttNum615 & (予備) &\\
\SetRow{unusingVariables}
\ttNum616 & (\EndFaceInCChamferMilling 用予備) &\\
\SetRow{unusingVariables}
\ttNum617 & (予備) &\\
\SetRow{unusingVariables}
\ttNum618 & (\EndFaceBoringMilling 用予備) &\\
\SetRow{unusingVariables}
\ttNum619 & (予備) &\\
\ttNum620 & \Dimple 測定時 近付き量 & 4.0 \\
\ttNum621 & \DimpleMilling 時 近付き量 & 2.0 \\
\SetRow{unusingVariables}
\ttNum622 & (予備) &\\
\SetRow{unusingVariables}
\ttNum623 & (予備) &\\
\SetRow{unusingVariables}
\ttNum624 & (予備) &\\
\end{multicollongtblr}


\clearpage
%%%%%%%%%%%%%%%%%%%%%%%%%%%%%%%%%%%%%%%%%%%%%%%%%%%%%%%%%%
%% subsection 11.5.2 %%%%%%%%%%%%%%%%%%%%%%%%%%%%%%%%%%%%%
%%%%%%%%%%%%%%%%%%%%%%%%%%%%%%%%%%%%%%%%%%%%%%%%%%%%%%%%%%
\subsection{\ttNum625\,-\ttNum649:残り代および1回あたりの削り代}
\ttNum625\,-\ttNum649については、各加工の\index{のこりけずりしろ@残り削り代}残り削り代や\index{1かいあたりのけずりしろ@1回あたりの削り代}1回あたりの削り代に関するものとする。\\

\begin{multicollongtblr}[white]{\ttNum625\,-\ttNum649:残り代および1回あたりの削り代}{cX[l]c}
変数 & 内容 & 設定例\\
\ttNum625 & \EndFacecutMilling1回あたりの$Z$方向削り代 & 4.0\\
\SetRow{unusingVariables}
\ttNum626 & (予備) & \\
\ttNum627 & \OutcutMilling1回あたりの削り代(直径) & 2.0\\
\ttNum628 & \OutcutMilling{} 仕上げ前 残り削り代(直径) & 1.0\\
\ttNum629 & \KeywayMilling1回あたりの削り代(\KeywayDepth) & 5.0\\
\ttNum630 & \KeywayMilling{} 仕上げ前 残り削り代(直径) & 1.0\\
\ttNum631 & \EndFaceOutCChamferMilling1回あたりの削り代(直径) & 2.000\\
\ttNum632 & \EndFaceOutCChamferMilling{} 仕上げ前 残り削り代(直径) & 1.500\\
\ttNum633 & \EndFaceInCChamferMilling1回あたりの削り代(直径) & 2.060\\
\ttNum634 & \EndFaceInCChamferMilling{} 仕上げ前 残り削り代(直径) & 1.000\\
\ttNum635 & \EndFaceBoringMilling1回あたりの削り代 & 1.500\\
\ttNum636 & \EndFaceBoringMilling{} 仕上げ前 残り削り代 & 0.500\\
\SetRow{unusingVariables}
\ttNum637 & (予備) & \\
\SetRow{unusingVariables}
\ttNum638 & (予備) & \\
\SetRow{unusingVariables}
\ttNum639 & (予備) & \\
\SetRow{unusingVariables}
\ttNum640 & (予備) & \\
\SetRow{unusingVariables}
\ttNum641 & (予備) & \\
\SetRow{unusingVariables}
\ttNum642 & (予備) & \\
\SetRow{unusingVariables}
\ttNum643 & (予備) & \\
\SetRow{unusingVariables}
\ttNum644 & (予備) & \\
\SetRow{unusingVariables}
\ttNum645 & (予備) & \\
\SetRow{unusingVariables}
\ttNum646 & (予備) & \\
\SetRow{unusingVariables}
\ttNum647 & (予備) & \\
\SetRow{unusingVariables}
\ttNum648 & (予備) & \\
\SetRow{unusingVariables}
\ttNum649 & (予備) & \\
\end{multicollongtblr}


\clearpage
%%%%%%%%%%%%%%%%%%%%%%%%%%%%%%%%%%%%%%%%%%%%%%%%%%%%%%%%%%
%% subsection 17.5.3 %%%%%%%%%%%%%%%%%%%%%%%%%%%%%%%%%%%%%
%%%%%%%%%%%%%%%%%%%%%%%%%%%%%%%%%%%%%%%%%%%%%%%%%%%%%%%%%%
\subsection{\ttNum650\,-\ttNum674:工具の送り速さ}
\ttNum650\,-\ttNum674については、工具の\index{おくりはやさ@送り速さ}送り速さに関するものとする。\\

\begin{multicollongtblr}[white]{\ttNum650\,-\ttNum674:工具の送り速さ}{cX[l]c}
変数 & 内容 & 設定例\\
\ttNum650 & 早送り$Z$:{\ttfamily T50}アプローチ以外 & \SpindleRapidTraverseZ\\
\ttNum651 & アプローチ・$XY$リトラクト:{\ttfamily T50}以外 & \SpindleRapidAproachFeedRateZ\\
\ttNum652 & 早送り$XY$:{\ttfamily T50}, 早送り$Z$:{\ttfamily T50}アプローチ & \SensorRapidTraverseXY\\
\ttNum653 & アプローチ:{\ttfamily T50} & \SensorRapidAproachFeedRateZ\\
\ttNum654 & アプローチ:工具長測定 & \ToolLengthMeasurementFeedRateZ\\
\ttNum655 & 測定時アプローチ:{\ttfamily T50}(両側芯出し測定) & \CenterMeasurementFeedRate\\
\ttNum656 & 測定時アプローチ:{\ttfamily T50}(片側測定) & \PosMeasurementFeedRate\\
\SetRow{unusingVariables}
\ttNum657 & (不使用) &\\
\ttNum658 & \EndFacecutMilling:直線 & \EndFaceLinearFeedRate\\
\ttNum659 & \EndFacecutMilling:コーナー & \EndFaceCornerFeedRate\\
\ttNum660 & \OutcutMilling:直線 & \OutcutLinearFeedRate\\
\ttNum661 & \OutcutMilling:コーナー & \OutcutCornerFeedRate\\
\ttNum662 & \KeywayMilling:直線 & \KeywayLinearFeedRate\\
\ttNum663 & \KeywayMilling:コーナー & \KeywayCornerFeedRate\\
\ttNum664 & \EndFaceOutCChamferMilling:直線 & \OutCChamferLinearFeedRate\\
\ttNum665 & \EndFaceOutCChamferMilling:コーナー & \OutCChamferCornerFeedRate\\
\ttNum666 & \EndFaceInCChamferMilling:直線 & \InCChamferLinearFeedRate\\
\ttNum667 & \EndFaceInCChamferMilling:コーナー & \InCChamferCornerFeedRate\\
\ttNum668 & \EndFaceBoringMilling:直線 & \EndFaceBoringLinearFeedRate\\
\ttNum669 & \EndFaceBoringMilling:コーナー & \EndFaceBoringCornerFeedRate\\
\ttNum670 & \Dimple 測定:表面アプローチ & \DimpleApproachFeedRate\\
\ttNum671 & \DimpleMilling:加工 & \DimpleProcessFeedRate\\
\SetRow{unusingVariables}
\ttNum672 & (予備) & \\
\SetRow{unusingVariables}
\ttNum673 & (予備) & \\
\SetRow{unusingVariables}
\ttNum674 & (予備) & \\
\end{multicollongtblr}


\clearpage
%%%%%%%%%%%%%%%%%%%%%%%%%%%%%%%%%%%%%%%%%%%%%%%%%%%%%%%%%%
%% subsection 17.5.4 %%%%%%%%%%%%%%%%%%%%%%%%%%%%%%%%%%%%%
%%%%%%%%%%%%%%%%%%%%%%%%%%%%%%%%%%%%%%%%%%%%%%%%%%%%%%%%%%
\subsection{\ttNum675\,-\ttNum699:スピンドルの回転数(切削時)}
\ttNum675\,-\ttNum699については、切削時の工具の\index{しゅじくかいてんすう@主軸回転数}主軸回転数に関するものとする。\\

\begin{multicollongtblr}[white]{\ttNum675\,-\ttNum699:主軸回転数(切削時)}{cX[l]c}
変数 & 内容 & 設定例\\
\ttNum675 & \EndFacecutMilling & \EndFaceSpindleSpeed\\
\ttNum676 & \OutcutMilling & \OutcutSpindleSpeed\\
\ttNum677 & \KeywayMilling & \KeywaySpindleSpeed\\
\ttNum678 & \EndFaceOutCChamferMilling & \OutCChamferSpindleSpeed\\
\ttNum679 & \EndFaceInCChamferMilling & \InCChamferSpindleSpeed\\
\ttNum680 & \EndFaceBoringMilling & \EndFaceBoringSpindleSpeed\\
\ttNum681 & \DimpleMilling & \DimpleSpindleSpeed\\
\SetRow{unusingVariables}
\ttNum682 & (予備) & \\
\SetRow{unusingVariables}
\ttNum683 & (予備) & \\
\SetRow{unusingVariables}
\ttNum684 & (予備) & \\
\SetRow{unusingVariables}
\ttNum685 & (予備) & \\
\SetRow{unusingVariables}
\ttNum686 & (予備) & \\
\SetRow{unusingVariables}
\ttNum687 & (予備) & \\
\SetRow{unusingVariables}
\ttNum688 & (予備) & \\
\SetRow{unusingVariables}
\ttNum689 & (予備) & \\
\SetRow{unusingVariables}
\ttNum690 & (予備) & \\
\SetRow{unusingVariables}
\ttNum691 & (予備) & \\
\SetRow{unusingVariables}
\ttNum692 & (予備) & \\
\SetRow{unusingVariables}
\ttNum693 & (予備) & \\
\SetRow{unusingVariables}
\ttNum694 & (予備) & \\
\SetRow{unusingVariables}
\ttNum695 & (予備) & \\
\SetRow{unusingVariables}
\ttNum696 & (予備) & \\
\SetRow{unusingVariables}
\ttNum697 & (予備) & \\
\SetRow{unusingVariables}
\ttNum698 & (予備) & \\
\SetRow{unusingVariables}
\ttNum699 & (予備) & \\
\end{multicollongtblr}



\clearpage
%%%%%%%%%%%%%%%%%%%%%%%%%%%%%%%%%%%%%%%%%%%%%%%%%%%%%%%%%%
%% section 11.6 %%%%%%%%%%%%%%%%%%%%%%%%%%%%%%%%%%%%%%%%%%
%%%%%%%%%%%%%%%%%%%%%%%%%%%%%%%%%%%%%%%%%%%%%%%%%%%%%%%%%%
\modHeadsection{\ttNum700\,-\ttNum750:\Dimple 用}
\ttNum700\,-\ttNum750については、主に\expandafterindex{\yomiDimple そくていようサブプログラム@\nameDimple 測定用サブプログラム}\Dimple 測定用サブプログラム\Gprgbox{O2x000x}で使用されるものとする。


%%%%%%%%%%%%%%%%%%%%%%%%%%%%%%%%%%%%%%%%%%%%%%%%%%%%%%%%%%
%% subsection 18.6.1 %%%%%%%%%%%%%%%%%%%%%%%%%%%%%%%%%%%%%
%%%%%%%%%%%%%%%%%%%%%%%%%%%%%%%%%%%%%%%%%%%%%%%%%%%%%%%%%%
\subsection{\ttNum700\,-\ttNum724:\Dimple~\DLone 用}

\begin{multicollongtblr}[white]{\ttNum700\,-\ttNum724:\Dimple~移動 \DLone 用}{cX[l]}
変数 & 内容\\
\ttNum700 & NCプログラム読込時のワーク座標系(\ttNum4012)\\
\ttNum701 & 工具別$Z$補正({\ttfamily T50}:\ttNum901050, {\ttfamily T3x}:0)\\
\ttNum702 & 工具別$XY$補正({\ttfamily T50}:\ttNum901050, {\ttfamily T3x}:\ttNum[2400+\ttNum4111]+\ttNum[2600+\ttNum4111])\\
\ttNum703 & 工具別{\ttfamily G\ttNum} ({\ttfamily T50}:31, {\ttfamily T3x}:01)\\
\ttNum704 & \TableCenter からワーク座標(\ttNum700)原点までの$X$距離\\
\ttNum705 & 傾き後の\TopEndFace 中心(機械座標)$X$ (\cf\pageeqref{eq:afterPhiTCenterFromO})\\
\ttNum706 & \TableCenter から傾き後の\TopEndFace 中心までの$Z$距離 (\cf\pageeqref{eq:afterPhiTCenterFromO})\\
\ttNum707 & \TableCenter から\DimpleFirstRow までの$Z$距離$Z-q$\\
\ttNum708 & トップ端中心から\DimpleFirstRow 中心までの$X$距離(\cf\pageeqref{eq:dimpleCenterDistance})\\
\ttNum709 & 傾き後\DimpleFirstRow 中心$X$移動距離(\cf\pageeqref{eq:afterPhidimpleCenterDistance})\\
\ttNum710 & 傾き後\DimpleFirstRow 中心$Z$移動距離(\cf\pageeqref{eq:afterPhidimpleCenterDistance})\\
\ttNum711 & 傾き後トップ端中心(ブロックエンド)$X$座標(\ttNum5001)\\
\ttNum712 & 傾き後トップ端中心(ブロックエンド)$Z$座標(\ttNum5003)\\
\ttNum713 & 傾き後\DimpleFirstRow 中心(ブロックエンド)$X$座標 (\ttNum5001)\\
\ttNum714 & 傾き後\DimpleFirstRow 中心(ブロックエンド)$Y$座標 (\ttNum5002)\\
\ttNum715 & 傾き後\DimpleFirstRow 中心(ブロックエンド)$Z$座標 (\ttNum5003)\\
\ttNum716 & BD内半径$+$\PlatingThk$-\text{\ttNum702}-\text{\ttNum620}$\\
\ttNum717 & (AC内半径$+$\PlatingThk$-\text{\ttNum702}-\text{\ttNum620}\text)\cos\phi$\\
\ttNum718 & \expandafterindex{しかこうよう\yomiDimpleDepth@試加工用\nameDimpleDepth}試加工用\nameDimpleDepth(\DimpleDepth $-0.1$)\\
\SetRow{unusingVariables}
\ttNum719 & (予備)\\
\SetRow{unusingVariables}
\ttNum720 & (予備)\\
\SetRow{unusingVariables}
\ttNum721 & (予備)\\
\SetRow{unusingVariables}
\ttNum722 & (予備)\\
\SetRow{unusingVariables}
\ttNum723 & (予備)\\
\ttNum724\color{red}$^*$ & 各面 ループ用数値(1:A, 2:C, 3:B, 4:D)\\
\end{multicollongtblr}
%%%%%%%%%%%%%%%%%%%%%%%%%%%%%%%%%%%%%%%%%%%%%%%%%%%%%%%%%%
%% hosoku %%%%%%%%%%%%%%%%%%%%%%%%%%%%%%%%%%%%%%%%%%%%%%%%
%%%%%%%%%%%%%%%%%%%%%%%%%%%%%%%%%%%%%%%%%%%%%%%%%%%%%%%%%%
\begin{marker}
\ttNum724は\DLtwoAC および\DLtwoBD で\index{RHS(コモンへんすう)@RHS(コモン変数)}RHSとして使用していることに注意
\end{marker}
%%%%%%%%%%%%%%%%%%%%%%%%%%%%%%%%%%%%%%%%%%%%%%%%%%%%%%%%%%
%%%%%%%%%%%%%%%%%%%%%%%%%%%%%%%%%%%%%%%%%%%%%%%%%%%%%%%%%%
%%%%%%%%%%%%%%%%%%%%%%%%%%%%%%%%%%%%%%%%%%%%%%%%%%%%%%%%%%


\clearpage
%%%%%%%%%%%%%%%%%%%%%%%%%%%%%%%%%%%%%%%%%%%%%%%%%%%%%%%%%%
%% subsection 18.6.2 %%%%%%%%%%%%%%%%%%%%%%%%%%%%%%%%%%%%%
%%%%%%%%%%%%%%%%%%%%%%%%%%%%%%%%%%%%%%%%%%%%%%%%%%%%%%%%%%
\subsection{\ttNum725\,-\ttNum749:\Dimple~\DLtwoAC\DLtwoBD 用}

\begin{multicollongtblr}[white]{\ttNum725\,-\ttNum744:\Dimple~移動 \DLtwoAC\DLtwoBD 用}{cX[l]}
変数 & 内容\\
\ttNum725 & プログラム読込時ブロックエンド$Y$ or $X$ (\ttNum5002, \ttNum5001)\\
\ttNum726 & プログラム読込時ブロックエンド$Z$ (\ttNum5003)\\
\ttNum727 & \Dimple~偶数列の列数\\
\ttNum728 & \Dimple~偶数列(一列)の\DimpleNum\\
\ttNum729 & \Dimple~奇数列(一列)の\DimpleNum\\
\ttNum730 & \Dimple~現在の列の\DimpleNum\\
\SetRow{unusingVariables}
\ttNum731 & (予備)\\
\SetRow{unusingVariables}
\ttNum732 & (予備)\\
\SetRow{unusingVariables}
\ttNum733 & (予備)\\
\SetRow{unusingVariables}
\ttNum734 & (予備)\\
\SetRow{unusingVariables}
\ttNum735 & (予備)\\
\SetRow{unusingVariables}
\ttNum736 & (予備)\\
\SetRow{unusingVariables}
\ttNum737 & (予備)\\
\SetRow{unusingVariables}
\ttNum738 & (予備)\\
\SetRow{unusingVariables}
\ttNum739 & (予備)\\
\SetRow{unusingVariables}
\ttNum740 & (予備)\\
\SetRow{unusingVariables}
\ttNum741 & (予備)\\
\SetRow{unusingVariables}
\ttNum742 & (予備)\\
\SetRow{unusingVariables}
\ttNum743 & (予備)\\
\SetRow{unusingVariables}
\ttNum744 & (予備)\\
\end{multicollongtblr}


%\clearpage
\begin{multicollongtblr}[white]{\ttNum745\,-\ttNum749:\Dimple~測定~\DMLthreeAC\DMLthreeBD 用}{cX[l]}
変数 & 内容\\
\SetRow{unusingVariables}
\ttNum745 & (予備)\\
\SetRow{unusingVariables}
\ttNum746 & (予備)\\
\SetRow{unusingVariables}
\ttNum747 & (予備)\\
\ttNum748 & プログラム読込時ブロックエンド$X$ or $Y$ (\ttNum5001, \ttNum5002)\\
\ttNum749\color{red}$^*$ & \Dimple~表面位置$X$ or $Y$測定値
\end{multicollongtblr}
%%%%%%%%%%%%%%%%%%%%%%%%%%%%%%%%%%%%%%%%%%%%%%%%%%%%%%%%%%
%% hosoku %%%%%%%%%%%%%%%%%%%%%%%%%%%%%%%%%%%%%%%%%%%%%%%%
%%%%%%%%%%%%%%%%%%%%%%%%%%%%%%%%%%%%%%%%%%%%%%%%%%%%%%%%%%
\begin{marker}
\ttNum749は\DLtwoAC および\DLtwoBD で\index{RHS(コモンへんすう)@RHS(コモン変数)}RHSとして使用していることに注意
\end{marker}
%%%%%%%%%%%%%%%%%%%%%%%%%%%%%%%%%%%%%%%%%%%%%%%%%%%%%%%%%%
%%%%%%%%%%%%%%%%%%%%%%%%%%%%%%%%%%%%%%%%%%%%%%%%%%%%%%%%%%
%%%%%%%%%%%%%%%%%%%%%%%%%%%%%%%%%%%%%%%%%%%%%%%%%%%%%%%%%%



\clearpage
%%%%%%%%%%%%%%%%%%%%%%%%%%%%%%%%%%%%%%%%%%%%%%%%%%%%%%%%%%
%% section 19.7 %%%%%%%%%%%%%%%%%%%%%%%%%%%%%%%%%%%%%%%%%%
%%%%%%%%%%%%%%%%%%%%%%%%%%%%%%%%%%%%%%%%%%%%%%%%%%%%%%%%%%
\modHeadsection{\ttNum900000\,-\ttNum900031, \ttNum900101\,-\ttNum900500:実測値}
\ttNum900000\,-\ttNum900500については、\index{じっそくち@実測値}実測値または計算値を格納する。


%%%%%%%%%%%%%%%%%%%%%%%%%%%%%%%%%%%%%%%%%%%%%%%%%%%%%%%%%%
%% subsection 19.7.1 %%%%%%%%%%%%%%%%%%%%%%%%%%%%%%%%%%%%%
%%%%%%%%%%%%%%%%%%%%%%%%%%%%%%%%%%%%%%%%%%%%%%%%%%%%%%%%%%
\subsection{\ttNum900001\,-\ttNum900049:\Dimple 以外}

\begin{multicollongtblr}[white]{\ttNum900000\,-\ttNum900005:外中心$X$ 両側 測定用 \MXOThickness}{cX[l]}
変数 & 内容\\
\ttNum900000 & $X$外中心測定 $-X$側測定値\\
\ttNum900001 & $X$外中心測定 $+X$側測定値\\
\ttNum900002 & $X$外中心 測定値\\
\ttNum900003 & $X$外幅 トップ側測定値\\
\ttNum900004 & $X$外幅 ボトム側測定値\\
\SetRow{unusingVariables}
\ttNum900005 & (予備)\\
\end{multicollongtblr}


\begin{multicollongtblr}[white]{\ttNum900006\,-\ttNum900011:外中心$Y$ 両側 測定用 \MYOThickness}{cX[l]}
変数 & 内容\\
\ttNum900006 & $Y$外中心測定 $-Y$側測定値\\
\ttNum900007 & $Y$外中心測定 $+Y$側測定値\\
\ttNum900008 & $Y$外中心 測定値\\
\ttNum900009 & $Y$外幅 トップ側測定値\\
\ttNum900010 & $Y$外幅 ボトム側測定値\\
\SetRow{unusingVariables}
\ttNum900011 & (予備)\\
\end{multicollongtblr}


%\clearpage
\begin{multicollongtblr}[white]{\ttNum900012\,-\ttNum900014:\KeywayCenter$X$ 片側 測定用 \MXOface}{cX[l]}
変数 & 内容\\
\ttNum900012 & $X$\KeywayCenter 測定 A側外面測定値\\
\SetRow{unusingVariables}
\ttNum900013 & (予備)\\
\SetRow{unusingVariables}
\ttNum900014 & (予備)\\
\end{multicollongtblr}


\clearpage
\begin{multicollongtblr}[white]{\ttNum900015\,-\ttNum900020:内中心$X$ 両側 測定用 \MXIWidth}{cX[l]}
変数 & 内容\\
\ttNum900015 & $X$内中心測定 $-X$側測定値\\
\ttNum900016 & $X$内中心測定 $+X$側測定値\\
\ttNum900017 & $X$内中心 測定値\\
\ttNum900018 & $X$内幅 トップ側測定値\\
\ttNum900019 & $X$内幅 ボトム側測定値\\
\SetRow{unusingVariables}
\ttNum900020 & (予備)\\
\end{multicollongtblr}


%\clearpage
\begin{multicollongtblr}[white]{\ttNum900021\,-\ttNum900026:内中心$Y$ 両側 測定用 \MYIWidth}{cX[l]}
変数 & 内容\\
\ttNum900021 & $Y$内中心測定 $-Y$側測定値\\
\ttNum900022 & $Y$内中心測定 $+Y$側測定値\\
\ttNum900023 & $Y$内中心 測定値\\
\ttNum900024 & $Y$内幅 トップ側測定値\\
\ttNum900025 & $Y$内幅 ボトム側測定値\\
\SetRow{unusingVariables}
\ttNum900026 & (予備)\\
\end{multicollongtblr}


%\clearpage
\begin{multicollongtblr}[white]{\ttNum900027\,-\ttNum900029:外削中心$X$ 片側 測定用 \MXIface}{cX[l]}
変数 & 内容\\
\ttNum900027 & $X$外削中心測定 内面測定値\\
\SetRow{unusingVariables}
\ttNum900028 & (予備)\\
\SetRow{unusingVariables}
\ttNum900029 & (予備)\\
\end{multicollongtblr}


\clearpage
\begin{multicollongtblr}[white]{\ttNum900030\,-\ttNum900037:\CenterlineEndFaceDif{} 片側 測定用 \Mcenterline}{cX[l]}
変数 & 内容\\
\ttNum900030 & \CenterlineEndFaceDifBD{} ボトム側測定値\\
\ttNum900031 & \CenterlineEndFaceDifBD{} トップ側測定値\\
\ttNum900032 & \CenterlineEndFaceDifBD{} 測定値\\
\SetRow{unusingVariables}
\ttNum900033 & (予備)\\
\ttNum900034 & \CenterlineEndFaceDifAC{} トップ側測定値\\
\ttNum900035 & \CenterlineEndFaceDifAC{} ボトム側測定値\\
\ttNum900036 & \CenterlineEndFaceDifAC{} 測定値\\
\SetRow{unusingVariables}
\ttNum900037 & (予備)\\
\end{multicollongtblr}


%\clearpage
\begin{multicollongtblr}[white]{\ttNum900038\,-\ttNum900040:\CurvedOutcutMilling 用 \KCurvedGaisakuRLeft}{cX[l]}
変数 & 内容\\
\ttNum900038 & \TopCurvedOutcutAngle{} 計算値\\
\ttNum900039 & \BottomCurvedOutcutAngle{} 計算値\\
\SetRow{unusingVariables}
\ttNum900040 & (予備)\\
\end{multicollongtblr}


%\clearpage
%%%%%%%%%%%%%%%%%%%%%%%%%%%%%%%%%%%%%%%%%%%%%%%%%%%%%%%%%%
%% subsection 18.7.2 %%%%%%%%%%%%%%%%%%%%%%%%%%%%%%%%%%%%%
%%%%%%%%%%%%%%%%%%%%%%%%%%%%%%%%%%%%%%%%%%%%%%%%%%%%%%%%%%
\subsection{\ttNum900101\,-\ttNum900500:\Dimple}

%%%%%%%%%%%%%%%%%%%%%%%%%%%%%%%%%%%%%%%%%%%%%%%%%%%%%%%%%%
%% subsubsection 18.7.2.1 %%%%%%%%%%%%%%%%%%%%%%%%%%%%%%%%
%%%%%%%%%%%%%%%%%%%%%%%%%%%%%%%%%%%%%%%%%%%%%%%%%%%%%%%%%%
\subsubsection{\ttNum900101\,-\ttNum900300:\Dimple AC面}

\begin{multicollongtblr}[white]{\ttNum900101\,-\ttNum900300:\Dimple AC表面位置 測定値 \DMLthreeAC}{cX[l]}
変数 & 内容\\
\ttNum900101\,-\ttNum900200 & \AfaceDimple~表面位置$X$ 測定値\\
\ttNum900201\,-\ttNum900300 & \CfaceDimple~表面位置$X$ 測定値
\end{multicollongtblr}

%%%%%%%%%%%%%%%%%%%%%%%%%%%%%%%%%%%%%%%%%%%%%%%%%%%%%%%%%%
%% subsubsection 18.7.2.2 %%%%%%%%%%%%%%%%%%%%%%%%%%%%%%%%
%%%%%%%%%%%%%%%%%%%%%%%%%%%%%%%%%%%%%%%%%%%%%%%%%%%%%%%%%%
\subsubsection{\ttNum900301\,-\ttNum900500:\Dimple BD面}

\begin{multicollongtblr}[white]{\ttNum900301\,-\ttNum900500:\Dimple BD表面位置 測定値 \DMLthreeBD}{cX[l]}
変数 & 内容\\
\ttNum900301\,-\ttNum900400 & \BfaceDimple~表面位置$Y$ 測定値\\
\ttNum900401\,-\ttNum900500 & \DfaceDimple~表面位置$Y$ 測定値
\end{multicollongtblr}



\clearpage
%%%%%%%%%%%%%%%%%%%%%%%%%%%%%%%%%%%%%%%%%%%%%%%%%%%%%%%%%%
%% section 11.8 %%%%%%%%%%%%%%%%%%%%%%%%%%%%%%%%%%%%%%%%%%
%%%%%%%%%%%%%%%%%%%%%%%%%%%%%%%%%%%%%%%%%%%%%%%%%%%%%%%%%%
\modHeadsection{\ttNum901000\,-\ttNum901024:パレット・ジグ}
\ttNum901000\,-\ttNum901024については、主に\index{パレット}パレットや\index{ジグ}ジグに関するものとする。\\

\begin{multicollongtblr}[white]{\ttNum901000\,-\ttNum901024:主にパレット・ジグ}{cX[l]c}
変数 & 内容 & 設定例\\
\SetRow{unusingVariables}
\ttNum901000 & (予備) &\\
\ttNum901001 & パレット\ttNum1 \JigCenter 機械座標$X$ & -550.019\\
\ttNum901002 & パレット\ttNum1 \JigCenter 機械座標$Y$ & -739.006\\
\ttNum901003 & パレット\ttNum1 \JigCenter 機械座標$Z$ & -1149.974\\
\ttNum901004 & パレット\ttNum1 \JigCenter 機械座標$B$ & 0.073\\
\ttNum901005 & パレット\ttNum2 \JigCenter 機械座標$X$ & -550.019\\
\ttNum901006 & パレット\ttNum2 \JigCenter 機械座標$Y$ & -739.555\\
\ttNum901007 & パレット\ttNum2 \JigCenter 機械座標$Z$ & -1149.974\\
\ttNum901008 & パレット\ttNum2 \JigCenter 機械座標$B$ & 0.073\\
\ttNum901009 & 工具中心機械座標$C$ & 0\\
\SetRow{unusingVariables}
\ttNum901010 & (予備) &\\
\ttNum901011 & パレット\ttNum1 \JigLength$2l$(機械座標系$B$0 $Z$方向) & 660.0\\
\ttNum901012 & パレット\ttNum1 ジグ内側幅(機械座標系$B$0 $Z$方向) & 410.0\\
\ttNum901013 & パレット\ttNum1 ジグ幅(機械座標系$B$0 $X$方向) & 455.0\\
\SetRow{unusingVariables}
\ttNum901014 & パレット\ttNum1 ジグ底面-固定用ボルト取付具間距離(高さ, $Y$方向) &\\
\SetRow{unusingVariables}
\ttNum901015 & パレット\ttNum1 ジグ底面-固定用ボルト取付具間距離(幅, 機械座標系$B$0 $X$方向) &\\
\SetRow{unusingVariables}
\ttNum901016 & (予備) &\\
\ttNum901017 & パレット\ttNum2 \JigLength$2l$(機械座標系$B$0における$Z$方向) & 660.0\\
\ttNum901018 & パレット\ttNum2 ジグ内側幅(機械座標系$B$0における$Z$方向) & 410.0\\
\ttNum901019 & パレット\ttNum2 ジグ幅(機械座標系$B$0における$X$方向) & 455.0\\
\ttNum901020 & パレット\ttNum2 ジグ底面-固定用ボルト取付具間距離(高さ, $Y$方向) & 257.0\\
\ttNum901021 & パレット\ttNum2 ジグ底面-固定用ボルト取付具間距離(幅, 機械座標系$B$0 $X$方向) & 247.0\\
\SetRow{unusingVariables}
\ttNum901022 & (予備) &\\
\SetRow{unusingVariables}
\ttNum901023 & (予備) &\\
\SetRow{unusingVariables}
\ttNum901024 & (予備) &\\
\end{multicollongtblr}

\clearpage
%%%%%%%%%%%%%%%%%%%%%%%%%%%%%%%%%%%%%%%%%%%%%%%%%%%%%%%%%%
%% hosoku %%%%%%%%%%%%%%%%%%%%%%%%%%%%%%%%%%%%%%%%%%%%%%%%
%%%%%%%%%%%%%%%%%%%%%%%%%%%%%%%%%%%%%%%%%%%%%%%%%%%%%%%%%%
\begin{hosoku}
その他の(\index{ずめん(ジグ)@図面(ジグ)}図面上の)\index{すんぽう(ジグ)@寸法(ジグ)}寸法として、
\begin{enumerate}[label=\sarrow]
\item \TableCenter と C面側ジグ端 との水平距離:196.5
\item 受板の円の半径$\rho$:100
\item 受板の鉛直方向の幅$\sigma$:40
\item \TableCenter と 受板の円の中心 との水平距離$\Delta$:201.5
\item 受板の円の中心 と 受板の水平方向の底 との距離:70
\end{enumerate}
\end{hosoku}
%%%%%%%%%%%%%%%%%%%%%%%%%%%%%%%%%%%%%%%%%%%%%%%%%%%%%%%%%%
%%%%%%%%%%%%%%%%%%%%%%%%%%%%%%%%%%%%%%%%%%%%%%%%%%%%%%%%%%
%%%%%%%%%%%%%%%%%%%%%%%%%%%%%%%%%%%%%%%%%%%%%%%%%%%%%%%%%%



\clearpage
%%%%%%%%%%%%%%%%%%%%%%%%%%%%%%%%%%%%%%%%%%%%%%%%%%%%%%%%%%
%% section 17.9 %%%%%%%%%%%%%%%%%%%%%%%%%%%%%%%%%%%%%%%%%%
%%%%%%%%%%%%%%%%%%%%%%%%%%%%%%%%%%%%%%%%%%%%%%%%%%%%%%%%%%
\modHeadsection{\ttNum901050\,-\ttNum901074:タッチセンサーブローブ}
\ttNum901050\,-\ttNum901074については、主に工具{\ttfamily T50}\index{タッチセンサープローブ}タッチセンサープローブおよび\index{こうぐそくてい@工具測定}工具測定用\index{タッチセンサーブローブ(こうぐそくてい)@タッチセンサーブローブ(工具測定)}タッチセンサーブローブに関するものとする。

%%%%%%%%%%%%%%%%%%%%%%%%%%%%%%%%%%%%%%%%%%%%%%%%%%%%%%%%%%
%% marker %%%%%%%%%%%%%%%%%%%%%%%%%%%%%%%%%%%%%%%%%%%%%%%%
%%%%%%%%%%%%%%%%%%%%%%%%%%%%%%%%%%%%%%%%%%%%%%%%%%%%%%%%%%
\begin{marker}
これらの値は2023/09/26機械設置時のもの
\end{marker}
%%%%%%%%%%%%%%%%%%%%%%%%%%%%%%%%%%%%%%%%%%%%%%%%%%%%%%%%%%
%%%%%%%%%%%%%%%%%%%%%%%%%%%%%%%%%%%%%%%%%%%%%%%%%%%%%%%%%%
%%%%%%%%%%%%%%%%%%%%%%%%%%%%%%%%%%%%%%%%%%%%%%%%%%%%%%%%%%
\begin{multicollongtblr}[white]{\ttNum901050\,-\ttNum901074:工具{\ttfamily T50}および工具測定用タッチセンサーブローブ}{cX[l]c}
変数 & 内容 & 設定例\\
\ttNum901050 & {\ttfamily T50}先端球 半径 & 5.0\\
\ttNum901051 & {\ttfamily T50}軸部分 半径 & 3.75\\
\SetRow{unusingVariables}
\ttNum901052 & (予備) &\\
\ttNum901053 & 信号遅れ補正(送り速さ50mm/min時) & 0.040\\
\ttNum901054 & {\ttfamily T50}中心$X+$補正 & -0.016507\\
\ttNum901055 & {\ttfamily T50}中心$Y+$補正 & -0.12\\
\SetRow{unusingVariables}
\ttNum901056 & (予備) &\\
\ttNum901057 & 測定距離 & 5\\
\ttNum901058 & 工具測定用タッチセンサープローブ表面とプログラム原点($Z$0)との距離 & 785.529\\
\SetRow{unusingVariables}
\ttNum901059 & (予備) &\\
\ttNum901060 & 工具破損検出の許容差 & 1.0\\
\SetRow{unusingVariables}
\ttNum901061 & (予備) &\\
\ttNum901062 & 工具測定用タッチセンサープローブの位置$X$ & -30.374\\
\ttNum901063 & 工具測定用タッチセンサープローブの位置$Y$ & -913.761\\
\ttNum901064 & 工具測定用タッチセンサープローブの位置$Z$ & -820.000\\
\SetRow{unusingVariables}
\ttNum901065 & (予備) &\\
\SetRow{unusingVariables}
\ttNum901066 & (予備) &\\
\SetRow{unusingVariables}
\ttNum901067 & (予備) &\\
\SetRow{unusingVariables}
\ttNum901068 & (予備) &\\
\SetRow{unusingVariables}
\ttNum901069 & (予備) &\\
\SetRow{unusingVariables}
\ttNum901070 & (予備) &\\
\SetRow{unusingVariables}
\ttNum901071 & (予備) &\\
\SetRow{unusingVariables}
\ttNum901072 & (予備) &\\
\SetRow{unusingVariables}
\ttNum901073 & (予備) &\\
\SetRow{unusingVariables}
\ttNum901074 & (予備) &\\
\end{multicollongtblr}



\clearpage
%%%%%%%%%%%%%%%%%%%%%%%%%%%%%%%%%%%%%%%%%%%%%%%%%%%%%%%%%%
%% section 11.9 %%%%%%%%%%%%%%%%%%%%%%%%%%%%%%%%%%%%%%%%%%
%%%%%%%%%%%%%%%%%%%%%%%%%%%%%%%%%%%%%%%%%%%%%%%%%%%%%%%%%%
\modHeadsection{\ttNum901100\,-\ttNum901149:工具}
\ttNum901100\,-\ttNum901149については、\index{こうぐ@工具}工具に関するもの(工具長や工具径およびその摩耗量を除く)とする。
%%%%%%%%%%%%%%%%%%%%%%%%%%%%%%%%%%%%%%%%%%%%%%%%%%%%%%%%%%
%% hosoku %%%%%%%%%%%%%%%%%%%%%%%%%%%%%%%%%%%%%%%%%%%%%%%%
%%%%%%%%%%%%%%%%%%%%%%%%%%%%%%%%%%%%%%%%%%%%%%%%%%%%%%%%%%
\begin{hosoku}
工具長・工具径・工具の摩耗量といったオフセットの値は、\index{システムへんすう@システム変数}システム変数を参照
\end{hosoku}
%%%%%%%%%%%%%%%%%%%%%%%%%%%%%%%%%%%%%%%%%%%%%%%%%%%%%%%%%%
%%%%%%%%%%%%%%%%%%%%%%%%%%%%%%%%%%%%%%%%%%%%%%%%%%%%%%%%%%
%%%%%%%%%%%%%%%%%%%%%%%%%%%%%%%%%%%%%%%%%%%%%%%%%%%%%%%%%%

%%%%%%%%%%%%%%%%%%%%%%%%%%%%%%%%%%%%%%%%%%%%%%%%%%%%%%%%%%
%% subsection 17.9.1 %%%%%%%%%%%%%%%%%%%%%%%%%%%%%%%%%%%%%
%%%%%%%%%%%%%%%%%%%%%%%%%%%%%%%%%%%%%%%%%%%%%%%%%%%%%%%%%%
\subsection{\ttNum901100\,-\ttNum901124}

\begin{multicollongtblr}[white]{\ttNum901100\,-\ttNum901124:工具({\ttfamily T50}除く)}{cX[l]c}
変数 & 内容 & 設定例\\
\SetRow{unusingVariables}
\ttNum901100 & (予備) &\\
\ttNum901101 & 工具{\ttfamily T02}(フェイスミル)最大刃径(直径)DCX公称値$\phi'_\mathrm D$ & 113.5\\
\SetRow{unusingVariables}
\ttNum901102 & (\EndFacecutMilling 工具{\ttfamily T02}-{\ttfamily T05}用 予備) &\\
\SetRow{unusingVariables}
\ttNum901103 & (\EndFacecutMilling 工具{\ttfamily T02}-{\ttfamily T05}用 予備) &\\
\SetRow{unusingVariables}
\ttNum901104 & (\EndFacecutMilling 工具{\ttfamily T02}-{\ttfamily T05}用 予備) &\\
\ttNum901105 & 工具{\ttfamily T06}(サイドカッター)厚さ$t$ & 6.89\\
\SetRow{unusingVariables}
\ttNum901106 & (\KeywayMilling 工具{\ttfamily T06}, {\ttfamily T07}用 予備) &\\
\SetRow{unusingVariables}
\ttNum901107 & (\KeywayMilling 工具{\ttfamily T06}, {\ttfamily T07}用 予備) &\\
\SetRow{unusingVariables}
\ttNum901108 & (\KeywayMilling 工具{\ttfamily T06}, {\ttfamily T07}用 予備) &\\
\ttNum901109 & 工具{\ttfamily T08}(サイドカッター)厚さ$t$ & 5.0\\
\SetRow{unusingVariables}
\ttNum901110 & (\KeywayMilling 工具{\ttfamily T08}-{\ttfamily T10}用 予備) &\\
\SetRow{unusingVariables}
\ttNum901111 & (\KeywayMilling 工具{\ttfamily T08}-{\ttfamily T10}用 予備) &\\
\SetRow{unusingVariables}
\ttNum901112 & (\KeywayMilling 工具{\ttfamily T08}-{\ttfamily T10}用 予備) &\\
\ttNum901113 & 工具{\ttfamily T11}(テーパエンドミル)参照直径用 工具長補正値 & 2.0\\
\SetRow{unusingVariables}
\ttNum901114 & (工具{\ttfamily T11}用 予備) &\\
\ttNum901115 & 工具{\ttfamily T12}(テーパエンドミル)参照直径用 工具長補正値 & 2.0\\
\SetRow{unusingVariables}
\ttNum901116 & (工具{\ttfamily T12}用 予備) &\\
\ttNum901117 & 工具{\ttfamily T13}(テーパエンドミル)参照直径用 工具長補正値 & 2.0\\
\SetRow{unusingVariables}
\ttNum901118 & (\EndFaceCChamferMilling 工具{\ttfamily T13}-{\ttfamily T15}用 予備) &\\
\SetRow{unusingVariables}
\ttNum901119 & (\EndFaceCChamferMilling 工具{\ttfamily T13}-{\ttfamily T15}用 予備) &\\
\SetRow{unusingVariables}
\ttNum901120 & (\EndFaceCChamferMilling 工具{\ttfamily T13}-{\ttfamily T15}用 予備) &\\
\SetRow{unusingVariables}
\ttNum901121 & (\OutcutMilling 工具{\ttfamily T16}-{\ttfamily T20}用 予備) &\\
\SetRow{unusingVariables}
\ttNum901122 & (\OutcutMilling 工具{\ttfamily T16}-{\ttfamily T20}用 予備) &\\
\SetRow{unusingVariables}
\ttNum901123 & (\OutcutMilling 工具{\ttfamily T16}-{\ttfamily T20}用 予備) &\\
\SetRow{unusingVariables}
\ttNum901124 & (\OutcutMilling 工具{\ttfamily T16}-{\ttfamily T20}用 予備) &\\
\end{multicollongtblr}

\clearpage
%%%%%%%%%%%%%%%%%%%%%%%%%%%%%%%%%%%%%%%%%%%%%%%%%%%%%%%%%%
%% subsection 17.9.2 %%%%%%%%%%%%%%%%%%%%%%%%%%%%%%%%%%%%%
%%%%%%%%%%%%%%%%%%%%%%%%%%%%%%%%%%%%%%%%%%%%%%%%%%%%%%%%%%
\subsection{\ttNum901125\,-\ttNum901149}

\begin{multicollongtblr}[white]{\ttNum901124\,-\ttNum901149:工具({\ttfamily T50}除く)続き}{cX[l]c}
変数 & 内容 & 設定例\\
\SetRow{unusingVariables}
\ttNum901125 & (\OutcutMilling 工具{\ttfamily T16}-{\ttfamily T20}用 予備) &\\
\SetRow{unusingVariables}
\ttNum901126 & (\OutcutMilling 工具{\ttfamily T16}-{\ttfamily T20}用 予備) &\\
\ttNum901127 & 工具{\ttfamily T31}(Tスロットカッター)厚さ & 8.0\\
\ttNum901128 & 工具{\ttfamily T31}(Tスロットカッター)シャンク直径 & 25.0\\
\ttNum901129 & 工具{\ttfamily T31}(Tスロットカッター)中心$X+$補正 & 0.00\\
\ttNum901130 & 工具{\ttfamily T31}(Tスロットカッター)中心$Y+$補正 & -0.030\\
\ttNum901131 & 工具{\ttfamily T32}(Tスロットカッター)厚さ & 8.0\\
\ttNum901132 & 工具{\ttfamily T32}(Tスロットカッター)シャンク直径 & 25.0\\
\ttNum901133 & 工具{\ttfamily T32}(Tスロットカッター)中心$X+$補正 & 0.015\\
\ttNum901134 & 工具{\ttfamily T32}(Tスロットカッター)中心$Y+$補正 & -0.030\\
\SetRow{unusingVariables}
\ttNum901135 & (\DimpleMilling 工具用 予備) &\\
\SetRow{unusingVariables}
\ttNum901136 & (\DimpleMilling 工具用 予備) &\\
\ttNum901137 & 工具{\ttfamily T33}(Tスロットカッター)中心$X+$補正 & 0.00\\
\ttNum901138 & 工具{\ttfamily T33}(Tスロットカッター)中心$Y+$補正 & 0.00\\
\SetRow{unusingVariables}
\ttNum901139 & (予備) &\\
\SetRow{unusingVariables}
\ttNum901140 & (予備) &\\
\SetRow{unusingVariables}
\ttNum901141 & (予備) &\\
\SetRow{unusingVariables}
\ttNum901142 & (予備) &\\
\SetRow{unusingVariables}
\ttNum901143 & (予備) &\\
\SetRow{unusingVariables}
\ttNum901144 & (予備) &\\
\SetRow{unusingVariables}
\ttNum901145 & (予備) &\\
\SetRow{unusingVariables}
\ttNum901146 & (予備) &\\
\SetRow{unusingVariables}
\ttNum901147 & (予備) &\\
\SetRow{unusingVariables}
\ttNum901148 & (予備) &\\
\SetRow{unusingVariables}
\ttNum901149 & (予備) &\\
\end{multicollongtblr}



\clearpage
%%%%%%%%%%%%%%%%%%%%%%%%%%%%%%%%%%%%%%%%%%%%%%%%%%%%%%%%%%
%% section 18.10 %%%%%%%%%%%%%%%%%%%%%%%%%%%%%%%%%%%%%%%%%%
%%%%%%%%%%%%%%%%%%%%%%%%%%%%%%%%%%%%%%%%%%%%%%%%%%%%%%%%%%
\modHeadsection{未使用(使用可)のコモン変数}
この章の各々の表に載せていない\index{コモンへんすう(みしよう)@コモン変数(未使用)}コモン変数は(\dateUnusedVariables 時点において)未使用であり、新たに用いても問題ない(他のNCプログラムと競合しない)。
これらのコモン変数を、以下にまとめておく。
\begin{enumerate}[label=\sarrow]
\item \ttNum575-\ttNum599
\item \ttNum750-\ttNum999
\item \ttNum900050-\ttNum900100
\item \ttNum900501-\ttNum900999
\item \ttNum901025-\ttNum901049
\item \ttNum901075-\ttNum901999
\item \ttNum901150-\ttNum907399
\end{enumerate}

%!TEX root = ../RfCPN.tex


\modHeadchapter[]{\expandafterindex{コモンへんすう(\yomiMMC)@コモン変数(\nameMMC)}コモン変数(\nameMMC)\TBW}
ここでは\MMC の加工システムで使用している\expandafterindex{コモンへんすう(\yomiMMC)@コモン変数(\nameMMC)}コモン変数について述べる。



%%%%%%%%%%%%%%%%%%%%%%%%%%%%%%%%%%%%%%%%%%%%%%%%%%%%%%%%%%
%% section 17.1 %%%%%%%%%%%%%%%%%%%%%%%%%%%%%%%%%%%%%%%%%%
%%%%%%%%%%%%%%%%%%%%%%%%%%%%%%%%%%%%%%%%%%%%%%%%%%%%%%%%%%
\modHeadsection{コモン変数 (コモン変数の範囲)}
(to be written...)

%!TEX root = ../RPA_for_Creating_Program_Note.tex


\modHeadchapter[loC]{関連する著作物およびその提示}
\DMname における\index{ソフトウェア}ソフトウェアに関して、\index{バージョンかんり@バージョン管理}バージョン管理や\index{イシューかんり@イシュー管理}イシュー管理はオンライン上の\index{バージョンかんりシステム@バージョン管理システム}バージョン管理システム, \index{ソースコードかんりシステム@ソースコード管理システム}ソースコード管理(\index{SCM}SCM)システム, \index{リポジトリホスティングサービス}リポジトリホスティングサービスを用いて行われている。
これによりコードの共有・バージョン管理・イシュー管理・ビルド・テストなどの機能を用いることができ、生産性が大きく向上している。

このように、開発・\index{ほしゅ@保守}保守等の生産性の向上を目的としたとき、作成された\index{ちょさくぶつ@著作物}著作物をオンライン上に公開・提示することはメリットが大きい。
一方で、その著作物の中には\index{きみつじこう@機密事項}機密事項を含んでいたり、外部の法人・個人による著作物等も含まれるため、安易に公開してはならないものも存在する。

そのため、ここでは\DMname に関する著作物およびその公表・提示について一定の\index{きじゅん(こうかい・ていじ)@基準(公開・提示)}基準を設ける。



%%%%%%%%%%%%%%%%%%%%%%%%%%%%%%%%%%%%%%%%%%%%%%%%%%%%%%%%%%
%% section 20.1 %%%%%%%%%%%%%%%%%%%%%%%%%%%%%%%%%%%%%%%%%%
%%%%%%%%%%%%%%%%%%%%%%%%%%%%%%%%%%%%%%%%%%%%%%%%%%%%%%%%%%
\modHeadsection{関連する著作物}
\index{マシニングセンタ}マシニングセンタによる\index{モールド}モールドの加工に対して、当社の従業員によって作成されたソフトウェア関連の著作物(以下、\index{かんれんちょさくぶつ@関連著作物}\textbf{関連著作物})として、主に以下のものが挙げられる。
\begin{enumerate}
\item 本書
\item 位置情報等の数値計算用プログラム
\item \index{ないけいテーパひょう(すうちけいさんよう)@内径テーパ表(数値計算用)}数値計算用の内径テーパ(コーナーR含む)データ表
\item \index{しようスペーサけいさんようプログラム@使用スペーサ計算用プログラム}使用スペーサ計算用プログラム
\item \index{G-codeプログラム}G-codeプログラム
%\item モールドのRDB
\item \index{3D CADモデリングようひながた(モールド)@3D CADモデリング用雛型(モールド)}モールドの3D CADモデリング用雛型
\item \index{3D CADモデリングようひながた(ないけいテーパ)@3D CADモデリング用雛型(内径テーパ)}内径テーパの3D CADモデリング用雛型
\end{enumerate}
これらは\index{ちょさくぶつのていぎ@著作物の定義}著作物の定義に該当し、\index{ちょさくけんほう@著作権法}著作権法の保護対象となる。



%%%%%%%%%%%%%%%%%%%%%%%%%%%%%%%%%%%%%%%%%%%%%%%%%%%%%%%%%%
%% section 20.2 %%%%%%%%%%%%%%%%%%%%%%%%%%%%%%%%%%%%%%%%%%
%%%%%%%%%%%%%%%%%%%%%%%%%%%%%%%%%%%%%%%%%%%%%%%%%%%%%%%%%%
\modHeadsection{関連著作物の著作権および著作権者}


%%%%%%%%%%%%%%%%%%%%%%%%%%%%%%%%%%%%%%%%%%%%%%%%%%%%%%%%%%
%% subsection 20.2.1 %%%%%%%%%%%%%%%%%%%%%%%%%%%%%%%%%%%%%
%%%%%%%%%%%%%%%%%%%%%%%%%%%%%%%%%%%%%%%%%%%%%%%%%%%%%%%%%%
\subsection{著作人格権}
すべての\index{かんれんちょさくぶつ@関連著作物}関連著作物の\index{ちょさくじんかくけん@著作人格権}著作人格権は、その\index{ちょさくしゃ@著作者}著作者に帰属する。


%%%%%%%%%%%%%%%%%%%%%%%%%%%%%%%%%%%%%%%%%%%%%%%%%%%%%%%%%%
%% subsection 20.2.2 %%%%%%%%%%%%%%%%%%%%%%%%%%%%%%%%%%%%%
%%%%%%%%%%%%%%%%%%%%%%%%%%%%%%%%%%%%%%%%%%%%%%%%%%%%%%%%%%
\subsection{著作財産権}
当社において作成された\DMname における関連著作物が職務上作成された著作物(\index{しょくむちょさくぶつ@職務著作物}職務著作物)に該当する場合、その\index{ちょさくざいさんけん@著作財産権}著作財産権は当社に帰属する。

関連著作物が職務著作物に該当しない場合、その著作財産権は著作者個人に帰属する。



\clearpage
%%%%%%%%%%%%%%%%%%%%%%%%%%%%%%%%%%%%%%%%%%%%%%%%%%%%%%%%%%
%% section 20.4 %%%%%%%%%%%%%%%%%%%%%%%%%%%%%%%%%%%%%%%%%%
%%%%%%%%%%%%%%%%%%%%%%%%%%%%%%%%%%%%%%%%%%%%%%%%%%%%%%%%%%
\modHeadsection{関連著作物の公表}
\index{かんれんちょさくぶつ@関連著作物}関連著作物の\index{ちょさくしゃ@著作者}著作者は、その\index{ちょさくぶつ@著作物}著作物でまだ公表されていないもの(その同意を得ないで公表された著作物を含む)を公表・提示する権利を有する(当該著作物を原著作物とする二次的著作物についても同様)。


%%%%%%%%%%%%%%%%%%%%%%%%%%%%%%%%%%%%%%%%%%%%%%%%%%%%%%%%%%
%% subsection 20.4.1 %%%%%%%%%%%%%%%%%%%%%%%%%%%%%%%%%%%%%
%%%%%%%%%%%%%%%%%%%%%%%%%%%%%%%%%%%%%%%%%%%%%%%%%%%%%%%%%%
\subsection{公表する関連著作物}

%%%%%%%%%%%%%%%%%%%%%%%%%%%%%%%%%%%%%%%%%%%%%%%%%%%%%%%%%%
%% subsubsection 20.4.2.1 %%%%%%%%%%%%%%%%%%%%%%%%%%%%%%%%
%%%%%%%%%%%%%%%%%%%%%%%%%%%%%%%%%%%%%%%%%%%%%%%%%%%%%%%%%%
\subsubsection{生産性の向上および著作権者の同意}
関連著作物については、開発・保守等の生産性の向上を目的に、原則としてすべてオンライン上に提示する。
なお、次節(非公表にする関連著作物)に該当する著作物に関してはその限りではない。

ただし提示は、その著作物におけるすべての\index{ちょさくじんかくけん@著作人格権}著作人格権の保有者(\index{ちょさくしゃ@著作者}著作者)およびすべての\index{ちょさくざいさんけん@著作財産権}著作財産権の保有者の、全員の同意の下で行われることを前提とする。


%%%%%%%%%%%%%%%%%%%%%%%%%%%%%%%%%%%%%%%%%%%%%%%%%%%%%%%%%%
%% subsubsection 20.4.2.2 %%%%%%%%%%%%%%%%%%%%%%%%%%%%%%%%
%%%%%%%%%%%%%%%%%%%%%%%%%%%%%%%%%%%%%%%%%%%%%%%%%%%%%%%%%%
\subsubsection{個人の著作権者による提示にの権利\label{subsec:individualright}}
著作人格権および著作財産権の保有者が同一人物であり、かつ1人の個人のみである場合は、その個人が提示の行為を行ってよいものとする。

%%%%%%%%%%%%%%%%%%%%%%%%%%%%%%%%%%%%%%%%%%%%%%%%%%%%%%%%%%
%% subsubsection 20.4.2.3 %%%%%%%%%%%%%%%%%%%%%%%%%%%%%%%%
%%%%%%%%%%%%%%%%%%%%%%%%%%%%%%%%%%%%%%%%%%%%%%%%%%%%%%%%%%
\subsubsection{データ保護とプライバシー}
公表の際は、\index{こじんじょうほうほごほう@個人情報保護法}\href{https://elaws.e-gov.go.jp/document?lawid=415AC0000000057}{個人情報の保護に関する法律}(個人情報保護法)\cite{online:eGovPersonalInfoProtectionLaw}に基づいて、\index{データほご@データ保護}データ保護・\index{プライバシーほご@プライバシー保護}プライバシー保護に十分に配慮しなければならない。


%%%%%%%%%%%%%%%%%%%%%%%%%%%%%%%%%%%%%%%%%%%%%%%%%%%%%%%%%%
%% subsection 20.4.2 %%%%%%%%%%%%%%%%%%%%%%%%%%%%%%%%%%%%%
%%%%%%%%%%%%%%%%%%%%%%%%%%%%%%%%%%%%%%%%%%%%%%%%%%%%%%%%%%
\subsection{非公表にする関連著作物}

%%%%%%%%%%%%%%%%%%%%%%%%%%%%%%%%%%%%%%%%%%%%%%%%%%%%%%%%%%
%% subsubsection 20.4.2.1 %%%%%%%%%%%%%%%%%%%%%%%%%%%%%%%%
%%%%%%%%%%%%%%%%%%%%%%%%%%%%%%%%%%%%%%%%%%%%%%%%%%%%%%%%%%
\subsubsection{機密情報の保護および提示の範囲\label{subsec:notopenwork}}
作成した\index{メインプログラム}メインプログラムやモールドの\index{データベース(モールド)}データベースについては、個々の\index{めいさい(モールド)@明細(モールド)}明細の情報(\index{きみつじこう@機密事項}機密事項)を推察できるデータを含む。
このような機密事項を含む著作物については、原則として提示はしないものとする。

提示する場合は(個々のものすべてでなく)代表的・典型的なものに留めるか、あるいは許可を得た者のみが閲覧可能な状態として提示を行うものとする。

%%%%%%%%%%%%%%%%%%%%%%%%%%%%%%%%%%%%%%%%%%%%%%%%%%%%%%%%%%
%% subsubsection 20.4.2.2 %%%%%%%%%%%%%%%%%%%%%%%%%%%%%%%%
%%%%%%%%%%%%%%%%%%%%%%%%%%%%%%%%%%%%%%%%%%%%%%%%%%%%%%%%%%
\subsubsection{外部作成の著作物および著作権者の同意\label{subsec:standardscopyrightsSubcontractor}}
\index{プログラム}プログラムの中には外注先で作成されたものも存在する。
このような\index{ちょさくけんしゃ@著作権者}著作権者(特に著作財産権者)が当社の従業員または当社自体ではない著作物については、原則として公表しない。
公表する場合は、すべての著作人格権者およびすべての著作財産権者の同意の下に行われるものとする。



\clearpage
~\vfill
\begin{Column}{\DMname の関連著作物}
\DMname の社内で作成された\index{ソフトウェアかんれんちょさくぶつ@ソフトウェア関連著作物}ソフトウェア関連著作物については、先にも述べた通りその一部がオンライン上に提示されている。
これは、その\DMname の立上げに関わる必要な(\index{ソフトウェア}ソフトウェアにおける)社内の業務の一切が、\index{かんれんちょさくぶつ@関連著作物}関連著作物の作成開始時から作成終了後(保守・管理を含む)に至るまで、1人の一般職である著作者個人の独力に(管理職・スタッフにより)一任されており、以下のような状態にあることが背景にある。
\tcbline*
\begin{enumerate}[label=\Roman*]
\item 当社がある目的を持って構想した著作物は(ソフトウェアに関しては)全く存在しない。
\item
著作者(一般職)は主に以下のような作業を単独かつ独力で行っており、通常の業務範囲および業務量から大きく逸脱しているのは明白である。
  \begin{enumerate}
  \item[-] ソフトウェアの観点からみた現状の業務フローの調査・整理
  \item[-] 現行の業務フローにおける問題点・改善可能点の抽出
  \item[-] マシニングセンタ導入後の業務フローの計画の策定
  \item[-] マシニングセンタの稼働に対する要件定義
  \item[-] マシニングセンタの稼働に対するシステム設計
  \item[-] マシニングセンタの稼働に対する詳細設計
  \item[-] ソフトウェア開発に関わる諸規定の策定および作成
  \item[-] ソフトウェア開発に関わる諸標準の策定および作成
  \item[-] マシニングセンタ内におけるモールドの幾何学的形状の解析計算および体系化
  \item[-] 明細ごとの具体的な数値の自動取得システムの設計
  \item[-] 明細ごとの具体的な数値の自動取得システムの構築および作成
  \item[-] 加工用サブプログラムの作成
  \item[-] 加工用メインプログラムの作成
  \item[-] プログラムの実装
  \item[-] プログラムに対する統合試運転を除く試運転
  \item[-] プログラムの修正保守
  \item[-] プログラムの機能追加保守
  \item[-] モールドの関係データベースの設計
  \item[-] モールドの関係データベースの作成
  \item[-] 加工用メインプログラムの自動作成システムの設計
  \item[-] 加工用メインプログラムの自動作成システムの構築および作成
  \item[-] 関連ドキュメントの作成および管理
  \end{enumerate}
\item 関連著作物は著作者個人の氏名あるいはアカウントの下に、オンライン上に提示されている。
\item ソフトウェア作成時において、その\index{ちょさくぶつ@著作物}著作物についての別段の定め等は本書を除いて一切ない。
\end{enumerate}
\tcbline*
したがって、いずれの要件も満たしていないため、その\index{ちょさくざいさんけん@著作財産権}著作財産権は著作者個人に帰属する。
関連著作物(一部)のオンライン上の提示は、\pageautoref{subsec:individualright}に基づいて行われている。
\end{Column}
\cleardoublepage



%%%%%%%%%%%%%%%%%%%%%%%%%%%%%%%%%%%%%%%%%%%%%%%%%%%%%%%%%
%% Appendices %%%%%%%%%%%%%%%%%%%%%%%%%%%%%%%%%%%%%%%%%%%
%%%%%%%%%%%%%%%%%%%%%%%%%%%%%%%%%%%%%%%%%%%%%%%%%%%%%%%%%
\begin{appendices}
\Appendixpart
%!TEX root = ../RPA_for_Creating_Program_Note.tex


ここでは\DMname の主な\index{システムへんすう@システム変数}システム変数を記載する。
なお、使用の欄は、●は作成したプログラムで使用しているもの、◯はバンドルのプログラムのみで使用されているものを示す。



%%%%%%%%%%%%%%%%%%%%%%%%%%%%%%%%%%%%%%%%%%%%%%%%%%%%%%%%%%
%% section A.1 %%%%%%%%%%%%%%%%%%%%%%%%%%%%%%%%%%%%%%%%%%%
%%%%%%%%%%%%%%%%%%%%%%%%%%%%%%%%%%%%%%%%%%%%%%%%%%%%%%%%%%
\modHeadsection{\ttNum1000\,-\ttNum1999\TBW}

\begin{3columnstable}[white]{\ttNum1000-\ttNum1000: \index{パレット}{パレット}\TBW}{|Sc|Sl|Sc|}{番号}{内容\hspace*{0.65\textwidth}~}{使用}
\ttNum1000 & パレット\ttNum~~0:\ttNum1, 1:\ttNum2 & ●\\\hline
\ttNum1001 & & ◯\\\hline
\ttNum1002 & & ◯\\\hline
\ttNum1003 & & ◯\\
\end{3columnstable}

\begin{3columnstable}[white]{\ttNum1004-\ttNum1005: タッチセンサー電源}{|Sc|Sl|Sc|}{番号}{内容\hspace*{0.65\textwidth}~}{使用}
\ttNum1004 & タッチセンサー電源~~0: off, 1: on & ●\\\hline
\ttNum1005 & タッチセンサー電池残量~~0: OK, 1: low & ●
\end{3columnstable}

\begin{3columnstable}[white]{\ttNum1010-\ttNum1015: \TBW}{|Sc|Sl|Sc|}{番号}{内容\hspace*{0.65\textwidth}~}{使用}
\ttNum1010 & & ◯\\\hline
\ttNum1011 & & ◯\\\hline
\ttNum1012 & & ◯\\\hline
\ttNum1013 & & ◯\\\hline
\ttNum1014 & & ◯\\\hline
\ttNum1015 & & ◯\\
\end{3columnstable}



\clearpage
%%%%%%%%%%%%%%%%%%%%%%%%%%%%%%%%%%%%%%%%%%%%%%%%%%%%%%%%%%
%% section A.2 %%%%%%%%%%%%%%%%%%%%%%%%%%%%%%%%%%%%%%%%%%%
%%%%%%%%%%%%%%%%%%%%%%%%%%%%%%%%%%%%%%%%%%%%%%%%%%%%%%%%%%
\modHeadsection{\ttNum2000\,-\ttNum2999\TBW}

\begin{3columnstable}[white]{\ttNum2001-\ttNum2800: 工具補正}{|Sc|Sl|Sc|}{番号}{内容\hspace*{0.65\textwidth}~}{使用}
\ttNum2000+xxx & 工具長補正 \ttNum xxx補正量 [xxx=1-200](摩耗0とした値) & ●\\\hline
\ttNum2200+xxx & 工具長補正 \ttNum xxx摩耗 [xxx=1-200] & ●\\\hline
\ttNum2400+xxx & 工具径補正 \ttNum xxx補正量 [xxx=1-200](摩耗0とした値) & ●\\\hline
\ttNum2600+xxx & 工具径補正 \ttNum xxx摩耗 [xxx=1-200] & ●
\end{3columnstable}



%\clearpage
%%%%%%%%%%%%%%%%%%%%%%%%%%%%%%%%%%%%%%%%%%%%%%%%%%%%%%%%%%
%% section A.3 %%%%%%%%%%%%%%%%%%%%%%%%%%%%%%%%%%%%%%%%%%%
%%%%%%%%%%%%%%%%%%%%%%%%%%%%%%%%%%%%%%%%%%%%%%%%%%%%%%%%%%
\modHeadsection{\ttNum3000\,-\ttNum3999\TBW}

\begin{3columnstable}[white]{\ttNum3000: アラーム}{|Sc|Sl|Sc|}{番号}{内容\hspace*{0.65\textwidth}~}{使用}
\ttNum3000 & アラーム & ●
\end{3columnstable}

\begin{3columnstable}[white]{\ttNum3003: \TBW}{|Sc|Sl|Sc|}{番号}{内容\hspace*{0.65\textwidth}~}{使用}
\ttNum3003 & & ◯\\\hline
\ttNum3006 & & ◯
\end{3columnstable}

\begin{3columnstable}[white]{\ttNum3011-\ttNum3011: 時計・時刻}{|Sc|Sl|Sc|}{番号}{内容\hspace*{0.65\textwidth}~}{使用}
\ttNum3011 & 現在の年月日(yyyymmdd) &
\end{3columnstable}



\clearpage
%%%%%%%%%%%%%%%%%%%%%%%%%%%%%%%%%%%%%%%%%%%%%%%%%%%%%%%%%%
%% section A.4 %%%%%%%%%%%%%%%%%%%%%%%%%%%%%%%%%%%%%%%%%%%
%%%%%%%%%%%%%%%%%%%%%%%%%%%%%%%%%%%%%%%%%%%%%%%%%%%%%%%%%%
\modHeadsection{\ttNum4000\,-\ttNum4999\TBW}

\begin{3columnstable}[white]{\ttNum4001-\ttNum4020: 直前までのブロックで指令されたG指令モーダル情報}{|Sc|Sl|Sc|}{番号}{内容\hspace*{0.65\textwidth}~}{使用}
\ttNum4001 & グループ01:\verb|G00|, \verb|G01|, \verb|G02|, \verb|G03| & ◯\\\hline
\ttNum4002 & グループ02:選択平面\verb|G17|, \verb|G18|, \verb|G19| & \\\hline
\ttNum4003 & グループ03:\verb|G90|, \verb|G91| & ◯\\\hline
\ttNum4004 & グループ04:\verb|G22|, \verb|G23| & \\\hline
\ttNum4005 & グループ05:\verb|G94| & \\\hline
\ttNum4006 & グループ06:\verb|G21| & \\\hline
\ttNum4007 & グループ07:工具径補正\verb|G40|, \verb|G41|, \verb|G43| & \\\hline
\ttNum4008 & グループ08:工具長補正\verb|G43|, \verb|G44|, \verb|G49| & \\\hline
\ttNum4009 & グループ09:固定サイクル\verb|G80| & \\\hline
\ttNum4010 & グループ10:\verb|G98| & \\\hline
\ttNum4011 & グループ11:スケーリング\verb|G50| & \\\hline
\ttNum4012 & グループ12:ワーク座標系\verb|G54|-\verb|G59|, \verb|G54.1|  & ● \\\hline
\ttNum4013 & グループ13:高精度制御\verb|G61.1| & \\\hline
\ttNum4014 & グループ14:\verb|G67| & \\\hline
\ttNum4015 & グループ15:法線制御\verb|G40.1| & \\\hline
\ttNum4016 & グループ16:\verb|G69| & \\\hline
\ttNum4017 & グループ17:周速一定制御\verb|G97| & \\\hline
\ttNum4018 & グループ18:極座標指令\verb|G15| & \\\hline
\ttNum4019 & グループ19:G指令ミラーイメージ\verb|G50.1| & \\\hline
\rowcolor{unusingVariables}
\ttNum4020 & null & \\
\end{3columnstable}
%%%%%%%%%%%%%%%%%%%%%%%%%%%%%%%%%%%%%%%%%%%%%%%%%%%%%%%%%%
%% hosoku %%%%%%%%%%%%%%%%%%%%%%%%%%%%%%%%%%%%%%%%%%%%%%%%
%%%%%%%%%%%%%%%%%%%%%%%%%%%%%%%%%%%%%%%%%%%%%%%%%%%%%%%%%%
\begin{hosoku}
グループについては\pageautoref{chap:VII.H}を参照。
ここでは主な\verb|G#|を挙げている。
\end{hosoku}
%%%%%%%%%%%%%%%%%%%%%%%%%%%%%%%%%%%%%%%%%%%%%%%%%%%%%%%%%%
%%%%%%%%%%%%%%%%%%%%%%%%%%%%%%%%%%%%%%%%%%%%%%%%%%%%%%%%%%
%%%%%%%%%%%%%%%%%%%%%%%%%%%%%%%%%%%%%%%%%%%%%%%%%%%%%%%%%%


\begin{3columnstable}[white]{\ttNum4021-\ttNum4021}{|Sc|Sl|Sc|}{番号}{内容\hspace*{0.65\textwidth}~}{使用}
\ttNum4021 & & \\
\end{3columnstable}


\clearpage
\begin{3columnstable}[white]{\ttNum4102-\ttNum4129: 直前までのブロックで指令されたモーダル情報}{|Sc|Sl|Sc|}{番号}{内容\hspace*{0.65\textwidth}~}{使用}
\ttNum4102 &  & \\\hline
\rowcolor{unusingVariables}
\ttNum4103 & null & \\\hline
\rowcolor{unusingVariables}
$\cdots$ & $\cdots$ & \\\hline
\ttNum4107 & \verb|D#| 工具径補正コード\ttNum & ●\\\hline
\ttNum4108 &  & \\\hline
\ttNum4109 & \verb|F#| Fコード\ttNum &\\\hline
\rowcolor{unusingVariables}
\ttNum4110 & null & \\\hline
\ttNum4111 & \verb|H#| 工具長補正コード\ttNum & ●\\\hline
\rowcolor{unusingVariables}
\ttNum4112 & null & \\\hline
\ttNum4113 & \verb|M#| Mコード\ttNum &\\\hline
\ttNum4114 & \verb|N#| シーケンス\ttNum &\\\hline
\ttNum4115 & \verb|O#| プログラム\ttNum &\\\hline
\rowcolor{unusingVariables}
\ttNum4116 & null & \\\hline
\rowcolor{unusingVariables}
$\cdots$ & $\cdots$ & \\\hline
\ttNum4119 & \verb|S#| Sコード\ttNum &\\\hline
\ttNum4120 & \verb|T#| 工具コード\ttNum & \\\hline
\rowcolor{unusingVariables}
\ttNum4121 & null & \\\hline
\rowcolor{unusingVariables}
$\cdots$ & $\cdots$ & \\
\end{3columnstable}


%\clearpage
\begin{3columnstable}[white]{\ttNum4130-\ttNum4140}{|Sc|Sl|Sc|}{番号}{内容\hspace*{0.65\textwidth}~}{使用}
\ttNum4130 & & ◯\\\hline
\rowcolor{unusingVariables}
\ttNum4131 & null & \\\hline
\rowcolor{unusingVariables}
$\cdots$ & $\cdots$ & \\
\end{3columnstable}


%\clearpage
\begin{3columnstable}[white]{\ttNum4999-\ttNum4999}{|Sc|Sl|Sc|}{番号}{内容\hspace*{0.65\textwidth}~}{使用}
\rowcolor{unusingVariables}
\ttNum4999 & null & \\
\end{3columnstable}



\clearpage
%%%%%%%%%%%%%%%%%%%%%%%%%%%%%%%%%%%%%%%%%%%%%%%%%%%%%%%%%%
%% section A.5 %%%%%%%%%%%%%%%%%%%%%%%%%%%%%%%%%%%%%%%%%%%
%%%%%%%%%%%%%%%%%%%%%%%%%%%%%%%%%%%%%%%%%%%%%%%%%%%%%%%%%%
\modHeadsection{\ttNum5000\,-\ttNum5999\TBW}

\begin{3columnstable}[white]{\ttNum5001-\ttNum5329: 位置情報}{|Sc|Sl|Sc|}{番号}{内容\hspace*{0.65\textwidth}~}{使用}
\rowcolor{unusingVariables}
\ttNum5000 & null &\\\hline
\ttNum500x\footnote{\ttNum500x:工具補正値が加味された値(ワーク座標系に表示される数値)。途中でスキップがオンになったときはそのときの値。}
       & ブロック終点位置(現在のワーク座標系)[x=1-9] & ●\\\hline
\rowcolor{unusingVariables}
\ttNum5010 & null &\\\hline
\rowcolor{unusingVariables}
\ttNum5011 & null &\\\hline
\ttNum502x & 現在の機械座標系の座標 [x=1-9] & ●\\\hline
\ttNum504x & 現在のワーク座標系の座標 [x=1-9] & ●\\\hline
\ttNum506x & スキップ座標 [x=1-9](工具補正0とした値) & ●\\\hline
\ttNum5203 & & ◯\\\hline
\ttNum522x & ワーク座標系G54原点の機械座標 [x=1-9] & ●\\\hline
\ttNum524x & ワーク座標系G55原点の機械座標 [x=1-9] & ●\\\hline
\ttNum526x & ワーク座標系G56原点の機械座標 [x=1-9] & ●\\\hline
\ttNum528x & ワーク座標系G57原点の機械座標 [x=1-9] & ●\\\hline
\ttNum530x & ワーク座標系G58原点の機械座標 [x=1-9] &\\\hline
\ttNum532x & ワーク座標系G59原点の機械座標 [x=1-9] &
\end{3columnstable}



%\clearpage
%%%%%%%%%%%%%%%%%%%%%%%%%%%%%%%%%%%%%%%%%%%%%%%%%%%%%%%%%%
%% section A.06 %%%%%%%%%%%%%%%%%%%%%%%%%%%%%%%%%%%%%%%%%%
%%%%%%%%%%%%%%%%%%%%%%%%%%%%%%%%%%%%%%%%%%%%%%%%%%%%%%%%%%
\modHeadsection{\ttNum7000\,-\ttNum7999\TBW}

\begin{3columnstable}[white]{\ttNum7000-\ttNum7009}{|Sc|Sl|Sc|}{番号}{内容\hspace*{0.65\textwidth}~}{使用}
\rowcolor{unusingVariables}
\ttNum7000 & null &\\\hline
\ttNum700x & パレット中心の機械座標 [x=1-9] & ◯\\
\end{3columnstable}


\begin{3columnstable}[white]{\ttNum7010-\ttNum7100}{|Sc|Sl|Sc|}{番号}{内容\hspace*{0.65\textwidth}~}{使用}
\rowcolor{unusingVariables}
\ttNum7010 & null &\\\hline
\rowcolor{unusingVariables}
\ttNum7100 & null &\\
\end{3columnstable}



\clearpage
%%%%%%%%%%%%%%%%%%%%%%%%%%%%%%%%%%%%%%%%%%%%%%%%%%%%%%%%%%
%% section A.07 %%%%%%%%%%%%%%%%%%%%%%%%%%%%%%%%%%%%%%%%%%
%%%%%%%%%%%%%%%%%%%%%%%%%%%%%%%%%%%%%%%%%%%%%%%%%%%%%%%%%%
\modHeadsection{\ttNum10000\,-\ttNum11999:工具長補正\TBW}

\begin{3columnstable}[white]{\ttNum10000-\ttNum11999}{|Sc|Sl|Sc|}{番号}{内容\hspace*{0.65\textwidth}~}{使用}
\ttNum10000+xxx & 工具長補正 \ttNum xxx補正量 [xxx=1-200](摩耗0とした値) & ◯\\\hline
\rowcolor{unusingVariables}
\ttNum10201 & null &\\\hline
\ttNum11000+xxx & 工具長 \ttNum xxx摩耗量 [xxx=1-200] & ◯
\end{3columnstable}



%\clearpage
%%%%%%%%%%%%%%%%%%%%%%%%%%%%%%%%%%%%%%%%%%%%%%%%%%%%%%%%%%
%% section A.08 %%%%%%%%%%%%%%%%%%%%%%%%%%%%%%%%%%%%%%%%%%
%%%%%%%%%%%%%%%%%%%%%%%%%%%%%%%%%%%%%%%%%%%%%%%%%%%%%%%%%%
\modHeadsection{\ttNum16000\,-\ttNum17999:工具径補正\TBW}

\begin{3columnstable}[white]{\ttNum16001-\ttNum17999}{|Sc|Sl|Sc|}{番号}{内容\hspace*{0.65\textwidth}~}{使用}
\ttNum16000+xxx & 工具径補正 \ttNum xxx補正量 [xxx=1-200](摩耗0とした値) & ◯\\\hline
\ttNum17000+xxx & 工具径 \ttNum xxx摩耗量 [xxx=1-200] & ◯\\
\end{3columnstable}



%\clearpage
%%%%%%%%%%%%%%%%%%%%%%%%%%%%%%%%%%%%%%%%%%%%%%%%%%%%%%%%%%
%% section A.09 %%%%%%%%%%%%%%%%%%%%%%%%%%%%%%%%%%%%%%%%%%
%%%%%%%%%%%%%%%%%%%%%%%%%%%%%%%%%%%%%%%%%%%%%%%%%%%%%%%%%%
\modHeadsection{\ttNum100000\,-\ttNum100999\TBW}

\begin{3columnstable}[white]{\ttNum100000-\ttNum100011}{|Sc|Sl|Sc|}{番号}{内容\hspace*{0.65\textwidth}~}{使用}
\ttNum100000 & & ◯\\\hline
\ttNum100001 & & \\\hline
\ttNum100002 & & ◯\\\hline
\rowcolor{unusingVariables}
\ttNum100003 & null & \\\hline
\rowcolor{unusingVariables}
$\cdots$ & $\cdots$ & \\\hline
\ttNum100010 & & ◯\\\hline
\rowcolor{unusingVariables}
\ttNum100011 & null & \\
\end{3columnstable}

%!TEX root = ../RPA_for_Creating_Program_Note.tex



%%%%%%%%%%%%%%%%%%%%%%%%%%%%%%%%%%%%%%%%%%%%%%%%%%%%%%%%%%
%% section 13.1 %%%%%%%%%%%%%%%%%%%%%%%%%%%%%%%%%%%%%%%%%%
%%%%%%%%%%%%%%%%%%%%%%%%%%%%%%%%%%%%%%%%%%%%%%%%%%%%%%%%%%
\modHeadsection{\ttNum4000\,-\ttNum4999}

\begin{2commonvariables}{\ttNum4111-\ttNum4120: 直前までのブロックで指令されたモーダル情報}{番号}{内容\hspace*{0.72\textwidth}~}
\ttNum4111 & \verb|H#| 工具長補正コード\ttNum\\\hline
\ttNum4120 & \verb|T#| 工具コード\ttNum
\end{2commonvariables}


%%%%%%%%%%%%%%%%%%%%%%%%%%%%%%%%%%%%%%%%%%%%%%%%%%%%%%%%%%
%% section 13.1 %%%%%%%%%%%%%%%%%%%%%%%%%%%%%%%%%%%%%%%%%%
%%%%%%%%%%%%%%%%%%%%%%%%%%%%%%%%%%%%%%%%%%%%%%%%%%%%%%%%%%
\modHeadsection{\ttNum5000\,-\ttNum5999}

\begin{2commonvariables}{\ttNum5021-\ttNum5328: 位置情報}{番号}{内容\hspace*{0.72\textwidth}~}
\ttNum502x & 現在の機械座標系の座標 1:X, 2:Y, 3:Z, 4:B\\\hline
\ttNum504x & 現在のワーク座標系の座標 1:X, 2:Y, 3:Z, 4:B\\\hline
\ttNum506x & スキップ座標 1:X, 2:Y, 3:Z, 4:B(工具補正0とした値)\\\hline
\ttNum522x & ワーク座標系G54原点の機械座標 1:X, 2:Y, 3:Z, 4:B\\\hline
\ttNum524x & ワーク座標系G55原点の機械座標 1:X, 2:Y, 3:Z, 4:B\\\hline
\ttNum526x & ワーク座標系G56原点の機械座標 1:X, 2:Y, 3:Z, 4:B\\\hline
\ttNum528x & ワーク座標系G57原点の機械座標 1:X, 2:Y, 3:Z, 4:B\\\hline
\ttNum530x & ワーク座標系G58原点の機械座標 1:X, 2:Y, 3:Z, 4:B\\\hline
\ttNum532x & ワーク座標系G59原点の機械座標 1:X, 2:Y, 3:Z, 4:B\\
\end{2commonvariables}



%!TEX root = ../RPA_for_Creating_Program_Note.tex


\modHeadchapter[loC,lot]{引数指定}



%%%%%%%%%%%%%%%%%%%%%%%%%%%%%%%%%%%%%%%%%%%%%%%%%%%%%%%%%%
%% section F.1 %%%%%%%%%%%%%%%%%%%%%%%%%%%%%%%%%%%%%%%%%%%
%%%%%%%%%%%%%%%%%%%%%%%%%%%%%%%%%%%%%%%%%%%%%%%%%%%%%%%%%%
\modHeadsection{引数の指定}
\index{ひきすう@引数}引数の指定の仕方は2通りある。
ここではそれらを\index{ひきすうしてい@引数指定}引数指定Iおよび引数指定IIと呼ぶことにする。


%%%%%%%%%%%%%%%%%%%%%%%%%%%%%%%%%%%%%%%%%%%%%%%%%%%%%%%%%%
%% subsection 10.4.1 %%%%%%%%%%%%%%%%%%%%%%%%%%%%%%%%%%%%%
%%%%%%%%%%%%%%%%%%%%%%%%%%%%%%%%%%%%%%%%%%%%%%%%%%%%%%%%%%
\subsection{引数指定I}
\index{ひきすうしていI@引数指定I}引数指定Iでは、\index{ひきすうアドレス@引数アドレス}引数アドレスとしてA-Zのアルファベットをそれぞれ1回ずつ用いることができる。
そのため使用できる引数の数は、アルファベットの数(26個)である。
ただし、通常はG, L, N, O, Pは用いることができず、実質的に21個が使用可能な数である。


%%%%%%%%%%%%%%%%%%%%%%%%%%%%%%%%%%%%%%%%%%%%%%%%%%%%%%%%%%
%% subsection 10.4.1 %%%%%%%%%%%%%%%%%%%%%%%%%%%%%%%%%%%%%
%%%%%%%%%%%%%%%%%%%%%%%%%%%%%%%%%%%%%%%%%%%%%%%%%%%%%%%%%%
\subsection{引数指定II}
\index{ひきすうしていII@引数指定II}引数指定IIでは、引数アドレスとしてA, B, Cを1回とI, J, Kを10組まで用いることができる。
そのため使用できる引数の数は、33個である。

A, B, Cにはそれぞれ\ttNum1, \ttNum2, \ttNum3が割り当てられ、I, J, Kは入力の順序でマクロの\ttNum4から順番に\ttNum33まで割り当てられる。



\clearpage
%%%%%%%%%%%%%%%%%%%%%%%%%%%%%%%%%%%%%%%%%%%%%%%%%%%%%%%%%%
%% section F.2 %%%%%%%%%%%%%%%%%%%%%%%%%%%%%%%%%%%%%%%%%%%
%%%%%%%%%%%%%%%%%%%%%%%%%%%%%%%%%%%%%%%%%%%%%%%%%%%%%%%%%%
\modHeadsection{引数アドレスとそのローカル変数}
原則として、引数の数が22個以上必要でない限り、\DMC では引数指定Iを用いるものとする。


%%%%%%%%%%%%%%%%%%%%%%%%%%%%%%%%%%%%%%%%%%%%%%%%%%%%%%%%%%
%% subsection C.2.1 %%%%%%%%%%%%%%%%%%%%%%%%%%%%%%%%%%%%%%
%%%%%%%%%%%%%%%%%%%%%%%%%%%%%%%%%%%%%%%%%%%%%%%%%%%%%%%%%%
\subsection{引数指定I 一覧}
\index{ひきすうアドレス(ひきすうしていI)@引数アドレス(引数指定I)}引数指定Iの引数アドレスとそれに対応する\index{ローカルへんすう(ひきすう)@ローカル変数(引数)}ローカル変数は以下の通りである。\\
\noindent%
\begin{minipage}[t]{0.66\textwidth}
%%%%%%%%%%%%%%%%%%%%%%%%%%%%%%%%%%%%%%%%%%%%%%%%%%%%%%%%%%
%% captionof %%%%%%%%%%%%%%%%%%%%%%%%%%%%%%%%%%%%%%%%%%%%%
%%%%%%%%%%%%%%%%%%%%%%%%%%%%%%%%%%%%%%%%%%%%%%%%%%%%%%%%%%
\begin{twocolbreaktblr}{引数指定I 一覧}{cc|[3pt, white!0!]|cc|[3pt, white!0!]|cc|[3pt, white!0!]|cc}
\cmidrule[r=0]{1-2}\cmidrule[lr=0]{3-4}\cmidrule[lr=0]{5-6}\cmidrule[l=0]{7-8}
記号 & 変数 & 記号 & 変数 & 記号 & 変数 & 記号 & 変数\\
\cmidrule[r=0]{1-2}\cmidrule[lr=0]{3-4}\cmidrule[lr=0]{5-6}\cmidrule[l=0]{7-8}
A & \ttfamily\#01 & H & \ttfamily\#11 & R & \ttfamily\#18 & X & \ttfamily\#24\\
\cmidrule[r=0]{1-2}\cmidrule[lr=0]{3-4}\cmidrule[lr=0]{5-6}\cmidrule[l=0]{7-8}
B & \ttfamily\#02 & I & \ttfamily\#04 & S & \ttfamily\#19 & Y & \ttfamily\#25\\
\cmidrule[r=0]{1-2}\cmidrule[lr=0]{3-4}\cmidrule[lr=0]{5-6}\cmidrule[l=0]{7-8}
C & \ttfamily\#03 & J & \ttfamily\#05 & T & \ttfamily\#20 & Z & \ttfamily\#26\\
\cmidrule[r=0]{1-2}\cmidrule[lr=0]{3-4}\cmidrule[lr=0]{5-6}\cmidrule[l=0]{7-8}
D & \ttfamily\#07 & K & \ttfamily\#06 & U & \ttfamily\#21\\
\cmidrule[r=0]{1-2}\cmidrule[lr=0]{3-4}\cmidrule[lr=0]{5-6}\cmidrule[l=0]{7-8}
E & \ttfamily\#08 & M & \ttfamily\#13 & V & \ttfamily\#22\\
\cmidrule[r=0]{1-2}\cmidrule[lr=0]{3-4}\cmidrule[lr=0]{5-6}\cmidrule[l=0]{7-8}
F & \ttfamily\#09 & Q & \ttfamily\#17 & W & \ttfamily\#23\\
\cmidrule[r=0]{1-2}\cmidrule[lr=0]{3-4}\cmidrule[lr=0]{5-6}\cmidrule[l=0]{7-8}
\end{twocolbreaktblr}%
\end{minipage}%
\begin{minipage}[t]{0.34\textwidth}
%%%%%%%%%%%%%%%%%%%%%%%%%%%%%%%%%%%%%%%%%%%%%%%%%%%%%%%%%%
%% captionof %%%%%%%%%%%%%%%%%%%%%%%%%%%%%%%%%%%%%%%%%%%%%
%%%%%%%%%%%%%%%%%%%%%%%%%%%%%%%%%%%%%%%%%%%%%%%%%%%%%%%%%%
\begin{twocolbreaktblr}{通常指定不可な引数}{cc}
\cmidrule{1-Z}
記号 & 変数\\
\cmidrule{1-Z}
G & \ttfamily\#10\\
\cmidrule{1-Z}
L & \ttfamily\#12\\
\cmidrule{1-Z}
N & \ttfamily\#14\\
\cmidrule{1-Z}
O & \ttfamily\#15\\
\cmidrule{1-Z}
P & \ttfamily\#16\\
\cmidrule{1-Z}
\end{twocolbreaktblr}%
\end{minipage}


%%%%%%%%%%%%%%%%%%%%%%%%%%%%%%%%%%%%%%%%%%%%%%%%%%%%%%%%%%
%% subsection C.2.2 %%%%%%%%%%%%%%%%%%%%%%%%%%%%%%%%%%%%%%
%%%%%%%%%%%%%%%%%%%%%%%%%%%%%%%%%%%%%%%%%%%%%%%%%%%%%%%%%%
\subsection{引数指定II 一覧}
\index{ひきすうアドレス(ひきすうしていII)@引数アドレス(引数指定II)}引数指定IIの引数アドレスとそれに対応するローカル変数は以下の通りである。
なお、I, J, Kの添字は便宜上記述したものであり、実際のNCプログラムにはすべて同じI, J, Kの記号を用いる。\\

%%%%%%%%%%%%%%%%%%%%%%%%%%%%%%%%%%%%%%%%%%%%%%%%%%%%%%%%%%
%% captionof %%%%%%%%%%%%%%%%%%%%%%%%%%%%%%%%%%%%%%%%%%%%%
%%%%%%%%%%%%%%%%%%%%%%%%%%%%%%%%%%%%%%%%%%%%%%%%%%%%%%%%%%
\begin{twocolbreaktblr}{引数指定II 一覧}{cc|[3pt, white!0!]|cc|[3pt, white!0!]|cc|[3pt, white!0!]|cc}
\cmidrule[r=0]{1-2}\cmidrule[lr=0]{3-4}\cmidrule[lr=0]{5-6}\cmidrule[l=0]{7-8}
記号 & 変数 & 記号 & 変数 & 記号 & 変数 & 記号 & 変数\\
\cmidrule[r=0]{1-2}\cmidrule[lr=0]{3-4}\cmidrule[lr=0]{5-6}\cmidrule[l=0]{7-8}
A & \ttfamily\#01 & I$_3$ & \ttfamily\#10 & I$_6$ & \ttfamily\#19 & I$_9$ & \ttfamily\#28\\
\cmidrule[r=0]{1-2}\cmidrule[lr=0]{3-4}\cmidrule[lr=0]{5-6}\cmidrule[l=0]{7-8}
B & \ttfamily\#02 & J$_3$ & \ttfamily\#11 & J$_6$ & \ttfamily\#20 & J$_9$ & \ttfamily\#29\\
\cmidrule[r=0]{1-2}\cmidrule[lr=0]{3-4}\cmidrule[lr=0]{5-6}\cmidrule[l=0]{7-8}
C & \ttfamily\#03 & K$_3$ & \ttfamily\#12 & K$_6$ & \ttfamily\#21 & K$_9$ & \ttfamily\#30\\
\cmidrule[r=0]{1-2}\cmidrule[lr=0]{3-4}\cmidrule[lr=0]{5-6}\cmidrule[l=0]{7-8}
I$_1$ & \ttfamily\#04 & I$_4$ & \ttfamily\#13 & I$_7$ & \ttfamily\#22 & I$_{10}$ & \ttfamily\#31\\
\cmidrule[r=0]{1-2}\cmidrule[lr=0]{3-4}\cmidrule[lr=0]{5-6}\cmidrule[l=0]{7-8}
J$_1$ & \ttfamily\#05 & J$_4$ & \ttfamily\#14 & J$_7$ & \ttfamily\#23 & J$_{10}$ & \ttfamily\#32\\
\cmidrule[r=0]{1-2}\cmidrule[lr=0]{3-4}\cmidrule[lr=0]{5-6}\cmidrule[l=0]{7-8}
K$_1$ & \ttfamily\#06 & K$_4$ & \ttfamily\#15 & K$_7$ & \ttfamily\#24 & K$_{10}$ & \ttfamily\#33\\
\cmidrule[r=0]{1-2}\cmidrule[lr=0]{3-4}\cmidrule[lr=0]{5-6}\cmidrule[l=0]{7-8}
I$_2$ & \ttfamily\#07 & I$_5$ & \ttfamily\#16 & I$_8$ & \ttfamily\#25\\
\cmidrule[r=0]{1-2}\cmidrule[lr=0]{3-4}\cmidrule[lr=0]{5-6}\cmidrule[l=0]{7-8}
J$_2$ & \ttfamily\#08 & J$_5$ & \ttfamily\#17 & J$_8$ & \ttfamily\#26\\
\cmidrule[r=0]{1-2}\cmidrule[lr=0]{3-4}\cmidrule[lr=0]{5-6}\cmidrule[l=0]{7-8}
K$_2$ & \ttfamily\#09 & K$_5$ & \ttfamily\#18 & K$_8$ & \ttfamily\#27\\
\cmidrule[r=0]{1-2}\cmidrule[lr=0]{3-4}\cmidrule[lr=0]{5-6}\cmidrule[l=0]{7-8}
\end{twocolbreaktblr}%


\clearpage
~\vfill
%%%%%%%%%%%%%%%%%%%%%%%%%%%%%%%%%%%%%%%%%%%%%%%%%%%%%%%%%%
%% Column %%%%%%%%%%%%%%%%%%%%%%%%%%%%%%%%%%%%%%%%%%%%%%%%
%%%%%%%%%%%%%%%%%%%%%%%%%%%%%%%%%%%%%%%%%%%%%%%%%%%%%%%%%%
\begin{Column}{コード記述 注意点・ミスの例}
\begin{enumerate}
\item \ttNum\hx の付け忘れ
\item \verb|[ ]|の数の不整合
\item \verb|IF[...]THEN M00|
\item \verb|G.. GOTO...|
\item \verb|G90 G53 Z0|\\
      \verb|G91 G28 Z0|(\index{こうぐちょうほせい@工具長補正}工具長補正があるとエラー)
\item \verb|(...) O...|(最初の\index{プログラムばんごう@プログラム番号}プログラム番号の前に\index{コメント(プログラム)}コメントがあると\index{エラー}エラー)
\end{enumerate}
\tcbline*
\begin{enumerate}
\item
{\ttfamily G31}はすべての工具に適用できる。\\
ただし、\index{スキップきのう@スキップ機能}スキップ機能を有するのは\index{タッチセンサープローブ}タッチセンサープローブのみ。\\
また、タッチセンサープローブの\index{でんげん(タッチセンサープローブ)@電源(タッチセンサープローブ)}電源が入っていない状態で用いると工具は移動せず、\index{いどうかいしてん@移動開始点}移動開始点がそのまま\index{ブロックエンド}ブロックエンドとなり、\index{プログラム(G-code)}プログラムは進行する。
\item {\ttfamily G00}では\verb|F|値の記述はできるが適用されない
\item {\ttfamily M98}で呼び出したプログラムは、レベル1の\index{かいそう(プログラム)@階層(プログラム)}階層(\index{メインプログラム}メインプログラム)として実行される。
\end{enumerate}
\end{Column}
%%%%%%%%%%%%%%%%%%%%%%%%%%%%%%%%%%%%%%%%%%%%%%%%%%%%%%%%%%
%%%%%%%%%%%%%%%%%%%%%%%%%%%%%%%%%%%%%%%%%%%%%%%%%%%%%%%%%%
%%%%%%%%%%%%%%%%%%%%%%%%%%%%%%%%%%%%%%%%%%%%%%%%%%%%%%%%%%
\end{appendices}

\addtocontents{toc}{\protect\end{tocBox}}
\clearrightpage