%!TEX root = ../RPA_for_Creating_Program_Note.tex

%%%%%%%%%%%%%%%%%%%%%%%%%%%%%%%%%%%%%%%%%%%%%%
%%%%% DOCUMENTCLASS %%%%%%%%%%%%%%%%%%%%%%%%%%
%%%%%%%%%%%%%%%%%%%%%%%%%%%%%%%%%%%%%%%%%%%%%%
\documentclass[11pt, twoside, open=any]{scrbook}

\let\tmpcleardoublepage\cleardoublepage
\let\tmpclearpage\clearpage
\let\tmpnewpage\newpage
%!TEX root = ../RPA_for_Creating_Program_Note.tex

% Encoding and Language Settings
\usepackage[utf8]{inputenc} % Allows input of utf8 characters.
\usepackage{babel} % Multilingual support for LaTeX.
\usepackage[japanese]{pxbabel} % Japanese support for babel.
\usepackage[style=english]{csquotes} % Context sensitive quotation facilities.

% Font and Math Settings
%\usepackage[T1]{fontenc} % for _ in \jobname
\usepackage{amsmath, amssymb} % Enhanced math support in LaTeX.
\usepackage{eulervm} % Euler virtual math fonts.

%%%%%%%%%%%%%%%%%%%%%%%%%%%%%%%%%%%%%%%%%%%%%%
%%%%% DOTFILL %%%%%%%%%%%%%%%%%%%%%%%%%%%%%%%%
%%%%%%%%%%%%%%%%%%%%%%%%%%%%%%%%%%%%%%%%%%%%%%
%\patchcmd{\@dottedtocline}{\hbox{.}}{\hbox{$\cdot$}}{}{}
% Layout and Table Settings
\usepackage{geometry} % Provides an easy and flexible user interface to customize page layout.
\usepackage{array} % Extends array and tabular environments.
\usepackage{tabularray} % Allows tables to break across pages.
\UseTblrLibrary{booktabs, diagbox}
\usepackage{colortbl} % Adds color to LaTeX tables.

% Graphics and Color Settings
\usepackage[dvipsnames]{xcolor} % Provides driver-independent color extensions for LaTeX and pdfLaTeX.
%%% tikz, pgf %%%
\def\pgfsysdriver{pgfsys-dvipdfmx.def} % Specifies the driver for PGF, a lower-level language for creating graphics.
\usepackage{pgfplots} % A tool to create 2D and 3D plots in LaTeX.
\pgfplotsset{compat=1.18} % Specifies the version of pgfplots to use for compatibility.

% Box Settings
\usepackage[most]{tcolorbox} % Provides an environment for colored and framed text boxes with a heading line.
\usepackage{tikzpagenodes}

% Bibliography Settings
\usepackage[backend=biber, style=numeric-comp, natbib=true]{biblatex} % references
\addbibresource{./RfCPN_a0_preamble/RfCPN_reference.bib}

% List Settings
\usepackage{enumitem} % Control layout of itemize, enumerate, description.

% Footnote
\usepackage{footnotehyper} % Improve on LaTeX's footnote handling.

% Make index
\usepackage{makeidx}
\makeindex

% Hyperlink Settings
\usepackage[dvipdfmx]{hyperref} % Adds support for hyperlinks.
\usepackage{pxjahyper} % Adjusts hyperref for pLaTeX and upLaTeX.

% Header and Footer Settings
\usepackage{scrlayer-fancyhdr} % Combines the features of fancyhdr with KOMA-Script's scrlayer.

% Date and Time
\usepackage{datetime} % Date and time handling.
\usepackage{bxjaholiday} % Support for Japanese holidays.

% Page Layout
\usepackage{lastpage} % Reference last page for Page N of M type footers.
\usepackage{pdflscape} % Make landscape pages display as landscape.
\usepackage{afterpage} % Execute command after the next page break.

% Table of Contents and Headers
\usepackage{titletoc} % Alternative headings for toc/lof/lot.
\usepackage{appendix} % Extra control of appendices.

% Caption
\usepackage{caption} % Customizes captions in floating environments.
%\usepackage{subcaption} % Support for sub-captions.

% Line Spacing
\usepackage{setspace} % Set space between lines.

% Citation
%\usepackage{cite} % Improved citation handling.
%\usepackage{cleveref} % Intelligent cross-referencing.

% Conditional Processing
\usepackage{ifthen} % Conditional commands.

% Loop
\usepackage{pgffor} % Foreach loop structure.

% Arrow
\usepackage{extarrows} % Extra Arrows beyond those provided in AMSmath.

% Units
\usepackage{units} % Typeset units.

% Box Adjustment
\usepackage{adjustbox} % Graphics package-alike macros for "general" boxes.

%
\usepackage{subfiles}

% List Display
\usepackage{jvlisting} % For including code listings with Japanese comments.

% Other
\usepackage{scrhack} % Fix koma-script interaction with other packages.

%%%%% TIKZ LIBRARY etc %%%%%%%%%%%%%%%%%%%%%%%%%%%%
% Arrows
%\usetikzlibrary{arrows} % Arrow tip library.
\usetikzlibrary{arrows.meta} % Advanced arrow tip library.
% Calculations
\usetikzlibrary{calc} % Coordinate calculations.
% Decorations
%\usetikzlibrary{decorations} % General decoration library.
%\usetikzlibrary{decorations.fractals} % Fractal decorations.
%\usetikzlibrary{decorations.markings} % Arbitrary markings on paths.
%\usetikzlibrary{decorations.pathmorphing} % "Morphing" decorations.
%\usetikzlibrary{decorations.shapes} % Shape decorations.
%\usetikzlibrary{decorations.text} % Text decorations.
%\usepgfmodule{decorations} % Decoration library module.
% Matrix
%\usetikzlibrary{matrix} % Matrix library.
% Plot Marks
%\usetikzlibrary{plotmarks} % Plot mark library.
% Positioning
%\usetikzlibrary{positioning} % Improved positioning of nodes.
% Shadows
%\usetikzlibrary{shadows} % Shadow library.
% Shapes
\usetikzlibrary{shapes} % Shape library.
% Trees
%\usetikzlibrary{trees} % Tree library.
% tcolorbox Libraries
\tcbuselibrary{listings} % Enables the use of listings within tcolorboxes.
\tcbuselibrary{breakable} % Allows tcolorboxes to break across pages.
%\tcbuselibrary{skins} % Provides additional skins to customize the appearance of tcolorboxes.
%\tcbuselibrary{theorems} % Provides theorem environments within tcolorboxes.
%!TEX root = ../RPA_for_Creating_Program_Note.tex

\makeatletter

%!TEX root = ../RPA_for_Creating_Program_Note.tex


%%%%%%%%%%%%%%%%%%%%%%%%%%%%%%%%%%%%%%%%%%%%%%
%%%%% MOULDCOODINATE %%%%%%%%%%%%%%%%%%%%%%%%%
%%%%%%%%%%%%%%%%%%%%%%%%%%%%%%%%%%%%%%%%%%%%%%
\def\mouldCoordinate{%
\begin{tikzpicture}[scale=1, every node/.style={scale=0.8}]
% 値の計算
\pgfmathsetmacro{\Ax}{9.6} %A:T_iのx座標
\pgfmathsetmacro{\Ay}{3.5} %A:T_iのy座標
\pgfmathsetmacro{\Bx}{2.0+(\Ax)} %B:T_oのx座標
\pgfmathsetmacro{\Cy}{-4.2}                  %C:B_iのy座標
\pgfmathsetmacro{\Ri}{sqrt((\Ax)^2+(\Ay)^2)} %R_iの長さ
\pgfmathsetmacro{\Cx}{sqrt((\Ri)^2-(\Cy)^2)} %C:B_iのx座標
\pgfmathsetmacro{\Ro}{sqrt((\Bx)^2+(\Ay)^2)} %R_oの長さ
\pgfmathsetmacro{\Dx}{sqrt((\Ro)^2-(\Cy)^2)} %D:B_oのx座標
\pgfmathsetmacro{\Rc}{(\Ri+\Ro)/2}           %R_cの長さ
\pgfmathsetmacro{\Ex}{sqrt((\Rc)^2-(\Ay)^2)} %E:湾曲中心線トップ端のx座標
\pgfmathsetmacro{\Fx}{sqrt((\Rc)^2-(\Cy)^2)} %F:湾曲中心線ボトム端のx座標
\pgfmathsetmacro{\Ub}{2.4}                   %Ub:受板-モールド接点のy座標
\pgfmathsetmacro{\Ux}{sqrt((\Ri)^2-(\Ub)^2)} %Ux:受板-モールド接点のx座標
\pgfmathsetmacro{\Uxo}{sqrt((\Ro)^2-(\Ub)^2)} %Ux:受板-モールド接点のx座標
\pgfmathsetmacro{\Hx}{1+(\Ri)} %H:テーブルの中心
\pgfmathsetmacro{\Ix}{1.90} %I:テーブルx方向の長さの半分
% 座標系を描画
\draw[-latex, line width=0.5pt] (-0.4, 0) -- (14.5, 0) node[below] {\textbf{Re}};
\draw[-latex, line width=0.5pt] (0, -4) -- (0, 4) node[below right] {\textbf{Im}};
% 座標を定義
\coordinate (O) at (  0, 0); % 原点
\coordinate (A) at (\Ax, \Ay); %
\coordinate (B) at (\Bx, \Ay);
\coordinate (C) at (\Cx, \Cy);
\coordinate (D) at (\Dx, \Cy);
\coordinate (E) at (\Ex, \Ay);
\coordinate (F) at (\Fx, \Cy);
\coordinate (Rc) at (\Rc, 0);
\coordinate (Ri) at (\Ri, 0);
\coordinate (Ro) at (\Ro, 0);
\coordinate (Ut) at (\Ux, \Ub);
\coordinate (Ub) at (\Ux, -\Ub);
\coordinate (Uto) at (\Uxo, \Ub);
\coordinate (Ubo) at (\Uxo, -\Ub);
\coordinate (Tal) at (\Hx-\Ix, \Ub);
\coordinate (Tbl) at (\Hx-\Ix, -\Ub);
\coordinate (Tar) at (\Hx+\Ix, \Ub);
\coordinate (Tbr) at (\Hx+\Ix, -\Ub);
\coordinate (Lt) at (\Hx+\Ix+0.4, \Ub);
\coordinate (Lb) at (\Hx+\Ix+0.4, -\Ub);
\coordinate (Lc) at (\Hx+\Ix+0.4, 0);
\coordinate (Fc) at (\Hx+\Ix+1.0, 0);
\coordinate (Ft) at (\Hx+\Ix+1.0, \Ay);
\coordinate (Fb) at (\Hx+\Ix+1.0, \Cy);
% 点を描画
\fill (O) circle (2pt);
\fill (A) circle (2pt);
\fill (B) circle (2pt);
\fill (C) circle (2pt);
\fill (D) circle (2pt);
\fill (E) circle (2pt);
\fill (Rc) circle (2pt);
\fill (Ro) circle (2pt);
\fill (Ri) circle (2pt);
\fill (Ut) circle (2pt);
\fill (Ub) circle (2pt);
% 点にラベルを付ける
\node at (O) [below left] {O};
\node at (A) [above] {T$_\mathrm i$};
\node at (B) [above] {T$_\mathrm o$};
\node at (C) [below] {B$_\mathrm i$};
\node at (D) [below] {B$_\mathrm o$};
\node at (Rc) [above right] {$R_\mathrm c$};
\node at (Ro) [below right] {$R_\mathrm o$};
\node at (Ri) [below left] {$R_\mathrm i$};
\node at (Ut) [right] {U$_\mathrm T$};
\node at (Ub) [right] {U$_\mathrm B$};
% モールド外形
\draw[line width=0.75pt, fill=ffwwqq, fill opacity=0.1]
  let \p1=(A), \p2=(C), \p3=(B), \p4=(D), \n1={atan2(\y1,\x1)}, \n2={atan2(\y2,\x2)}, \n3={atan2(\y3,\x3)}, \n4={atan2(\y4,\x4)}
    in (A) -- (B) -- (\n3:\Ro) arc (\n3:\n4:\Ro) -- (C) -- (\n2:\Ri) arc (\n2:\n1:\Ri) -- cycle;
% モールド中心線
\draw[dotted, line width=0.5pt] let \p1=(E), \p2=(F), \n1={atan2(\y1,\x1)}, \n2={atan2(\y2,\x2)}
  in (\n1:\Rc) arc (\n1:\n2:\Rc);
% テーブル
\draw (Ut) -- (Tal) -- (Tbl) -- (Ub);
\draw (Uto) -- (Tar) -- (Tbr) -- (Ubo);
\draw[dotted] (Tar) -- (Lt);
\draw[dotted] (Tbr) -- (Lb);
\draw[dotted] (Tar) -- (Lt);
\draw[latex-latex, line width=0.5pt] (Lc) -- (Lt) node[midway, right] {$l$};
\draw[latex-latex, line width=0.5pt] (Lc) -- (Lb) node[midway, right] {$l$};
% 振分け
\draw[dotted] (B) -- (Ft);
\draw[dotted] (D) -- (Fb);
\draw[latex-latex, line width=0.5pt] (Fc) -- (Ft) node[midway, right] {$f_\mathrm T$};
\draw[latex-latex, line width=0.5pt] (Fc) -- (Fb) node[midway, right] {$f_\mathrm B$};
% 半径
\draw[dotted, line width=0.5pt] (O) -- (A) node[midway, above left] {$R_\mathrm i$} ;
\draw[dotted, line width=0.5pt] (O) -- (B) node[midway, below right] {$R_\mathrm o$} ;
\draw[dotted, line width=0.5pt] (O) -- (Ub) node[midway, above right] {$R_\mathrm i$} ;
% 角度
\draw[line width=0.5pt, fill=ffwwqq, fill opacity=0.1]
  let \p1=(Ri), \p2=(A), \n1={atan2(\y1,\x1)}, \n2={atan2(\y2,\x2)}
   in (\n1:2) arc (\n1:\n2:2) node[midway, right, opacity=1] {$\alpha_{\mathrm T_\mathrm i}$} -- (O);
\draw[line width=0.5pt, fill=qqzzqq, fill opacity=0.1]
  let \p1=(Ro), \p2=(B), \n1={atan2(\y1,\x1)}, \n2={atan2(\y2,\x2)}
  in (\n1:3.2) arc (\n1:\n2:3.2) node[midway, right, opacity=1] {$\alpha_{\mathrm T_\mathrm o}$} -- (O) -- cycle ;
\draw[line width=0.5pt, fill=wwqqcc, fill opacity=0.1]
  let \p1=(Ri), \p2=(Ub), \n1={atan2(\y1,\x1)}, \n2={atan2(\y2,\x2)}
  in (\n1:2.7) arc (\n1:\n2:2.7) node[midway, right, opacity=1] {$\alpha_{\mathrm U_\mathrm B}$} -- (O);
\end{tikzpicture}%
}

%%%%%%%%%%%%%%%%%%%%%%%%%%%%%%%%%%%%%%%%%%%%%%
%%%%% MOULDWITHUKEITA %%%%%%%%%%%%%%%%%%%%%%%%
%%%%%%%%%%%%%%%%%%%%%%%%%%%%%%%%%%%%%%%%%%%%%%
\def\mouldwithukeita{%
\begin{tikzpicture}[scale=0.5,
                    every node/.style={scale=0.4},
                    >={Latex[length=1mm, width=0.75mm]},]
% 値の計算
\pgfmathsetmacro{\Ax}{12}                    %A:T_iのx座標
\pgfmathsetmacro{\Ay}{3.5}                   %A:T_iのy座標
\pgfmathsetmacro{\Bx}{2.0+(\Ax)}             %B:T_oのx座標
\pgfmathsetmacro{\Cy}{-4.2}                  %C:B_iのy座標
\pgfmathsetmacro{\TUb}{2.4}                  %TUb:テーブル-モールドの交点y座標
\pgfmathsetmacro{\Ix}{2.7}                   %I:テーブルx方向の長さの半分
\pgfmathsetmacro{\Uw}{0.75}                  %Uw:受板の幅
\pgfmathsetmacro{\Ul}{0.45}                  %Ul:受板の長さ
\pgfmathsetmacro{\Ur}{1.35}                   %Uw:受板の半径
\pgfmathsetmacro{\Ri}{sqrt((\Ax)^2+(\Ay)^2)} %R_iの長さ
\pgfmathsetmacro{\Hx}{0.1+(\Ri)}             %H:テーブルの中心
% 値の計算
\pgfmathsetmacro{\Cx}{sqrt((\Ri)^2-(\Cy)^2)}    %C:B_iのx座標
\pgfmathsetmacro{\Ro}{sqrt((\Bx)^2+(\Ay)^2)}    %R_oの長さ
\pgfmathsetmacro{\Dx}{sqrt((\Ro)^2-(\Cy)^2)}    %D:B_oのx座標
\pgfmathsetmacro{\Rc}{(\Ri+\Ro)/2}              %R_cの長さ
\pgfmathsetmacro{\Ex}{sqrt((\Rc)^2-(\Ay)^2)}    %E:湾曲中心線トップ端のx座標
\pgfmathsetmacro{\Fx}{sqrt((\Rc)^2-(\Cy)^2)}    %F:湾曲中心線ボトム端のx座標
\pgfmathsetmacro{\TUx}{sqrt((\Ri)^2-(\TUb)^2)}  %TUx:テーブル-モールド内側との交点のx座標
\pgfmathsetmacro{\TUxo}{sqrt((\Ro)^2-(\TUb)^2)} %TUxo:テーブル-モールド外側との交点x座標
\pgfmathsetmacro{\Ub}{\Ri*sin(asin((\TUb-\Uw/2)/(\Ri-\Ur)))} %Ub:ボトム側受板-モールド接点のy座標
\pgfmathsetmacro{\Ux}{sqrt((\Ri)^2-(\Ub)^2)}    %Ux:トップ側受板-モールド接点のx座標
\pgfmathsetmacro{\Ucx}{\Ux-sqrt((\Ur)^2-((\Uw)/2-(\TUb-\Ub))^2)} %Uw:受板の半径
\pgfmathsetmacro{\TBUx}{\Ucx+sqrt((\Ur)^2-((\Uw)/2)^2)} %TBUx:テーブルと受板の交点x座標
% 座標の定義
\coordinate (A) at (\Ax, \Ay); %
\coordinate (B) at (\Bx, \Ay);
\coordinate (C) at (\Cx, \Cy);
\coordinate (D) at (\Dx, \Cy);
\coordinate (E) at (\Ex, \Ay);
\coordinate (F) at (\Fx, \Cy);
\coordinate (Ut) at (\Ux, \Ub);
\coordinate (TUt) at (\TUx, \TUb);
\coordinate (Ub) at (\Ux, -\Ub);
\coordinate (TUb) at (\TUx, -\TUb);
\coordinate (TUto) at (\TUxo, \TUb);
\coordinate (TUbo) at (\TUxo, -\TUb);
\coordinate (Tal) at (\Hx-\Ix, \TUb);
\coordinate (Tbl) at (\Hx-\Ix, -\TUb);
\coordinate (Tar) at (\Hx+\Ix, \TUb);
\coordinate (Tbr) at (\Hx+\Ix, -\TUb);
\coordinate (Lt) at (\Hx+\Ix+0.4, \TUb);
\coordinate (Lb) at (\Hx+\Ix+0.4, -\TUb);
\coordinate (Lc) at (\Hx+\Ix+0.4, 0);
\coordinate (Fc) at (\Hx+\Ix+1.0, 0);
\coordinate (TUc) at (\Ucx, \TUb-\Uw/2);
\coordinate (BUc) at (\Ucx, \Uw/2-\TUb);
\coordinate (TTU) at (\TBUx, \TUb);
\coordinate (TTUex) at (\TBUx-\Ul, \TUb);
\coordinate (TUex) at (\TBUx-\Ul, \TUb-\Uw);
\coordinate (TU) at (\TBUx, \TUb-\Uw);
\coordinate (TBU) at (\TBUx, -\TUb);
\coordinate (TBUex) at (\TBUx-\Ul, -\TUb);
\coordinate (BUex) at (\TBUx-\Ul, -\TUb+\Uw);
\coordinate (BU) at (\TBUx, -\TUb+\Uw);
% モールド外形
\draw[line width=0.5pt, fill=ffwwqq, fill opacity=0.1]
  let \p1=(A), \p2=(C), \p3=(B), \p4=(D), \n1={atan2(\y1,\x1)}, \n2={atan2(\y2,\x2)}, \n3={atan2(\y3,\x3)}, \n4={atan2(\y4,\x4)}
    in (A) -- (B) -- (\n3:\Ro) arc (\n3:\n4:\Ro) -- (C) -- (\n2:\Ri) arc (\n2:\n1:\Ri) -- cycle;
% モールド中心線
\draw[dotted, line width=0.5pt] let \p1=(E), \p2=(F), \n1={atan2(\y1,\x1)}, \n2={atan2(\y2,\x2)}
  in (\n1:\Rc) arc (\n1:\n2:\Rc);
% テーブル
\draw (TUt) -- (Tal) -- (Tbl) -- (TUb);
\draw (TUto) -- (Tar) -- (Tbr) -- (TUbo);
% 受板
\draw[fill=gray!15!] let \p1=(TUc), \p2=(TU), \p3=(TTU), \n1={atan2(\y2-\y1,\x2-\x1)}, \n2={atan2(\y3-\y1,\x3-\x1)},
    in (TTU) -- (TTUex) -- (TUex) -- (TU) arc[start angle=\n1, end angle=\n2, radius=\Ur] -- cycle;
\draw[fill=gray!15!] let \p1=(BUc), \p2=(BU), \p3=(TBU), \n1={atan2(\y2-\y1,\x2-\x1)}, \n2={atan2(\y3-\y1,\x3-\x1)},
    in (TBU) -- (TBUex) -- (BUex) -- (BU) arc[start angle=\n1, end angle=\n2, radius=\Ur] -- cycle;
\draw[<->] (BUc) -- (Ub) node[midway, above] {$\rho$};
\draw[<->] ([xshift=-1.25mm]TTUex) -- ([xshift=-1.25mm]TUex) node[midway, left] {$\sigma$};
% 点を描画
\fill (Ut) circle (1pt);
\fill (Ub) circle (1pt);
\fill (TUc) circle (1pt);
\fill (BUc) circle (1pt);
% 点にラベルを付ける
\node at (Ut) [right] {$R_\mathrm ie^{i\alpha_{\mathrm U_\mathrm T}}$};
\node at (Ub) [right] {$R_\mathrm ie^{i\alpha_{\mathrm U_\mathrm B}}$};
\node at (TUc) [above] {U$_\mathrm T$};
\node at (BUc) [below] {U$_\mathrm B$};
\end{tikzpicture}%
}


%%%%%%%%%%%%%%%%%%%%%%%%%%%%%%%%%%%%%%%%%%%%%%
%%%%% DATE %%%%%%%%%%%%%%%%%%%%%%%%%%%%%%%%%%%
%%%%%%%%%%%%%%%%%%%%%%%%%%%%%%%%%%%%%%%%%%%%%%
\newcommand{\customtoday}{\the\year/\two@digits{\the\month}/\two@digits{\the\day}}
\newcommand{\customdate}{\customtoday\ \currenttime\ (\jadayofweek{\the\year}{\the\month}{\the\day})}
\newcommand{\customtodayap}{\ifnum\currenthour<12 \customtoday\,a.m.\else\customtoday\,p.m.\fi}
%%%%%%%%%%%%%%%%%%%%%%%%%%%%%%%%%%%%%%%%%%%%%%
%%%%% NEWIF %%%%%%%%%%%%%%%%%%%%%%%%%%%%%%%%%%
%%%%%%%%%%%%%%%%%%%%%%%%%%%%%%%%%%%%%%%%%%%%%%
\newif\if@backmatter%\@backmattertrue
\newif\if@frontmatter%\@frontmattertrue
\newif\if@appendix%\@appendixfalse
%%%%%%%%%%%%%%%%%%%%%%%%%%%%%%%%%%%%%%%%%%%%%%
%%%%% CLEARLEFTPAGE %%%%%%%%%%%%%%%%%%%%%%%%%%
%%%%%%%%%%%%%%%%%%%%%%%%%%%%%%%%%%%%%%%%%%%%%%
\newcommand{\thispageNum}{\the\value{page}}
\newcommand{\clearleftpage}{%
  \if@twoside%
    \ifodd\thispageNum%
      \clearpage%
    \else
      \cleardoublepage%
    \fi
  \else%
    \clearpage%
  \fi
}
\newcommand{\clearrightpage}{%
  \if@twoside%
    \ifodd\thispageNum%
      \cleardoublepage%
    \else%
      \clearpage%
    \fi
  \else%
    \clearpage%
  \fi
}
%%%%%%%%%%%%%%%%%%%%%%%%%%%%%%%%%%%%%%%%%%%%%%
%%%%% DIMENSION %%%%%%%%%%%%%%%%%%%%%%%%%%%%%%
%%%%%%%%%%%%%%%%%%%%%%%%%%%%%%%%%%%%%%%%%%%%%%
\ifluatex
  \usepackage{luatexja}
  \ltjsetparameter{kanjiskip=0.0pt plus 0.4pt minus 0.5pt}
  \ltjsetparameter{xkanjiskip=2.40555pt plus 1.0pt minus 1.0pt}
  \newcommand{\hk}{\hspace{\ltjgetparameter{kanjiskip}}}
  \newcommand{\hx}{\hspace{\ltjgetparameter{xkanjiskip}}}
\else
  \setlength{\kanjiskip}{0.0pt plus 0.4pt minus 0.5pt}
  \setlength{\xkanjiskip}{2.40555pt plus 1.0pt minus 1.0pt}
  \newcommand{\hk}{\hspace{\kanjiskip}}
  \newcommand{\hx}{\hspace{\xkanjiskip}}
\fi
%%%%%%%%%%%%%%%%%%%%%%%%%%%%%%%%%%%%%%%%%%%%%%
%%%%% CDOTFILL LIKE TOC %%%%%%%%%%%%%%%%%%%%%%
%%%%%%%%%%%%%%%%%%%%%%%%%%%%%%%%%%%%%%%%%%%%%%
%\newcommand\cdotfill{%
%  \leaders\hbox{$\m@th\mkern\@dotsep mu\hbox{$\cdot$}\mkern \@dotsep mu$}\hfill\kern\z@
%}
%%%%%%%%%%%%%%%%%%%%%%%%%%%%%%%%%%%%%%%%%%%%%%
%%%%% COLOR %%%%%%%%%%%%%%%%%%%%%%%%%%%%%%%%%%
%%%%%%%%%%%%%%%%%%%%%%%%%%%%%%%%%%%%%%%%%%%%%%
\definecolor{ai}     {rgb}{0.2039, 0.3765, 0.4314}
\definecolor{kon}    {rgb}{0.0000, 0.2000, 0.4000}
\definecolor{konpeki}{rgb}{0.0902, 0.5098, 0.7333}
\definecolor{moegi}  {rgb}{0.3020, 0.5961, 0.1882}
\definecolor{sssec}  {rgb}{0.7333, 0.5, 0.7333}
\definecolor{sora}   {rgb}{0.1451, 0.7216, 0.8039}
\definecolor{sumire} {rgb}{0.3882, 0.2157, 0.5922}
\definecolor{wwqqcc} {rgb}{0.4, 0, 0.8}
\definecolor{qqzzqq} {rgb}{0, 0.6, 0}
\definecolor{ffwwqq} {rgb}{1, 0.4, 0}
%%%%%%%%%%%%%%%%%%%%%%%%%%%%%%%%%%%%%%%%%%%%%%
%%%%% REF %%%%%%%%%%%%%%%%%%%%%%%%%%%%%%%%%%%%
%%%%%%%%%%%%%%%%%%%%%%%%%%%%%%%%%%%%%%%%%%%%%%
%%%%% AUTOREFNAME %%%%%%%%%%%%%%%%%%%%%%%%%%%%
%\newcommand{\subfigureautorefname}{\figureautorefname} % subfigure --> figure
%\newcommand{\subtableautorefname}{\tableautorefname}
%%%%% PAGEREF %%%%%%%%%%%%%%%%%%%%%%%%%%%%%%%%%%%
\newcommand{\pageautoref}[1]{%
  \ifthenelse{\equal{\pageref{#1}}{\thepage}}%
    {\autoref{#1}}%
    {\autoref{#1}~[p.\pageref{#1}]}%
}
\newcommand{\pageeqref}[1]{%
  \ifthenelse{\equal{\pageref{#1}}{\thepage}}%
    {\eqref{#1}}%
    {\eqref{#1}~[p.\pageref{#1}]}%
}
%%%%%%%%%%%%%%%%%%%%%%%%%%%%%%%%%%%%%%%%%%%%%%
%%%%% FOR LISTINGS %%%%%%%%%%%%%%%%%%%%%%%%%%%
%%%%%%%%%%%%%%%%%%%%%%%%%%%%%%%%%%%%%%%%%%%%%%
%%%%% CAPTOINOF %%%%%%%%%%%%%%%%%%%%%%%%%%%%%%
\setlength{\abovecaptionskip}{0pt}
\newcommand{\modcaptionof}[2]{%
  \captionof{#1}{%
    \csname #1name\endcsname\thechapter.%
    \ifnum\value{#1}<10 0\fi
      \arabic{#1}. #2}}
%%%%%%%%%%%%%%%%%%%%%%%%%%%%%%%%%%%%%%%%%%%%%%
%%%%% FOR TOC %%%%%%%%%%%%%%%%%%%%%%%%%%%%%%%%
%%%%%%%%%%%%%%%%%%%%%%%%%%%%%%%%%%%%%%%%%%%%%%
%%%%% TOC LINE %%%%%%%%%%%%%%%%%%%%%%%%%%%%%%%
\newcommand\PartSeparateline[1]{\addtocontents{#1}{\protect\par\protect\hrulefill\protect\par\protect\vspace*{-10pt}}}%
\newcommand\tocAPartSeparateline[3]{\addtocontents{#1}{\protect\par\protect\vspace*{#2}\protect\hrule width 0.5\linewidth\protect\par\protect\vspace*{#3}}}%
%\newcommand\tableAPartSeparateline{\tocAPartSeparateline{lot}{2pt}{2pt}}
%%%%% PART FOR APPENDIX %%%%%%%%%%%%%%%%%%%%%%%%%%%%%%%%%%%%%%%%%
\newcommand{\Appendixpart}{
  \clearrightpage
  \part*{\partname\ \thepart\hx の補遺\label{Apart:\thepart}}
  \addcontentsline{toc}{part}{\partname\ \thepart\hx の補遺}
}
%%%%%%%%%%%%%%%%%%%%%%%%%%%%%%%%%%%%%%%%%%%%%%
%%%%% AUTO LABELING %%%%%%%%%%%%%%%%%%%%%%%%%%
%%%%%%%%%%%%%%%%%%%%%%%%%%%%%%%%%%%%%%%%%%%%%%
%%%%% FOR CHAPTER OR APPENDIX %%%%%%%%%%%%%%%%%%%%%%%%%%%%%
\newcommand{\modHeadchapter}[2][]{%
  \clearrightpage\clearleftpage
  \let\newpage\relax
  \ifx\relax#1\relax
    \ifx\@chapapp\appendixname
      \chapter{#2\label{app:\thechapter}}%
    \else
      \chapter{#2\label{chap:\thechapter}}%
    \fi
  \else%
    \ifx\@chapapp\appendixname
      \chapter[#1]{#2\label{app:\thechapter}}%
    \else
      \chapter[#1]{#2\label{chap:\thechapter}}%
    \fi
  \fi
  \thispagestyle{main}%
  \indentspace%
  \let\newpage\tmpnewpage
}
%%%%% FOR SECTION %%%%%%%%%%%%%%%%%%%%%%%%%%%%%
%%%%% TO LABEL SECTION  %%%%%%
\newcommand{\modHeadsection}[2][]{%
  \ifx\relax#1\relax
    \section{#2\label{sec:\thesection}}%
  \else%
    \section[#1]{#2\label{sec:\thesection}}%
  \fi
}
%%%%%%%%%%%%%%%%%%%%%%%%%%%%%%%%%%%%%%%%%%%%%%
%%%%% NEWCOLORBOX %%%%%%%%%%%%%%%%%%%%%%%%%%%%
%%%%%%%%%%%%%%%%%%%%%%%%%%%%%%%%%%%%%%%%%%%%%%
\newcounter{GlobalFootnote}% difine a counter Global Footnote
%%%%% PART %%%%%
\newcommand{\tablePartnname}{\thepart}
\newtcolorbox[auto counter, number within=part]{tablePart}[2][]{Columnbox, title={\termblue{\tablePartnname:#2}}, #1, after title={}}
%%%%% COLUMN %%%%%
\newcommand{\Columnname}{Column}
\newtcolorbox[auto counter, number within=chapter, list inside=loC]{Column}[2][]{%
  Columnbox, title={#2}, #1, list entry={\numberline{\Columnname~\thetcbcounter}{\protect\hspace*{41pt}#2}}}
%\newcommand{\tcb@cnt@Columnautorefname}{Column}
%%%%% FIGBOX %%%%%
\newtcolorbox{Figbox}[1][]{Figurebox, #1}
%%%%% TABBOX %%%%%
\newtcolorbox{Tabbox}[1][]{Tabularbox, #1}
%\renewcommand{\tableautorefname}{表}
%%%%% HOSOKU %%%%%
\newcommand{\hosokuname}{補}
\definecolor{hosoku}{cmyk}{0, 0, 0, .15}
\newtcolorbox[auto counter, number within=chapter]{hosoku}[1][]{hosokubox, #1}
\newcommand{\tcb@cnt@hosokuautorefname}{補足}
%%%%% TWOCTABLE %%%%%
\newtcolorbox[number within=chapter]{twoCtable}[2][]{twoCtablebox, title={#2}, #1}
%%%%% OTHER TIKZ DEFINITION %%%%%%%%%%%%%%%%%%
\def\pgfname{\textsc{pgf}}
\def\tikzname{Ti\textit{k}Z}
\tikzfading[name=fade ball, inner color=transparent!60, outer color=transparent!30]
\def\sball#1{\tikz \shade [ball color=#1, path fading=fade ball] (0,0) circle (.7ex);}
\def\terminal#1#2{\tikz[baseline=(a.base)] \node (a) [terminal, bottom color=#2] {\small #1};}
\def\termblue#1{\terminal{\color{blue}\fontsize{8pt}{0pt}\textbf{#1}}{gray!25}\hskip6pt}
%%%%%%%%%%%%%%%%%%%%%%%%%%%%%%%%%%%%%%%%%%%%%%
%%%%% DECLARE %%%%%%%%%%%%%%%%%%%%%%%%%%%%%%%%
%%%%%%%%%%%%%%%%%%%%%%%%%%%%%%%%%%%%%%%%%%%%%%
%%%%% DECLAREMATHOPERATOR %%%%%%%%%%%%%%%%%%%%
\DeclareMathOperator{\IP}{Im}
\DeclareMathOperator{\RP}{Re}
%%%%% DECLAREROBUSTCOMMAND %%%%%%%%%%%%%%%%%%%%
\DeclareRobustCommand{\bDiv}{\nonscript\mskip-\medmuskip\mkern5mu\mathbin
  {\operator@font div}\penalty900
  \mkern5mu\nonscript\mskip-\medmuskip}
\DeclareRobustCommand{\pod}[1]{\allowbreak
  \if@display\mkern18mu\else\mkern8mu\fi(#1)}
\DeclareRobustCommand{\pDiv}[1]{\pod{{\operator@font div}\mkern6mu#1}}
\DeclareRobustCommand{\Div}[1]{\allowbreak\if@display\mkern18mu
  \else\mkern12mu\fi{\operator@font div}\,\,#1}
%%%%% DECLARENEWTOC %%%%%%%%%%%%%%%%%%%%
\DeclareNewTOC[owner=\jobname, name=Part]{lop}
\DeclareNewTOC[owner=\jobname, name=Column]{loC}
%%%%% DECLARENEWLAYER %%%%%%%%%%%%%%%%%%%%
\newcommand{\setallpageWatermark}[4]{%
  \DeclareNewLayer[
    foreground,
    page,
    contents={%
      \begin{tikzpicture}[remember picture, overlay]
        \node[rotate=#2, scale=#3, anchor=center, opacity=#4, text=lightgray] at (current page.center){\scshape#1};%
      \end{tikzpicture}%
    }%
  ]{WatermarkLayer}
}
%%%%%%%%%%%%%%%%%%%%%%%%%%%%%%%%%%%%%%%%%%%%%%
%%%%% LINK %%%%%%%%%%%%%%%%%%%%%%%%%%%%%%%%%%%
%%%%%%%%%%%%%%%%%%%%%%%%%%%%%%%%%%%%%%%%%%%%%%
%%%%% LINK NAME %%%%%%%%%%%%%%%%%%
\newcommand{\linkLaTeX}{\href{https://www.latex-project.org/}{\LaTeX}}
\newcommand{\linkLaTeXProject}{\href{https://www.latex-project.org/}{\LaTeX\ Project}}
\newcommand{\linkAMS}{\href{https://www.latex-project.org/}{American Mathematical Society}}
\newcommand{\linkTeXLive}{\href{https://tug.org/texlive/}{\TeX\ Live}}
\newcommand{\linkTeXUsersGroup}{\href{http://www.tug.org/}{\TeX\ Users Group}}
\newcommand{\linkBibLaTeX}{\href{https://ctan.org/pkg/biblatex}{Bib\LaTeX}}
\newcommand{\linkBiber}{\href{https://ctan.org/pkg/biber}{Biber}}
\newcommand{\linkupmendex}{\href{https://ctan.org/pkg/upmendex}{upmendex}}
\newcommand{\linkPGFTikZ}{\href{https://github.com/pgf-tikz/pgf}{\pgfname/\tikzname}}
\newcommand{\linkTeXStudio}{\href{https://texstudio.org/}{\TeX\ Studio}}
\newcommand{\linkSumatraPDF}{\href{https://www.sumatrapdfreader.org/}{Sumatra PDF}}
\newcommand{\linkVSCode}{\href{https://code.visualstudio.com/}{VS Code}}
\newcommand{\linkExcel}{\href{https://www.microsoft.com/ja-jp/microsoft-365/excel}{Excel}}
\newcommand{\linkMicrosoftCopilot}{\href{https://www.bing.com/}{Microsoft Copilot}}
\newcommand{\linkMicrosoftCorp}{\href{https://www.microsoft.com/}{Microsoft Corporation}}
\newcommand{\linkPython}{\href{https://www.python.org/}{Python}}
\newcommand{\linkPythonSF}{\href{https://www.python.org/psf-landing/}{Python Software Foundation}}
\newcommand{\linkGitHub}{\href{https://github.com/}{GitHub}}
\newcommand{\linkGitHubInc}{\href{https://github.com/}{GitHub, Inc}}
\newcommand{\linkDocker}{\href{https://www.docker.com/}{Docker}}
\newcommand{\linkDockerInc}{\href{https://www.docker.com/}{Docker, Inc}}
\newcommand{\linkUbuntu}{\href{https://ubuntu.com/}{Ubuntu}}
\newcommand{\linkCanonicalLtd}{\href{https://canonical.com/}{Canonical Ltd}}
\newcommand{\linkSQLite}{\href{https://www.sqlite.org/}{SQLite}}
\newcommand{\linkSQLiteConsortium}{\href{https://www.sqlite.org/consortium.html}{SQLite Consortium}}
\newcommand{\linkChatGPT}{\href{https://openai.com/chatgpt}{ChatGPT}}
\newcommand{\linkOpenAI}{\href{https://www.openai.com/}{OpenAI}}
%\newcommand\nextsectionlink[1]{\addtocounter{section}\@ne
%                               \hyperlink{section.\thechapter.\the\c@section}{#1}%
%                               \addtocounter{section}{-\@ne}}
%\newcommand\previoussectionlink[1]{\addtocounter{section}{-\@ne}
%                                   \hyperlink{section.\thechapter.\the\c@section}{#1}%
%                                   \addtocounter{section}{\@ne}}
%\newcommand\previouschapterlink[1]{\addtocounter{chapter}{-\@ne}
%                                   \hyperlink{chapter.\the\c@chapter}{#1}%
%                                   \addtocounter{chapter}{\@ne}}
%%%%%%%%%%%%%%%%%%%%%%%%%%%%%%%%%%%%%%%%%%%%%%
%%%%% OTHER DEFINITION %%%%%%%%%%%%%%%%%%%%%%%
%%%%%%%%%%%%%%%%%%%%%%%%%%%%%%%%%%%%%%%%%%%%%%
%%%%% TO GET CHAPTER TITLE %%%%%%
\newcommand\Chaptername{} % initialize \Chaptername
\let\old@chapter\@chapter
\def\@chapter[#1]#2{\gdef\Chaptername{#2}\old@chapter[#1]{#2}}
%%%%% TO GET SECTION TITLE %%%%%%
\newcommand\Sectionname{} % initialize \Sectionname
\let\Sectionmark\sectionmark
\def\sectionmark#1{\def\Sectionname{#1}\Sectionmark{#1}}
%%%%% TO PRG NAME %%%%%%
\newcommand\MainEx{O1916} % an example for main program
\newcommand\MXOThickness{O110001} % X外側中心計測
\newcommand\MYOThickness{O110002} % Y外側中心計測
\newcommand\MXIWidth{O130001} % X内側中心計測
\newcommand\MYIWidth{O130002} % Y内側中心計測
\newcommand\MXface{O140001} % 外削X基準面計測
\newcommand\MYcenterline{O150002} % 通り芯Y
\newcommand\MXcenterline{O150003} % 通り芯X(Z測定)
\newcommand\DLone{O210003} % 内面溝用 レベル1
\newcommand\DLtwoAC{O220001} % 内面溝用 レベル2 AC
\newcommand\DLtwoBD{O220002} % 内面溝用 レベル2 BD
\newcommand\DMLthreeAC{O230001} % 内面溝 測定用 レベル3 AC
\newcommand\DMLthreeBD{O230002} % 内面溝 測定用 レベル3 BD
\newcommand\KTanmenRight{O410000} % 端面用 右回り
\newcommand\KGaisakuRLeft{O420000} % 外側 左回り
\newcommand\KMizoConerLeft{O430000} % 溝用 左回り
\newcommand\KSotoMentoriRLeft{O440000} % 外側面取用 左回り
\newcommand\KUchiMentoriRLeft{O450000} % 内側面取用 左回り
\newcommand\KOLeft{O490005} % 外 左回り
\newcommand\DKLthreeAC{O530001} % 内面溝 加工用 レベル3 AC
\newcommand\DKLthreeBD{O530002} % 内面溝 加工用 レベル3 BD
\newcommand\OsensorOn{O910001} % タッチセンサーON
\newcommand\OsensorOff{O910002} % タッチセンサーOFF
%%%%% MACHINING NAME %%%%%%%%%%%%%%%%%%
\newcommand{\DMname}{Dマシニング}
\newcommand{\MMname}{Mマシニング}
%%%%% OTHER DEFINITION %%%%%%%%%%%%%%%%%%
\newcommand\ttNum{\ifmmode{\text{\texttt\#}}\else\texttt\#\fi}
\newcommand\cf{{\itshape cf.\,}}
\newcommand{\TBW}{\texorpdfstring{\small{\color{red}\,*}}{}}
%%%%% FOR PART WITH TABLE %%%%%%%%%%%%%%%%%%
\newcommand{\tPart}[4][]{
  \begingroup
  \clearrightpage
%  \@openrightfalse
  \let\oldnewpage\newpage
  \let\newpage\relax
  \part{#2\label{part:\thepart}}
  \addcontentsline{lop}{part}{\protect\numberline{\thepart}#2}%
  \ifx#1\relax\else\addcontentsline{#1}{part}{\protect\numberline{\thepart}#2}\fi
  \let\newpage\oldnewpage
  \clearpage
%  \@openrightfalse
  \thispagestyle{emptydate}
  \vspace*{0.1\textheight}%
  \begin{tablePart}{#3}
  #4
  \end{tablePart}%
%  \@openrightfalse
  \@mainmattertrue\pagestyle{main}
  \endgroup
}
%%%%% NOTATION TABLE %%%%%%%%%%%%%%%%%%
\newenvironment{Notation}[2]
{%
  \rowcolors{3}{gray!10}{white}
  \setlength\cellspacetoplimit{4pt}
  \setlength\cellspacebottomlimit{4pt}
  \if\relax\detokenize{#1}\relax
  \else
    \captionsetup{justification=centering}
    \setlength{\abovecaptionskip}{-7pt}
    \modcaptionof{table}{#1} % 追加したキャプション
    \addtocounter{table}{-1}
  \fi
  \begin{longtable}{|c|Sl|c|}
  \hline
  \rowcolor{orange!20}
  \textbf{記号} & \textbf{内容}\hspace*{0.75\textwidth} & \textbf{#2}\\
  \hline
  \endfirsthead
  \hline
  \rowcolor{orange!20}
  \textbf{記号} & \textbf{内容} & \textbf{#2}\\
  \hline
  \endhead
  \hline
  \multicolumn{3}{|r|}{\scriptsize 次ページへ続く} \\
  \hline
  \endfoot
  \hline
  \endlastfoot
}
{%
  \end{longtable}
}

%%%%%%%%%%%%%%%%%%%%%%%%%%%%%%%%%%%%%%%%%%%%%%
%%%%% GEOMETRY %%%%%%%%%%%%%%%%%%%%%%%%%%%%%%%
%%%%%%%%%%%%%%%%%%%%%%%%%%%%%%%%%%%%%%%%%%%%%%
\geometry{
  a4paper, % paper size
  centering,
  textwidth={6.5in},
  includehead,  % include the head of the page
%  headheight = 13.6pt,
  includefoot,  % include the foot of the page
  top=15.0truemm,
  bottom=-0.5truemm,
}
%%%%%%%%%%%%%%%%%%%%%%%%%%%%%%%%%%%%%%%%%%%%%%
%%%%% HEYPERSETUP %%%%%%%%%%%%%%%%%%%%%%%%%%%%
%%%%%%%%%%%%%%%%%%%%%%%%%%%%%%%%%%%%%%%%%%%%%%
\hypersetup{
%  pdfcreationdate=date,
%  pdfcreator={upLaTeX with hyperref}, % creator for PDF subjct field
  pdftitle={モールド関連 -主に幾何学的性質-}, % title for PDF subjct field
  pdfsubject={Mould-Related - Mainly Geometric Properties}, % text for PDF subjct field
  pdfauthor={Kurahashi Nobuaki},  % text for PDF Author field
  pdfkeywords={mould, mold}, % keywords
%  pdfproducer=producer, % dvipdfmx
  linktoc=all,
%  linktocpage=false,   % (if it is true) make page number, not text, be link on TOC, LOF and LOT
  pdfcenterwindow=false, % position the document window center of the screen
  pdffitwindow=true,     % resize document window to fit document size
  bookmarksnumbered=true,
  bookmarksopen=true, %bookmarks open
  pdfstartview={FitH}, % Fit, FitV, FitH, FitB
  pdfpagemode=UseThumbs, % set default mode of PDF display
  unicode=true,
  pdfencoding=unicode,   % PDFDocEncoding or Unicode
  colorlinks=true,     % color links
  linkcolor=ai,      % color of links
  urlcolor=ai,         % color of urls
  citecolor=sora,      % color of citation links
}
%%%%%%%%%%%%%%%%%%%%%%%%%%%%%%%%%%%%%%%%%%%%%%
%%%%% DISPLAYBREAK %%%%%%%%%%%%%%%%%%%%%%%%%%%
%%%%%%%%%%%%%%%%%%%%%%%%%%%%%%%%%%%%%%%%%%%%%%
\allowdisplaybreaks
%%%%% TO NO PAGE BREAK IN TOC %%%
%\pretocmd{\modHeadchapter}{\addtocontents{lot}{\protect\nopagebreak}}{}{}
%\pretocmd{\modHeadsection}{\ifnum\value{section}=0\addtocontents{lot}{\protect\nopagebreak}\fi}{}{}
%\pretocmd{\subsection}{\ifnum\value{subsection}=0\addtocontents{toc}{\protect\nopagebreak[4]}\fi}{}{}
%\pretocmd{\subsubsection}{\ifnum\value{subsubsection}=0\addtocontents{toc}{\protect\nopagebreak}\fi}{}{}
%%%%%%%%%%%%%%%%%%%%%%%%%%%%%%%%%%%%%%%%%%%%%%
%%%%% UNIT LENGTH %%%%%%%%%%%%%%%%%%%%%%%%%%%%
%%%%%%%%%%%%%%%%%%%%%%%%%%%%%%%%%%%%%%%%%%%%%%
\setlength{\unitlength}{1pt}
%%%%%%%%%%%%%%%%%%%%%%%%%%%%%%%%%%%%%%%%%%%%%%
%%%%% LINESPREAD %%%%%%%%%%%%%%%%%%%%%%%%%%%%%
%%%%%%%%%%%%%%%%%%%%%%%%%%%%%%%%%%%%%%%%%%%%%%
\linespread{1.15}\selectfont
%%%%%%%%%%%%%%%%%%%%%%%%%%%%%%%%%%%%%%%%%%%%%%
%%%%% PARINDENT %%%%%%%%%%%%%%%%%%%%%%%%%%%%%%
%%%%%%%%%%%%%%%%%%%%%%%%%%%%%%%%%%%%%%%%%%%%%%
\newcommand{\indentspace}{\setlength\parindent{11pt}}
\indentspace
%\def \globalscale {0.83}
%%%%%%%%%%%%%%%%%%%%%%%%%%%%%%%%%%%%%%%%%%%%%%
%%%%% FOOTNOTE %%%%%%%%%%%%%%%%%%%%%%%%%%%%%%%
%%%%%%%%%%%%%%%%%%%%%%%%%%%%%%%%%%%%%%%%%%%%%%
\renewcommand*{\footnoteautorefname}{脚注}
\interfootnotelinepenalty=10000
\counterwithout{footnote}{chapter}
\def\@makefnmark{\hbox{}\hbox{\@textsuperscript{\normalfont\@thefnmark}}\hbox{}}
\deffootnote[1em]{1em}{1em}{\textsuperscript{\thefootnotemark}}
\renewcommand\footnoterule{%
  \kern3pt
  \hrule\@width.75\columnwidth
  \kern2.6pt
}
\makesavenoteenv{twoCtable}
\makesavenoteenv{longtable}
\makesavenoteenv{tablePart}
\makesavenoteenv{Column}
%%%%%%%%%%%%%%%%%%%%%%%%%%%%%%%%%%%%%%%%%%%%%%
%%%%% HEADER AND FOOTER %%%%%%%%%%%%%%%%%%%%%%
%%%%%%%%%%%%%%%%%%%%%%%%%%%%%%%%%%%%%%%%%%%%%%
%\pagestyle{fancy}
\renewcommand{\headrulewidth}{1.5pt}
\renewcommand{\footrulewidth}{0pt}
\newcommand{\commonheadfoot}{
  \fancyhead{}
  \fancyfoot{}
  \fancyfoot[LO]{\tiny\customdate} % footer left fields for main odd pages
  \fancyfoot[RE]{\tiny\customdate} % footer right fields for main even pages
}
\fancypagestyle{emptydate}{\renewcommand{\headrulewidth}{0pt}\commonheadfoot}
\fancypagestyle{front}{
  \commonheadfoot
  \fancyhead[RO]{\thepage}
  \fancyhead[LE]{\thepage}
}
\fancypagestyle{main}{
  \commonheadfoot
  \fancyhead[RO]{\hyperref[sec:\thesection]{{\small\nouppercase\rightmark}}}
  \fancyhead[LO]{$\nicefrac{\thepage\,}{\pageref{LastPage}}$~~{\hyperref[part:\thepart]{\small\partname.\,\thepart}}} % header right fields for all main odd pages
  \fancyhead[RE]{{\hyperref[part:\thepart]{\small\partname.\,\thepart}}~~$\nicefrac{\thepage\,}{\pageref{LastPage}}$} % header right fields for all main odd pages
  \fancyhead[LE]{\hyperref[\ifx\@chapapp\appendixname app:\else chap:\fi\thechapter]{{\sffamily\bfseries\thechapter.\hskip0.75em\nouppercase\Chaptername}}} % header right fields for all main even pages
}
\fancypagestyle{plainheadfront}{
  \commonheadfoot
  %\fancyhead[C]{}
  \fancyhead[LO]{\thepage}
  \fancyhead[RE]{\thepage}
}
\fancypagestyle{plainhead}{
  \commonheadfoot
  \fancyhead[RO]{\hyperref[part:\thepart]{\small\partname.\,\thepart}~~~$\nicefrac{\thepage\,}{\pageref{LastPage}}$}
  \fancyhead[LE]{$\nicefrac{\thepage\,}{\pageref{LastPage}}$}
}
\fancypagestyle{plainheadback}{
  \commonheadfoot
  \fancyhead[RO]{$\nicefrac{\thepage\,}{\pageref{LastPage}}$}
  \fancyhead[LE]{$\nicefrac{\thepage\,}{\pageref{LastPage}}$}
}
\renewcommand*\frontmatter{%
  \if@twoside\cleardoubleoddpage\else\clearpage\fi
  \@frontmattertrue\@mainmatterfalse\@backmatterfalse\pagenumbering{roman}\pagestyle{plainheadfront}%
}
\renewcommand*\mainmatter{%
  \if@twoside\cleardoubleoddpage\else\clearpage\fi
  \@frontmatterfalse\@mainmattertrue\@backmatterfalse\pagenumbering{arabic}\pagestyle{main}%
}
\renewcommand*\backmatter{%
  \if@openright\cleardoubleoddpage\else\clearpage\fi
  \@frontmatterfalse\@mainmatterfalse\@backmattertrue\pagestyle{plainheadback}
}
%\AtBeginEnvironment{theindex}{%
%  \let\oldhyperpage\hyperpage
%  \renewcommand{\hyperpage}[1]{\cdotfill~\oldhyperpage{#1}~}
%}
\AtEndEnvironment{theindex}{%
  \thispagestyle{plainheadback}
  \pagestyle{plainheadback}
}
%%%%% WATERMARK %%%%%%%%%%%%%%%%%%%%%%%%%%%%%%%%
\AddLayersToPageStyle{@everystyle@}{WatermarkLayer}
%%%%% SETLIST %%%%%%%%%%%%%%%%%%%%%%%%%%%%%%%%
\setlist[enumerate]{listparindent=\parindent, parsep=0pt, partopsep=0pt, topsep=3pt, itemsep=3pt, leftmargin=*}
\setlist[enumerate, 1]{leftmargin=\leftmargini}
%%%%% EQUATION %%%%%%%%%%%%%%%%%%%%%%%%%%%%
\renewcommand{\theequation}{\thesection.\arabic{equation}}
\@addtoreset{equation}{section}
%%%%% CAPTION STYLE %%%%%%%%%%%%%%%%%%%%%%%%%%%%
\captionsetup[figure]{%
  width=.8\textwidth,
  format=hang,
  labelfont={bf, sf},
  labelsep={colon},
  labelformat=simple,
  font={small},
}
\captionsetup[lstlisting]{
  justification=raggedright,
  singlelinecheck=false,
  position=above,
  aboveskip=1.1pt,
  belowskip=0pt,
  labelformat={empty},
  labelfont={bf, sf},
  labelsep={space},
  font={bf, large, sf},
  hypcap=false,
}
\captionsetup[table]{
  justification=raggedright,
  singlelinecheck=false,
  position=above,
  aboveskip=4pt,
  belowskip=5pt,
  labelformat={empty},
  labelfont={bf, sf},
  labelsep={space},
  font={bf, large, sf},
  hypcap=false,
}
%%%%% STYLE OF TABLE OF CONTENTS %%%%%
%\RedeclareSectionCommand[
%  tocindent=0em,
%  tocnumwidth=4.25em
%]{part}
%\addtokomafont{partentry}{\def\autodot{}}
%\RedeclareSectionCommand[
%  tocindent=1.5em,
%  tocnumwidth=1.5em
%]{chapter}
%\RedeclareSectionCommand[
%  tocindent=3em,
%  tocnumwidth=2.5em,
%  tocentrynumberformat=\entrynumberwithdot
%]{section}
%\RedeclareSectionCommand[
%  tocindent=4.5em,
%  tocnumwidth=2.5em,
%  tocentrynumberformat=\entrynumberwithdot
%]{subsection}
%%%%%%%%%%%%%%%%%%%%%%%%%%%%%%%%%%%%%%%%%%%%%%
%%%%% FOR STYLE OF TOC %%%%%%%%%%%%%%%%%%%%%%%
%%%%%%%%%%%%%%%%%%%%%%%%%%%%%%%%%%%%%%%%%%%%%%
\setcounter{tocdepth}{3}
\renewcommand\contentsname{\texorpdfstring{\hbox to 1.9em{目次}}{目次}}
\patchcmd{\l@part}{\begingroup}{\begingroup\begingroup\tikzset{every node/.style={rectangle,fill=blue!20,rounded corners}}}{}{}
\patchcmd{\l@part}{\endgroup}{\endgroup\endgroup}{}{}
%%%%% STYLE OF LIST FOR APPENDIX %%%%%
\AtBeginEnvironment{appendices}{%
  \@appendixtrue%
  \patchcmd{\part}{\newpage}{\relax}{}{}%
  \pretocmd{\part}{\tocAPartSeparateline{toc}{15pt}{-10pt}}{}{}{}% add page break before parts, except part 1
 % \patchcmd{\part}{\tableAPartSeparateline}{\relax}{}{}%
}
\AfterEndEnvironment{appendices}{%
  \patchcmd{\part}{\tocAPartSeparateline}{\relax}{}{}%
  \pretocmd{\part}{\addtocontents{toc}{\protect\newpage}}{}{}{}% add page break before parts, except the first one
  \@appendixfalse%
}
%%%%% STYLE OF PART FOR LIST %%%%%
%\newcommand{\tocpartnumdeco}[1]{%
%  \begin{tikzpicture}[baseline]
%    \node[draw, rounded corners, fill=blue!20, text=black, minimum width=1.5em, align=center, yshift=4.5pt] (partnumber) {#1};
%  \end{tikzpicture}
%}
%\renewcommand*{\addparttocentry}[2]{%
%  \addtocentrydefault{part}{\tocpartnumdeco{#1}}{#2}%
%}
%\renewcommand*{\partformat}{\thepart\autodot\enskip}
%%%%%%%%%%%%%%%%%%%%%%%%%%%%%%%%%%%%%%%%%%%%%%
%%%%% FOR STYLE OF LOP %%%%%%%%%%%%%%%%%%%%%%%
%%%%%%%%%%%%%%%%%%%%%%%%%%%%%%%%%%%%%%%%%%%%%%
\renewcommand\listoflopname{\texorpdfstring{\hbox to 2.75em{大項目}}{大項目}}
%%%%%%%%%%%%%%%%%%%%%%%%%%%%%%%%%%%%%%%%%%%%%%
%%%%% FOR STYLE OF LOF %%%%%%%%%%%%%%%%%%%%%%%
%%%%%%%%%%%%%%%%%%%%%%%%%%%%%%%%%%%%%%%%%%%%%%
\renewcommand\listfigurename{\texorpdfstring{\hbox to 2.75em{図目次}}{図目次}}
\renewcommand{\figurename}{図}
\renewcommand{\figureautorefname}{\figurename} % figure --> 図
%\setcounter{lofdepth}{2}
%\renewcommand*\l@figure{\@dottedtocline{1}{1.5em}{3.2em}}
%\renewcommand*{\l@subfigure}{\@dottedxxxline{\ext@subfigure}{2}{4.7em}{2.3em}}
%%%%%%%%%%%%%%%%%%%%%%%%%%%%%%%%%%%%%%%%%%%%%%
%%%%% FOR STYLE OF LOC %%%%%%%%%%%%%%%%%%%%%%%
%%%%%%%%%%%%%%%%%%%%%%%%%%%%%%%%%%%%%%%%%%%%%%
\renewcommand\listofloCname{Column一覧}
%%%%%%%%%%%%%%%%%%%%%%%%%%%%%%%%%%%%%%%%%%%%%%
%%%%% FOR STYLE OF LOT %%%%%%%%%%%%%%%%%%%%%%%
%%%%%%%%%%%%%%%%%%%%%%%%%%%%%%%%%%%%%%%%%%%%%%
\renewcommand\listtablename{\texorpdfstring{\hbox to 2.75em{表目次}}{表目次}}
\renewcommand{\tablename}{表}
\AtBeginDocument{\renewcommand{\thetable}{}}
%%%%%%%%%%%%%%%%%%%%%%%%%%%%%%%%%%%%%%%%%%%%%%
%%%%% FOR STYLE OF LISTLINGS %%%%%%%%%%%%%%%%%
%%%%%%%%%%%%%%%%%%%%%%%%%%%%%%%%%%%%%%%%%%%%%%
\renewcommand{\lstlistlistingname}{プログラム 目次}
\renewcommand{\lstlistingname}{{\scshape Prg: }}
\AtBeginDocument{%
%  \counterwithout{lstlisting}{chapter}
  \renewcommand{\thelstlisting}{}%
}
\renewcommand*\l@lstlisting{\@dottedtocline{1}{0.6em}{3.2em}}
\lstdefinelanguage{GcodeBasic}{
  sensitive=true,%
  basicstyle=\small\ttfamily\linespread{1}\selectfont,%
  abovecaptionskip=0pt,
  belowcaptionskip=0pt,
%  aboveskip=0\baselineskip,
%  belowskip=0\baselineskip,
  numbers=left,%
  numbersep=6pt,%
  numberstyle=\ttfamily\scriptsize\color{black!75!},%
  breaklines=true,%
  breakindent=3pt,%
  postbreak=\mbox{\textcolor{blue}{$\hookrightarrow$}\,},%
  frame=single,%
  columns=fixed,%
  basewidth=0.53em,%
  lineskip=-1.2pt,
  comment=[l]{(},%
  commentstyle=\footnotesize\color{black!60!green!100}\slshape,%
  stringstyle={},%
  alsoletter={},%
  keywords={},%
  keywordstyle=\bfseries,%
}
\lstdefinestyle{Gcode-more}{
  language=GcodeBasic,
  nolol,
  morekeywords={[20]IF, GOTO, THEN, WHILE, DO1, DO2, END1, END2},%
  keywordstyle={[20]\fontfamily{pcr}\selectfont\bfseries},%
  emph={G90, G91},
  emphstyle={\bfseries\color{red}},
  emph={[2]G53, G54, G55, G56, G57},
  emphstyle={[2]\bfseries\color{blue}},
  emph={[3]G40, G41, G42, G43, G44},
  emphstyle={[3]\bfseries\color{orange}},
  emph={[4]G00, G01, G02, G03, G31},
  emphstyle={[4]\bfseries\color{magenta}},
  emph={[5]G28, G30},
  emphstyle={[5]\bfseries\color{violet}},
  emph={[6]G04, G10, G13, G17, G49, G58, G65, G80},
  emphstyle={[6]\bfseries\color{cyan}},
  emph={[11]
    T01, T02, T06, T11, T13, T16, T31, T50,
    M00, M01, M03, M04, M05, M06, M08, M09,
    M10, M11, M19,
    M30, M32, M33,
    M48, M49,
    M60, M61, M62, M63,
    M71, M72, M73, M74, M78, M79,
    M98, M99,
    M117, M214, M262},
  emphstyle={[11]\fontfamily{pcr}\bfseries\color{yellow!100!green!80!black!100!}},
  emph={[22]SIN, COS, SQRT, FIX, FUP, ROUND, ABS},%
  emphstyle={[22]\fontfamily{pcr}\selectfont\bfseries},%
  emph={[31]
    P140001, P110002, P130001, P130002, P150002, P150003,
    P210003, P220001, P220002, P230001, P230002, P530001, P530002,
    P410000, P420000, P430000, P440000, P450000, P490005,
    P910001, P910002},%
  emphstyle={[31]\bfseries\color{cyan}},%
}
\lstdefinestyle{Gcode-bundle}{
  language=GcodeBasic,
  nolol,
  morekeywords={[120]WHILE, THEN, DO1, DO2, END1, END2},%
  keywordstyle={[120]\fontfamily{pcr}\selectfont\bfseries},%
  emph={[102]},
  emphstyle={[102]\bfseries\color{blue}},
  emph={[111]M10, M11, M19, M30, M32, M33, M48, M49,  M61, M62, M74, M78, M262},
  emphstyle={[111]\fontfamily{pcr}\bfseries\color{yellow!100!green!80!black!100!}},
  emph={[112]M98P9001},
  emphstyle={[112]\bfseries\color{cyan}},
  emph={[122]SIN, COS, SQRT, FIX, FUP, ABS},%
  emphstyle={[122]\fontfamily{pcr}\selectfont\bfseries},%
}
%\lstdefinestyle{Gcode-ren}{
%  language=GcodeBasic,
%  nolol,
%  morekeywords={[220]IF, WHILE, THEN, DO1, DO2, END1, END2},%
%  keywordstyle={[220]\fontfamily{pcr}\selectfont\bfseries},%
%  emph={[202]},
%  emphstyle={[202]\bfseries\color{blue}},
%  emph={[211]M0, M5, M19, M32, M33, M99, M214},
%  emphstyle={[211]\fontfamily{pcr}\bfseries\color{yellow!100!green!80!black!100!}},
%  emph={[222]SIN, COS, TAN, ATAN, SQRT, FIX, FUP, ROUND, DPRNT},%
%  emphstyle={[222]\fontfamily{pcr}\selectfont\bfseries},%
%}
%%%%%%%%%%%%%%%%%%%%%%%%%%%%%%%%%%%%%%%%%%%%%%
%%%%% FOR STYLE OF INDEX %%%%%%%%%%%%%%%%%%%%%
%%%%%%%%%%%%%%%%%%%%%%%%%%%%%%%%%%%%%%%%%%%%%%
\renewcommand{\indexname}{\texorpdfstring{\hbox to 1.9em{索引}}{索引}}
%%%%%%%%%%%%%%%%%%%%%%%%%%%%%%%%%%%%%%%%%%%%%%
%%%%% FOR STYLE OF BIBLATEX %%%%%%%%%%%%%%%%%%
%%%%%%%%%%%%%%%%%%%%%%%%%%%%%%%%%%%%%%%%%%%%%%
\renewcommand{\bibname}{\texorpdfstring{\hbox to 3.75em{参考文献}}{参考文献}}
\ExecuteBibliographyOptions{
  sorting=none,
  refsegment=chapter,
  hyperref=true,
  block=nbpar,
  subentry=true,
  citecounter=true,
}
\appto\bibfont{\footnotesize\setstretch{1.1}}
\DeclareFieldFormat{labelnumberwidth}{\mkbibbrackets{#1}\hspace{-6pt}}
%%%%%%%%%%%%%%%%%%%%%%%%%%%%%%%%%%%%%%%%%%%%%%
%%%%% FOR STYLE OF PART %%%%%%%%%%%%%%%%%%%%%%
%%%%%%%%%%%%%%%%%%%%%%%%%%%%%%%%%%%%%%%%%%%%%%
\renewcommand{\partautorefname}{part}  % part --> part
\renewcommand\partpagestyle{emptydate}
%\renewcommand*{\partformat}{\begin{gtfamily}\thepart\end{gtfamily}}
%\renewcommand{\partname}{}
%\renewcommand{\thepart}{第\hspace{2truemm}\Roman{part}\hspace{2truemm}部} %
%\renewcommand*{\addparttocentry}[2]{\addtocentrydefault{part}{#1}{第#2部}}
%%%%%%%%%%%%%%%%%%%%%%%%%%%%%%%%
%%%%% FOR STYLE OF CHAPTER %%%%%
%%%%%%%%%%%%%%%%%%%%%%%%%%%%%%%%
\renewcommand{\chapterautorefname}{章}  % chapter --> 章
\renewcommand\chapterpagestyle{\if@frontmatter plainheadfront\else plainhead\fi}
\renewcommand*{\chapterformat}{%
  \begin{tikzpicture}[every node/.style={signal, draw, text=white, signal to=nowhere},
                      baseline=-8.5pt]
    \node[fill=\if@appendix white\else Maroon!65\fi,
          text=\if@appendix Maroon!65\else white\fi,
          signal from=east,
          inner sep=5pt,
          text height=1.5ex,
          text depth=2pt]
      at (0, 0) {\LARGE{\fontfamily{pplx}\bfseries\,\thepart.~\thechapter~~\relax}};
  \end{tikzpicture}~~%
}
%%%%% STYLE OF APPENDICES %%%%%%%%%%%%%%%%%%%%%%%%%%%%
\renewcommand{\setthesection}{\Alph{section}}
%\renewcommand{\appendixname}{補\hskip0.5em 遺} % appendix --> 補 遺
\renewcommand{\appendixautorefname}{補遺\!} % appendix --> 補遺
%%%%%%%%%%%%%%%%%%%%%%%%%%%%%%%%%%%%%%%%%%%%%%
%%%%% FOR STYLE OF SECTION %%%%%%%%%%%%%%%%%%%
%%%%%%%%%%%%%%%%%%%%%%%%%%%%%%%%%%%%%%%%%%%%%%
\renewcommand{\sectionautorefname}{節\!} % section --> 節
\setcounter{secnumdepth}{4}
\renewcommand*{\sectionformat}{%
  \begin{tikzpicture}[every node/.style={signal, draw, text=white, signal to=nowhere},
                      baseline=-5.25pt]
    \node[fill=Green!65!,
          text=white,
          signal from=east,
          inner sep=5pt,
          text height=1.3ex,
          text depth=0pt]
      at (0, 0) {\Large{\fontfamily{pplx}\bfseries\,\thesection~\relax}};
  \end{tikzpicture}~~%
}
%%%%%%%%%%%%%%%%%%%%%%%%%%%%%%%%%%%%%%%%%%%%%%
%%%%% FOR STYLE OF SUBSECTION %%%%%%%%%%%%%%%%
%%%%%%%%%%%%%%%%%%%%%%%%%%%%%%%%%%%%%%%%%%%%%%
\renewcommand{\subsectionautorefname}{\sectionautorefname} % subsection --> section
\renewcommand*{\subsectionformat}{%
  \begin{tikzpicture}[every node/.style={signal, draw, text=white, signal to=nowhere},
                      baseline=-4pt]
    \node[fill=blue!50!,
          text=white,
          signal from=east,
          inner sep=5pt,
          text height=1.25ex,
          text depth=0pt]
      at (0, 0) {\large{\fontfamily{pplx}\bfseries\,\thesubsection~\relax}};
  \end{tikzpicture}~~%
}
%%%%%%%%%%%%%%%%%%%%%%%%%%%%%%%%%%%%%%%%%%%%%%
%%%%% FOR STYLE OF SUBSUBSECTION %%%%%%%%%%%%%
%%%%%%%%%%%%%%%%%%%%%%%%%%%%%%%%%%%%%%%%%%%%%%
\renewcommand{\subsubsectionautorefname}{\sectionautorefname} % subsubsection --> section
\renewcommand*{\subsubsectionformat}{%
  \begin{tikzpicture}[every node/.style={signal, draw, text=white, signal to=nowhere},
                      baseline=-3.75pt]
    \node[fill=Green!50!Blue!40!,
          text=white,
          signal from=east,
          inner sep=5pt,
          text height=1.25ex,
          text depth=0pt]
      at (0, 0) {\small{\fontfamily{pplx}\bfseries\,\thesubsubsection~\relax}};
  \end{tikzpicture}~~%
}
%%%%%%%%%%%%%%%%%%%%%%%%%%%%%%%%%%%%%%%%%%%%%%
%%%%% FOR STYLE OF PARAGRAPH %%%%%%%%%%%%%%%%%
%%%%%%%%%%%%%%%%%%%%%%%%%%%%%%%%%%%%%%%%%%%%%%
%for scrbook.cls
\RedeclareSectionCommand[%
  style=section,%
  level=4,%
  indent=0pt,%
  afterindent=false,
  beforeskip=3.25ex \@plus1ex \@minus.2ex,%
  afterskip=0.1ex \@plus.1ex \@minus.1ex,% -1em から変更
  tocindentfollows=subsubsection,%
  tocstyle=section,%
  tocindent=10em,%
  tocnumwidth=5em,% def: 5em
  font=\raggedsection\normalfont\sectfont\gtfamily\nobreak\sball{blue}~
]{paragraph}
%for book.cls
%\renewcommand\paragraph[1]{%
%  \@startsection{paragraph}{\paragraphnumdepth}{0pt}%
%  {3.25ex \@plus1ex \@minus.2ex}% \@plus, \@minusは伸び縮みできるスペースの長さ
%  {0.1ex\@plus.1ex \@minus.1ex}% ここが正だと改行されて、値だけ垂直スペースが入る
%  {\raggedsection\normalfont\sectfont\gtfamily\nobreak\size@paragraph\sball{blue}~}{#1}\noindent
%}
%%%%%%%%%%%%%%%%%%%%%%%%%%%%%%%%%%%%%%%%%%%%%%
%%%%% FOR STYLE OF SUBPARAGRAPH %%%%%%%%%%%%%%
%%%%%%%%%%%%%%%%%%%%%%%%%%%%%%%%%%%%%%%%%%%%%%
\RedeclareSectionCommand[%
  style=section,%
  level=5,%
  indent=\scr@parindent,%
  afterindent=false,
  beforeskip=0.5ex \@plus1ex \@minus.2ex,% 3.25ex \@plus1ex \@minus .2ex から変更
  afterskip=0.1ex \@plus.1ex \@minus.1ex,% -1em から変更
  tocstyle=section,%
  tocindent=12em,%
  tocnumwidth=6em%
]{subparagraph}
%%%%%%%%%%%%%%%%%%%%%%%%%%%%%%%%%%%%%%%%%%%%%%
%%%%% FOR STYLE OF TCBSET %%%%%%%%%%%%%%%%%%%%
%%%%%%%%%%%%%%%%%%%%%%%%%%%%%%%%%%%%%%%%%%%%%%
\definecolor{myheadercolor}{rgb}{0.68, 0.85, 0.90}
\tcbset{%
  %%%%% COLUMNBOX STYLE %%%%%
  Tabularbox/.style={%
    fonttitle=\gtfamily\bfseries,%
    breakable,%
    enhanced jigsaw,%
    left=.5ex,%
    right=.5ex,%
    bicolor,%
    colbacklower=black!10!white,%
    before upper={%
      \setcounter{GlobalFootnote}{\value{footnote}}% GlobalFootnoteValue=footnoteValue
      \let\oldfootnote=\footnote% \oldfootnote=\footnote
      \def\footnote{\stepcounter{GlobalFootnote}\oldfootnote[\arabic{GlobalFootnote}]}% define \footnote to \footnote using counter GlobalFootnote
      \renewcommand\thempfootnote{\arabic{mpfootnote}}% arabic footnote
    },
    after upper={%
      \setcounter{footnote}{\value{GlobalFootnote}}% footnoteValue=GlobalFootnoteValue
      \let\footnote=\oldfootnote% \footnote=\oldfootnote
    },
  },%
  %%%%% COLUMNBOX STYLE %%%%%
  Columnbox/.style={%
    Tabularbox,
    after title=\hfill\termblue{\Columnname~\thetcbcounter},%
  },%
  %%%%% FORMULA STYLE %%%%%
  Formulabox/.style={%
    Tabularbox,
    after title=\hfill\termblue{\Formulaname~\thetcbcounter},%
  },%
  %%%%% FIGUREBOX STYLE %%%%%
  Figurebox/.style={%
    notitle,%
    height=\textwidth,
%    width=\textwidth,
    center upper,%
    center lower,%
    arc=5pt,%
    outer arc=2pt,%
    boxrule=1pt,%
    boxsep=3mm,%
    valign=center,%
    halign=center,%
    left=0pt,%
    right=0pt,%
    colback=green!3!white,%
    colframe=black!25!white,%
    before={\centering},
  },
  %%%%% HIGHLIGHT MATH STYLE %%%%%
%  highlight math/.style={%
%    enhanced,%
%    arc=2pt,%
%    boxrule=0pt,%
%    frame hidden,%
%    fuzzy halo=1pt with blue,%
%    left=0pt,%
%    right=0pt,%
%    top=.4mm,%
%    bottom=.4mm,%
%    colback=yellow!40!white,%
%  },%
  %%%%% HOSOKUBOX STYLE %%%%%
  hosokubox/.style={%
    title={\termblue{\hosokuname~\thetcbcounter}~},%
    attach title to upper,%
    breakable,%
    enhanced jigsaw,%
    size=fbox,%
    arc=0pt,%
    middle=1mm,%
    colback=hosoku,%
    colframe=hosoku,%
    drop lifted shadow={blue!100!white!50!},%
    skin first is subskin of={enhanced jigsaw}{no shadow},%
    skin middle is subskin of={enhanced jigsaw}{no shadow},%
    skin last is subskin of={enhanced jigsaw}{drop lifted shadow={blue!100!white!50!}},%
    segmentation style={draw=black!50!white},%
    after=\smallskip\noindent{\color{white}},%
  },
  %%%%% TWOCTABLEBOX STYLE %%%%%
  twoCtablebox/.style={%
    breakable,%
    enhanced,%
    fonttitle=\bfseries,%
    fontupper=\small\sffamily,%
    colframe=black!50!black,%
    colbacktitle=blue!10!white,%
    coltitle=black,%
    top=0pt,%
    bottom=0pt,%
    left=0pt,%
    right=0pt,%
    enlarge top by=-5pt,
    enlarge bottom by=5pt,
    before upper={%
      \setcounter{GlobalFootnote}{\value{footnote}}% GlobalFootnoteValue=footnoteValue
      \let\oldfootnote=\footnote% \oldfootnote=\footnote
      \def\footnote{\stepcounter{GlobalFootnote}\oldfootnote[\arabic{GlobalFootnote}]}% define \footnote to \footnote using counter GlobalFootnote
      \renewcommand\thempfootnote{\arabic{mpfootnote}}% arabic footnote
%      \renewcommand{\arraystretch}{1.2}% 行の高さを調整
      \setlength{\LTpre}{0pt}%
      \setlength{\LTpost}{0pt}%
      \setlength{\LTleft}{0pt}%
      \setlength{\LTright}{0pt}%
      \setlength\cellspacetoplimit{3.5pt}
      \setlength\cellspacebottomlimit{3.5pt}
      \begin{longtable}{@{}c|Sl@{\extracolsep{\fill}}c}%
      \addtocounter{table}{-1}%
    },%
    after upper={%
      \end{longtable}%
      \setcounter{footnote}{\value{GlobalFootnote}}% footnoteValue=GlobalFootnoteValue
      \let\footnote=\oldfootnote% \footnote=\oldfootnote
    },
  },
}
%%%%%%%%%%%%%%%%%%%%%%%%%%%%%%%%%%%%%%%%%%%%%%
%%%%% FOR STYLE OF TIKZSET %%%%%%%%%%%%%%%%%%%
%%%%%%%%%%%%%%%%%%%%%%%%%%%%%%%%%%%%%%%%%%%%%%
\tikzset{
  %%%%% SECTIONFORMAT STYLE %%%%%
%  sect/.style={signal, draw, text=white},
%  section/.style={sect, fill=konpeki!100!, signal to=east, inner sep=3pt},
%  subsection/.style={sect, fill=moegi!90!, signal to=nowhere, inner sep=3pt},
%  subsubsection/.style={sect, fill=sssec!100!, signal to=nowhere, inner sep=3pt},
  %%%%% TERMINAL STYLE %%%%%
  terminal/.style={%
    rectangle,%
    minimum size=10pt,%
    rounded corners=1.5mm,%
    thin,%
    draw=black!75,%
    top color=white,%
    font=\fontfamily{pplx},%
    inner sep=3pt,%
    inner xsep=3pt,%
    text height=1ex,%
    text depth=0pt,%
  },
  %%%%% BMATRIX STYLE %%%%%
%  every left delimiter/.style={xshift=.5em},
%  every right delimiter/.style={xshift=-.5em},
%  bmatrix/.style={matrix of math nodes, left delimiter=[, right delimiter=],},
}

\makeatother