%!TEX root = ../RPA_for_Creating_Program_Note.tex

\makeatletter

%!TEX root = ../RPA_for_Creating_Program_Note.tex


%%%%%%%%%%%%%%%%%%%%%%%%%%%%%%%%%%%%%%%%%%%%%%
%%%%% MOULDCOODINATE %%%%%%%%%%%%%%%%%%%%%%%%%
%%%%%%%%%%%%%%%%%%%%%%%%%%%%%%%%%%%%%%%%%%%%%%
\def\mouldCoordinate{%
\begin{tikzpicture}[scale=1, every node/.style={scale=0.8}]
% 値の計算
\pgfmathsetmacro{\Ax}{9.6} %A:T_iのx座標
\pgfmathsetmacro{\Ay}{3.5} %A:T_iのy座標
\pgfmathsetmacro{\Bx}{2.0+(\Ax)} %B:T_oのx座標
\pgfmathsetmacro{\Cy}{-4.2}                  %C:B_iのy座標
\pgfmathsetmacro{\Ri}{sqrt((\Ax)^2+(\Ay)^2)} %R_iの長さ
\pgfmathsetmacro{\Cx}{sqrt((\Ri)^2-(\Cy)^2)} %C:B_iのx座標
\pgfmathsetmacro{\Ro}{sqrt((\Bx)^2+(\Ay)^2)} %R_oの長さ
\pgfmathsetmacro{\Dx}{sqrt((\Ro)^2-(\Cy)^2)} %D:B_oのx座標
\pgfmathsetmacro{\Rc}{(\Ri+\Ro)/2}           %R_cの長さ
\pgfmathsetmacro{\Ex}{sqrt((\Rc)^2-(\Ay)^2)} %E:湾曲中心線トップ端のx座標
\pgfmathsetmacro{\Fx}{sqrt((\Rc)^2-(\Cy)^2)} %F:湾曲中心線ボトム端のx座標
\pgfmathsetmacro{\Ub}{2.4}                   %Ub:受板-モールド接点のy座標
\pgfmathsetmacro{\Ux}{sqrt((\Ri)^2-(\Ub)^2)} %Ux:受板-モールド接点のx座標
\pgfmathsetmacro{\Uxo}{sqrt((\Ro)^2-(\Ub)^2)} %Ux:受板-モールド接点のx座標
\pgfmathsetmacro{\Hx}{1+(\Ri)} %H:テーブルの中心
\pgfmathsetmacro{\Ix}{1.90} %I:テーブルx方向の長さの半分
% 座標系を描画
\draw[-latex, line width=0.5pt] (-0.4, 0) -- (14.5, 0) node[below] {\textbf{Re}};
\draw[-latex, line width=0.5pt] (0, -4) -- (0, 4) node[below right] {\textbf{Im}};
% 座標を定義
\coordinate (O) at (  0, 0); % 原点
\coordinate (A) at (\Ax, \Ay); %
\coordinate (B) at (\Bx, \Ay);
\coordinate (C) at (\Cx, \Cy);
\coordinate (D) at (\Dx, \Cy);
\coordinate (E) at (\Ex, \Ay);
\coordinate (F) at (\Fx, \Cy);
\coordinate (Rc) at (\Rc, 0);
\coordinate (Ri) at (\Ri, 0);
\coordinate (Ro) at (\Ro, 0);
\coordinate (Ut) at (\Ux, \Ub);
\coordinate (Ub) at (\Ux, -\Ub);
\coordinate (Uto) at (\Uxo, \Ub);
\coordinate (Ubo) at (\Uxo, -\Ub);
\coordinate (Tal) at (\Hx-\Ix, \Ub);
\coordinate (Tbl) at (\Hx-\Ix, -\Ub);
\coordinate (Tar) at (\Hx+\Ix, \Ub);
\coordinate (Tbr) at (\Hx+\Ix, -\Ub);
\coordinate (Lt) at (\Hx+\Ix+0.4, \Ub);
\coordinate (Lb) at (\Hx+\Ix+0.4, -\Ub);
\coordinate (Lc) at (\Hx+\Ix+0.4, 0);
\coordinate (Fc) at (\Hx+\Ix+1.0, 0);
\coordinate (Ft) at (\Hx+\Ix+1.0, \Ay);
\coordinate (Fb) at (\Hx+\Ix+1.0, \Cy);
% 点を描画
\fill (O) circle (2pt);
\fill (A) circle (2pt);
\fill (B) circle (2pt);
\fill (C) circle (2pt);
\fill (D) circle (2pt);
\fill (E) circle (2pt);
\fill (Rc) circle (2pt);
\fill (Ro) circle (2pt);
\fill (Ri) circle (2pt);
\fill (Ut) circle (2pt);
\fill (Ub) circle (2pt);
% 点にラベルを付ける
\node at (O) [below left] {O};
\node at (A) [above] {T$_\mathrm i$};
\node at (B) [above] {T$_\mathrm o$};
\node at (C) [below] {B$_\mathrm i$};
\node at (D) [below] {B$_\mathrm o$};
\node at (Rc) [above right] {$R_\mathrm c$};
\node at (Ro) [below right] {$R_\mathrm o$};
\node at (Ri) [below left] {$R_\mathrm i$};
\node at (Ut) [right] {U$_\mathrm T$};
\node at (Ub) [right] {U$_\mathrm B$};
% モールド外形
\draw[line width=0.75pt, fill=ffwwqq, fill opacity=0.1]
  let \p1=(A), \p2=(C), \p3=(B), \p4=(D), \n1={atan2(\y1,\x1)}, \n2={atan2(\y2,\x2)}, \n3={atan2(\y3,\x3)}, \n4={atan2(\y4,\x4)}
    in (A) -- (B) -- (\n3:\Ro) arc (\n3:\n4:\Ro) -- (C) -- (\n2:\Ri) arc (\n2:\n1:\Ri) -- cycle;
% モールド中心線
\draw[dotted, line width=0.5pt] let \p1=(E), \p2=(F), \n1={atan2(\y1,\x1)}, \n2={atan2(\y2,\x2)}
  in (\n1:\Rc) arc (\n1:\n2:\Rc);
% テーブル
\draw (Ut) -- (Tal) -- (Tbl) -- (Ub);
\draw (Uto) -- (Tar) -- (Tbr) -- (Ubo);
\draw[dotted] (Tar) -- (Lt);
\draw[dotted] (Tbr) -- (Lb);
\draw[dotted] (Tar) -- (Lt);
\draw[latex-latex, line width=0.5pt] (Lc) -- (Lt) node[midway, right] {$l$};
\draw[latex-latex, line width=0.5pt] (Lc) -- (Lb) node[midway, right] {$l$};
% 振分け
\draw[dotted] (B) -- (Ft);
\draw[dotted] (D) -- (Fb);
\draw[latex-latex, line width=0.5pt] (Fc) -- (Ft) node[midway, right] {$f_\mathrm T$};
\draw[latex-latex, line width=0.5pt] (Fc) -- (Fb) node[midway, right] {$f_\mathrm B$};
% 半径
\draw[dotted, line width=0.5pt] (O) -- (A) node[midway, above left] {$R_\mathrm i$} ;
\draw[dotted, line width=0.5pt] (O) -- (B) node[midway, below right] {$R_\mathrm o$} ;
\draw[dotted, line width=0.5pt] (O) -- (Ub) node[midway, above right] {$R_\mathrm i$} ;
% 角度
\draw[line width=0.5pt, fill=ffwwqq, fill opacity=0.1]
  let \p1=(Ri), \p2=(A), \n1={atan2(\y1,\x1)}, \n2={atan2(\y2,\x2)}
   in (\n1:2) arc (\n1:\n2:2) node[midway, right, opacity=1] {$\alpha_{\mathrm T_\mathrm i}$} -- (O);
\draw[line width=0.5pt, fill=qqzzqq, fill opacity=0.1]
  let \p1=(Ro), \p2=(B), \n1={atan2(\y1,\x1)}, \n2={atan2(\y2,\x2)}
  in (\n1:3.2) arc (\n1:\n2:3.2) node[midway, right, opacity=1] {$\alpha_{\mathrm T_\mathrm o}$} -- (O) -- cycle ;
\draw[line width=0.5pt, fill=wwqqcc, fill opacity=0.1]
  let \p1=(Ri), \p2=(Ub), \n1={atan2(\y1,\x1)}, \n2={atan2(\y2,\x2)}
  in (\n1:2.7) arc (\n1:\n2:2.7) node[midway, right, opacity=1] {$\alpha_{\mathrm U_\mathrm B}$} -- (O);
\end{tikzpicture}%
}

%%%%%%%%%%%%%%%%%%%%%%%%%%%%%%%%%%%%%%%%%%%%%%
%%%%% MOULDWITHUKEITA %%%%%%%%%%%%%%%%%%%%%%%%
%%%%%%%%%%%%%%%%%%%%%%%%%%%%%%%%%%%%%%%%%%%%%%
\def\mouldwithukeita{%
\begin{tikzpicture}[scale=0.5,
                    every node/.style={scale=0.4},
                    >={Latex[length=1mm, width=0.75mm]},]
% 値の計算
\pgfmathsetmacro{\Ax}{12}                    %A:T_iのx座標
\pgfmathsetmacro{\Ay}{3.5}                   %A:T_iのy座標
\pgfmathsetmacro{\Bx}{2.0+(\Ax)}             %B:T_oのx座標
\pgfmathsetmacro{\Cy}{-4.2}                  %C:B_iのy座標
\pgfmathsetmacro{\TUb}{2.4}                  %TUb:テーブル-モールドの交点y座標
\pgfmathsetmacro{\Ix}{2.7}                   %I:テーブルx方向の長さの半分
\pgfmathsetmacro{\Uw}{0.75}                  %Uw:受板の幅
\pgfmathsetmacro{\Ul}{0.45}                  %Ul:受板の長さ
\pgfmathsetmacro{\Ur}{1.35}                   %Uw:受板の半径
\pgfmathsetmacro{\Ri}{sqrt((\Ax)^2+(\Ay)^2)} %R_iの長さ
\pgfmathsetmacro{\Hx}{0.1+(\Ri)}             %H:テーブルの中心
% 値の計算
\pgfmathsetmacro{\Cx}{sqrt((\Ri)^2-(\Cy)^2)}    %C:B_iのx座標
\pgfmathsetmacro{\Ro}{sqrt((\Bx)^2+(\Ay)^2)}    %R_oの長さ
\pgfmathsetmacro{\Dx}{sqrt((\Ro)^2-(\Cy)^2)}    %D:B_oのx座標
\pgfmathsetmacro{\Rc}{(\Ri+\Ro)/2}              %R_cの長さ
\pgfmathsetmacro{\Ex}{sqrt((\Rc)^2-(\Ay)^2)}    %E:湾曲中心線トップ端のx座標
\pgfmathsetmacro{\Fx}{sqrt((\Rc)^2-(\Cy)^2)}    %F:湾曲中心線ボトム端のx座標
\pgfmathsetmacro{\TUx}{sqrt((\Ri)^2-(\TUb)^2)}  %TUx:テーブル-モールド内側との交点のx座標
\pgfmathsetmacro{\TUxo}{sqrt((\Ro)^2-(\TUb)^2)} %TUxo:テーブル-モールド外側との交点x座標
\pgfmathsetmacro{\Ub}{\Ri*sin(asin((\TUb-\Uw/2)/(\Ri-\Ur)))} %Ub:ボトム側受板-モールド接点のy座標
\pgfmathsetmacro{\Ux}{sqrt((\Ri)^2-(\Ub)^2)}    %Ux:トップ側受板-モールド接点のx座標
\pgfmathsetmacro{\Ucx}{\Ux-sqrt((\Ur)^2-((\Uw)/2-(\TUb-\Ub))^2)} %Uw:受板の半径
\pgfmathsetmacro{\TBUx}{\Ucx+sqrt((\Ur)^2-((\Uw)/2)^2)} %TBUx:テーブルと受板の交点x座標
% 座標の定義
\coordinate (A) at (\Ax, \Ay); %
\coordinate (B) at (\Bx, \Ay);
\coordinate (C) at (\Cx, \Cy);
\coordinate (D) at (\Dx, \Cy);
\coordinate (E) at (\Ex, \Ay);
\coordinate (F) at (\Fx, \Cy);
\coordinate (Ut) at (\Ux, \Ub);
\coordinate (TUt) at (\TUx, \TUb);
\coordinate (Ub) at (\Ux, -\Ub);
\coordinate (TUb) at (\TUx, -\TUb);
\coordinate (TUto) at (\TUxo, \TUb);
\coordinate (TUbo) at (\TUxo, -\TUb);
\coordinate (Tal) at (\Hx-\Ix, \TUb);
\coordinate (Tbl) at (\Hx-\Ix, -\TUb);
\coordinate (Tar) at (\Hx+\Ix, \TUb);
\coordinate (Tbr) at (\Hx+\Ix, -\TUb);
\coordinate (Lt) at (\Hx+\Ix+0.4, \TUb);
\coordinate (Lb) at (\Hx+\Ix+0.4, -\TUb);
\coordinate (Lc) at (\Hx+\Ix+0.4, 0);
\coordinate (Fc) at (\Hx+\Ix+1.0, 0);
\coordinate (TUc) at (\Ucx, \TUb-\Uw/2);
\coordinate (BUc) at (\Ucx, \Uw/2-\TUb);
\coordinate (TTU) at (\TBUx, \TUb);
\coordinate (TTUex) at (\TBUx-\Ul, \TUb);
\coordinate (TUex) at (\TBUx-\Ul, \TUb-\Uw);
\coordinate (TU) at (\TBUx, \TUb-\Uw);
\coordinate (TBU) at (\TBUx, -\TUb);
\coordinate (TBUex) at (\TBUx-\Ul, -\TUb);
\coordinate (BUex) at (\TBUx-\Ul, -\TUb+\Uw);
\coordinate (BU) at (\TBUx, -\TUb+\Uw);
% モールド外形
\draw[line width=0.5pt, fill=ffwwqq, fill opacity=0.1]
  let \p1=(A), \p2=(C), \p3=(B), \p4=(D), \n1={atan2(\y1,\x1)}, \n2={atan2(\y2,\x2)}, \n3={atan2(\y3,\x3)}, \n4={atan2(\y4,\x4)}
    in (A) -- (B) -- (\n3:\Ro) arc (\n3:\n4:\Ro) -- (C) -- (\n2:\Ri) arc (\n2:\n1:\Ri) -- cycle;
% モールド中心線
\draw[dotted, line width=0.5pt] let \p1=(E), \p2=(F), \n1={atan2(\y1,\x1)}, \n2={atan2(\y2,\x2)}
  in (\n1:\Rc) arc (\n1:\n2:\Rc);
% テーブル
\draw (TUt) -- (Tal) -- (Tbl) -- (TUb);
\draw (TUto) -- (Tar) -- (Tbr) -- (TUbo);
% 受板
\draw[fill=gray!15!] let \p1=(TUc), \p2=(TU), \p3=(TTU), \n1={atan2(\y2-\y1,\x2-\x1)}, \n2={atan2(\y3-\y1,\x3-\x1)},
    in (TTU) -- (TTUex) -- (TUex) -- (TU) arc[start angle=\n1, end angle=\n2, radius=\Ur] -- cycle;
\draw[fill=gray!15!] let \p1=(BUc), \p2=(BU), \p3=(TBU), \n1={atan2(\y2-\y1,\x2-\x1)}, \n2={atan2(\y3-\y1,\x3-\x1)},
    in (TBU) -- (TBUex) -- (BUex) -- (BU) arc[start angle=\n1, end angle=\n2, radius=\Ur] -- cycle;
\draw[<->] (BUc) -- (Ub) node[midway, above] {$\rho$};
\draw[<->] ([xshift=-1.25mm]TTUex) -- ([xshift=-1.25mm]TUex) node[midway, left] {$\sigma$};
% 点を描画
\fill (Ut) circle (1pt);
\fill (Ub) circle (1pt);
\fill (TUc) circle (1pt);
\fill (BUc) circle (1pt);
% 点にラベルを付ける
\node at (Ut) [right] {$R_\mathrm ie^{i\alpha_{\mathrm U_\mathrm T}}$};
\node at (Ub) [right] {$R_\mathrm ie^{i\alpha_{\mathrm U_\mathrm B}}$};
\node at (TUc) [above] {U$_\mathrm T$};
\node at (BUc) [below] {U$_\mathrm B$};
\end{tikzpicture}%
}


%%%%%%%%%%%%%%%%%%%%%%%%%%%%%%%%%%%%%%%%%%%%%%
%%%%% DATE %%%%%%%%%%%%%%%%%%%%%%%%%%%%%%%%%%%
%%%%%%%%%%%%%%%%%%%%%%%%%%%%%%%%%%%%%%%%%%%%%%
\newcommand{\customtoday}{\the\year/\two@digits{\the\month}/\two@digits{\the\day}}
\newcommand{\customdate}{\customtoday\ \currenttime\ (\jadayofweek{\the\year}{\the\month}{\the\day})}
\newcommand{\customtodayap}{\ifnum\currenthour<12 \customtoday\,a.m.\else\customtoday\,p.m.\fi}
%%%%%%%%%%%%%%%%%%%%%%%%%%%%%%%%%%%%%%%%%%%%%%
%%%%% NEWIF %%%%%%%%%%%%%%%%%%%%%%%%%%%%%%%%%%
%%%%%%%%%%%%%%%%%%%%%%%%%%%%%%%%%%%%%%%%%%%%%%
\newif\if@backmatter%\@backmattertrue
\newif\if@frontmatter%\@frontmattertrue
\newif\if@appendix%\@appendixfalse
%%%%%%%%%%%%%%%%%%%%%%%%%%%%%%%%%%%%%%%%%%%%%%
%%%%% CLEARLEFTPAGE %%%%%%%%%%%%%%%%%%%%%%%%%%
%%%%%%%%%%%%%%%%%%%%%%%%%%%%%%%%%%%%%%%%%%%%%%
\newcommand{\thispageNum}{\the\value{page}}
\newcommand{\clearleftpage}{%
  \if@twoside%
    \ifodd\thispageNum%
      \clearpage%
    \else
      \cleardoublepage%
    \fi
  \else%
    \clearpage%
  \fi
}
\newcommand{\clearrightpage}{%
  \if@twoside%
    \ifodd\thispageNum%
      \cleardoublepage%
    \else%
      \clearpage%
    \fi
  \else%
    \clearpage%
  \fi
}
%%%%%%%%%%%%%%%%%%%%%%%%%%%%%%%%%%%%%%%%%%%%%%
%%%%% DIMENSION %%%%%%%%%%%%%%%%%%%%%%%%%%%%%%
%%%%%%%%%%%%%%%%%%%%%%%%%%%%%%%%%%%%%%%%%%%%%%
\ifluatex
  \usepackage{luatexja}
  \ltjsetparameter{kanjiskip=0.0pt plus 0.4pt minus 0.5pt}
  \ltjsetparameter{xkanjiskip=2.40555pt plus 1.0pt minus 1.0pt}
  \newcommand{\hk}{\hspace{\ltjgetparameter{kanjiskip}}}
  \newcommand{\hx}{\hspace{\ltjgetparameter{xkanjiskip}}}
\else
  \setlength{\kanjiskip}{0.0pt plus 0.4pt minus 0.5pt}
  \setlength{\xkanjiskip}{2.40555pt plus 1.0pt minus 1.0pt}
  \newcommand{\hk}{\hspace{\kanjiskip}}
  \newcommand{\hx}{\hspace{\xkanjiskip}}
\fi
%%%%%%%%%%%%%%%%%%%%%%%%%%%%%%%%%%%%%%%%%%%%%%
%%%%% CDOTFILL LIKE TOC %%%%%%%%%%%%%%%%%%%%%%
%%%%%%%%%%%%%%%%%%%%%%%%%%%%%%%%%%%%%%%%%%%%%%
%\newcommand\cdotfill{%
%  \leaders\hbox{$\m@th\mkern\@dotsep mu\hbox{$\cdot$}\mkern \@dotsep mu$}\hfill\kern\z@
%}
%%%%%%%%%%%%%%%%%%%%%%%%%%%%%%%%%%%%%%%%%%%%%%
%%%%% COLOR %%%%%%%%%%%%%%%%%%%%%%%%%%%%%%%%%%
%%%%%%%%%%%%%%%%%%%%%%%%%%%%%%%%%%%%%%%%%%%%%%
\definecolor{ai}     {rgb}{0.2039, 0.3765, 0.4314}
\definecolor{kon}    {rgb}{0.0000, 0.2000, 0.4000}
\definecolor{konpeki}{rgb}{0.0902, 0.5098, 0.7333}
\definecolor{moegi}  {rgb}{0.3020, 0.5961, 0.1882}
\definecolor{sssec}  {rgb}{0.7333, 0.5, 0.7333}
\definecolor{sora}   {rgb}{0.1451, 0.7216, 0.8039}
\definecolor{sumire} {rgb}{0.3882, 0.2157, 0.5922}
\definecolor{wwqqcc} {rgb}{0.4, 0, 0.8}
\definecolor{qqzzqq} {rgb}{0, 0.6, 0}
\definecolor{ffwwqq} {rgb}{1, 0.4, 0}
%%%%%%%%%%%%%%%%%%%%%%%%%%%%%%%%%%%%%%%%%%%%%%
%%%%% REF %%%%%%%%%%%%%%%%%%%%%%%%%%%%%%%%%%%%
%%%%%%%%%%%%%%%%%%%%%%%%%%%%%%%%%%%%%%%%%%%%%%
%%%%% AUTOREFNAME %%%%%%%%%%%%%%%%%%%%%%%%%%%%
%\newcommand{\subfigureautorefname}{\figureautorefname} % subfigure --> figure
%\newcommand{\subtableautorefname}{\tableautorefname}
%%%%% PAGEREF %%%%%%%%%%%%%%%%%%%%%%%%%%%%%%%%%%%
\newcommand{\pageautoref}[1]{%
  \ifthenelse{\equal{\pageref{#1}}{\thepage}}%
    {\autoref{#1}}%
    {\autoref{#1}~[p.\pageref{#1}]}%
}
\newcommand{\pageeqref}[1]{%
  \ifthenelse{\equal{\pageref{#1}}{\thepage}}%
    {\eqref{#1}}%
    {\eqref{#1}~[p.\pageref{#1}]}%
}
%%%%%%%%%%%%%%%%%%%%%%%%%%%%%%%%%%%%%%%%%%%%%%
%%%%% FOR LISTINGS %%%%%%%%%%%%%%%%%%%%%%%%%%%
%%%%%%%%%%%%%%%%%%%%%%%%%%%%%%%%%%%%%%%%%%%%%%
%%%%% CAPTOINOF %%%%%%%%%%%%%%%%%%%%%%%%%%%%%%
\setlength{\abovecaptionskip}{0pt}
\newcommand{\modcaptionof}[2]{%
  \captionof{#1}{%
    \csname #1name\endcsname\thechapter.%
    \ifnum\value{#1}<10 0\fi
      \arabic{#1}. #2}}
%%%%%%%%%%%%%%%%%%%%%%%%%%%%%%%%%%%%%%%%%%%%%%
%%%%% FOR REFERENCES %%%%%%%%%%%%%%%%%%%%%%%%%
%%%%%%%%%%%%%%%%%%%%%%%%%%%%%%%%%%%%%%%%%%%%%%
\newcommand{\Articlename}{論文}
\newcommand{\Bookname}{書籍}
\newcommand{\OnlineSourcename}{ウェブサイト}
\defbibheading{articles}[\Articlename]{\section*{#1}}
\defbibheading{online}[\OnlineSourcename]{\section*{#1}}
%%%%% DECLAREFIELDFORMAT %%%%%%%%%%%%%%%%%%%%
\DeclareFieldFormat{urldate}{%
  (\textbf{urlseen~}\thefield{urlyear}/\ifnum\thefield{urlmonth}<10 0\fi\thefield{urlmonth})%
}
%%%%%%%%%%%%%%%%%%%%%%%%%%%%%%%%%%%%%%%%%%%%%%
%%%%% FOR TOC %%%%%%%%%%%%%%%%%%%%%%%%%%%%%%%%
%%%%%%%%%%%%%%%%%%%%%%%%%%%%%%%%%%%%%%%%%%%%%%
%%%%% TOC LINE %%%%%%%%%%%%%%%%%%%%%%%%%%%%%%%
\newcommand\PartSeparateline[1]{\addtocontents{#1}{\protect\par\protect\hrulefill\protect\par\protect\vspace*{-10pt}}}%
\newcommand\tocAPartSeparateline[3]{\addtocontents{#1}{\protect\par\protect\vspace*{#2}\protect\hrule width 0.5\linewidth\protect\par\protect\vspace*{#3}}}%
%\newcommand\tableAPartSeparateline{\tocAPartSeparateline{lot}{2pt}{2pt}}
%%%%% PART FOR APPENDIX %%%%%%%%%%%%%%%%%%%%%%%%%%%%%%%%%%%%%%%%%
\newcommand{\Appendixpart}{
  \clearrightpage
  \part*{\partname\ \thepart\hx の補遺\label{Apart:\thepart}}
  \addcontentsline{toc}{part}{\partname\ \thepart\hx の補遺}
}
%%%%%%%%%%%%%%%%%%%%%%%%%%%%%%%%%%%%%%%%%%%%%%
%%%%% AUTO LABELING %%%%%%%%%%%%%%%%%%%%%%%%%%
%%%%%%%%%%%%%%%%%%%%%%%%%%%%%%%%%%%%%%%%%%%%%%
%%%%% FOR CHAPTER OR APPENDIX %%%%%%%%%%%%%%%%%%%%%%%%%%%%%
\newcommand{\modHeadchapter}[2][]{%
  \clearrightpage\clearleftpage
  \let\newpage\relax
  \ifx\relax#1\relax
    \ifx\@chapapp\appendixname
      \chapter{#2\label{app:\thechapter}}%
    \else
      \chapter{#2\label{chap:\thechapter}}%
    \fi
  \else%
    \ifx\@chapapp\appendixname
      \chapter[#1]{#2\label{app:\thechapter}}%
    \else
      \chapter[#1]{#2\label{chap:\thechapter}}%
    \fi
  \fi
  \thispagestyle{main}%
  \indentspace%
  \let\newpage\tmpnewpage
}
%%%%% FOR SECTION %%%%%%%%%%%%%%%%%%%%%%%%%%%%%
%%%%% TO LABEL SECTION  %%%%%%
\newcommand{\modHeadsection}[2][]{%
  \ifx\relax#1\relax
    \section{#2\label{sec:\thesection}}%
  \else%
    \section[#1]{#2\label{sec:\thesection}}%
  \fi
}
%%%%%%%%%%%%%%%%%%%%%%%%%%%%%%%%%%%%%%%%%%%%%%
%%%%% NEWCOLORBOX %%%%%%%%%%%%%%%%%%%%%%%%%%%%
%%%%%%%%%%%%%%%%%%%%%%%%%%%%%%%%%%%%%%%%%%%%%%
\newcounter{GlobalFootnote}% difine a counter Global Footnote
%%%%% PART %%%%%
\newcommand{\tablePartnname}{\thepart}
\newtcolorbox[auto counter, number within=part]{tablePart}[2][]{Columnbox, title={\termblue{\tablePartnname:#2}}, #1, after title={}}
%%%%% COLUMN %%%%%
\newcommand{\Columnname}{Column}
\newtcolorbox[auto counter, number within=chapter, list inside=loC]{Column}[2][]{%
  Columnbox, title={#2}, #1, list entry={\numberline{\Columnname~\thetcbcounter}{\protect\hspace*{41pt}#2}}}
%\newcommand{\tcb@cnt@Columnautorefname}{Column}
%%%%% Formula %%%%%
\newcommand{\Formulaname}{Eq.}
\newtcolorbox[auto counter, number within=chapter]{Formula}[2][]{%
  Formulabox, title={#2}, #1}
\newcommand{\tcb@cnt@Formulaautorefname}{Eq.\!}
%%%%% FIGBOX %%%%%
\newtcolorbox{Figbox}[1][]{Figurebox, #1}
%%%%% TABBOX %%%%%
\newtcolorbox{Tabbox}[1][]{Tabularbox, #1}
%\renewcommand{\tableautorefname}{表}
%%%%% HOSOKU %%%%%
\newcommand{\hosokuname}{補}
\definecolor{hosoku}{cmyk}{0, 0, 0, .15}
\newtcolorbox[auto counter, number within=chapter]{hosoku}[1][]{hosokubox, #1}
\newcommand{\tcb@cnt@hosokuautorefname}{補足}
%%%%% TWOCTABLE %%%%%
\newtcolorbox[number within=chapter]{twoCtable}[2][]{twoCtablebox, title={#2}, #1}
%%%%% OTHER TIKZ DEFINITION %%%%%%%%%%%%%%%%%%
\def\pgfname{\textsc{pgf}}
\def\tikzname{Ti\textit{k}Z}
\tikzfading[name=fade ball, inner color=transparent!60, outer color=transparent!30]
\def\sball#1{\tikz \shade [ball color=#1, path fading=fade ball] (0,0) circle (.7ex);}
\def\terminal#1#2{\tikz[baseline=(a.base)] \node (a) [terminal, bottom color=#2] {\small #1};}
\def\termblue#1{\terminal{\color{blue}\fontsize{8pt}{0pt}\textbf{#1}}{gray!25}\hskip6pt}
%%%%%%%%%%%%%%%%%%%%%%%%%%%%%%%%%%%%%%%%%%%%%%
%%%%% DECLARE %%%%%%%%%%%%%%%%%%%%%%%%%%%%%%%%
%%%%%%%%%%%%%%%%%%%%%%%%%%%%%%%%%%%%%%%%%%%%%%
%%%%% DECLAREMATHOPERATOR %%%%%%%%%%%%%%%%%%%%
\DeclareMathOperator{\IP}{Im}
\DeclareMathOperator{\RP}{Re}
%%%%% DECLAREROBUSTCOMMAND %%%%%%%%%%%%%%%%%%%%
\DeclareRobustCommand{\bDiv}{\nonscript\mskip-\medmuskip\mkern5mu\mathbin
  {\operator@font div}\penalty900
  \mkern5mu\nonscript\mskip-\medmuskip}
\DeclareRobustCommand{\pod}[1]{\allowbreak
  \if@display\mkern18mu\else\mkern8mu\fi(#1)}
\DeclareRobustCommand{\pDiv}[1]{\pod{{\operator@font div}\mkern6mu#1}}
\DeclareRobustCommand{\Div}[1]{\allowbreak\if@display\mkern18mu
  \else\mkern12mu\fi{\operator@font div}\,\,#1}
%%%%% DECLARENEWTOC %%%%%%%%%%%%%%%%%%%%
\DeclareNewTOC[owner=\jobname, name=Part]{lop}
\DeclareNewTOC[owner=\jobname, name=Column]{loC}
%%%%% DECLARENEWLAYER %%%%%%%%%%%%%%%%%%%%
\newcommand{\setallpageWatermark}[4]{%
  \DeclareNewLayer[
    foreground,
    page,
    contents={%
      \begin{tikzpicture}[remember picture, overlay]
        \node[rotate=#2, scale=#3, anchor=center, opacity=#4, text=lightgray] at (current page.center){\scshape#1};%
      \end{tikzpicture}%
    }%
  ]{WatermarkLayer}
}
%%%%%%%%%%%%%%%%%%%%%%%%%%%%%%%%%%%%%%%%%%%%%%
%%%%% LINK %%%%%%%%%%%%%%%%%%%%%%%%%%%%%%%%%%%
%%%%%%%%%%%%%%%%%%%%%%%%%%%%%%%%%%%%%%%%%%%%%%
%%%%% LINK NAME %%%%%%%%%%%%%%%%%%
\newcommand{\linkLaTeX}{\href{https://www.latex-project.org/}{\LaTeX}}
\newcommand{\linkLaTeXProject}{\href{https://www.latex-project.org/}{\LaTeX\ Project}}
\newcommand{\linkAMS}{\href{https://www.latex-project.org/}{American Mathematical Society}}
\newcommand{\linkTeXLive}{\href{https://tug.org/texlive/}{\TeX\ Live}}
\newcommand{\linkTeXUsersGroup}{\href{http://www.tug.org/}{\TeX\ Users Group}}
\newcommand{\linkBibLaTeX}{\href{https://ctan.org/pkg/biblatex}{Bib\LaTeX}}
\newcommand{\linkBiber}{\href{https://ctan.org/pkg/biber}{Biber}}
\newcommand{\linkupmendex}{\href{https://ctan.org/pkg/upmendex}{upmendex}}
\newcommand{\linkPGFTikZ}{\href{https://github.com/pgf-tikz/pgf}{\pgfname/\tikzname}}
\newcommand{\linkTeXStudio}{\href{https://texstudio.org/}{\TeX\ Studio}}
\newcommand{\linkSumatraPDF}{\href{https://www.sumatrapdfreader.org/}{Sumatra PDF}}
\newcommand{\linkVSCode}{\href{https://code.visualstudio.com/}{VS Code}}
\newcommand{\linkExcel}{\href{https://www.microsoft.com/ja-jp/microsoft-365/excel}{Excel}}
\newcommand{\linkMicrosoftCopilot}{\href{https://www.bing.com/}{Microsoft Copilot}}
\newcommand{\linkMicrosoftCorp}{\href{https://www.microsoft.com/}{Microsoft Corporation}}
\newcommand{\linkPython}{\href{https://www.python.org/}{Python}}
\newcommand{\linkPythonSF}{\href{https://www.python.org/psf-landing/}{Python Software Foundation}}
\newcommand{\linkGitHub}{\href{https://github.com/}{GitHub}}
\newcommand{\linkGitHubDesktop}{\href{https://desktop.github.com/}{GitHub Desktop}}
\newcommand{\linkGitHubInc}{\href{https://github.com/}{GitHub, Inc}}
\newcommand{\linkDocker}{\href{https://www.docker.com/}{Docker}}
\newcommand{\linkDockerInc}{\href{https://www.docker.com/}{Docker, Inc}}
\newcommand{\linkUbuntu}{\href{https://ubuntu.com/}{Ubuntu}}
\newcommand{\linkCanonicalLtd}{\href{https://canonical.com/}{Canonical Ltd}}
\newcommand{\linkSQLite}{\href{https://www.sqlite.org/}{SQLite}}
\newcommand{\linkSQLiteConsortium}{\href{https://www.sqlite.org/consortium.html}{SQLite Consortium}}
\newcommand{\linkChatGPT}{\href{https://openai.com/chatgpt}{ChatGPT}}
\newcommand{\linkOpenAI}{\href{https://www.openai.com/}{OpenAI}}
%\newcommand\nextsectionlink[1]{\addtocounter{section}\@ne
%                               \hyperlink{section.\thechapter.\the\c@section}{#1}%
%                               \addtocounter{section}{-\@ne}}
%\newcommand\previoussectionlink[1]{\addtocounter{section}{-\@ne}
%                                   \hyperlink{section.\thechapter.\the\c@section}{#1}%
%                                   \addtocounter{section}{\@ne}}
%\newcommand\previouschapterlink[1]{\addtocounter{chapter}{-\@ne}
%                                   \hyperlink{chapter.\the\c@chapter}{#1}%
%                                   \addtocounter{chapter}{\@ne}}
%%%%%%%%%%%%%%%%%%%%%%%%%%%%%%%%%%%%%%%%%%%%%%
%%%%% OTHER DEFINITION %%%%%%%%%%%%%%%%%%%%%%%
%%%%%%%%%%%%%%%%%%%%%%%%%%%%%%%%%%%%%%%%%%%%%%
%%%%% TO GET CHAPTER TITLE %%%%%%
\newcommand\Chaptername{} % initialize \Chaptername
\let\old@chapter\@chapter
\def\@chapter[#1]#2{\gdef\Chaptername{#2}\old@chapter[#1]{#2}}
%%%%% TO GET SECTION TITLE %%%%%%
\newcommand\Sectionname{} % initialize \Sectionname
\let\Sectionmark\sectionmark
\def\sectionmark#1{\def\Sectionname{#1}\Sectionmark{#1}}
%%%%% TO PRG NAME %%%%%%
\newcommand\MainEx{O1916} % an example for main program
\newcommand\MXOThickness{O110001} % X外側中心計測
\newcommand\MYOThickness{O110002} % Y外側中心計測
\newcommand\MXOface{O120001} % 溝X基準面計測
\newcommand\MXIWidth{O130001} % X内側中心計測
\newcommand\MYIWidth{O130002} % Y内側中心計測
\newcommand\MXIface{O140001} % 外削X基準面計測
\newcommand\MYcenterline{O150002} % 通り芯Y
\newcommand\MXcenterline{O150003} % 通り芯X(Z測定)
\newcommand\DLone{O210003} % 内面溝用 レベル1
\newcommand\DLtwoAC{O220001} % 内面溝用 レベル2 AC
\newcommand\DLtwoBD{O220002} % 内面溝用 レベル2 BD
\newcommand\DMLthreeAC{O230001} % 内面溝 測定用 レベル3 AC
\newcommand\DMLthreeBD{O230002} % 内面溝 測定用 レベル3 BD
\newcommand\KTanmenRight{O410000} % 端面用 右回り
\newcommand\KGaisakuRLeft{O420000} % 外側 左回り
\newcommand\KMizoConerLeft{O430000} % 溝用 左回り
\newcommand\KSotoMentoriRLeft{O440000} % 外側面取用 左回り
\newcommand\KUchiMentoriRLeft{O450000} % 内側面取用 左回り
\newcommand\KOLeft{O490005} % 外 左回り
\newcommand\DKLthreeAC{O530001} % 内面溝 加工用 レベル3 AC
\newcommand\DKLthreeBD{O530002} % 内面溝 加工用 レベル3 BD
\newcommand\OpauseCheck{O900003} % タッチセンサーON
\newcommand\OsensorOn{O910001} % タッチセンサーON
\newcommand\OsensorOff{O910002} % タッチセンサーOFF
%%%%% MACHINING NAME %%%%%%%%%%%%%%%%%%
\newcommand{\DMname}{Dマシニング}
\newcommand{\MMname}{Mマシニング}
\newcommand{\dimple}{内面溝}
%%%%% OTHER DEFINITION %%%%%%%%%%%%%%%%%%
\newcommand\ttNum{\ifmmode{\text{\texttt\#}}\else\texttt\#\fi}
\newcommand\cf{{\itshape cf.\,}}
\newcommand{\TBW}{\texorpdfstring{\small{\color{red}\,*}}{}}
%%%%% FOR PART WITH TABLE %%%%%%%%%%%%%%%%%%
\newcommand{\tPart}[4][]{
  \begingroup
  \clearrightpage
%  \@openrightfalse
  \let\oldnewpage\newpage
  \let\newpage\relax
  \part{#2\label{part:\thepart}}
  \addcontentsline{lop}{part}{\protect\numberline{\thepart}#2}%
  \ifx#1\relax\else\addcontentsline{#1}{part}{\protect\numberline{\thepart}#2}\fi
  \let\newpage\oldnewpage
  \clearpage
%  \@openrightfalse
  \thispagestyle{emptydate}
  \vspace*{0.1\textheight}%
  \begin{tablePart}{#3}
  #4
  \end{tablePart}%
%  \@openrightfalse
  \@mainmattertrue\pagestyle{main}
  \endgroup
}
%%%%% NOTATION TABLE %%%%%%%%%%%%%%%%%%
\newenvironment{Notation}[2]
{%
  \rowcolors{3}{gray!10}{white}
  \setlength\cellspacetoplimit{4pt}
  \setlength\cellspacebottomlimit{4pt}
  \if\relax\detokenize{#1}\relax
  \else
    \captionsetup{justification=centering}
    \setlength{\abovecaptionskip}{-7pt}
    \modcaptionof{table}{#1} % 追加したキャプション
    \addtocounter{table}{-1}
  \fi
  \begin{longtable}{|c|Sl|c|}
  \hline
  \rowcolor{orange!20}
  \textbf{記号} & \textbf{内容}\hspace*{0.75\textwidth} & \textbf{#2}\\
  \hline
  \endfirsthead
  \hline
  \rowcolor{orange!20}
  \textbf{記号} & \textbf{内容} & \textbf{#2}\\
  \hline
  \endhead
  \hline
  \multicolumn{3}{|r|}{\scriptsize 次ページへ続く} \\
  \hline
  \endfoot
  \hline
  \endlastfoot
}
{%
  \end{longtable}
}