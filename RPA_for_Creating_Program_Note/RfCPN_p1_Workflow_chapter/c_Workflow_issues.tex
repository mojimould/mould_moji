%!TEX root = ../RPA_for_Creating_Program_Note.tex


現時点(\DMname 設置時点)における\MMname での作業について、特にソフトウェアの観点から問題点・改善可能な点をピックアップしていく。



%%%%%%%%%%%%%%%%%%%%%%%%%%%%%%%%%%%%%%%%%%%%%%%%%%%%%%%%%%
%% section 1.2 %%%%%%%%%%%%%%%%%%%%%%%%%%%%%%%%%%%%%%%%%%%
%%%%%%%%%%%%%%%%%%%%%%%%%%%%%%%%%%%%%%%%%%%%%%%%%%%%%%%%%%
\modHeadsection{機械による加工前の工程\TBW}
\MMname では、各々の芯出し計測の工程では「\index{みつびしマシニングセンタむじんかシステム@三菱マシニングセンタ無人化システム}三菱マシニングセンタ無人化システム」, 各々の加工工程では当社の従業員(一般職)により作成された\index{サブプログラム}サブプログラムを用いられている。
これらのサブプログラムを用いることにより、各々の明細における必要な寸法値等を引数に格納して用いればよい状態になっている。
\MMname において「プログラムを作成する」というのは、この具体的な引数等を入力していくことを意味している。

したがって、まず第一に、「該当する図面を用意して現場の作業員がプログラムを作成する」という時点で明らかな問題点であることがわかる。
さらに、これが問題点(改善可能な点)だとこれまで一切(スタッフ・管理職を含めて)認識されなかったことが(大きな)問題点として挙げられる。

つまり、(製造業にも関わらず)当社がソフトウェアエンジニアリングをあからさまに蔑ろにしてきたことが根本的な問題として存在する。
そのため、\index{ソフトウェアエンジニアリング}ソフトウェアエンジニアリングに対する当社の著しい能力の低さが露呈した形となっている。



%%%%%%%%%%%%%%%%%%%%%%%%%%%%%%%%%%%%%%%%%%%%%%%%%%%%%%%%%%
%% subsection 01.2.1 %%%%%%%%%%%%%%%%%%%%%%%%%%%%%%%%%%%%%
%%%%%%%%%%%%%%%%%%%%%%%%%%%%%%%%%%%%%%%%%%%%%%%%%%%%%%%%%%
\subsection{\TBW}
(to be written...)
