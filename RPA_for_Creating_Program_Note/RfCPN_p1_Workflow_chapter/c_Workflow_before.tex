%!TEX root = ../RPA_for_Creating_Program_Note.tex


モールドの端面加工・外削加工・溝加工等は主に三菱製横型マシニングセンタ(以下、\MMname)にて行われている。
新たに導入する横型マシニングセンタ(以下、\DMname)での工程は、\dimple の測定・加工を除けば\MMname と同様である。
そこで、まずは\MMname ではどのようなフローで業務が行われているかを(ソフトウェアの観点から)みることにする。

なお、ここでは\MMname の\expandafterindex{No.1パレット(\MMname)@No.1パレット(\MMname)}No.1パレットで加工を行うものを対象とする
%% footnote %%%%%%%%%%%%%%%%%%%%%
\footnote{\expandafterindex{No.2パレット(\MMname)@No.2パレット(\MMname)}No.2パレットでは、径の大きなものや丸形のもの等の加工が主に行われる。}。
%%%%%%%%%%%%%%%%%%%%%%%%%%%%%%%%%



%%%%%%%%%%%%%%%%%%%%%%%%%%%%%%%%%%%%%%%%%%%%%%%%%%%%%%%%%%
%% section 1.1 %%%%%%%%%%%%%%%%%%%%%%%%%%%%%%%%%%%%%%%%%%%
%%%%%%%%%%%%%%%%%%%%%%%%%%%%%%%%%%%%%%%%%%%%%%%%%%%%%%%%%%
\modHeadsection{\MMname における工程(加工前)}
\MMname において、ある明細のモールドを加工をする際に、以下のような流れで作業が行われる。


%%%%%%%%%%%%%%%%%%%%%%%%%%%%%%%%%%%%%%%%%%%%%%%%%%%%%%%%%%
%% subsection 01.1.1 %%%%%%%%%%%%%%%%%%%%%%%%%%%%%%%%%%%%%
%%%%%%%%%%%%%%%%%%%%%%%%%%%%%%%%%%%%%%%%%%%%%%%%%%%%%%%%%%
\subsection{図面の確認}
\begin{enumerate}
\item 対象となる明細の図面を用意する
\item 他に内容が類似する明細の図面があれば、それも併せて用意する
\end{enumerate}
%%%%%%%%%%%%%%%%%%%%%%%%%%%%%%%%%%%%%%%%%%%%%%%%%%%%%%%%%%
%% PARAMETER %%%%%%%%%%%%%%%%%%%%%%%%%%%%%%%%%%%%%%%%%%%%%
%%%%%%%%%%%%%%%%%%%%%%%%%%%%%%%%%%%%%%%%%%%%%%%%%%%%%%%%%%
\begin{Parameter}{必要なパラメータ}
図面の有無
\end{Parameter}
%%%%%%%%%%%%%%%%%%%%%%%%%%%%%%%%%%%%%%%%%%%%%%%%%%%%%%%%%%
%%%%%%%%%%%%%%%%%%%%%%%%%%%%%%%%%%%%%%%%%%%%%%%%%%%%%%%%%%
%%%%%%%%%%%%%%%%%%%%%%%%%%%%%%%%%%%%%%%%%%%%%%%%%%%%%%%%%%


%%%%%%%%%%%%%%%%%%%%%%%%%%%%%%%%%%%%%%%%%%%%%%%%%%%%%%%%%%
%% subsection 01.1.2 %%%%%%%%%%%%%%%%%%%%%%%%%%%%%%%%%%%%%
%%%%%%%%%%%%%%%%%%%%%%%%%%%%%%%%%%%%%%%%%%%%%%%%%%%%%%%%%%
\subsection{加工部分の有無の確認}

%%%%%%%%%%%%%%%%%%%%%%%%%%%%%%%%%%%%%%%%%%%%%%%%%%%%%%%%%%
%% subsubsection 01.1.2.2 %%%%%%%%%%%%%%%%%%%%%%%%%%%%%%%%
%%%%%%%%%%%%%%%%%%%%%%%%%%%%%%%%%%%%%%%%%%%%%%%%%%%%%%%%%%
\subsubsection{端面部分}
\index{たんめんかこう@端面加工}端面の加工については、全明細に共通の形で存在する。

%%%%%%%%%%%%%%%%%%%%%%%%%%%%%%%%%%%%%%%%%%%%%%%%%%%%%%%%%%
%% subsubsection 01.1.2.2 %%%%%%%%%%%%%%%%%%%%%%%%%%%%%%%%
%%%%%%%%%%%%%%%%%%%%%%%%%%%%%%%%%%%%%%%%%%%%%%%%%%%%%%%%%%
\subsubsection{外削部分}
\index{がいさくかこう@外削加工}外削の加工については、明細により外削の有無または形状の違いが存在する。
\begin{enumerate}
\item トップ側またはボトム側の\index{がいさく@外削}外削の有無を確認する
\item \index{がいさく@外削}外削の形状を確認し、使用する工具を決定する
\item \index{わんきょくにそったがいさく@湾曲に沿った外削}外削が湾曲に沿ったものかどうかも確認する
\end{enumerate}
%%%%%%%%%%%%%%%%%%%%%%%%%%%%%%%%%%%%%%%%%%%%%%%%%%%%%%%%%%
%% PARAMETER %%%%%%%%%%%%%%%%%%%%%%%%%%%%%%%%%%%%%%%%%%%%%
%%%%%%%%%%%%%%%%%%%%%%%%%%%%%%%%%%%%%%%%%%%%%%%%%%%%%%%%%%
\begin{Parameter}{必要なパラメータ}
トップ外削の有無・ボトム外削の有無・トップ外削の形状・ボトム外削の形状
\end{Parameter}
%%%%%%%%%%%%%%%%%%%%%%%%%%%%%%%%%%%%%%%%%%%%%%%%%%%%%%%%%%
%%%%%%%%%%%%%%%%%%%%%%%%%%%%%%%%%%%%%%%%%%%%%%%%%%%%%%%%%%
%%%%%%%%%%%%%%%%%%%%%%%%%%%%%%%%%%%%%%%%%%%%%%%%%%%%%%%%%%

\clearpage
%%%%%%%%%%%%%%%%%%%%%%%%%%%%%%%%%%%%%%%%%%%%%%%%%%%%%%%%%%
%% subsubsection 01.1.2.3 %%%%%%%%%%%%%%%%%%%%%%%%%%%%%%%%
%%%%%%%%%%%%%%%%%%%%%%%%%%%%%%%%%%%%%%%%%%%%%%%%%%%%%%%%%%
\subsubsection{溝部分}
\index{みぞかこう@溝加工}溝の加工については、全明細に存在し、明細により形状の違いが存在する。
\begin{enumerate}
\item \index{みぞ@溝}溝の形状を確認し、使用する\index{サブプログラム}サブプログラムの判断を行う
\item \index{みぞはば@溝幅}溝幅を確認し、使用する工具の判断を行う
\end{enumerate}
%%%%%%%%%%%%%%%%%%%%%%%%%%%%%%%%%%%%%%%%%%%%%%%%%%%%%%%%%%
%% PARAMETER %%%%%%%%%%%%%%%%%%%%%%%%%%%%%%%%%%%%%%%%%%%%%
%%%%%%%%%%%%%%%%%%%%%%%%%%%%%%%%%%%%%%%%%%%%%%%%%%%%%%%%%%
\begin{Parameter}{必要なパラメータ}
溝の形状・トップ外削の有無・溝幅
\end{Parameter}
%%%%%%%%%%%%%%%%%%%%%%%%%%%%%%%%%%%%%%%%%%%%%%%%%%%%%%%%%%
%%%%%%%%%%%%%%%%%%%%%%%%%%%%%%%%%%%%%%%%%%%%%%%%%%%%%%%%%%
%%%%%%%%%%%%%%%%%%%%%%%%%%%%%%%%%%%%%%%%%%%%%%%%%%%%%%%%%%

%\clearpage
%%%%%%%%%%%%%%%%%%%%%%%%%%%%%%%%%%%%%%%%%%%%%%%%%%%%%%%%%%
%% subsubsection 01.1.2.4 %%%%%%%%%%%%%%%%%%%%%%%%%%%%%%%%
%%%%%%%%%%%%%%%%%%%%%%%%%%%%%%%%%%%%%%%%%%%%%%%%%%%%%%%%%%
\subsubsection{端面の面取部分}
\index{たんめん@端面}端面の\index{めんとりかこう@面取加工}面取の加工については、全明細に存在し、明細により形状の違いが存在する。
\begin{enumerate}
\item 面取がC面取であれば、マシニングによる加工を行うか判断を行う
\item \index{Cめんとり(たんめん)@C面取(端面)}C面取の角度を確認し、使用する工具を決定する
\end{enumerate}

%\clearpage
%%%%%%%%%%%%%%%%%%%%%%%%%%%%%%%%%%%%%%%%%%%%%%%%%%%%%%%%%%
%% subsubsection 01.1.2.4 %%%%%%%%%%%%%%%%%%%%%%%%%%%%%%%%
%%%%%%%%%%%%%%%%%%%%%%%%%%%%%%%%%%%%%%%%%%%%%%%%%%%%%%%%%%
\subsubsection{端面の座ぐり部分\TBW}
(to be written...)


%\clearpage
%%%%%%%%%%%%%%%%%%%%%%%%%%%%%%%%%%%%%%%%%%%%%%%%%%%%%%%%%%
%% subsection 01.1.3 %%%%%%%%%%%%%%%%%%%%%%%%%%%%%%%%%%%%%
%%%%%%%%%%%%%%%%%%%%%%%%%%%%%%%%%%%%%%%%%%%%%%%%%%%%%%%%%%
\subsection{加工部分の寸法の確認}

%%%%%%%%%%%%%%%%%%%%%%%%%%%%%%%%%%%%%%%%%%%%%%%%%%%%%%%%%%
%% subsubsection 01.1.3.1 %%%%%%%%%%%%%%%%%%%%%%%%%%%%%%%%
%%%%%%%%%%%%%%%%%%%%%%%%%%%%%%%%%%%%%%%%%%%%%%%%%%%%%%%%%%
\subsubsection{端面における寸法}
\begin{enumerate}
\item \index{ぜんちょう(ワーク)@全長(ワーク)}全長の\index{こうさ(ぜんちょう)@公差(全長)}公差を確認し、\index{トップふりわけちょう@トップ振分長}トップ振分長および\index{ボトムふりわけちょう@ボトム振分長}ボトム振分長の\index{こうさ(ふりわけちょう)@公差(振分長)}公差の判断を行う
\item トップ側・ボトム側の\index{ふりわけちょう@振分長}振分長を確認し、\index{スペーサ}スペーサによる調整が必要か判断を行う
\item 使用するスペーサおよび再振分長は、専用の計算プログラム(\index{Excel VBA}Excel VBA)を用いて決定する
\item \index{がいけい@外径}外径・端面部の\index{にくあつ@肉厚}肉厚・\index{コーナー(たんめん)@コーナー(端面)}コーナーRの大きさを確認し、それに応じて\index{こうぐけいほせいち@工具径補正値}工具径補正値を決定する
\end{enumerate}
%%%%%%%%%%%%%%%%%%%%%%%%%%%%%%%%%%%%%%%%%%%%%%%%%%%%%%%%%%
%% PARAMETER %%%%%%%%%%%%%%%%%%%%%%%%%%%%%%%%%%%%%%%%%%%%%
%%%%%%%%%%%%%%%%%%%%%%%%%%%%%%%%%%%%%%%%%%%%%%%%%%%%%%%%%%
\begin{Parameter}{必要なパラメータ}
\paragraph*{再振分長}
全長・トップ振分長・ボトム振分長・AC外径・ジグの長さ・受板の幅
\tcbline*
\paragraph*{トップ端面}
トップ再振分長・AC外径・BD外径・トップ端外側コーナーR・トップ端AC内径・トップ端BD内径
\tcbline*
\paragraph*{ボトム端面}
ボトム再振分長・AC外径・BD外径・ボトム端外側コーナーR・ボトム端AC内径・ボトム端BD内径
\end{Parameter}
%%%%%%%%%%%%%%%%%%%%%%%%%%%%%%%%%%%%%%%%%%%%%%%%%%%%%%%%%%
%%%%%%%%%%%%%%%%%%%%%%%%%%%%%%%%%%%%%%%%%%%%%%%%%%%%%%%%%%
%%%%%%%%%%%%%%%%%%%%%%%%%%%%%%%%%%%%%%%%%%%%%%%%%%%%%%%%%%

\clearpage
%%%%%%%%%%%%%%%%%%%%%%%%%%%%%%%%%%%%%%%%%%%%%%%%%%%%%%%%%%
%% subsubsection 01.1.3.2 %%%%%%%%%%%%%%%%%%%%%%%%%%%%%%%%
%%%%%%%%%%%%%%%%%%%%%%%%%%%%%%%%%%%%%%%%%%%%%%%%%%%%%%%%%%
\subsubsection{外削における寸法}
\begin{enumerate}
\item \index{がいさくちょう@外削長}外削長と\index{みぞ@溝}溝の位置を確認し、実際に加工する外削の長さの判断を行う
\item トップ側・ボトム側の両方に外削のある場合は、どちら側が基準であるのかを確認する
\item 端面における内径・端面におけるA側肉厚・めっき膜厚から、外削中心$X$位置用のパラメータを手動による計算で決定する
\item 湾曲に沿った外削の場合は、\index{かたむきかく(がいさく)@傾き角(外削)}傾き角を手動による計算で決定する
\end{enumerate}
%%%%%%%%%%%%%%%%%%%%%%%%%%%%%%%%%%%%%%%%%%%%%%%%%%%%%%%%%%
%% PARAMETER %%%%%%%%%%%%%%%%%%%%%%%%%%%%%%%%%%%%%%%%%%%%%
%%%%%%%%%%%%%%%%%%%%%%%%%%%%%%%%%%%%%%%%%%%%%%%%%%%%%%%%%%
\begin{Parameter}{必要なパラメータ}
\paragraph*{ボトム側の外削のみの場合}
ボトム外削AC径・ボトム外削BD径・ボトム端AC内径・ボトム端A側肉厚・めっき膜厚・ボトム外削長
\tcbline*
\paragraph*{トップ側の外削のみの場合}
トップ外削AC径・トップ外削BD径・トップ端AC内径・トップ端A側肉厚・めっき膜厚・トップ外削長・溝位置・溝幅
\tcbline*
\paragraph*{両方に外削があり、ボトム側が基準の場合}
ボトム外削AC径・ボトム外削BD径・ボトム端AC内径・ボトム端A側肉厚・めっき膜厚・ボトム外削長・
トップ外削AC径・トップ外削BD径・トップ外削長・溝位置・溝幅・
通り芯
\tcbline*
\paragraph*{両方に外削があり、トップ側が基準の場合}
トップ外削AC径・トップ外削BD径・トップ端AC内径・トップ端A側肉厚・トップ外削長・溝位置・溝幅・
ボトム外削AC径・ボトム外削BD径・めっき膜厚・ボトム外削長・
通り芯
\tcbline*
\paragraph*{湾曲に沿った外削の場合}
(以上に加えて)中心湾曲
\end{Parameter}
%%%%%%%%%%%%%%%%%%%%%%%%%%%%%%%%%%%%%%%%%%%%%%%%%%%%%%%%%%
%%%%%%%%%%%%%%%%%%%%%%%%%%%%%%%%%%%%%%%%%%%%%%%%%%%%%%%%%%
%%%%%%%%%%%%%%%%%%%%%%%%%%%%%%%%%%%%%%%%%%%%%%%%%%%%%%%%%%

%\clearpage
%%%%%%%%%%%%%%%%%%%%%%%%%%%%%%%%%%%%%%%%%%%%%%%%%%%%%%%%%%
%% subsubsection 01.1.3.3 %%%%%%%%%%%%%%%%%%%%%%%%%%%%%%%%
%%%%%%%%%%%%%%%%%%%%%%%%%%%%%%%%%%%%%%%%%%%%%%%%%%%%%%%%%%
\subsubsection{溝における寸法}
\begin{enumerate}
\item 溝の形状を確認し、必要に応じて加工における径の決定する
\item \index{みぞちゅうしん@溝中心}溝中心の基準を確認し、溝中心の$X$位置を手動で計算し、決定する
\item \index{みぞはば@溝幅}溝幅を確認し、加工の回数を決定する
\end{enumerate}
%%%%%%%%%%%%%%%%%%%%%%%%%%%%%%%%%%%%%%%%%%%%%%%%%%%%%%%%%%
%% PARAMETER %%%%%%%%%%%%%%%%%%%%%%%%%%%%%%%%%%%%%%%%%%%%%
%%%%%%%%%%%%%%%%%%%%%%%%%%%%%%%%%%%%%%%%%%%%%%%%%%%%%%%%%%
\begin{Parameter}{必要なパラメータ}
\paragraph*{湾曲中心が基準の場合}
溝AC径・溝BD径・溝位置・溝幅・中心湾曲・溝コーナーRまたはC
\tcbline*
\paragraph*{外削中心が基準の場合}
溝AC径・溝BD径・溝位置・溝幅・溝コーナーRまたはC
\tcbline*
\paragraph*{A側溝深さが基準の場合}
(以上に加えて)A側溝深さ
\end{Parameter}
%%%%%%%%%%%%%%%%%%%%%%%%%%%%%%%%%%%%%%%%%%%%%%%%%%%%%%%%%%
%%%%%%%%%%%%%%%%%%%%%%%%%%%%%%%%%%%%%%%%%%%%%%%%%%%%%%%%%%
%%%%%%%%%%%%%%%%%%%%%%%%%%%%%%%%%%%%%%%%%%%%%%%%%%%%%%%%%%

%%%%%%%%%%%%%%%%%%%%%%%%%%%%%%%%%%%%%%%%%%%%%%%%%%%%%%%%%%
%% subsubsection 01.1.3.4 %%%%%%%%%%%%%%%%%%%%%%%%%%%%%%%%
%%%%%%%%%%%%%%%%%%%%%%%%%%%%%%%%%%%%%%%%%%%%%%%%%%%%%%%%%%
\subsubsection{面取における寸法}
\begin{enumerate}
\item 外側C面取の場合は、外削の有無を確認し、加工の径を決定する
\end{enumerate}
%%%%%%%%%%%%%%%%%%%%%%%%%%%%%%%%%%%%%%%%%%%%%%%%%%%%%%%%%%
%% PARAMETER %%%%%%%%%%%%%%%%%%%%%%%%%%%%%%%%%%%%%%%%%%%%%
%%%%%%%%%%%%%%%%%%%%%%%%%%%%%%%%%%%%%%%%%%%%%%%%%%%%%%%%%%
\begin{Parameter}{必要なパラメータ}
\paragraph*{トップ端外側C面取}
AC外径またはトップ外削AC径・BD外径またはトップ外削BD径・トップ端外側C面取長・外径またはトップ端外削コーナーR
\tcbline*
\paragraph*{ボトム端外側C面取}
AC外径またはボトム外削AC径・BD外径またはボトム外削BD径・ボトム端外側C面取長・外径またはボトム端外削コーナーR
\tcbline*
\paragraph*{トップ端内側C面取}
トップ端AC内径・トップ端BD内径・めっき膜厚・トップ端内側C面取長・トップ端内径コーナーR
\tcbline*
\paragraph*{ボトム端内側C面取}
ボトム端AC内径・ボトム端BD内径・めっき膜厚・ボトム端内側C面取長・ボトム端内径コーナーR
\end{Parameter}
%%%%%%%%%%%%%%%%%%%%%%%%%%%%%%%%%%%%%%%%%%%%%%%%%%%%%%%%%%
%%%%%%%%%%%%%%%%%%%%%%%%%%%%%%%%%%%%%%%%%%%%%%%%%%%%%%%%%%
%%%%%%%%%%%%%%%%%%%%%%%%%%%%%%%%%%%%%%%%%%%%%%%%%%%%%%%%%%

\clearpage
%%%%%%%%%%%%%%%%%%%%%%%%%%%%%%%%%%%%%%%%%%%%%%%%%%%%%%%%%%
%% subsubsection 01.1.3.4 %%%%%%%%%%%%%%%%%%%%%%%%%%%%%%%%
%%%%%%%%%%%%%%%%%%%%%%%%%%%%%%%%%%%%%%%%%%%%%%%%%%%%%%%%%%
\subsubsection{座ぐりにおける寸法\TBW}
(to be written...)


%\clearpage
%%%%%%%%%%%%%%%%%%%%%%%%%%%%%%%%%%%%%%%%%%%%%%%%%%%%%%%%%%
%% subsection 01.1.4 %%%%%%%%%%%%%%%%%%%%%%%%%%%%%%%%%%%%%
%%%%%%%%%%%%%%%%%%%%%%%%%%%%%%%%%%%%%%%%%%%%%%%%%%%%%%%%%%
\subsection{プログラムの入力}
\begin{enumerate}
\item 原則として、\index{プログラムばんごう@プログラム番号}プログラム番号は\index{せいひんばんごう@製品番号}製品番号と一致させる\\
ただし、加工内容が同一のものである場合は、既存のプログラムをそのまま流用する
\item 各々の加工部分およびその形状に対する\index{サブプログラム}サブプログラムを決定する
\item 各々のサブプログラムに対し、適切な寸法値を手動で計算する
\item 各々のサブプログラムに対し、計算した寸法値・\index{こうぐばんごう@工具番号}工具番号・\index{おくりはやさ@送り速さ}送り速さ・\index{しゅじくかいてんすう@主軸回転数}主軸回転数を格納する
\item \index{さいふりわけちょう@再振分長}再振分長の寸法に応じて、\index{Zほうこうくりあらんすへいめん@$Z$方向クリアランス平面}$Z$方向クリアランス平面の位置を決定する
\item マシニングにてメインプログラムを直接編集し、必要なコードまたは数値を記入する
\item 必要に応じて、一時停止用のコードを代入する
\item 工具径または工具長の補正が必要な場合は、別途専用画面にて編集を行う
\end{enumerate}


%\clearpage
%%%%%%%%%%%%%%%%%%%%%%%%%%%%%%%%%%%%%%%%%%%%%%%%%%%%%%%%%%
%% subsection 01.1.4 %%%%%%%%%%%%%%%%%%%%%%%%%%%%%%%%%%%%%
%%%%%%%%%%%%%%%%%%%%%%%%%%%%%%%%%%%%%%%%%%%%%%%%%%%%%%%%%%
\subsection{ワークの設置}
\begin{enumerate}
\item \index{スペーサ}スペーサが必要な場合は、適切なスペーサをジグの受板に設置する
\item ワークの大きさを考慮して、\index{ワークこていようボルト@ワーク固定用ボルト}ワーク固定用ボルトの長さを適宜決定し、ジグに設置する
\item \index{さいふりわけちょう@再振分長}再振分長に応じた位置にワークを設置し、固定する
\item トップ側およびボトム側の、ジグからの\index{はりだしちょう@張出長}張出長を\index{メジャー}メジャーを用いて測定する
\item 測定した張出長から、端面加工における\index{ぜんけずりしろ(たんめん)@全削り代(端面)}全削り代を手動でおおまかに計算する
\end{enumerate}


%\clearpage
%%%%%%%%%%%%%%%%%%%%%%%%%%%%%%%%%%%%%%%%%%%%%%%%%%%%%%%%%%
%% subsection 01.1.5 %%%%%%%%%%%%%%%%%%%%%%%%%%%%%%%%%%%%%
%%%%%%%%%%%%%%%%%%%%%%%%%%%%%%%%%%%%%%%%%%%%%%%%%%%%%%%%%%
\subsection{ワーク設置後の調整}
\begin{enumerate}
\item トップ側およびボトム側の全削り代に応じて、端面加工の回数を設定する
\item トップ端およびボトム端の外径中心の位置を\index{メジャー}メジャーで測定する
\item 測定した中心位置を用いて、ワーク座標系原点の設定を行う
\end{enumerate}
これらの設定は、\index{メインプログラム}メインプログラムを直接手動で編集する形で行われる。



\clearpage
%%%%%%%%%%%%%%%%%%%%%%%%%%%%%%%%%%%%%%%%%%%%%%%%%%%%%%%%%%
%% section 1.2 %%%%%%%%%%%%%%%%%%%%%%%%%%%%%%%%%%%%%%%%%%%
%%%%%%%%%%%%%%%%%%%%%%%%%%%%%%%%%%%%%%%%%%%%%%%%%%%%%%%%%%
\modHeadsection{\MMname における工程(加工中)}


%%%%%%%%%%%%%%%%%%%%%%%%%%%%%%%%%%%%%%%%%%%%%%%%%%%%%%%%%%
%% subsection 01.2.1 %%%%%%%%%%%%%%%%%%%%%%%%%%%%%%%%%%%%%
%%%%%%%%%%%%%%%%%%%%%%%%%%%%%%%%%%%%%%%%%%%%%%%%%%%%%%%%%%
\subsection{芯出し測定後}
\begin{enumerate}
\item 各々のワーク座標系原点の測定後、必要に応じて\index{ワークざひょうけいげんてん@ワーク座標系原点}ワーク座標系原点の設定変更を行う
\item 各々の測定箇所の$Z$位置の変更を、必要に応じて行う
\end{enumerate}
これらの設定は、\index{メインプログラム}メインプログラムを直接手動で編集する形で行われる。


%%%%%%%%%%%%%%%%%%%%%%%%%%%%%%%%%%%%%%%%%%%%%%%%%%%%%%%%%%
%% subsection 01.2.1 %%%%%%%%%%%%%%%%%%%%%%%%%%%%%%%%%%%%%
%%%%%%%%%%%%%%%%%%%%%%%%%%%%%%%%%%%%%%%%%%%%%%%%%%%%%%%%%%
\subsection{端面加工中}
\begin{enumerate}
\item 必要に応じて、\index{1かいあたりのけずりしろ(たんめん)@1回あたりの削り代(端面)}1回あたりの削り代を調整する
\end{enumerate}


%%%%%%%%%%%%%%%%%%%%%%%%%%%%%%%%%%%%%%%%%%%%%%%%%%%%%%%%%%
%% subsection 01.2.1 %%%%%%%%%%%%%%%%%%%%%%%%%%%%%%%%%%%%%
%%%%%%%%%%%%%%%%%%%%%%%%%%%%%%%%%%%%%%%%%%%%%%%%%%%%%%%%%%
\subsection{外削加工中}
\begin{enumerate}
\item 必要に応じて\index{しあげかこう(がいさく)@仕上加工(外削)}仕上加工前に\index{いちじていし@一時停止}一時停止を行い、\index{Aがわにくあつ@A側肉厚}A側肉厚および\index{がいさくけい@外削径}外削径の測定を行う
\item A側肉厚を調整する場合は、該当する芯出し測定部分のパラメータをメインプログラムから直接手動で編集する
\item 外削径を調整する場合は、該当する加工部分のパラメータをメインプログラムから直接手動で編集する
\item 加工の回数を変更する場合は、該当する加工部分をメインプログラムから直接手動で編集する
\end{enumerate}


%%%%%%%%%%%%%%%%%%%%%%%%%%%%%%%%%%%%%%%%%%%%%%%%%%%%%%%%%%
%% subsection 01.2.1 %%%%%%%%%%%%%%%%%%%%%%%%%%%%%%%%%%%%%
%%%%%%%%%%%%%%%%%%%%%%%%%%%%%%%%%%%%%%%%%%%%%%%%%%%%%%%%%%
\subsection{溝加工中}
\begin{enumerate}
\item 必要に応じて仕上加工前に一時停止を行い、A側溝深さおよび溝径の測定を行う
\item A側溝深さを調整する場合は、該当する芯出し測定部分のパラメータをメインプログラムから直接手動で編集する
\item 溝径を調整する場合は、該当する加工部分のパラメータをメインプログラムから直接手動で編集する
\item 加工の回数を変更する場合は、該当する加工部分をメインプログラムから直接手動で編集する
\item 必要に応じて、\index{ブロックゲージ}ブロックゲージによる溝幅の測定を行う
\end{enumerate}


%%%%%%%%%%%%%%%%%%%%%%%%%%%%%%%%%%%%%%%%%%%%%%%%%%%%%%%%%%
%% subsection 01.2.1 %%%%%%%%%%%%%%%%%%%%%%%%%%%%%%%%%%%%%
%%%%%%%%%%%%%%%%%%%%%%%%%%%%%%%%%%%%%%%%%%%%%%%%%%%%%%%%%%
\subsection{外側C面取加工中}
\begin{enumerate}
\item 必要に応じて仕上加工前に一時停止を行い、C面取の測定・位置の確認を行う
\item C面取の位置を調整する場合は、該当する加工部分のパラメータをメインプログラムから直接手動で編集する
\item 加工の回数を変更する場合は、該当する加工部分をメインプログラムから直接手動で編集する
\end{enumerate}


%%%%%%%%%%%%%%%%%%%%%%%%%%%%%%%%%%%%%%%%%%%%%%%%%%%%%%%%%%
%% subsection 01.2.1 %%%%%%%%%%%%%%%%%%%%%%%%%%%%%%%%%%%%%
%%%%%%%%%%%%%%%%%%%%%%%%%%%%%%%%%%%%%%%%%%%%%%%%%%%%%%%%%%
\subsection{内側C面取加工中}
\begin{enumerate}
\item 必要に応じて仕上加工前に一時停止を行い、C面取の測定・位置の確認を行う
\item C面取の位置を調整する場合は、該当する加工部分のパラメータをメインプログラムから直接手動で編集する
\item 加工の回数を変更する場合は、該当する加工部分をメインプログラムから直接手動で編集する
\end{enumerate}


\clearpage
%%%%%%%%%%%%%%%%%%%%%%%%%%%%%%%%%%%%%%%%%%%%%%%%%%%%%%%%%%
%% subsection 01.2.1 %%%%%%%%%%%%%%%%%%%%%%%%%%%%%%%%%%%%%
%%%%%%%%%%%%%%%%%%%%%%%%%%%%%%%%%%%%%%%%%%%%%%%%%%%%%%%%%%
\subsection{座ぐり加工中\TBW}
(to be written...)



\clearpage
%%%%%%%%%%%%%%%%%%%%%%%%%%%%%%%%%%%%%%%%%%%%%%%%%%%%%%%%%%
%% section 01.3 %%%%%%%%%%%%%%%%%%%%%%%%%%%%%%%%%%%%%%%%%%
%%%%%%%%%%%%%%%%%%%%%%%%%%%%%%%%%%%%%%%%%%%%%%%%%%%%%%%%%%
\modHeadsection{\MMname における工程(加工後)}


%%%%%%%%%%%%%%%%%%%%%%%%%%%%%%%%%%%%%%%%%%%%%%%%%%%%%%%%%%
%% subsection 01.3.1 %%%%%%%%%%%%%%%%%%%%%%%%%%%%%%%%%%%%%
%%%%%%%%%%%%%%%%%%%%%%%%%%%%%%%%%%%%%%%%%%%%%%%%%%%%%%%%%%
\subsection{ワークの取外し}
\begin{enumerate}
\item 必要に応じて、\index{ワークこていようボルト@ワーク固定用ボルト}ワーク固定用ボルトを緩める前に、計測器で寸法を確認する
\item 固定用ボルトを緩め、ワークを取り出し、軽く洗浄(拭取り)を行う
\item \index{リフター}リフターまたは\index{クレーン}クレーンを用いて、面取り作業台にワークを移動する
\end{enumerate}


%%%%%%%%%%%%%%%%%%%%%%%%%%%%%%%%%%%%%%%%%%%%%%%%%%%%%%%%%%
%% subsection 01.3.2 %%%%%%%%%%%%%%%%%%%%%%%%%%%%%%%%%%%%%
%%%%%%%%%%%%%%%%%%%%%%%%%%%%%%%%%%%%%%%%%%%%%%%%%%%%%%%%%%
\subsection{外観の確認・寸法の検査}
\begin{enumerate}
\item 外観に異常がないか確認を行う
\item 計測器を用いて寸法の確認を行う
\item 所定の用紙に、指定箇所の公差を考慮した寸法値を、手動で計算を行い記入する
\item 必要に応じて、所定の用紙に計測した値の記入を行う
\end{enumerate}


%%%%%%%%%%%%%%%%%%%%%%%%%%%%%%%%%%%%%%%%%%%%%%%%%%%%%%%%%%
%% subsection 01.3.3 %%%%%%%%%%%%%%%%%%%%%%%%%%%%%%%%%%%%%
%%%%%%%%%%%%%%%%%%%%%%%%%%%%%%%%%%%%%%%%%%%%%%%%%%%%%%%%%%
\subsection{手動による仕上げ加工}
\begin{enumerate}
\item 所定の寸法の面取りを、\index{てもちけんまき@手持ち研磨機}手持ち研磨機を用いて手動で行う
\item ばり等を除去を、やすりを用いて全体的に手動で行う
\item 端面の肉厚に応じて刻印の大きさを決定し、所定の刻印を手動で打ち込む
\item リフターまたはクレーンを用いて、所定の置き場に移動する
\end{enumerate}

