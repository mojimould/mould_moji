%!TEX root = ./RPA_for_Creating_Program_Note.tex


\modHeadchapter{計算の必要な数値計算(サブプログラム用)\TBW}
ここでは主に\index{サブプログラム}サブプログラムの記述に際して、\index{すうちけいさん@数値計算}数値計算に必要な部分をピックアップする。
なお、ここでは主に\DMC について述べるため、\index{スペーサ}スペーサに関するものは省略する。



%%%%%%%%%%%%%%%%%%%%%%%%%%%%%%%%%%%%%%%%%%%%%%%%%%%%%%%%%%
%% section H.1 %%%%%%%%%%%%%%%%%%%%%%%%%%%%%%%%%%%%%%%%%%%
%%%%%%%%%%%%%%%%%%%%%%%%%%%%%%%%%%%%%%%%%%%%%%%%%%%%%%%%%%
\modHeadsection{外削の数値情報}


%%%%%%%%%%%%%%%%%%%%%%%%%%%%%%%%%%%%%%%%%%%%%%%%%%%%%%%%%%
%% subsection H.3.1 %%%%%%%%%%%%%%%%%%%%%%%%%%%%%%%%%%%%%%
%%%%%%%%%%%%%%%%%%%%%%%%%%%%%%%%%%%%%%%%%%%%%%%%%%%%%%%%%%
\subsection{外削中心:ボトムA側肉厚基準の場合}
\index{テーブルちゅうしん@テーブル中心}テーブル中心\index{P(テーブルちゅうしん)@P(テーブル中心)}Pを\index{げんてんP@原点P}原点とした\index{ボトムがわのがいさくちゅうしん@ボトム側の外削中心}ボトム側外削中心$\mathfrak B_\mathrm c'$の(おおよその)$X$座標は、\pageeqref{eq:gaisakucenterBt}より、
\begin{align*}
  \HLbox{%
    \Delta_x'\cos\theta
    -\frac{\sqrt{R_\mathrm o^2-f_\mathrm B^2}+\sqrt{R_\mathrm i^2-f_\mathrm B^2}}2
    -\frac{w_\mathrm B}2
    -\tau_\mathrm B
    +\frac{\mathfrak W_\mathrm B}2
  }\ .
\end{align*}
このとき、測定したA側内面b$_\mathrm o'$の$X$座標が\pageeqref{eq:gaisakucenterBr}となるように、原点$\mathfrak B_\mathrm c'$を定める。
\begin{align*}
  \HLbox{-\left(\frac{\mathfrak W_\mathrm B}2-\tau_\mathrm B+\mu\right)}\ .
\end{align*}
トップ側にも外削がある場合、測定で定めた$\mathfrak B_\mathrm c'$の$X$座標$\mathcal G_{\mathrm Bx}$および\index{とおりしん@通り芯}通り芯$T_x$を用いて\pageeqref{eq:BbasedTx}で与えられる。
\begin{align*}
  \HLbox{-\mathcal G_{Bx}+T_x}\ .
\end{align*}


%%%%%%%%%%%%%%%%%%%%%%%%%%%%%%%%%%%%%%%%%%%%%%%%%%%%%%%%%%
%% subsection H.3.2 %%%%%%%%%%%%%%%%%%%%%%%%%%%%%%%%%%%%%%
%%%%%%%%%%%%%%%%%%%%%%%%%%%%%%%%%%%%%%%%%%%%%%%%%%%%%%%%%%
\subsection{外削中心:トップA面肉厚基準の場合}
\index{テーブルちゅうしん@テーブル中心}テーブル中心\index{P(テーブルちゅうしん)@P(テーブル中心)}Pを\index{げんてんP@原点P}原点とした\index{トップがわのがいさくちゅうしん@トップ側の外削中心}トップ側外削中心$\mathfrak T_\mathrm c'$の(おおよその)$X$座標は、\pageeqref{eq:gaisakucenterTt}より、
\begin{align*}
  \HLbox{%
    \frac{\sqrt{R_\mathrm o^2-f_\mathrm T^2}+\sqrt{R_\mathrm i^2-f_\mathrm T^2}}2
    -\Delta_x'\cos\theta
    +\frac{w_\mathrm T}2
    +\tau_\mathrm T
    -\frac{\mathfrak W_\mathrm T}2
  }\ .
\end{align*}
このとき、測定したA側内面t$_\mathrm o'$の$X$座標が\pageeqref{eq:gaisakucenterTr}となるように、原点$\mathfrak T_\mathrm c'$を定める。
\begin{align*}
  \HLbox{\frac{\mathfrak W_\mathrm T}2-\tau_\mathrm T+\mu}~.
\end{align*}
ボトム側にも外削がある場合、測定で定めた$\mathfrak T_\mathrm c'$の$X$座標$\mathcal G_{\mathrm Tx}$および\index{とおりしん@通り芯}通り芯$T_x$を用いて\pageeqref{eq:TbasedTx}で与えられる。
\begin{align*}
  \HLbox{-\mathcal G_{Tx}+T_x}\ .
\end{align*}


%%%%%%%%%%%%%%%%%%%%%%%%%%%%%%%%%%%%%%%%%%%%%%%%%%%%%%%%%%
%% subsection 30.3.3 %%%%%%%%%%%%%%%%%%%%%%%%%%%%%%%%%%%%%
%%%%%%%%%%%%%%%%%%%%%%%%%%%%%%%%%%%%%%%%%%%%%%%%%%%%%%%%%%
\subsection{\OutcutLength}

%%%%%%%%%%%%%%%%%%%%%%%%%%%%%%%%%%%%%%%%%%%%%%%%%%%%%%%%%%
%% subsubsection 30.3.3.1 %%%%%%%%%%%%%%%%%%%%%%%%%%%%%%%%
%%%%%%%%%%%%%%%%%%%%%%%%%%%%%%%%%%%%%%%%%%%%%%%%%%%%%%%%%%
\subsubsection{ボトムの外削}
ボトム側の外削における\index{こうぐ@工具}工具の先端の$Z$座標は、\BottomOutcutLength を$h_\mathrm B$として、
\begin{align*}
  \HLbox{f_\mathrm B'-h_\mathrm B}\ .
\end{align*}

%\clearpage
%%%%%%%%%%%%%%%%%%%%%%%%%%%%%%%%%%%%%%%%%%%%%%%%%%%%%%%%%%
%% subsubsection 30.3.3.1 %%%%%%%%%%%%%%%%%%%%%%%%%%%%%%%%
%%%%%%%%%%%%%%%%%%%%%%%%%%%%%%%%%%%%%%%%%%%%%%%%%%%%%%%%%%
\subsubsection{トップの外削}
トップ側の外削における工具の先端の$Z$座標は、\TopOutcutLength, \KeywayPos, \KeywayWidth をそれぞれ$h_\mathrm T$, $\kappa_p$, $\kappa_w$として、
\begin{alignat*}{3}
  & \HLbox{f_\mathrm T'-h_\mathrm T} & \quad & \Big(\text{if}~h_\mathrm T > \kappa_p+\kappa_w\Big)\\
  & \HLbox{f_\mathrm T'-\left(\kappa_p+1[\mathrm{mm}]\right)} & \quad  & \Big(\text{if}~h_\mathrm T = \kappa_p+\kappa_w\Big)
\end{alignat*}


%\clearpage
%%%%%%%%%%%%%%%%%%%%%%%%%%%%%%%%%%%%%%%%%%%%%%%%%%%%%%%%%%
%% subsection 30.3.4 %%%%%%%%%%%%%%%%%%%%%%%%%%%%%%%%%%%%%
%%%%%%%%%%%%%%%%%%%%%%%%%%%%%%%%%%%%%%%%%%%%%%%%%%%%%%%%%%
\subsection{湾曲に沿った外削\TBW}
(to be written...)



\clearpage
%%%%%%%%%%%%%%%%%%%%%%%%%%%%%%%%%%%%%%%%%%%%%%%%%%%%%%%%%%
%% section 30.4 %%%%%%%%%%%%%%%%%%%%%%%%%%%%%%%%%%%%%%%%%%
%%%%%%%%%%%%%%%%%%%%%%%%%%%%%%%%%%%%%%%%%%%%%%%%%%%%%%%%%%
\modHeadsection{\Keyway の数値情報}


%%%%%%%%%%%%%%%%%%%%%%%%%%%%%%%%%%%%%%%%%%%%%%%%%%%%%%%%%%
%% subsection 30.4.1 %%%%%%%%%%%%%%%%%%%%%%%%%%%%%%%%%%%%%
%%%%%%%%%%%%%%%%%%%%%%%%%%%%%%%%%%%%%%%%%%%%%%%%%%%%%%%%%%
\subsection{\KeywayCenter\texorpdfstring{$Z$}{Z}}
\KeywayPos$\kappa_p$および\KeywayWidth$\kappa_w$に対し、\index{テーブルちゅうしん@テーブル中心}テーブル中心\index{P(テーブルちゅうしん)@P(テーブル中心)}Pを\index{げんてんP@原点P}原点とした\KeywayCenter M$'$の$Z$座標は、\pageeqref{eq:mizocenterZ}より
\begin{align*}
  \HLbox{f_\mathrm T'-\kappa_p-\frac{\kappa_w}2}\ .
\end{align*}


%%%%%%%%%%%%%%%%%%%%%%%%%%%%%%%%%%%%%%%%%%%%%%%%%%%%%%%%%%
%% subsection 30.4.2 %%%%%%%%%%%%%%%%%%%%%%%%%%%%%%%%%%%%%
%%%%%%%%%%%%%%%%%%%%%%%%%%%%%%%%%%%%%%%%%%%%%%%%%%%%%%%%%%
\subsection{湾曲中心が基準の場合}
\index{トップたんのそとがわちゅうしん@トップ端の外側中心}トップ端の外側中心T$_\mathrm c'$と\KeywayCenter M$'$との$X$方向の差は、\pageeqref{eq:difTopMizoCenter}より、
\begin{align*}
  \HLbox{%
    \sqrt{R_\mathrm c^2-\left(f_\mathrm T-\kappa_p-\frac{\kappa_w}2\right)^2}
    -\frac{\sqrt{R_\mathrm o^2-f_\mathrm T^2}+\sqrt{R_\mathrm i^2-f_\mathrm T^2}}2%
  }\ .
\end{align*}


%%%%%%%%%%%%%%%%%%%%%%%%%%%%%%%%%%%%%%%%%%%%%%%%%%%%%%%%%%
%% subsection 30.4.3 %%%%%%%%%%%%%%%%%%%%%%%%%%%%%%%%%%%%%
%%%%%%%%%%%%%%%%%%%%%%%%%%%%%%%%%%%%%%%%%%%%%%%%%%%%%%%%%%
\subsection{外削中心が基準の場合}
\KeywayCenter は\index{トップがわのがいさくちゅうしん@トップ側の外削中心}トップ側の外削中心とする。


%%%%%%%%%%%%%%%%%%%%%%%%%%%%%%%%%%%%%%%%%%%%%%%%%%%%%%%%%%
%% subsection 30.4.4 %%%%%%%%%%%%%%%%%%%%%%%%%%%%%%%%%%%%%
%%%%%%%%%%%%%%%%%%%%%%%%%%%%%%%%%%%%%%%%%%%%%%%%%%%%%%%%%%
\subsection{\AsideKeywayDepth 指定の場合}

%%%%%%%%%%%%%%%%%%%%%%%%%%%%%%%%%%%%%%%%%%%%%%%%%%%%%%%%%%
%% subsubsection 30.4.4.1 %%%%%%%%%%%%%%%%%%%%%%%%%%%%%%%%
%%%%%%%%%%%%%%%%%%%%%%%%%%%%%%%%%%%%%%%%%%%%%%%%%%%%%%%%%%
\subsubsection{外削のない場合}
\AsideKeywayDepth$\kappa_d$は、その測定値$\kappa_d'$が\index{ずめん(モールド)@図面(モールド)}図面上の値となるように与えられるものとする。このとき\pageeqref{eq:keydepthDif1}より、
\begin{align*}
  \HLbox{%
    \kappa_d
    = \frac{2\kappa_d'-\kappa_w\sin\zeta}{1+\cos^2\zeta}\cos\zeta
      +\sqrt{R_\mathrm o^2-\left(f_\mathrm T-\kappa_p-\frac{\kappa_w}2\right)^2}
      -\sqrt{R_\mathrm o^2-\left(f_\mathrm T-\kappa_p\right)^2}%
  }\ .
\end{align*}
ここで$\zeta$は\pageeqref{eq:angleZeta}より、
\begin{align*}
  \HLbox{%
    \tan\zeta
    = \frac{\sqrt{R_\mathrm o^2-\left(f_\mathrm T-\kappa_p-\kappa_w\right)^2}
            -\sqrt{R_\mathrm o^2-\left(f_\mathrm T-\kappa_p\right)^2}}
           {\kappa_w}%
  }\ .
\end{align*}
\AsideKeywayDepth$\kappa_d$に対し、\KeywayCenter の位置の$X$座標は\pageeqref{eq:mizocenterA}より、
\begin{align*}
  \HLbox{%
    \sqrt{R_\mathrm o^2-\left(f_\mathrm T-\kappa_p-\frac{\kappa_w}2\right)^2}
    -\kappa_d
    -\frac{W_{mx}}2
    -\Delta_x%
  }\ .
\end{align*}
また\expandafterindex{Aがわがいめん(\yomiKeywayCenter)@A側外面(\nameKeywayCenter)}A側外面の\index{じっそくち@実測値}実測値を$\mathcal G_m$とすると、\KeywayCenter と$G_m$との$X$座標の差は、\pageeqref{eq:mizocenterAd}より、
\begin{align*}
  \HLbox{-\frac{W_{mx}}2-\kappa_d}\ .
\end{align*}

\clearpage
%%%%%%%%%%%%%%%%%%%%%%%%%%%%%%%%%%%%%%%%%%%%%%%%%%%%%%%%%%
%% subsubsection 30.4.4.2 %%%%%%%%%%%%%%%%%%%%%%%%%%%%%%%%
%%%%%%%%%%%%%%%%%%%%%%%%%%%%%%%%%%%%%%%%%%%%%%%%%%%%%%%%%%
\subsubsection{外削のある場合}
\AsideKeywayDepth$\kappa_d$に対し、トップ外削$X$中心を$\mathcal G_{\mathrm Tx}$とすると、$\mathcal G_{\mathrm Tx}$と\KeywayCenter との$X$座標の差は、\pageeqref{eq:mizocenterAG}より、
\begin{align*}
  \HLbox{\frac{\mathfrak W_x}2-\kappa_d-\frac{W_{mx}}2}\ .
\end{align*}



\clearpage
%%%%%%%%%%%%%%%%%%%%%%%%%%%%%%%%%%%%%%%%%%%%%%%%%%%%%%%%%%
%% section 9.2 %%%%%%%%%%%%%%%%%%%%%%%%%%%%%%%%%%%%%%%%%%%
%%%%%%%%%%%%%%%%%%%%%%%%%%%%%%%%%%%%%%%%%%%%%%%%%%%%%%%%%%
\modHeadsection{端面外C面取の数値情報\TBW}
(to be written...)



%\clearpage
%%%%%%%%%%%%%%%%%%%%%%%%%%%%%%%%%%%%%%%%%%%%%%%%%%%%%%%%%%
%% section 9.2 %%%%%%%%%%%%%%%%%%%%%%%%%%%%%%%%%%%%%%%%%%%
%%%%%%%%%%%%%%%%%%%%%%%%%%%%%%%%%%%%%%%%%%%%%%%%%%%%%%%%%%
\modHeadsection{端面内C面取の数値情報\TBW}
(to be written...)



%%%%%%%%%%%%%%%%%%%%%%%%%%%%%%%%%%%%%%%%%%%%%%%%%%%%%%%%%%
%% section 9.2 %%%%%%%%%%%%%%%%%%%%%%%%%%%%%%%%%%%%%%%%%%%
%%%%%%%%%%%%%%%%%%%%%%%%%%%%%%%%%%%%%%%%%%%%%%%%%%%%%%%%%%
\modHeadsection{端面外R面取の数値情報\TBW}
(to be written...)



%\clearpage
%%%%%%%%%%%%%%%%%%%%%%%%%%%%%%%%%%%%%%%%%%%%%%%%%%%%%%%%%%
%% section 9.2 %%%%%%%%%%%%%%%%%%%%%%%%%%%%%%%%%%%%%%%%%%%
%%%%%%%%%%%%%%%%%%%%%%%%%%%%%%%%%%%%%%%%%%%%%%%%%%%%%%%%%%
\modHeadsection{端面内R面取の数値情報\TBW}
(to be written...)


%\clearpage
%%%%%%%%%%%%%%%%%%%%%%%%%%%%%%%%%%%%%%%%%%%%%%%%%%%%%%%%%%
%% section 9.2 %%%%%%%%%%%%%%%%%%%%%%%%%%%%%%%%%%%%%%%%%%%
%%%%%%%%%%%%%%%%%%%%%%%%%%%%%%%%%%%%%%%%%%%%%%%%%%%%%%%%%%
\modHeadsection{\TanmenZaguri の数値情報\TBW}
(to be written...)



\clearpage
%%%%%%%%%%%%%%%%%%%%%%%%%%%%%%%%%%%%%%%%%%%%%%%%%%%%%%%%%%
%% section H.2 %%%%%%%%%%%%%%%%%%%%%%%%%%%%%%%%%%%%%%%%%%%
%%%%%%%%%%%%%%%%%%%%%%%%%%%%%%%%%%%%%%%%%%%%%%%%%%%%%%%%%%
\modHeadsection{\Dimple の数値情報\TBW}
(to be written...)
