%!TEX root = ./RPA_for_Creating_Program_Note.tex


\addtocontents{toc}{\protect\cleardoublepage}
%%%%%%%%%%%%%%%%%%%%%%%%%%%%%%%%%%%%%%%%%%%%%%%%%%%%%%%%%
%% Part Numerical Calculation %%%%%%%%%%%%%%%%%%%%%%%%%%%
%%%%%%%%%%%%%%%%%%%%%%%%%%%%%%%%%%%%%%%%%%%%%%%%%%%%%%%%%
\addtocontents{toc}{\protect\begin{tocBox}{\tmppartnum}}%
\tPart{解析計算に基づく数値解析\label{part:NC}}{%
\paragraph*{目標(なにがしたいか?)}
\index{めいさい(モールド)@明細(モールド)}明細ごとに異なる\index{すんぽう@寸法}寸法・形状を持つすべての\index{モールド}モールドに対し、\index{プログラム}プログラムの作成に必要な\textbf{数値情報および\index{じょうけんぶんき@条件分岐}条件分岐情報等が自動的に得られるシステムを構築}する。
\tcbline*
\paragraph*{手段(どうやって?)}
前段階で導出した\textbf{解析的な情報}を用いて、各明細における具体的な\textbf{数値的な情報}に自動的に変換するシステムの構築を試みる。
\tcbline*
\paragraph*{背景(なぜ?)}
一般に、個々の製品や\index{こうぐ@工具}工具等の寸法は異なり、明細ごとに固有の寸法・形状を持つ。
プログラムの作成の際には、それらをすべて考慮した\textbf{具体的な数値情報}を指定する必要がある。

 こうした数値情報は\textbf{明細ごとに膨大にある}が、現時点(\DMC 設置時点)において、こうした手続きは\textbf{明細ごとに手作業}で行われている
%% footnote %%%%%%%%%%%%%%%%%%%%%
\footnote{さらには、「\index{ずめん(モールド)@図面(モールド)}図面の作成後、それを見て\index{プログラム}プログラムを作成する」という明らかに異常な(奇妙な)事態が放置され続けている。}。
%%%%%%%%%%%%%%%%%%%%%%%%%%%%%%%%%

 したがって、こうした手続きのシステム化を行い、可能な限り\textbf{自動化}することが喫緊の課題である。
そうすることで、\textbf{危険を伴う作業の削減(\index{あんぜんせい@安全性}安全性の向上)}や、\textbf{品質の低下の防止}に大きく寄与できることが期待される。
また副次的効果として、作業効率・人的資源・安全(security)・\index{ほしゅ@保守}保守などのいずれの観点からみた\textbf{能率の低下の防止}にも大きく貢献することも自ずと期待される。
}{%
\paragraph*{結論(どうなった?)}
各明細のモールドにおける固有数値情報の入力により、プログラムの作成に必要な数値情報および条件分岐情報等が(人による手動計算を介することなく)自動的に得られるシステムを構築した。
\tcbline*
\paragraph*{次の段階(それで?)}
(to be written...)
}

%%%%%%%%%%%%%%%%%%%%%%%%%%%%%%%%%%%%%%%%%%%%%%%%%%%%%%%%%%
%% chapters %%%%%%%%%%%%%%%%%%%%%%%%%%%%%%%%%%%%%%%%%%%%%%
%%%%%%%%%%%%%%%%%%%%%%%%%%%%%%%%%%%%%%%%%%%%%%%%%%%%%%%%%%
%!TEX root = ./RPA_for_Creating_Program_Note.tex


\modHeadchapter[lot]{入力する数値情報・パラメータ}
寸法・形状等の数値情報や条件分岐情報は、\index{めいさい(モールド)@明細(モールド)}明細ごとに固有である。
そのため、その固有情報は入力する必要がある。
ここではそうした入力する必要のある情報をまとめておく。
なお、\index{にゅうりょくするすうちじょうほう@入力する数値情報}入力する数値情報に関しては、原則として\index{ずめん@図面}図面上の\index{すんぽう@寸法}寸法をそのまま入力する形となるような方針とする
%% footnote %%%%%%%%%%%%%%%%%%%%%
\footnote{ただし、これらの値にはそれぞれの公差が考慮されている。}。
%%%%%%%%%%%%%%%%%%%%%%%%%%%%%%%%%



%%%%%%%%%%%%%%%%%%%%%%%%%%%%%%%%%%%%%%%%%%%%%%%%%%%%%%%%%%
%% section 30.1 %%%%%%%%%%%%%%%%%%%%%%%%%%%%%%%%%%%%%%%%%%
%%%%%%%%%%%%%%%%%%%%%%%%%%%%%%%%%%%%%%%%%%%%%%%%%%%%%%%%%%
\modHeadsection{湾曲・振分けに関する入力数値}

\begin{multicollongtblr}{入力情報:湾曲・振分け}{X[l]c}
内容 & 型\\
\CenterCurvatureExists & boolean\\
\CenterCurvatureRadius & float\\
\TopAlocationLength & float\\
\BottomAlocationLength & float\\
\end{multicollongtblr}



%%%%%%%%%%%%%%%%%%%%%%%%%%%%%%%%%%%%%%%%%%%%%%%%%%%%%%%%%%
%% section 30.2 %%%%%%%%%%%%%%%%%%%%%%%%%%%%%%%%%%%%%%%%%%
%%%%%%%%%%%%%%%%%%%%%%%%%%%%%%%%%%%%%%%%%%%%%%%%%%%%%%%%%%
\modHeadsection{外形・内形に関する入力数値}

\begin{multicollongtblr}{入力情報:外形}{X[l]c}
内容 & 型\\
\ACOD & float\\
\BDOD & float\\
\ODCornerR & float\\
\end{multicollongtblr}

\begin{multicollongtblr}{入力情報:内形}{X[l]c}
内容 & 型\\
\PlatingThk & float\\
内径テーパ表 Taper No. & list\\
\end{multicollongtblr}
%%%%%%%%%%%%%%%%%%%%%%%%%%%%%%%%%%%%%%%%%%%%%%%%%%%%%%%%%%
%% marker %%%%%%%%%%%%%%%%%%%%%%%%%%%%%%%%%%%%%%%%%%%%%%%%
%%%%%%%%%%%%%%%%%%%%%%%%%%%%%%%%%%%%%%%%%%%%%%%%%%%%%%%%%%
\begin{marker}
\index{ないけいテーパひょう@内径テーパ表}内径テーパ表には、\index{ないけいコーナーR@内形コーナーR}内形コーナーRの\index{すんぽう(ないけいコーナーR)@寸法(内形コーナーR)}寸法情報も記載されているものとする。
\end{marker}
%%%%%%%%%%%%%%%%%%%%%%%%%%%%%%%%%%%%%%%%%%%%%%%%%%%%%%%%%%
%%%%%%%%%%%%%%%%%%%%%%%%%%%%%%%%%%%%%%%%%%%%%%%%%%%%%%%%%%
%%%%%%%%%%%%%%%%%%%%%%%%%%%%%%%%%%%%%%%%%%%%%%%%%%%%%%%%%%



\clearpage
%%%%%%%%%%%%%%%%%%%%%%%%%%%%%%%%%%%%%%%%%%%%%%%%%%%%%%%%%%
%% section 30.3 %%%%%%%%%%%%%%%%%%%%%%%%%%%%%%%%%%%%%%%%%%
%%%%%%%%%%%%%%%%%%%%%%%%%%%%%%%%%%%%%%%%%%%%%%%%%%%%%%%%%%
\modHeadsection{外削に関する入力数値}

\begin{multicollongtblr}{入力情報:ボトム外削}{X[l]c}
内容 & 型\\
\BottomOutcutExists & boolean\\
\BottomOutcutType & enum\\
ボトム外削 A側肉厚 & float\\
\BottomOutcutAC & float\\
\BottomOutcutBD & float\\
\BottomOutcutLength & float\\
\BottomOutcutConerR & float\\
\end{multicollongtblr}

\begin{multicollongtblr}{入力情報:トップ外削}{X[l]c}
内容 & 型\\
\TopOutcutExists & boolean\\
\TopOutcutType & enum\\
トップ外削 A側肉厚 & float\\
\TopOutcutAC & float\\
\TopOutcutBD & float\\
\TopOutcutLength & float\\
\TopOutcutCornerR & float\\
\end{multicollongtblr}

\begin{multicollongtblr}{入力情報:両外削}{X[l]c}
内容 & 型\\
外削中心基準 & enum\\
\CenterlineEndFaceDif & float\\
\end{multicollongtblr}



\clearpage
%%%%%%%%%%%%%%%%%%%%%%%%%%%%%%%%%%%%%%%%%%%%%%%%%%%%%%%%%%
%% section 30.4 %%%%%%%%%%%%%%%%%%%%%%%%%%%%%%%%%%%%%%%%%%
%%%%%%%%%%%%%%%%%%%%%%%%%%%%%%%%%%%%%%%%%%%%%%%%%%%%%%%%%%
\modHeadsection{\Keyway に関する入力数値}

\begin{multicollongtblr}{入力情報:\Keyway}{X[l]c}
内容 & 型\\
\KeywayType & enum\\
\AsideKeywayDepth 指定 有無 & boolean\\
\KeywayACOD & float\\
\KeywayBDOD & float\\
\KeywayPos & float\\
\KeywayWidth & float\\
\AsideKeywayDepth & float\\
\KeywayCornerR & float\\
\KeywayCornerC & float\\
\end{multicollongtblr}



%\clearpage
%%%%%%%%%%%%%%%%%%%%%%%%%%%%%%%%%%%%%%%%%%%%%%%%%%%%%%%%%%
%% section 30.5 %%%%%%%%%%%%%%%%%%%%%%%%%%%%%%%%%%%%%%%%%%
%%%%%%%%%%%%%%%%%%%%%%%%%%%%%%%%%%%%%%%%%%%%%%%%%%%%%%%%%%
\modHeadsection{\Dimple に関する入力数値}

\begin{multicollongtblr}{入力情報:\Dimple}{X[l]c}
内容 & 型\\
トップ端と\Dimple1列目までの距離 & float\\
\Dimple 鉛直方向ピッチ & float\\
\Dimple 水平方向ピッチ & float\\
\Dimple 奇数列の長さ & float\\
\Dimple 偶数列の長さ & float\\
\Dimple 列数 & enum\\
\Dimple 深さ & float\\
\Dimple 半径(工具小半径) & float\\
\end{multicollongtblr}



\clearpage
%%%%%%%%%%%%%%%%%%%%%%%%%%%%%%%%%%%%%%%%%%%%%%%%%%%%%%%%%%
%% section 30.6 %%%%%%%%%%%%%%%%%%%%%%%%%%%%%%%%%%%%%%%%%%
%%%%%%%%%%%%%%%%%%%%%%%%%%%%%%%%%%%%%%%%%%%%%%%%%%%%%%%%%%
\modHeadsection{端面面取に関する入力数値}


%%%%%%%%%%%%%%%%%%%%%%%%%%%%%%%%%%%%%%%%%%%%%%%%%%%%%%%%%%
%% subsection 30.6.1 %%%%%%%%%%%%%%%%%%%%%%%%%%%%%%%%%%%%%
%%%%%%%%%%%%%%%%%%%%%%%%%%%%%%%%%%%%%%%%%%%%%%%%%%%%%%%%%%
\subsection{ボトム端面面取に関する入力情報}

\begin{multicollongtblr}{入力情報:ボトム\nameEndFaceOutChamfer}{X[l]c}
内容 & 型\\
\BottomEndFaceOutChamferExists & boolean\\
\BottomEndFaceOutChamferLength & float\\
\BottomEndFaceOutChamferAngle & float\\
\end{multicollongtblr}

\begin{multicollongtblr}{入力情報:ボトム\nameEndFaceOutRoundChamfer}{X[l]c}
内容 & 型\\
\BottomEndFaceOutRoundChamferExists & boolean\\
\BottomEndFaceOutRoundChamferRadius & float\\
\end{multicollongtblr}

\begin{multicollongtblr}{入力情報:ボトム\nameEndFaceInChamfer}{X[l]c}
内容 & 型\\
\BottomEndFaceInChamferExists & boolean\\
\BottomEndFaceInChamferLength & float\\
\BottomEndFaceInChamferAngle & integer\\
\end{multicollongtblr}

\begin{multicollongtblr}{入力情報:ボトム\nameEndFaceInRoundChamfer}{X[l]c}
内容 & 型\\
\BottomFaceInRoundChamferExsits & boolean\\
\BottomFaceInRoundChamferRadius & float\\
\end{multicollongtblr}



\clearpage
%%%%%%%%%%%%%%%%%%%%%%%%%%%%%%%%%%%%%%%%%%%%%%%%%%%%%%%%%%
%% subsection 30.6.2 %%%%%%%%%%%%%%%%%%%%%%%%%%%%%%%%%%%%%
%%%%%%%%%%%%%%%%%%%%%%%%%%%%%%%%%%%%%%%%%%%%%%%%%%%%%%%%%%
\subsection{トップ端面面取に関する入力情報}

\begin{multicollongtblr}{入力情報:トップ\nameEndFaceOutChamfer}{X[l]c}
内容 & 型\\
\TopEndFaceOutChamferExists & boolean\\
\TopEndFaceOutChamferLength & float\\
\TopEndFaceOutChamferAngle & integer\\
\end{multicollongtblr}

\begin{multicollongtblr}{入力情報:トップ\nameEndFaceOutRoundChamfer}{X[l]c}
内容 & 型\\
\TopEndFaceOutRoundChamferExists & boolean\\
\TopEndFaceOutRoundChamferRadius & float\\
\end{multicollongtblr}

\begin{multicollongtblr}{入力情報:トップ\nameEndFaceInChamfer}{X[l]c}
内容 & 型\\
\TopEndFaceInChamferExists & boolean\\
\TopEndFaceInChamferLength & float\\
\TopEndFaceInChamferAngle & integer\\
\end{multicollongtblr}

\begin{multicollongtblr}{入力情報:トップ\nameEndFaceInRoundChamfer}{X[l]c}
内容 & 型\\
\TopEndFaceInRoundChamferExists & boolean\\
\TopEndFaceInRoundChamferRadius & float\\
\end{multicollongtblr}

%%%%%%%%%%%%%%%%%%%%%%%%%%%%%%%%%%%%%%%%%%%%%%%%%%%%%%%%%%
%% subsection 30.6.2 %%%%%%%%%%%%%%%%%%%%%%%%%%%%%%%%%%%%%
%%%%%%%%%%%%%%%%%%%%%%%%%%%%%%%%%%%%%%%%%%%%%%%%%%%%%%%%%%
\subsection{\EndFaceBoring に関する入力情報}

\begin{multicollongtblr}{入力情報:\EndFaceBoring}{X[l]c}
内容 & 型\\
\EndFaceBoringExists & boolean\\
\EndFaceBoringWidth & float\\
\EndFaceBoringDepth & float\\
\EndFaceBoringCornerR & float\\
\EndFaceBoringLength & float\\
\end{multicollongtblr}


\clearrightpage

%!TEX root = ./RPA_for_Creating_Program_Note.tex


基本的に、\index{すうちじょうほう@数値情報}数値情報については数値計算用の言語を用いて行うため、その詳細は別ドキュメントに譲る。
ここでは各明細用の\index{メインプログラム}メインプログラムの記述に際して、\index{すうちけいさん@数値計算}数値計算に必要な部分をピックアップする。
なお、ここでは主に\DMname について述べるため、\index{スペーサ}スペーサに関するものは省略する。


%%%%%%%%%%%%%%%%%%%%%%%%%%%%%%%%%%%%%%%%%%%%%%%%%%%%%%%%%%
%% section 30.1 %%%%%%%%%%%%%%%%%%%%%%%%%%%%%%%%%%%%%%%%%%
%%%%%%%%%%%%%%%%%%%%%%%%%%%%%%%%%%%%%%%%%%%%%%%%%%%%%%%%%%
\modHeadsection{再振分長・再張出長・均等振分角の数値情報}
各パラメータを以下とする。
\begin{align*}
  \varDelta_x' = \varDelta_x+\sqrt{R_\mathrm i'-\bar l^2}\ , \quad
  R_\mathrm i' = R_\mathrm c-\frac{W_x}2-\rho\ ,\quad
  \bar l = l-\frac\sigma2\ ,\quad
  f_d = \frac{f_\mathrm B-f_\mathrm T}2\ .
\end{align*}


%%%%%%%%%%%%%%%%%%%%%%%%%%%%%%%%%%%%%%%%%%%%%%%%%%%%%%%%%%
%% subsection 30.1.1 %%%%%%%%%%%%%%%%%%%%%%%%%%%%%%%%%%%%%
%%%%%%%%%%%%%%%%%%%%%%%%%%%%%%%%%%%%%%%%%%%%%%%%%%%%%%%%%%
\subsection{再振分長}
\index{テーブル}テーブルを$-\theta$だけ回転させて調整したトップ・ボトム側の\index{さいふりわけちょう@再振分長}振分長$f'_\mathrm T$, $f'_\mathrm B$は、\pageeqref{eq:saifuriwake}より、
\begin{align*}
  \text{トップ側:}\quad
  & \HLbox{f_\mathrm T' = f_\mathrm T+\varDelta_x'\!\sin\theta}\ ,\\
  \text{ボトム側:}\quad
  & \HLbox{f_\mathrm B' = (f_\mathrm T+f_\mathrm B)-f_\mathrm T'}\ .
\end{align*}


%%%%%%%%%%%%%%%%%%%%%%%%%%%%%%%%%%%%%%%%%%%%%%%%%%%%%%%%%%
%% subsection 30.1.2 %%%%%%%%%%%%%%%%%%%%%%%%%%%%%%%%%%%%%
%%%%%%%%%%%%%%%%%%%%%%%%%%%%%%%%%%%%%%%%%%%%%%%%%%%%%%%%%%
\subsection{再張出長}
テーブルを$-\theta$だけ回転させた後の\index{ジグ}ジグ(長さ$2l$)からの\index{さいはりだしちょう@再張出長}張出長に換算すると、それぞれ
\begin{align*}
  \text{トップ側:}\quad
  & \HLbox{f_\mathrm T'-l}\ ,\\
  \text{ボトム側:}\quad
  & \HLbox{f_\mathrm B'-l}\ .
\end{align*}


%%%%%%%%%%%%%%%%%%%%%%%%%%%%%%%%%%%%%%%%%%%%%%%%%%%%%%%%%%
%% subsection 30.1.3 %%%%%%%%%%%%%%%%%%%%%%%%%%%%%%%%%%%%%
%%%%%%%%%%%%%%%%%%%%%%%%%%%%%%%%%%%%%%%%%%%%%%%%%%%%%%%%%%
\subsection{均等振分角}
$f'_\mathrm T$および$f'_\mathrm B$が均等になり、かつトップ側およびボトム側の\index{たんめん@端面}端面が$X$軸方向に平行になるときの\index{かたむきかく(ふりわけちょうせい)@傾き角(振分調整)}回転角$\theta_\mathrm T'$, $\theta_\mathrm B'$は、\pageeqref{eq:saifuriwakeangle}よりそれぞれ、
\begin{align*}
  \text{トップ側:}\quad
  & \HLbox{\theta_\mathrm T' = -\sin^{-1}\frac{f_d}{\varDelta_x'}}\ ,\\
  \text{ボトム側:}\quad
  & \HLbox{\theta_\mathrm B' = \pi-\sin^{-1}\frac{f_d}{\varDelta_x'}}\ .
\end{align*}



\clearpage
%%%%%%%%%%%%%%%%%%%%%%%%%%%%%%%%%%%%%%%%%%%%%%%%%%%%%%%%%%
%% section 30.2 %%%%%%%%%%%%%%%%%%%%%%%%%%%%%%%%%%%%%%%%%%
%%%%%%%%%%%%%%%%%%%%%%%%%%%%%%%%%%%%%%%%%%%%%%%%%%%%%%%%%%
\modHeadsection{原点設定の数値情報}


%%%%%%%%%%%%%%%%%%%%%%%%%%%%%%%%%%%%%%%%%%%%%%%%%%%%%%%%%%
%% subsection 30.2.1 %%%%%%%%%%%%%%%%%%%%%%%%%%%%%%%%%%%%%
%%%%%%%%%%%%%%%%%%%%%%%%%%%%%%%%%%%%%%%%%%%%%%%%%%%%%%%%%%
\subsection{ボトム側の外側中心\texorpdfstring{$X$}{X}}

%%%%%%%%%%%%%%%%%%%%%%%%%%%%%%%%%%%%%%%%%%%%%%%%%%%%%%%%%%
%% subsubsection 30.2.1.1 %%%%%%%%%%%%%%%%%%%%%%%%%%%%%%%%
%%%%%%%%%%%%%%%%%%%%%%%%%%%%%%%%%%%%%%%%%%%%%%%%%%%%%%%%%%
\subsubsection{ボトム端の外側中心\texorpdfstring{$X$}{X}}
\index{テーブルちゅうしん@テーブル中心}テーブル中心\index{P(テーブルちゅうしん)@P(テーブル中心)}Pを\index{げんてんP@原点P}原点とした、($-\theta$回転後の)\index{ボトムたんのそとがわちゅうしん@ボトム端の外側中心}ボトム端の外径中心の$X$位置は、\pageeqref{eq:tableBc}より、
\begin{align*}
  \HLbox{%
    \varDelta_x'\cos\theta
    -\frac{\sqrt{R_\mathrm o^2-f_\mathrm B^2}+\sqrt{R_\mathrm i^2-f_\mathrm B^2}}2%
  }\ .
\end{align*}

%%%%%%%%%%%%%%%%%%%%%%%%%%%%%%%%%%%%%%%%%%%%%%%%%%%%%%%%%%
%% subsubsection 30.2.1.2 %%%%%%%%%%%%%%%%%%%%%%%%%%%%%%%%
%%%%%%%%%%%%%%%%%%%%%%%%%%%%%%%%%%%%%%%%%%%%%%%%%%%%%%%%%%
\subsubsection{ボトム側の外削中心\texorpdfstring{$X$}{X}(ボトムA側肉厚基準)}
\index{テーブルちゅうしん@テーブル中心}テーブル中心\index{P(テーブルちゅうしん)@P(テーブル中心)}Pを\index{げんてんP@原点P}原点とした、($-\theta$回転後の)\index{ボトムAがわにくあつ@ボトムA側肉厚}ボトムA側肉厚を\index{きじゅん(ボトムAがわにくあつ)@基準(ボトムA側肉厚)}基準とした\index{ボトムがわのがいさくちゅうしん@ボトム側の外削中心}ボトム側の外削中心$\mathfrak B_\mathrm c'$の(おおよその)$X$座標は、\pageeqref{eq:gaisakucenterBt}より、
\begin{align*}
  \HLbox{%
    \varDelta_x'\cos\theta
    -\frac{\sqrt{R_\mathrm o^2-f_\mathrm B^2}+\sqrt{R_\mathrm i^2-f_\mathrm B^2}}2
    -\frac{w_\mathrm B}2
    -\tau_\mathrm B
    +\frac{\mathfrak W_\mathrm B}2
  }\ .
\end{align*}

%%%%%%%%%%%%%%%%%%%%%%%%%%%%%%%%%%%%%%%%%%%%%%%%%%%%%%%%%%
%% subsubsection 30.2.1.2 %%%%%%%%%%%%%%%%%%%%%%%%%%%%%%%%
%%%%%%%%%%%%%%%%%%%%%%%%%%%%%%%%%%%%%%%%%%%%%%%%%%%%%%%%%%
\subsubsection{ボトム側の外削中心\texorpdfstring{$X$}{X}(トップA側肉厚基準)}
\index{テーブルちゅうしん@テーブル中心}テーブル中心\index{P(テーブルちゅうしん)@P(テーブル中心)}Pを\index{げんてんP@原点P}原点とした、($-\theta$回転後の)\index{トップAがわにくあつ@トップA側肉厚}トップA側肉厚を\index{きじゅん(トップAがわにくあつ)@基準(トップA側肉厚)}基準とした\index{トップがわのがいさくちゅうしん@トップ側の外削中心}ボトム側の外削中心$\mathfrak B_\mathrm c'$の(おおよその)$X$座標は、\pageeqref{eq:gaisakucenterTt}より、
\begin{align*}
  \HLbox{%
    -\left(
      \frac{\sqrt{R_\mathrm o^2-f_\mathrm T^2}+\sqrt{R_\mathrm i^2-f_\mathrm T^2}}2
      -\varDelta_x'\cos\theta
      +\frac{w_\mathrm T}2
      +\tau_\mathrm T
      -\frac{\mathfrak W_\mathrm T}2
    \right)
    +T_x
  }\ .
\end{align*}


%%%%%%%%%%%%%%%%%%%%%%%%%%%%%%%%%%%%%%%%%%%%%%%%%%%%%%%%%%
%% subsection 30.2.2 %%%%%%%%%%%%%%%%%%%%%%%%%%%%%%%%%%%%%
%%%%%%%%%%%%%%%%%%%%%%%%%%%%%%%%%%%%%%%%%%%%%%%%%%%%%%%%%%
\subsection{ボトム側の内側中心\texorpdfstring{$X$}{X}}

%%%%%%%%%%%%%%%%%%%%%%%%%%%%%%%%%%%%%%%%%%%%%%%%%%%%%%%%%%
%% subsubsection 30.2.2.1 %%%%%%%%%%%%%%%%%%%%%%%%%%%%%%%%
%%%%%%%%%%%%%%%%%%%%%%%%%%%%%%%%%%%%%%%%%%%%%%%%%%%%%%%%%%
\subsubsection{ボトム端の湾曲中心\texorpdfstring{$X$}{X}}
\index{テーブルちゅうしん@テーブル中心}テーブル中心\index{P(テーブルちゅうしん)@P(テーブル中心)}Pを\index{げんてんP@原点P}原点とした、($-\theta$回転後の)\index{ボトムたんのわんきょくちゅうしん@ボトム端の湾曲中心}ボトム端の湾曲中心の$X$値は、\pageeqref{eq:tableBRc}より、
\begin{align*}
  \HLbox{\varDelta_x'\!\cos\theta-\sqrt{R_\mathrm c^2-f_\mathrm B^2}}~.
\end{align*}

%%%%%%%%%%%%%%%%%%%%%%%%%%%%%%%%%%%%%%%%%%%%%%%%%%%%%%%%%%
%% subsubsection 30.2.2.2 %%%%%%%%%%%%%%%%%%%%%%%%%%%%%%%%
%%%%%%%%%%%%%%%%%%%%%%%%%%%%%%%%%%%%%%%%%%%%%%%%%%%%%%%%%%
\subsubsection{ボトム端の内側中心\texorpdfstring{$X$}{X}}
\index{テーブルちゅうしん@テーブル中心}テーブル中心\index{P(テーブルちゅうしん)@P(テーブル中心)}Pを\index{げんてんP@原点P}原点とした、($-\theta$回転後の)\index{ボトムたんのうちがわちゅうしん@ボトム端の内側中心}ボトム端の内側中心は、ボトム端の湾曲中心をもって代用してもよいものとする。


\clearpage
%%%%%%%%%%%%%%%%%%%%%%%%%%%%%%%%%%%%%%%%%%%%%%%%%%%%%%%%%%
%% subsection 30.2.3 %%%%%%%%%%%%%%%%%%%%%%%%%%%%%%%%%%%%%
%%%%%%%%%%%%%%%%%%%%%%%%%%%%%%%%%%%%%%%%%%%%%%%%%%%%%%%%%%
\subsection{トップ側の外側中心\texorpdfstring{$X$}{X}}

%%%%%%%%%%%%%%%%%%%%%%%%%%%%%%%%%%%%%%%%%%%%%%%%%%%%%%%%%%
%% subsubsection 30.2.3.1 %%%%%%%%%%%%%%%%%%%%%%%%%%%%%%%%
%%%%%%%%%%%%%%%%%%%%%%%%%%%%%%%%%%%%%%%%%%%%%%%%%%%%%%%%%%
\subsubsection{トップ端の外側中心\texorpdfstring{$X$}{X}}
\index{テーブルちゅうしん@テーブル中心}テーブル中心\index{P(テーブルちゅうしん)@P(テーブル中心)}Pを\index{げんてんP@原点P}原点とした、($-\theta$回転後の\index{トップたんのそとがわちゅうしん@トップ端の外側中心}外側中心の$X$位置は、\pageeqref{eq:tableTc}より、
\begin{align*}
  \HLbox{%
    \frac{\sqrt{R_\mathrm o^2-f_\mathrm T^2}+\sqrt{R_\mathrm i^2-f_\mathrm T^2}}2-\varDelta_x'\cos\theta%
  }~.
\end{align*}

%%%%%%%%%%%%%%%%%%%%%%%%%%%%%%%%%%%%%%%%%%%%%%%%%%%%%%%%%%
%% subsubsection 30.2.3.2 %%%%%%%%%%%%%%%%%%%%%%%%%%%%%%%%
%%%%%%%%%%%%%%%%%%%%%%%%%%%%%%%%%%%%%%%%%%%%%%%%%%%%%%%%%%
\subsubsection{トップ側の外削中心\texorpdfstring{$X$}{X}(ボトムA側肉厚基準)}
\index{テーブルちゅうしん@テーブル中心}テーブル中心\index{P(テーブルちゅうしん)@P(テーブル中心)}Pを\index{げんてんP@原点P}原点とした、($-\theta$回転後の)\index{ボトムAがわにくあつ@ボトムA側肉厚}ボトムA側肉厚を\index{きじゅん(ボトムAがわにくあつ)@基準(ボトムA側肉厚)}基準とした\index{トップがわのがいさくちゅうしん@トップ側の外削中心}トップ側の外削中心$\mathfrak T_\mathrm c'$の(おおよその)$X$座標は、\pageeqref{eq:gaisakucenterBt}より、
\begin{align*}
  \HLbox{%
    -\left(
      \varDelta_x'\cos\theta
      -\frac{\sqrt{R_\mathrm o^2-f_\mathrm B^2}+\sqrt{R_\mathrm i^2-f_\mathrm B^2}}2
      -\frac{w_\mathrm B}2
      -\tau_\mathrm B
      +\frac{\mathfrak W_\mathrm B}2
    \right)
    +T_x
  }\ .
\end{align*}

%%%%%%%%%%%%%%%%%%%%%%%%%%%%%%%%%%%%%%%%%%%%%%%%%%%%%%%%%%
%% subsubsection 30.2.3.3 %%%%%%%%%%%%%%%%%%%%%%%%%%%%%%%%
%%%%%%%%%%%%%%%%%%%%%%%%%%%%%%%%%%%%%%%%%%%%%%%%%%%%%%%%%%
\subsubsection{トップ側の外削中心\texorpdfstring{$X$}{X}(トップA側肉厚基準)}
\index{テーブルちゅうしん@テーブル中心}テーブル中心\index{P(テーブルちゅうしん)@P(テーブル中心)}Pを\index{げんてんP@原点P}原点とした、($-\theta$回転後の)\index{トップAがわにくあつ@トップA側肉厚}トップA側肉厚を\index{きじゅん(トップAがわにくあつ)@基準(トップA側肉厚)}基準とした\index{トップがわのがいさくちゅうしん@トップ側の外削中心}トップ側の外削中心$\mathfrak T_\mathrm c'$の(おおよその)$X$座標は、\pageeqref{eq:gaisakucenterTt}より、
\begin{align*}
  \HLbox{%
    \frac{\sqrt{R_\mathrm o^2-f_\mathrm T^2}+\sqrt{R_\mathrm i^2-f_\mathrm T^2}}2
    -\varDelta_x'\cos\theta
    +\frac{w_\mathrm T}2
    +\tau_\mathrm T
    -\frac{\mathfrak W_\mathrm T}2
  }\ .
\end{align*}

%%%%%%%%%%%%%%%%%%%%%%%%%%%%%%%%%%%%%%%%%%%%%%%%%%%%%%%%%%
%% subsubsection 30.2.3.4 %%%%%%%%%%%%%%%%%%%%%%%%%%%%%%%%
%%%%%%%%%%%%%%%%%%%%%%%%%%%%%%%%%%%%%%%%%%%%%%%%%%%%%%%%%%
\subsubsection{溝中心\texorpdfstring{$X$}{X}(A側溝深さ基準・外削なし)}
\index{みぞいち@溝位置}溝位置$\kappa_p$, \index{みぞはば@溝幅}溝幅$\kappa_w$, \index{Aがわみぞふかさ@A側溝深さ}A側溝深さ$\kappa_d'$, \index{みぞACけい@溝AC径}溝AC径$W_{mx}$に対し、\index{みぞちゅうしん@溝中心}溝中心M$'$の$X$座標は\pageeqref{eq:mizocenterA}より、
\begin{gather*}
  \HLbox{%
    \sqrt{R_\mathrm o^2-\left(f_\mathrm T-\kappa_p-\frac{\kappa_w}2\right)^{\!2}}
    -\kappa_d
    -\frac{W_{mx}}2
    -\varDelta_x%
  }\\[9pt]
  \left(
  \kappa_d
  = \frac{2\kappa_d'-\kappa_w\sin\zeta}{1+\cos^2\zeta}\cos\zeta
    +\sqrt{R_\mathrm o^2-\left(f_\mathrm T-\kappa_p-\frac{\kappa_w}2\right)^{\!2}}
    -\sqrt{R_\mathrm o^2-\left(f_\mathrm T-\kappa_p\right)^2}
  \right).
\end{gather*}


\clearpage
%%%%%%%%%%%%%%%%%%%%%%%%%%%%%%%%%%%%%%%%%%%%%%%%%%%%%%%%%%
%% subsection 30.2.4 %%%%%%%%%%%%%%%%%%%%%%%%%%%%%%%%%%%%%
%%%%%%%%%%%%%%%%%%%%%%%%%%%%%%%%%%%%%%%%%%%%%%%%%%%%%%%%%%
\subsection{トップ側の内側中心\texorpdfstring{$X$}{X}}

%%%%%%%%%%%%%%%%%%%%%%%%%%%%%%%%%%%%%%%%%%%%%%%%%%%%%%%%%%
%% subsubsection 30.4.1.1 %%%%%%%%%%%%%%%%%%%%%%%%%%%%%%%%
%%%%%%%%%%%%%%%%%%%%%%%%%%%%%%%%%%%%%%%%%%%%%%%%%%%%%%%%%%
\subsubsection{トップ端の湾曲中心\texorpdfstring{$X$}{X}}
\index{テーブルちゅうしん@テーブル中心}テーブル中心\index{P(テーブルちゅうしん)@P(テーブル中心)}Pを\index{げんてんP@原点P}原点とした、($-\theta$回転後の)\index{トップたんのわんきょくちゅうしん@トップ端の湾曲中心}トップ端の湾曲中心の$X$値は、\pageeqref{eq:tableTRc}より、
\begin{align*}
  \HLbox{\sqrt{R_\mathrm c^2-f_\mathrm T^2}-\varDelta_x'\!\cos\theta}~.
\end{align*}

%%%%%%%%%%%%%%%%%%%%%%%%%%%%%%%%%%%%%%%%%%%%%%%%%%%%%%%%%%
%% subsubsection 30.4.1.2 %%%%%%%%%%%%%%%%%%%%%%%%%%%%%%%%
%%%%%%%%%%%%%%%%%%%%%%%%%%%%%%%%%%%%%%%%%%%%%%%%%%%%%%%%%%
\subsubsection{トップ端の内側中心\texorpdfstring{$X$}{X}}
\index{テーブルちゅうしん@テーブル中心}テーブル中心\index{P(テーブルちゅうしん)@P(テーブル中心)}Pを\index{げんてんP@原点P}原点とした、($-\theta$回転後の)\index{トップたんのうちがわちゅうしん@トップ端の内側中心}トップ端の内側中心は、トップ端の湾曲中心をもって代用してもよいものとする。


%%%%%%%%%%%%%%%%%%%%%%%%%%%%%%%%%%%%%%%%%%%%%%%%%%%%%%%%%%
%% subsection 30.2.1 %%%%%%%%%%%%%%%%%%%%%%%%%%%%%%%%%%%%%
%%%%%%%%%%%%%%%%%%%%%%%%%%%%%%%%%%%%%%%%%%%%%%%%%%%%%%%%%%
\subsection{外側中心・内側中心\texorpdfstring{$Y$}{Y}}
\index{ジグ}ジグの底の$Y$座標を$\varDelta_y$とすると、\index{そとがわちゅうしんY@外側中心$Y$}外側中心および\index{うちがわちゅうしんY@内側中心$Y$}内側中心$Y$座標は、
\begin{align*}
  \HLbox{\varDelta_y+\frac{W_y}2}\ .
\end{align*}


%%%%%%%%%%%%%%%%%%%%%%%%%%%%%%%%%%%%%%%%%%%%%%%%%%%%%%%%%
%% Appendiodes %%%%%%%%%%%%%%%%%%%%%%%%%%%%%%%%%%%%%%%%%%
%%%%%%%%%%%%%%%%%%%%%%%%%%%%%%%%%%%%%%%%%%%%%%%%%%%%%%%%%
\begin{appendices}
\Appendixpart
%!TEX root = ./RPA_for_Creating_Program_Note.tex


\modHeadchapter{計算の必要な数値計算(サブプログラム用)\TBW}
ここでは主に\index{サブプログラム}サブプログラムの記述に際して、\index{すうちけいさん@数値計算}数値計算に必要な部分をピックアップする。
なお、ここでは主に\DMC について述べるため、\index{スペーサ}スペーサに関するものは省略する。



%%%%%%%%%%%%%%%%%%%%%%%%%%%%%%%%%%%%%%%%%%%%%%%%%%%%%%%%%%
%% section H.1 %%%%%%%%%%%%%%%%%%%%%%%%%%%%%%%%%%%%%%%%%%%
%%%%%%%%%%%%%%%%%%%%%%%%%%%%%%%%%%%%%%%%%%%%%%%%%%%%%%%%%%
\modHeadsection{外削の数値情報}


%%%%%%%%%%%%%%%%%%%%%%%%%%%%%%%%%%%%%%%%%%%%%%%%%%%%%%%%%%
%% subsection H.3.1 %%%%%%%%%%%%%%%%%%%%%%%%%%%%%%%%%%%%%%
%%%%%%%%%%%%%%%%%%%%%%%%%%%%%%%%%%%%%%%%%%%%%%%%%%%%%%%%%%
\subsection{外削中心:ボトムA側肉厚基準の場合}
\index{テーブルちゅうしん@テーブル中心}テーブル中心\index{P(テーブルちゅうしん)@P(テーブル中心)}Pを\index{げんてんP@原点P}原点とした\index{ボトムがわのがいさくちゅうしん@ボトム側の外削中心}ボトム側外削中心$\mathfrak B_\mathrm c'$の(おおよその)$X$座標は、\pageeqref{eq:gaisakucenterBt}より、
\begin{align*}
  \HLbox{%
    \Delta_x'\cos\theta
    -\frac{\sqrt{R_\mathrm o^2-f_\mathrm B^2}+\sqrt{R_\mathrm i^2-f_\mathrm B^2}}2
    -\frac{w_\mathrm B}2
    -\tau_\mathrm B
    +\frac{\mathfrak W_\mathrm B}2
  }\ .
\end{align*}
このとき、測定したA側内面b$_\mathrm o'$の$X$座標が\pageeqref{eq:gaisakucenterBr}となるように、原点$\mathfrak B_\mathrm c'$を定める。
\begin{align*}
  \HLbox{-\left(\frac{\mathfrak W_\mathrm B}2-\tau_\mathrm B+\mu\right)}\ .
\end{align*}
トップ側にも外削がある場合、測定で定めた$\mathfrak B_\mathrm c'$の$X$座標$\mathcal G_{\mathrm Bx}$および\CenterlineEndFaceDifAC$T_x$を用いて\pageeqref{eq:BbasedTx}で与えられる。
\begin{align*}
  \HLbox{-\mathcal G_{Bx}+T_x}\ .
\end{align*}


%%%%%%%%%%%%%%%%%%%%%%%%%%%%%%%%%%%%%%%%%%%%%%%%%%%%%%%%%%
%% subsection H.3.2 %%%%%%%%%%%%%%%%%%%%%%%%%%%%%%%%%%%%%%
%%%%%%%%%%%%%%%%%%%%%%%%%%%%%%%%%%%%%%%%%%%%%%%%%%%%%%%%%%
\subsection{外削中心:トップA面肉厚基準の場合}
\index{テーブルちゅうしん@テーブル中心}テーブル中心\index{P(テーブルちゅうしん)@P(テーブル中心)}Pを\index{げんてんP@原点P}原点とした\index{トップがわのがいさくちゅうしん@トップ側の外削中心}トップ側外削中心$\mathfrak T_\mathrm c'$の(おおよその)$X$座標は、\pageeqref{eq:gaisakucenterTt}より、
\begin{align*}
  \HLbox{%
    \frac{\sqrt{R_\mathrm o^2-f_\mathrm T^2}+\sqrt{R_\mathrm i^2-f_\mathrm T^2}}2
    -\Delta_x'\cos\theta
    +\frac{w_\mathrm T}2
    +\tau_\mathrm T
    -\frac{\mathfrak W_\mathrm T}2
  }\ .
\end{align*}
このとき、測定したA側内面t$_\mathrm o'$の$X$座標が\pageeqref{eq:gaisakucenterTr}となるように、原点$\mathfrak T_\mathrm c'$を定める。
\begin{align*}
  \HLbox{\frac{\mathfrak W_\mathrm T}2-\tau_\mathrm T+\mu}~.
\end{align*}
ボトム側にも外削がある場合、測定で定めた$\mathfrak T_\mathrm c'$の$X$座標$\mathcal G_{\mathrm Tx}$および\CenterlineEndFaceDifAC$T_x$を用いて\pageeqref{eq:TbasedTx}で与えられる。
\begin{align*}
  \HLbox{-\mathcal G_{Tx}+T_x}\ .
\end{align*}


%%%%%%%%%%%%%%%%%%%%%%%%%%%%%%%%%%%%%%%%%%%%%%%%%%%%%%%%%%
%% subsection 30.3.3 %%%%%%%%%%%%%%%%%%%%%%%%%%%%%%%%%%%%%
%%%%%%%%%%%%%%%%%%%%%%%%%%%%%%%%%%%%%%%%%%%%%%%%%%%%%%%%%%
\subsection{\OutcutLength}

%%%%%%%%%%%%%%%%%%%%%%%%%%%%%%%%%%%%%%%%%%%%%%%%%%%%%%%%%%
%% subsubsection 30.3.3.1 %%%%%%%%%%%%%%%%%%%%%%%%%%%%%%%%
%%%%%%%%%%%%%%%%%%%%%%%%%%%%%%%%%%%%%%%%%%%%%%%%%%%%%%%%%%
\subsubsection{ボトムの外削}
ボトム側の外削における\index{こうぐ@工具}工具の先端の$Z$座標は、\BottomOutcutLength を$h_\mathrm B$として、
\begin{align*}
  \HLbox{f_\mathrm B'-h_\mathrm B}\ .
\end{align*}

%\clearpage
%%%%%%%%%%%%%%%%%%%%%%%%%%%%%%%%%%%%%%%%%%%%%%%%%%%%%%%%%%
%% subsubsection 30.3.3.1 %%%%%%%%%%%%%%%%%%%%%%%%%%%%%%%%
%%%%%%%%%%%%%%%%%%%%%%%%%%%%%%%%%%%%%%%%%%%%%%%%%%%%%%%%%%
\subsubsection{トップの外削}
トップ側の外削における工具の先端の$Z$座標は、\TopOutcutLength, \KeywayPos, \KeywayWidth をそれぞれ$h_\mathrm T$, $\kappa_p$, $\kappa_w$として、
\begin{alignat*}{3}
  & \HLbox{f_\mathrm T'-h_\mathrm T} & \quad & \Big(\text{if}~h_\mathrm T > \kappa_p+\kappa_w\Big)\\
  & \HLbox{f_\mathrm T'-\left(\kappa_p+1[\mathrm{mm}]\right)} & \quad  & \Big(\text{if}~h_\mathrm T = \kappa_p+\kappa_w\Big)
\end{alignat*}


%\clearpage
%%%%%%%%%%%%%%%%%%%%%%%%%%%%%%%%%%%%%%%%%%%%%%%%%%%%%%%%%%
%% subsection 30.3.4 %%%%%%%%%%%%%%%%%%%%%%%%%%%%%%%%%%%%%
%%%%%%%%%%%%%%%%%%%%%%%%%%%%%%%%%%%%%%%%%%%%%%%%%%%%%%%%%%
\subsection{湾曲に沿った外削\TBW}
(to be written...)



\clearpage
%%%%%%%%%%%%%%%%%%%%%%%%%%%%%%%%%%%%%%%%%%%%%%%%%%%%%%%%%%
%% section 30.4 %%%%%%%%%%%%%%%%%%%%%%%%%%%%%%%%%%%%%%%%%%
%%%%%%%%%%%%%%%%%%%%%%%%%%%%%%%%%%%%%%%%%%%%%%%%%%%%%%%%%%
\modHeadsection{\Keyway の数値情報}


%%%%%%%%%%%%%%%%%%%%%%%%%%%%%%%%%%%%%%%%%%%%%%%%%%%%%%%%%%
%% subsection 30.4.1 %%%%%%%%%%%%%%%%%%%%%%%%%%%%%%%%%%%%%
%%%%%%%%%%%%%%%%%%%%%%%%%%%%%%%%%%%%%%%%%%%%%%%%%%%%%%%%%%
\subsection{\KeywayCenter\texorpdfstring{$Z$}{Z}}
\KeywayPos$\kappa_p$および\KeywayWidth$\kappa_w$に対し、\index{テーブルちゅうしん@テーブル中心}テーブル中心\index{P(テーブルちゅうしん)@P(テーブル中心)}Pを\index{げんてんP@原点P}原点とした\KeywayCenter M$'$の$Z$座標は、\pageeqref{eq:mizocenterZ}より
\begin{align*}
  \HLbox{f_\mathrm T'-\kappa_p-\frac{\kappa_w}2}\ .
\end{align*}


%%%%%%%%%%%%%%%%%%%%%%%%%%%%%%%%%%%%%%%%%%%%%%%%%%%%%%%%%%
%% subsection 30.4.2 %%%%%%%%%%%%%%%%%%%%%%%%%%%%%%%%%%%%%
%%%%%%%%%%%%%%%%%%%%%%%%%%%%%%%%%%%%%%%%%%%%%%%%%%%%%%%%%%
\subsection{湾曲中心が基準の場合}
\index{トップたんのそとがわちゅうしん@トップ端の外側中心}トップ端の外側中心T$_\mathrm c'$と\KeywayCenter M$'$との$X$方向の差は、\pageeqref{eq:difTopMizoCenter}より、
\begin{align*}
  \HLbox{%
    \sqrt{R_\mathrm c^2-\left(f_\mathrm T-\kappa_p-\frac{\kappa_w}2\right)^2}
    -\frac{\sqrt{R_\mathrm o^2-f_\mathrm T^2}+\sqrt{R_\mathrm i^2-f_\mathrm T^2}}2%
  }\ .
\end{align*}


%%%%%%%%%%%%%%%%%%%%%%%%%%%%%%%%%%%%%%%%%%%%%%%%%%%%%%%%%%
%% subsection 30.4.3 %%%%%%%%%%%%%%%%%%%%%%%%%%%%%%%%%%%%%
%%%%%%%%%%%%%%%%%%%%%%%%%%%%%%%%%%%%%%%%%%%%%%%%%%%%%%%%%%
\subsection{外削中心が基準の場合}
\KeywayCenter は\index{トップがわのがいさくちゅうしん@トップ側の外削中心}トップ側の外削中心とする。


%%%%%%%%%%%%%%%%%%%%%%%%%%%%%%%%%%%%%%%%%%%%%%%%%%%%%%%%%%
%% subsection 30.4.4 %%%%%%%%%%%%%%%%%%%%%%%%%%%%%%%%%%%%%
%%%%%%%%%%%%%%%%%%%%%%%%%%%%%%%%%%%%%%%%%%%%%%%%%%%%%%%%%%
\subsection{\AsideKeywayDepth 指定の場合}

%%%%%%%%%%%%%%%%%%%%%%%%%%%%%%%%%%%%%%%%%%%%%%%%%%%%%%%%%%
%% subsubsection 30.4.4.1 %%%%%%%%%%%%%%%%%%%%%%%%%%%%%%%%
%%%%%%%%%%%%%%%%%%%%%%%%%%%%%%%%%%%%%%%%%%%%%%%%%%%%%%%%%%
\subsubsection{外削のない場合}
\AsideKeywayDepth$\kappa_d$は、その測定値$\kappa_d'$が\index{ずめん(モールド)@図面(モールド)}図面上の値となるように与えられるものとする。このとき\pageeqref{eq:keydepthDif1}より、
\begin{align*}
  \HLbox{%
    \kappa_d
    = \frac{2\kappa_d'-\kappa_w\sin\zeta}{1+\cos^2\zeta}\cos\zeta
      +\sqrt{R_\mathrm o^2-\left(f_\mathrm T-\kappa_p-\frac{\kappa_w}2\right)^2}
      -\sqrt{R_\mathrm o^2-\left(f_\mathrm T-\kappa_p\right)^2}%
  }\ .
\end{align*}
ここで$\zeta$は\pageeqref{eq:angleZeta}より、
\begin{align*}
  \HLbox{%
    \tan\zeta
    = \frac{\sqrt{R_\mathrm o^2-\left(f_\mathrm T-\kappa_p-\kappa_w\right)^2}
            -\sqrt{R_\mathrm o^2-\left(f_\mathrm T-\kappa_p\right)^2}}
           {\kappa_w}%
  }\ .
\end{align*}
\AsideKeywayDepth$\kappa_d$に対し、\KeywayCenter の位置の$X$座標は\pageeqref{eq:mizocenterA}より、
\begin{align*}
  \HLbox{%
    \sqrt{R_\mathrm o^2-\left(f_\mathrm T-\kappa_p-\frac{\kappa_w}2\right)^2}
    -\kappa_d
    -\frac{W_{mx}}2
    -\Delta_x%
  }\ .
\end{align*}
また\expandafterindex{Aがわがいめん(\yomiKeywayCenter)@A側外面(\nameKeywayCenter)}A側外面の\index{じっそくち@実測値}実測値を$\mathcal G_m$とすると、\KeywayCenter と$G_m$との$X$座標の差は、\pageeqref{eq:mizocenterAd}より、
\begin{align*}
  \HLbox{-\frac{W_{mx}}2-\kappa_d}\ .
\end{align*}

\clearpage
%%%%%%%%%%%%%%%%%%%%%%%%%%%%%%%%%%%%%%%%%%%%%%%%%%%%%%%%%%
%% subsubsection 30.4.4.2 %%%%%%%%%%%%%%%%%%%%%%%%%%%%%%%%
%%%%%%%%%%%%%%%%%%%%%%%%%%%%%%%%%%%%%%%%%%%%%%%%%%%%%%%%%%
\subsubsection{外削のある場合}
\AsideKeywayDepth$\kappa_d$に対し、トップ外削$X$中心を$\mathcal G_{\mathrm Tx}$とすると、$\mathcal G_{\mathrm Tx}$と\KeywayCenter との$X$座標の差は、\pageeqref{eq:mizocenterAG}より、
\begin{align*}
  \HLbox{\frac{\mathfrak W_x}2-\kappa_d-\frac{W_{mx}}2}\ .
\end{align*}



\clearpage
%%%%%%%%%%%%%%%%%%%%%%%%%%%%%%%%%%%%%%%%%%%%%%%%%%%%%%%%%%
%% section 9.2 %%%%%%%%%%%%%%%%%%%%%%%%%%%%%%%%%%%%%%%%%%%
%%%%%%%%%%%%%%%%%%%%%%%%%%%%%%%%%%%%%%%%%%%%%%%%%%%%%%%%%%
\modHeadsection{\EndFaceOutChamfer の数値情報\TBW}
(to be written...)



%\clearpage
%%%%%%%%%%%%%%%%%%%%%%%%%%%%%%%%%%%%%%%%%%%%%%%%%%%%%%%%%%
%% section 9.2 %%%%%%%%%%%%%%%%%%%%%%%%%%%%%%%%%%%%%%%%%%%
%%%%%%%%%%%%%%%%%%%%%%%%%%%%%%%%%%%%%%%%%%%%%%%%%%%%%%%%%%
\modHeadsection{\EndFaceInChamfer の数値情報\TBW}
(to be written...)



%%%%%%%%%%%%%%%%%%%%%%%%%%%%%%%%%%%%%%%%%%%%%%%%%%%%%%%%%%
%% section 9.2 %%%%%%%%%%%%%%%%%%%%%%%%%%%%%%%%%%%%%%%%%%%
%%%%%%%%%%%%%%%%%%%%%%%%%%%%%%%%%%%%%%%%%%%%%%%%%%%%%%%%%%
\modHeadsection{端面外R面取の数値情報\TBW}
(to be written...)



%\clearpage
%%%%%%%%%%%%%%%%%%%%%%%%%%%%%%%%%%%%%%%%%%%%%%%%%%%%%%%%%%
%% section 9.2 %%%%%%%%%%%%%%%%%%%%%%%%%%%%%%%%%%%%%%%%%%%
%%%%%%%%%%%%%%%%%%%%%%%%%%%%%%%%%%%%%%%%%%%%%%%%%%%%%%%%%%
\modHeadsection{端面内R面取の数値情報\TBW}
(to be written...)


%\clearpage
%%%%%%%%%%%%%%%%%%%%%%%%%%%%%%%%%%%%%%%%%%%%%%%%%%%%%%%%%%
%% section 9.2 %%%%%%%%%%%%%%%%%%%%%%%%%%%%%%%%%%%%%%%%%%%
%%%%%%%%%%%%%%%%%%%%%%%%%%%%%%%%%%%%%%%%%%%%%%%%%%%%%%%%%%
\modHeadsection{\EndFaceBoring の数値情報\TBW}
(to be written...)



\clearpage
%%%%%%%%%%%%%%%%%%%%%%%%%%%%%%%%%%%%%%%%%%%%%%%%%%%%%%%%%%
%% section H.2 %%%%%%%%%%%%%%%%%%%%%%%%%%%%%%%%%%%%%%%%%%%
%%%%%%%%%%%%%%%%%%%%%%%%%%%%%%%%%%%%%%%%%%%%%%%%%%%%%%%%%%
\modHeadsection{\Dimple の数値情報\TBW}
(to be written...)

\end{appendices}

\addtocontents{toc}{\protect\end{tocBox}}
