%!TEX root = ./RPA_for_Creating_Program_Note.tex


\modHeadchapter[lot]{入力する数値情報・パラメータ}
寸法・形状等の数値情報や条件分岐情報は、\index{めいさい(モールド)@明細(モールド)}明細ごとに固有である。
そのため、その固有情報は入力する必要がある。
ここではそうした入力する必要のある情報をまとめておく。
なお、\index{にゅうりょくするすうちじょうほう@入力する数値情報}入力する数値情報に関しては、原則として\index{ずめん@図面}図面上の\index{すんぽう@寸法}寸法をそのまま入力する形となるような方針とする
%% footnote %%%%%%%%%%%%%%%%%%%%%
\footnote{ただし、これらの値にはそれぞれの公差が考慮されている。}。
%%%%%%%%%%%%%%%%%%%%%%%%%%%%%%%%%



%%%%%%%%%%%%%%%%%%%%%%%%%%%%%%%%%%%%%%%%%%%%%%%%%%%%%%%%%%
%% section 30.1 %%%%%%%%%%%%%%%%%%%%%%%%%%%%%%%%%%%%%%%%%%
%%%%%%%%%%%%%%%%%%%%%%%%%%%%%%%%%%%%%%%%%%%%%%%%%%%%%%%%%%
\modHeadsection{湾曲・振分けに関する入力数値}

\begin{multicollongtblr}{入力情報:湾曲・振分け}{X[l]c}
内容 & 型\\
中心湾曲の有無 & boolean\\
中心湾曲 & float\\
トップ振分長 & float\\
ボトム振分長 & float\\
\end{multicollongtblr}



%%%%%%%%%%%%%%%%%%%%%%%%%%%%%%%%%%%%%%%%%%%%%%%%%%%%%%%%%%
%% section 30.2 %%%%%%%%%%%%%%%%%%%%%%%%%%%%%%%%%%%%%%%%%%
%%%%%%%%%%%%%%%%%%%%%%%%%%%%%%%%%%%%%%%%%%%%%%%%%%%%%%%%%%
\modHeadsection{外形・内形に関する入力数値}

\begin{multicollongtblr}{入力情報:外形}{X[l]c}
内容 & 型\\
AC外径 & float\\
BD外径 & float\\
外形 コーナーR & float\\
\end{multicollongtblr}

\begin{multicollongtblr}{入力情報:内形}{X[l]c}
内容 & 型\\
内面めっき膜厚 & float\\
内形テーパ表 Taper No. & list\\
\end{multicollongtblr}
%%%%%%%%%%%%%%%%%%%%%%%%%%%%%%%%%%%%%%%%%%%%%%%%%%%%%%%%%%
%% marker %%%%%%%%%%%%%%%%%%%%%%%%%%%%%%%%%%%%%%%%%%%%%%%%
%%%%%%%%%%%%%%%%%%%%%%%%%%%%%%%%%%%%%%%%%%%%%%%%%%%%%%%%%%
\begin{marker}
\index{ないけいテーパひょう@内径テーパ表}内径テーパ表には、\index{ないけいのコーナーR@内形のコーナーR}内形のコーナーRの数値情報も記載されているものとする。
\end{marker}
%%%%%%%%%%%%%%%%%%%%%%%%%%%%%%%%%%%%%%%%%%%%%%%%%%%%%%%%%%
%%%%%%%%%%%%%%%%%%%%%%%%%%%%%%%%%%%%%%%%%%%%%%%%%%%%%%%%%%
%%%%%%%%%%%%%%%%%%%%%%%%%%%%%%%%%%%%%%%%%%%%%%%%%%%%%%%%%%



\clearpage
%%%%%%%%%%%%%%%%%%%%%%%%%%%%%%%%%%%%%%%%%%%%%%%%%%%%%%%%%%
%% section 30.3 %%%%%%%%%%%%%%%%%%%%%%%%%%%%%%%%%%%%%%%%%%
%%%%%%%%%%%%%%%%%%%%%%%%%%%%%%%%%%%%%%%%%%%%%%%%%%%%%%%%%%
\modHeadsection{外削に関する入力数値}

\begin{multicollongtblr}{入力情報:ボトム外削}{X[l]c}
内容 & 型\\
ボトム外削の有無 & boolean\\
面取の有無 & boolean\\
ボトム外削 A側肉厚 & float\\
ボトム外削 AC径 & float\\
ボトム外削 BD径 & float\\
ボトム外削長 & float\\
ボトム外削 コーナーR & float\\
\end{multicollongtblr}

\begin{multicollongtblr}{入力情報:トップ外削}{X[l]c}
内容 & 型\\
トップ外削の有無 & boolean\\
面取の有無 & boolean\\
トップ外削 A側肉厚 & float\\
トップ外削 AC径 & float\\
トップ外削 BD径 & float\\
トップ外削長 & float\\
トップ外削 コーナーR & float\\
\end{multicollongtblr}

\begin{multicollongtblr}{入力情報:両外削}{X[l]c}
内容 & 型\\
中心基準 & enum\\
通り芯 & float\\
\end{multicollongtblr}



\clearpage
%%%%%%%%%%%%%%%%%%%%%%%%%%%%%%%%%%%%%%%%%%%%%%%%%%%%%%%%%%
%% section 30.4 %%%%%%%%%%%%%%%%%%%%%%%%%%%%%%%%%%%%%%%%%%
%%%%%%%%%%%%%%%%%%%%%%%%%%%%%%%%%%%%%%%%%%%%%%%%%%%%%%%%%%
\modHeadsection{溝に関する入力数値}

\begin{multicollongtblr}{入力情報:溝}{X[l]c}
内容 & 型\\
溝の種類 & enum\\
A側溝深さ指定 有無 & boolean\\
溝AC径 & float\\
溝BD径 & float\\
溝位置 & float\\
溝幅 & float\\
A側溝深さ & float\\
溝 コーナーR & float\\
溝 コーナーC & float\\
\end{multicollongtblr}



%\clearpage
%%%%%%%%%%%%%%%%%%%%%%%%%%%%%%%%%%%%%%%%%%%%%%%%%%%%%%%%%%
%% section 30.5 %%%%%%%%%%%%%%%%%%%%%%%%%%%%%%%%%%%%%%%%%%
%%%%%%%%%%%%%%%%%%%%%%%%%%%%%%%%%%%%%%%%%%%%%%%%%%%%%%%%%%
\modHeadsection{\dimple に関する入力数値}

\begin{multicollongtblr}{入力情報:\dimple}{X[l]c}
内容 & 型\\
トップ端と\dimple1列目までの距離 & float\\
\dimple 鉛直方向ピッチ & float\\
\dimple 水平方向ピッチ & float\\
\dimple 奇数列の長さ & float\\
\dimple 偶数列の長さ & float\\
\dimple 列数 & enum\\
\dimple 深さ & float\\
\dimple 半径(工具小半径) & float\\
\end{multicollongtblr}



\clearpage
%%%%%%%%%%%%%%%%%%%%%%%%%%%%%%%%%%%%%%%%%%%%%%%%%%%%%%%%%%
%% section 30.6 %%%%%%%%%%%%%%%%%%%%%%%%%%%%%%%%%%%%%%%%%%
%%%%%%%%%%%%%%%%%%%%%%%%%%%%%%%%%%%%%%%%%%%%%%%%%%%%%%%%%%
\modHeadsection{端面の面取に関する入力数値}


%%%%%%%%%%%%%%%%%%%%%%%%%%%%%%%%%%%%%%%%%%%%%%%%%%%%%%%%%%
%% subsection 30.6.1 %%%%%%%%%%%%%%%%%%%%%%%%%%%%%%%%%%%%%
%%%%%%%%%%%%%%%%%%%%%%%%%%%%%%%%%%%%%%%%%%%%%%%%%%%%%%%%%%
\subsection{ボトム端の面取に関する入力情報}

\begin{multicollongtblr}{入力情報:ボトム端 外C面取}{X[l]c}
内容 & 型\\
ボトム端 外C面取の有無 & boolean\\
ボトム端 外C面取長 & float\\
ボトム端 外C面取 角度 & float\\
\end{multicollongtblr}

\begin{multicollongtblr}{入力情報:ボトム端 外R面取}{X[l]c}
内容 & 型\\
ボトム端 外R面取の有無 & boolean\\
ボトム端 外R面取長 & float\\
\end{multicollongtblr}

\begin{multicollongtblr}{入力情報:ボトム端 内C面取}{X[l]c}
内容 & 型\\
ボトム端 内C面取の有無 & boolean\\
ボトム端 内C面取長 & float\\
ボトム端 内C面取 角度 & integer\\
\end{multicollongtblr}

\begin{multicollongtblr}{入力情報:ボトム端 内R面取}{X[l]c}
内容 & 型\\
ボトム端 内R面取の有無 & boolean\\
ボトム端 内R面取長 & float\\
\end{multicollongtblr}



\clearpage
%%%%%%%%%%%%%%%%%%%%%%%%%%%%%%%%%%%%%%%%%%%%%%%%%%%%%%%%%%
%% subsection 30.6.2 %%%%%%%%%%%%%%%%%%%%%%%%%%%%%%%%%%%%%
%%%%%%%%%%%%%%%%%%%%%%%%%%%%%%%%%%%%%%%%%%%%%%%%%%%%%%%%%%
\subsection{トップ端の面取に関する入力情報}

\begin{multicollongtblr}{入力情報:トップ端 外C面取}{X[l]c}
内容 & 型\\
トップ端 外C面取の有無 & boolean\\
トップ端 外C面取長 & float\\
トップ端 外C面取 角度 & integer\\
\end{multicollongtblr}

\begin{multicollongtblr}{入力情報:トップ端 外R面取}{X[l]c}
内容 & 型\\
トップ端 外R面取の有無 & boolean\\
トップ端 外R面取長 & float\\
\end{multicollongtblr}

\begin{multicollongtblr}{入力情報:トップ端 内C面取}{X[l]c}
内容 & 型\\
トップ端 内C面取の有無 & boolean\\
トップ端 内C面取長 & float\\
トップ端 内C面取 角度 & integer\\
\end{multicollongtblr}

\begin{multicollongtblr}{入力情報:トップ端 内R面取}{X[l]c}
内容 & 型\\
トップ端 内R面取の有無 & boolean\\
トップ端 内R面取長 & float\\
\end{multicollongtblr}

\clearrightpage
