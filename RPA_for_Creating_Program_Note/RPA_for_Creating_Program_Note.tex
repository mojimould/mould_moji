%!TEX encoding = UTF-8 Unicode

%!TEX root = ../RPA_for_Creating_Program_Note.tex

%%%%% DOCUMENTCLASS %%%%%%%%%%%%%%%%%%%%%%%%%%
\documentclass[11pt, twoside, open=any]{scrbook}

\let\tmpcleardoublepage\cleardoublepage
\let\tmpclearpage\clearpage
\let\tmpnewpage\newpage
\input{./RfCPN_a0_preamble/RfCPN_packages}
%!TEX root = ../RPA_for_Creating_Program_Note.tex

\makeatletter

%!TEX root = ../RPA_for_Creating_Program_Note.tex


%%%%%%%%%%%%%%%%%%%%%%%%%%%%%%%%%%%%%%%%%%%%%%
%%%%% MOULDCOODINATE %%%%%%%%%%%%%%%%%%%%%%%%%
%%%%%%%%%%%%%%%%%%%%%%%%%%%%%%%%%%%%%%%%%%%%%%
\def\mouldCoordinate{%
\begin{tikzpicture}[scale=1, every node/.style={scale=0.8}]
% 値の計算
\pgfmathsetmacro{\Ax}{9.6} %A:T_iのx座標
\pgfmathsetmacro{\Ay}{3.5} %A:T_iのy座標
\pgfmathsetmacro{\Bx}{2.0+(\Ax)} %B:T_oのx座標
\pgfmathsetmacro{\Cy}{-4.2}                  %C:B_iのy座標
\pgfmathsetmacro{\Ri}{sqrt((\Ax)^2+(\Ay)^2)} %R_iの長さ
\pgfmathsetmacro{\Cx}{sqrt((\Ri)^2-(\Cy)^2)} %C:B_iのx座標
\pgfmathsetmacro{\Ro}{sqrt((\Bx)^2+(\Ay)^2)} %R_oの長さ
\pgfmathsetmacro{\Dx}{sqrt((\Ro)^2-(\Cy)^2)} %D:B_oのx座標
\pgfmathsetmacro{\Rc}{(\Ri+\Ro)/2}           %R_cの長さ
\pgfmathsetmacro{\Ex}{sqrt((\Rc)^2-(\Ay)^2)} %E:湾曲中心線トップ端のx座標
\pgfmathsetmacro{\Fx}{sqrt((\Rc)^2-(\Cy)^2)} %F:湾曲中心線ボトム端のx座標
\pgfmathsetmacro{\Ub}{2.4}                   %Ub:受板-モールド接点のy座標
\pgfmathsetmacro{\Ux}{sqrt((\Ri)^2-(\Ub)^2)} %Ux:受板-モールド接点のx座標
\pgfmathsetmacro{\Uxo}{sqrt((\Ro)^2-(\Ub)^2)} %Ux:受板-モールド接点のx座標
\pgfmathsetmacro{\Hx}{1+(\Ri)} %H:テーブルの中心
\pgfmathsetmacro{\Ix}{1.90} %I:テーブルx方向の長さの半分
% 座標系を描画
\draw[-latex, line width=0.5pt] (-0.4, 0) -- (14.5, 0) node[below] {\textbf{Re}};
\draw[-latex, line width=0.5pt] (0, -4) -- (0, 4) node[below right] {\textbf{Im}};
% 座標を定義
\coordinate (O) at (  0, 0); % 原点
\coordinate (A) at (\Ax, \Ay); %
\coordinate (B) at (\Bx, \Ay);
\coordinate (C) at (\Cx, \Cy);
\coordinate (D) at (\Dx, \Cy);
\coordinate (E) at (\Ex, \Ay);
\coordinate (F) at (\Fx, \Cy);
\coordinate (Rc) at (\Rc, 0);
\coordinate (Ri) at (\Ri, 0);
\coordinate (Ro) at (\Ro, 0);
\coordinate (Ut) at (\Ux, \Ub);
\coordinate (Ub) at (\Ux, -\Ub);
\coordinate (Uto) at (\Uxo, \Ub);
\coordinate (Ubo) at (\Uxo, -\Ub);
\coordinate (Tal) at (\Hx-\Ix, \Ub);
\coordinate (Tbl) at (\Hx-\Ix, -\Ub);
\coordinate (Tar) at (\Hx+\Ix, \Ub);
\coordinate (Tbr) at (\Hx+\Ix, -\Ub);
\coordinate (Lt) at (\Hx+\Ix+0.4, \Ub);
\coordinate (Lb) at (\Hx+\Ix+0.4, -\Ub);
\coordinate (Lc) at (\Hx+\Ix+0.4, 0);
\coordinate (Fc) at (\Hx+\Ix+1.0, 0);
\coordinate (Ft) at (\Hx+\Ix+1.0, \Ay);
\coordinate (Fb) at (\Hx+\Ix+1.0, \Cy);
% 点を描画
\fill (O) circle (2pt);
\fill (A) circle (2pt);
\fill (B) circle (2pt);
\fill (C) circle (2pt);
\fill (D) circle (2pt);
\fill (E) circle (2pt);
\fill (Rc) circle (2pt);
\fill (Ro) circle (2pt);
\fill (Ri) circle (2pt);
\fill (Ut) circle (2pt);
\fill (Ub) circle (2pt);
% 点にラベルを付ける
\node at (O) [below left] {O};
\node at (A) [above] {T$_\mathrm i$};
\node at (B) [above] {T$_\mathrm o$};
\node at (C) [below] {B$_\mathrm i$};
\node at (D) [below] {B$_\mathrm o$};
\node at (Rc) [above right] {$R_\mathrm c$};
\node at (Ro) [below right] {$R_\mathrm o$};
\node at (Ri) [below left] {$R_\mathrm i$};
\node at (Ut) [right] {U$_\mathrm T$};
\node at (Ub) [right] {U$_\mathrm B$};
% モールド外形
\draw[line width=0.75pt, fill=ffwwqq, fill opacity=0.1]
  let \p1=(A), \p2=(C), \p3=(B), \p4=(D), \n1={atan2(\y1,\x1)}, \n2={atan2(\y2,\x2)}, \n3={atan2(\y3,\x3)}, \n4={atan2(\y4,\x4)}
    in (A) -- (B) -- (\n3:\Ro) arc (\n3:\n4:\Ro) -- (C) -- (\n2:\Ri) arc (\n2:\n1:\Ri) -- cycle;
% モールド中心線
\draw[dotted, line width=0.5pt] let \p1=(E), \p2=(F), \n1={atan2(\y1,\x1)}, \n2={atan2(\y2,\x2)}
  in (\n1:\Rc) arc (\n1:\n2:\Rc);
% テーブル
\draw (Ut) -- (Tal) -- (Tbl) -- (Ub);
\draw (Uto) -- (Tar) -- (Tbr) -- (Ubo);
\draw[dotted] (Tar) -- (Lt);
\draw[dotted] (Tbr) -- (Lb);
\draw[dotted] (Tar) -- (Lt);
\draw[latex-latex, line width=0.5pt] (Lc) -- (Lt) node[midway, right] {$l$};
\draw[latex-latex, line width=0.5pt] (Lc) -- (Lb) node[midway, right] {$l$};
% 振分け
\draw[dotted] (B) -- (Ft);
\draw[dotted] (D) -- (Fb);
\draw[latex-latex, line width=0.5pt] (Fc) -- (Ft) node[midway, right] {$f_\mathrm T$};
\draw[latex-latex, line width=0.5pt] (Fc) -- (Fb) node[midway, right] {$f_\mathrm B$};
% 半径
\draw[dotted, line width=0.5pt] (O) -- (A) node[midway, above left] {$R_\mathrm i$} ;
\draw[dotted, line width=0.5pt] (O) -- (B) node[midway, below right] {$R_\mathrm o$} ;
\draw[dotted, line width=0.5pt] (O) -- (Ub) node[midway, above right] {$R_\mathrm i$} ;
% 角度
\draw[line width=0.5pt, fill=ffwwqq, fill opacity=0.1]
  let \p1=(Ri), \p2=(A), \n1={atan2(\y1,\x1)}, \n2={atan2(\y2,\x2)}
   in (\n1:2) arc (\n1:\n2:2) node[midway, right, opacity=1] {$\alpha_{\mathrm T_\mathrm i}$} -- (O);
\draw[line width=0.5pt, fill=qqzzqq, fill opacity=0.1]
  let \p1=(Ro), \p2=(B), \n1={atan2(\y1,\x1)}, \n2={atan2(\y2,\x2)}
  in (\n1:3.2) arc (\n1:\n2:3.2) node[midway, right, opacity=1] {$\alpha_{\mathrm T_\mathrm o}$} -- (O) -- cycle ;
\draw[line width=0.5pt, fill=wwqqcc, fill opacity=0.1]
  let \p1=(Ri), \p2=(Ub), \n1={atan2(\y1,\x1)}, \n2={atan2(\y2,\x2)}
  in (\n1:2.7) arc (\n1:\n2:2.7) node[midway, right, opacity=1] {$\alpha_{\mathrm U_\mathrm B}$} -- (O);
\end{tikzpicture}%
}

%%%%%%%%%%%%%%%%%%%%%%%%%%%%%%%%%%%%%%%%%%%%%%
%%%%% MOULDWITHUKEITA %%%%%%%%%%%%%%%%%%%%%%%%
%%%%%%%%%%%%%%%%%%%%%%%%%%%%%%%%%%%%%%%%%%%%%%
\def\mouldwithukeita{%
\begin{tikzpicture}[scale=0.5,
                    every node/.style={scale=0.4},
                    >={Latex[length=1mm, width=0.75mm]},]
% 値の計算
\pgfmathsetmacro{\Ax}{12}                    %A:T_iのx座標
\pgfmathsetmacro{\Ay}{3.5}                   %A:T_iのy座標
\pgfmathsetmacro{\Bx}{2.0+(\Ax)}             %B:T_oのx座標
\pgfmathsetmacro{\Cy}{-4.2}                  %C:B_iのy座標
\pgfmathsetmacro{\TUb}{2.4}                  %TUb:テーブル-モールドの交点y座標
\pgfmathsetmacro{\Ix}{2.7}                   %I:テーブルx方向の長さの半分
\pgfmathsetmacro{\Uw}{0.75}                  %Uw:受板の幅
\pgfmathsetmacro{\Ul}{0.45}                  %Ul:受板の長さ
\pgfmathsetmacro{\Ur}{1.35}                   %Uw:受板の半径
\pgfmathsetmacro{\Ri}{sqrt((\Ax)^2+(\Ay)^2)} %R_iの長さ
\pgfmathsetmacro{\Hx}{0.1+(\Ri)}             %H:テーブルの中心
% 値の計算
\pgfmathsetmacro{\Cx}{sqrt((\Ri)^2-(\Cy)^2)}    %C:B_iのx座標
\pgfmathsetmacro{\Ro}{sqrt((\Bx)^2+(\Ay)^2)}    %R_oの長さ
\pgfmathsetmacro{\Dx}{sqrt((\Ro)^2-(\Cy)^2)}    %D:B_oのx座標
\pgfmathsetmacro{\Rc}{(\Ri+\Ro)/2}              %R_cの長さ
\pgfmathsetmacro{\Ex}{sqrt((\Rc)^2-(\Ay)^2)}    %E:湾曲中心線トップ端のx座標
\pgfmathsetmacro{\Fx}{sqrt((\Rc)^2-(\Cy)^2)}    %F:湾曲中心線ボトム端のx座標
\pgfmathsetmacro{\TUx}{sqrt((\Ri)^2-(\TUb)^2)}  %TUx:テーブル-モールド内側との交点のx座標
\pgfmathsetmacro{\TUxo}{sqrt((\Ro)^2-(\TUb)^2)} %TUxo:テーブル-モールド外側との交点x座標
\pgfmathsetmacro{\Ub}{\Ri*sin(asin((\TUb-\Uw/2)/(\Ri-\Ur)))} %Ub:ボトム側受板-モールド接点のy座標
\pgfmathsetmacro{\Ux}{sqrt((\Ri)^2-(\Ub)^2)}    %Ux:トップ側受板-モールド接点のx座標
\pgfmathsetmacro{\Ucx}{\Ux-sqrt((\Ur)^2-((\Uw)/2-(\TUb-\Ub))^2)} %Uw:受板の半径
\pgfmathsetmacro{\TBUx}{\Ucx+sqrt((\Ur)^2-((\Uw)/2)^2)} %TBUx:テーブルと受板の交点x座標
% 座標の定義
\coordinate (A) at (\Ax, \Ay); %
\coordinate (B) at (\Bx, \Ay);
\coordinate (C) at (\Cx, \Cy);
\coordinate (D) at (\Dx, \Cy);
\coordinate (E) at (\Ex, \Ay);
\coordinate (F) at (\Fx, \Cy);
\coordinate (Ut) at (\Ux, \Ub);
\coordinate (TUt) at (\TUx, \TUb);
\coordinate (Ub) at (\Ux, -\Ub);
\coordinate (TUb) at (\TUx, -\TUb);
\coordinate (TUto) at (\TUxo, \TUb);
\coordinate (TUbo) at (\TUxo, -\TUb);
\coordinate (Tal) at (\Hx-\Ix, \TUb);
\coordinate (Tbl) at (\Hx-\Ix, -\TUb);
\coordinate (Tar) at (\Hx+\Ix, \TUb);
\coordinate (Tbr) at (\Hx+\Ix, -\TUb);
\coordinate (Lt) at (\Hx+\Ix+0.4, \TUb);
\coordinate (Lb) at (\Hx+\Ix+0.4, -\TUb);
\coordinate (Lc) at (\Hx+\Ix+0.4, 0);
\coordinate (Fc) at (\Hx+\Ix+1.0, 0);
\coordinate (TUc) at (\Ucx, \TUb-\Uw/2);
\coordinate (BUc) at (\Ucx, \Uw/2-\TUb);
\coordinate (TTU) at (\TBUx, \TUb);
\coordinate (TTUex) at (\TBUx-\Ul, \TUb);
\coordinate (TUex) at (\TBUx-\Ul, \TUb-\Uw);
\coordinate (TU) at (\TBUx, \TUb-\Uw);
\coordinate (TBU) at (\TBUx, -\TUb);
\coordinate (TBUex) at (\TBUx-\Ul, -\TUb);
\coordinate (BUex) at (\TBUx-\Ul, -\TUb+\Uw);
\coordinate (BU) at (\TBUx, -\TUb+\Uw);
% モールド外形
\draw[line width=0.5pt, fill=ffwwqq, fill opacity=0.1]
  let \p1=(A), \p2=(C), \p3=(B), \p4=(D), \n1={atan2(\y1,\x1)}, \n2={atan2(\y2,\x2)}, \n3={atan2(\y3,\x3)}, \n4={atan2(\y4,\x4)}
    in (A) -- (B) -- (\n3:\Ro) arc (\n3:\n4:\Ro) -- (C) -- (\n2:\Ri) arc (\n2:\n1:\Ri) -- cycle;
% モールド中心線
\draw[dotted, line width=0.5pt] let \p1=(E), \p2=(F), \n1={atan2(\y1,\x1)}, \n2={atan2(\y2,\x2)}
  in (\n1:\Rc) arc (\n1:\n2:\Rc);
% テーブル
\draw (TUt) -- (Tal) -- (Tbl) -- (TUb);
\draw (TUto) -- (Tar) -- (Tbr) -- (TUbo);
% 受板
\draw[fill=gray!15!] let \p1=(TUc), \p2=(TU), \p3=(TTU), \n1={atan2(\y2-\y1,\x2-\x1)}, \n2={atan2(\y3-\y1,\x3-\x1)},
    in (TTU) -- (TTUex) -- (TUex) -- (TU) arc[start angle=\n1, end angle=\n2, radius=\Ur] -- cycle;
\draw[fill=gray!15!] let \p1=(BUc), \p2=(BU), \p3=(TBU), \n1={atan2(\y2-\y1,\x2-\x1)}, \n2={atan2(\y3-\y1,\x3-\x1)},
    in (TBU) -- (TBUex) -- (BUex) -- (BU) arc[start angle=\n1, end angle=\n2, radius=\Ur] -- cycle;
\draw[<->] (BUc) -- (Ub) node[midway, above] {$\rho$};
\draw[<->] ([xshift=-1.25mm]TTUex) -- ([xshift=-1.25mm]TUex) node[midway, left] {$\sigma$};
% 点を描画
\fill (Ut) circle (1pt);
\fill (Ub) circle (1pt);
\fill (TUc) circle (1pt);
\fill (BUc) circle (1pt);
% 点にラベルを付ける
\node at (Ut) [right] {$R_\mathrm ie^{i\alpha_{\mathrm U_\mathrm T}}$};
\node at (Ub) [right] {$R_\mathrm ie^{i\alpha_{\mathrm U_\mathrm B}}$};
\node at (TUc) [above] {U$_\mathrm T$};
\node at (BUc) [below] {U$_\mathrm B$};
\end{tikzpicture}%
}

%!TEX root = ../RPA_for_Creating_Program_Note.tex


%%%%%%%%%%%%%%%%%%%%%%%%%%%%%%%%%%%%%%%%%%%%%%
%%%%% OTHER DEFINITION %%%%%%%%%%%%%%%%%%%%%%%
%%%%%%%%%%%%%%%%%%%%%%%%%%%%%%%%%%%%%%%%%%%%%%
\newcommand\ttNum{\ifmmode{\text{\texttt\#}}\else\texttt\#\fi}
\newcommand\cf{{\itshape cf.\,}}
\newcommand{\TBW}{\texorpdfstring{\small{\color{red}\,*}}{}}
%%%%% COLOR %%%%%%%%%%%%%%%%%%%%%%%%%%%%%%%%%%
\definecolor{ai}     {rgb}{0.2039, 0.3765, 0.4314}
\definecolor{kon}    {rgb}{0.0000, 0.2000, 0.4000}
\definecolor{konpeki}{rgb}{0.0902, 0.5098, 0.7333}
\definecolor{moegi}  {rgb}{0.3020, 0.5961, 0.1882}
\definecolor{sssec}  {rgb}{0.7333, 0.5, 0.7333}
\definecolor{sora}   {rgb}{0.1451, 0.7216, 0.8039}
\definecolor{sumire} {rgb}{0.3882, 0.2157, 0.5922}
\definecolor{wwqqcc} {rgb}{0.4, 0, 0.8}
\definecolor{qqzzqq} {rgb}{0, 0.6, 0}
\definecolor{ffwwqq} {rgb}{1, 0.4, 0}
\colorlet{unusingVariables}{gray!35!}
%%%%% DATE %%%%%%%%%%%%%%%%%%%%%%%%%%%%%%%%%%%
\newcommand{\customtoday}{\the\year/\two@digits{\the\month}/\two@digits{\the\day}}
\newcommand{\customdate}{\customtoday\ \currenttime\ (\jadayofweek{\the\year}{\the\month}{\the\day})}
\newcommand{\customtodayap}{\ifnum\currenthour<12 \customtoday\,a.m.\else\customtoday\,p.m.\fi}
%%%%% MACHINING NAME %%%%%%%%%%%%%%%%%%
\newcommand{\DMname}{Dマシニング}
\newcommand{\MMname}{Mマシニング}
\newcommand{\dimple}{ディンプル}
\newcommand{\dimplekana}{ディンプル}
%%%%% TO PRG NAME %%%%%%
\newcommand\nameMainEx{O1916} % main program example
\newcommand\MainEx{\prgbox{\nameMainEx}}
\newcommand\nameMXOThickness{O110001} % X outer center measurement
\newcommand\MXOThickness{\prgbox{\nameMXOThickness}}
\newcommand\nameMYOThickness{O110002} % Y outer center measurement
\newcommand\MYOThickness{\prgbox{\nameMYOThickness}}
\newcommand\nameMXOface{O120001} % keyway X reference surface measurement
\newcommand\MXOface{\prgbox{\nameMXOface}}
\newcommand\nameMXIWidth{O130001} % X inner center measurement
\newcommand\MXIWidth{\prgbox{\nameMXIWidth}}
\newcommand\nameMYIWidth{O130002} % Y inner center measurement
\newcommand\MYIWidth{\prgbox{\nameMYIWidth}}
\newcommand\nameMXIface{O140001} % outer cut X reference surface measurement
\newcommand\MXIface{\prgbox{\nameMXIface}}
\newcommand\nameMYcenterline{O150002} % Y centerline measurement
\newcommand\MYcenterline{\prgbox{\nameMYcenterline}}
\newcommand\nameMXcenterline{O150003} % X centerline measurement (Z measurement)
\newcommand\MXcenterline{\prgbox{\nameMXcenterline}}
\newcommand\nameDLone{O210003} % for dimple level 1
\newcommand\DLone{\prgbox{\nameDLone}}
\newcommand\nameDLtwoAC{O220001} % for dimple level 2 AC
\newcommand\DLtwoAC{\prgbox{\nameDLtwoAC}}
\newcommand\nameDLtwoBD{O220002} % for dimple level 2 BD
\newcommand\DLtwoBD{\prgbox{\nameDLtwoBD}}
\newcommand\nameDMLthreeAC{O230001} % for dimple level 3 AC measurement
\newcommand\DMLthreeAC{\prgbox{\nameDMLthreeAC}}
\newcommand\nameDMLthreeBD{O230002} % for dimple level 3 BD measurement
\newcommand\DMLthreeBD{\prgbox{\nameDMLthreeBD}}
\newcommand\nameKTanmenRight{O410000} % end face left turn
\newcommand\KTanmenRight{\prgbox{\nameKTanmenRight}}
\newcommand\nameKGaisakuRLeft{O420000} % outer cut left turn
\newcommand\KGaisakuRLeft{\prgbox{\nameKGaisakuRLeft}}
\newcommand\nameKMizoConerLeft{O430000} % keyway left turn
\newcommand\KMizoConerLeft{\prgbox{\nameKMizoConerLeft}}
\newcommand\nameKSotoMentoriRLeft{O440000} % 外側面取用 左回り
\newcommand\KSotoMentoriRLeft{\prgbox{\nameKSotoMentoriRLeft}}
\newcommand\nameKUchiMentoriRLeft{O450000} % 内側面取用 左回り
\newcommand\KUchiMentoriRLeft{\prgbox{\nameKUchiMentoriRLeft}}
\newcommand\nameKOLeftAR{O490005} % 外 左回り above right
\newcommand\KOLeftAR{\prgbox{\nameKOLeftAR}}
\newcommand\nameKILeftAC{O490007} % 内 右回り above center
\newcommand\KILeftAC{\prgbox{\nameKILeftAC}}
\newcommand\nameDKLthreeAC{O530001} % 内面溝 加工用 レベル3 AC
\newcommand\DKLthreeAC{\prgbox{\nameDKLthreeAC}}
\newcommand\nameDKLthreeBD{O530002} % 内面溝 加工用 レベル3 BD
\newcommand\DKLthreeBD{\prgbox{\nameDKLthreeBD}}
\newcommand\nameOpauseCheck{O900003} % move and pause in front of the door
\newcommand\OpauseCheck{\prgbox{\nameOpauseCheck}}
\newcommand\nameOsensorOn{O910001} % touch sensor ON
\newcommand\OsensorOn{\prgbox{\nameOsensorOn}}
\newcommand\nameOsensorOff{O910002} % touch sensor OFF
\newcommand\OsensorOff{\prgbox{\nameOsensorOff}}
\newcommand\nameOwarmingupA{O915100} % for warming up
\newcommand\OwarmingupA{\prgbox{\nameOwarmingupA}}
\newcommand\nameOwarmingup{O915101} % subprg for warming up
\newcommand\Owarmingup{\prgbox{\nameOwarmingup}}
\newcommand\nameOtoolLengthA{O917100} % specified tool length correction auto measurement
\newcommand\OtoolLengthA{\prgbox{\nameOtoolLengthA}}
\newcommand\nameOtoolLength{O919100} % tool length correction auto measure
\newcommand\OtoolLength{\prgbox{\nameOtoolLength}}
%%%%% LINK %%%%%%%%%%%%%%%%%%%%%%%%%%%%%%%%%%%
\newcommand{\linkLaTeX}{\href{https://www.latex-project.org/}{\LaTeX}}
\newcommand{\linkLaTeXProject}{\href{https://www.latex-project.org/}{\LaTeX\ Project}}
\newcommand{\linkAMS}{\href{https://www.latex-project.org/}{American Mathematical Society}}
\newcommand{\linkTeXLive}{\href{https://tug.org/texlive/}{\TeX\ Live}}
\newcommand{\linkTeXUsersGroup}{\href{http://www.tug.org/}{\TeX\ Users Group}}
\newcommand{\linkBibLaTeX}{\href{https://ctan.org/pkg/biblatex}{Bib\LaTeX}}
\newcommand{\linkBiber}{\href{https://ctan.org/pkg/biber}{Biber}}
\newcommand{\linkupmendex}{\href{https://ctan.org/pkg/upmendex}{upmendex}}
\newcommand{\linkPGFTikZ}{\href{https://github.com/pgf-tikz/pgf}{\pgfname/\tikzname}}
\newcommand{\linkTeXStudio}{\href{https://texstudio.org/}{\TeX\ Studio}}
\newcommand{\linkSumatraPDF}{\href{https://www.sumatrapdfreader.org/}{Sumatra PDF}}
\newcommand{\linkVSCode}{\href{https://code.visualstudio.com/}{VS Code}}
\newcommand{\linkWolframAlpha}{\href{https://www.wolframalpha.com/}{Wolfram\textbar Alpha}}
\newcommand{\linkWolfram}{\href{https://www.wolfram.com/}{Wolfram Research}}
\newcommand{\linkExcel}{\href{https://www.microsoft.com/ja-jp/microsoft-365/excel}{Excel}}
\newcommand{\linkMicrosoftCopilot}{\href{https://www.bing.com/}{Microsoft Copilot}}
\newcommand{\linkMicrosoftCorp}{\href{https://www.microsoft.com/}{Microsoft Corporation}}
\newcommand{\linkPython}{\href{https://www.python.org/}{Python}}
\newcommand{\linkPythonSF}{\href{https://www.python.org/psf-landing/}{Python Software Foundation}}
\newcommand{\linkGitHub}{\href{https://github.com/}{GitHub}}
\newcommand{\linkGitHubDesktop}{\href{https://desktop.github.com/}{GitHub Desktop}}
\newcommand{\linkGitHubInc}{\href{https://github.com/}{GitHub, Inc}}
\newcommand{\linkDocker}{\href{https://www.docker.com/}{Docker}}
\newcommand{\linkDockerInc}{\href{https://www.docker.com/}{Docker, Inc}}
\newcommand{\linkUbuntu}{\href{https://ubuntu.com/}{Ubuntu}}
\newcommand{\linkCanonicalLtd}{\href{https://canonical.com/}{Canonical Ltd}}
\newcommand{\linkSQLite}{\href{https://www.sqlite.org/}{SQLite}}
\newcommand{\linkSQLiteConsortium}{\href{https://www.sqlite.org/consortium.html}{SQLite Consortium}}
\newcommand{\linkChatGPT}{\href{https://openai.com/chatgpt}{ChatGPT}}
\newcommand{\linkOpenAI}{\href{https://www.openai.com/}{OpenAI}}


%%%%%%%%%%%%%%%%%%%%%%%%%%%%%%%%%%%%%%%%%%%%%%
%%%%% NEWIF %%%%%%%%%%%%%%%%%%%%%%%%%%%%%%%%%%
%%%%%%%%%%%%%%%%%%%%%%%%%%%%%%%%%%%%%%%%%%%%%%
\newif\if@backmatter%\@backmattertrue
\newif\if@frontmatter%\@frontmattertrue
\newif\if@appendix%\@appendixfalse
%%%%%%%%%%%%%%%%%%%%%%%%%%%%%%%%%%%%%%%%%%%%%%
%%%%% DIMENSION %%%%%%%%%%%%%%%%%%%%%%%%%%%%%%
%%%%%%%%%%%%%%%%%%%%%%%%%%%%%%%%%%%%%%%%%%%%%%
\ifluatex
  \usepackage{luatexja}
  \ltjsetparameter{kanjiskip=0.0pt plus 0.4pt minus 0.5pt}
  \ltjsetparameter{xkanjiskip=2.40555pt plus 1.0pt minus 1.0pt}
  \newcommand{\hk}{\hspace{\ltjgetparameter{kanjiskip}}}
  \newcommand{\hx}{\hspace{\ltjgetparameter{xkanjiskip}}}
\else
  \setlength{\kanjiskip}{0.0pt plus 0.4pt minus 0.5pt}
  \setlength{\xkanjiskip}{2.40555pt plus 1.0pt minus 1.0pt}
  \newcommand{\hk}{\hspace{\kanjiskip}}
  \newcommand{\hx}{\hspace{\xkanjiskip}}
\fi
%%%%%%%%%%%%%%%%%%%%%%%%%%%%%%%%%%%%%%%%%%%%%%
%%%%% CLEARLEFTPAGE %%%%%%%%%%%%%%%%%%%%%%%%%%
%%%%%%%%%%%%%%%%%%%%%%%%%%%%%%%%%%%%%%%%%%%%%%
\newcommand{\thispageNum}{\the\value{page}}
\newcommand{\clearleftpage}{%
  \if@twoside%
    \ifodd\thispageNum%
      \clearpage%
    \else
      \cleardoublepage%
    \fi
  \else%
    \clearpage%
  \fi
}
\newcommand{\clearrightpage}{%
  \if@twoside%
    \ifodd\thispageNum%
      \cleardoublepage%
    \else%
      \clearpage%
    \fi
  \else%
    \clearpage%
  \fi
}
%%%%%%%%%%%%%%%%%%%%%%%%%%%%%%%%%%%%%%%%%%%%%%
%%%%% REF %%%%%%%%%%%%%%%%%%%%%%%%%%%%%%%%%%%%
%%%%%%%%%%%%%%%%%%%%%%%%%%%%%%%%%%%%%%%%%%%%%%
%%%%% AUTOREFNAME %%%%%%%%%%%%%%%%%%%%%%%%%%%%
%\newcommand{\subfigureautorefname}{\figureautorefname} % subfigure --> figure
%\newcommand{\subtableautorefname}{\tableautorefname}
%%%%% PAGEREF %%%%%%%%%%%%%%%%%%%%%%%%%%%%%%%%%%%
\newcommand{\pageautoref}[1]{%
  \ifthenelse{\equal{\pageref{#1}}{\thepage}}%
    {\autoref{#1}}%
    {\autoref{#1}~[p.\pageref{#1}]}%
}
\newcommand{\pageeqref}[1]{%
  \ifthenelse{\equal{\pageref{#1}}{\thepage}}%
    {\eqref{#1}}%
    {\eqref{#1}~[p.\pageref{#1}]}%
}
%%%%%%%%%%%%%%%%%%%%%%%%%%%%%%%%%%%%%%%%%%%%%%
%%%%% FOR REFERENCES %%%%%%%%%%%%%%%%%%%%%%%%%
%%%%%%%%%%%%%%%%%%%%%%%%%%%%%%%%%%%%%%%%%%%%%%
\newcommand{\Articlename}{論文}
\newcommand{\Bookname}{書籍}
\newcommand{\OnlineSourcename}{ウェブサイト}
\defbibheading{articles}[\Articlename]{\section*{#1}}
\defbibheading{online}[\OnlineSourcename]{\section*{#1}}
%%%%% DECLAREFIELDFORMAT %%%%%%%%%%%%%%%%%%%%
\DeclareFieldFormat{urldate}{%
  (\textbf{urlseen~}\thefield{urlyear}/\ifnum\thefield{urlmonth}<10 0\fi\thefield{urlmonth})%
}
%%%%%%%%%%%%%%%%%%%%%%%%%%%%%%%%%%%%%%%%%%%%%%
%%%%% FOR TOC %%%%%%%%%%%%%%%%%%%%%%%%%%%%%%%%
%%%%%%%%%%%%%%%%%%%%%%%%%%%%%%%%%%%%%%%%%%%%%%
%%%%% TOC LINE %%%%%%%%%%%%%%%%%%%%%%%%%%%%%%%
%\newcommand\PartSeparateline[1]{\addtocontents{#1}{\protect\par\protect\hrulefill\protect\par\protect\vspace*{-10pt}}}%
%%%%% tPart %%%%%%%%%%%%%%%%%%%%%%%%%%%%%%%%%%%%%%%%%
\newcounter{tocwatermark}
\setcounter{tocwatermark}{1}
\def\tmppartnum{\Roman{tocwatermark}}%
\newcommand{\tPart}[4][]{%
  \addtocounter{tocwatermark}{1}
  \def\tmppartnum{\Roman{tocwatermark}}%
%  \def\temppartpage{part:\the\numexpr\value{part}+1}%
  \begingroup
  \clearrightpage
%  \@openrightfalse
  \let\oldnewpage\newpage
  \let\newpage\relax
  \part{#2\label{part:\thepart}}
  \addtocontents{lop}{\protect\vspace*{-2.5mm}}%
  \addcontentsline{lop}{part}{\protect\numberline{\thepart}#2}%
  \ifx#1\relax\else%
    \foreach \x in {#1}{\addcontentsline{\x}{part}{\protect\numberline{\thepart}#2}}%
  \fi
  \let\newpage\oldnewpage
  \clearpage
%  \@openrightfalse
  \thispagestyle{emptydate}
  \vspace*{0.1\textheight}%
  \begin{tablePart}{#3}
  #4
  \end{tablePart}%
%  \@openrightfalse
  \@mainmattertrue\pagestyle{main}
  \endgroup
}
%%%%% PART FOR APPENDIX %%%%%%%%%%%%%%%%%%%%%%%%%%%%%%%%%%%%%%%%%
\newcommand{\Appendixpart}{
  \clearrightpage
  \part*{\partname\ \thepart\hx の補遺\label{Apart:\thepart}}
  \addcontentsline{toc}{part}{\partname\ \thepart\hx の補遺}
}
%%%%%%%%%%%%%%%%%%%%%%%%%%%%%%%%%%%%%%%%%%%%%%
%%%%% AUTO LABELING %%%%%%%%%%%%%%%%%%%%%%%%%%
%%%%%%%%%%%%%%%%%%%%%%%%%%%%%%%%%%%%%%%%%%%%%%
%%%%% FOR CHAPTER OR APPENDIX %%%%%%%%%%%%%%%%%%%%%%%%%%%%%
\newcommand{\modHeadchapter}[2][]{%
  \clearrightpage\clearleftpage
  \let\newpage\relax
  \ifx\@chapapp\appendixname
    \chapter{#2\label{app:\thepart.\thechapter}}%
  \else
    \chapter{#2\label{chap:\thepart.\thechapter}}%
  \fi
  \ifx#1\relax\else%
    \foreach \x in {#1}{\addcontentsline{\x}{chapter}{\protect\numberline{\thechapter}#2}}%
  \fi
  \thispagestyle{main}%
  \indentspace%
  \let\newpage\tmpnewpage
}
%%%%% FOR SECTION %%%%%%%%%%%%%%%%%%%%%%%%%%%%%
\newcommand{\modHeadsection}[2][]{%
  \ifx\relax#1\relax
    \section{#2\label{sec:\thepart.\thesection}}%
  \else%
    \section[#1]{#2\label{sec:\thepart.\thesection}}%
  \fi
}
%%%%% CAPTOINOF %%%%%%%%%%%%%%%%%%%%%%%%%%%%%%
\setlength{\abovecaptionskip}{0pt}
\newcommand{\modcaptionof}[2]{%
  \captionof{#1}{%
    \csname #1name\endcsname\thechapter.%
    \ifnum\value{#1}<10 0\fi
      \arabic{#1}. #2}}
%%%%%%%%%%%%%%%%%%%%%%%%%%%%%%%%%%%%%%%%%%%%%%
%%%%% NEW ENVIRONMENT %%%%%%%%%%%%%%%%%%%%%%%%
%%%%%%%%%%%%%%%%%%%%%%%%%%%%%%%%%%%%%%%%%%%%%%
%!TEX root = ../RPA_for_Creating_Program_Note.tex


%%%%%%%%%%%%%%%%%%%%%%%%%%%%%%%%%%%%%%%%%%%%%%
%%%%% NEW ENVIRONMENT %%%%%%%%%%%%%%%%%%%%%%%%
%%%%%%%%%%%%%%%%%%%%%%%%%%%%%%%%%%%%%%%%%%%%%%
%%%%% COMMONTABLE %%%%%%%%%%%%%%%%%%
\newenvironment{commonTable}[5]{%
  \rowcolors{3}{#1}{white}
  \if\relax\detokenize{#2}\relax
  \else
%    \setlength{\abovecaptionskip}{0pt}
    \modcaptionof{table}{#2} % 追加したキャプション
    \addtocounter{table}{-1}
  \fi
  \setlength{\LTpre}{3pt}%
  \setlength{\LTpost}{5pt}%
  \setlength{\LTleft}{0pt}%
  \setlength{\LTright}{\fill}%
  \begin{longtable}{#3}
  \hline
  \rowcolor{orange!20}
  #4\\
  \hline
  \endfirsthead
  \hline
  \rowcolor{orange!20}
  #4\\
  \hline
  \endhead
  \hline
  \multicolumn{#5}{|r|}{\scriptsize 次ページへ続く} \\
  \hline
  \endfoot
  \hline
  \endlastfoot
}
{%
  \end{longtable}
}
%%%%% 2columnstable %%%%%
\newenvironment{2columnstable}[5][gray!15!]{%
  \setlength\cellspacetoplimit{4.5pt}%
  \setlength\cellspacebottomlimit{4.5pt}%
  \commonTable{#1}{#2}{#3}{\textbf{#4} & \textbf{#5}}{2}%
}
{\endcommonTable}
%%%%% 3columnstable %%%%%
\newenvironment{3columnstable}[6][gray!15!]{%
  \setlength\cellspacetoplimit{4.5pt}%
  \setlength\cellspacebottomlimit{4.5pt}%
  \commonTable{#1}{#2}{#3}{\textbf{#4} & \textbf{#5} & \textbf{#6}}{3}%
}
{\endcommonTable}
%%%%% 4columnstable %%%%%
\newenvironment{4columnstable}[7][gray!15!]{%
  \setlength\cellspacetoplimit{4.5pt}%
  \setlength\cellspacebottomlimit{4.5pt}%
  \commonTable{#1}{#2}{#3}{\textbf{#4} & \textbf{#5} & \textbf{#6} & \textbf{#7}}{4}%
}
{\endcommonTable}

%%%%% NEWTCOLORBOX %%%%%%%%%%%%%%%%%%%%%%%%%%%
\input{./RfCPN_a0_preamble/RfCPN_definitions_newtcolorboxes}
%%%%%%%%%%%%%%%%%%%%%%%%%%%%%%%%%%%%%%%%%%%%%%
%%%%% NEWTCBOX %%%%%%%%%%%%%%%%%%%%%%%%%%%%%%
%%%%%%%%%%%%%%%%%%%%%%%%%%%%%%%%%%%%%%%%%%%%%%
\newtcbox{\prgbox}{
  enhanced,
  nobeforeafter,
  tcbox raise base,
  boxrule=0.4pt,
  top=0mm,
  bottom=0mm,
  right=0mm,
  left=4mm,
  arc=1pt,
  boxsep=3pt,
  left skip=2pt,
  right skip=2pt,
  before upper={\vphantom{dl}},
  colframe=blue!50!black,
  coltext=blue!25!black,
  colback=blue!10!white,
  fontupper=\ttfamily,
  overlay={%
    \begin{tcbclipinterior}\fill[blue!75!green!50!white] (frame.south west)
    rectangle node[text=white, font=\sffamily\bfseries\tiny, rotate=90] {Prg} ([xshift=4mm]frame.north west);\end{tcbclipinterior}%
  },
}
\robustify{\prgbox}
\pdfstringdefDisableCommands{%
  \def\prgbox#1{`#1'}%
}
%%%%% OTHER TIKZ DEFINITION %%%%%%%%%%%%%%%%%%
\def\pgfname{\textsc{pgf}}
\def\tikzname{Ti\textit{k}Z}
\tikzfading[name=fade ball, inner color=transparent!60, outer color=transparent!30]
\def\sball#1{\tikz \shade [ball color=#1, path fading=fade ball] (0,0) circle (.7ex);}
\def\terminal#1#2{\tikz[baseline=(a.base)] \node (a) [terminal, bottom color=#2] {\small #1};}
\def\termblue#1{\terminal{\color{blue}\fontsize{8pt}{0pt}\textbf{#1}}{gray!25}\hskip6pt}
%%%%%%%%%%%%%%%%%%%%%%%%%%%%%%%%%%%%%%%%%%%%%%
%%%%% DECLARE %%%%%%%%%%%%%%%%%%%%%%%%%%%%%%%%
%%%%%%%%%%%%%%%%%%%%%%%%%%%%%%%%%%%%%%%%%%%%%%
%%%%% DECLARENEWTOC %%%%%%%%%%%%%%%%%%%%
\DeclareNewTOC[owner=\jobname, name=Part]{lop}
\DeclareNewTOC[owner=\jobname, name=Column]{loC}
%%%%% DECLARENEWLAYER %%%%%%%%%%%%%%%%%%%%
\newcommand{\setallpageWatermark}[4]{%
  \DeclareNewLayer[
    foreground,
    page,
    contents={%
      \begin{tikzpicture}[remember picture, overlay]
        \node[rotate=#2, scale=#3, anchor=center, opacity=#4, text=lightgray] at (current page.center){\scshape#1};%
      \end{tikzpicture}%
    }%
  ]{WatermarkLayer}
}
%%%%%%%%%%%%%%%%%%%%%%%%%%%%%%%%%%%%%%%%%%%%%%
%%%%% OTHER DEFINITION %%%%%%%%%%%%%%%%%%%%%%%
%%%%%%%%%%%%%%%%%%%%%%%%%%%%%%%%%%%%%%%%%%%%%%
%%%%% LINK %%%%%%%%%%%%%%%%%%%%%%%%%%%%%%%%%%%
%\newcommand\nextsectionlink[1]{\addtocounter{section}\@ne
%                               \hyperlink{section.\thechapter.\the\c@section}{#1}%
%                               \addtocounter{section}{-\@ne}}
%\newcommand\previoussectionlink[1]{\addtocounter{section}{-\@ne}
%                                   \hyperlink{section.\thechapter.\the\c@section}{#1}%
%                                   \addtocounter{section}{\@ne}}
%\newcommand\previouschapterlink[1]{\addtocounter{chapter}{-\@ne}
%                                   \hyperlink{chapter.\the\c@chapter}{#1}%
%                                   \addtocounter{chapter}{\@ne}}
%%%%% TO GET CHAPTER TITLE %%%%%%
\newcommand\Chaptername{} % initialize \Chaptername
\let\old@chapter\@chapter
\def\@chapter[#1]#2{\gdef\Chaptername{#2}\old@chapter[#1]{#2}}
%%%%% TO GET SECTION TITLE %%%%%%
\newcommand\Sectionname{} % initialize \Sectionname
\let\Sectionmark\sectionmark
\def\sectionmark#1{\def\Sectionname{#1}\Sectionmark{#1}}

%%%%% BASIC %%%%%%%%%%%%%%%%%%%%%%%%%%%%%
%!TEX root = ../RPA_for_Creating_Program_Note.tex


%%%%%%%%%%%%%%%%%%%%%%%%%%%%%%%%%%%%%%%%%%%%%%
%%%%% GEOMETRY %%%%%%%%%%%%%%%%%%%%%%%%%%%%%%%
%%%%%%%%%%%%%%%%%%%%%%%%%%%%%%%%%%%%%%%%%%%%%%
\geometry{
  a4paper, % paper size
  centering,
  textwidth={6.5in},
  includehead,  % include the head of the page
%  headheight = 13.6pt,
  includefoot,  % include the foot of the page
  top=15.0truemm,
  bottom=-0.5truemm,
}
%%%%%%%%%%%%%%%%%%%%%%%%%%%%%%%%%%%%%%%%%%%%%%
%%%%% HEYPERSETUP %%%%%%%%%%%%%%%%%%%%%%%%%%%%
%%%%%%%%%%%%%%%%%%%%%%%%%%%%%%%%%%%%%%%%%%%%%%
\hypersetup{
%  pdfcreationdate=date,
%  pdfcreator={upLaTeX with hyperref}, % creator for PDF subjct field
  pdftitle={RPAに向けたRDBSの構築と応用}, % title for PDF subjct field
  pdfsubject={プログラム作成業務の自動化}, % text for PDF subjct field
  pdfauthor={Kurahashi Nobuaki},  % text for PDF Author field
  pdfkeywords={
    mold,
    mould,
    RDB,
    RDBS,
    RPA,
  }, % keywords
%  pdfproducer=producer, % dvipdfmx
  linktoc=all,
%  linktocpage=false,   % (if it is true) make page number, not text, be link on TOC, LOF and LOT
  pdfcenterwindow=false, % position the document window center of the screen
  pdffitwindow=true,     % resize document window to fit document size
  bookmarksnumbered=true,
  bookmarksopen=true, %bookmarks open
  pdfstartview={FitH}, % Fit, FitV, FitH, FitB
  pdfpagemode=UseThumbs, % set default mode of PDF display
  unicode=true,
  pdfencoding=unicode,   % PDFDocEncoding or Unicode
  colorlinks=true,     % color links
  linkcolor=ai,      % color of links
  urlcolor=ai,         % color of urls
  citecolor=sora,      % color of citation links
}
%%%%%%%%%%%%%%%%%%%%%%%%%%%%%%%%%%%%%%%%%%%%%%
%%%%% DISPLAYBREAK %%%%%%%%%%%%%%%%%%%%%%%%%%%
%%%%%%%%%%%%%%%%%%%%%%%%%%%%%%%%%%%%%%%%%%%%%%
\allowdisplaybreaks
%%%%%%%%%%%%%%%%%%%%%%%%%%%%%%%%%%%%%%%%%%%%%%
%%%%% UNIT LENGTH %%%%%%%%%%%%%%%%%%%%%%%%%%%%
%%%%%%%%%%%%%%%%%%%%%%%%%%%%%%%%%%%%%%%%%%%%%%
\setlength{\unitlength}{1pt}
%%%%%%%%%%%%%%%%%%%%%%%%%%%%%%%%%%%%%%%%%%%%%%
%%%%% LINESPREAD %%%%%%%%%%%%%%%%%%%%%%%%%%%%%
%%%%%%%%%%%%%%%%%%%%%%%%%%%%%%%%%%%%%%%%%%%%%%
\linespread{1.15}\selectfont
%%%%%%%%%%%%%%%%%%%%%%%%%%%%%%%%%%%%%%%%%%%%%%
%%%%% PARINDENT %%%%%%%%%%%%%%%%%%%%%%%%%%%%%%
%%%%%%%%%%%%%%%%%%%%%%%%%%%%%%%%%%%%%%%%%%%%%%
\newcommand{\indentspace}{\setlength\parindent{11pt}}
\indentspace
%%%%% EQUATION %%%%%%%%%%%%%%%%%%%%%%%%%%%%
\renewcommand{\theequation}{\thesection.\arabic{equation}}
\@addtoreset{equation}{section}

%%%%% TITLE %%%%%%%%%%%%%%%%%%%%%%%%%%%%%
%!TEX root = ../RPA_for_Creating_Program_Note.tex

\titlehead{\hfill\small\customdate}

\subject{\Large--- プログラム作成業務の自動化 ---}

\title{\huge RPA実現に向けたRDBSの設計と構築}

%\subtitle{\ \\ソフトウェア視点による作業管理}

\date{}

\publishers{}

\author{\ \\\small Kurahashi Nobuaki}

\uppertitleback{\small
This document was created using \TeX{} (\linkLaTeX\kern.15em2$_{\textstyle\varepsilon}$), specifically utilizing tools such as \linkTeXLive{} 2023, up\LaTeX, \linkBibLaTeX\ (\linkBiber), \linkupmendex, \linkPGFTikZ, and many useful packages.\\
The document was edited using \linkTeXStudio{} and \linkSumatraPDF.\\
Numerical calculations were performed using \linkExcel{} and \linkPython.\\
The source codes for the G-code programs were written using \linkVSCode.\\
Version and issue control for these documents were managed using \linkGitHub.\\
%The environment for these tools was managed using \linkDocker{} and \linkUbuntu.\\
%The database used was \linkSQLite.\\
Thanks to these tools, with the all-around support of \linkChatGPT-4 (\linkMicrosoftCopilot), the creation of the document and system was made possible,
even while navigating solo and finding my way in the quiet corners.
}

\lowertitleback{{\scriptsize\relax%
\begin{spacing}{1.2}%
\setlength{\tabcolsep}{1.8pt}%
\hfill%
\begin{tabular}{rl}
\TeX{} and \linkLaTeX & are trademarks of the \linkAMS.\\
%\linkTeXLive & is developed by Karl Berry and \linkTeXUsersGroup.\\
%\linkBibLaTeX & is developed by PhilippLehman and Fran\c cois Charette.\\
%\linkPGFTikZ & are developed by TillTantau.\\
%\linkTeXStudio & is developed by Benito van der Zander.\\
%\linkSumatraPDF & is developed by KrzysztofKowalczyk and SimonB\"unzli.\\
\linkExcel, \linkVSCode{} and \linkMicrosoftCopilot & are trademarks of \linkMicrosoftCorp.\\
\linkPython & is a trademark of \linkPythonSF.\\
\linkGitHub & is a trademark of \linkGitHubInc.\\
\linkDocker & is a trademark of \linkDockerInc.\\
\linkUbuntu & is a trademark of \linkCanonicalLtd.\\
\linkSQLite & is a trademark of \linkSQLiteConsortium.\\
\linkChatGPT & is a trademark of \linkOpenAI.
\end{tabular}\\
\hrulefill\\
Copyright © 2023.\\
This document (except for bundled programs) and the system are owned by the individual writer, not any corporation. All rights reserved.
\end{spacing}
\thispagestyle{emptydate}
}}
%%%%% PAGESTYLES %%%%%%%%%%%%%%%%%%%%%%%%%%%%%
\input{./RfCPN_a0_preamble/RfCPN_styles_pagestyles}
%%%%% FOR STYLE OF TOC %%%%%%%%%%%%%%%%%%%%%%%
\input{./RfCPN_a0_preamble/RfCPN_styles_list_toc}
%%%%% FOR STYLE OF OTHER LISTS %%%%%%%%%%%%%%%
\input{./RfCPN_a0_preamble/RfCPN_styles_list_other}
%%%%% FOR STYLE OF COMPONENTS %%%%%%%%%%%%%%%
\input{./RfCPN_a0_preamble/RfCPN_styles_components}
%%%%% FOR STYLE OF LISTLISTINGS %%%%%%%%%%%%%%%
\input{./RfCPN_a0_preamble/RfCPN_styles_listlistings}

%%%%%%%%%%%%%%%%%%%%%%%%%%%%%%%%%%%%%%%%%%%%%%
%%%%% FOR FOOTNOTE %%%%%%%%%%%%%%%%%%%%%%%%%%%
%%%%%%%%%%%%%%%%%%%%%%%%%%%%%%%%%%%%%%%%%%%%%%
\renewcommand*{\footnoteautorefname}{脚注}
\interfootnotelinepenalty=10000
\counterwithout{footnote}{chapter}
\def\@makefnmark{\hbox{}\hbox{\@textsuperscript{\normalfont\@thefnmark}}\hbox{}}
\deffootnote[1em]{1em}{1em}{\textsuperscript{\thefootnotemark}}
\renewcommand\footnoterule{%
  \kern3pt
  \hrule\@width.75\columnwidth
  \kern2.6pt
}
\makesavenoteenv{longtable}
\makesavenoteenv{tablePart}
\makesavenoteenv{Column}
%%%%%%%%%%%%%%%%%%%%%%%%%%%%%%%%%%%%%%%%%%%%%%
%%%%% SETLIST %%%%%%%%%%%%%%%%%%%%%%%%%%%%%%%%
%%%%%%%%%%%%%%%%%%%%%%%%%%%%%%%%%%%%%%%%%%%%%%
\setlist[enumerate]{listparindent=\parindent, parsep=0pt, partopsep=0pt, topsep=3pt, itemsep=3pt, leftmargin=*}
\setlist[enumerate, 1]{leftmargin=\leftmargini}
%%%%%%%%%%%%%%%%%%%%%%%%%%%%%%%%%%%%%%%%%%%%%%
%%%%% CAPTION STYLE %%%%%%%%%%%%%%%%%%%%%%%%%%
%%%%%%%%%%%%%%%%%%%%%%%%%%%%%%%%%%%%%%%%%%%%%%
\captionsetup[figure]{%
  width=.8\textwidth,
  format=hang,
  labelfont={bf, sf},
  labelsep={colon},
  labelformat=simple,
%  font={small},
}
\captionsetup[lstlisting]{
  justification=raggedright,
  singlelinecheck=false,
  position=above,
  aboveskip=1.1pt,
  belowskip=0pt,
  labelformat={empty},
  labelfont={bf, sf},
  labelsep={space},
  font={bf, large, sf},
  hypcap=false,
}
\captionsetup[table]{
  justification=raggedright,
  singlelinecheck=false,
  position=above,
  aboveskip=4pt,
  belowskip=5pt,
  labelformat={empty},
  labelfont={bf, sf},
  labelsep={space},
  font={bf, large, sf},
  hypcap=false,
}
%%%%%%%%%%%%%%%%%%%%%%%%%%%%%%%%%%%%%%%%%%%%%%
%%%%% FOR STYLE OF TIKZSET %%%%%%%%%%%%%%%%%%%
%%%%%%%%%%%%%%%%%%%%%%%%%%%%%%%%%%%%%%%%%%%%%%
\tikzset{
  %%%%% SECTIONFORMAT STYLE %%%%%
%  sect/.style={signal, draw, text=white},
%  section/.style={sect, fill=konpeki!100!, signal to=east, inner sep=3pt},
%  subsection/.style={sect, fill=moegi!90!, signal to=nowhere, inner sep=3pt},
%  subsubsection/.style={sect, fill=sssec!100!, signal to=nowhere, inner sep=3pt},
  %%%%% TERMINAL STYLE %%%%%
  terminal/.style={%
    rectangle,%
    minimum size=10pt,%
    rounded corners=1.5mm,%
    thin,%
    draw=black!75,%
    top color=white,%
    font=\fontfamily{pplx},%
    inner sep=3pt,%
    inner xsep=3pt,%
    text height=1ex,%
    text depth=0pt,%
  },
  %%%%% BMATRIX STYLE %%%%%
%  every left delimiter/.style={xshift=.5em},
%  every right delimiter/.style={xshift=-.5em},
%  bmatrix/.style={matrix of math nodes, left delimiter=[, right delimiter=],},
}
%%%%% FOR STYLE OF TCBSET %%%%%%%%%%%%%%%%%%%%
%!TEX root = ../RPA_for_Creating_Program_Note.tex


%%%%%%%%%%%%%%%%%%%%%%%%%%%%%%%%%%%%%%%%%%%%%%
%%%%% FOR STYLE OF TCBSET %%%%%%%%%%%%%%%%%%%%
%%%%%%%%%%%%%%%%%%%%%%%%%%%%%%%%%%%%%%%%%%%%%%
\tcbset{%
  %%%%% COLUMNBOX STYLE %%%%%
  Tabularbox/.style={%
    fonttitle=\gtfamily\bfseries,%
    breakable,%
    enhanced jigsaw,%
    left=.5ex,%
    right=.5ex,%
    bicolor,%
    colbacklower=black!10!white,%
    before upper={%
      \setcounter{GlobalFootnote}{\value{footnote}}% GlobalFootnoteValue=footnoteValue
      \let\oldfootnote=\footnote% \oldfootnote=\footnote
      \def\footnote{\stepcounter{GlobalFootnote}\oldfootnote[\arabic{GlobalFootnote}]}% define \footnote to \footnote using counter GlobalFootnote
      \renewcommand\thempfootnote{\arabic{mpfootnote}}% arabic footnote
    },
    after upper={%
      \setcounter{footnote}{\value{GlobalFootnote}}% footnoteValue=GlobalFootnoteValue
      \let\footnote=\oldfootnote% \footnote=\oldfootnote
    },
    after=\smallskip\noindent,
  },%
  %%%%% COLUMNBOX STYLE %%%%%
  Columnbox/.style={%
    Tabularbox,
    colback=black!02!white!,%
    after title=\hfill\termblue{\Columnname~\thetcbcounter},%
  },%
  %%%%% FORMULA STYLE %%%%%
  Formulabox/.style={%
    Tabularbox,
    after title=\hfill\termblue{\Formulaname~\thetcbcounter},%
  },%
  %%%%% FIGUREBOX STYLE %%%%%
  Figurebox/.style={%
    notitle,%
    height=\textwidth,
%    width=\textwidth,
    center upper,%
    center lower,%
    arc=5pt,%
    outer arc=2pt,%
    boxrule=1pt,%
    boxsep=3mm,%
    valign=center,%
    halign=center,%
    left=0pt,%
    right=0pt,%
    colback=green!3!white,%
    colframe=black!25!white,%
    before={\centering},
  },
  %%%%% HIGHLIGHT MATH STYLE %%%%%
%  highlight math/.style={%
%    enhanced,%
%    arc=2pt,%
%    boxrule=0pt,%
%    frame hidden,%
%    fuzzy halo=1pt with blue,%
%    left=0pt,%
%    right=0pt,%
%    top=.4mm,%
%    bottom=.4mm,%
%    colback=yellow!40!white,%
%  },%
  %%%%% HOSOKUBOX STYLE %%%%%
  hosokubox/.style={%
    title={\termblue{\hosokuname~\thetcbcounter}~},%
    attach title to upper,%
    breakable,%
    enhanced jigsaw,%
    size=fbox,%
    arc=0pt,%
    middle=1mm,%
    colback=hosoku,%
    colframe=hosoku,%
    drop lifted shadow={blue!100!white!50!},%
    skin first is subskin of={enhanced jigsaw}{no shadow},%
    skin middle is subskin of={enhanced jigsaw}{no shadow},%
    skin last is subskin of={enhanced jigsaw}{drop lifted shadow={blue!100!white!50!}},%
    segmentation style={draw=black!50!white},%
    after=\smallskip\noindent{\color{white}},%
  },
  %%%%% ISSUEBOX STYLE %%%%%
  Issuebox/.style={%
    Tabularbox,
    colback=black!02!white!,%
    after title=\hfill\termblue{\issuename},%
  },%
  %%%%% LIST COMMON STYLE %%%%%
  listcommonstyle/.style={%
    breakable,%
    enhanced jigsaw,%
    before=\par\bigskip\noindent,%
    enlargepage flexible=\baselineskip,%
    pad at break*=3mm,%
    before skip=0mm,%
    top=4mm,%
    middle=4.5mm,
    height fixed for=first and middle,%
    drop fuzzy shadow,
  },
  %%%%% TOCSTYLE STYLE %%%%%
  tocstyle/.style={%
    listcommonstyle,
    colback=green!0.5!white,%
    colframe=green!40!black!60!,%
    watermark color=green!60!yellow!13!white,%
  },
  %%%%% LOCSTYLE STYLE %%%%%
  lofstyle/.style={%
    listcommonstyle,
    colback=cyan!1!white,%
    colframe=cyan!80!black!100!,%
  },
  %%%%% LOTSTYLE STYLE %%%%%
  lotstyle/.style={lofstyle},
  %%%%% LOTSTYLE STYLE %%%%%
  lolstyle/.style={lofstyle},
  %%%%% LOCSTYLE STYLE %%%%%
  loCstyle/.style={lofstyle},
  %%%%% LOPSTYLE STYLE %%%%%
  lopstyle/.style={%
    listcommonstyle,
    fonttitle=\sffamily\bfseries\large,%
    colback=yellow!2!white,%
    colframe=red!40!black!60!,%
    colbacktitle=red!50!yellow!70!black!60!,%
    top=8mm,
    bottom=4mm,
    after skip=5mm,
    attach boxed title to top left={%
      xshift=10mm,%
      yshift=-0.25mm-\tcboxedtitleheight/2,%
      yshifttext=2mm-\tcboxedtitleheight/2,%
    },
    boxed title style={%
      enhanced,%
      boxrule=0.5mm,%
      frame code={%
        \path[tcb fill frame] ([xshift=-4mm]frame.west) -- (frame.north west)
        -- (frame.north east) -- ([xshift=4mm]frame.east)
        -- (frame.south east) -- (frame.south west) -- cycle;
      },
      interior code={
        \path[tcb fill interior] ([xshift=-2mm]interior.west)
        -- (interior.north west) -- (interior.north east)
        -- ([xshift=2mm]interior.east) -- (interior.south east) -- (interior.south west)
        -- cycle;
      },
    },
  },
}

%%%%%%%%%%%%%%%%%%%%%%%%%%%%%%%%%%%%%%%%%%%%%%
%%%%% NEWTBLRTHEME %%%%%%%%%%%%%%%%%%%%%%%%%%%
\NewTblrTheme{commontblr}{
}
%%%%%%%%%%%%%%%%%%%%%%%%%%%%%%%%%%%%%%%%%%%%%%
%%%%%%%%%%%%%%%%%%%%%%%%%%%%%%%%%%%%%%%%%%%%%%
%%%%% OTHER %%%%%%%%%%%%%%%%%%%%%%%%%%%%%%%%%%
%%%%%%%%%%%%%%%%%%%%%%%%%%%%%%%%%%%%%%%%%%%%%%
\setlength\baselineskip{16pt}
\setlength\normalbaselineskip{\baselineskip}
\let\cleardoublepage\tmpcleardoublepage
\let\clearpage\tmpclearpage
\makeatother

\usepackage{refcheck}
\newcommand{\thisdocumentversion}{version 0.1.0}
\newcommand{\dateTourokuKougu}{2024/02\hx}
\newcommand{\dateKouguSpeed}{2024/02/16\hx}
\newcommand{\dateKouguRotation}{2024/02/16\hx}
\newcommand{\dateUnusedVariables}{2024/02/16\hx}
\setallpageWatermark{Copy}{atan(297/210)+5.1}{32.1}{0.175}
%\setallpageWatermark{Draft}{atan(297/210)+5.1}{25.3}
%\RequirePackage{plautopatch}

\begin{document}
%%%%%%%%%%%%%%%%%%%%%%%%%%%%%%%%%%%%%%%%%%%%%%%%%%%%%%%%%%%%%%%%%%%%
%%       %%%%%%%%%%%%%%%%%%%%%%%%%%%%%%%%%%%%%%%%%%%%%%%%%%%%%%%%%%%
%% TITLE %%%%%%%%%%%%%%%%%%%%%%%%%%%%%%%%%%%%%%%%%%%%%%%%%%%%%%%%%%%
%%       %%%%%%%%%%%%%%%%%%%%%%%%%%%%%%%%%%%%%%%%%%%%%%%%%%%%%%%%%%%
%%%%%%%%%%%%%%%%%%%%%%%%%%%%%%%%%%%%%%%%%%%%%%%%%%%%%%%%%%%%%%%%%%%%
\addtocounter{page}{1}
\let\cleardoublepage\relax
\maketitle
\let\cleardoublepage\tmpcleardoublepage



%%%%%%%%%%%%%%%%%%%%%%%%%%%%%%%%%%%%%%%%%%%%%%%%%%%%%%%%%%%
%%             %%%%%%%%%%%%%%%%%%%%%%%%%%%%%%%%%%%%%%%%%%%%
%%             %%%%%%%%%%%%%%%%%%%%%%%%%%%%%%%%%%%%%%%%%%%%
%% FRONTMATTER %%%%%%%%%%%%%%%%%%%%%%%%%%%%%%%%%%%%%%%%%%%%
%%             %%%%%%%%%%%%%%%%%%%%%%%%%%%%%%%%%%%%%%%%%%%%
%%             %%%%%%%%%%%%%%%%%%%%%%%%%%%%%%%%%%%%%%%%%%%%
%%%%%%%%%%%%%%%%%%%%%%%%%%%%%%%%%%%%%%%%%%%%%%%%%%%%%%%%%%%
\frontmatter



%%%%%%%%%%%%%%%%%%%%%%%%%%%%%%%%%%%%%%%%%%%%%%%%%%%%%%%%%%%%%%%%%%%%
%%               %%%%%%%%%%%%%%%%%%%%%%%%%%%%%%%%%%%%%%%%%%%%%%%%%%%
%% LIST OF PARTS %%%%%%%%%%%%%%%%%%%%%%%%%%%%%%%%%%%%%%%%%%%%%%%%%%%
%%               %%%%%%%%%%%%%%%%%%%%%%%%%%%%%%%%%%%%%%%%%%%%%%%%%%%
%%%%%%%%%%%%%%%%%%%%%%%%%%%%%%%%%%%%%%%%%%%%%%%%%%%%%%%%%%%%%%%%%%%%
\setcounter{page}{2}
\begingroup
\let\cleardoublepage\relax
\renewcommand{\contentsname}{\partcontentsname}
\phantomsection
\addchaptertocentry{}{\listoflopname}
\listoftoc{lop} % insert lop
\let\cleardoublepage\tmpcleardoublepage
\endgroup



%%%%%%%%%%%%%%%%%%%%%%%%%%%%%%%%%%%%%%%%%%%%%%%%%%%%%%%%%%%%%%%%%%%%
%%                   %%%%%%%%%%%%%%%%%%%%%%%%%%%%%%%%%%%%%%%%%%%%%%%
%% TABLE OF CONTENTS %%%%%%%%%%%%%%%%%%%%%%%%%%%%%%%%%%%%%%%%%%%%%%%
%%                   %%%%%%%%%%%%%%%%%%%%%%%%%%%%%%%%%%%%%%%%%%%%%%%
%%%%%%%%%%%%%%%%%%%%%%%%%%%%%%%%%%%%%%%%%%%%%%%%%%%%%%%%%%%%%%%%%%%%
\clearrightpage\clearpage
\phantomsection
\addchaptertocentry{}{\contentsname}
\tableofcontents % insert toc



%%%%%%%%%%%%%%%%%%%%%%%%%%%%%%%%%%%%%%%%%%%%%%%%%%%%%%%%%%%%%%%%%%%%
%%                  %%%%%%%%%%%%%%%%%%%%%%%%%%%%%%%%%%%%%%%%%%%%%%%%
%% LIST OF TABLES   %%%%%%%%%%%%%%%%%%%%%%%%%%%%%%%%%%%%%%%%%%%%%%%%
%%                  %%%%%%%%%%%%%%%%%%%%%%%%%%%%%%%%%%%%%%%%%%%%%%%%
%%%%%%%%%%%%%%%%%%%%%%%%%%%%%%%%%%%%%%%%%%%%%%%%%%%%%%%%%%%%%%%%%%%%
\clearrightpage
\phantomsection
\addchaptertocentry{}{\listtablename}
\listoftables  % insert lot



%%%%%%%%%%%%%%%%%%%%%%%%%%%%%%%%%%%%%%%%%%%%%%%%%%%%%%%%%%%%%%%%%%%%
%%                  %%%%%%%%%%%%%%%%%%%%%%%%%%%%%%%%%%%%%%%%%%%%%%%%
%% LIST OF LISTINGS %%%%%%%%%%%%%%%%%%%%%%%%%%%%%%%%%%%%%%%%%%%%%%%%
%%                  %%%%%%%%%%%%%%%%%%%%%%%%%%%%%%%%%%%%%%%%%%%%%%%%
%%%%%%%%%%%%%%%%%%%%%%%%%%%%%%%%%%%%%%%%%%%%%%%%%%%%%%%%%%%%%%%%%%%%
\clearrightpage
\phantomsection
\addchaptertocentry{}{\lstlistlistingname}
\lstlistoflistings % insert lol



\PartSeparateline{toc}%
\addtocontents{toc}{\protect\vspace*{8mm}}
%%%%%%%%%%%%%%%%%%%%%%%%%%%%%%%%%%%%%%%%%%%%%%%%%%%%%%%%%%
%%            %%%%%%%%%%%%%%%%%%%%%%%%%%%%%%%%%%%%%%%%%%%%
%%            %%%%%%%%%%%%%%%%%%%%%%%%%%%%%%%%%%%%%%%%%%%%
%% MAINMATTER %%%%%%%%%%%%%%%%%%%%%%%%%%%%%%%%%%%%%%%%%%%%
%%            %%%%%%%%%%%%%%%%%%%%%%%%%%%%%%%%%%%%%%%%%%%%
%%            %%%%%%%%%%%%%%%%%%%%%%%%%%%%%%%%%%%%%%%%%%%%
%%%%%%%%%%%%%%%%%%%%%%%%%%%%%%%%%%%%%%%%%%%%%%%%%%%%%%%%%%
\mainmatter




%%%%%%%%%%%%%%%%%%%%%%%%%%%%%%%%%%%%%%%%%%%%%%%%%%%%%%%%%
%%                    %%%%%%%%%%%%%%%%%%%%%%%%%%%%%%%%%%%
%%                    %%%%%%%%%%%%%%%%%%%%%%%%%%%%%%%%%%%
%% Part Work Flow     %%%%%%%%%%%%%%%%%%%%%%%%%%%%%%%%%%%
%%                    %%%%%%%%%%%%%%%%%%%%%%%%%%%%%%%%%%%
%%                    %%%%%%%%%%%%%%%%%%%%%%%%%%%%%%%%%%%
%%%%%%%%%%%%%%%%%%%%%%%%%%%%%%%%%%%%%%%%%%%%%%%%%%%%%%%%%
\edef\temp{\protect{\the\numexpr\value{part}+1\relax}}
\addtocontents{toc}{\protect\begin{tcolorbox}[parttocstyle={Contents}{\temp}]}
\tPart{業務フロー}{概要}{%
\paragraph*{目標(なにがしたいか?)}
\textbf{ソフトウェアの視点}から、横型マシニングセンタにおける\textbf{現在の業務の流れを明確に}する。
それに基づき、新たなマシニング\textbf{導入後の業務の流れを明確に}する。

\tcbline*
\paragraph*{手段(どうやって?)}
関係者に詳しく聞き取りを行い、ソフトウェアの観点における現在の業務の流れを把握する。
それに基づいて解決可能な課題を抽出し、改善を行った上で導入後の業務の流れを取り決める。

\tcbline*
\paragraph*{背景(なぜ?)}
新たなマシニングについて、ハードウェア視点では大きな問題はないという形で設置に至った。

 しかしソフトウェアの観点から見ると、事態は惨憺たるものである。
「ソフトウェア関する管理・業務」を行う部門はおろか、担当者(特に管理職)さえ社内に存在しない。
そのため業務の流れすら体系的に把握されておらず、したがって\index{ぎょうむきてい@業務規程}業務規程や開発プロセスの計画も事実上皆無に等しい
%% footnote %%%%%%%%%%%%%%%%%%%%%
\footnote{つまり、ソフトウェアの観点からすると、もはや企業としての体をなしていない。}。
%%%%%%%%%%%%%%%%%%%%%%%%%%%%%%%%%

 このような状況のため、システム開発プロセスの最初期の段階である、\index{ぎょうむフロー@業務フロー}\textbf{業務フロー}の把握から初めなければならない。

\tcbline*
\paragraph*{結論(どうなった?)}
ソフトウェアの視点に立って\MMname における業務の流れを把握し、それに基づいて\DMname 導入後の業務の流れを取り決めた。
}
%!TEX root = ../RPA_for_Creating_Program_Note.tex


%%%%%%%%%%%%%%%%%%%%%%%%%%%%%%%%%%%%%%%%%%%%%%%%%%%%%%%%%
%% Part Work Flow     %%%%%%%%%%%%%%%%%%%%%%%%%%%%%%%%%%%
%%%%%%%%%%%%%%%%%%%%%%%%%%%%%%%%%%%%%%%%%%%%%%%%%%%%%%%%%
\addtocontents{toc}{\protect\begin{tocBox}{\tmppartnum}}%
\tPart{ソフトウェア視点における業務フロー}{概要}{%
\paragraph*{目標(なにがしたいか?)}
\index{よこがたマシニングセンタ@横型マシニングセンタ}横型マシニングセンタにおける作業において、\textbf{ソフトウェアの視点から現在の業務の流れを明確化}する。
それに基づき、新たなマシニングセンタ\textbf{導入後の業務の流れを計画}する。

\tcbline*
\paragraph*{手段(どうやって?)}
関係者に詳しく聞き取りを行い、ソフトウェアの観点における現在の業務の流れを把握する。
それに基づいて解決可能な課題を抽出し、改善を施した上で導入後の業務の流れを取り決める。

\tcbline*
\paragraph*{背景(なぜ?)}
新たなマシニングセンタについて、ハードウェア視点では大きな問題はないという形で設置に至った。

 一方、ソフトウェア視点においては、事態は惨憺たるものである。
「ソフトウェア関する管理・業務」を行う部門はおろか、担当者(特に管理職)さえ社内に存在しない。
そのため業務の流れすら体系的に把握されておらず、したがって\index{ぎょうむきてい@業務規程}業務規程や開発プロセスの計画も事実上皆無である
%% footnote %%%%%%%%%%%%%%%%%%%%%
\footnote{つまり、ソフトウェアの観点からすると、もはや企業としての体をなしていない。
事実に基づいて客観的に判断すると、著しくモラルが低いと言わざるを得ない状態にある。
なお、これは2024/03現在、全く事態は変わっていない。}。
%%%%%%%%%%%%%%%%%%%%%%%%%%%%%%%%%

 このように、客観的事実として、長期にわたり業務の中枢となる部分が全く機能していないため、システムの\index{かいはつプロセス@開発プロセス}開発プロセスの最初期の段階である\textbf{現在の\index{ぎょうむフロー@業務フロー}業務フローを把握}する段階から着手しなければならない。

\tcbline*
\paragraph*{結論(どうなった?)}
ソフトウェアの視点に立って現状の横型マシニングセンタにおける業務の流れを把握し、それに基づいて新たなマシニングセンタ導入後の業務の流れを取り決めた。
}

%%%%%%%%%%%%%%%%%%%%%%%%%%%%%%%%%%%%%%%%%%%%%%%%%%%%%%%%%%
%% chapter 1 %%%%%%%%%%%%%%%%%%%%%%%%%%%%%%%%%%%%%%%%%%%%%
%%%%%%%%%%%%%%%%%%%%%%%%%%%%%%%%%%%%%%%%%%%%%%%%%%%%%%%%%%
\modHeadchapter{横型マシニングにおける業務フロー}
%!TEX root = ../RPA_for_Creating_Program_Note.tex


\modHeadchapter{現状の業務フローの理解}
新たに導入する\index{よこがたマシニングセンタ@横型マシニングセンタ}横型マシニングセンタ(以下、\textbf{\DMname})での工程は、\dimple の測定・加工を除けば\MMname と同様である。
そこで、まずは\MMname ではどのようなフローで業務が行われているかを(ソフトウェアの観点から)みることにする。
%%%%%%%%%%%%%%%%%%%%%%%%%%%%%%%%%%%%%%%%%%%%%%%%%%%%%%%%%%
%% marker %%%%%%%%%%%%%%%%%%%%%%%%%%%%%%%%%%%%%%%%%%%%%%%%
%%%%%%%%%%%%%%%%%%%%%%%%%%%%%%%%%%%%%%%%%%%%%%%%%%%%%%%%%%
\begin{marker}
ここでは主に\MMname の\expandafterindex{No.1パレット(\MMname)@No.1パレット(\MMname)}No.1パレットで加工を行うものを対象とする
%% footnote %%%%%%%%%%%%%%%%%%%%%
\footnote{\expandafterindex{No.2パレット(\MMname)@No.2パレット(\MMname)}No.2パレットでは、\index{おおがたのモールド@大型のモールド}径の大きなものや\index{まるがたのモールド@丸型のモールド}丸形のもの等の加工が主に行われる。}。
%%%%%%%%%%%%%%%%%%%%%%%%%%%%%%%%%
\end{marker}
%%%%%%%%%%%%%%%%%%%%%%%%%%%%%%%%%%%%%%%%%%%%%%%%%%%%%%%%%%
%%%%%%%%%%%%%%%%%%%%%%%%%%%%%%%%%%%%%%%%%%%%%%%%%%%%%%%%%%
%%%%%%%%%%%%%%%%%%%%%%%%%%%%%%%%%%%%%%%%%%%%%%%%%%%%%%%%%%
%%%%%%%%%%%%%%%%%%%%%%%%%%%%%%%%%%%%%%%%%%%%%%%%%%%%%%%%%%
%% marker %%%%%%%%%%%%%%%%%%%%%%%%%%%%%%%%%%%%%%%%%%%%%%%%
%%%%%%%%%%%%%%%%%%%%%%%%%%%%%%%%%%%%%%%%%%%%%%%%%%%%%%%%%%
\begin{marker}
ここで挙げている必要なパラメータ(\index{すんぽう@寸法}寸法)には、その\index{こうさ@公差}公差も考慮されているものとする。
\end{marker}
%%%%%%%%%%%%%%%%%%%%%%%%%%%%%%%%%%%%%%%%%%%%%%%%%%%%%%%%%%
%%%%%%%%%%%%%%%%%%%%%%%%%%%%%%%%%%%%%%%%%%%%%%%%%%%%%%%%%%
%%%%%%%%%%%%%%%%%%%%%%%%%%%%%%%%%%%%%%%%%%%%%%%%%%%%%%%%%%



%%%%%%%%%%%%%%%%%%%%%%%%%%%%%%%%%%%%%%%%%%%%%%%%%%%%%%%%%%
%% section 1.1 %%%%%%%%%%%%%%%%%%%%%%%%%%%%%%%%%%%%%%%%%%%
%%%%%%%%%%%%%%%%%%%%%%%%%%%%%%%%%%%%%%%%%%%%%%%%%%%%%%%%%%
\modHeadsection{現在のマシニングセンタの操作方法\TBW}



%%%%%%%%%%%%%%%%%%%%%%%%%%%%%%%%%%%%%%%%%%%%%%%%%%%%%%%%%%
%% section 1.2 %%%%%%%%%%%%%%%%%%%%%%%%%%%%%%%%%%%%%%%%%%%
%%%%%%%%%%%%%%%%%%%%%%%%%%%%%%%%%%%%%%%%%%%%%%%%%%%%%%%%%%
\modHeadsection{使用ソフトウェアおよびツール\TBW}
\index{モールド}モールドの\index{たんめんかこう@端面加工}端面加工・\index{がいさくかこう@外削加工}外削加工・\index{ないさくかこう@内削加工}内削加工・\index{みぞかこう@溝加工}溝加工・\index{ざぐりかこう(たんめん)@座ぐり加工(端面)}端面の座ぐり加工・\index{そとがわCめんとりかこう(たんめん)@外側C面取加工(端面)}端面の外側C面取加工・\index{うちがわCめんとりかこう(たんめん)@内側C面取加工(端面)}端面の内側C面取等は主に三菱製横型マシニングセンタ(以下、\textbf{\MMname})にて行われている。



%%%%%%%%%%%%%%%%%%%%%%%%%%%%%%%%%%%%%%%%%%%%%%%%%%%%%%%%%%
%% section 1.3 %%%%%%%%%%%%%%%%%%%%%%%%%%%%%%%%%%%%%%%%%%%
%%%%%%%%%%%%%%%%%%%%%%%%%%%%%%%%%%%%%%%%%%%%%%%%%%%%%%%%%%
\modHeadsection{生産ラインの流れ\TBW}
\MMname において、ある\index{めいさい(モールド)@明細(モールド)}明細のモールドを加工をする際に、以下のような流れで作業が行われる。


%%%%%%%%%%%%%%%%%%%%%%%%%%%%%%%%%%%%%%%%%%%%%%%%%%%%%%%%%%
%% subsection 01.1.1 %%%%%%%%%%%%%%%%%%%%%%%%%%%%%%%%%%%%%
%%%%%%%%%%%%%%%%%%%%%%%%%%%%%%%%%%%%%%%%%%%%%%%%%%%%%%%%%%
\subsection{図面の確認}
\begin{enumerate}
\item 対象となる明細の\index{ずめん(モールド)@図面(モールド)}図面を用意する
\item 他に内容が類似する明細の図面があれば、それも併せて用意する
\end{enumerate}
%%%%%%%%%%%%%%%%%%%%%%%%%%%%%%%%%%%%%%%%%%%%%%%%%%%%%%%%%%
%% PARAMETER %%%%%%%%%%%%%%%%%%%%%%%%%%%%%%%%%%%%%%%%%%%%%
%%%%%%%%%%%%%%%%%%%%%%%%%%%%%%%%%%%%%%%%%%%%%%%%%%%%%%%%%%
\begin{Parameter}{必要なパラメータ}
\PMbox{図面の有無}%
\PMbox{図面番号}%
\end{Parameter}
%%%%%%%%%%%%%%%%%%%%%%%%%%%%%%%%%%%%%%%%%%%%%%%%%%%%%%%%%%
%%%%%%%%%%%%%%%%%%%%%%%%%%%%%%%%%%%%%%%%%%%%%%%%%%%%%%%%%%
%%%%%%%%%%%%%%%%%%%%%%%%%%%%%%%%%%%%%%%%%%%%%%%%%%%%%%%%%%


%%%%%%%%%%%%%%%%%%%%%%%%%%%%%%%%%%%%%%%%%%%%%%%%%%%%%%%%%%
%% subsection 01.1.2 %%%%%%%%%%%%%%%%%%%%%%%%%%%%%%%%%%%%%
%%%%%%%%%%%%%%%%%%%%%%%%%%%%%%%%%%%%%%%%%%%%%%%%%%%%%%%%%%
\subsection{加工部分の有無の確認}

%%%%%%%%%%%%%%%%%%%%%%%%%%%%%%%%%%%%%%%%%%%%%%%%%%%%%%%%%%
%% subsubsection 01.1.2.2 %%%%%%%%%%%%%%%%%%%%%%%%%%%%%%%%
%%%%%%%%%%%%%%%%%%%%%%%%%%%%%%%%%%%%%%%%%%%%%%%%%%%%%%%%%%
\subsubsection{端面部分}
\index{たんめんかこう@端面加工}端面の加工については、全明細に共通の形で存在する。

%%%%%%%%%%%%%%%%%%%%%%%%%%%%%%%%%%%%%%%%%%%%%%%%%%%%%%%%%%
%% subsubsection 01.1.2.2 %%%%%%%%%%%%%%%%%%%%%%%%%%%%%%%%
%%%%%%%%%%%%%%%%%%%%%%%%%%%%%%%%%%%%%%%%%%%%%%%%%%%%%%%%%%
\subsubsection{外削部分}
\index{がいさくかこう@外削加工}外削の加工については、明細により\index{がいさくのうむ@外削の有無}外削の有無または\index{がいさくのけいじょう@外削の形状}形状の違いが存在する。
\begin{enumerate}
\item トップ側またはボトム側の外削の有無を確認する
\item 外削の形状を確認し、使用する\index{こうぐ(がいさく)@工具(外削)}工具を決定する
\item \index{わんきょくにそったがいさく@湾曲に沿った外削}外削が湾曲に沿ったものかどうかも確認する
\end{enumerate}
%\clearpage
%%%%%%%%%%%%%%%%%%%%%%%%%%%%%%%%%%%%%%%%%%%%%%%%%%%%%%%%%%
%% PARAMETER %%%%%%%%%%%%%%%%%%%%%%%%%%%%%%%%%%%%%%%%%%%%%
%%%%%%%%%%%%%%%%%%%%%%%%%%%%%%%%%%%%%%%%%%%%%%%%%%%%%%%%%%
\begin{Parameter}{必要なパラメータ}
\PMbox{トップ外削の有無}%
\PMbox{ボトム外削の有無}%
\PMbox{トップ外削の形状}%
\PMbox{ボトム外削の形状}%
\end{Parameter}
%%%%%%%%%%%%%%%%%%%%%%%%%%%%%%%%%%%%%%%%%%%%%%%%%%%%%%%%%%
%%%%%%%%%%%%%%%%%%%%%%%%%%%%%%%%%%%%%%%%%%%%%%%%%%%%%%%%%%
%%%%%%%%%%%%%%%%%%%%%%%%%%%%%%%%%%%%%%%%%%%%%%%%%%%%%%%%%%

%\clearpage
%%%%%%%%%%%%%%%%%%%%%%%%%%%%%%%%%%%%%%%%%%%%%%%%%%%%%%%%%%
%% subsubsection 01.1.2.3 %%%%%%%%%%%%%%%%%%%%%%%%%%%%%%%%
%%%%%%%%%%%%%%%%%%%%%%%%%%%%%%%%%%%%%%%%%%%%%%%%%%%%%%%%%%
\subsubsection{溝部分}
\index{みぞかこう@溝加工}溝の加工については、全明細のトップ側に存在し、明細により形状の違いが存在する。
\begin{enumerate}
\item \index{みぞのけいじょう@溝の形状}溝の形状を確認し、使用する\index{サブプログラム(みぞ)@サブプログラム(溝)}サブプログラムの判断を行う
\item \index{みぞはば@溝幅}溝幅を確認し、使用する\index{こうぐ(みぞ)@工具(溝)}工具の判断を行う
\end{enumerate}
%%%%%%%%%%%%%%%%%%%%%%%%%%%%%%%%%%%%%%%%%%%%%%%%%%%%%%%%%%
%% PARAMETER %%%%%%%%%%%%%%%%%%%%%%%%%%%%%%%%%%%%%%%%%%%%%
%%%%%%%%%%%%%%%%%%%%%%%%%%%%%%%%%%%%%%%%%%%%%%%%%%%%%%%%%%
\begin{Parameter}{必要なパラメータ}
\PMbox{溝の形状}\PMbox{トップ外削の有無}\PMbox{溝幅}
\end{Parameter}
%%%%%%%%%%%%%%%%%%%%%%%%%%%%%%%%%%%%%%%%%%%%%%%%%%%%%%%%%%
%%%%%%%%%%%%%%%%%%%%%%%%%%%%%%%%%%%%%%%%%%%%%%%%%%%%%%%%%%
%%%%%%%%%%%%%%%%%%%%%%%%%%%%%%%%%%%%%%%%%%%%%%%%%%%%%%%%%%

%\clearpage
%%%%%%%%%%%%%%%%%%%%%%%%%%%%%%%%%%%%%%%%%%%%%%%%%%%%%%%%%%
%% subsubsection 01.1.2.4 %%%%%%%%%%%%%%%%%%%%%%%%%%%%%%%%
%%%%%%%%%%%%%%%%%%%%%%%%%%%%%%%%%%%%%%%%%%%%%%%%%%%%%%%%%%
\subsubsection{端面の面取部分}
\index{めんとりかこう(たんめん)@面取加工(端面)}端面の面取の加工については、全明細に存在し、明細により形状の違いが存在する。
\begin{enumerate}
\item 面取がC面取であれば、マシニングセンタによる加工を行うか判断を行う
\item \index{Cめんとり(たんめん)@C面取(端面)}C面取の角度を確認し、使用する\index{こうぐ(Cめんとり)@工具(C面取)}工具を決定する
\end{enumerate}
%%%%%%%%%%%%%%%%%%%%%%%%%%%%%%%%%%%%%%%%%%%%%%%%%%%%%%%%%%
%% PARAMETER %%%%%%%%%%%%%%%%%%%%%%%%%%%%%%%%%%%%%%%%%%%%%
%%%%%%%%%%%%%%%%%%%%%%%%%%%%%%%%%%%%%%%%%%%%%%%%%%%%%%%%%%
\begin{Parameter}{必要なパラメータ}
\PMbox{面取の形状}\PMbox{C面取長}\PMbox{C面取の角度}\PMbox{トップ外削の有無}
\end{Parameter}
%%%%%%%%%%%%%%%%%%%%%%%%%%%%%%%%%%%%%%%%%%%%%%%%%%%%%%%%%%
%%%%%%%%%%%%%%%%%%%%%%%%%%%%%%%%%%%%%%%%%%%%%%%%%%%%%%%%%%
%%%%%%%%%%%%%%%%%%%%%%%%%%%%%%%%%%%%%%%%%%%%%%%%%%%%%%%%%%

%\clearpage
%%%%%%%%%%%%%%%%%%%%%%%%%%%%%%%%%%%%%%%%%%%%%%%%%%%%%%%%%%
%% subsubsection 01.1.2.4 %%%%%%%%%%%%%%%%%%%%%%%%%%%%%%%%
%%%%%%%%%%%%%%%%%%%%%%%%%%%%%%%%%%%%%%%%%%%%%%%%%%%%%%%%%%
\subsubsection{端面の座ぐり部分\TBW}
(to be written...)
%%%%%%%%%%%%%%%%%%%%%%%%%%%%%%%%%%%%%%%%%%%%%%%%%%%%%%%%%%
%% PARAMETER %%%%%%%%%%%%%%%%%%%%%%%%%%%%%%%%%%%%%%%%%%%%%
%%%%%%%%%%%%%%%%%%%%%%%%%%%%%%%%%%%%%%%%%%%%%%%%%%%%%%%%%%
\begin{Parameter}{必要なパラメータ}
\PMbox{座ぐりの有無}
\end{Parameter}
%%%%%%%%%%%%%%%%%%%%%%%%%%%%%%%%%%%%%%%%%%%%%%%%%%%%%%%%%%
%%%%%%%%%%%%%%%%%%%%%%%%%%%%%%%%%%%%%%%%%%%%%%%%%%%%%%%%%%
%%%%%%%%%%%%%%%%%%%%%%%%%%%%%%%%%%%%%%%%%%%%%%%%%%%%%%%%%%


\clearpage
%%%%%%%%%%%%%%%%%%%%%%%%%%%%%%%%%%%%%%%%%%%%%%%%%%%%%%%%%%
%% subsection 01.1.3 %%%%%%%%%%%%%%%%%%%%%%%%%%%%%%%%%%%%%
%%%%%%%%%%%%%%%%%%%%%%%%%%%%%%%%%%%%%%%%%%%%%%%%%%%%%%%%%%
\subsection{加工部分の寸法の確認}

%%%%%%%%%%%%%%%%%%%%%%%%%%%%%%%%%%%%%%%%%%%%%%%%%%%%%%%%%%
%% subsubsection 01.1.3.1 %%%%%%%%%%%%%%%%%%%%%%%%%%%%%%%%
\subsubsection{端面における寸法}
\begin{enumerate}
\item \index{こうさ(ぜんちょう)@公差(全長)}全長の公差を確認し、\index{トップふりわけちょう@トップ振分長}トップ振分長および\index{ボトムふりわけちょう@ボトム振分長}ボトム振分長の\index{こうさ(ふりわけちょう)@公差(振分長)}公差の判断を行う
\item トップ側・ボトム側の\index{ふりわけちょう@振分長}振分長を確認し、\index{スペーサ}スペーサによる調整が必要か判断を行う
\item 使用するスペーサおよび\index{さいふりわけちょう@再振分長}再振分長は、専用の計算\index{プログラム(Excel VBA)}プログラム(\index{Excel VBA}Excel VBA)を用いて決定する
\item \index{がいけい@外径}外径・\index{にくあつ(たんめん)@肉厚(端面)}端面部の肉厚・\index{コーナーR(たんめん)@コーナーR(端面)}コーナーRの大きさを確認し、それに応じて\index{こうぐけいほせいち@工具径補正値}工具径補正値を決定する
\item 振分長に応じて、$Z$方向の\index{クリアランスへいめん(Zほうこう)@クリアランス平面($Z$方向)}クリアランス平面の位置を決定する
\end{enumerate}
%%%%%%%%%%%%%%%%%%%%%%%%%%%%%%%%%%%%%%%%%%%%%%%%%%%%%%%%%%
%% PARAMETER %%%%%%%%%%%%%%%%%%%%%%%%%%%%%%%%%%%%%%%%%%%%%
%%%%%%%%%%%%%%%%%%%%%%%%%%%%%%%%%%%%%%%%%%%%%%%%%%%%%%%%%%
\begin{Parameter}{必要なパラメータ}
\paragraph*{再振分長}
\PMbox{全長}\PMbox{トップ振分長}\PMbox{ボトム振分長}\PMbox{AC外径}\PMbox{ジグの長さ}\PMbox{受板の幅}
\tcbline*
\paragraph*{トップ端面}
\PMbox{トップ再振分長}\PMbox{AC外径}\PMbox{BD外径}\PMbox{外径コーナーR}\\
\PMbox{トップ端AC内径}\PMbox{トップ端BD内径}\\
\PMbox{トップ側$Z$方向クリアランス平面距離}
\tcbline*
\paragraph*{ボトム端面}
\PMbox{ボトム再振分長}\PMbox{AC外径}\PMbox{BD外径}\PMbox{外径コーナーR}\\
\PMbox{ボトム端AC内径}\PMbox{ボトム端BD内径}\\
\PMbox{ボトム側$Z$方向クリアランス平面距離}
\end{Parameter}
%%%%%%%%%%%%%%%%%%%%%%%%%%%%%%%%%%%%%%%%%%%%%%%%%%%%%%%%%%
%%%%%%%%%%%%%%%%%%%%%%%%%%%%%%%%%%%%%%%%%%%%%%%%%%%%%%%%%%
%%%%%%%%%%%%%%%%%%%%%%%%%%%%%%%%%%%%%%%%%%%%%%%%%%%%%%%%%%

%\clearpage
%%%%%%%%%%%%%%%%%%%%%%%%%%%%%%%%%%%%%%%%%%%%%%%%%%%%%%%%%%
%% subsubsection 01.1.3.2 %%%%%%%%%%%%%%%%%%%%%%%%%%%%%%%%
%%%%%%%%%%%%%%%%%%%%%%%%%%%%%%%%%%%%%%%%%%%%%%%%%%%%%%%%%%
\subsubsection{外削における寸法}
\begin{enumerate}
\item \index{がいさくちょう@外削長}外削長と\index{みぞ@溝}溝の位置を確認し、実際に加工する外削の長さの判断を行う
\item トップ側・ボトム側の両方に外削のある場合は、どちら側が\index{きじゅん(がいさくちゅうしん)@基準(外削中心)}基準であるのかを確認する
\item \index{ないけい(たんめん)@内径(端面)}端面の内径・\index{Aがわにくあつ(がいさく)@A側肉厚(外削)}外削部のA側肉厚・内面の\index{めっきまくあつ@めっき膜厚}めっき膜厚から、\index{がいさくちゅうしん@外削中心}外削中心$X$位置用のパラメータを手動による計算で決定する
\item \index{わんきょくにそったがいさく@湾曲に沿った外削}湾曲に沿った外削の場合は、\index{かたむきかく(がいさく)@傾き角(外削)}傾き角を手動による計算で決定する
\end{enumerate}
%%%%%%%%%%%%%%%%%%%%%%%%%%%%%%%%%%%%%%%%%%%%%%%%%%%%%%%%%%
%% PARAMETER %%%%%%%%%%%%%%%%%%%%%%%%%%%%%%%%%%%%%%%%%%%%%
%%%%%%%%%%%%%%%%%%%%%%%%%%%%%%%%%%%%%%%%%%%%%%%%%%%%%%%%%%
\begin{Parameter}{必要なパラメータ}
\paragraph*{ボトム側の外削のみの場合}
\PMbox{ボトム外削AC径}\PMbox{ボトム外削BD径}\PMbox{ボトム外削コーナーR}\\
\PMbox{ボトム外削長}\\
\PMbox{ボトム端AC内径}\PMbox{ボトム側A側肉厚}\PMbox{めっき膜厚}
\tcbline*
\paragraph*{トップ側の外削のみの場合}
\PMbox{トップ外削AC径}\PMbox{トップ外削BD径}\PMbox{トップ外削コーナーR}\\
\PMbox{トップ外削長}\PMbox{溝位置}\PMbox{溝幅}\\
\PMbox{トップ端AC内径}\PMbox{トップ外削A側肉厚}\PMbox{めっき膜厚}
\tcbline*
\paragraph*{両方に外削があり、ボトム側が基準の場合}
\PMbox{ボトム外削AC径}\PMbox{ボトム外削BD径}\PMbox{ボトム外削コーナーR}\\
\PMbox{ボトム外削長}\\
\PMbox{ボトム端AC内径}\PMbox{ボトム外削A側肉厚}\PMbox{めっき膜厚}\\
\PMbox{トップ外削AC径}\PMbox{トップ外削BD径}\PMbox{トップ外削コーナーR}\\
\PMbox{トップ外削長}\PMbox{溝位置}\PMbox{溝幅}\PMbox{通り芯}
\tcbline*
\paragraph*{両方に外削があり、トップ側が基準の場合}
\PMbox{トップ外削AC径}\PMbox{トップ外削BD径}\PMbox{トップ外削コーナーR}\\
\PMbox{トップ外削長}\PMbox{溝位置}\PMbox{溝幅}\\
\PMbox{トップ端AC内径}\PMbox{トップ外削A側肉厚}\PMbox{めっき膜厚}\\
\PMbox{ボトム外削AC径}\PMbox{ボトム外削BD径}\PMbox{ボトム外削コーナーR}\\
\PMbox{ボトム外削長}\PMbox{通り芯}
\tcbline*
\paragraph*{湾曲に沿った外削の場合}
(以上に加えて)\PMbox{中心湾曲}
\end{Parameter}
%%%%%%%%%%%%%%%%%%%%%%%%%%%%%%%%%%%%%%%%%%%%%%%%%%%%%%%%%%
%%%%%%%%%%%%%%%%%%%%%%%%%%%%%%%%%%%%%%%%%%%%%%%%%%%%%%%%%%
%%%%%%%%%%%%%%%%%%%%%%%%%%%%%%%%%%%%%%%%%%%%%%%%%%%%%%%%%%

%\clearpage
%%%%%%%%%%%%%%%%%%%%%%%%%%%%%%%%%%%%%%%%%%%%%%%%%%%%%%%%%%
%% subsubsection 01.1.3.3 %%%%%%%%%%%%%%%%%%%%%%%%%%%%%%%%
%%%%%%%%%%%%%%%%%%%%%%%%%%%%%%%%%%%%%%%%%%%%%%%%%%%%%%%%%%
\subsubsection{溝における寸法}
\begin{enumerate}
\item \index{みぞのけいじょう@溝の形状}溝の形状を確認し、必要に応じて加工における径の決定する
\item \index{きじゅん(みぞちゅうしん)@基準(溝中心)}溝中心の基準を確認し、\index{みぞちゅうしん@溝中心}溝中心の$X$位置を手動で計算し、決定する
\item \index{みぞはば@溝幅}溝幅を確認し、\index{かこうかいすう(みぞはば)@加工回数(溝幅)}加工の回数を決定する
\end{enumerate}
%%%%%%%%%%%%%%%%%%%%%%%%%%%%%%%%%%%%%%%%%%%%%%%%%%%%%%%%%%
%% PARAMETER %%%%%%%%%%%%%%%%%%%%%%%%%%%%%%%%%%%%%%%%%%%%%
%%%%%%%%%%%%%%%%%%%%%%%%%%%%%%%%%%%%%%%%%%%%%%%%%%%%%%%%%%
\begin{Parameter}{必要なパラメータ}
\paragraph*{湾曲中心が基準の場合}
\PMbox{溝AC径}\PMbox{溝BD径}\PMbox{溝位置}\PMbox{溝幅}\PMbox{中心湾曲}\\
\PMbox{溝コーナーR}または\PMbox{溝コーナーC}
\tcbline*
\paragraph*{外削中心が基準の場合}
\PMbox{溝AC径}\PMbox{溝BD径}\PMbox{溝位置}\PMbox{溝幅}\\
\PMbox{溝コーナーR}または\PMbox{溝コーナーC}
\tcbline*
\paragraph*{A側溝深さが基準の場合}
(以上に加えて)\PMbox{A側溝深さ}
\end{Parameter}
%%%%%%%%%%%%%%%%%%%%%%%%%%%%%%%%%%%%%%%%%%%%%%%%%%%%%%%%%%
%%%%%%%%%%%%%%%%%%%%%%%%%%%%%%%%%%%%%%%%%%%%%%%%%%%%%%%%%%
%%%%%%%%%%%%%%%%%%%%%%%%%%%%%%%%%%%%%%%%%%%%%%%%%%%%%%%%%%

\clearpage
%%%%%%%%%%%%%%%%%%%%%%%%%%%%%%%%%%%%%%%%%%%%%%%%%%%%%%%%%%
%% subsubsection 01.1.3.4 %%%%%%%%%%%%%%%%%%%%%%%%%%%%%%%%
%%%%%%%%%%%%%%%%%%%%%%%%%%%%%%%%%%%%%%%%%%%%%%%%%%%%%%%%%%
\subsubsection{端面の面取における寸法}
\begin{enumerate}
\item \index{そとがわCめんとり(たんめん)@外側C面取}端面の外側C面取の場合は、\index{がいさくのうむ@外削の有無}外削の有無を確認し、加工の径を決定する
\item \index{Cめんとり(がいさくせんたん)@C面取(外削先端)}外削先端部のC面取の場合は、\index{がいさくのけいじょう@外削の形状}外削の形状を確認し、\index{こうぐ(がいさく)@工具(外削)}工具を決定する
\end{enumerate}
%%%%%%%%%%%%%%%%%%%%%%%%%%%%%%%%%%%%%%%%%%%%%%%%%%%%%%%%%%
%% PARAMETER %%%%%%%%%%%%%%%%%%%%%%%%%%%%%%%%%%%%%%%%%%%%%
%%%%%%%%%%%%%%%%%%%%%%%%%%%%%%%%%%%%%%%%%%%%%%%%%%%%%%%%%%
\begin{Parameter}{必要なパラメータ}
\paragraph*{トップ端外側C面取:外削のない場合}
\PMbox{AC外径}\PMbox{BD外径}\PMbox{トップ端外側C面取長}\PMbox{外径コーナーR}
\tcbline*
\paragraph*{トップ端外側C面取:外削のある場合}
\PMbox{トップ外削AC径}\PMbox{トップ外削BD径}\PMbox{トップ端外削コーナーR}\\
\PMbox{トップ端外側C面取長}
\tcbline*
\paragraph*{ボトム端外側C面取:外削のない場合}
\PMbox{AC外径}\PMbox{BD外径}\PMbox{ボトム端外側C面取長}\PMbox{外径コーナーR}
\tcbline*
\paragraph*{ボトム端外側C面取:外削のある場合}
\PMbox{ボトム外削AC径}\PMbox{ボトム外削BD径}\PMbox{ボトム端外削コーナーR}\\
\PMbox{ボトム端外側C面取長}
\tcbline*
\paragraph*{トップ端内側C面取}
\PMbox{トップ端AC内径}\PMbox{トップ端BD内径}\PMbox{トップ端内径コーナーR}\\
\PMbox{トップ端内側C面取長}\PMbox{めっき膜厚}
\tcbline*
\paragraph*{ボトム端内側C面取}
\PMbox{ボトム端AC内径}\PMbox{ボトム端BD内径}\PMbox{ボトム端内径コーナーR}\\
\PMbox{ボトム端内側C面取長}\PMbox{めっき膜厚}
\end{Parameter}
%%%%%%%%%%%%%%%%%%%%%%%%%%%%%%%%%%%%%%%%%%%%%%%%%%%%%%%%%%
%%%%%%%%%%%%%%%%%%%%%%%%%%%%%%%%%%%%%%%%%%%%%%%%%%%%%%%%%%
%%%%%%%%%%%%%%%%%%%%%%%%%%%%%%%%%%%%%%%%%%%%%%%%%%%%%%%%%%

%\clearpage
%%%%%%%%%%%%%%%%%%%%%%%%%%%%%%%%%%%%%%%%%%%%%%%%%%%%%%%%%%
%% subsubsection 01.1.3.4 %%%%%%%%%%%%%%%%%%%%%%%%%%%%%%%%
%%%%%%%%%%%%%%%%%%%%%%%%%%%%%%%%%%%%%%%%%%%%%%%%%%%%%%%%%%
\subsubsection{座ぐりにおける寸法\TBW}
(to be written...)
%%%%%%%%%%%%%%%%%%%%%%%%%%%%%%%%%%%%%%%%%%%%%%%%%%%%%%%%%%
%% PARAMETER %%%%%%%%%%%%%%%%%%%%%%%%%%%%%%%%%%%%%%%%%%%%%
%%%%%%%%%%%%%%%%%%%%%%%%%%%%%%%%%%%%%%%%%%%%%%%%%%%%%%%%%%
\begin{Parameter}{必要なパラメータ}
\PMbox{座ぐりの位置}\PMbox{座ぐりの長さ}\PMbox{座ぐりコーナーR}\PMbox{座ぐり深さ}\PMbox{トップAC外径}
\end{Parameter}
%%%%%%%%%%%%%%%%%%%%%%%%%%%%%%%%%%%%%%%%%%%%%%%%%%%%%%%%%%
%%%%%%%%%%%%%%%%%%%%%%%%%%%%%%%%%%%%%%%%%%%%%%%%%%%%%%%%%%
%%%%%%%%%%%%%%%%%%%%%%%%%%%%%%%%%%%%%%%%%%%%%%%%%%%%%%%%%%


\clearpage
%%%%%%%%%%%%%%%%%%%%%%%%%%%%%%%%%%%%%%%%%%%%%%%%%%%%%%%%%%
%% subsection 01.1.4 %%%%%%%%%%%%%%%%%%%%%%%%%%%%%%%%%%%%%
%%%%%%%%%%%%%%%%%%%%%%%%%%%%%%%%%%%%%%%%%%%%%%%%%%%%%%%%%%
\subsection{プログラムの入力}
\begin{enumerate}
\item 原則として、\index{プログラムばんごう@プログラム番号}プログラム番号は\index{せいひんばんごう@製品番号}製品番号と一致させる\\
ただし、加工内容が同一のものである場合は、既存のプログラムをそのまま流用する
\item 各々の加工部分およびその形状に対する\index{サブプログラム}サブプログラムを決定する
\item 各々のサブプログラムに対し、適切な寸法値を手動で計算する
\item 各々のサブプログラムに対し、計算した寸法値・\index{こうぐばんごう@工具番号}工具番号・\index{おくりはやさ@送り速さ}送り速さ・\index{しゅじくかいてんすう@主軸回転数}主軸回転数を格納する
\item \index{さいふりわけちょう@再振分長}再振分長の寸法に応じて、\index{クリアランスへいめん(Zほうこう)@クリアランス平面($Z$方向)}$Z$方向クリアランス平面の位置を決定する
\item マシニングセンタの操作画面にて\index{メインプログラム}メインプログラムを直接編集し、必要なコードまたは数値を記入する
\item 必要に応じて、\index{いちじていし(プログラム)@一時停止(プログラム)}一時停止用のコードを代入する
\item \index{こうぐけいほせい@工具径補正}工具径または\index{こうぐちょうほせい@工具長補正}工具長の補正が必要な場合は、別途専用画面にて手動で編集を行う
\end{enumerate}


%\clearpage
%%%%%%%%%%%%%%%%%%%%%%%%%%%%%%%%%%%%%%%%%%%%%%%%%%%%%%%%%%
%% subsection 01.1.4 %%%%%%%%%%%%%%%%%%%%%%%%%%%%%%%%%%%%%
%%%%%%%%%%%%%%%%%%%%%%%%%%%%%%%%%%%%%%%%%%%%%%%%%%%%%%%%%%
\subsection{ワークの設置}
\begin{enumerate}
\item \index{スペーサ}スペーサが必要な場合は、適切なスペーサを\index{ジグ}ジグの\index{うけいた@受板}受板に設置する
\item \index{ワーク}ワークの大きさを考慮して、\index{ワークこていようボルト@ワーク固定用ボルト}ワーク固定用ボルトの長さを目分量で適宜決定し、ジグに設置する
\item \index{さいふりわけちょう@再振分長}再振分長に応じた位置に\index{ワーク}ワークを設置し、固定する
\item トップ側およびボトム側の、ジグからの\index{はりだしちょう@張出長}張出長を\index{メジャー}メジャーを用いて測定する
\item 測定した張出長から、\index{たんめんかこう@端面加工}端面加工における\index{ぜんけずりしろ(たんめん)@全削り代(端面)}全削り代を手動でおおまかに計算する
\end{enumerate}
%%%%%%%%%%%%%%%%%%%%%%%%%%%%%%%%%%%%%%%%%%%%%%%%%%%%%%%%%%
%% PARAMETER %%%%%%%%%%%%%%%%%%%%%%%%%%%%%%%%%%%%%%%%%%%%%
%%%%%%%%%%%%%%%%%%%%%%%%%%%%%%%%%%%%%%%%%%%%%%%%%%%%%%%%%%
\begin{Parameter}{必要なパラメータ}
\paragraph*{ワーク固定用ボルト}
\PMbox{AC外径}\PMbox{BD外径}\\
\PMbox{ジグ床面とボルト取付具(上)間の距離}\PMbox{受板とボルト取付具(横)間の距離}
\tcbline*
\paragraph*{端面の削り代}
\PMbox{ジグの長さ}\PMbox{トップ再振分長}\PMbox{ボトム再振分長}\PMbox{端面加工1回あたりの削り代}\\
\PMbox{トップ側張出長実測値}\PMbox{ボトム側張出長実測値}
\end{Parameter}
%%%%%%%%%%%%%%%%%%%%%%%%%%%%%%%%%%%%%%%%%%%%%%%%%%%%%%%%%%
%%%%%%%%%%%%%%%%%%%%%%%%%%%%%%%%%%%%%%%%%%%%%%%%%%%%%%%%%%
%%%%%%%%%%%%%%%%%%%%%%%%%%%%%%%%%%%%%%%%%%%%%%%%%%%%%%%%%%


\clearpage
%%%%%%%%%%%%%%%%%%%%%%%%%%%%%%%%%%%%%%%%%%%%%%%%%%%%%%%%%%
%% subsection 01.1.5 %%%%%%%%%%%%%%%%%%%%%%%%%%%%%%%%%%%%%
%%%%%%%%%%%%%%%%%%%%%%%%%%%%%%%%%%%%%%%%%%%%%%%%%%%%%%%%%%
\subsection{ワーク設置後の調整}
\begin{enumerate}
\item トップ側およびボトム側の\index{ぜんけずりしろ(たんめん)@全削り代(端面)}全削り代に応じて、\index{かこうかいすう(たんめんかこう)@加工回数(端面加工)}端面加工の回数を設定する
\item トップ端およびボトム端の\index{がいけいちゅうしん@外径中心}外径中心の位置を\index{メジャー}メジャーで測定する
\item 測定した中心位置を用いて、\index{ワークざひょうけいげんてん@ワーク座標系原点}ワーク座標系原点の設定を行う
\item \expandafterindex{テーブルのかいてんちゅうしん(\MMname)@テーブルの回転中心(\MMname)}テーブルの回転中心とのずれを考慮して、端面の$Z$方向の長さを調整する
\item \index{とおりしん@通り芯}通り芯がある場合\expandafterindex{テーブルのかいてんちゅうしん(\MMname)@テーブルの回転中心(\MMname)}テーブルの回転中心とのずれを考慮して、\index{がいさくけいのちゅうしん@外削径の中心}の$X$方向の長さを調整する
\end{enumerate}
これらの設定は、マシニングセンタの操作画面から\index{メインプログラム}メインプログラムを直接手動で編集する形で行われる。



\clearpage
%%%%%%%%%%%%%%%%%%%%%%%%%%%%%%%%%%%%%%%%%%%%%%%%%%%%%%%%%%
%% section 1.2 %%%%%%%%%%%%%%%%%%%%%%%%%%%%%%%%%%%%%%%%%%%
%%%%%%%%%%%%%%%%%%%%%%%%%%%%%%%%%%%%%%%%%%%%%%%%%%%%%%%%%%
\modHeadsection{\MMname における工程(加工中)}


%%%%%%%%%%%%%%%%%%%%%%%%%%%%%%%%%%%%%%%%%%%%%%%%%%%%%%%%%%
%% subsection 01.2.1 %%%%%%%%%%%%%%%%%%%%%%%%%%%%%%%%%%%%%
%%%%%%%%%%%%%%%%%%%%%%%%%%%%%%%%%%%%%%%%%%%%%%%%%%%%%%%%%%
\subsection{芯出し測定後}
\begin{enumerate}
\item 各々の\index{ワークざひょうけいげんてん@ワーク座標系原点}ワーク座標系原点の測定後、必要に応じてワーク座標系原点の設定変更を行う
\item 各々の測定箇所の$Z$位置の変更を、必要に応じて行う
\end{enumerate}
これらの設定は、\index{マシニングセンタ}マシニングセンタの操作画面から\index{メインプログラム}メインプログラムを直接手動で編集する形で行われる。


%%%%%%%%%%%%%%%%%%%%%%%%%%%%%%%%%%%%%%%%%%%%%%%%%%%%%%%%%%
%% subsection 01.2.1 %%%%%%%%%%%%%%%%%%%%%%%%%%%%%%%%%%%%%
%%%%%%%%%%%%%%%%%%%%%%%%%%%%%%%%%%%%%%%%%%%%%%%%%%%%%%%%%%
\subsection{端面加工中}
\begin{enumerate}
\item 必要に応じて、\index{1かいあたりのけずりしろ(たんめん)@1回あたりの削り代(端面)}1回あたりの削り代を調整する
\end{enumerate}


%%%%%%%%%%%%%%%%%%%%%%%%%%%%%%%%%%%%%%%%%%%%%%%%%%%%%%%%%%
%% subsection 01.2.1 %%%%%%%%%%%%%%%%%%%%%%%%%%%%%%%%%%%%%
%%%%%%%%%%%%%%%%%%%%%%%%%%%%%%%%%%%%%%%%%%%%%%%%%%%%%%%%%%
\subsection{外削加工中}
\begin{enumerate}
\item 必要に応じて\index{しあげかこう(がいさく)@仕上げ加工(外削)}仕上加工前に\index{いちじていし(プログラム)@一時停止(プログラム)}一時停止を行い、\index{Aがわにくあつ@A側肉厚}A側肉厚および\index{がいさくけい@外削径}外削径の測定を行う
\item A側肉厚を調整する場合は、該当する\index{しんだしそくてい(がいさくちゅうしん@芯出し測定(外削中心)}芯出し測定部分のパラメータをメインプログラムから直接手動で編集する
\item \index{がいさくけい@外削径}外削径を調整する場合は、該当する加工部分のパラメータをマシニングセンタの操作画面から\index{メインプログラム}メインプログラムを直接手動で編集する
\item \index{かこうかいすう(がいさく)@加工回数(外削)}加工の回数を変更する場合は、該当する加工部分をマシニングセンタの操作画面からメインプログラムを直接手動で編集する
\end{enumerate}


%%%%%%%%%%%%%%%%%%%%%%%%%%%%%%%%%%%%%%%%%%%%%%%%%%%%%%%%%%
%% subsection 01.2.1 %%%%%%%%%%%%%%%%%%%%%%%%%%%%%%%%%%%%%
%%%%%%%%%%%%%%%%%%%%%%%%%%%%%%%%%%%%%%%%%%%%%%%%%%%%%%%%%%
\subsection{溝加工中}
\begin{enumerate}
\item 必要に応じて\index{しあげかこう(みぞ)@仕上げ加工(溝)}仕上加工前に\index{いちじていし(プログラム)@一時停止(プログラム)}一時停止を行い、\index{Aがわみぞふかさ@A側溝深さ}A側溝深さおよび\index{みぞけい@溝径}溝径の測定を行う
\item A側溝深さを調整する場合は、該当する\index{しんだしそくてい(みぞちゅうしん)@芯出し測定(溝)}芯出し測定部分のパラメータをマシニングセンタの操作画面からメインプログラムを直接手動で編集する
\item 溝径を調整する場合は、該当する加工部分のパラメータをマシニングセンタの操作画面からメインプログラムを直接手動で編集する
\item \index{かこうかいすう(みぞ)@加工回数(溝)}加工の回数を変更する場合は、該当する加工部分をマシニングセンタの操作画面からメインプログラムを直接手動で編集する
\item 必要に応じて、\index{ブロックゲージ}ブロックゲージによる\index{みぞはば@溝幅}溝幅の測定を行う
\end{enumerate}


%%%%%%%%%%%%%%%%%%%%%%%%%%%%%%%%%%%%%%%%%%%%%%%%%%%%%%%%%%
%% subsection 01.2.1 %%%%%%%%%%%%%%%%%%%%%%%%%%%%%%%%%%%%%
%%%%%%%%%%%%%%%%%%%%%%%%%%%%%%%%%%%%%%%%%%%%%%%%%%%%%%%%%%
\subsection{端面の外側C面取加工中}
\begin{enumerate}
\item 必要に応じて\index{しあげかこう(そとがわCめんとり)@仕上げ加工(外側C面取)}仕上加工前に\index{いちじていし(プログラム)@一時停止(プログラム)}一時停止を行い、\index{Cめんとり@C面取}C面取の測定・位置の確認を行う
\item C面取の位置を調整する場合は、該当する加工部分のパラメータをマシニングセンタの操作画面からメインプログラムを直接手動で編集する
\item \index{かこうかいすう(たんめんそとがわCめんとり)@加工回数(端面外側C面取)}加工の回数を変更する場合は、該当する加工部分をマシニングセンタの操作画面からメインプログラムを直接手動で編集する
\end{enumerate}


\clearpage
%%%%%%%%%%%%%%%%%%%%%%%%%%%%%%%%%%%%%%%%%%%%%%%%%%%%%%%%%%
%% subsection 01.2.1 %%%%%%%%%%%%%%%%%%%%%%%%%%%%%%%%%%%%%
%%%%%%%%%%%%%%%%%%%%%%%%%%%%%%%%%%%%%%%%%%%%%%%%%%%%%%%%%%
\subsection{端面の内側C面取加工中}
\begin{enumerate}
\item 必要に応じて\index{しあげかこう(うちがわCめんとり)@仕上げ加工(内側C面取)}仕上加工前に\index{いちじていし@一時停止}一時停止を行い、C面取の測定・位置の確認を行う
\item C面取の位置を調整する場合は、該当する加工部分のパラメータをマシニングセンタの操作画面からメインプログラムを直接手動で編集する
\item \index{かこうかいすう(たんめんうちがわCめんとり)@加工回数(端面内側C面取)}加工の回数を変更する場合は、該当する加工部分をマシニングセンタの操作画面からメインプログラムを直接手動で編集する
\end{enumerate}


%\clearpage
%%%%%%%%%%%%%%%%%%%%%%%%%%%%%%%%%%%%%%%%%%%%%%%%%%%%%%%%%%
%% subsection 01.2.1 %%%%%%%%%%%%%%%%%%%%%%%%%%%%%%%%%%%%%
%%%%%%%%%%%%%%%%%%%%%%%%%%%%%%%%%%%%%%%%%%%%%%%%%%%%%%%%%%
\subsection{座ぐり加工中\TBW}
(to be written...)



\clearpage
%%%%%%%%%%%%%%%%%%%%%%%%%%%%%%%%%%%%%%%%%%%%%%%%%%%%%%%%%%
%% section 01.3 %%%%%%%%%%%%%%%%%%%%%%%%%%%%%%%%%%%%%%%%%%
%%%%%%%%%%%%%%%%%%%%%%%%%%%%%%%%%%%%%%%%%%%%%%%%%%%%%%%%%%
\modHeadsection{\MMname における工程(加工後)}


%%%%%%%%%%%%%%%%%%%%%%%%%%%%%%%%%%%%%%%%%%%%%%%%%%%%%%%%%%
%% subsection 01.3.1 %%%%%%%%%%%%%%%%%%%%%%%%%%%%%%%%%%%%%
%%%%%%%%%%%%%%%%%%%%%%%%%%%%%%%%%%%%%%%%%%%%%%%%%%%%%%%%%%
\subsection{ワークの取外し}
\begin{enumerate}
\item 必要に応じて、\index{ワークこていようボルト@ワーク固定用ボルト}ワーク固定用ボルトを緩める前に、各種\index{そくていき@測定器}測定器で\index{すんぽう@寸法}寸法を確認する
\item クーラント用の液およびエアーブローを用いて軽く洗浄を行い、固定用ボルトを緩めて\index{ワーク}ワークを取り出し、軽く拭取りを行う
\item \index{リフター}リフターまたは\index{クレーン}クレーンを用いて、\index{めんとりようさぎょうだい@面取用作業台}面取用作業台にワークを移動する
\end{enumerate}


%%%%%%%%%%%%%%%%%%%%%%%%%%%%%%%%%%%%%%%%%%%%%%%%%%%%%%%%%%
%% subsection 01.3.2 %%%%%%%%%%%%%%%%%%%%%%%%%%%%%%%%%%%%%
%%%%%%%%%%%%%%%%%%%%%%%%%%%%%%%%%%%%%%%%%%%%%%%%%%%%%%%%%%
\subsection{外観の確認・寸法の検査}
\begin{enumerate}
\item \index{がいかん(ワーク)@外観(ワーク)}外観に異常がないか確認を行う
\item \index{そくていき@測定器}測定器を用いて\index{すんぽう@寸法}寸法の確認を行う
\item 所定の用紙に、指定箇所の\index{こうさ@公差}公差を考慮した寸法値を、手動で計算を行い手動で記入する
\item 必要に応じて、所定の用紙に測定値の記入を行う
\end{enumerate}


%%%%%%%%%%%%%%%%%%%%%%%%%%%%%%%%%%%%%%%%%%%%%%%%%%%%%%%%%%
%% subsection 01.3.3 %%%%%%%%%%%%%%%%%%%%%%%%%%%%%%%%%%%%%
%%%%%%%%%%%%%%%%%%%%%%%%%%%%%%%%%%%%%%%%%%%%%%%%%%%%%%%%%%
\subsection{手動による仕上げ加工}
\begin{enumerate}
\item 所定の寸法の\index{めんとり(たんめん)@面取(端面)}端面の面取を、\index{てもちけんまき@手持ち研磨機}手持ち研磨機を用いて手動で行う
\item \index{ばり}ばり等を除去を、\index{やすり}やすりを用いて全体的に手動で行う
\item \index{にくあつ(たんめん)@肉厚(端面)}端面の肉厚に応じて\index{こくいん@刻印}刻印の大きさを決定する
\item 明細のによる指定に応じて、刻印の位置を調整する
\item リフターまたはクレーンを用いて、所定の置き場に移動する
\end{enumerate}



%%%%%%%%%%%%%%%%%%%%%%%%%%%%%%%%%%%%%%%%%%%%%%%%%%%%%%%%%%
%% chapter 2 %%%%%%%%%%%%%%%%%%%%%%%%%%%%%%%%%%%%%%%%%%%%%
%%%%%%%%%%%%%%%%%%%%%%%%%%%%%%%%%%%%%%%%%%%%%%%%%%%%%%%%%%
\modHeadchapter{現状の問題点}
%!TEX root = ../RfCPN.tex


\modHeadchapter{イシュー・問題の特定}
先に述べた\expandafterindex{ぎょうむフロー(\yomiMMC)@業務フロー(\nameMMC)}業務フローを通して、\MMC に関する\index{イシュー}イシュー(issue)および\index{もんだい(problem)@問題(problem)}問題(problem)の特定を試みる。
なお{\color{red}赤色}(\,\sarrow[red]\!)の項目は、ソフトウェア側(\index{NCプログラム}NCプログラムの内容)で対処・改善できると考えられるものを示す
%% footnote %%%%%%%%%%%%%%%%%%%%%
\footnote{つまり、システムの管理・監督者の能力および責任によるところが大きなものである。}。
%%%%%%%%%%%%%%%%%%%%%%%%%%%%%%%%%



%%%%%%%%%%%%%%%%%%%%%%%%%%%%%%%%%%%%%%%%%%%%%%%%%%%%%%%%%%
%% section 02.01 %%%%%%%%%%%%%%%%%%%%%%%%%%%%%%%%%%%%%%%%%
%%%%%%%%%%%%%%%%%%%%%%%%%%%%%%%%%%%%%%%%%%%%%%%%%%%%%%%%%%
\modHeadsection{安全(safety)に関するイシュー・問題}

\begin{Issues}{オペレータの精神的高ストレス下での作業}
通常、\index{NCプログラム}NCプログラムの編集はシステム管理者・設計者・\index{プログラマ}プログラマ等が行う高度な業務である
\begin{enumerate}[label=\sarrow]
\item[{\sarrow[red]}]作業者に対してNCメインプログラムの直接編集を強いている状態になっている
\item[{\sarrow[red]}]プログラマが存在しない
\end{enumerate}
\end{Issues}

\begin{Issues}{\index{オペレータ}オペレータの機内の侵入に伴うリスク}
一般に、\index{きないへのしんにゅう@機内への侵入}機内への侵入は、転倒・巻き込まれ等のリスクを伴う
\begin{enumerate}[label=\sarrow]
\item[{\sarrow[red]}]機内に侵入し直接測定をしないと、ワークの\index{かこうげんてん(がいさんち)@加工原点(概算値)}加工原点の概算値が見出だせない状態にある
\item[{\sarrow[red]}]機内に侵入しないと、\index{はのこうかん(フェイスミル)@刃の交換(フェイスミル)}フェイスミルの刃の交換ができない状態にある
\item 機内に侵入しないと、内部の掃除ができない状態にある
\end{enumerate}
\end{Issues}

\begin{Issues}{\index{てもちけんまき@手持ち研磨機}手持ち研磨機の使用に伴うリスク}
一般に、\index{てもちけんまき@手持ち研磨機}手持ち研磨機による加工は、巻き込まれや粉塵の付着・吸引等のリスクを伴う
\begin{enumerate}[label=\sarrow]
\item[{\sarrow[red]}]寸法の小さな\EndFaceChamferMilling が\MMC で行われず、\index{てもちけんまき@手持ち研磨機}手持ち研磨機により手作業で行われている状態にある
\item[{\sarrow[red]}]\EndFaceChamferMilling に関する解析的な幾何情報を導出しないまま放置され続けている
\end{enumerate}
\end{Issues}

\begin{Issues}{柵への衝突に伴うリスク}
\begin{enumerate}[label=\sarrow]
\item 後付けされた\expandafterindex{あんぜんさく(\yomiMMC)@安全柵(\nameMMC)}安全柵(と称されている柵)が、衝突の\index{リスク(しょうとつ)@リスク(衝突)}リスクを生み出している
\item 一部のみ(出入口のみ)を着目し、周囲(柵の周辺)が軽視され、\index{あんぜんたいさく@安全対策}安全対策が機能せず反対に危険リスクを生み出している
\end{enumerate}
\end{Issues}

\clearpage
\begin{Issues}{\index{リフター}リフターへの衝突に伴うリスク}
\begin{enumerate}[label=\sarrow]
\item ワークの\index{うけいれけんさ(ワーク)@受入検査(ワーク)}受入検査の際に\index{リフター}リフターや\index{クレーン}クレーンが必ず\index{あんぜんつうろ@安全通路}安全通路を通る構造にある
\item 安全性より生産性を優先する\index{けいえいほうしん(とうしゃ)@経営方針(当社)}経営方針となっている
\end{enumerate}
\end{Issues}


\clearpage
%%%%%%%%%%%%%%%%%%%%%%%%%%%%%%%%%%%%%%%%%%%%%%%%%%%%%%%%%%
%% section 02.02 %%%%%%%%%%%%%%%%%%%%%%%%%%%%%%%%%%%%%%%%%
%%%%%%%%%%%%%%%%%%%%%%%%%%%%%%%%%%%%%%%%%%%%%%%%%%%%%%%%%%
\modHeadsection{\index{ひんしつ@品質}品質に関するイシュー・問題}


%%%%%%%%%%%%%%%%%%%%%%%%%%%%%%%%%%%%%%%%%%%%%%%%%%%%%%%%%%
%% subsection 02.02.01 %%%%%%%%%%%%%%%%%%%%%%%%%%%%%%%%%%%
%%%%%%%%%%%%%%%%%%%%%%%%%%%%%%%%%%%%%%%%%%%%%%%%%%%%%%%%%%
\subsection{測定における品質}

\begin{Issues}{\KeywayCenterMeasurement(AC方向)の不安定性}
\begin{enumerate}[label=\sarrow]
\item[{\sarrow[red]}]\KeywayCenter(AC方向)の測定が、手計算による関節的な方法で導出されている
\item[{\sarrow[red]}]自動化が十分可能にもかかわらず、事態が放置され続けている
\item[{\sarrow[red]}]\CenterCurvature に伴う誤差を含み、位置(特に\AsideKeywayDepth)が安定しない
\end{enumerate}
\end{Issues}

\begin{Issues}{\KeywayCenterMeasurement(BD方向)の不安定性}
\begin{enumerate}[label=\sarrow]
\item[{\sarrow[red]}]\KeywayCenter(BD方向)が、端面におけるそれとして与えられている
\item[{\sarrow[red]}]BD方向の真直度に伴う誤差を含み、位置が安定しない
\end{enumerate}
\end{Issues}

\begin{Issues}{\CenterlineEndFaceDifMeasurement の不安定性}
\begin{enumerate}[label=\sarrow]
\item[{\sarrow[red]}]\CenterlineEndFaceDifMeasurement が、手動による\index{ハンドルそうさ@ハンドル操作}ハンドル操作で行われている
\item[{\sarrow[red]}]手作業によるため、測定位置や送り速さが安定しない
\item[{\sarrow[red]}]自動化が十分可能にもかかわらず、事態が放置され続けている
\end{enumerate}
\end{Issues}

\begin{Issues}{\TLMeasurement の故障}
\begin{enumerate}[label=\sarrow]
\item\TLMeasurement 用装置が、物理的に壊れている
\item 現物合わせで\TLMeasurement を行っている
\end{enumerate}
\end{Issues}

\begin{Issues}{\TopOutcutCenter と\BottomOutcutCenter との差の未検出}
\begin{enumerate}[label=\sarrow]
\item[{\sarrow[red]}]測定した\TopOutcutCenter と\BottomOutcutCenter に大きな差があっても検出できない
\item[{\sarrow[red]}]\index{へんにく@偏肉}偏肉等による\OutcutCenter の偏りが検出ができない
\end{enumerate}
\end{Issues}


%%%%%%%%%%%%%%%%%%%%%%%%%%%%%%%%%%%%%%%%%%%%%%%%%%%%%%%%%%
%% subsection 02.02.02 %%%%%%%%%%%%%%%%%%%%%%%%%%%%%%%%%%%
%%%%%%%%%%%%%%%%%%%%%%%%%%%%%%%%%%%%%%%%%%%%%%%%%%%%%%%%%%
\subsection{\OutcutMilling における品質}

\begin{Issues}{\CurvedOutcutMilling の不安定性}
\begin{enumerate}[label=\sarrow]
\item[{\sarrow[red]}]手作業による\index{ハンドルそうさ@ハンドル操作}ハンドル操作により測定が行われている
\item[{\sarrow[red]}]\index{NCプログラム}NCプログラム内の\OutcutLength の寸法に誤りが存在している
\item[{\sarrow[red]}]自動化が十分可能にもかかわらず、事態が放置され続けている
\end{enumerate}
\end{Issues}


\clearpage
%%%%%%%%%%%%%%%%%%%%%%%%%%%%%%%%%%%%%%%%%%%%%%%%%%%%%%%%%%
%% subsection 02.02.03 %%%%%%%%%%%%%%%%%%%%%%%%%%%%%%%%%%%
%%%%%%%%%%%%%%%%%%%%%%%%%%%%%%%%%%%%%%%%%%%%%%%%%%%%%%%%%%
\subsection{\KeywayMilling における品質}

\begin{Issues}{\KeywayMilling のかえりの\index{てさぎょう@手作業}手作業による除去}
\begin{enumerate}[label=\sarrow]
\item[{\sarrow[red]}]長方形・8角形(C面取)の\KeywayMilling の一部の頂点にかえりが残る
\item[{\sarrow[red]}]手作業でかえりを除去するため、頂点部が歪な形状になる
\end{enumerate}
\end{Issues}


%\clearpage
%%%%%%%%%%%%%%%%%%%%%%%%%%%%%%%%%%%%%%%%%%%%%%%%%%%%%%%%%%
%% subsection 02.02.04 %%%%%%%%%%%%%%%%%%%%%%%%%%%%%%%%%%%
%%%%%%%%%%%%%%%%%%%%%%%%%%%%%%%%%%%%%%%%%%%%%%%%%%%%%%%%%%
\subsection{\EndFaceChamferMilling における品質}

\begin{Issues}{\index{てさぎょう@手作業}手作業による\EndFaceChamferMilling}
\begin{enumerate}[label=\sarrow]
\item[{\sarrow[red]}]
寸法の小さな\EndFaceChamferMilling が\index{てもちけんまき@手持ち研磨機}手持ち研磨機により\index{てさぎょう@手作業}手作業で行われている
\item[{\sarrow[red]}]大半の場合は自動化が十分可能にもかかわらず、事態が放置され続けている
\end{enumerate}
\end{Issues}

%%%%%%%%%%%%%%%%%%%%%%%%%%%%%%%%%%%%%%%%%%%%%%%%%%%%%%%%%%
%% subsection 02.02.04.1 %%%%%%%%%%%%%%%%%%%%%%%%%%%%%%%%%
%%%%%%%%%%%%%%%%%%%%%%%%%%%%%%%%%%%%%%%%%%%%%%%%%%%%%%%%%%
\subsubsection{\EndFaceOutCChamferMilling における品質}

\begin{Issues}{\EndFaceOutCChamferMilling の位置調整}
\begin{enumerate}[label=\sarrow]
\item[{\sarrow[red]}]\EndFaceOutCChamferMilling のAC方向の位置調整が手作業により行われている
\item[{\sarrow[red]}]目分量により位置調整がなされているため、位置が安定しない
\item[{\sarrow[red]}]自動化が十分可能にもかかわらず、事態が放置され続けている
\end{enumerate}
\end{Issues}

%%%%%%%%%%%%%%%%%%%%%%%%%%%%%%%%%%%%%%%%%%%%%%%%%%%%%%%%%%
%% subsection 02.02.04.2 %%%%%%%%%%%%%%%%%%%%%%%%%%%%%%%%%
%%%%%%%%%%%%%%%%%%%%%%%%%%%%%%%%%%%%%%%%%%%%%%%%%%%%%%%%%%
\subsubsection{\EndFaceInCChamferMilling における品質}

\begin{Issues}{\EndFaceInCChamferMilling の位置調整}
\begin{enumerate}[label=\sarrow]
\item[{\sarrow[red]}]\EndFaceInCChamferMilling のAC方向の位置調整が手作業により行われている
\item[{\sarrow[red]}]目分量により位置調整がなされているため、位置が安定しない
\item[{\sarrow[red]}]自動化が十分可能にもかかわらず、事態が放置され続けている
\end{enumerate}
\end{Issues}

\begin{Issues}{\EndFaceInCChamferMilling の始点・終点の位置}
\begin{enumerate}[label=\sarrow]
\item[{\sarrow[red]}]\EndFaceInCChamferMilling の始点および終点が直線部分にあり、加工の跡が残りやすい
\item[{\sarrow[red]}]始点および終点が同じ点になっているため、工具が少し摩耗しただけでも跡が残る
\end{enumerate}
\end{Issues}

\begin{Issues}{\EndFaceInCChamferMilling のコーナーRの補正}
\begin{enumerate}[label=\sarrow]
\item\index{しんがね@芯金}芯金の摩耗等により、\EndFaceInCChamfer のコーナーが変化しうる
\item[{\sarrow[red]}]\EndFaceInCChamferMilling のコーナーの値が一定なため、変化に対応ができない
\end{enumerate}
\end{Issues}


\clearpage
%%%%%%%%%%%%%%%%%%%%%%%%%%%%%%%%%%%%%%%%%%%%%%%%%%%%%%%%%%
%% subsection 02.02.02 %%%%%%%%%%%%%%%%%%%%%%%%%%%%%%%%%%%
%%%%%%%%%%%%%%%%%%%%%%%%%%%%%%%%%%%%%%%%%%%%%%%%%%%%%%%%%%
\subsection{\EndFaceBoringMilling における品質}

\begin{Issues}{\EndFaceBoringMilling の寸法の誤り}
\begin{enumerate}[label=\sarrow]
\item[{\sarrow[red]}]\EndFaceBoringMilling のAC方向の寸法が、図面のものと異なっている
\item 後工程で用いる端面座ぐり部の穴あけ用ジグを、誤った寸法のものに手作業で修正する必要がある
\end{enumerate}
\end{Issues}


%\clearpage
%%%%%%%%%%%%%%%%%%%%%%%%%%%%%%%%%%%%%%%%%%%%%%%%%%%%%%%%%%
%% subsection 02.02.02 %%%%%%%%%%%%%%%%%%%%%%%%%%%%%%%%%%%
%%%%%%%%%%%%%%%%%%%%%%%%%%%%%%%%%%%%%%%%%%%%%%%%%%%%%%%%%%
\subsection{加工全般における品質}

\begin{Issues}{仕上げ前加工・仕上げ加工の分離の非統一}
一般にどの加工においても、削り代がより少ないほど仕上がりがきれいなものになる
\begin{enumerate}[label=\sarrow]
\item[{\sarrow[red]}]仕上げ前の加工と仕上げの加工とで、削り代が統一されていない
\end{enumerate}
\end{Issues}

\begin{Issues}{手書きによる\index{きかいかこうすんぽううけいれチェックひょう@機械加工寸法受入チェック表}機械加工寸法受入チェック表}
\begin{enumerate}[label=\sarrow]
\item[{\sarrow[red]}]\DrawingNumber・向先・公称寸法・寸法公差許容範囲を、作業員が手書きで記述しなければならない状態にある
\item[{\sarrow[red]}]必要な項目が記載されていないことが頻繁にある
\item[{\sarrow[red]}]不必要な項目が記載されていることが頻繁にある
\item[{\sarrow[red]}]全明細の分を1つの形式で無理にまとめようとしており、かつできていない
\item[{\sarrow[red]}]自動化が十分可能にもかかわらず、事態が放置され続けている
\item[{\sarrow[red]}]しわ寄せがすべて作業員に押し付けられており、記載ミスを誘発・頻発させている
\end{enumerate}
\end{Issues}


%\clearpage
%%%%%%%%%%%%%%%%%%%%%%%%%%%%%%%%%%%%%%%%%%%%%%%%%%%%%%%%%%
%% subsection 02.02.07 %%%%%%%%%%%%%%%%%%%%%%%%%%%%%%%%%%%
%%%%%%%%%%%%%%%%%%%%%%%%%%%%%%%%%%%%%%%%%%%%%%%%%%%%%%%%%%
\subsection{工具における品質}

\begin{Issues}{工具の不整理・不整頓}
\begin{enumerate}[label=\sarrow]
\item 工具が整理されておらず、使用可能かも不明な工具が設置されている
\item 工具が整頓されておらず、無分別に工具が設置されている
\end{enumerate}
\end{Issues}



\clearpage
%%%%%%%%%%%%%%%%%%%%%%%%%%%%%%%%%%%%%%%%%%%%%%%%%%%%%%%%%%
%% section 02.03 %%%%%%%%%%%%%%%%%%%%%%%%%%%%%%%%%%%%%%%%%
%%%%%%%%%%%%%%%%%%%%%%%%%%%%%%%%%%%%%%%%%%%%%%%%%%%%%%%%%%
\modHeadsection{\index{さぎょうこうりつ@作業効率}作業効率に関するイシュー・問題}


%%%%%%%%%%%%%%%%%%%%%%%%%%%%%%%%%%%%%%%%%%%%%%%%%%%%%%%%%%
%% subsection 02.03.01 %%%%%%%%%%%%%%%%%%%%%%%%%%%%%%%%%%%
%%%%%%%%%%%%%%%%%%%%%%%%%%%%%%%%%%%%%%%%%%%%%%%%%%%%%%%%%%
\subsection{\index{NCプログラム}NCプログラムの作成における作業効率}

\begin{Issues}{\KeywayCenter 位置の手動計算による導出}
\begin{enumerate}[label=\sarrow]
\item[{\sarrow[red]}]\KeywayCenter の$X$座標を、電卓により手作業で計算して導出している
\item[{\sarrow[red]}]自動化が十分可能にもかかわらず、事態が放置され続けている
\end{enumerate}
\end{Issues}

\begin{Issues}{\AsideKeywayDepth 指定時の\KeywayCenter 位置の関節的導出}
\begin{enumerate}[label=\sarrow]
\item[{\sarrow[red]}]\AsideKeywayDepth 指定時の\KeywayCenter 位置を、直接的な測定ではなく、端面部の外側中心を基準に関節的に導出している
\item[{\sarrow[red]}]\CenterCurvature に伴う誤差を含み、\AsideKeywayDepth が安定しない
\item[{\sarrow[red]}]手動による位置調整の頻度が必然的に多くなっている
\end{enumerate}
\end{Issues}

\begin{Issues}{\index{NCメインプログラム}NCメインプログラムの手動作成}
\begin{enumerate}[label=\sarrow]
\item[{\sarrow[red]}]\index{NCメインプログラム}NCメインプログラムが人手により手動で作成されている
\item[{\sarrow[red]}]\index{NCメインプログラム}NCメインプログラムのパターン化がなされていない
\item[{\sarrow[red]}]\index{プログラマ}プログラマとしての能力が必要な\index{NCメインプログラム}NCメインプログラムの作成業務が、(スタッフや管理職でなく)作業者に課されている
\item[{\sarrow[red]}]作業者に\index{プログラマ}プログラマとしての能力が問われ、\index{きょういくコスト@教育コスト}教育コストを多くかかる状態になっている
\item[{\sarrow[red]}]自動化が十分可能にもかかわらず、事態が放置され続けている
\end{enumerate}
\end{Issues}


%%%%%%%%%%%%%%%%%%%%%%%%%%%%%%%%%%%%%%%%%%%%%%%%%%%%%%%%%%
%% subsection 02.03.02 %%%%%%%%%%%%%%%%%%%%%%%%%%%%%%%%%%%
%%%%%%%%%%%%%%%%%%%%%%%%%%%%%%%%%%%%%%%%%%%%%%%%%%%%%%%%%%
\subsection{測定における作業効率}

\begin{Issues}{測定基準値の手動による実測}
\begin{enumerate}[label=\sarrow]
\item[{\sarrow[red]}]\index{ワークざひょうけいげんてん@ワーク座標系原点}ワーク座標系原点の概算値を見出すために、オペレータが直接機内で測定している
\item[{\sarrow[red]}]機械的に見出だせるにもかかわらず、放置され続けている
\end{enumerate}
\end{Issues}

\begin{Issues}{測定時の送り速さ}
\begin{enumerate}[label=\sarrow]
\item[{\sarrow[red]}]すべての測定時の送り速さが1つの値で制御されている状態にある
\item[{\sarrow[red]}]すべての測定に対し、最も小さい送り速さに合わせる必要があり、必然的に測定に時間がかかってしまう
\end{enumerate}
\end{Issues}

\begin{Issues}{\CenterlineEndFaceDifMeasurement の作業効率}
\begin{enumerate}[label=\sarrow]
\item[{\sarrow[red]}]\CenterlineEndFaceDifMeasurement が、手動による\index{ハンドルそうさ@ハンドル操作}ハンドル操作で行われている
\item[{\sarrow[red]}]自動化が可能にもかかわらず、事態が放置され続けている
\end{enumerate}
\end{Issues}


\clearpage
%%%%%%%%%%%%%%%%%%%%%%%%%%%%%%%%%%%%%%%%%%%%%%%%%%%%%%%%%%
%% subsection 02.03.03 %%%%%%%%%%%%%%%%%%%%%%%%%%%%%%%%%%%
%%%%%%%%%%%%%%%%%%%%%%%%%%%%%%%%%%%%%%%%%%%%%%%%%%%%%%%%%%
\subsection{\EndFacecutMilling における作業効率}

\begin{Issues}{\EndFacecutMilling の手動による\TDCorrection}
\begin{enumerate}[label=\sarrow]
\item[{\sarrow[red]}]\EndFacecutMilling の基準点が外側中心として与えられており、\index{めいさい@明細}明細に依存する\indexTDFaceMill\nameTDCorrection を行わなければならない
\item[{\sarrow[red]}]作業者が手動で\indexTDFaceMill\nameTDCorrection 値を編集しなければならない状態にある
\item[{\sarrow[red]}]自動補正が可能にもかかわらず、事態が放置され続けている
\end{enumerate}
\end{Issues}

\begin{Issues}{\EndFacecutMilling の手動による加工回数の変更}
\begin{enumerate}[label=\sarrow]
\item[{\sarrow[red]}]\EndFacecutMilling の回数を変更する際、作業者が手動により直接メインプログラムを編集しなければならない状態にある
\end{enumerate}
\end{Issues}


%%%%%%%%%%%%%%%%%%%%%%%%%%%%%%%%%%%%%%%%%%%%%%%%%%%%%%%%%%
%% subsection 02.03.05 %%%%%%%%%%%%%%%%%%%%%%%%%%%%%%%%%%%
%%%%%%%%%%%%%%%%%%%%%%%%%%%%%%%%%%%%%%%%%%%%%%%%%%%%%%%%%%
\subsection{\KeywayMilling における作業効率}

\begin{Issues}{\KeywayMilling 回数($Z$方向)の手動による設定}
\begin{enumerate}[label=\sarrow]
\item[{\sarrow[red]}]\KeywayWidth・工具幅に応じた\KeywayMilling の回数を、作業者が決定しなければならない状態にある
\item[{\sarrow[red]}]\KeywayMilling の回数の変更は、作業者が\index{NCメインプログラム}NCメインプログラムを直接編集しなければならない状態にある
\item[{\sarrow[red]}]自動化が十分可能にもかかわらず、事態が放置され続けている
\end{enumerate}
\end{Issues}


%%%%%%%%%%%%%%%%%%%%%%%%%%%%%%%%%%%%%%%%%%%%%%%%%%%%%%%%%%
%% subsection 02.03.06 %%%%%%%%%%%%%%%%%%%%%%%%%%%%%%%%%%%
%%%%%%%%%%%%%%%%%%%%%%%%%%%%%%%%%%%%%%%%%%%%%%%%%%%%%%%%%%
\subsection{\EndFaceChamferMilling における作業効率}

\begin{Issues}{手作業による\EndFaceChamferMilling の加工}
\begin{enumerate}[label=\sarrow]
\item[{\sarrow[red]}]寸法の小さな\EndFaceChamferMilling が、手持ち研磨機を用いて手作業で加工されている
\item[{\sarrow[red]}]大半の場合は自動化が十分可能にもかかわらず、事態が放置され続けている
\end{enumerate}
\end{Issues}


%%%%%%%%%%%%%%%%%%%%%%%%%%%%%%%%%%%%%%%%%%%%%%%%%%%%%%%%%%
%% subsection 02.03.07 %%%%%%%%%%%%%%%%%%%%%%%%%%%%%%%%%%%
%%%%%%%%%%%%%%%%%%%%%%%%%%%%%%%%%%%%%%%%%%%%%%%%%%%%%%%%%%
\subsection{\index{とくしゅなかこう@特殊な加工}特殊な加工における作業効率}

\begin{Issues}{\CurvedOutcutMilling の\index{ハンドルそうさ@ハンドル操作}ハンドル操作による測定}
\begin{enumerate}[label=\sarrow]
\item[{\sarrow[red]}]手作業による\index{ハンドルそうさ@ハンドル操作}ハンドル操作により測定が行われている
\item[{\sarrow[red]}]自動化が十分可能にもかかわらず、事態が放置され続けている
\end{enumerate}
\end{Issues}

\begin{Issues}{\Keyway 8角形コーナーRの加工の未対応}
\begin{enumerate}[label=\sarrow]
\item[{\sarrow[red]}]8角形コーナーRの形状をした\KeywayMilling に対応できていない
\end{enumerate}
\end{Issues}


\clearpage
%%%%%%%%%%%%%%%%%%%%%%%%%%%%%%%%%%%%%%%%%%%%%%%%%%%%%%%%%%
%% subsection 02.03.08 %%%%%%%%%%%%%%%%%%%%%%%%%%%%%%%%%%%
%%%%%%%%%%%%%%%%%%%%%%%%%%%%%%%%%%%%%%%%%%%%%%%%%%%%%%%%%%
\subsection{加工全般における作業効率}

\begin{Issues}{\Table 回転による振分調整の未対応}
\begin{enumerate}[label=\sarrow]
\item[{\sarrow[red]}]振分調整が\Spacer を用いた方法のみでしかできない状態にある
\item[{\sarrow[red]}]再振分けを行う際に、必ず\Spacer の取付けまたは取外し作業を行わなければならない
\end{enumerate}
\end{Issues}

\begin{Issues}{メインプログラム編集による\index{おくりそくど@送り速度}送り速度の変更}
\begin{enumerate}[label=\sarrow]
\item[{\sarrow[red]}]\index{おくりそくど@送り速度}送り速度を変更するために、\index{NCメインプログラム}NCメインプログラムを直接編集しなければならない
\end{enumerate}
\end{Issues}

\begin{Issues}{メインプログラム編集による\index{しゅじくかいてんすう@主軸回転数}主軸回転数の変更}
\begin{enumerate}[label=\sarrow]
\item[{\sarrow[red]}]\index{しゅじくかいてんすう@主軸回転数}主軸回転数を変更するために、\index{NCメインプログラム}NCメインプログラムを直接編集しなければならない
\end{enumerate}
\end{Issues}


\clearpage
%%%%%%%%%%%%%%%%%%%%%%%%%%%%%%%%%%%%%%%%%%%%%%%%%%%%%%%%%%
%% section 02.03 %%%%%%%%%%%%%%%%%%%%%%%%%%%%%%%%%%%%%%%%%
%%%%%%%%%%%%%%%%%%%%%%%%%%%%%%%%%%%%%%%%%%%%%%%%%%%%%%%%%%
\modHeadsection{\index{しんらいせい@信頼性}信頼性に関するイシュー・問題}

\begin{Issues}{\index{だんきうんてん@暖機運転}暖機運転}
\begin{enumerate}[label=\sarrow]
\item[{\sarrow[red]}]十分な\index{だんきうんてん@暖機運転}暖機運転がなされていない
\end{enumerate}
\end{Issues}

\begin{Issues}{パラメタ誤入力の予防策}
\begin{enumerate}[label=\sarrow]
\item[{\sarrow[red]}]\index{NCメインプログラム}NCメインプログラム・\index{NCサブプログラム}NCサブプログラムともに、\index{ひきすう@引数}引数に対する誤入力の予防措置がなされていない
\item[{\sarrow[red]}]\index{NCメインプログラム}NCメインプログラム・\index{NCサブプログラム}NCサブプログラムともに、\index{コモンへんすう@コモン変数}コモン変数に対する誤入力の予防措置がなされていない
\end{enumerate}
\end{Issues}

\begin{Issues}{ワーク設置ミスによる衝突の防止策}
\begin{enumerate}[label=\sarrow]
\item[{\sarrow[red]}]ワークの設置にミスがあった場合に起因する衝突の防止策がなにも取られていない
\end{enumerate}
\end{Issues}

\begin{Issues}{\KeywayMilling 時 \index{がいぶワークざひょうけい@外部ワーク座標系}外部ワーク座標系による衝突の予防策}
\begin{enumerate}[label=\sarrow]
\item[{\sarrow[red]}]\index{がいぶワークざひょうけい@外部ワーク座標系}外部ワーク座標系による衝突の予防措置がなされていない
\item[{\sarrow[red]}]特に影響の大きい\KeywayMilling にさえなされていない
\end{enumerate}
\end{Issues}



\clearpage
%%%%%%%%%%%%%%%%%%%%%%%%%%%%%%%%%%%%%%%%%%%%%%%%%%%%%%%%%%
%% section 02.04 %%%%%%%%%%%%%%%%%%%%%%%%%%%%%%%%%%%%%%%%%
%%%%%%%%%%%%%%%%%%%%%%%%%%%%%%%%%%%%%%%%%%%%%%%%%%%%%%%%%%
\modHeadsection{全般的な問題点}
そもそも、以下のような非常に大きな問題が存在し続けている。

\begin{Issues}{根本的な問題}
\begin{enumerate}[label=\sarrow]
\item[{\sarrow[red]}] \expandafterindex{\yomiDrawing(モールド)@\nameDrawing(モールド)}\nameDrawing 作成と\index{NCプログラム}NCプログラム作成作業が直列の工程となっている
\item[{\sarrow[red]}] \expandafterindex{\yomiDrawing のさくせい)@\nameDrawing の作成}\nameDrawing の作成が、\index{CADソフトウェア}CADソフトウェアを用いてすべて手動でなされている
\item[{\sarrow[red]}] \index{モールド}モールドに関する\index{データベース@データベース}データベースに相当するものが存在しない
\end{enumerate}
\end{Issues}
これらについては、\pageautoref{part:XI}以降で取り扱う。



%%%%%%%%%%%%%%%%%%%%%%%%%%%%%%%%%%%%%%%%%%%%%%%%%%%%%%%%%%
%% chapter 3 %%%%%%%%%%%%%%%%%%%%%%%%%%%%%%%%%%%%%%%%%%%%%
%%%%%%%%%%%%%%%%%%%%%%%%%%%%%%%%%%%%%%%%%%%%%%%%%%%%%%%%%%
\modHeadchapter{\DMname 導入後の業務フロー\TBW}
%!TEX root = ../RfCPN.tex


\modHeadchapter{改善の余地の特定\TBW}
\MMC における改善項目は、そのほとんどが\DMC にも適用できることが想定される。


%%%%%%%%%%%%%%%%%%%%%%%%%%%%%%%%%%%%%%%%%%%%%%%%%%%%%%%%%%
%% section 3.2 %%%%%%%%%%%%%%%%%%%%%%%%%%%%%%%%%%%%%%%%%%%
%%%%%%%%%%%%%%%%%%%%%%%%%%%%%%%%%%%%%%%%%%%%%%%%%%%%%%%%%%
\modHeadsection{新たな技術の導入\TBW}
(to be written...)



%%%%%%%%%%%%%%%%%%%%%%%%%%%%%%%%%%%%%%%%%%%%%%%%%%%%%%%%%%
%% section 3.2 %%%%%%%%%%%%%%%%%%%%%%%%%%%%%%%%%%%%%%%%%%%
%%%%%%%%%%%%%%%%%%%%%%%%%%%%%%%%%%%%%%%%%%%%%%%%%%%%%%%%%%
\modHeadsection{業務プロセスの最適化\TBW}
(to be written...)



%%%%%%%%%%%%%%%%%%%%%%%%%%%%%%%%%%%%%%%%%%%%%%%%%%%%%%%%%%
%% section 3.2 %%%%%%%%%%%%%%%%%%%%%%%%%%%%%%%%%%%%%%%%%%%
%%%%%%%%%%%%%%%%%%%%%%%%%%%%%%%%%%%%%%%%%%%%%%%%%%%%%%%%%%
\modHeadsection{教育・トレーニングの強化}
(to be written...)


\begin{appendices}
%%%%%%%%%%%%%%%%%%%%%%%%%%%%%%%%%%%%%%%%%%%%%%%%%%%%%%%%%
%% Appendix Start %%%%%%%%%%%%%%%%%%%%%%%%%%%%%%%%%%%%%%%
%%%%%%%%%%%%%%%%%%%%%%%%%%%%%%%%%%%%%%%%%%%%%%%%%%%%%%%%%
%\Appendixpart

\end{appendices}
%%%%%%%%%%%%%%%%%%%%%%%%%%%%%%%%%%%%%%%%%%%%%%%%%%%%%%%%%
%% Appendix End %%%%%%%%%%%%%%%%%%%%%%%%%%%%%%%%%%%%%%%%%
%%%%%%%%%%%%%%%%%%%%%%%%%%%%%%%%%%%%%%%%%%%%%%%%%%%%%%%%%
\addtocontents{toc}{\protect\end{tocBox}}
\clearrightpage

\addtocontents{toc}{\protect\end{tcolorbox}}
\addtocontents{toc}{\protect\cleardoublepage} % add page break





%%%%%%%%%%%%%%%%%%%%%%%%%%%%%%%%%%%%%%%%%%%%%%%%%%%%%%%%%
%%               %%%%%%%%%%%%%%%%%%%%%%%%%%%%%%%%%%%%%%%%
%%               %%%%%%%%%%%%%%%%%%%%%%%%%%%%%%%%%%%%%%%%
%% Part Rules    %%%%%%%%%%%%%%%%%%%%%%%%%%%%%%%%%%%%%%%%
%%               %%%%%%%%%%%%%%%%%%%%%%%%%%%%%%%%%%%%%%%%
%%               %%%%%%%%%%%%%%%%%%%%%%%%%%%%%%%%%%%%%%%%
%%%%%%%%%%%%%%%%%%%%%%%%%%%%%%%%%%%%%%%%%%%%%%%%%%%%%%%%%
\addtocontents{toc}{\protect\begin{tcolorbox}[parttocstyle={Contents}{\the\numexpr\value{part}+1\relax}]}
\tPart{諸規程の策定}{概要}{%
\paragraph*{目標(なにがしたいか?)}
ソフトウェアおよびソフトウェアエンジニアリングに関する\textbf{業務を管理状態に}おく。
\tcbline*
\paragraph*{手段(どうやって?)}
マシニングを用いる業務が関わる(特にソフトウェア視点による)\textbf{規程の策定}を試みる。
\tcbline*
\paragraph*{背景(なぜ?)}
すべての業務において、何の決まりごともなく管理状態におくことは不可能である。

 しかし現時点(\DMname 設置時点)において、「ソフトウェアエンジニアリングに関する管理・業務を行う部門」は存在しておらず、専任の担当者さえも存在しない。
そのため、\textbf{ソフトウェアエンジニアリング視点による規程も存在しない}
%% footnote %%%%%%%%%%%%%%%%%%%%%
\footnote{「ソフトウェア」を「ハードウェア」に置き換えてみると、これがどれほどの異常事態であるかがわかる。}。
%%%%%%%%%%%%%%%%%%%%%%%%%%%%%%%%%
事実、社内のほとんどの業務におけるシステム化や自動化について、その\index{ようけんていぎ@要件定義}\textbf{要件定義を行うことすらままならない}状態が長く放置されている。

 したがって、(\DMname 関連に限らず)こうした視点による規程を設けることは焦眉の課題である。
\tcbline*
\paragraph*{結論(どうなった?)}
ソフトウェアの観点のもと、\DMname 関連の業務が関わるソフトウェア開発・保守等の規程を策定した。
}
%!TEX root = ../RPA_for_Creating_Program_Note.tex


\addtocontents{toc}{\protect\cleardoublepage}
%%%%%%%%%%%%%%%%%%%%%%%%%%%%%%%%%%%%%%%%%%%%%%%%%%%%%%%%%
%% Part Rules    %%%%%%%%%%%%%%%%%%%%%%%%%%%%%%%%%%%%%%%%
%%%%%%%%%%%%%%%%%%%%%%%%%%%%%%%%%%%%%%%%%%%%%%%%%%%%%%%%%
\addtocontents{toc}{\protect\begin{tocBox}{\tmppartnum}}%
\tPart{諸規程の策定}{概要}{%
\paragraph*{目標(なにがしたいか?)}
\index{ソフトウェア}ソフトウェアおよび\index{ソフトウェアエンジニアリング}ソフトウェアエンジニアリングに関する\textbf{業務を管理状態に}おく。
\tcbline*
\paragraph*{手段(どうやって?)}
\index{よこがたマシニング@横型マシニング}横型マシニングを用いる業務が関わる(特にソフトウェア視点による)\textbf{規程の策定}を試みる。
\tcbline*
\paragraph*{背景(なぜ?)}
すべての業務において、何の決まりごともなく管理状態におくことは不可能である。

 しかし現時点(\DMname 設置時点)において、「ソフトウェアエンジニアリングに関する管理・業務を行う部門」は存在しておらず、専任の担当者さえも存在しない。
そのため、\textbf{ソフトウェアエンジニアリング視点による規程も存在しない}
%% footnote %%%%%%%%%%%%%%%%%%%%%
\footnote{「ソフトウェア」を「\index{ハードウェア}ハードウェア」に置き換えてみると、これがどれほどの異常事態であるかがわかる。}。
%%%%%%%%%%%%%%%%%%%%%%%%%%%%%%%%%
事実、社内のほとんどの業務におけるシステム化や自動化について、その\index{ようけんていぎ@要件定義}\textbf{要件定義を行うことすらままならない}状態が長く放置されている。

 したがって、(\DMname 関連に限らず)こうした視点による規程を設けることは焦眉の課題である。
\tcbline*
\paragraph*{結論(どうなった?)}
ソフトウェアの観点のもと、\DMname 関連の業務が関わる\index{ソフトウェアかいはつ@ソフトウェア開発}ソフトウェア開発・保守等の規程を策定した。
}

%%%%%%%%%%%%%%%%%%%%%%%%%%%%%%%%%%%%%%%%%%%%%%%%%%%%%%%%%%
%% chapter 04 %%%%%%%%%%%%%%%%%%%%%%%%%%%%%%%%%%%%%%%%%%%%
%%%%%%%%%%%%%%%%%%%%%%%%%%%%%%%%%%%%%%%%%%%%%%%%%%%%%%%%%%
\modHeadchapter{情報処理技術者および技術水準に関する規程}
%!TEX root = ../RPA_for_Creating_Program_Note.tex



%%%%%%%%%%%%%%%%%%%%%%%%%%%%%%%%%%%%%%%%%%%%%%%%%%%%%%%%%%
%% section 12.1 %%%%%%%%%%%%%%%%%%%%%%%%%%%%%%%%%%%%%%%%%%
%%%%%%%%%%%%%%%%%%%%%%%%%%%%%%%%%%%%%%%%%%%%%%%%%%%%%%%%%%
\modHeadsection{目的\TBW}
\begin{enumerate}
\item 情報処理技術者に目標を示し、刺激を与えることによって、その技術の向上に資すること
\item 情報処理技術者として備えるべき能力についての水準を示すことにより、職業教育・企業内教育等における教育の水準の確保に資すること
\item 情報処理技術者の採用を行う際に役立つよう客観的な評価の尺度を提供すること
\end{enumerate}

%%%%%%%%%%%%%%%%%%%%%%%%%%%%%%%%%%%%%%%%%%%%%%%%%%%%%%%%%%
%% subsection 12.1.1 %%%%%%%%%%%%%%%%%%%%%%%%%%%%%%%%%%%%%
%%%%%%%%%%%%%%%%%%%%%%%%%%%%%%%%%%%%%%%%%%%%%%%%%%%%%%%%%%
\subsection{情報処理技術者の区分\TBW}


%%%%%%%%%%%%%%%%%%%%%%%%%%%%%%%%%%%%%%%%%%%%%%%%%%%%%%%%%%
%% chapter 05 %%%%%%%%%%%%%%%%%%%%%%%%%%%%%%%%%%%%%%%%%%%%
%%%%%%%%%%%%%%%%%%%%%%%%%%%%%%%%%%%%%%%%%%%%%%%%%%%%%%%%%%
\modHeadchapter{システムおよびソフトウェアの作成に関する規程}
%!TEX root = ../RfCPN.tex


\modHeadchapter{システムおよびソフトウェアの作成に関する規程}



%%%%%%%%%%%%%%%%%%%%%%%%%%%%%%%%%%%%%%%%%%%%%%%%%%%%%%%%%%
%% section 12.1 %%%%%%%%%%%%%%%%%%%%%%%%%%%%%%%%%%%%%%%%%%
%%%%%%%%%%%%%%%%%%%%%%%%%%%%%%%%%%%%%%%%%%%%%%%%%%%%%%%%%%
\modHeadsection{開発プロセス}

%%%%%%%%%%%%%%%%%%%%%%%%%%%%%%%%%%%%%%%%%%%%%%%%%%%%%%%%%%
%% subsection 12.1.1 %%%%%%%%%%%%%%%%%%%%%%%%%%%%%%%%%%%%%
%%%%%%%%%%%%%%%%%%%%%%%%%%%%%%%%%%%%%%%%%%%%%%%%%%%%%%%%%%
\subsection{要件定義}
新規機能の開発または既存機能の改修を行う際は、\index{ようけんていぎ@要件定義}要件定義を行うものとする。
要件定義については関係者全員でレビューを実施し、必要に応じて要件定義書の作成・更新を行う。

なおこのプロセスは、区分\ref{item:ITseg4}\hx に該当する者が行うものとする。

%%%%%%%%%%%%%%%%%%%%%%%%%%%%%%%%%%%%%%%%%%%%%%%%%%%%%%%%%%
%% subsection 12.1.2 %%%%%%%%%%%%%%%%%%%%%%%%%%%%%%%%%%%%%
%%%%%%%%%%%%%%%%%%%%%%%%%%%%%%%%%%%%%%%%%%%%%%%%%%%%%%%%%%
\subsection{基本設計}
要件定義書の内容に基づいて\index{きほんせっけい@基本設計}基本設計を行うものとする。
基本設計については関係者全員でレビューを実施し、必要に応じて基本設計書の作成・更新を行う。

なおこのプロセスは、区分\ref{item:ITseg4}\hx に該当する者、あるいは区分\ref{item:ITseg4}\hx に該当する者の指導のもとで区分\ref{item:ITseg3}\hx に該当する者が行うものとする。

%%%%%%%%%%%%%%%%%%%%%%%%%%%%%%%%%%%%%%%%%%%%%%%%%%%%%%%%%%
%% subsection 12.1.3 %%%%%%%%%%%%%%%%%%%%%%%%%%%%%%%%%%%%%
%%%%%%%%%%%%%%%%%%%%%%%%%%%%%%%%%%%%%%%%%%%%%%%%%%%%%%%%%%
\subsection{詳細設計}
基本設計書の内容に基づいて\index{しょうさいせっけい@詳細設計}詳細設計を行うものとする。
詳細設計書については関係者全員でレビューを実施し、必要に応じて詳細設計書を作成・更新する。

なおこのプロセスは、区分\ref{item:ITseg4}\hx に該当する者、あるいは区分\ref{item:ITseg4}\hx に該当する者の指導のもとで区分\ref{item:ITseg3}\hx に該当する者が行うものとする。

%%%%%%%%%%%%%%%%%%%%%%%%%%%%%%%%%%%%%%%%%%%%%%%%%%%%%%%%%%
%% subsection 12.1.4 %%%%%%%%%%%%%%%%%%%%%%%%%%%%%%%%%%%%%
%%%%%%%%%%%%%%%%%%%%%%%%%%%%%%%%%%%%%%%%%%%%%%%%%%%%%%%%%%
\subsection{コードの記述}
詳細設計書の内容に基づいて、\index{コード}コードの記述を行うものとする。
記述は、\index{コーディングルール}コーディングルールに従って行う。

なおこのプロセスは、区分\ref{item:ITseg3}\hx に該当する者が行うものとする。

%%%%%%%%%%%%%%%%%%%%%%%%%%%%%%%%%%%%%%%%%%%%%%%%%%%%%%%%%%
%% subsection 12.1.5 %%%%%%%%%%%%%%%%%%%%%%%%%%%%%%%%%%%%%
%%%%%%%%%%%%%%%%%%%%%%%%%%%%%%%%%%%%%%%%%%%%%%%%%%%%%%%%%%
\subsection{コードレビュー}
コードの記述の完了に伴い、そのコードは\index{コードレビュー}コードレビューを受けるものとする。
レビューの結果にもと、必要があれば修正・改善を行う。

%%%%%%%%%%%%%%%%%%%%%%%%%%%%%%%%%%%%%%%%%%%%%%%%%%%%%%%%%%
%% subsection 12.1.6 %%%%%%%%%%%%%%%%%%%%%%%%%%%%%%%%%%%%%
%%%%%%%%%%%%%%%%%%%%%%%%%%%%%%%%%%%%%%%%%%%%%%%%%%%%%%%%%%
\subsection{テスト}
コードレビューを受けたコードについて、正しく動作することを確認するためにテストを行うものとする。
テストの方法は予め定めておき、その方法に従ってテストを実施する。
テスト結果に基づいて、必要があれば修正・改善を行う。

なおこのプロセスは、区分\ref{item:ITseg3}\hx に該当する者が行うものとする。

\clearpage
%%%%%%%%%%%%%%%%%%%%%%%%%%%%%%%%%%%%%%%%%%%%%%%%%%%%%%%%%%
%% subsection 12.1.7 %%%%%%%%%%%%%%%%%%%%%%%%%%%%%%%%%%%%%
%%%%%%%%%%%%%%%%%%%%%%%%%%%%%%%%%%%%%%%%%%%%%%%%%%%%%%%%%%
\subsection{テスト環境での動作確認}
テスト結果に問題のないことが確認された場合、コードを\index{テストかんきょう@テスト環境}テスト環境にデプロイし、動作の確認を実施する。
確認事項は予め定めておき、それに従って確認を実施する。
動作確認の結果に基づいて、必要があれば修正・改善を行う。

なおこのプロセスは、区分\ref{item:ITseg3}\hx に該当する者が行うものとする。

%%%%%%%%%%%%%%%%%%%%%%%%%%%%%%%%%%%%%%%%%%%%%%%%%%%%%%%%%%
%% subsection 12.1.8 %%%%%%%%%%%%%%%%%%%%%%%%%%%%%%%%%%%%%
%%%%%%%%%%%%%%%%%%%%%%%%%%%%%%%%%%%%%%%%%%%%%%%%%%%%%%%%%%
\subsection{本番環境へのリリース}
\index{テストかんきょう@テスト環境}テスト環境による動作に問題のないことが確認された場合、\index{ほんばんかんきょう@本番環境}本番環境にリリースを行うものとする。

なおこのプロセスは、区分\ref{item:ITseg4}\hx に該当する者、あるいは区分\ref{item:ITseg4}\hx に該当する者の指導のもとで区分\ref{item:ITseg3}\hx に該当する者が行うものとする。

%%%%%%%%%%%%%%%%%%%%%%%%%%%%%%%%%%%%%%%%%%%%%%%%%%%%%%%%%%
%% subsection 12.1.9 %%%%%%%%%%%%%%%%%%%%%%%%%%%%%%%%%%%%%
%%%%%%%%%%%%%%%%%%%%%%%%%%%%%%%%%%%%%%%%%%%%%%%%%%%%%%%%%%
\subsection{変更管理}
開発プロセス中に要件・設計・コード等が変更される場合、\index{へんこうようきゅう@変更要求}変更要求書を作成し、関係者全員でレビューを行うものとする。
\index{へんこうようきゅうしょ@変更要求書}変更要求書には、変更の理由・影響範囲・必要なリソース等を明記する。
変更が承認された場合、変更要求書に基づいて関連する文書やコードの更新を行う。

なお、区分\ref{item:ITseg4}\hx に該当する者が管理を行うものとする。

%%%%%%%%%%%%%%%%%%%%%%%%%%%%%%%%%%%%%%%%%%%%%%%%%%%%%%%%%%
%% subsection 12.1.10 %%%%%%%%%%%%%%%%%%%%%%%%%%%%%%%%%%%%
%%%%%%%%%%%%%%%%%%%%%%%%%%%%%%%%%%%%%%%%%%%%%%%%%%%%%%%%%%
\subsection{リスク管理}
プロジェクトの開始時に、\index{リスクかんり@リスク管理}リスク管理計画を行うものとする。
\index{リスクかんりけいかく@リスク管理計画}リスク管理計画には、潜在的なリスクの特定、リスクの評価、リスク対策の策定、リスクの監視と管理が含まれる。
リスク管理計画はプロジェクト期間中に定期的に見直し、更新を実施する。

なお、区分\ref{item:ITseg4}\hx に該当する者が管理を行うものとする。

%%%%%%%%%%%%%%%%%%%%%%%%%%%%%%%%%%%%%%%%%%%%%%%%%%%%%%%%%%
%% subsection 12.1.10 %%%%%%%%%%%%%%%%%%%%%%%%%%%%%%%%%%%%
%%%%%%%%%%%%%%%%%%%%%%%%%%%%%%%%%%%%%%%%%%%%%%%%%%%%%%%%%%
\subsection{ドキュメンテーション}
各々の業務において、その全体像を理解するために、適切な\index{ドキュメンテーション}ドキュメンテーションの作成を行うものとする。
これには、\index{ようけんていぎしょ@要件定義書}要件定義書・\index{せっけいしょ@設計書}設計書・\index{テストけいかく@テスト計画}テスト計画・\index{ユーザーズマニュアル}ユーザーズマニュアル・\index{さぎょうてじゅんしょ@作業手順書}作業手順書等が含まれる。
ドキュメンテーションは常に最新の状態を保ち、関係者が容易にアクセスできる場所に保存を行う。

なお、これらはそのプロジェクトの各段階の責任者に相当する者が作成し、区分\ref{item:ITseg4}\hx に該当する者が管理を行うものとする。

%%%%%%%%%%%%%%%%%%%%%%%%%%%%%%%%%%%%%%%%%%%%%%%%%%%%%%%%%%
%% subsection 12.1.11 %%%%%%%%%%%%%%%%%%%%%%%%%%%%%%%%%%%%
%%%%%%%%%%%%%%%%%%%%%%%%%%%%%%%%%%%%%%%%%%%%%%%%%%%%%%%%%%
\subsection{緊急時の対応}
\index{ほんばんかんきょう@本番環境}本番環境で重大な問題が発生した際は、速やかに直属の上司および\index{システムかんりしゃ@システム管理者}システム管理者に報告する。
システム管理者は速やかに問題の原因の特定を試み、適切な対応策の実施を行う。



\clearpage
%%%%%%%%%%%%%%%%%%%%%%%%%%%%%%%%%%%%%%%%%%%%%%%%%%%%%%%%%%
%% section 20.2 %%%%%%%%%%%%%%%%%%%%%%%%%%%%%%%%%%%%%%%%%%
%%%%%%%%%%%%%%%%%%%%%%%%%%%%%%%%%%%%%%%%%%%%%%%%%%%%%%%%%%
\modHeadsection{コードレビュー}
\index{コードレビュー}コードレビューとは、\index{ソースコード}ソースコードの作成者とは別の人物がコードを詳しく調べて問題がないか検討することであり、プログラムの品質を維持するために欠かせない工程である。

%%%%%%%%%%%%%%%%%%%%%%%%%%%%%%%%%%%%%%%%%%%%%%%%%%%%%%%%%%
%% subsection 08.2.1 %%%%%%%%%%%%%%%%%%%%%%%%%%%%%%%%%%%%%
%%%%%%%%%%%%%%%%%%%%%%%%%%%%%%%%%%%%%%%%%%%%%%%%%%%%%%%%%%
\subsection{レビューおよびレビュアー}
新規に作成または修正したソースコードは、コードレビューを受けるものとする。
コードレビューを行う者(\index{レビュアー(コードレビュー)}レビュアー)は、そのコードに関連する技術水準を有することが要求され、可能な限り作成者とは異なる他の開発者であることが望ましい。

なお、レビュアーの選定は区分\ref{item:ITseg3}\hx に該当する者が行うものとする。

%%%%%%%%%%%%%%%%%%%%%%%%%%%%%%%%%%%%%%%%%%%%%%%%%%%%%%%%%%
%% subsection 08.2.2 %%%%%%%%%%%%%%%%%%%%%%%%%%%%%%%%%%%%%
%%%%%%%%%%%%%%%%%%%%%%%%%%%%%%%%%%%%%%%%%%%%%%%%%%%%%%%%%%
\subsection{レビューの範囲}
コードレビューの対象となるコードは、新規に作成されたコード、修正されたコード、およびそれらのコードに直接影響を与える可能性のある既存のコードとする。
レビューの範囲は、レビュアーとコードの作成者が協議して決定する。

%%%%%%%%%%%%%%%%%%%%%%%%%%%%%%%%%%%%%%%%%%%%%%%%%%%%%%%%%%
%% subsection 12.3.2 %%%%%%%%%%%%%%%%%%%%%%%%%%%%%%%%%%%%%
%%%%%%%%%%%%%%%%%%%%%%%%%%%%%%%%%%%%%%%%%%%%%%%%%%%%%%%%%%
\subsection{コードレビューの承認}
コードレビューでは、以下のすべての観点について確認を行い、これらをすべて満たしている場合にのみ承認を行うものとする。
\begin{enumerate}[label=\sarrow]
\item コーディングルールに従っていること
\item 意図したとおりの実装が行われていること
\item 不具合につながる部分が存在しないこと
\end{enumerate}

%%%%%%%%%%%%%%%%%%%%%%%%%%%%%%%%%%%%%%%%%%%%%%%%%%%%%%%%%%
%% subsection 12.3.2 %%%%%%%%%%%%%%%%%%%%%%%%%%%%%%%%%%%%%
%%%%%%%%%%%%%%%%%%%%%%%%%%%%%%%%%%%%%%%%%%%%%%%%%%%%%%%%%%
\subsection{フィードバックの提供}
レビュアーは、レビューの結果を明確に伝えるために、具体的なコメントや改善提案の提供を行う。
フィードバックは建設的であるべきであり、問題だけでなく良い点や改善の提案も含めることが推奨される。

%%%%%%%%%%%%%%%%%%%%%%%%%%%%%%%%%%%%%%%%%%%%%%%%%%%%%%%%%%
%% subsection 12.3.3 %%%%%%%%%%%%%%%%%%%%%%%%%%%%%%%%%%%%%
%%%%%%%%%%%%%%%%%%%%%%%%%%%%%%%%%%%%%%%%%%%%%%%%%%%%%%%%%%
\subsection{コードレビューの時期}
開発プロセスにおけるコードレビューの実施タイミングは、予め定められたスケジュールに従うものとする。



\clearpage
%%%%%%%%%%%%%%%%%%%%%%%%%%%%%%%%%%%%%%%%%%%%%%%%%%%%%%%%%%
%% section 12.3 %%%%%%%%%%%%%%%%%%%%%%%%%%%%%%%%%%%%%%%%%%
%%%%%%%%%%%%%%%%%%%%%%%%%%%%%%%%%%%%%%%%%%%%%%%%%%%%%%%%%%
\modHeadsection{バージョン管理}
\index{バージョンかんりシステム@バージョン管理システム}バージョン管理システムとは、\index{ソースコード}ソースコードや\index{ドキュメンテーション}ドキュメンテーション等の\index{へんこうりれき(ソースコード)@変更履歴(ソースコード)}変更履歴を追跡し、必要に応じて以前の\index{バージョン}バージョンに戻すことができるシステムである。
これにより複数の開発者による同時作業や、過去のバージョンに戻すこと等ができ、生産性の向上に大きく寄与する。

%%%%%%%%%%%%%%%%%%%%%%%%%%%%%%%%%%%%%%%%%%%%%%%%%%%%%%%%%%
%% subsection 12.3.1 %%%%%%%%%%%%%%%%%%%%%%%%%%%%%%%%%%%%%
%%%%%%%%%%%%%%%%%%%%%%%%%%%%%%%%%%%%%%%%%%%%%%%%%%%%%%%%%%
\subsection{バージョン管理の目的}
ソフトウェア開発の際は、ソースコードの変更履歴を追跡し、必要に応じて以前のバージョンに戻すことができるように、バージョン管理を行うものとする。

%%%%%%%%%%%%%%%%%%%%%%%%%%%%%%%%%%%%%%%%%%%%%%%%%%%%%%%%%%
%% subsection 12.3.2 %%%%%%%%%%%%%%%%%%%%%%%%%%%%%%%%%%%%%
%%%%%%%%%%%%%%%%%%%%%%%%%%%%%%%%%%%%%%%%%%%%%%%%%%%%%%%%%%
\subsection{バージョン管理の手順}
各々の文書またはソースコードは、それぞれ1つのバージョン管理システムを用いて管理を行うものとする。
新規に作成または修正したソースコードは、バージョン管理システムにコミットする。
各コミットには、変更内容を説明する文言を必ず含める。

%%%%%%%%%%%%%%%%%%%%%%%%%%%%%%%%%%%%%%%%%%%%%%%%%%%%%%%%%%
%% subsection 12.3.3 %%%%%%%%%%%%%%%%%%%%%%%%%%%%%%%%%%%%%
%%%%%%%%%%%%%%%%%%%%%%%%%%%%%%%%%%%%%%%%%%%%%%%%%%%%%%%%%%
\subsection{コミットおよびコードレビュー}
バージョン管理システムにコミットする前に、そのソースコードはコードレビューを受けるものとする。
コードレビューの結果に基づいて、必要に応じてソースコードの修正・改善を行う。
問題がないと判断された場合、コミットを行う。

%%%%%%%%%%%%%%%%%%%%%%%%%%%%%%%%%%%%%%%%%%%%%%%%%%%%%%%%%%
%% subsection 12.3.4 %%%%%%%%%%%%%%%%%%%%%%%%%%%%%%%%%%%%%
%%%%%%%%%%%%%%%%%%%%%%%%%%%%%%%%%%%%%%%%%%%%%%%%%%%%%%%%%%
\subsection{変更およびコミットの時期}
バージョン変更およびコミットの実施時期は、ソースコードの新規作成または修正が完了した時点に行うものとする。



\clearpage
%%%%%%%%%%%%%%%%%%%%%%%%%%%%%%%%%%%%%%%%%%%%%%%%%%%%%%%%%%
%% section 12.4 %%%%%%%%%%%%%%%%%%%%%%%%%%%%%%%%%%%%%%%%%%
%%%%%%%%%%%%%%%%%%%%%%%%%%%%%%%%%%%%%%%%%%%%%%%%%%%%%%%%%%
\modHeadsection{テスト}

%%%%%%%%%%%%%%%%%%%%%%%%%%%%%%%%%%%%%%%%%%%%%%%%%%%%%%%%%%
%% subsection 12.4.1 %%%%%%%%%%%%%%%%%%%%%%%%%%%%%%%%%%%%%
%%%%%%%%%%%%%%%%%%%%%%%%%%%%%%%%%%%%%%%%%%%%%%%%%%%%%%%%%%
\subsection{テストの目的}
ソフトウェア開発の際は、ソフトウェアが期待通りの動作をすることを確認するために、テストを行うものとする。

%%%%%%%%%%%%%%%%%%%%%%%%%%%%%%%%%%%%%%%%%%%%%%%%%%%%%%%%%%
%% subsection 12.4.2 %%%%%%%%%%%%%%%%%%%%%%%%%%%%%%%%%%%%%
%%%%%%%%%%%%%%%%%%%%%%%%%%%%%%%%%%%%%%%%%%%%%%%%%%%%%%%%%%
\subsection{テストケースの作成}
テストケースは、要件定義書および設計書に基づいて作成するものとする。
すべての機能・シナリオをカバーするように、テストケースを作成する。

なお、テストケースの作成は、品質管理の能力を有するものが行うものとする。

%%%%%%%%%%%%%%%%%%%%%%%%%%%%%%%%%%%%%%%%%%%%%%%%%%%%%%%%%%
%% subsection 12.4.3 %%%%%%%%%%%%%%%%%%%%%%%%%%%%%%%%%%%%%
%%%%%%%%%%%%%%%%%%%%%%%%%%%%%%%%%%%%%%%%%%%%%%%%%%%%%%%%%%
\subsection{テストの手順}
新規に作成または修正したソースコードは、テストを行うものとする。
テストは、予め定められたテスト計画に従って実施する。
テストの結果に基づいて、必要に応じてソースコードの修正・改善を行う。

%%%%%%%%%%%%%%%%%%%%%%%%%%%%%%%%%%%%%%%%%%%%%%%%%%%%%%%%%%
%% subsection 12.4.4 %%%%%%%%%%%%%%%%%%%%%%%%%%%%%%%%%%%%%
%%%%%%%%%%%%%%%%%%%%%%%%%%%%%%%%%%%%%%%%%%%%%%%%%%%%%%%%%%
\subsection{テストの承認}
すべてのテストケースが成功した場合にのみ、そのソースコードは承認されるものとする。

%%%%%%%%%%%%%%%%%%%%%%%%%%%%%%%%%%%%%%%%%%%%%%%%%%%%%%%%%%
%% subsection 12.4.5 %%%%%%%%%%%%%%%%%%%%%%%%%%%%%%%%%%%%%
%%%%%%%%%%%%%%%%%%%%%%%%%%%%%%%%%%%%%%%%%%%%%%%%%%%%%%%%%%
\subsection{テストの時期}
開発プロセスにおけるテストの実施タイミングは、ソースコードの新規作成または修正が完了した時点とする。

%%%%%%%%%%%%%%%%%%%%%%%%%%%%%%%%%%%%%%%%%%%%%%%%%%%%%%%%%%
%% subsection 12.4.5 %%%%%%%%%%%%%%%%%%%%%%%%%%%%%%%%%%%%%
%%%%%%%%%%%%%%%%%%%%%%%%%%%%%%%%%%%%%%%%%%%%%%%%%%%%%%%%%%
\subsection{リグレッションテスト}
プログラムの変更や修正等により新たなコードを追加した際は、その新たなコードが既存の機能に影響を与えていないことを確認するため、必要に応じてリグレッションテストを行うものとする。



\clearpage
%%%%%%%%%%%%%%%%%%%%%%%%%%%%%%%%%%%%%%%%%%%%%%%%%%%%%%%%%%
%% section 12.5 %%%%%%%%%%%%%%%%%%%%%%%%%%%%%%%%%%%%%%%%%%
%%%%%%%%%%%%%%%%%%%%%%%%%%%%%%%%%%%%%%%%%%%%%%%%%%%%%%%%%%
\modHeadsection{ドキュメンテーション}

%%%%%%%%%%%%%%%%%%%%%%%%%%%%%%%%%%%%%%%%%%%%%%%%%%%%%%%%%%
%% subsection 12.5.1 %%%%%%%%%%%%%%%%%%%%%%%%%%%%%%%%%%%%%
%%%%%%%%%%%%%%%%%%%%%%%%%%%%%%%%%%%%%%%%%%%%%%%%%%%%%%%%%%
\subsection{ドキュメンテーションの目的}
ソフトウェア開発の際は、ソフトウェアの使用方法、設計、変更履歴などを記録し、開発者やユーザーが理解しやすくするためにドキュメンテーションを行うものとする。

%%%%%%%%%%%%%%%%%%%%%%%%%%%%%%%%%%%%%%%%%%%%%%%%%%%%%%%%%%
%% subsection 12.5.2 %%%%%%%%%%%%%%%%%%%%%%%%%%%%%%%%%%%%%
%%%%%%%%%%%%%%%%%%%%%%%%%%%%%%%%%%%%%%%%%%%%%%%%%%%%%%%%%%
\subsection{ドキュメンテーションの作成}
各々の業務において、その全体像を理解するために、適切なドキュメンテーションの作成を行うものとする。
これには、要件定義書、基本設計書、詳細設計書、ユーザーマニュアル等が含まれる。

%%%%%%%%%%%%%%%%%%%%%%%%%%%%%%%%%%%%%%%%%%%%%%%%%%%%%%%%%%
%% subsection 12.5.3 %%%%%%%%%%%%%%%%%%%%%%%%%%%%%%%%%%%%%
%%%%%%%%%%%%%%%%%%%%%%%%%%%%%%%%%%%%%%%%%%%%%%%%%%%%%%%%%%
\subsection{ドキュメンテーションの更新}
ドキュメンテーションは常に最新の状態を保つものとする。
新規に作成または修正したソースコードに関連するドキュメンテーションは、ソースコードの新規作成または修正が完了した時点で更新する。

%%%%%%%%%%%%%%%%%%%%%%%%%%%%%%%%%%%%%%%%%%%%%%%%%%%%%%%%%%
%% subsection 12.5.4 %%%%%%%%%%%%%%%%%%%%%%%%%%%%%%%%%%%%%
%%%%%%%%%%%%%%%%%%%%%%%%%%%%%%%%%%%%%%%%%%%%%%%%%%%%%%%%%%
\subsection{ドキュメンテーションの保存}
ドキュメンテーションは関係者が容易にアクセスできる場所に保存または提示を行うものとする。
また、最新でないドキュメンテーションは原則として提示せず、必要のない限り紙媒体(印刷物)・電子媒体にかかわらず、速やかに破棄をする。



%%%%%%%%%%%%%%%%%%%%%%%%%%%%%%%%%%%%%%%%%%%%%%%%%%%%%%%%%%
%% chapter 06 %%%%%%%%%%%%%%%%%%%%%%%%%%%%%%%%%%%%%%%%%%%%
%%%%%%%%%%%%%%%%%%%%%%%%%%%%%%%%%%%%%%%%%%%%%%%%%%%%%%%%%%
\modHeadchapter{マシニングセンタにおける工具の取扱い}
%!TEX root = ../RPA_for_Creating_Program_Note.tex


\modHeadchapter{マシニングセンタにおける工具の取扱い}

%%%%%%%%%%%%%%%%%%%%%%%%%%%%%%%%%%%%%%%%%%%%%%%%%%%%%%%%%%
%% section 06.1 %%%%%%%%%%%%%%%%%%%%%%%%%%%%%%%%%%%%%%%%%%
%%%%%%%%%%%%%%%%%%%%%%%%%%%%%%%%%%%%%%%%%%%%%%%%%%%%%%%%%%
\modHeadsection{工具番号}

%%%%%%%%%%%%%%%%%%%%%%%%%%%%%%%%%%%%%%%%%%%%%%%%%%%%%%%%%%
%% subsection 06.1.1 %%%%%%%%%%%%%%%%%%%%%%%%%%%%%%%%%%%%%
%%%%%%%%%%%%%%%%%%%%%%%%%%%%%%%%%%%%%%%%%%%%%%%%%%%%%%%%%%
\subsection{工具番号の設定}
各々の工具は、一意の\index{こうぐばんごう@工具番号}工具番号を持つものとする。
また、工具番号の設定は、予め定められた規則にしたがって行うものとする。

%%%%%%%%%%%%%%%%%%%%%%%%%%%%%%%%%%%%%%%%%%%%%%%%%%%%%%%%%%
%% subsection 05.1.2 %%%%%%%%%%%%%%%%%%%%%%%%%%%%%%%%%%%%%
%%%%%%%%%%%%%%%%%%%%%%%%%%%%%%%%%%%%%%%%%%%%%%%%%%%%%%%%%%
\subsection{工具の登録}
工具番号が設定された工具は、その番号を用いて\index{マシニングセンタ}マシニングセンタに登録を行うものとする。

%%%%%%%%%%%%%%%%%%%%%%%%%%%%%%%%%%%%%%%%%%%%%%%%%%%%%%%%%%
%% subsection 05.1.3 %%%%%%%%%%%%%%%%%%%%%%%%%%%%%%%%%%%%%
%%%%%%%%%%%%%%%%%%%%%%%%%%%%%%%%%%%%%%%%%%%%%%%%%%%%%%%%%%
\subsection{工具番号の管理}
すべての\index{とうろくこうぐ@登録工具}登録工具とそれらの工具番号は、定期的に確認および更新を行うものとする。
新規に工具が追加された場合や、交換・修理・廃棄等により既存の工具番号の変更が生じた場合は、速やかに工具の\index{とうろくじょうほう@登録情報}登録情報を更新する。
また現在の登録工具番号の情報は、関係者が容易にアクセスできる場所に保存または提示を行う。


%%%%%%%%%%%%%%%%%%%%%%%%%%%%%%%%%%%%%%%%%%%%%%%%%%%%%%%%%%
%% section 05.2 %%%%%%%%%%%%%%%%%%%%%%%%%%%%%%%%%%%%%%%%%%
%%%%%%%%%%%%%%%%%%%%%%%%%%%%%%%%%%%%%%%%%%%%%%%%%%%%%%%%%%
\modHeadsection{工具補正値の設定}
登録された工具は、工具の設置時に\index{こうぐちょう@工具長}工具長および\index{こうぐけい@工具径}工具径を測定し、該当する登録番号の工具に対し適切な\index{こうぐちょうほせいち@工具長補正値}工具長補正値および\index{こうぐけいほせいち@工具径補正値}工具径補正値を設定する。
\index{まもう@摩耗}摩耗等で工具長または工具径に変更が生じた場合は、該当する登録番号の工具に対して適切な\index{こうぐまもうりょう@工具摩耗量}摩耗量を設定する。

また、この他に必要な補正量が存在する場合は、適当な\index{コモンへんすう@コモン変数}コモン変数を用いて設定を行うものとする。



%%%%%%%%%%%%%%%%%%%%%%%%%%%%%%%%%%%%%%%%%%%%%%%%%%%%%%%%%%
%% section 05.2 %%%%%%%%%%%%%%%%%%%%%%%%%%%%%%%%%%%%%%%%%%
%%%%%%%%%%%%%%%%%%%%%%%%%%%%%%%%%%%%%%%%%%%%%%%%%%%%%%%%%%
\modHeadsection{工具の送り速さ値および主軸回転数}

%%%%%%%%%%%%%%%%%%%%%%%%%%%%%%%%%%%%%%%%%%%%%%%%%%%%%%%%%%
%% subsection 05.2.1 %%%%%%%%%%%%%%%%%%%%%%%%%%%%%%%%%%%%%
%%%%%%%%%%%%%%%%%%%%%%%%%%%%%%%%%%%%%%%%%%%%%%%%%%%%%%%%%%
\subsection{送り速さ値および主軸回転数の設定}
各々の工具の\index{おくりはやさち@送り速さ値}送り速さ値および\index{しゅじくかいてんすう@主軸回転数}主軸回転数は、工具の最適なパフォーマンスを確保するために適切に設定するものとする。
送り速さ値および主軸回転数の設定は工具の種類・材料・加工条件等に基づいて行う。
また、現在の送り速さ値ならびに主軸回転数の情報は、関係者が容易にアクセスできる場所に保存または提示を行う。

%%%%%%%%%%%%%%%%%%%%%%%%%%%%%%%%%%%%%%%%%%%%%%%%%%%%%%%%%%
%% subsection 05.2.2 %%%%%%%%%%%%%%%%%%%%%%%%%%%%%%%%%%%%%
%%%%%%%%%%%%%%%%%%%%%%%%%%%%%%%%%%%%%%%%%%%%%%%%%%%%%%%%%%
\subsection{送り速さ値および主軸回転数の変更}
各々の工程における工具の送り速さ値および主軸回転数は、加工状況等に応じて必要があれば適時変更を行う。
送り速さ値または主軸回転数が変更が生じた場合は、速やかに現在の送り速さ値および主軸回転数の情報を更新する。


%%%%%%%%%%%%%%%%%%%%%%%%%%%%%%%%%%%%%%%%%%%%%%%%%%%%%%%%%%
%% Chapter 03 %%%%%%%%%%%%%%%%%%%%%%%%%%%%%%%%%%%%%%%%%%%%
%%%%%%%%%%%%%%%%%%%%%%%%%%%%%%%%%%%%%%%%%%%%%%%%%%%%%%%%%%
\modHeadchapter{著作物およびその公表}
%!TEX root = ../RPA_for_Creating_Program_Note.tex


ここでは作成されたソフトウェア関連の\index{ちょさくぶつ@著作物}著作物の\index{ちょさくけん@著作権}著作権について述べる。


%%%%%%%%%%%%%%%%%%%%%%%%%%%%%%%%%%%%%%%%%%%%%%%%%%%%%%%%%%
%% section 20.1 %%%%%%%%%%%%%%%%%%%%%%%%%%%%%%%%%%%%%%%%%%
%%%%%%%%%%%%%%%%%%%%%%%%%%%%%%%%%%%%%%%%%%%%%%%%%%%%%%%%%%
\modHeadsection{関連する著作物\TBW}
マシニングセンタによるモールドの加工に対して作成された(ソフトウェア関連の)著作物として、主に以下のものが挙げられる。
\begin{enumerate}
\item 本書
\item 位置情報等の数値計算用プログラム
\item 使用スペーサ計算用プログラム
\item バンドルのプログラムを除いたNCプログラム
\item モールドのRDB
\item モールドの3次元CADモデリング用テンプレート
\item 内面テーパの3次元CADモデリング用テンプレート
\end{enumerate}



%%%%%%%%%%%%%%%%%%%%%%%%%%%%%%%%%%%%%%%%%%%%%%%%%%%%%%%%%%
%% section 20.2 %%%%%%%%%%%%%%%%%%%%%%%%%%%%%%%%%%%%%%%%%%
%%%%%%%%%%%%%%%%%%%%%%%%%%%%%%%%%%%%%%%%%%%%%%%%%%%%%%%%%%
\modHeadsection{関連する著作物の著作権者\TBW}


%%%%%%%%%%%%%%%%%%%%%%%%%%%%%%%%%%%%%%%%%%%%%%%%%%%%%%%%%%
%% subsection 20.2.1 %%%%%%%%%%%%%%%%%%%%%%%%%%%%%%%%%%%%%
%%%%%%%%%%%%%%%%%%%%%%%%%%%%%%%%%%%%%%%%%%%%%%%%%%%%%%%%%%
\subsection{著作人格権\TBW}
上記のすべての関連著作物の著作人格権は、


%%%%%%%%%%%%%%%%%%%%%%%%%%%%%%%%%%%%%%%%%%%%%%%%%%%%%%%%%%
%% subsection 20.2.1 %%%%%%%%%%%%%%%%%%%%%%%%%%%%%%%%%%%%%
%%%%%%%%%%%%%%%%%%%%%%%%%%%%%%%%%%%%%%%%%%%%%%%%%%%%%%%%%%
\subsection{著作財産権\TBW}



\begin{appendices}
%%%%%%%%%%%%%%%%%%%%%%%%%%%%%%%%%%%%%%%%%%%%%%%%%%%%%%%%%
%% Appendix   %%%%%%%%%%%%%%%%%%%%%%%%%%%%%%%%%%%%%%%%%%%
%% Part Rules %%%%%%%%%%%%%%%%%%%%%%%%%%%%%%%%%%%%%%%%%%%
%% Start      %%%%%%%%%%%%%%%%%%%%%%%%%%%%%%%%%%%%%%%%%%%
%%%%%%%%%%%%%%%%%%%%%%%%%%%%%%%%%%%%%%%%%%%%%%%%%%%%%%%%%
%\Appendixpart

\end{appendices}
%%%%%%%%%%%%%%%%%%%%%%%%%%%%%%%%%%%%%%%%%%%%%%%%%%%%%%%%%
%% Appendix   %%%%%%%%%%%%%%%%%%%%%%%%%%%%%%%%%%%%%%%%%%%
%% Part Rules %%%%%%%%%%%%%%%%%%%%%%%%%%%%%%%%%%%%%%%%%%%
%% End        %%%%%%%%%%%%%%%%%%%%%%%%%%%%%%%%%%%%%%%%%%%
%%%%%%%%%%%%%%%%%%%%%%%%%%%%%%%%%%%%%%%%%%%%%%%%%%%%%%%%%
\addtocontents{toc}{\protect\end{tocBox}}
\clearrightpage

\addtocontents{toc}{\protect\end{tcolorbox}}
\addtocontents{toc}{\protect\cleardoublepage}% add page break




%%%%%%%%%%%%%%%%%%%%%%%%%%%%%%%%%%%%%%%%%%%%%%%%%%%%%%%%%
%%                    %%%%%%%%%%%%%%%%%%%%%%%%%%%%%%%%%%%
%%                    %%%%%%%%%%%%%%%%%%%%%%%%%%%%%%%%%%%
%% Part Work Flow     %%%%%%%%%%%%%%%%%%%%%%%%%%%%%%%%%%%
%%                    %%%%%%%%%%%%%%%%%%%%%%%%%%%%%%%%%%%
%%                    %%%%%%%%%%%%%%%%%%%%%%%%%%%%%%%%%%%
%%%%%%%%%%%%%%%%%%%%%%%%%%%%%%%%%%%%%%%%%%%%%%%%%%%%%%%%%
\addtocontents{toc}{\protect\begin{tcolorbox}[parttocstyle={Contents}{\the\numexpr\value{part}+1\relax}]}
\tPart{要件定義}{概要}{%
\paragraph*{目標(なにがしたいか?)}
(to be written...)

\tcbline*
\paragraph*{手段(どうやって?)}
(to be written...)

\tcbline*
\paragraph*{背景(なぜ?)}
(to be written...)\\
外注で作成したプログラムも、当社が具体的な要件定義すら行えず、実用に至っていない
%% footnote %%%%%%%%%%%%%%%%%%%%%
\footnote{プログラム自体は尤もな内容であり、レベルも十分なものである。
(第二製造室に関わらず)当社のソフトウェアに関する致命的なほどの無関心および能力の無さが顕わに露呈したものであり、自明な帰結である。}。
%%%%%%%%%%%%%%%%%%%%%%%%%%%%%%%%%

 したがって、システム開発プロセスの計画の策定を行うことが喫緊の課題である。

\tcbline*
\paragraph*{結論(どうなった?)}
\MMname における業務の流れを把握し、それを基に\DMname の稼働に向けたソフトウェア視点によるシステム開発プロセスの計画を策定した。
}
%!TEX root = ./RPA_for_Creating_Program_Note.tex


\addtocontents{toc}{\protect\cleardoublepage}
%%%%%%%%%%%%%%%%%%%%%%%%%%%%%%%%%%%%%%%%%%%%%%%%%%%%%%%%%
%% Part Requirement Definition %%%%%%%%%%%%%%%%%%%%%%%%%%
%%%%%%%%%%%%%%%%%%%%%%%%%%%%%%%%%%%%%%%%%%%%%%%%%%%%%%%%%
\addtocontents{toc}{\protect\begin{tocBox}{\tmppartnum}}%
\tPart{要件定義\TBW}{概要}{%
\paragraph*{目標(なにがしたいか?)}
(to be written...)

\tcbline*
\paragraph*{手段(どうやって?)}
(to be written...)

\tcbline*
\paragraph*{背景(なぜ?)}
(to be written...)\\
外注で作成した\index{プログラム(がいちゅう)@プログラム(外注)}プログラムも、当社が具体的な\index{ようけんていぎ@要件定義}要件定義すら行えず、実用に至っていない
%% footnote %%%%%%%%%%%%%%%%%%%%%
\footnote{プログラム自体は尤もな内容であり、レベルも十分なものである。
(第二製造室に関わらず)当社の\index{ソフトウェアエンジニアリング}ソフトウェアエンジニアリングに関する致命的なほどの無関心が顕わに露呈したものであり、自明な帰結である。}。
%%%%%%%%%%%%%%%%%%%%%%%%%%%%%%%%%

 したがって、\index{システムかいはつプロセス@システム開発プロセス}システム開発プロセスの計画の策定を行うことが喫緊の課題である。

\tcbline*
\paragraph*{結論(どうなった?)}
\MMname における業務の流れを把握し、それを基に\DMname の稼働に向けたソフトウェア視点によるシステム開発プロセスの計画を策定した。
}

%%%%%%%%%%%%%%%%%%%%%%%%%%%%%%%%%%%%%%%%%%%%%%%%%%%%%%%%%
%% chapters %%%%%%%%%%%%%%%%%%%%%%%%%%%%%%%%%%%%%%%%%%%%%%
%%%%%%%%%%%%%%%%%%%%%%%%%%%%%%%%%%%%%%%%%%%%%%%%%%%%%%%%%%
%!TEX root = ../RfCPN.tex


\modHeadchapter{目的・目標の明確化}
新たに導入するマシニングセンタで何を達成したいのか、その目的さえも明確にされていないのが現状である
%% footnote %%%%%%%%%%%%%%%%%%%%%
\footnote{\MMC を蔑ろに扱っている現状にもかかわらず、そもそもどのような論理で導入にまで至ったのか、常識的に考えて不思議でならない。}。
%%%%%%%%%%%%%%%%%%%%%%%%%%%%%%%%%
したがって、ここでは暫定的に目的を定め、具体的な目標の設定を行うことにする。



%%%%%%%%%%%%%%%%%%%%%%%%%%%%%%%%%%%%%%%%%%%%%%%%%%%%%%%%%%
%% section 04.01 %%%%%%%%%%%%%%%%%%%%%%%%%%%%%%%%%%%%%%%%%
%%%%%%%%%%%%%%%%%%%%%%%%%%%%%%%%%%%%%%%%%%%%%%%%%%%%%%%%%%
\modHeadsection{機械導入の目的}
(暫定的な)目的として、以下を採用する。
\begin{enumerate}[label=\sarrow]
\item 作業効率(\MMC 比)の大幅向上
\item 教育コスト(\MMC 比)の大幅削減
\item \Dimple 加工の内製化
\end{enumerate}



%%%%%%%%%%%%%%%%%%%%%%%%%%%%%%%%%%%%%%%%%%%%%%%%%%%%%%%%%%
%% section 04.02 %%%%%%%%%%%%%%%%%%%%%%%%%%%%%%%%%%%%%%%%%
%%%%%%%%%%%%%%%%%%%%%%%%%%%%%%%%%%%%%%%%%%%%%%%%%%%%%%%%%%
\modHeadsection{達成したい目標}
達成したい目標として主に以下が挙げられる。
\begin{enumerate}[label=\sarrow]
\item 諸規定の策定
\item 諸規定に則った、諸標準の策定
\item 諸標準に則った、各明細・各工程に対する\index{NCプログラム}NCプログラムの作成
\item 諸作業に対する属人性の大幅削減
\end{enumerate}
なお、属人性の削減については、以下のような方針をとるものとする。
\begin{enumerate}[label=\sarrow]
\item \textgt{加工の自動化}:手作業による測定・加工の削減・簡易化
\item \textgt{操作の自動化}:手作業によるマシニングセンタ画面操作の削減・簡易化
\item \textgt{書類生成の自動化}:手作業による書類記入の削減・簡易化
\item \textgt{コード生成の自動化}:手作業による\index{NCメインプログラム}NC(メイン)プログラム作成の簡易化
\end{enumerate}



\clearpage
%%%%%%%%%%%%%%%%%%%%%%%%%%%%%%%%%%%%%%%%%%%%%%%%%%%%%%%%%%
%% section 04.03 %%%%%%%%%%%%%%%%%%%%%%%%%%%%%%%%%%%%%%%%%
%%%%%%%%%%%%%%%%%%%%%%%%%%%%%%%%%%%%%%%%%%%%%%%%%%%%%%%%%%
\modHeadsection{諸作業の目標}


%%%%%%%%%%%%%%%%%%%%%%%%%%%%%%%%%%%%%%%%%%%%%%%%%%%%%%%%%%
%% subsection 04.03.01 %%%%%%%%%%%%%%%%%%%%%%%%%%%%%%%%%%%
%%%%%%%%%%%%%%%%%%%%%%%%%%%%%%%%%%%%%%%%%%%%%%%%%%%%%%%%%%
\subsection{\index{ワークのせっち@ワークの設置}ワークの設置における目標}
\begin{enumerate}
\item \Spacer による振分調整作業の廃止
\item ワーク\FixtureBolt の規格化
\end{enumerate}


%%%%%%%%%%%%%%%%%%%%%%%%%%%%%%%%%%%%%%%%%%%%%%%%%%%%%%%%%%
%% subsection 04.03.02 %%%%%%%%%%%%%%%%%%%%%%%%%%%%%%%%%%%
%%%%%%%%%%%%%%%%%%%%%%%%%%%%%%%%%%%%%%%%%%%%%%%%%%%%%%%%%%
\subsection{測定(原点設定・\CenterlineEndFaceDif)における目標}
\begin{enumerate}
\item \index{ワークざひょうげんてん@ワーク座標原点}ワーク座標原点概算値導出作業の廃止(解析的導出)
\item 測定箇所変更時の数値変更作業の自動化
\item AC方向\KeywayCenter 座標計算作業の廃止・自動化
\item \CenterlineEndFaceDifMeasurement 作業の廃止・自動化
\end{enumerate}


%%%%%%%%%%%%%%%%%%%%%%%%%%%%%%%%%%%%%%%%%%%%%%%%%%%%%%%%%%
%% subsection 04.03.03 %%%%%%%%%%%%%%%%%%%%%%%%%%%%%%%%%%%
%%%%%%%%%%%%%%%%%%%%%%%%%%%%%%%%%%%%%%%%%%%%%%%%%%%%%%%%%%
\subsection{\DimpleMeasurement における目標}
\begin{enumerate}
\item 短時間による\DimpleMeasurement(概ね6s/個 程度)
\end{enumerate}


%%%%%%%%%%%%%%%%%%%%%%%%%%%%%%%%%%%%%%%%%%%%%%%%%%%%%%%%%%
%% subsection 04.03.04 %%%%%%%%%%%%%%%%%%%%%%%%%%%%%%%%%%%
%%%%%%%%%%%%%%%%%%%%%%%%%%%%%%%%%%%%%%%%%%%%%%%%%%%%%%%%%%
\subsection{\DimpleMilling における目標\TBW}
(to be written...)


%%%%%%%%%%%%%%%%%%%%%%%%%%%%%%%%%%%%%%%%%%%%%%%%%%%%%%%%%%
%% subsection 04.03.05 %%%%%%%%%%%%%%%%%%%%%%%%%%%%%%%%%%%
%%%%%%%%%%%%%%%%%%%%%%%%%%%%%%%%%%%%%%%%%%%%%%%%%%%%%%%%%%
\subsection{\EndFacecutMilling における目標}
\begin{enumerate}
\item 加工回数変更作業の簡易化(半自動化)
\item \TDCValue 変更作業の廃止
\end{enumerate}


%%%%%%%%%%%%%%%%%%%%%%%%%%%%%%%%%%%%%%%%%%%%%%%%%%%%%%%%%%
%% subsection 04.03.06 %%%%%%%%%%%%%%%%%%%%%%%%%%%%%%%%%%%
%%%%%%%%%%%%%%%%%%%%%%%%%%%%%%%%%%%%%%%%%%%%%%%%%%%%%%%%%%
\subsection{\OutcutMilling における目標}
\begin{enumerate}
\item \CurvedOutcut 用測定作業の廃止
\item \CurvedOutcutMilling の自動化
\end{enumerate}


%%%%%%%%%%%%%%%%%%%%%%%%%%%%%%%%%%%%%%%%%%%%%%%%%%%%%%%%%%
%% subsection 04.03.07 %%%%%%%%%%%%%%%%%%%%%%%%%%%%%%%%%%%
%%%%%%%%%%%%%%%%%%%%%%%%%%%%%%%%%%%%%%%%%%%%%%%%%%%%%%%%%%
\subsection{\KeywayMilling における目標}
\begin{enumerate}
\item \KeywayPos・\KeywayWidth 調整作業の簡易化
\item $Z$方向加工回数変更作業の廃止・自動化
\end{enumerate}


%%%%%%%%%%%%%%%%%%%%%%%%%%%%%%%%%%%%%%%%%%%%%%%%%%%%%%%%%%
%% subsection 04.03.08 %%%%%%%%%%%%%%%%%%%%%%%%%%%%%%%%%%%
%%%%%%%%%%%%%%%%%%%%%%%%%%%%%%%%%%%%%%%%%%%%%%%%%%%%%%%%%%
\subsection{\EndFaceChamferMilling における目標}
\begin{enumerate}
\item 手作業による\EndFaceCChamferMilling の廃止・自動化
\item 手作業による\EndFaceRChamferMilling の簡易化・半自動化
\end{enumerate}


%%%%%%%%%%%%%%%%%%%%%%%%%%%%%%%%%%%%%%%%%%%%%%%%%%%%%%%%%%
%% subsection 04.03.09 %%%%%%%%%%%%%%%%%%%%%%%%%%%%%%%%%%%
%%%%%%%%%%%%%%%%%%%%%%%%%%%%%%%%%%%%%%%%%%%%%%%%%%%%%%%%%%
\subsection{\EndFaceBoringMilling における目標}
\begin{enumerate}
\item \EndFaceBoringWidth の計算間違いの訂正
\end{enumerate}


%%%%%%%%%%%%%%%%%%%%%%%%%%%%%%%%%%%%%%%%%%%%%%%%%%%%%%%%%%
%% subsection 04.03.10 %%%%%%%%%%%%%%%%%%%%%%%%%%%%%%%%%%%
%%%%%%%%%%%%%%%%%%%%%%%%%%%%%%%%%%%%%%%%%%%%%%%%%%%%%%%%%%
\subsection{\IncutBoringMilling における目標\TBW}
(to be written...)



\clearpage
%%%%%%%%%%%%%%%%%%%%%%%%%%%%%%%%%%%%%%%%%%%%%%%%%%%%%%%%%%
%% section 04.02 %%%%%%%%%%%%%%%%%%%%%%%%%%%%%%%%%%%%%%%%%
%%%%%%%%%%%%%%%%%%%%%%%%%%%%%%%%%%%%%%%%%%%%%%%%%%%%%%%%%%
\modHeadsection{目標の優先順位\TBW}
(to be written...)


\clearrightpage
%%%%%%%%%%%%%%%%%%%%%%%%%%%%%%%%%%%%%%%%%%%%%%%%%%%%%%%%%
%% Appendices %%%%%%%%%%%%%%%%%%%%%%%%%%%%%%%%%%%%%%%%%%%
%%%%%%%%%%%%%%%%%%%%%%%%%%%%%%%%%%%%%%%%%%%%%%%%%%%%%%%%%
\begin{appendices}
%\Appendixpart
\end{appendices}

\addtocontents{toc}{\protect\end{tocBox}}
\clearrightpage

\addtocontents{toc}{\protect\end{tcolorbox}}
\addtocontents{toc}{\protect\cleardoublepage}% add page break





%%%%%%%%%%%%%%%%%%%%%%%%%%%%%%%%%%%%%%%%%%%%%%%%%%%%%%%%%
%%                %%%%%%%%%%%%%%%%%%%%%%%%%%%%%%%%%%%%%%%
%%                %%%%%%%%%%%%%%%%%%%%%%%%%%%%%%%%%%%%%%%
%% Part Standards %%%%%%%%%%%%%%%%%%%%%%%%%%%%%%%%%%%%%%%
%%                %%%%%%%%%%%%%%%%%%%%%%%%%%%%%%%%%%%%%%%
%%                %%%%%%%%%%%%%%%%%%%%%%%%%%%%%%%%%%%%%%%
%%%%%%%%%%%%%%%%%%%%%%%%%%%%%%%%%%%%%%%%%%%%%%%%%%%%%%%%%
\addtocontents{toc}{\protect\begin{tcolorbox}[parttocstyle={Contents}{\the\numexpr\value{part}+1\relax}]}
\tPart[lot,loC]{諸標準の策定}{概要}{%
\paragraph*{目標(なにがしたいか?)}
プログラムの記述に関して一定の基準・規則を設けることで、一貫性および効率性を向上する
\tcbline*
\paragraph*{手段(どうやって?)}
\DMname についての(特にソフトウェアの観点による)\textbf{諸標準の策定}を試みる。
\tcbline*
\paragraph*{背景(なぜ?)}
プログラムの記述には、一般にG-codeが用いられる。
一方で、G-codeの記述に関してはその規格が多岐にわたり統一された標準が存在しない。
そのため、記述に際して社内の標準を参照する必要がある。

しかし、現時点(\DMname 設置時点)において、マシニングに関連した(ソフトウェア視点による)\textbf{社内の標準はほとんど存在しない}
%% footnote %%%%%%%%%%%%%%%%%%%%%
\footnote{つまり、マシニングについては長年にわたり管理業務が事実上放棄されている。
またその皺寄せの大部分が、作業者(一般職)に押し付けられている。}。
%%%%%%%%%%%%%%%%%%%%%%%%%%%%%%%%%

 したがって、プログラムの記述に関する標準の策定が急務である。
これにより、プログラムの記述における一貫性と効率性が向上し、業務の効率化(外注を含む)とコスト削減が期待される。
また標準を設けることにより、G-codeの記述に関する混乱を解消し、より高品質なソフトウェア開発を実現するための重要なステップとなる。
\tcbline*
\paragraph*{結論(どうなった?)}
\DMname 関連のソフトウェア開発における、(ソフトウェアの観点による)寸法・コードの記述法・工具・保守等の標準を策定した。
}
\input{./RfCPN_part/RfCPN_p4_Standards_Coding}
\addtocontents{toc}{\protect\end{tcolorbox}}
\addtocontents{toc}{\protect\cleardoublepage}% add page break




\PartSeparateline{lot}%
%%%%%%%%%%%%%%%%%%%%%%%%%%%%%%%%%%%%%%%%%%%%%%%%%%%%%%%%%
%%          %%%%%%%%%%%%%%%%%%%%%%%%%%%%%%%%%%%%%%%%%%%%%
%%          %%%%%%%%%%%%%%%%%%%%%%%%%%%%%%%%%%%%%%%%%%%%%
%% Part AC  %%%%%%%%%%%%%%%%%%%%%%%%%%%%%%%%%%%%%%%%%%%%%
%%          %%%%%%%%%%%%%%%%%%%%%%%%%%%%%%%%%%%%%%%%%%%%%
%%          %%%%%%%%%%%%%%%%%%%%%%%%%%%%%%%%%%%%%%%%%%%%%
%%%%%%%%%%%%%%%%%%%%%%%%%%%%%%%%%%%%%%%%%%%%%%%%%%%%%%%%%
\addtocontents{toc}{\protect\begin{tcolorbox}[parttocstyle={Contents}{\the\numexpr\value{part}+1\relax}]}
\tPart[lof,lot,loC]{幾何学性質の解析計算}{概要}{%
\paragraph*{目標(なにがしたいか?)}
マシニングにおけるモールドの加工に必要な幾何学的情報を抽象化(一般化)し、体系を構築する。
\tcbline*
\paragraph*{手段(どうやって?)}
マシニング内におけるモールドや工具等の幾何学的性質を抽象化(一般化)し、加工の際に必要となる位置・距離等の\textbf{解析的な導出}を試みる。
\tcbline*
\paragraph*{背景(なぜ?)}
ソフトウェアの観点から見た場合、高精度な加工を実現にはマシニング内におけるモールドや工具の\textbf{幾何学的性質}を正確に把握することが不可欠であることは言うまでもない。
マシニングセンタの操作・指示はこうした特性の理解が前提であり、いわば``常識"と言っても過言ではない
%% footnote %%%%%%%%%%%%%%%%%%%%%
\footnote{さらに、\index{CAD}CAD, \index{CAM}CAMを利用するというのであれば、これに加えてCADによる\index{3Dモデリング}3Dモデリングの技術、およびそれに伴うCAMの操作・設定技術も習熟しなければならず、教育コストが大きくかかることは容易に推測される。}。
%%%%%%%%%%%%%%%%%%%%%%%%%%%%%%%%%

 モールドの場合、必要な幾何学的性質のほとんどが直線または円で記述されるため、その体系化は(複雑ではあるが)難しいものではない。
しかし現時点(\DMname 設置時点)において、こうした性質の\textbf{体系化はほぼ全くなされていない}。
実際、\textbf{明細ごとに対応}をしている状態にある。

 したがって、(少なくとも実際の加工に関わる程度の)幾何学的性質の把握・体系化は急務である。
またこれにより品質や生産性の低下が大きく抑えられることが期待される。
\tcbline*
\paragraph*{結論(どうなった?)}
\DMname のおける加工の際に必要なすべての幾何的情報を解析的に導出し、体系化した。
}
%!TEX root = ./RPA_for_Creating_Program_Note.tex


\addtocontents{toc}{\protect\cleardoublepage}
\addtocontents{lot}{\protect\tcbline*}
\addtocontents{loC}{\protect\tcbline*}
%%%%%%%%%%%%%%%%%%%%%%%%%%%%%%%%%%%%%%%%%%%%%%%%%%%%%%%%%
%%                             %%%%%%%%%%%%%%%%%%%%%%%%%%
%% Part Analytical Calculation %%%%%%%%%%%%%%%%%%%%%%%%%%
%%                             %%%%%%%%%%%%%%%%%%%%%%%%%%
%%%%%%%%%%%%%%%%%%%%%%%%%%%%%%%%%%%%%%%%%%%%%%%%%%%%%%%%%
\addtocontents{toc}{\protect\begin{tocBox}{\tmppartnum}}%
\tPart[lot,lof,loC]{幾何学性質の解析計算}{概要}{%
\paragraph*{目標(なにがしたいか?)}
マシニングにおける\index{モールド}モールドの加工に必要な\index{きかがくてきせいしつ(ワーク)@幾何学的性質(ワーク)}幾何学的情報を抽象化(一般化)し、体系を構築する。
\tcbline*
\paragraph*{手段(どうやって?)}
マシニング内におけるモールドや工具・\index{ジグ}ジグの位置・距離等を抽象化(一般化)し、加工の際に必要となる\textbf{幾何学的性質の解析的な導出}を試みる。
\tcbline*
\paragraph*{背景(なぜ?)}
ソフトウェアの観点から見た場合、高精度な加工の実現にはマシニング内におけるワークや工具の\textbf{幾何学的性質の正確な把握}が不可欠である。
マシニングセンタの操作・指示はこうした性質の理解が前提であり、いわば``常識"と言っても過言ではない
%% footnote %%%%%%%%%%%%%%%%%%%%%
\footnote{さらに、\index{CAD}CAD, \index{CAM}CAMを利用するというのであれば、これに加えてCADによる\index{3Dモデリング}3Dモデリングの技術、およびそれに伴うCAMの操作・設定技術も習熟しなければならず、\index{きょういくコスト@教育コスト}教育コストが大きくかかることは容易に推測される。}。
%%%%%%%%%%%%%%%%%%%%%%%%%%%%%%%%%

 モールドの場合、必要な幾何学的性質のほとんどが直線または円で記述されるため、その体系化は(複雑ではあるが)難しいものではない。
しかし現時点(\DMname 設置時点)において、こうした性質の\textbf{体系化はほぼ全くなされていない}。
実際、\textbf{明細ごとに対応}をしている状態にある。

 したがって、(少なくとも実際の加工に関わる程度の)幾何学的性質の把握・体系化は急務である。
また、これにより品質や生産性の低下が大きく抑えられることも期待される。
\tcbline*
\paragraph*{結論(どうなった?)}
\DMname のおける加工の際に必要なほぼすべての幾何的情報を解析的に導出し、体系化した。
}




%%%%%%%%%%%%%%%%%%%%%%%%%%%%%%%%%%%%%%%%%%%%%%%%%%%%%%%%%%
%%           %%%%%%%%%%%%%%%%%%%%%%%%%%%%%%%%%%%%%%%%%%%%%
%% chapter 1 %%%%%%%%%%%%%%%%%%%%%%%%%%%%%%%%%%%%%%%%%%%%%
%%           %%%%%%%%%%%%%%%%%%%%%%%%%%%%%%%%%%%%%%%%%%%%%
%%%%%%%%%%%%%%%%%%%%%%%%%%%%%%%%%%%%%%%%%%%%%%%%%%%%%%%%%%
\modHeadchapter[loC,lof]{全長および振分けの幾何}
%!TEX root = ../RPA_for_Creating_Program_Note.tex



\index{ふりわけ@振分け}振分けの長さ(\index{ふりわけちょう@振分長}\textbf{振分長})は、トップ側とボトム側では一般に異なる。
しかし、加工をする際には\index{ジグのちゅうしん@ジグの中心}ジグの中心に対して両者の長さの差が小さいほうが一般的には好都合である。
そうした場合の対処法として、ここでは以下のような2つの方法を考える。
\begin{enumerate}
\item
適当な厚さの\index{スペーサ}\textbf{スペーサ}を\index{ワークとジグのせってん@ワークとジグの接点}ワークとジグの接点に取り付けることで、双方の振分長を調節
\item
適当な角度に\index{テーブル}テーブルを回転することで、双方の振分長を調節
\end{enumerate}
このとき、\index{ワーク}ワークがどのように移動するかを考える。

基本的な考え方として、\index{ちゅうしんわんきょく@中心湾曲}\textbf{中心湾曲半径}$R_\mathrm c$の円の中心を原点として$\Omega$だけ回転し、次にワークとの(スペーサを装着していない側の)接点を中心に$-\theta$だけ回転したと考えることができる。
なお、ここでは話の簡単化のため、もとの振分けではトップ側よりボトム側の振分長のほうが長いものとする。




%%%%%%%%%%%%%%%%%%%%%%%%%%%%%%%%%%%%%%%%%%%%%%%%%%%%%%%%%%
%% section 1.1 %%%%%%%%%%%%%%%%%%%%%%%%%%%%%%%%%%%%%%%%%%%
%%%%%%%%%%%%%%%%%%%%%%%%%%%%%%%%%%%%%%%%%%%%%%%%%%%%%%%%%%
\modHeadsection{ジグの接点部が点の場合}
まずは簡単のため、\index{ジグ}ジグの\index{ワーク}ワークとの接点部(\pageautoref{fig:mouldOnComplexPlane1}のU$_\mathrm T$, U$_\mathrm B$の部分)は点であるとして考える。
%%%%%%%%%%%%%%%%%%%%%%%%%%%%%%%%%%%%%%%%%%%%%%%%%%%%%%%%%%
%% figure %%%%%%%%%%%%%%%%%%%%%%%%%%%%%%%%%%%%%%%%%%%%%%%%
%%%%%%%%%%%%%%%%%%%%%%%%%%%%%%%%%%%%%%%%%%%%%%%%%%%%%%%%%%
\begin{figure}[p]
\centering%
\begin{Figlandscape}
\captionsetup{width=.75\textheight}
\begin{adjustbox}{%
  addcode={\begin{minipage}{\textheight}\centering}{%
    \captionof{figure}[湾曲中心Oを原点とした複素平面上のモールド]{%
      \index{わんきょくちゅうしん@湾曲中心}湾曲中心Oを\index{げんてんO@原点O}原点とした\index{ふくそへいめん@複素平面}複素平面上のモールド\newline
      T$_\mathrm o$, T$_\mathrm i$, B$_\mathrm o$, B$_\mathrm i$, U$_\mathrm T$, U$_\mathrm B$は点、%
      $R_\mathrm c$, $R_\mathrm o$, $R_\mathrm i$, $f_\mathrm T$, $f_\mathrm B$, $l$は長さ、%
      $\alpha_{\mathrm T_\mathrm o}$, $\alpha_{\mathrm T_\mathrm i}$, $\alpha_{\mathrm U_\mathrm B}$は角度を示す。%
      \label{fig:mouldOnComplexPlane1}%
    }%
    \end{minipage}%
  },
  rotate=90,
  max width=\textwidth,
  max height=\textheight,
  keepaspectratio}
\includegraphics{RfCPN_p5_pictures/mouldoverall.pdf}
\end{adjustbox}
\end{Figlandscape}%
\end{figure}
%%%%%%%%%%%%%%%%%%%%%%%%%%%%%%%%%%%%%%%%%%%%%%%%%%%%%%%%%%
%%%%%%%%%%%%%%%%%%%%%%%%%%%%%%%%%%%%%%%%%%%%%%%%%%%%%%%%%%
%%%%%%%%%%%%%%%%%%%%%%%%%%%%%%%%%%%%%%%%%%%%%%%%%%%%%%%%%%



%%%%%%%%%%%%%%%%%%%%%%%%%%%%%%%%%%%%%%%%%%%%%%%%%%%%%%%%%%
%% subsection 1.1.1 %%%%%%%%%%%%%%%%%%%%%%%%%%%%%%%%%%%%%%
%%%%%%%%%%%%%%%%%%%%%%%%%%%%%%%%%%%%%%%%%%%%%%%%%%%%%%%%%%
\subsection{スペーサを用いた再振分け}
\index{モールド}モールドの湾曲における円の中心\index{O(わんきょくちゅうしん)@O(湾曲中心)}Oを\index{げんてんO@原点O}原点とした\index{ふくそへいめん@複素平面}複素平面を考える
%% footnote %%%%%%%%%%%%%%%%%%%%%
\footnote{ここでは$0 < R_\mathrm c < \infty$ ($0 < \nicefrac1{R_\mathrm c} < \infty$)としている。
$R_\mathrm c \to \infty$ ($\nicefrac1{R_\mathrm c} \to 0$)の場合、すなわち\index{わんきょくのないモールド@湾曲のないモールド}湾曲のないまっすぐなモールドの場合は、別途考える必要がある。}。
%%%%%%%%%%%%%%%%%%%%%%%%%%%%%%%%%
このとき、\pageautoref{fig:mouldOnComplexPlane1}のように、$R_\mathrm c$, $R_\mathrm i$, $R_\mathrm o$, $f_\mathrm T$, $f_\mathrm B$, $l$, $\alpha_{\mathrm T_\mathrm i}$, $\alpha_{\mathrm T_\mathrm o}$, $\alpha_{\mathrm U_\mathrm B}$をとると、
\begin{subequations}
%% label{eq:constraintUpoint1}
%% label{eq:constraintUpoint2}
\begin{gather}
  \label{eq:constraintUpoint1}
  R_\mathrm o - R_\mathrm c = R_\mathrm c - R_\mathrm i = \frac{W_x}2~, \qquad
  \IP\left(R_\mathrm oe^{i\alpha_{\mathrm T_\mathrm o}} - R_\mathrm ie^{i\alpha_{\mathrm T_\mathrm i}}\right)
  = 0~,\\
  \label{eq:constraintUpoint2}
  \sin\alpha_{\mathrm T_\mathrm i} = \frac{f_\mathrm T}{R_\mathrm i}, \qquad
  \sin\alpha_{\mathrm U_\mathrm B} = \frac l{R_\mathrm i}, \qquad
  \tan\theta = \frac{\delta_\mathrm s}{2l}~.
\end{gather}
\end{subequations}
ここで$W_x$はモールドの(AC)外径、$\delta_\mathrm s$は\index{スペーサあつ@スペーサ厚}スペーサの厚さ(\textbf{スペーサ厚})である。
このときモールドを原点Oを中心に$\Omega$だけ回転し、さらに点U$_\mathrm B$($R_\mathrm i$, $-\alpha_{\mathrm U_\mathrm B}$)を中心に$-\theta$だけ回転すると、点T$_\mathrm i$($R_\mathrm i$, $\alpha_{\mathrm T_\mathrm i}$)は、
%% label{eq:afterftUpoint}
\begin{align}
  \notag
  & e^{-i\theta}\left\{R_\mathrm ie^{i(\alpha_{\mathrm T_\mathrm i} + \Omega)} - R_\mathrm ie^{-i\alpha_{\mathrm U_\mathrm B}}\right\}
    +R_\mathrm ie^{-i\alpha_{\mathrm U_\mathrm B}}\\
  &= R_\mathrm i
     \left\{
       e^{i(\alpha_{\mathrm T_\mathrm i} + \Omega - \theta)} - e^{-i(\alpha_{\mathrm U_\mathrm B} + \theta)} + e^{-i\alpha_{\mathrm U_\mathrm B}}
     \right\}
  \label{eq:afterftUpoint}
\end{align}
に移動する。
また同様に点T$_\mathrm o$($R_\mathrm o$, $\alpha_{\mathrm T_\mathrm o}$)は
\begin{align*}
  \notag
  R_\mathrm oe^{i(\alpha_{\mathrm T_\mathrm o} + \Omega - \theta)}
  -R_\mathrm i\left\{e^{-i(\alpha_{\mathrm U_\mathrm B} + \theta)}-e^{-i\alpha_{\mathrm U_\mathrm B}}\right\}
\end{align*}
に移動する。
したがって、これらの差
\begin{align*}
  \notag
  e^{i(\Omega - \theta)}\left(R_\mathrm oe^{i\alpha_{\mathrm T_\mathrm o}} - R_\mathrm ie^{i\alpha_{\mathrm T_\mathrm i}}\right)
\end{align*}
の虚部が0であればよい。
つまり、\pageeqref{eq:constraintUpoint1}より、$\Omega = \theta$である
%% footnote %%%%%%%%%%%%%%%%%%%%%
\footnote{ここでは$0 \leqq \Omega, \theta < \nicefrac \pi2$としている。}。
%%%%%%%%%%%%%%%%%%%%%%%%%%%%%%%%%

スペーサを入れた後の(トップ側の)振分長は、\pageeqref{eq:afterftUpoint}の虚部を見ればよい。
\begin{align*}
  R_\mathrm i\left\{\sin\alpha_{\mathrm T_\mathrm i} + \sin(\alpha_{\mathrm U_\mathrm B} + \theta) - \sin\alpha_{\mathrm U_\mathrm B}\right\}
  &= f_\mathrm T -l
     +R_\mathrm i\left(\sin\alpha_{\mathrm U_\mathrm B}\cos\theta + \cos\alpha_{\mathrm U_\mathrm B}\sin\theta\right)\\
  &= f_\mathrm T -l+l\cdot\frac{2l}{\sqrt{4l^2+\delta_\mathrm s^2}}
     +\sqrt{R_\mathrm i^2-l^2}\cdot\frac{\delta_\mathrm s}{\sqrt{4l^2+\delta_\mathrm s^2}}\\
  &= f_\mathrm T -l+\frac{2l^2+\delta_\mathrm s\sqrt{R_\mathrm i^2-l^2}}{\sqrt{4l^2+\delta_\mathrm s^2}}~.
\end{align*}
まとめると、厚さ$\delta_\mathrm s$の\index{スペーサ}スペーサを入れた後のトップ側の\index{ふりわけちょう@振分長}振分長$f'_\mathrm T$は、
\begin{align*}
  f'_\mathrm T
  = f_\mathrm T -l
    +\frac{2l^2+\delta_\mathrm s\sqrt{\left(R_\mathrm c-\nicefrac{W_x}2\right)^2-l^2}}{\sqrt{4l^2+\delta_\mathrm s^2}}~.
\end{align*}



%%%%%%%%%%%%%%%%%%%%%%%%%%%%%%%%%%%%%%%%%%%%%%%%%%%%%%%%%%
%% subsection 1.1.2 %%%%%%%%%%%%%%%%%%%%%%%%%%%%%%%%%%%%%%
%%%%%%%%%%%%%%%%%%%%%%%%%%%%%%%%%%%%%%%%%%%%%%%%%%%%%%%%%%
\subsection{振分長が均等になるスペーサ厚}
%%%%%%%%%%%%%%%%%%%%%%%%%%%%%%%
トップ側とボトム側の\index{きんとうふりわけ@均等振分}振分長が同じになるとき、$\delta_\mathrm s$は
\begin{align*}
  f'_\mathrm T - f_\mathrm T = \frac{f_\mathrm B - f_\mathrm T}2
\end{align*}
を満たす。
これより、
\begin{align*}
  \frac{2l^2+\delta_\mathrm s\sqrt{R_\mathrm i^2-l^2}}{\sqrt{4l^2+\delta_\mathrm s^2}} = l'\qquad
  \left(l' \equiv l + \frac{f_\mathrm B-f_\mathrm T}2\right)
\end{align*}
両辺を2乗すると、
\begin{gather*}
  4l^4+\delta_\mathrm s^2\left(R_\mathrm i^2-l^2\right)+4l^2\delta_\mathrm s\sqrt{R_\mathrm i^2-l^2}
  = l'^2\left(4l^2+\delta_\mathrm s^2\right)\\
  \longrightarrow\quad
  \delta_\mathrm s^2\left(R_\mathrm i^2-l^2-l'^2\right)
  +4l^2\delta_\mathrm s\sqrt{R_\mathrm i^2-l^2} -4l^2\left(l'^2 - l^2\right)
  = 0.
\end{gather*}
$\delta_\mathrm s > 0$より、
\begin{align*}
  \delta_\mathrm s
  &= \frac{\sqrt{4l^4\left(R_\mathrm i^2-l^2\right)
                 +4l^2\left(R_\mathrm i^2-l^2-l'^2\right)\left(l'^2 - l^2\right)}
           -2l^2\sqrt{R_\mathrm i^2-l^2}}{R_\mathrm i^2-l^2-l'^2}\\
  &= 2l\cdot\frac{l'\sqrt{R_\mathrm i^2-l'^2}-l\sqrt{R_\mathrm i^2-l^2}}{R_\mathrm i^2-l^2-l'^2}
\end{align*}
まとめると、求める\index{スペーサあつ@スペーサ厚}スペーサ厚$\delta_\mathrm s$は、
\begin{align*}
  \delta_\mathrm s
  = 2l\cdot
    \frac{\displaystyle
          \left(l+\frac{f_\mathrm B-f_\mathrm T}2\right)
          \sqrt{\left(R_\mathrm c-\frac{W_x}2\right)^2
                -\left(l+\frac{f_\mathrm B-f_\mathrm T}2\right)^2}
          -l\sqrt{\left(R_\mathrm c-\frac{W_x}2\right)^2-l^2}}
         {\displaystyle
          \left(R_\mathrm c-\frac{W_x}2\right)^2-l^2
          -\left(l+\frac{f_\mathrm B-f_\mathrm T}2\right)^2}~.
\end{align*}



\clearpage
%%%%%%%%%%%%%%%%%%%%%%%%%%%%%%%%%%%%%%%%%%%%%%%%%%%%%%%%%%
%% section 20.2 %%%%%%%%%%%%%%%%%%%%%%%%%%%%%%%%%%%%%%%%%%
%%%%%%%%%%%%%%%%%%%%%%%%%%%%%%%%%%%%%%%%%%%%%%%%%%%%%%%%%%
\modHeadsection{受板がある場合}
\index{ジグ}ジグの\index{ワーク}ワークと接する部品(\index{うけいた@受板}\textbf{受板})の大きさを考慮した場合を考える。
ワークに接する側の面は半径$\rho$の円弧(\index{うけいたのけい@受板の径}受板の径), 虚軸方向の厚み(\index{うけいたのはば@受板の幅}受板の幅)は$\sigma$とする。
また受板の虚軸負方向側の面は、ジグのそれと同じ平面上にあるものとする。
%%%%%%%%%%%%%%%%%%%%%%%%%%%%%%%%%%%%%%%%%%%%%%%%%%%%%%%%%%
%% figure %%%%%%%%%%%%%%%%%%%%%%%%%%%%%%%%%%%%%%%%%%%%%%%%
%%%%%%%%%%%%%%%%%%%%%%%%%%%%%%%%%%%%%%%%%%%%%%%%%%%%%%%%%%
\begin{figure}[p]%
\begin{Figbox}[valign=top]%
\resizebox{\linewidth-35pt}{!}{\includegraphics{RfCPN_p5_pictures/mouldUkeita.pdf}}%
\vfill~
\captionof{figure}[受板がある場合]{%
 \index{うけいた@受板}受板がある場合\newline
 $\rho$, $\sigma$は\index{うけいたのけい@受板の径}受板の径および\index{うけいたのはば@受板の幅}受板の幅を示し、U$_\mathrm T$, U$_\mathrm B$はそれぞれの受板の円弧の中心点を示す。
 受板があると\index{ワーク}ワークとの\index{せってん(ワークとうけいた)@接点(ワークと受板)}接点U$_\mathrm B$の位置も変化する。
 なお受板は、トップ側とボトム側で同じものであり、その片面はジグの片面と揃える形で装着されるものとする。
 \label{fig:mouldwithukeita}}
\end{Figbox}%
\end{figure}%
%%%%%%%%%%%%%%%%%%%%%%%%%%%%%%%%%%%%%%%%%%%%%%%%%%%%%%%%%%
%%%%%%%%%%%%%%%%%%%%%%%%%%%%%%%%%%%%%%%%%%%%%%%%%%%%%%%%%%
%%%%%%%%%%%%%%%%%%%%%%%%%%%%%%%%%%%%%%%%%%%%%%%%%%%%%%%%%%
\index{うけいたのちゅうしん@受板の中心}受板の径の中心を改めてU$_\mathrm B$とし、また\index{げんてんO@原点O}原点Oに対する\index{へんかく(うけいた)@偏角(受板)}偏角を改めて$-\alpha_{\mathrm U_\mathrm B}$とすると、これはU$_\mathrm B$($R_\mathrm i-\rho$, $-\alpha_{\mathrm U_\mathrm B}$)と表すことができる。
ただし、\pageeqref{eq:constraintUpoint2}は以下のようになる。
\begin{align*}
  \sin\alpha_{\mathrm U_\mathrm B} = \frac{\bar l}{R_\mathrm i-\rho}\quad, \quad
  \tan\psi = \frac{\delta_\mathrm s}{2\bar l} \quad
  \left(~\bar l \equiv l-\frac\sigma2~\right).
\end{align*}
これを原点Oを中心に$\Omega$だけ回転し、さらに点U$_\mathrm B$($R_\mathrm i-\rho$, $-\alpha_{\mathrm U_\mathrm B}$)を中心に点T$_\mathrm i$($R_\mathrm i$, $\alpha_{\mathrm T_\mathrm i}$)を$-\theta$だけ回転すると、
%% label{eq:afterftUfinite}
\begin{align}
  \notag
  & e^{-i\theta}\!
    \left\{R_\mathrm ie^{i(\alpha_{\mathrm T_\mathrm i}+\Omega)}
           -R_\mathrm i'e^{-i\alpha_{\mathrm U_\mathrm B}}\right\}
    +R_\mathrm i'e^{-i\alpha_{\mathrm U_\mathrm B}}\\
  & = R_\mathrm ie^{i(\alpha_{\mathrm T_\mathrm i}+\Omega-\theta)}
      -R_\mathrm i'\!
       \left\{e^{-i(\alpha_{\mathrm U_\mathrm B}+\theta)}-e^{-i\alpha_{\mathrm U_\mathrm B}}\right\}\qquad
    \big(R_\mathrm i' \equiv R_\mathrm i-\rho\big)
    \label{eq:afterftUfinite}
\end{align}
に移動する。
同様に点T$_\mathrm o$($R_\mathrm o$, $\alpha_{\mathrm T_\mathrm o}$)は
\begin{align*}
  R_\mathrm oe^{i(\alpha_{\mathrm T_\mathrm o}+\Omega-\theta)}
  -R_\mathrm i'\!
   \left\{e^{-i(\alpha_{\mathrm U_\mathrm B} + \theta)} - e^{-i\alpha_{\mathrm U_\mathrm B}}\right\}
\end{align*}
に移動する。
したがって、これらの差
\begin{align*}
  e^{i(\Omega-\theta)}
  \left(R_\mathrm oe^{i\alpha_{\mathrm T_\mathrm o}} - R_\mathrm ie^{i\alpha_{\mathrm T_\mathrm i}}\right)
\end{align*}
の虚部が0であればよい。
つまり、\pageeqref{eq:constraintUpoint1}より、受板がある場合も$\Omega = \theta$である。


%%%%%%%%%%%%%%%%%%%%%%%%%%%%%%%%%%%%%%%%%%%%%%%%%%%%%%%%%%
%% subsection 20.2.1 %%%%%%%%%%%%%%%%%%%%%%%%%%%%%%%%%%%%%
%%%%%%%%%%%%%%%%%%%%%%%%%%%%%%%%%%%%%%%%%%%%%%%%%%%%%%%%%%
\subsection{受板の接点}
\index{うけいた@受板}受板とワークとの(トップ側の)接点は、$R_\mathrm ie^{i\alpha_{\mathrm U_\mathrm B}}$で与えられる。
このとき厚さ$\delta_\mathrm s$のスペーサを取付けると、U$_\mathrm B$を中心に回転するが、それに伴い\index{ワークとジグのせってん@ワークとジグの接点}受板との接点の位置も変化する。

%%%%%%%%%%%%%%%%%%%%%%%%%%%%%%%%%%%%%%%%%%%%%%%%%%%%%%%%%%
%% subsubsection 19.2.1.1 %%%%%%%%%%%%%%%%%%%%%%%%%%%%%%%%
%%%%%%%%%%%%%%%%%%%%%%%%%%%%%%%%%%%%%%%%%%%%%%%%%%%%%%%%%%
\subsubsection{回転後のモールドの湾曲中心}
%%%%%%%%%%%%%%%%%%%%%%%%%
厚さ$\delta_\mathrm s$の\index{スペーサ}スペーサを挟むと、トップ側における\index{うけいたのちゅうしん@受板の中心}受板の円の中心U$_\mathrm B$は実軸方向に$\delta_\mathrm s$だけ移動するので、
\begin{align*}
  R_\mathrm i'e^{i\alpha_{\mathrm U_\mathrm B}}
  \quad\longrightarrow\quad
  \delta_\mathrm s+R_\mathrm i'e^{i\alpha_{\mathrm U_\mathrm B}}\ .
\end{align*}
よって、それぞれの受板の中心U$_\mathrm B$, U$_\mathrm T$を結んだ線分U$_\mathrm B$U$_\mathrm T$は、U$_\mathrm B$を中心に$-\psi$だけ傾いた線分U$_\mathrm B'$U$_\mathrm T'$となる
%% footnote %%%%%%%%%%%%%%%%%%%%%
\footnote{%
U$_\mathrm B'$U$_\mathrm T'$の長さは$\bar l\sec\psi$であり、U$_\mathrm B$U$_\mathrm T$の長さ$\bar l$より長くなることに注意。}。
%%%%%%%%%%%%%%%%%%%%%%%%%%%%%%%%%
\index{かいてんごのわんきょくちゅうしん@回転後の湾曲中心}回転後のワークの湾曲中心は、この線分の垂直二等分線上にあり、またそれぞれの受板の中心から$R_\mathrm i'$の距離の位置にある。
つまり、この傾いた線分U$_\mathrm B'$U$_\mathrm T'$の中点から、角度$\pi-\psi$, 大きさ$\sqrt{R_\mathrm i'^2-\frac{\delta_\mathrm s^2+(2\bar l)^2}4}$の位置に移動する。
したがって、回転後における湾曲中心O$'$は、
%% label{eq:afterOrgin}
\begin{align}
  \notag
  & \frac{\delta_\mathrm s}2+\sqrt{R_\mathrm i'^2-\bar l^2}
    +\sqrt{R_\mathrm i'^2-\frac{\delta_\mathrm s^2+(2\bar l)^2}4}e^{i(\pi-\psi)}\\
  & = \frac{\delta_\mathrm s}2+\sqrt{R_\mathrm i'^2-\bar l^2}-\sqrt{R_\mathrm i'^2-\frac{\delta_\mathrm s^2+(2\bar l)^2}4}\cos\psi
      +i\sqrt{R_\mathrm i'^2-\frac{\delta_\mathrm s^2+(2\bar l)^2}4}\sin\psi\ .
    \label{eq:afterOrgin}
\end{align}


\clearpage
%%%%%%%%%%%%%%%%%%%%%%%%%%%%%%%%%%%%%%%%%%%%%%%%%%%%%%%%%%
%% subsubsection 19.2.1.2 %%%%%%%%%%%%%%%%%%%%%%%%%%%%%%%%
%%%%%%%%%%%%%%%%%%%%%%%%%%%%%%%%%%%%%%%%%%%%%%%%%%%%%%%%%%
\subsubsection{回転後の接点(トップ側)}
\index{かいてんごのうけいたのちゅうしん@回転後の受板の中心}回転後のトップ側における受板の中心U$_\mathrm T'$と\index{かいてんごのわんきょくちゅうしん@回転後の湾曲中心}湾曲中心O$'$との差をとると、
\begin{align*}
  \frac{\delta_\mathrm s}2+\sqrt{R_\mathrm i'^2-\frac{\delta_\mathrm s^2+(2\bar l)^2}4}\cos\psi
  +i\left\{\bar l-\sqrt{R_\mathrm i'^2-\frac{\delta_\mathrm s^2+(2\bar l)^2}4}\sin\psi\right\}
  = R_\mathrm i'e^{i\alpha'_{\mathrm U_\mathrm T}}\ .
\end{align*}
ここで、
\begin{align*}
  \tan\alpha'_{\mathrm U_\mathrm T}
  = \frac{\displaystyle\bar l-\sqrt{R_\mathrm i'^2-\frac{\delta_\mathrm s^2+(2\bar l)^2}4}\sin\psi}
         {\displaystyle\frac{\delta_\mathrm s}2+\sqrt{R_\mathrm i'^2-\frac{\delta_\mathrm s^2+(2\bar l)^2}4}\cos\psi}\ .
\end{align*}
%%%%%%%%%%%%%%%%%%%%%%%%%%%%%%%%%%%%%%%%%%%%%%%%%%%%%%%%%%
%% hosoku %%%%%%%%%%%%%%%%%%%%%%%%%%%%%%%%%%%%%%%%%%%%%%%%
%%%%%%%%%%%%%%%%%%%%%%%%%%%%%%%%%%%%%%%%%%%%%%%%%%%%%%%%%%
\begin{hosoku}
これの大きさは、$\delta_\mathrm s\cos\psi-2\bar l\sin\psi = 0$より、
\begin{align*}
  \left\{\frac{\delta_\mathrm s}2+\sqrt{R_\mathrm i'^2-\frac{\delta_\mathrm s^2+(2\bar l)^2}4}\cos\psi\right\}^2
  +\left\{\bar l-\sqrt{R_\mathrm i'^2-\frac{\delta_\mathrm s^2+(2\bar l)^2}4}\sin\psi\right\}^2
  = R_\mathrm i'^2\ .
\end{align*}
\end{hosoku}
%%%%%%%%%%%%%%%%%%%%%%%%%%%%%%%%%%%%%%%%%%%%%%%%%%%%%%%%%%
%%%%%%%%%%%%%%%%%%%%%%%%%%%%%%%%%%%%%%%%%%%%%%%%%%%%%%%%%%
%%%%%%%%%%%%%%%%%%%%%%%%%%%%%%%%%%%%%%%%%%%%%%%%%%%%%%%%%%
よって、回転後の接点は以下で与えられる。
\begin{align*}
  &  R_\mathrm ie^{i\alpha'_{\mathrm U_\mathrm T}}
     +\frac{\delta_\mathrm s}2+\sqrt{R_\mathrm i'^2-\bar l^2}-\sqrt{R_\mathrm i'^2-\frac{\delta_\mathrm s^2+(2\bar l)^2}4}\cos\psi
     +i\sqrt{R_\mathrm i'^2-\frac{\delta_\mathrm s^2+(2\bar l)^2}4}\sin\psi\\
  &= \delta_\mathrm s+R_\mathrm i'e^{i\alpha_{\mathrm U_\mathrm B}}+\rho e^{i\alpha'_{\mathrm U_\mathrm T}}\ .
\end{align*}


%%%%%%%%%%%%%%%%%%%%%%%%%%%%%%%%%%%%%%%%%%%%%%%%%%%%%%%%%%
%% subsubsection 1.2.1.3 %%%%%%%%%%%%%%%%%%%%%%%%%%%%%%%%%
%%%%%%%%%%%%%%%%%%%%%%%%%%%%%%%%%%%%%%%%%%%%%%%%%%%%%%%%%%
\subsubsection{回転後の接点(ボトム側)}
回転後のボトム側における受板の中心U$_\mathrm B$と\index{わんきょくちゅうしん@湾曲中心}湾曲中心O$'$との差をとると、
\begin{align*}
  -\frac{\delta_\mathrm s}2+\sqrt{R_\mathrm i'^2-\frac{\delta_\mathrm s^2+(2\bar l)^2}4}\cos\psi
  -i\left\{\bar l+\sqrt{R_\mathrm i'^2-\frac{\delta_\mathrm s^2+(2\bar l)^2}4}\sin\psi\right\}
  = R_\mathrm i'e^{-i\alpha'_{\mathrm U_\mathrm B}}
\end{align*}
ここで、
\begin{align*}
  \tan\alpha'_{\mathrm U_\mathrm B}
  = \frac{\displaystyle\bar l+\sqrt{R_\mathrm i'^2-\frac{\delta_\mathrm s^2+(2\bar l)^2}4}\sin\psi}
         {\displaystyle-\frac{\delta_\mathrm s}2+\sqrt{R_\mathrm i'^2-\frac{\delta_\mathrm s^2+(2\bar l)^2}4}\cos\psi}
\end{align*}
よって、回転後の接点は以下で与えられる。
%% label{eq:afterUBcontact}
\begin{align}
  \notag
  &  R_\mathrm ie^{-i\alpha'_{\mathrm U_\mathrm B}}
     +\frac{\delta_\mathrm s}2+\sqrt{R_\mathrm i'^2-\bar l^2}-\sqrt{R_\mathrm i'^2-\frac{\delta_\mathrm s^2+(2\bar l)^2}4}\cos\psi
     +i\sqrt{R_\mathrm i'^2-\frac{\delta_\mathrm s^2+(2\bar l)^2}4}\sin\psi\\
  &= R_\mathrm i'e^{-i\alpha_{\mathrm U_\mathrm B}}+\rho e^{-i\alpha'_{\mathrm U_\mathrm B}}
   \label{eq:afterUBcontact}
\end{align}
%%%%%%%%%%%%%%%%%%%%%%%%%%%%%%%%%%%%%%%%%%%%%%%%%%%%%%%%%%
%% hosoku %%%%%%%%%%%%%%%%%%%%%%%%%%%%%%%%%%%%%%%%%%%%%%%%
%%%%%%%%%%%%%%%%%%%%%%%%%%%%%%%%%%%%%%%%%%%%%%%%%%%%%%%%%%
\begin{hosoku}
辺の長さが$R_i'$, $R_i'$, $2\bar l$の二等辺三角形$\triangle$OU$_\mathrm B$U$_\mathrm T$の部分が、回転後には辺の長さ$R_i'$, $R_i'$, $\sqrt{\delta_\mathrm s^2+(2\bar l)^2}$の二等辺三角形$\triangle$O$'$U$_\mathrm B'$U$_\mathrm T'$となる。
実際、$\cos2a = 1-2\sin^2\!a$より、
\begin{align*}
  \sin^2\frac{\alpha'_{\mathrm U_\mathrm T}+\alpha'_{\mathrm U_\mathrm B}}2
  = \frac{\delta_\mathrm s^2+(2\bar l)^2}{4R_\mathrm i'^2}\ .
\end{align*}
\end{hosoku}
%%%%%%%%%%%%%%%%%%%%%%%%%%%%%%%%%%%%%%%%%%%%%%%%%%%%%%%%%%
%%%%%%%%%%%%%%%%%%%%%%%%%%%%%%%%%%%%%%%%%%%%%%%%%%%%%%%%%%
%%%%%%%%%%%%%%%%%%%%%%%%%%%%%%%%%%%%%%%%%%%%%%%%%%%%%%%%%%


%%%%%%%%%%%%%%%%%%%%%%%%%%%%%%%%%%%%%%%%%%%%%%%%%%%%%%%%%%
%% subsection 1.2.2 %%%%%%%%%%%%%%%%%%%%%%%%%%%%%%%%%%%%%%
%%%%%%%%%%%%%%%%%%%%%%%%%%%%%%%%%%%%%%%%%%%%%%%%%%%%%%%%%%
\subsection{スペーサによるモールドの回転角}
厚さ$\delta_\mathrm s$の\index{スペーサ}スペーサを挿入すると、\index{わんきょくちゅうしん@湾曲中心}湾曲中心Oは、U$_\mathrm B$を中心に$-\left(\alpha'_{\mathrm U_\mathrm B}\!-\alpha_{\mathrm U_\mathrm B}\right)$だけ回転する。
実際、
\begin{align*}
  -R_\mathrm i'e^{-i\alpha'_{\mathrm U_\mathrm B}}+R_\mathrm i'e^{-i\alpha_{\mathrm U_\mathrm B}}
  &= R_\mathrm i'(\cos\alpha_{\mathrm U_\mathrm B}-\cos\alpha'_{\mathrm U_\mathrm B})
     +iR_\mathrm i'(\sin\alpha'_{\mathrm U_\mathrm B}-\sin\alpha_{\mathrm U_\mathrm B})
\end{align*}
であり、これは\index{かいてんごのわんきょくちゅうしん@回転後の湾曲中心}回転後の湾曲中心\pageeqref{eq:afterOrgin}に一致する。
つまり、$\alpha'_{\mathrm U_\mathrm B}\!-\alpha_{\mathrm U_\mathrm B}$が$\theta$に相当する。
%%%%%%%%%%%%%%%%%%%%%%%%%%%%%%%%%%%%%%%%%%%%%%%%%%%%%%%%%%
%% hosoku %%%%%%%%%%%%%%%%%%%%%%%%%%%%%%%%%%%%%%%%%%%%%%%%
%%%%%%%%%%%%%%%%%%%%%%%%%%%%%%%%%%%%%%%%%%%%%%%%%%%%%%%%%%
\begin{hosoku}
トップ側の接点U$_\mathrm T'$とボトム側の接点U$_\mathrm B'$の差をとると、
\begin{align*}
  R_\mathrm i\left(e^{i\alpha_{\mathrm U_\mathrm T}'}-e^{-i\alpha'_{\mathrm U_\mathrm B}}\right)
  &= \frac{R_\mathrm i'+\rho}{R_\mathrm i'}\left\{\delta_\mathrm s+i(2\bar l)\right\}
   = \frac{R_\mathrm i}{R_\mathrm i'}\sqrt{\delta_\mathrm s^2+(2\bar l)^2}e^{i(\nicefrac\pi2-\psi)}\ .
\end{align*}
したがって、厚さ$\delta_\mathrm s$のスペーサを挿入すると、両接点を通る直線は$-\psi$だけ傾くことがわかる。
また、その長さは受板中心間U$_\mathrm T'$U$_\mathrm B'$の距離$\sqrt{\delta_\mathrm s^2+(2\bar l)^2}$の$\nicefrac{R_i}{R_i'}$倍になっていることも確かめられる。
\end{hosoku}
%%%%%%%%%%%%%%%%%%%%%%%%%%%%%%%%%%%%%%%%%%%%%%%%%%%%%%%%%%
%%%%%%%%%%%%%%%%%%%%%%%%%%%%%%%%%%%%%%%%%%%%%%%%%%%%%%%%%%
%%%%%%%%%%%%%%%%%%%%%%%%%%%%%%%%%%%%%%%%%%%%%%%%%%%%%%%%%%


%%%%%%%%%%%%%%%%%%%%%%%%%%%%%%%%%%%%%%%%%%%%%%%%%%%%%%%%%%
%% subsection 1.2.3 %%%%%%%%%%%%%%%%%%%%%%%%%%%%%%%%%%%%%%
%%%%%%%%%%%%%%%%%%%%%%%%%%%%%%%%%%%%%%%%%%%%%%%%%%%%%%%%%%
\subsection{スペーサによる再振分け}

%%%%%%%%%%%%%%%%%%%%%%%%%%%%%%%%%%%%%%%%%%%%%%%%%%%%%%%%%%
%% subsubsection 1.2.3.1 %%%%%%%%%%%%%%%%%%%%%%%%%%%%%%%%%
%%%%%%%%%%%%%%%%%%%%%%%%%%%%%%%%%%%%%%%%%%%%%%%%%%%%%%%%%%
\subsubsection{再振分長}
\index{スペーサ}スペーサを入れた後のトップ側の\index{ふりわけちょう@振分長}振分長(\index{さいふりわけちょう@再振分長}\textbf{再振分長})$f'_\mathrm T$は、\pageeqref{eq:afterftUfinite}の虚部を見ればよい。
回転角は$-(\alpha'_{\mathrm U_\mathrm B}-\alpha_{\mathrm U_\mathrm B})$なので、
\begin{align*}
  f'_\mathrm T
  = R_\mathrm i\sin\alpha_{\mathrm T_\mathrm i}
    +R_\mathrm i'\left(\sin\alpha'_{\mathrm U_\mathrm B}-\sin\alpha_{\mathrm U_\mathrm B}\right)
  &= f_\mathrm T+\sqrt{R_\mathrm i'^2-\frac{\delta_\mathrm s^2+(2\bar l)^2}4}\sin\psi\\
  &= f_\mathrm T+\sqrt{R_\mathrm i'^2-\frac{\delta_\mathrm s^2+(2\bar l)^2}4}\frac{\delta_\mathrm s}{\sqrt{\delta_\mathrm s^2+(2\bar l)^2}}\ .
\end{align*}

%%%%%%%%%%%%%%%%%%%%%%%%%%%%%%%%%%%%%%%%%%%%%%%%%%%%%%%%%%
%% subsubsection 1.2.3.2 %%%%%%%%%%%%%%%%%%%%%%%%%%%%%%%%%
%%%%%%%%%%%%%%%%%%%%%%%%%%%%%%%%%%%%%%%%%%%%%%%%%%%%%%%%%%
\subsubsection{モールドの移動距離}
\pageeqref{eq:afterftUfinite}の実部は、
\begin{align*}
  & R_\mathrm i\cos\alpha_{\mathrm T_\mathrm i}
    -R_\mathrm i'(\cos\alpha'_{\mathrm U_\mathrm B}-\cos\alpha_{\mathrm U_\mathrm B})\\
  & = \sqrt{R_\mathrm i^2-f_\mathrm T^2}+\frac{\delta_\mathrm s}2+\sqrt{R_\mathrm i'^2-\bar l^2}
      -\sqrt{R_\mathrm i'^2-\frac{\delta_\mathrm s^2+(2\bar l)^2}4}\cos\psi
\end{align*}
となる。
よって、\index{スペーサ}スペーサを挿入することにより、ワークは水平・鉛直方向にそれぞれ、
\begin{subequations}
%% label{eq:spacerMoveHdistance}
\begin{alignat}{2}
  \label{eq:spacerMoveHdistance}
  \text{水平方向:}\quad
  & \frac{\delta_\mathrm s}2+\sqrt{R_\mathrm i'^2-\bar l^2}-\sqrt{R_\mathrm i'^2-\frac{\delta_\mathrm s^2+(2\bar l)^2}4}\frac{2\bar l}{\sqrt{\delta_\mathrm s^2+(2\bar l)^2}}\\
  \text{鉛直方向:}\quad
  & \sqrt{R_\mathrm i'^2-\frac{\delta_\mathrm s^2+(2\bar l)^2}4}\frac{\delta_\mathrm s}{\sqrt{\delta_\mathrm s^2+(2\bar l)^2}}
\end{alignat}
\end{subequations}
だけ移動することがわかる。



%\clearpage
%%%%%%%%%%%%%%%%%%%%%%%%%%%%%%%%%%%%%%%%%%%%%%%%%%%%%%%%%%
%% subsection 1.2.4 %%%%%%%%%%%%%%%%%%%%%%%%%%%%%%%%%%%%%%
%%%%%%%%%%%%%%%%%%%%%%%%%%%%%%%%%%%%%%%%%%%%%%%%%%%%%%%%%%
\subsection{再振分長が均等になるスペーサ厚}
トップ側とボトム側の\index{きんとうふりわけ@均等振分}振分長が同じになるとき、$\delta_\mathrm s$は
\begin{align*}
  \sqrt{R_\mathrm i'^2-\frac{\delta_\mathrm s^2+(2\bar l)^2}4}\frac{\delta_\mathrm s}{\sqrt{\delta_\mathrm s^2+(2\bar l)^2}} = f_d \qquad
  \left(f_d \equiv \frac{f_\mathrm B-f_\mathrm T}2\right)\ .
\end{align*}
を満たす。
両辺を2乗して$-4$倍すると、
\begin{align*}
  \delta_\mathrm s^2\left\{\delta_\mathrm s^2+(2\bar l)^2-4R_\mathrm i'^2\right\}+4f_d^2\left\{\delta_\mathrm s^2+(2\bar l)^2\right\}
  & = \delta_\mathrm s^4-2\left\{2R_\mathrm i'^2-2\bar l^2-2f_d^2\right\}\delta_\mathrm s^2+4f_d^2(2\bar l)^2\\
  & = 0\ .
\end{align*}
したがって、$f_\mathrm T = f_\mathrm B$ ($f_d = 0$)のとき$\delta_\mathrm s = 0$であることを考慮して、
\begin{align*}
  \delta_\mathrm s^2
  &= 2\left\{
       R_\mathrm i'^2-\bar l^2-f_d^2-\sqrt{\left(R_\mathrm i'^2-\bar l^2-f_d^2\right)^2-4f_d^2\bar l^2}\,
     \right\}\\
  &= \left(\sqrt{R_\mathrm i'^2-(\bar l-f_d)^2}-\sqrt{R_\mathrm i'^2-(\bar l+f_d)^2}\,\right)^2.
\end{align*}
なお、これはワークが水平・鉛直方向にそれぞれ、
\begin{align*}
  \text{水平方向:}~\frac{\delta_\mathrm s}2+\sqrt{R_\mathrm i^2-\bar l^2}-\frac{2\bar l}{\delta_\mathrm s}f_d\quad(\delta_\mathrm s>0)\ , \qquad
  \text{鉛直方向:}~\frac{f_\mathrm B-f_\mathrm T}2
\end{align*}
だけ移動することを意味する。
%%%%%%%%%%%%%%%%%%%%%%%%%%%%%%%%%%%%%%%%%%%%%%%%%%%%%%%%%%
%% hosoku %%%%%%%%%%%%%%%%%%%%%%%%%%%%%%%%%%%%%%%%%%%%%%%%
%%%%%%%%%%%%%%%%%%%%%%%%%%%%%%%%%%%%%%%%%%%%%%%%%%%%%%%%%%
\begin{hosoku}
改めてまとめると、厚さ$\delta_\mathrm s$のスペーサを(トップ側に)挿入した後の\index{トップふりわけちょう@トップ振分長}トップ側の振分長$f_\mathrm T'$は、
\begin{align*}
  f_\mathrm T'
  = f_\mathrm T+\sqrt{R_\mathrm i'^2-\frac{\delta_\mathrm s^2+(2\bar l)^2}4}\frac{\delta_\mathrm s}{\sqrt{\delta_\mathrm s^2+(2\bar l)^2}}\ .
\end{align*}
トップ側とボトム側の\index{きんとうふりわけ@均等振分}振分長が均等になるときの\index{スペーサあつ@スペーサ厚}スペーサ厚$\delta_\mathrm s'$は、
\begin{align*}
  \delta_\mathrm s' = \sqrt{R_\mathrm i'^2-(\bar l-f_d)^2}-\sqrt{R_\mathrm i'^2-(\bar l+f_d)^2}\ .
\end{align*}
ここで、
\begin{align*}
  R_\mathrm i' = R_\mathrm c-\frac{W_x}2-\rho\ ,\quad
  \bar l = l-\frac\sigma2\ ,\quad
  f_d = \frac{f_\mathrm B-f_\mathrm T}2\ .
\end{align*}
\end{hosoku}
%%%%%%%%%%%%%%%%%%%%%%%%%%%%%%%%%%%%%%%%%%%%%%%%%%%%%%%%%%
%%%%%%%%%%%%%%%%%%%%%%%%%%%%%%%%%%%%%%%%%%%%%%%%%%%%%%%%%%
%%%%%%%%%%%%%%%%%%%%%%%%%%%%%%%%%%%%%%%%%%%%%%%%%%%%%%%%%%



\clearpage
%%%%%%%%%%%%%%%%%%%%%%%%%%%%%%%%%%%%%%%%%%%%%%%%%%%%%%%%%%
%% section 21.3 %%%%%%%%%%%%%%%%%%%%%%%%%%%%%%%%%%%%%%%%%%
%%%%%%%%%%%%%%%%%%%%%%%%%%%%%%%%%%%%%%%%%%%%%%%%%%%%%%%%%%
\modHeadsection{テーブルの回転による振分長の調節}
これまでトップ・ボトム振分長の差を小さくするために\index{スペーサ}スペーサを用いる手法を考えてきた。
スペーサを取付けることは、本質的にはボトム側の\index{うけいた@受板}受板の点U$_\mathrm B$を中心に回転しているということである。
このとき回転の中心はU$_\mathrm B$である必要はなく、他の点でも問題ない。
したがって、スペーサを用いて回転をしなくても、\index{テーブル}テーブルそのものを回転するという手法が考えられる。
これは回転の中心が、受板の点U$_\mathrm B$から\index{テーブルちゅうしん@テーブル中心}\textbf{テーブル中心}Pに変わることに相当する。

\index{うけいたのちゅうしん@受板の中心}受板の中心U$_\mathrm B$と、テーブル中心Pとの実軸方向の距離を$\Delta$とすると、テーブル中心Pの$X$座標$\Delta'$は次で与えられる。
%% label{eq:tableCenter}
\begin{align}
  \label{eq:tableCenter}
  \Delta'
  = \Delta+R_\mathrm i'\cos\alpha_{\mathrm U_\mathrm B} = \Delta+\sqrt{R_\mathrm i'^2-\bar l^2}\ .
\end{align}
\index{トップがわのCがわそとたんてん@トップ側のC側外端点}トップ側のC側外端点T$_\mathrm i$($R_\mathrm i$, $\alpha_{\mathrm T_\mathrm i}$)を、原点Oを中心に$\Omega$だけ回転し、さらに点P\,($\Delta'$, $0$)を中心に$-\theta$だけ回転すると、
\begin{align}
  \label{eq:afterfttable}
  e^{-i\theta}\left\{R_\mathrm i^{i(\alpha_{\mathrm T_\mathrm i}+\Omega)}-\Delta'\right\}+\Delta'
  = R_\mathrm ie^{i(\alpha_{\mathrm T_\mathrm i}+\Omega-\theta)}+\Delta'\left(1-e^{-i\theta}\right)
\end{align}
に移動する。
同様に、\index{トップがわのAがわそとたんてん@トップ側のA側外端点}トップ側のA側外端点T$_\mathrm o$($R_\mathrm o$, $\alpha_{\mathrm T_\mathrm o}$)は、
\begin{align*}
  e^{-i\theta}\!\left\{R_\mathrm o^{i(\alpha_{\mathrm T_\mathrm o}+\Omega)}-\Delta'\right\}+\Delta'
  = R_\mathrm ie^{i(\alpha_{\mathrm T_\mathrm o}+\Omega-\theta)}+\Delta'\!\left(1-e^{-i\theta}\right)
\end{align*}
に移動する。
したがって、これらの差
\begin{align*}
  e^{i(\Omega-\theta)}
  \left(R_\mathrm oe^{i\alpha_{\mathrm T_\mathrm o}}-R_\mathrm ie^{i\alpha_{\mathrm T_\mathrm i}}\right)
\end{align*}
の虚部が$0$であればよい。
よって\pageeqref{eq:constraintUpoint1}より、この場合も$\Omega = \theta$となる。


%%%%%%%%%%%%%%%%%%%%%%%%%%%%%%%%%%%%%%%%%%%%%%%%%%%%%%%%%%
%% subsection 21.3.1 %%%%%%%%%%%%%%%%%%%%%%%%%%%%%%%%%%%%%
%%%%%%%%%%%%%%%%%%%%%%%%%%%%%%%%%%%%%%%%%%%%%%%%%%%%%%%%%%
\subsection{回転後のモールドの湾曲中心および受板との接点}

%%%%%%%%%%%%%%%%%%%%%%%%%%%%%%%%%%%%%%%%%%%%%%%%%%%%%%%%%%
%% subsubsection 21.3.1.1 %%%%%%%%%%%%%%%%%%%%%%%%%%%%%%%%
%%%%%%%%%%%%%%%%%%%%%%%%%%%%%%%%%%%%%%%%%%%%%%%%%%%%%%%%%%
\subsubsection{回転後のモールドの湾曲中心}
\index{かいてんごのわんきょくちゅうしん@回転後の湾曲中心}回転後の湾曲中心O$'$は、点Pを中心に$-\theta$だけ回転するので、
\begin{align*}
  \Delta'\!\left(1-e^{-i\theta}\right) = \Delta'(1-\cos\theta)+i\Delta'\sin\theta\ .
\end{align*}

%%%%%%%%%%%%%%%%%%%%%%%%%%%%%%%%%%%%%%%%%%%%%%%%%%%%%%%%%%
%% subsubsection 21.3.1.2 %%%%%%%%%%%%%%%%%%%%%%%%%%%%%%%%
%%%%%%%%%%%%%%%%%%%%%%%%%%%%%%%%%%%%%%%%%%%%%%%%%%%%%%%%%%
\subsubsection{回転後の接点}
回転後におけるトップ側の\index{ワークとうけいたのせってん@ワークと受板の接点}受板との接点は、点Pを中心に$-\theta$だけ回転するので、
\begin{align*}
  &  e^{-i\theta}\left(R_\mathrm ie^{i\alpha_{\mathrm U_\mathrm B}}-\Delta'\right)+\Delta'\\
  &= R_\mathrm ie^{i(\alpha_{\mathrm U_\mathrm B}-\theta)}+\Delta'\!\left(1-e^{-i\theta}\right)\\
  &= R_\mathrm i\cos(\alpha_{\mathrm U_\mathrm B}-\theta)+\Delta'(1-\cos\theta)
     +i\left\{R_\mathrm i\sin(\alpha_{\mathrm U_\mathrm B}-\theta)+i\Delta'\sin\theta\right\}\ .
\end{align*}
同様に、ボトム側の受板との接点は、
\begin{align*}
  &  e^{-i\theta}\left(R_\mathrm ie^{-i\alpha_{\mathrm U_\mathrm B}}-\Delta'\right)+\Delta'\\
  &= R_\mathrm ie^{-i(\alpha_{\mathrm U_\mathrm B}+\theta)}+\Delta'\!\left(1-e^{-i\theta}\right)\\
  &= R_\mathrm i\cos(\alpha_{\mathrm U_\mathrm B}+\theta)+\Delta'(1-\cos\theta)
     -i\left\{R_\mathrm i\sin(\alpha_{\mathrm U_\mathrm B}+\theta)-i\Delta'\sin\theta\right\}\ .
\end{align*}
%%%%%%%%%%%%%%%%%%%%%%%%%%%%%%%%%%%%%%%%%%%%%%%%%%%%%%%%%%
%% hosoku %%%%%%%%%%%%%%%%%%%%%%%%%%%%%%%%%%%%%%%%%%%%%%%%
%%%%%%%%%%%%%%%%%%%%%%%%%%%%%%%%%%%%%%%%%%%%%%%%%%%%%%%%%%
\begin{hosoku}
両接点との差をとると、
\begin{align*}
  2R_\mathrm i\sin\alpha_{\mathrm U_\mathrm B}\sin\theta+2iR_\mathrm i\sin\alpha_{\mathrm U_\mathrm B}\cos\theta
  = \frac{R_\mathrm i}{R_\mathrm i'}(2\bar l)e^{i(\pi-\theta)}
\end{align*}
となり、\index{うけいたのちゅうしん@受板の中心}受板の両中心間(長さ$2\bar l$)を結んだ線分を$\nicefrac{R_\mathrm i}{R_\mathrm i'}$倍し、虚軸から$-\theta$だけ傾けたものになっていることがわかる。
\end{hosoku}
%%%%%%%%%%%%%%%%%%%%%%%%%%%%%%%%%%%%%%%%%%%%%%%%%%%%%%%%%%
%%%%%%%%%%%%%%%%%%%%%%%%%%%%%%%%%%%%%%%%%%%%%%%%%%%%%%%%%%
%%%%%%%%%%%%%%%%%%%%%%%%%%%%%%%%%%%%%%%%%%%%%%%%%%%%%%%%%%


%%%%%%%%%%%%%%%%%%%%%%%%%%%%%%%%%%%%%%%%%%%%%%%%%%%%%%%%%%
%% subsection 21.3.2 %%%%%%%%%%%%%%%%%%%%%%%%%%%%%%%%%%%%%
%%%%%%%%%%%%%%%%%%%%%%%%%%%%%%%%%%%%%%%%%%%%%%%%%%%%%%%%%%
\subsection{回転後の振分長}
回転後の\index{トップさいふりわけちょう@トップ再振分長}トップ側の振分長$f_\mathrm T'$は、\pageeqref{eq:afterfttable}の虚部を見ればよいので、
\begin{subequations}
\label{eq:saifuriwake}
\begin{align}
  \label{eq:saifuriwakeT}
  f_\mathrm T'
  = f_\mathrm T+\Delta'\sin\theta
  = f_\mathrm T+\left(\Delta+\sqrt{R_\mathrm i'-\bar l^2}\right)\sin\theta\ .
\end{align}
同様に、\index{ボトムさいふりわけちょう@ボトム再振分長}ボトムの振分長$f_\mathrm B'$は、符号に注意して、
\begin{align}
  f_\mathrm B' = f_\mathrm B-\Delta'\sin\theta = (f_\mathrm T+f_\mathrm B)-f_\mathrm T'\ .
\end{align}
\end{subequations}
%%%%%%%%%%%%%%%%%%%%%%%%%%%%%%%%%%%%%%%%%%%%%%%%%%%%%%%%%%
%% hosoku %%%%%%%%%%%%%%%%%%%%%%%%%%%%%%%%%%%%%%%%%%%%%%%%
%%%%%%%%%%%%%%%%%%%%%%%%%%%%%%%%%%%%%%%%%%%%%%%%%%%%%%%%%%
\begin{hosoku}
\index{テーブルちゅうしん@テーブル中心}テーブル中心Pによる回転は\index{ふりわけちょう@振分長}振分長に影響しない(\index{たんめん@端面}端面を水平に戻す)ので、\index{わんきょくちゅうしん@湾曲中心}湾曲中心Oによる回転だけが影響する。
よって、振分長は$\Delta'\sin\theta$だけ変化する。
\end{hosoku}
%%%%%%%%%%%%%%%%%%%%%%%%%%%%%%%%%%%%%%%%%%%%%%%%%%%%%%%%%%
%%%%%%%%%%%%%%%%%%%%%%%%%%%%%%%%%%%%%%%%%%%%%%%%%%%%%%%%%%
%%%%%%%%%%%%%%%%%%%%%%%%%%%%%%%%%%%%%%%%%%%%%%%%%%%%%%%%%%
なお、\pageeqref{eq:afterfttable}の実部は、
\begin{align*}
  R_\mathrm i\cos\alpha_{\mathrm T_\mathrm i}+\Delta'(1-\cos\theta)
  = \sqrt{R_\mathrm i^2-\bar l^2}+\left(\Delta+\sqrt{R_\mathrm i'-\bar l^2}\right)(1-\cos\theta)\ .
\end{align*}
となるので、実軸の正方向に動くことがわかる。


%%%%%%%%%%%%%%%%%%%%%%%%%%%%%%%%%%%%%%%%%%%%%%%%%%%%%%%%%%
%% subsection 21.3.3 %%%%%%%%%%%%%%%%%%%%%%%%%%%%%%%%%%%%%
%%%%%%%%%%%%%%%%%%%%%%%%%%%%%%%%%%%%%%%%%%%%%%%%%%%%%%%%%%
\subsection{振分長を指定したときの回転角}
トップ側の振分長が$f_\mathrm T'$となる回転角$\theta$は、\pageeqref{eq:saifuriwakeT}より、
\begin{align*}
  \sin\theta = \frac{f_\mathrm T'-f_\mathrm T}{\Delta'}
\end{align*}
で与えられる。
特に、トップ側およびボトム側の\index{きんとうふりわけ@均等振分}振分長が同じになるとき、$\theta$は
\begin{align*}
  \Delta'\sin\theta = f_d \qquad \left(f_d \equiv \frac{f_\mathrm B-f_\mathrm T}2\right)
\end{align*}
であればいいので、
\begin{align}
  \label{eq:saifuriwakeangle}
  \sin\theta = \frac{f_d}{\Delta'}
  = \frac{f_\mathrm B-f_\mathrm T}{2\left(\Delta+\sqrt{R_\mathrm i'-\bar l^2}\right)}~.
\end{align}
%%%%%%%%%%%%%%%%%%%%%%%%%%%%%%%%%%%%%%%%%%%%%%%%%%%%%%%%%%
%% hosoku %%%%%%%%%%%%%%%%%%%%%%%%%%%%%%%%%%%%%%%%%%%%%%%%
%%%%%%%%%%%%%%%%%%%%%%%%%%%%%%%%%%%%%%%%%%%%%%%%%%%%%%%%%%
\begin{hosoku}
改めてまとめると、テーブルを$-\theta$だけ傾けた後の\index{トップさいふりわけちょう@トップ再振分長}トップ側の振分長$f_\mathrm T'$は、
\begin{align*}
  f_\mathrm T' = f_\mathrm T+\left(\Delta+\sqrt{R_\mathrm i'-\bar l^2}\right)\sin\theta\ .
\end{align*}
トップ側とボトム側の振分長が均等になるときの\index{かたむきかく(ふりわけちょうせい)@傾き角(振分調整)}傾き角$\theta'$は、
\begin{align*}
  \sin\theta' = \frac{f_\mathrm B-f_\mathrm T}{2\left(\Delta+\sqrt{R_\mathrm i'-\bar l^2}\right)}\ .
\end{align*}
ここで、
\begin{align*}
  R_\mathrm i' = R_\mathrm c-\frac{W_x}2-\rho\ ,\quad
  \bar l = l-\frac\sigma2\ ,\quad
  f_d = \frac{f_\mathrm B-f_\mathrm T}2\ .
\end{align*}
\end{hosoku}
%%%%%%%%%%%%%%%%%%%%%%%%%%%%%%%%%%%%%%%%%%%%%%%%%%%%%%%%%%
%%%%%%%%%%%%%%%%%%%%%%%%%%%%%%%%%%%%%%%%%%%%%%%%%%%%%%%%%%
%%%%%%%%%%%%%%%%%%%%%%%%%%%%%%%%%%%%%%%%%%%%%%%%%%%%%%%%%%



%\clearpage
%%%%%%%%%%%%%%%%%%%%%%%%%%%%%%%%%%%%%%%%%%%%%%%%%%%%%%%%%%
%% section 21.3 %%%%%%%%%%%%%%%%%%%%%%%%%%%%%%%%%%%%%%%%%%
%%%%%%%%%%%%%%%%%%%%%%%%%%%%%%%%%%%%%%%%%%%%%%%%%%%%%%%%%%
\modHeadsection{湾曲のないモールド}
\index{わんきょくのないモールド@湾曲のないモールド}湾曲のないモールド($R_\mathrm c^{-1}= 0$)については、回転して調整する必要性がないため、スペーサによる調整も回転による調整も必要がない。
つまり$\delta_\mathrm s = 0$あるいは$\theta = 0$である。

このとき、\index{きんとうふりわけ@均等振分}トップ側およびボトム側の振分長を同じにするには、単純に$f_d$だけ移動すればよい。

~\vfill
%%%%%%%%%%%%%%%%%%%%%%%%%%%%%%%%%%%%%%%%%%%%%%%%%%%%%%%%%%
%% Column %%%%%%%%%%%%%%%%%%%%%%%%%%%%%%%%%%%%%%%%%%%%%%%%
%%%%%%%%%%%%%%%%%%%%%%%%%%%%%%%%%%%%%%%%%%%%%%%%%%%%%%%%%%
\begin{Column}{湾曲半径無限大の極限\TBW}

\end{Column}
%%%%%%%%%%%%%%%%%%%%%%%%%%%%%%%%%%%%%%%%%%%%%%%%%%%%%%%%%%
%%%%%%%%%%%%%%%%%%%%%%%%%%%%%%%%%%%%%%%%%%%%%%%%%%%%%%%%%%
%%%%%%%%%%%%%%%%%%%%%%%%%%%%%%%%%%%%%%%%%%%%%%%%%%%%%%%%%%





%%%%%%%%%%%%%%%%%%%%%%%%%%%%%%%%%%%%%%%%%%%%%%%%%%%%%%%%%%
%%           %%%%%%%%%%%%%%%%%%%%%%%%%%%%%%%%%%%%%%%%%%%%%
%% chapter 2 %%%%%%%%%%%%%%%%%%%%%%%%%%%%%%%%%%%%%%%%%%%%%
%%           %%%%%%%%%%%%%%%%%%%%%%%%%%%%%%%%%%%%%%%%%%%%%
%%%%%%%%%%%%%%%%%%%%%%%%%%%%%%%%%%%%%%%%%%%%%%%%%%%%%%%%%%
\modHeadchapter[loC]{端面(外径)の幾何}
\input{RfCPN_p5_Analytical_Calculation_chapter/c_RfCPN_end_face}




%%%%%%%%%%%%%%%%%%%%%%%%%%%%%%%%%%%%%%%%%%%%%%%%%%%%%%%%%%
%%           %%%%%%%%%%%%%%%%%%%%%%%%%%%%%%%%%%%%%%%%%%%%%
%% chapter 3 %%%%%%%%%%%%%%%%%%%%%%%%%%%%%%%%%%%%%%%%%%%%%
%%           %%%%%%%%%%%%%%%%%%%%%%%%%%%%%%%%%%%%%%%%%%%%%
%%%%%%%%%%%%%%%%%%%%%%%%%%%%%%%%%%%%%%%%%%%%%%%%%%%%%%%%%%
\modHeadchapter{外削の幾何}
\input{RfCPN_p5_Analytical_Calculation_chapter/c_RfCPN_exterior_milling}





%%%%%%%%%%%%%%%%%%%%%%%%%%%%%%%%%%%%%%%%%%%%%%%%%%%%%%%%%%
%%           %%%%%%%%%%%%%%%%%%%%%%%%%%%%%%%%%%%%%%%%%%%%%
%% chapter 4 %%%%%%%%%%%%%%%%%%%%%%%%%%%%%%%%%%%%%%%%%%%%%
%%           %%%%%%%%%%%%%%%%%%%%%%%%%%%%%%%%%%%%%%%%%%%%%
%%%%%%%%%%%%%%%%%%%%%%%%%%%%%%%%%%%%%%%%%%%%%%%%%%%%%%%%%%
\modHeadchapter[loC]{溝の幾何}
%!TEX root = ../RfCPN.tex


\modHeadchapter[loColumn]{\Keyway の幾何}
ここでは主に、\textbf{\Keyway}に関する測定・加工に必要な\expandafterindex{きかてきせいしつ(\yomiKeyway)@幾何的性質(\nameKeyway)}幾何的性質を考える。



%%%%%%%%%%%%%%%%%%%%%%%%%%%%%%%%%%%%%%%%%%%%%%%%%%%%%%%%%%
%% section 22.1 %%%%%%%%%%%%%%%%%%%%%%%%%%%%%%%%%%%%%%%%%%
%%%%%%%%%%%%%%%%%%%%%%%%%%%%%%%%%%%%%%%%%%%%%%%%%%%%%%%%%%
\modHeadsection{\Keyway の基本事項}
\textbf{\Keyway}に関しては、その基準が以下のように与えられる場合が考えられる。
\begin{enumerate}[label=\sarrow]
\item \KeywayCenter Mが、ワークの\CenterCurvatureLine 上にある場合
\item \KeywayCenter Mが、(トップ側)\index{がいさくちゅうしん@外削中心}外削径の中心線上にある場合
\item \AsideKeywayDepth に指定がある場合
\end{enumerate}
なお、\KeywayPos(\EndFace から\Keyway までの長さ)・\KeywayWidth・\AsideKeywayDepth をそれぞれ$\kappa_p$, $\kappa_w$, $\kappa_d$とする。
このときいずれの場合も、$y$方向(機内における$Z$方向
%% footnote %%%%%%%%%%%%%%%%%%%%%
\footnote{計算上の$xy$座標($x$:実軸, $y$:虚軸)と、機内における$XZ$座標とが混在する形で話を進めているので注意されたし。})
%%%%%%%%%%%%%%%%%%%%%%%%%%%%%%%%%
の\index{せっさくはんい@切削範囲}切削範囲は、\TableCenter Pを\index{げんてん@原点}原点として、
\begin{align*}
  \big[f_\mathrm T'-(\kappa_p+\kappa_w)\ ,\ f_\mathrm T'-\kappa_p\big]
\end{align*}
であり、また\KeywayCenter M$'$の$y$座標($Z$座標)はこの切削範囲の中央
%% label{eq:mizocenterZ}
\begin{align}
  \label{eq:mizocenterZ}
  f_\mathrm T'-\bar\kappa_w \qquad
  \ab(\bar\kappa_w \equiv \kappa_p+\frac{\kappa_w}2)
\end{align}
で与えられる。



\clearpage
%%%%%%%%%%%%%%%%%%%%%%%%%%%%%%%%%%%%%%%%%%%%%%%%%%%%%%%%%%
%% section 22.02 %%%%%%%%%%%%%%%%%%%%%%%%%%%%%%%%%%%%%%%%%
%%%%%%%%%%%%%%%%%%%%%%%%%%%%%%%%%%%%%%%%%%%%%%%%%%%%%%%%%%
\modHeadsection{\CurvatureCenter が基準の場合}
\TopCurvatureCenter T$_{R_\mathrm c}'$と\KeywayCenter M$'$との$x$方向の差は、
\begin{align*}
  \sqrt{R_\mathrm c^2-\ab(f_\mathrm T-\bar\kappa_w)^2}
  -\sqrt{R_\mathrm c^2-f_\mathrm T^2}\ .
\end{align*}
実際の測定では、\TopCurvatureCenter T$_{R_\mathrm c}'$ではなく\index{トップたんのそとがわちゅうしん@トップ端の外側中心}トップ端の外側中心T$_\mathrm c'$が測定されるので、T$_\mathrm c'$とM$'$との$x$方向の差を考える必要がある。
すなわち、
%% label{eq:difTopKeywayCenter}
\begin{align}
  \label{eq:difTopKeywayCenter}
  \sqrt{R_\mathrm c^2-\ab(f_\mathrm T-\bar\kappa_w)^2}
  -\frac{\sqrt{R_\mathrm o^2-f_\mathrm T^2}+\sqrt{R_\mathrm i^2-f_\mathrm T^2}}2\ .
\end{align}


%%%%%%%%%%%%%%%%%%%%%%%%%%%%%%%%%%%%%%%%%%%%%%%%%%%%%%%%%%
%% subsection 23.02.1 %%%%%%%%%%%%%%%%%%%%%%%%%%%%%%%%%%%%
%%%%%%%%%%%%%%%%%%%%%%%%%%%%%%%%%%%%%%%%%%%%%%%%%%%%%%%%%%
\subsection{\Spacer を用いた場合の\KeywayCenter(\CurvatureCenter 基準)}
\KeywayCenter M$'$が\index{ワーク}ワークの\CenterCurvatureLine 上にある場合、\TableCenter Pを原点とした$x$座標は、\pageeqref{eq:spacerTRc}より、
\begin{align*}
  -\Delta_x+\sqrt{R_\mathrm c^2-(f_\mathrm T-\bar\kappa_w)^2}+\frac{\delta_\mathrm s}2
  -\sqrt{R_\mathrm i'^2-\frac{\delta_\mathrm s^2+(2\bar l)^2}4}\frac{2\bar l}{\sqrt{\delta_\mathrm s^2+\ab(2\bar l)^2}}
\end{align*}
となる。
なお実際の作業では、簡単のため、\TopEndFace の外側中心T$_\mathrm c'$を測定し、それを\TopCurvatureCenter T$_{R_\mathrm c}'$とみなして\KeywayCenter M$'$の位置を計算することが多い。
実測した外側中心の$X$・$Y$座標$G_{\mathrm Tx}$, $G_{\mathrm Ty}$を\TopCurvatureCenter のそれとみなすと、機内における\KeywayCenter M$'$の位置は、\TableCenter Pを原点として、
\begin{align*}
  \ab(
    G_{\mathrm Tx}
    +\sqrt{R_\mathrm c^2-(f_\mathrm T-\bar\kappa_w)^2}
    -\sqrt{R_\mathrm c^2-f_\mathrm T^2}\ ,\
    G_{\mathrm Ty}~,~
    f_\mathrm T'-\bar\kappa_w
  )\ .
\end{align*}
\TopCurvatureCenter とみなさずに正確に求めるなら、これに\pageeqref{eq:TRc-Tc}を引けばよい。
その場合の$X$座標は、
%% label{eq:Mreal}
\begin{align}
  \label{eq:Mreal}
  G_{\mathrm Tx}
  +\sqrt{R_\mathrm c^2-(f_\mathrm T-\bar\kappa_w)^2}
  -\frac{\sqrt{R_\mathrm o^2-f_\mathrm T^2}+\sqrt{R_\mathrm i^2-f_\mathrm T^2}}2\ .
\end{align}


%%%%%%%%%%%%%%%%%%%%%%%%%%%%%%%%%%%%%%%%%%%%%%%%%%%%%%%%%%
%% subsection 04.2.2 %%%%%%%%%%%%%%%%%%%%%%%%%%%%%%%%%%%%%
%%%%%%%%%%%%%%%%%%%%%%%%%%%%%%%%%%%%%%%%%%%%%%%%%%%%%%%%%%
\subsection{\Table を傾けた場合の\KeywayCenter(\CurvatureCenter 基準)}
\KeywayCenter M$'$が\CenterCurvatureLine 上にある場合、\TableCenter Pを原点とした$x$座標は、\pageeqref{eq:tableTRc}より、
\begin{align*}
  \sqrt{R_\mathrm c^2-(f_\mathrm T-\bar\kappa_w)^2}
  -\Delta_x'\cos\theta\ .
\end{align*}
実測した\index{そとがわちゅうしん@外側中心}外側中心の$X$, $Y$座標$G_{\mathrm Tx}$, $G_{\mathrm Ty}$を\TopCurvatureCenter のそれとみなした場合とそうでない場合は、\pageeqref{eq:Mreal}で与えられる。



\clearpage
%%%%%%%%%%%%%%%%%%%%%%%%%%%%%%%%%%%%%%%%%%%%%%%%%%%%%%%%%%
%% section 04.2 %%%%%%%%%%%%%%%%%%%%%%%%%%%%%%%%%%%%%%%%%%
%%%%%%%%%%%%%%%%%%%%%%%%%%%%%%%%%%%%%%%%%%%%%%%%%%%%%%%%%%
\modHeadsection{\OutcutCenter が基準の場合}
\KeywayCenter M$'$が\TopOutcutCenter 線上にある場合、機内におけるその位置座標は、
\begin{align*}
  \ab(
    -\mathcal G_{Bx}+T_x\ ,\
    \mathcal G_{By}\ ,\
    f_\mathrm T'-\bar\kappa_w
  ) \qquad
  \text{または}\qquad
  \ab(
    \mathcal G_{Bx}\ ,\
    \mathcal G_{By}\ ,\
    f_\mathrm T'-\bar\kappa_w
  )\ .
\end{align*}
ただし、前者は\BottomOutcut を基準にした(ボトム基準の\CenterlineEndFaceDif がある)場合であり、後者は\TopOutcut を基準にした場合である。



%\clearpage
%%%%%%%%%%%%%%%%%%%%%%%%%%%%%%%%%%%%%%%%%%%%%%%%%%%%%%%%%%
%% section 04.3 %%%%%%%%%%%%%%%%%%%%%%%%%%%%%%%%%%%%%%%%%%
%%%%%%%%%%%%%%%%%%%%%%%%%%%%%%%%%%%%%%%%%%%%%%%%%%%%%%%%%%
\modHeadsection{\AsideKeywayDepth が基準の場合}


%%%%%%%%%%%%%%%%%%%%%%%%%%%%%%%%%%%%%%%%%%%%%%%%%%%%%%%%%%
%% subsection 4.3.1 %%%%%%%%%%%%%%%%%%%%%%%%%%%%%%%%%%%%%%
%%%%%%%%%%%%%%%%%%%%%%%%%%%%%%%%%%%%%%%%%%%%%%%%%%%%%%%%%%
\subsection{\Outcut のない場合}
トップ側に\Outcut がなく、\textbf{\AsideKeywayDepth}$\kappa_d$が指定されている場合を考える。
\KeywayCenter の位置の$X$座標は、\TableCenter Pを\expandafterindex{げんてん(\yomiKeywayMilling)@原点(\nameKeywayMilling)}原点として、
%% label{eq:mizocenterA}
\begin{align}
  \label{eq:mizocenterA}
  \sqrt{R_\mathrm o^2-(f_\mathrm T-\bar\kappa_w)^2}-\kappa_d-\frac{W_{mx}}2
  -\Delta_x'
\end{align}
で与えられる。
ここで、$W_{mx}$は\KeywayACOD を表す。
なお実際の作業では、\index{Aがわがいめん@A側外面}A側外面の\KeywayCenter に相当する箇所を直接測定し、その位置を基準として\expandafterindex{げんてん(\yomiKeywayMilling)@原点(\nameKeywayMilling)}原点を割り出す。
\TopEndFace における中心の$X$座標(\index{じっそくち@実測値}実測値)$G_{\mathrm Tx}$がわかっている場合、\KeywayCenter の$X$座標$G_{mx}$は\pageeqref{eq:Mreal}で与えられ、また\KeywayCenter に対する\index{Aがわがいめん@A側外面}A側外面と$G_{\mathrm Tx}$との差(の絶対値)は、
%% label{eq:mizocenterA}
\begin{align}
  \label{eq:mizocenterAd}
  \frac{W_{mx}}2+\kappa_d
  +\sqrt{R_\mathrm c^2-(f_\mathrm T-\bar\kappa_w)^2}
  -\frac{\sqrt{R_\mathrm o^2-f_\mathrm T^2}+\sqrt{R_\mathrm i^2-f_\mathrm T^2}}2\ .
\end{align}


%%%%%%%%%%%%%%%%%%%%%%%%%%%%%%%%%%%%%%%%%%%%%%%%%%%%%%%%%%
%% subsection 4.3.2 %%%%%%%%%%%%%%%%%%%%%%%%%%%%%%%%%%%%%%
%%%%%%%%%%%%%%%%%%%%%%%%%%%%%%%%%%%%%%%%%%%%%%%%%%%%%%%%%%
\subsection{\Outcut のある場合}
トップ側に\Outcut があり、かつ\AsideKeywayDepth$\kappa_d$が指定されている場合を考える。
\TopOutcutCenter の実測値を$\mathcal G_{\mathrm Tx}$とすると、\KeywayCenter$G_{mx}$との差($G_{mx}-\mathcal G_{\mathrm Tx}$)は
%% label{eq:mizocenterAG}
\begin{align}
  \label{eq:mizocenterAG}
  \frac{\mathfrak W_x}2-\kappa_d-\frac{W_{mx}}2\ .
\end{align}



\clearpage
%%%%%%%%%%%%%%%%%%%%%%%%%%%%%%%%%%%%%%%%%%%%%%%%%%%%%%%%%%
%% section 36.05 %%%%%%%%%%%%%%%%%%%%%%%%%%%%%%%%%%%%%%%%%
%%%%%%%%%%%%%%%%%%%%%%%%%%%%%%%%%%%%%%%%%%%%%%%%%%%%%%%%%%
\modHeadsection{測定上の\KeywayDepth}
ここでは特に\AsideKeywayDepth$\kappa_d$に限って話を進める。
トップ側に\Outcut がある場合、$\kappa_d$は\index{Aがわがいさくめん@A側外削面}A側外削面と\Keyway のA側面との差で与えられる。
しかし\Outcut がない場合、つまり\index{Aがわがいめん@A側外面}A側外面に湾曲がある場合は、$\kappa_d$の値は自明ではない。


%%%%%%%%%%%%%%%%%%%%%%%%%%%%%%%%%%%%%%%%%%%%%%%%%%%%%%%%%%
%% subsection 36.05.01 %%%%%%%%%%%%%%%%%%%%%%%%%%%%%%%%%%%
%%%%%%%%%%%%%%%%%%%%%%%%%%%%%%%%%%%%%%%%%%%%%%%%%%%%%%%%%%
\subsection{図面上の\KeywayDepth}
単純に考えると、\index{ずめん@図面}図面上において\AsideKeywayDepth が$\kappa_d$のとき、これは\KeywayCenter の位置におけるA側外面から\EndFace と水平な方向に$\kappa_d$という意味で与えられる。
このとき、\Keyway のA面側の水平方向の位置は、\pageeqref{eq:mizocenterA}より、
\begin{align*}
  \sqrt{R_\mathrm o^2-(f_\mathrm T-\bar\kappa_w)^2}-\kappa_d-\Delta_x'\ .
\end{align*}
\KeywayCenter の$X$座標$G_{mx}$が与えられている場合は、
\begin{align*}
  G_{mx}+\frac{W_{mx}}2\ .
\end{align*}


%%%%%%%%%%%%%%%%%%%%%%%%%%%%%%%%%%%%%%%%%%%%%%%%%%%%%%%%%%
%% subsection 36.05.2 %%%%%%%%%%%%%%%%%%%%%%%%%%%%%%%%%%%%
%%%%%%%%%%%%%%%%%%%%%%%%%%%%%%%%%%%%%%%%%%%%%%%%%%%%%%%%%%
\subsection{測定上の傾き}
通常、実測には\index{マイクロメータ}マイクロメータ(\index{デプスゲージ}デプスゲージ)が用いられる。
このとき、測定は\index{そくていき@測定器}測定器を湾曲に沿った形にして行われる。
したがって、その測定値は\KeywayWidth の両端に対する外面の傾斜だけ傾いた形で与えられる。
\KeywayWidth の両端に対する外面の$XZ$位置は、\KeywayCenter の$X$座標を$G_{mx}$として、
\begin{align*}
  \text{トップ側:}&~~
  \ab(
  G_{mx}+\frac{W_x}2
  -\sqrt{R_\mathrm o^2-(f_\mathrm T-\bar\kappa_w)^2}
  +\sqrt{R_\mathrm o^2-(f_\mathrm T-\kappa_p)^2}~,~~
  f_\mathrm T-\kappa_p
  )\ ,\\
  \text{ボトム側:}&~~
  \ab(
  G_{mx}+\frac{W_x}2
  +\sqrt{R_\mathrm o^2-(f_\mathrm T-\kappa_p-\kappa_w)^2}
  -\sqrt{R_\mathrm o^2-(f_\mathrm T-\bar\kappa_w)^2}~,~~
  f_\mathrm T-\kappa_p-\kappa_w
  )\ .
\end{align*}
またその差は、
\begin{align*}
  \ab(
  \sqrt{R_\mathrm o^2-(f_\mathrm T-\kappa_p)^2}
  -\sqrt{R_\mathrm o^2-(f_\mathrm T-\kappa_p-\kappa_w)^2}~,~~
  \kappa_w
  )\ .
\end{align*}
したがって、測定の際の傾斜の角度(\KeywayDepthMeasurementAngle)$\zeta$ ($> 0$)は、
%% label{eq:angleZeta}
\begin{align}
  \label{eq:angleZeta}
  \tan\zeta
  = \frac{\sqrt{R_\mathrm o^2-\ab(f_\mathrm T-\kappa_p-\kappa_w)^2}
          -\sqrt{R_\mathrm o^2-\ab(f_\mathrm T-\kappa_p)^2}}
         {\kappa_w}\ .
\end{align}


\clearpage
%%%%%%%%%%%%%%%%%%%%%%%%%%%%%%%%%%%%%%%%%%%%%%%%%%%%%%%%%%
%% subsection 36.05.03 %%%%%%%%%%%%%%%%%%%%%%%%%%%%%%%%%%%
%%%%%%%%%%%%%%%%%%%%%%%%%%%%%%%%%%%%%%%%%%%%%%%%%%%%%%%%%%
\subsection{測定における\KeywayDepth 補正}%\label{subsec:keywayDepthDif}
測定の際は、測定器を\Keyway のトップ側およびボトム側に寄せる形で測定し、その平均値を測定値としている。
つまり、トップ側に寄っているときは測定器の針の付け根が\KeywayPos にあるところ、ボトム側に寄っているときは針の先端が\KeywayWidth ボトム側にあるところで測定を行っている。
この平均値を$\kappa_d'$とする。

トップ側およびボトム側の\AsideKeywayDepth をそれぞれ$\kappa_s$, $\kappa_l$とすると、
\begin{align*}
  \kappa_l-\kappa_s = \kappa_w\tan\zeta \quad,\qquad
  \kappa_d' = \frac{\kappa_l\cos\zeta+\kappa_s\sec\zeta}2
\end{align*}
であり、
\begin{align*}
  \kappa_l = \frac{2\kappa_d'\cos\zeta+\kappa_w\tan\zeta}{1+\cos^2\zeta}~~, \quad
  \kappa_s = \frac{2\kappa_d'-\kappa_w\sin\zeta}{1+\cos^2\zeta}\cos\zeta\ .
\end{align*}
%%%%%%%%%%%%%%%%%%%%%%%%%%%%%%%%%%%%%%%%%%%%%%%%%%%%%%%%%%
%% hosoku %%%%%%%%%%%%%%%%%%%%%%%%%%%%%%%%%%%%%%%%%%%%%%%%
%%%%%%%%%%%%%%%%%%%%%%%%%%%%%%%%%%%%%%%%%%%%%%%%%%%%%%%%%%
\begin{hosoku}
$\kappa_l = a+b$, $\kappa_s = a-b$とすると、
\begin{align*}
  a = \frac{\kappa_l+\kappa_s}2~~, \quad
  b = \frac{\kappa_l-\kappa_s}2\,\ab(= \frac12\kappa_w\tan\zeta)
\end{align*}
であり、
\begin{align*}
  2\kappa_d' = a(\cos\zeta+\sec\zeta)+b(\cos\zeta-\sec\zeta)
  \quad\longrightarrow\quad
  a = \frac{2\kappa_d'\cos\zeta+b\sin^2\zeta}{1+\cos^2\zeta}\ .
\end{align*}
また、$\kappa_l$, $\kappa_s$はその平均値$a$からその差の半分$b$の和あるいは差として得られる。
\begin{align*}
  \kappa_l
  &= \frac{2\kappa_d'\cos\zeta+\frac{\kappa_w\tan\zeta}2\sin^2\zeta}{1+\cos^2\zeta}+\frac12\kappa_w\tan\zeta
   = \frac{2\kappa_d'\cos\zeta+\kappa_w\tan\zeta}{1+\cos^2\zeta}\ ,\\
  \kappa_s
  &= \frac{2\kappa_d'\cos\zeta+\frac{\kappa_w\tan\zeta}2\sin^2\zeta}{1+\cos^2\zeta}-\frac12\kappa_w\tan\zeta
   = \frac{2\kappa_d'-\kappa_w\sin\zeta}{1+\cos^2\zeta}\cos\zeta\ .
\end{align*}
\end{hosoku}
%%%%%%%%%%%%%%%%%%%%%%%%%%%%%%%%%%%%%%%%%%%%%%%%%%%%%%%%%%
%%%%%%%%%%%%%%%%%%%%%%%%%%%%%%%%%%%%%%%%%%%%%%%%%%%%%%%%%%
%%%%%%%%%%%%%%%%%%%%%%%%%%%%%%%%%%%%%%%%%%%%%%%%%%%%%%%%%%
$\kappa_d$および$\kappa_s$の差は
\begin{align*}
  \kappa_d-\kappa_s
  &= \sqrt{R_\mathrm o^2-(f_\mathrm T-\bar\kappa_w)^2}
     -\sqrt{R_\mathrm o^2-(f_\mathrm T-\kappa_p)^2}
\end{align*}
なので、
\begin{subequations}
%% label{eq:keydepthDif1}
\begin{align}
  \label{eq:keydepthDif1}
  \kappa_d
  &= \frac{2\kappa_d'-\kappa_w\sin\zeta}{1+\cos^2\zeta}\cos\zeta
     +\sqrt{R_\mathrm o^2-(f_\mathrm T-\bar\kappa_w)^2}
     -\sqrt{R_\mathrm o^2-(f_\mathrm T-\kappa_p)^2}\ .
\end{align}
あるいは、
%% label{eq:keydepthDif2}
\begin{align}
  \label{eq:keydepthDif2}
  \kappa_d'
  &= \frac{1+\cos^2\zeta}{2\cos\zeta}
     \ab\{
     \kappa_d
     -\sqrt{R_\mathrm o^2-(f_\mathrm T-\bar\kappa_w)^2}
     +\sqrt{R_\mathrm o^2-(f_\mathrm T-\kappa_p)^2}
     \}
     +\frac12\kappa_w\sin\zeta\ .
\end{align}
\end{subequations}
したがって、$\kappa_d$を\Drawing 上の数値とする場合は\pageeqref{eq:keydepthDif2}を、$\kappa_d'$を\Drawing 上の数値とする場合は、\pageeqref{eq:keydepthDif1}を用いて補正すればよい。

\clearpage
~\vfill
%%%%%%%%%%%%%%%%%%%%%%%%%%%%%%%%%%%%%%%%%%%%%%%%%%%%%%%%%%
%% Column %%%%%%%%%%%%%%%%%%%%%%%%%%%%%%%%%%%%%%%%%%%%%%%%
%%%%%%%%%%%%%%%%%%%%%%%%%%%%%%%%%%%%%%%%%%%%%%%%%%%%%%%%%%
\begin{\Columnname}{測定における\KeywayDepth 補正の\expandafterindex{きんじけいさん(\yomiKeywayDepth ほせい)@近似計算(\nameKeywayDepth 補正)}近似計算}
\pageeqref{eq:angleZeta}および\pageeqref{eq:keydepthDif2}より、$R\to\infty$ ($R^{-1}\to0$)に対して、
\begin{align*}
  \tan\zeta \xlongrightarrow{R\to\infty} 0~, \quad
  \kappa_d' \xlongrightarrow{R\to\infty} \kappa_d\ .
\end{align*}
もう少し詳しく見ると、\index{テイラーてんかい@テイラー展開}テイラー展開(\index{マクローリンてんかい@マクローリン展開}マクローリン展開)\pageautoref{formula:taylorexpansion}より、
\begin{align*}
  \tan\zeta
  = \frac{R_\mathrm o}{\kappa_w}
     \ab\{
     \sqrt{1-\ab(\frac{f_\mathrm T-\kappa_p-\kappa_w}{R_\mathrm o})^2}
     -\sqrt{1-\ab(\frac{f_\mathrm T-\kappa_p}{R_\mathrm o})^2}
     \}
  = \frac{f_\mathrm T-\bar\kappa_w}{R_\mathrm o}+o\ab(R_\mathrm o^{-3})
\end{align*}
であり、これより、
\begin{align*}
  \sin\zeta = \frac{f_\mathrm T-\bar\kappa_w}{R_\mathrm o}+o\ab(R_\mathrm o^{-3})~~, \quad
  \cos\zeta = 1-\frac12\ab(\frac{f_\mathrm T-\bar\kappa_w}{R_\mathrm o})^2
              +o\ab(R_\mathrm o^{-4})\ .
\end{align*}
したがって、\pageeqref{eq:keydepthDif1}より、
\begin{align*}
  \kappa_d
  &= \kappa_d'-\frac{\kappa_w}2\frac{f_\mathrm T-\bar\kappa_w}{R_\mathrm o}
     +R_\mathrm o
      \ab\{
      \sqrt{1-\ab(\frac{f_\mathrm T-\bar\kappa_w}{R_\mathrm o})^2}
      -\sqrt{1-\ab(\frac{f_\mathrm T-\kappa_p}{R_\mathrm o})^2}
      \}\\
  &= \kappa_d'+\frac{\kappa_w^2}{8R_\mathrm o}
     +o\ab(R_\mathrm o^{-3})\ .
\end{align*}
これから、$\kappa_d > \kappa_d'$であることもわかる。
\end{\Columnname}
%%%%%%%%%%%%%%%%%%%%%%%%%%%%%%%%%%%%%%%%%%%%%%%%%%%%%%%%%%
%%%%%%%%%%%%%%%%%%%%%%%%%%%%%%%%%%%%%%%%%%%%%%%%%%%%%%%%%%
%%%%%%%%%%%%%%%%%%%%%%%%%%%%%%%%%%%%%%%%%%%%%%%%%%%%%%%%%%



\clearpage
%%%%%%%%%%%%%%%%%%%%%%%%%%%%%%%%%%%%%%%%%%%%%%%%%%%%%%%%%%
%% section 36.06 %%%%%%%%%%%%%%%%%%%%%%%%%%%%%%%%%%%%%%%%%
%%%%%%%%%%%%%%%%%%%%%%%%%%%%%%%%%%%%%%%%%%%%%%%%%%%%%%%%%%
\modHeadsection{\CsideKeywayDepth}
\CsideKeywayDepth を測定する際は、ボトム側に測定器(\index{デプスゲージ}デプスゲージ)を当てて測定を行うのが主である。

測定の際の傾斜の角度(\KeywayDepthMeasurementAngle)$\zeta_\textrm C$ ($> 0$)は、
%% label{eq:angleZetaC}
\begin{align}
%  \label{eq:angleZetaC}
  -\tan\zeta_\textrm C
  = \frac{\sqrt{R_\mathrm i^2-\ab(f_\mathrm T-\kappa_p-\kappa_w)^2}
          -\sqrt{R_\mathrm i^2-\ab(f_\mathrm T-\kappa_p)^2}}
         {\kappa_w}\ .
\end{align}
トップ側およびボトム側の\CsideKeywayDepth をそれぞれ$\kappa_{l\textrm C}$, $\kappa_{s\textrm C}$とすると、
\begin{align*}
  \kappa_{l\textrm C}-\kappa_{s\textrm C} = -\kappa_w\tan\zeta_\textrm C \quad, \qquad
  \kappa_{d\textrm C}' = \frac{\kappa_{s\textrm C}\cos\zeta_\textrm C+\kappa_{l\textrm C}\sec\zeta_C}2\ .
\end{align*}
ここで$\kappa_{d\textrm C}'$は$\kappa_d'$と同様に平均値を示す。
これより、
\begin{align*}
  \kappa_{s\textrm C}
  = \frac{2\kappa_{d\textrm C}'\cos\zeta_\textrm C-\kappa_w\tan\zeta_\textrm C}{1+\cos^2\zeta_\textrm C}
  = \kappa_{d\textrm C}'-\frac{\kappa_w}2\frac{f_\textrm T-\bar\kappa_w}{R_i}
    +o\ab(R_i^{-2})\ .
\end{align*}



\clearpage
%%%%%%%%%%%%%%%%%%%%%%%%%%%%%%%%%%%%%%%%%%%%%%%%%%%%%%%%%%
%% section 36.06 %%%%%%%%%%%%%%%%%%%%%%%%%%%%%%%%%%%%%%%%%
%%%%%%%%%%%%%%%%%%%%%%%%%%%%%%%%%%%%%%%%%%%%%%%%%%%%%%%%%%
\modHeadsection{\Keyway のコーナー}
ここでは\Keyway のコーナーの種類に着目して話を進める。


%%%%%%%%%%%%%%%%%%%%%%%%%%%%%%%%%%%%%%%%%%%%%%%%%%%%%%%%%%
%% subsection 36.06.01 %%%%%%%%%%%%%%%%%%%%%%%%%%%%%%%%%%%
%%%%%%%%%%%%%%%%%%%%%%%%%%%%%%%%%%%%%%%%%%%%%%%%%%%%%%%%%%
\subsection{\Keyway:コーナーRの場合}
\Keyway のコーナーRの大きさを$R_\mathrm K$とする。
このとき、AC方向およびBD方向の\KeywayWidth に対して、コーナーを除いた部分(直線部分)の長さは、それぞれ
\begin{align*}
  \text{AC方向:}\quad W_{\mathrm mx}-2R_\mathrm K~, \qquad
  \text{BD方向:}\quad W_{\mathrm my}-2R_\mathrm K\ .
\end{align*}


%%%%%%%%%%%%%%%%%%%%%%%%%%%%%%%%%%%%%%%%%%%%%%%%%%%%%%%%%%
%% subsection 36.06.02 %%%%%%%%%%%%%%%%%%%%%%%%%%%%%%%%%%%
%%%%%%%%%%%%%%%%%%%%%%%%%%%%%%%%%%%%%%%%%%%%%%%%%%%%%%%%%%
\subsection{\Keyway:コーナーC(8角形)の場合}
\Keyway のコーナーCの大きさを$C_\mathrm K$とする。
このとき、AC方向およびBD方向の\KeywayWidth に対して、コーナーを除いた部分の長さは、それぞれ
\begin{align*}
  \text{AC方向:}\quad W_{\mathrm mx}-2C_\mathrm K~, \qquad
  \text{BD方向:}\quad W_{\mathrm my}-2C_\mathrm K\ .
\end{align*}


%%%%%%%%%%%%%%%%%%%%%%%%%%%%%%%%%%%%%%%%%%%%%%%%%%%%%%%%%%
%% subsection 36.06.03 %%%%%%%%%%%%%%%%%%%%%%%%%%%%%%%%%%%
%%%%%%%%%%%%%%%%%%%%%%%%%%%%%%%%%%%%%%%%%%%%%%%%%%%%%%%%%%
\subsection{\Keyway:8角形コーナーRの場合}
\Keyway がコーナーC(大きさ$C_\mathrm K$)の8角形であり、かつその頂点がコーナーR(大きさ$R_\mathrm{K'}$)で丸めてある場合を考える。

%%%%%%%%%%%%%%%%%%%%%%%%%%%%%%%%%%%%%%%%%%%%%%%%%%%%%%%%%%
%% subsubsection 36.06.03.01 %%%%%%%%%%%%%%%%%%%%%%%%%%%%%
%%%%%%%%%%%%%%%%%%%%%%%%%%%%%%%%%%%%%%%%%%%%%%%%%%%%%%%%%%
\subsubsection{各辺の長さ}
このとき、AC方向およびBD方向の\KeywayWidth に対して、コーナーを除いた部分の長さは、それぞれ
\begin{align*}
  \text{AC方向:}\quad W_{\mathrm mx}-2C_\mathrm K-2\sqrt2\left(1-\frac1{\sqrt2}\right)R_\mathrm{K'}~, \quad
  \text{BD方向:}\quad W_{\mathrm my}-2C_\mathrm K-2\sqrt2\left(1-\frac1{\sqrt2}\right)R_\mathrm{K'}\ .
\end{align*}
コーナーC部の直線部分に対する$X$, $Y$方向の長さは、それぞれ
\begin{align*}
  C_\mathrm K-2\left(1-\frac1{\sqrt2}\right)R_\mathrm{K'}\ .
\end{align*}

%%%%%%%%%%%%%%%%%%%%%%%%%%%%%%%%%%%%%%%%%%%%%%%%%%%%%%%%%%
%% subsubsection 36.06.03.01 %%%%%%%%%%%%%%%%%%%%%%%%%%%%%
%%%%%%%%%%%%%%%%%%%%%%%%%%%%%%%%%%%%%%%%%%%%%%%%%%%%%%%%%%
\subsubsection{コーナーRの中心および接点の相対位置}
例として、工具から見て右上のコーナーCにおける、右下($X+Y-$)側の頂点に着目する。
このときコーナーRにおける、右下($X+Y-$)側のコーナーR接点を(0, 0)とすると、コーナーRの中心および左上($X-Y+$)側のコーナーR接点の位置は、それぞれ
\begin{align*}
  \text{中心:}\quad \ab(-R_\mathrm{K'},~0)~,\qquad
  \text{左上側接点}\quad \ab(-\left(1-\frac1{\sqrt2}\right)R_\mathrm{K'}~,~\frac{R_\mathrm{K'}}{\sqrt2})\ .
\end{align*}
同様に、工具から見て右上のコーナーCにおける、右上($X-Y+$)側の頂点に着目する。
このときコーナーRにおける、右下($X+Y-$)側のコーナーR接点を(0, 0)とすると、コーナーRの中心および左上($X-Y+$)側のコーナーR接点の位置は、それぞれ
\begin{align*}
  \text{中心:}\quad \ab(-\frac{R_\mathrm{K'}}{\sqrt2}~,~-\frac{R_\mathrm{K'}}{\sqrt2})~,\qquad
  \text{左上側接点}\quad \ab(-\frac{R_\mathrm{K'}}{\sqrt2}~,~\left(1-\frac1{\sqrt2}\right)R_\mathrm{K'})\ .
\end{align*}
その他のコーナーRについても同様であり、これを適当に回転させたものとなる。





%%%%%%%%%%%%%%%%%%%%%%%%%%%%%%%%%%%%%%%%%%%%%%%%%%%%%%%%%%
%%           %%%%%%%%%%%%%%%%%%%%%%%%%%%%%%%%%%%%%%%%%%%%%
%% chapter 5 %%%%%%%%%%%%%%%%%%%%%%%%%%%%%%%%%%%%%%%%%%%%%
%%           %%%%%%%%%%%%%%%%%%%%%%%%%%%%%%%%%%%%%%%%%%%%%
%%%%%%%%%%%%%%%%%%%%%%%%%%%%%%%%%%%%%%%%%%%%%%%%%%%%%%%%%%
\modHeadchapter{C面取の幾何}
\input{RfCPN_p5_Analytical_Calculation_chapter/c_RfCPN_Cmentori}




%%%%%%%%%%%%%%%%%%%%%%%%%%%%%%%%%%%%%%%%%%%%%%%%%%%%%%%%%%
%%            %%%%%%%%%%%%%%%%%%%%%%%%%%%%%%%%%%%%%%%%%%%%
%% chapter 24 %%%%%%%%%%%%%%%%%%%%%%%%%%%%%%%%%%%%%%%%%%%%
%%            %%%%%%%%%%%%%%%%%%%%%%%%%%%%%%%%%%%%%%%%%%%%
%%%%%%%%%%%%%%%%%%%%%%%%%%%%%%%%%%%%%%%%%%%%%%%%%%%%%%%%%%
\modHeadchapter{R面取の幾何}
%!TEX root = ../RPA_for_Creating_Program_Note.tex



ここでは主に、端面外側および内側の\index{Rめんとり@R面取}\textbf{R面取}について考える。
R面取加工では、\index{ボールエンドミル}ボールエンドミルの工具を用いて行われるのが一般的である。
ボールエンドミルは\DMname には搭載しない予定であり、基本的にはR面取はハンドグラインダーを用いて人の手で行う。
しかし面取の寸法が大きくなると、削る体積は3次関数的に増加する。
そのため作業者にかかる負担も大きくなり、マシニング外での作業時間も大きく増えることになる。

そこで、C面取加工で用いられる\index{テーパエンドミル}テーパエンドミルを用いて、R面取部分の一部を切削することを考える。

%%%%%%%%%%%%%%%%%%%%%%%%%%%%%%%%%%%%%%%%%%%%%%%%%%%%%%%%%%
%% section 25.1 %%%%%%%%%%%%%%%%%%%%%%%%%%%%%%%%%%%%%%%%%%
%%%%%%%%%%%%%%%%%%%%%%%%%%%%%%%%%%%%%%%%%%%%%%%%%%%%%%%%%%
\modHeadsection{テーパエンドミルによる面取}
トップ端面およびボトム端面における外側R面取の大きさを、それぞれ$r_\mathrm{To}$, $r_\mathrm{Bo}$とする。
このとき、片角$\xi_\mathrm e$のテーパエンドミルに対して、
\begin{align*}
  c_\mathrm{To} &= r_\mathrm{To}\left(1+\cot\xi_\mathrm e-\csc\xi_\mathrm e\right)\\
  c_\mathrm{Bo} &= r_\mathrm{Bo}\left(1+\cot\xi_\mathrm e-\csc\xi_\mathrm e\right)
\end{align*}
のC面取とみなして加工を行うと、R面取に接する形で加工を行うことができる。
内側R面取の大きさを$r_\mathrm{Ti}$, $r_\mathrm{Bi}$についても同様に、
\begin{align*}
  c_\mathrm{Ti} &= r_\mathrm{Ti}\left(1+\cot\xi_\mathrm e-\csc\xi_\mathrm e\right)\\
  c_\mathrm{Bi} &= r_\mathrm{Bi}\left(1+\cot\xi_\mathrm e-\csc\xi_\mathrm e\right)
\end{align*}
とみなすことができる。
特に、$\xi_\mathrm e = \nicefrac\pi{12}$\,($15^\circ$), $\nicefrac\pi6$\,($30^\circ$), $\nicefrac\pi4$\,($45^\circ$)のとき、それぞれ
\begin{align*}
  c_\mathrm{To} &= \big(3+\sqrt3-\sqrt6-\sqrt2\big)r_\mathrm{To} \sim 0.8683\cdot r_\mathrm{To}\\
  c_\mathrm{To} &= \big(\sqrt3-1\big)r_\mathrm{To} \sim 0.7321\cdot r_\mathrm{To}\\
  c_\mathrm{To} &= \big(2-\sqrt2\big)r_\mathrm{To} \sim 0.5858\cdot r_\mathrm{To}
\end{align*}

%%%%%%%%%%%%%%%%%%%%%%%%%%%%%%%%%%%%%%%%%%%%%%%%%%%%%%%%%%
%% hosoku %%%%%%%%%%%%%%%%%%%%%%%%%%%%%%%%%%%%%%%%%%%%%%%%
%%%%%%%%%%%%%%%%%%%%%%%%%%%%%%%%%%%%%%%%%%%%%%%%%%%%%%%%%%
\begin{hosoku}
原点を中心とする半径$r_\mathrm{To}$の円の第一象限に対し、$y$軸との角度が$\xi_\mathrm e$となる接線を考えればよい。
このとき、原点と接点を結んだ直線の傾きは$\tan\xi_\mathrm e$なので、接線の傾きは$-\cot\xi_\mathrm e$となる。
またこの接線は接点($r_\mathrm{To}\cos\xi_\mathrm e$, $r_\mathrm{To}\sin\xi_\mathrm e$)あるいは(0, $r\sec\xi_\mathrm e$)等を通ることを用いると、接線$y = -\cot\xi_\mathrm e+r_\mathrm{To}\csc\xi_\mathrm e$が得られる。
あとは$x = r_\mathrm{To}$のときの位置と、端点($r_\mathrm{To}$, $r_\mathrm{To}$)との差をみればよい。
\end{hosoku}
%%%%%%%%%%%%%%%%%%%%%%%%%%%%%%%%%%%%%%%%%%%%%%%%%%%%%%%%%%
%%%%%%%%%%%%%%%%%%%%%%%%%%%%%%%%%%%%%%%%%%%%%%%%%%%%%%%%%%
%%%%%%%%%%%%%%%%%%%%%%%%%%%%%%%%%%%%%%%%%%%%%%%%%%%%%%%%%%



\clearpage
%%%%%%%%%%%%%%%%%%%%%%%%%%%%%%%%%%%%%%%%%%%%%%%%%%%%%%%%%%
%% section 24.2 %%%%%%%%%%%%%%%%%%%%%%%%%%%%%%%%%%%%%%%%%%
%%%%%%%%%%%%%%%%%%%%%%%%%%%%%%%%%%%%%%%%%%%%%%%%%%%%%%%%%%
\modHeadsection[中心座標\texorpdfstring{$X$}{X}の移動]{中心座標$X$の移動}
C面取とみなして加工を行うため、外削のある場合を除いて$X$方向への移動を考える必要がある。
移動の大きさは、通常のC面取の場合と同じである。
\begin{align*}
  \text{外面取 トップ側:}&~~
  \sqrt{R_\mathrm c^2-\left(f_\mathrm T-c_\mathrm{To}\right)^2}-\sqrt{R_\mathrm c^2-f_\mathrm T^2}\ ,\\
  \text{外面取 ボトム側:}&~~
  \sqrt{R_\mathrm c^2-f_\mathrm B^2}-\sqrt{R_\mathrm c^2-\left(f_\mathrm B-c_\mathrm{Bo}\right)^2}\ ,\\
  \text{内面取 トップ側:}&~~
  \sqrt{R_\mathrm c^2-\left(f_\mathrm T-c_\mathrm{Ti}\right)^2}-\sqrt{R_\mathrm c^2-f_\mathrm T^2}\ ,\\
  \text{内面取 ボトム側:}&~~
  \sqrt{R_\mathrm c^2-f_\mathrm B^2}-\sqrt{R_\mathrm c^2-\left(f_\mathrm B-c_\mathrm{Bi}\right)^2}\ .
\end{align*}








%%%%%%%%%%%%%%%%%%%%%%%%%%%%%%%%%%%%%%%%%%%%%%%%%%%%%%%%%%
%%           %%%%%%%%%%%%%%%%%%%%%%%%%%%%%%%%%%%%%%%%%%%%%
%% chapter 6 %%%%%%%%%%%%%%%%%%%%%%%%%%%%%%%%%%%%%%%%%%%%%
%%           %%%%%%%%%%%%%%%%%%%%%%%%%%%%%%%%%%%%%%%%%%%%%
%%%%%%%%%%%%%%%%%%%%%%%%%%%%%%%%%%%%%%%%%%%%%%%%%%%%%%%%%%
\modHeadchapter{座ぐりの幾何}
%!TEX root = ../RPA_for_Creating_Program_Note.tex


\modHeadchapter{端面における座ぐりの幾何}
ここでは主に、トップ端面における\index{ざぐり(トップたん)@座ぐり(トップ端)}\textbf{座ぐり}に関する測定・加工に必要な\expandafterindex{きかがくてきせいしつ(ざぐり)@幾何学的性質(座ぐり)}幾何学的性質を考える。



%%%%%%%%%%%%%%%%%%%%%%%%%%%%%%%%%%%%%%%%%%%%%%%%%%%%%%%%%%
%% section 06.1 %%%%%%%%%%%%%%%%%%%%%%%%%%%%%%%%%%%%%%%%%%
%%%%%%%%%%%%%%%%%%%%%%%%%%%%%%%%%%%%%%%%%%%%%%%%%%%%%%%%%%
\modHeadsection{座ぐりの位置\TBW}
(to be written...)




%%%%%%%%%%%%%%%%%%%%%%%%%%%%%%%%%%%%%%%%%%%%%%%%%%%%%%%%%%
%%           %%%%%%%%%%%%%%%%%%%%%%%%%%%%%%%%%%%%%%%%%%%%%
%% chapter 7 %%%%%%%%%%%%%%%%%%%%%%%%%%%%%%%%%%%%%%%%%%%%%
%%           %%%%%%%%%%%%%%%%%%%%%%%%%%%%%%%%%%%%%%%%%%%%%
%%%%%%%%%%%%%%%%%%%%%%%%%%%%%%%%%%%%%%%%%%%%%%%%%%%%%%%%%%
\modHeadchapter[loC]{\dimple の幾何}
%!TEX root = ../RPA_for_Creating_Program_Note.tex



ここでは主に\expandafterindex{\dimplekana@\dimple}\textbf{\dimple}に関する計測・加工に必要な幾何学的性質を考える。

なお、\dimple の加工は\MMname で行うことはできず、\DMname のみで行う。
また\DMname では、振分長の調整についてスペーサを用いた方法は行わず、テーブルの回転を用いた方法のみで行う方針である。
したがって、スペーサを用いた方法の場合は考慮する必要がない。
そのため以降では、(\dimple に関する計測・加工については)テーブルを$-\theta$だけ回転した場合についてのみを考えることにする。




%%%%%%%%%%%%%%%%%%%%%%%%%%%%%%%%%%%%%%%%%%%%%%%%%%%%%%%%%%
%% section 6.1 %%%%%%%%%%%%%%%%%%%%%%%%%%%%%%%%%%%%%%%%%%%
%%%%%%%%%%%%%%%%%%%%%%%%%%%%%%%%%%%%%%%%%%%%%%%%%%%%%%%%%%
\modHeadsection{\dimple の表記法}
初めに、\dimple に関する表記法を簡単にまとめておく。
なお\dimple はトップ側にあるため、トップ側が工具側に向いているものとして話を進める。
%%%%%%%%%%%%%%%%%%%%%%%%%%%%%%%%%%%%%%%%%%%%%%%%%%%%%%%%%%
%% tcolorbox %%%%%%%%%%%%%%%%%%%%%%%%%%%%%%%%%%%%%%%%%%%%%
%%%%%%%%%%%%%%%%%%%%%%%%%%%%%%%%%%%%%%%%%%%%%%%%%%%%%%%%%%
\begin{tcolorbox}[title={\dimple に関する表記法}, fonttitle=\gtfamily\bfseries, breakable, enhanced jigsaw]
\begin{enumerate}
\item
\subparagraph*{列の数えかた}
\dimple が$m$列あるとき、トップ側から順に1列目, 2列目, …,$m$列目のように数える。

\item
\subparagraph*{列内の個数の数えかた}
各々の列の\dimple は、AC面側については工具側からみて下から順に、BD面については工具側からみて右から順に1つ目,2つ目,…のように数える。

\item
\subparagraph*{\dimple の寸法}
トップ端面から1列目までの距離を$q$, 鉛直・水平方向の\expandafterindex{ピッチ(\dimplekana)@ピッチ(\dimple)}ピッチをそれぞれ$p_z$, $p_x$とし、\expandafterindex{iれつめのながさ(\dimplekana)@$i$列目の長さ(\dimple)}$i$列目の長さをそれぞれ$d_i$とする。

特に、\expandafterindex{きすうれつめのながさ(\dimplekana)@奇数列目の長さ(\dimple)}奇数列目の長さが全て同じ場合はその長さを$d_\mathrm o$, \expandafterindex{ぐうすうれつめのながさ(\dimplekana)@偶数列目の長さ(\dimple)}偶数列目の長さが全て同じ場合はその長さを$d_\mathrm e$とも表記する。
(\pageautoref{fn:generallyDimpleN}および\pageautoref{hosoku:generallyDimpleN}参照)

\item
\subparagraph*{内径テーパ表の寸法}
\index{ないけいテーパひょう@内径テーパ表}内径テーパ表における\index{トップたんからのきょり(ないけいテーパひょう)@トップ端からの距離(内径テーパ表)}トップ端からの距離を$\lambda_i$ ($i = 0$, $1$, $2$, $\cdots$), それに対するAC・BD側\index{ないけい(ないけいテーパひょう)@内径(内径テーパ表)}内径をそれぞれ$w_{\mathrm Ai}$, $w_{\mathrm Bi}$とする。
(\pageautoref{hosoku:example4taper}参照)

\item
\subparagraph*{内径の(近似)寸法}
トップ端から$\lambda$の位置のAC内径を$w_{\mathrm A\lambda}$と表す。
このとき$w_{\mathrm A\lambda}$は、$\lambda_j \leqq \lambda < \lambda_{j+1}$に対する$w_{\mathrm Aj}$, $w_{\mathrm Aj+1}$の\index{かじゅうさんじゅつへいきん(ないけい)@加重算術平均(内径)}加重算術平均(\index{ウェイトさんじゅつへいきん(ないけい)@ウェイト算術平均(内径)}ウェイト算術平均・\index{おもみつきさんじゅつへいきん(ないけい)@重み付き算術平均(内径)}重み付き算術平均)
\begin{align*}
  w_{\mathrm A\lambda}
  = \frac{(\lambda-\lambda_j)w_{\mathrm Aj+1}+(\lambda_{j+1}-\lambda)w_{\mathrm Aj}}{\lambda_{j+1}-\lambda_j}
  \qquad
  \Big(\lambda_j \leqq \lambda < \lambda_{j+1}\Big)
\end{align*}
とみなすことにする。($w_{\mathrm B\lambda}$についても同様)

\item
\subparagraph*{めっき厚を含めた内径の(近似)寸法}
\index{めっきまくあつ@めっき膜厚}めっき膜厚$\mu$を考慮したAC・BD内径$w'_{\mathrm A\lambda}$, $w'_{\mathrm B\lambda}$をそれぞれ以下のように表す。
\begin{align*}
  w'_{\mathrm A\lambda} \equiv w_{\mathrm A\lambda}+2\mu~, \quad
  w'_{\mathrm B\lambda} \equiv w_{\mathrm B\lambda}+2\mu~.
\end{align*}
\end{enumerate}
\end{tcolorbox}\noindent
%%%%%%%%%%%%%%%%%%%%%%%%%%%%%%%%%%%%%%%%%%%%%%%%%%%%%%%%%%
%%%%%%%%%%%%%%%%%%%%%%%%%%%%%%%%%%%%%%%%%%%%%%%%%%%%%%%%%%
%%%%%%%%%%%%%%%%%%%%%%%%%%%%%%%%%%%%%%%%%%%%%%%%%%%%%%%%%%
このとき$m$列目の\dimple の個数$n_m$は、$n_m = \nicefrac{d_m}{p_x}+1$となる
%% footnote %%%%%%%%%%%%%%%%%%%%%
\footnote{\label{fn:generallyDimpleN}%
たいていの場合、\expandafterindex{きすうれつのこすう(\dimplekana)@奇数列の個数(\dimple)}奇数列の個数は全て同じ数$n_\mathrm o$であり、\expandafterindex{ぐうすうれつのこすう(\dimplekana)@偶数列の個数(\dimple)}偶数列の個数も全て同じ$n_\mathrm e$である。
また$|n_\mathrm o-n_\mathrm d| = 1$である。}。
%%%%%%%%%%%%%%%%%%%%%%%%%%%%%%%%%
%%%%%%%%%%%%%%%%%%%%%%%%%%%%%%%%%%%%%%%%%%%%%%%%%%%%%%%%%%
%% hosoku %%%%%%%%%%%%%%%%%%%%%%%%%%%%%%%%%%%%%%%%%%%%%%%%
%%%%%%%%%%%%%%%%%%%%%%%%%%%%%%%%%%%%%%%%%%%%%%%%%%%%%%%%%%
\begin{hosoku}[label=hosoku:example4taper]
たとえば内径テーパ表の値が25mm\index{ピッチ(ないけいテーパひょう)@ピッチ(内径テーパ表)}ピッチの場合、$\lambda_0=0$, $\lambda_1=25$, $\lambda_2=50$, $\cdots$とし、それぞれのACおよびBD側\index{ないけい(ないけいテーパひょう)@内径(内径テーパ表)}内径を$w_{\mathrm A0}$, $w_{\mathrm A1}$, $w_{\mathrm A2}$, $\cdots$および$w_{\mathrm B0}$, $w_{\mathrm B1}$, $w_{\mathrm B2}$, $\cdots$とする、という意味である。
ここでは離散値である$\lambda_i$を、連続値$\lambda$に(近似的に)置きかえている。
実際、たとえば$\lambda = \lambda_j$のとき$w_{\mathrm Aj} = w_{\mathrm A\lambda}$となることがわかる。
\end{hosoku}\relax
%%%%%%%%%%%%%%%%%%%%%%%%%%%%%%%%%%%%%%%%%%%%%%%%%%%%%%%%%%
%%%%%%%%%%%%%%%%%%%%%%%%%%%%%%%%%%%%%%%%%%%%%%%%%%%%%%%%%%
%%%%%%%%%%%%%%%%%%%%%%%%%%%%%%%%%%%%%%%%%%%%%%%%%%%%%%%%%%
%%%%%%%%%%%%%%%%%%%%%%%%%%%%%%%%%%%%%%%%%%%%%%%%%%%%%%%%%%
%% hosoku %%%%%%%%%%%%%%%%%%%%%%%%%%%%%%%%%%%%%%%%%%%%%%%%
%%%%%%%%%%%%%%%%%%%%%%%%%%%%%%%%%%%%%%%%%%%%%%%%%%%%%%%%%%
\begin{hosoku}
内径テーパ表の\index{ピッチ(ないけいテーパひょう)@ピッチ(内径テーパ表)}ピッチ$\lambda_{i+1}-\lambda_i$は常に一定の場合が多い。
$\lambda_{i+1}-\lambda_i$が$i$について常に一定であれば、$\lambda_j \leqq z < \lambda_{j+1}$となる$j$は、
\begin{align*}
  j = z \bDiv (\lambda_{i+1}-\lambda_i) = \left\lfloor\frac z{\lambda_{i+1}-\lambda_i}\right\rfloor
\end{align*}
のように表すことができる。
\end{hosoku}
%%%%%%%%%%%%%%%%%%%%%%%%%%%%%%%%%%%%%%%%%%%%%%%%%%%%%%%%%%
%%%%%%%%%%%%%%%%%%%%%%%%%%%%%%%%%%%%%%%%%%%%%%%%%%%%%%%%%%
%%%%%%%%%%%%%%%%%%%%%%%%%%%%%%%%%%%%%%%%%%%%%%%%%%%%%%%%%%
%%%%%%%%%%%%%%%%%%%%%%%%%%%%%%%%%%%%%%%%%%%%%%%%%%%%%%%%%%
%% Column %%%%%%%%%%%%%%%%%%%%%%%%%%%%%%%%%%%%%%%%%%%%%%%%
%%%%%%%%%%%%%%%%%%%%%%%%%%%%%%%%%%%%%%%%%%%%%%%%%%%%%%%%%%
\begin{Column}{商$\boldsymbol{\bDiv}$と余り$\boldsymbol{\bmod}$とガウス括弧$\boldsymbol{\lfloor\,\rfloor}$}
\renewcommand\theequation{c\thechapter.\arabic{equation}}
\setcounter{equation}{0}
\paragraph*{$\boldsymbol\bDiv$と$\boldsymbol\bmod$}
割り算の余りを表す記号としては\index{mod(あまり)@$\bmod$(余り)}$\bmod$が広く使われる。
商を表す記号は一般的な数学のテキスト等ではあまり用いられないが、プログラミング言語等では\index{div(しょう)@$\bDiv$(商)}$\bDiv$を用いられることがある。
これに倣って、ここでは商には$\bDiv$, 余りには$\bmod$を用いている。

 一般に、実数$a$, $b$ ($b\neq0$)に対して$a = bq+r$ ($0 \leqq r < |b|$)を満たす整数$q$を\index{しょう(div)@商($\bDiv$)}商、$r$を\index{あまり(mod)@余り($\bmod$)}余りと呼び、このとき$a \bDiv b = q$および$a \bmod b = r$のように表される。
なお、ここでは簡単のため、$q \geqq 0$として考えることにする。
\tcbline*
\paragraph*{ガウス括弧}
\index{ガウスかっこ@ガウス括弧}$\lfloor x\rfloor$は、$x \in R$ に対して$x$を超えない最大の整数。
簡単にいうと、($x > 0$の場合は)小数点以下を切り捨てた整数部分を表す。
\index{ガウスきごう@ガウス記号}ガウス記号, \index{ゆかかんすう@床関数}床関数(floor function)などとも呼ばれる。
\end{Column}




\clearpage
%%%%%%%%%%%%%%%%%%%%%%%%%%%%%%%%%%%%%%%%%%%%%%%%%%%%%%%%%%
%% section 6.2 %%%%%%%%%%%%%%%%%%%%%%%%%%%%%%%%%%%%%%%%%%%
%%%%%%%%%%%%%%%%%%%%%%%%%%%%%%%%%%%%%%%%%%%%%%%%%%%%%%%%%%
\modHeadsection{基本方針}
\dimple の加工における留意事項の1つに、モールドの内面(特にトップ端)と工具が接触してしまう\index{アンダーカット}アンダーカットというものがある
%% footnote %%%%%%%%%%%%%%%%%%%%%
\footnote{\dimple の測定・加工ではとりわけアンダーカットが生じやすい、という意味である。
その他の計測・加工についても当然アンダーカットは十分に生じうる。}。
%%%%%%%%%%%%%%%%%%%%%%%%%%%%%%%%%
特にA面側は工具へ向かう方向に\index{わんきょく@湾曲}湾曲があるため、アンダーカットが生じやすい。
そこで、アンダーカットを避けつつ加工ができるようにするため、モールドをいくらか(湾曲と反対側に)傾けて加工を行う。
その\expandafterindex{かたむきかく(\dimplekana)@傾き角(\dimple)}傾き角$\phi$ ($0 \leqq \phi < \nicefrac\pi2$)について、ここでは次の2点を基準に考えることにする。
\begin{tcolorbox}[title=A面の\dimple, fonttitle=\gtfamily\bfseries]
\begin{enumerate}
\item[a)] A側内面のトップ端点
\item[b)] A側内面の\dimple1列目(トップ端から$q$)の位置
\end{enumerate}
\end{tcolorbox}\noindent
この2点を通る直線と鉛直方向との角度を、傾き角$-\phi$とする
%% footnote %%%%%%%%%%%%%%%%%%%%%
\footnote{振分長の調整に用いたテーブルの\index{かたむきかく(ふりわけちょうせい)@傾き角(振分調整)}傾き角$\theta$と混同しないように注意。}。
%%%%%%%%%%%%%%%%%%%%%%%%%%%%%%%%%
なお、\index{トップたんのACないけい@トップ端のAC内径}トップ端のAC内径は$w'_{\mathrm A0}$で代用してもよいものとする。
このとき$\phi > 0$となる(C面側に傾く)場合は$\phi$だけ傾けて加工を行う。
一方、$\phi \leqq 0$となる(A面側に傾く)場合は、そもそもアンダーカットが生じないので、傾けずにそのまま加工を行うものとする。
%%%%%%%%%%%%%%%%%%%%%%%%%%%%%%%%%%%%%%%%%%%%%%%%%%%%%%%%%%
%% hosoku %%%%%%%%%%%%%%%%%%%%%%%%%%%%%%%%%%%%%%%%%%%%%%%%
%%%%%%%%%%%%%%%%%%%%%%%%%%%%%%%%%%%%%%%%%%%%%%%%%%%%%%%%%%
\begin{hosoku}
ここでは\dimple の工具として、\index{Tスロットカッター}Tスロットカッターを考えている。
しかし、当然ながら\index{こうぐけい@工具径}工具径は有限であるため、いくら適切に傾けたところで限界はある。
ここではその限界として、A側内面のトップ端の$X$座標と、それと最も$X$座標が近い\dimple との($X$方向の)距離を算出する。
そしてそれを工具径と比べることで、どこまでの範囲を加工するかを決定する。
加工できない部分に\dimple がある場合は、別の工具(\index{アングルヘッド}アングルヘッド)を使用して加工を行う。
\end{hosoku}
%%%%%%%%%%%%%%%%%%%%%%%%%%%%%%%%%%%%%%%%%%%%%%%%%%%%%%%%%%
%%%%%%%%%%%%%%%%%%%%%%%%%%%%%%%%%%%%%%%%%%%%%%%%%%%%%%%%%%
%%%%%%%%%%%%%%%%%%%%%%%%%%%%%%%%%%%%%%%%%%%%%%%%%%%%%%%%%%
%%%%%%%%%%%%%%%%%%%%%%%%%%%%%%%%%%%%%%%%%%%%%%%%%%%%%%%%%%
%% Column %%%%%%%%%%%%%%%%%%%%%%%%%%%%%%%%%%%%%%%%%%%%%%%%
%%%%%%%%%%%%%%%%%%%%%%%%%%%%%%%%%%%%%%%%%%%%%%%%%%%%%%%%%%
\begin{Column}{曲率と傾き}
内面A側・C側の\index{わんきょく(ないめん)@湾曲(内面)}湾曲をそれぞれ$\mathcal R_\mathrm o$, $\mathcal R_\mathrm i$とすると、\index{きょくりつ(ないめん)@曲率(内面)}曲率はそれぞれ$\mathcal R_\mathrm o^{-1} < R_\mathrm c^{-1} < \mathcal R_\mathrm i^{-1}$である。
そのため、(トップ側の)A側の$\mathcal R_\mathrm o$を基準にするとより緩やかに、C側の$\mathcal R_\mathrm i$を基準にするとよりきつく傾くことになる。
また、トップ端から($Z$方向に)遠い点を基準にするとより緩やかに、近い点を基準にするとよりきつく傾くことになる。
\end{Column}
%%%%%%%%%%%%%%%%%%%%%%%%%%%%%%%%%%%%%%%%%%%%%%%%%%%%%%%%%%
%%%%%%%%%%%%%%%%%%%%%%%%%%%%%%%%%%%%%%%%%%%%%%%%%%%%%%%%%%
%%%%%%%%%%%%%%%%%%%%%%%%%%%%%%%%%%%%%%%%%%%%%%%%%%%%%%%%%%

以下ではこの傾き角$\phi$と、回転後の\dimple や内面の位置を定量的に与えることを試みる。




\clearpage
%%%%%%%%%%%%%%%%%%%%%%%%%%%%%%%%%%%%%%%%%%%%%%%%%%%%%%%%%6
%% section 6.3 %%%%%%%%%%%%%%%%%%%%%%%%%%%%%%%%%%%%%%%%%%%
%%%%%%%%%%%%%%%%%%%%%%%%%%%%%%%%%%%%%%%%%%%%%%%%%%%%%%%%%%
\modHeadsection{\dimple の位置と傾き角(傾き前)}
\pageeqref{eq:tableTRc}より、テーブルを$-\theta$傾けて振分長の調整を行った場合、テーブル中心Pを原点とした\index{ちゅうしんわんきょくせん@中心湾曲線}中心湾曲線のトップ端における$X$座標は、
\begin{align*}
  R_\mathrm c\cos\alpha_\mathrm c-\varDelta'\cos\theta = \sqrt{R_\mathrm c^2-f_\mathrm T^2}-\varDelta'\cos\theta
\end{align*}
で与えられる。
これは\index{タッチセンサー}タッチセンサーによる\expandafterindex{そくていかいしてん(\dimplekana)@測定開始点(\dimple)}測定の開始点として用いることができる。
一方で、それ以外の作業では、\index{トップたんのないけいちゅうしん@トップ端の内径中心}トップ端における内径の中心座標$g_t$を直接測定するので、それを用いることにする
%% footnote %%%%%%%%%%%%%%%%%%%%%
\footnote{これは\index{ちゅうしんわんきょくせん@中心湾曲線}中心湾曲線上にない点であるが、\index{こうさ@公差}公差の範囲内であるものとして、ここではこれで代用する。}。
%%%%%%%%%%%%%%%%%%%%%%%%%%%%%%%%%
よって、テーブル中心Pを原点とした場合における、\expandafterindex{\dimplekana1れつめ@\dimple1列目}\dimple1列目中央の(だいたいの)位置
%% footnote %%%%%%%%%%%%%%%%%%%%%
\footnote{$w_{\mathrm Aq}$, $w_{\mathrm Bq}$は\index{わんきょくちゅうしん@湾曲中心}湾曲中心\index{O(わんきょくちゅうしん)@O(湾曲中心)}O(0, 0)方向への長さであるため正確ではないことに注意。}\relax
%%%%%%%%%%%%%%%%%%%%%%%%%%%%%%%%%
は、次で与えられる。
\begin{align*}
\begin{array}{rl}
  \text{A面($+X$方向):}
  & \displaystyle
    \left(
      g_{tx}+\mathcal L_0+\frac{w'_{\mathrm Aq}}2~,~
      g_{ty}~,~
      f_t'-q
    \right),\\[12pt]
  \text{C面($-X$方向):}
  & \displaystyle
    \left(
      g_{tx}+\mathcal L_0-\frac{w'_{\mathrm Aq}}2~,~
      g_{ty}~,~
      f_t'-q
    \right),\\[12pt]
  \text{B面($+Y$方向):}
  & \displaystyle
    \left(
      g_{tx}+\mathcal L_0~,~
      g_{ty}+\frac{w'_{\mathrm Bq}}2~,~
      f_t'-q
    \right),\\[12pt]
  \text{D面($-Y$方向):}
  & \displaystyle
    \left(
      g_{tx}+\mathcal L_0~,~
      g_{ty}-\frac{w'_{\mathrm Bq}}2~,~
      f_t'-q
    \right).
\end{array}
\end{align*}
ここで、\expandafterindex{\dimplekana iれつめ@\dimple$i$列目}$i$列目の湾曲中心と\index{トップたんのわんきょくちゅうしん@トップ端の湾曲中心}トップ端の湾曲中心との$X$座標の差を、
%% label{eq:}
\begin{align}
  \label{eq:dimpleCenterDistance}
  \mathcal L_i
  \equiv \sqrt{R_\mathrm c^2-\left\{f_\mathrm T-q-(i-1)p_z\right\}^2}-\sqrt{R_\mathrm c^2-f_\mathrm T^2}
\end{align}
と表した。
なお、$i$列目の湾曲中心と$j$列目の湾曲中心との$X$座標の差を
\begin{align*}
  \mathcal L_{i,j}
  \equiv \mathcal L_i-\mathcal L_j
  = \sqrt{R_\mathrm c^2-\left(f_\mathrm T-q-(i-1)p_z\right)^2}
    -\sqrt{R_\mathrm c^2-\left\{f_\mathrm T-q-(j-1)p_z\right\}^2}
\end{align*}
と表すことにする。



%%%%%%%%%%%%%%%%%%%%%%%%%%%%%%%%%%%%%%%%%%%%%%%%%%%%%%%%%%
%% subsection 5.3.1 %%%%%%%%%%%%%%%%%%%%%%%%%%%%%%%%%%%%%%
%%%%%%%%%%%%%%%%%%%%%%%%%%%%%%%%%%%%%%%%%%%%%%%%%%%%%%%%%%
\subsection{\dimple の\texorpdfstring{$X$}{X}座標(傾き前)}
テーブル中心Pを原点としたとき、傾き前の\expandafterindex{\dimplekana iれつめjばんめ@\dimple$i$列目$j$番目}$i$列目$j$番目の\dimple の$X$座標は、
%% label{eq:dPosXBefore}
\begin{align}
  \notag
  \text{A面:}\quad
  \mathcal D_{xi,\mathrm A}
  &= g_{tx}+\mathcal L_i+\frac{w'_{\mathrm Aq+(i-1)p_z}}2\\
  \label{eq:dPosXBefore}
  \text{C面:}\quad
  \mathcal D_{xi,\mathrm C}
  &= g_{tx}+\mathcal L_i-\frac{w'_{\mathrm Aq+(i-1)p_z}}2\\
  \notag
  \text{B, D面:}\quad
  \mathcal D_{xij,\mathrm B}
  &= g_{tx}+\mathcal L_i+\frac{d_i}2-(j-1)p_x
\end{align}
なお、A・C面については$j$に依らないことがわかる。
そのため、たとえば$\mathcal D_{xij,\mathrm A}$ではなく、$\mathcal D_{xi,\mathrm A}$のように表記している。



%%%%%%%%%%%%%%%%%%%%%%%%%%%%%%%%%%%%%%%%%%%%%%%%%%%%%%%%%%
%% subsection 5.3.2 %%%%%%%%%%%%%%%%%%%%%%%%%%%%%%%%%%%%%%
%%%%%%%%%%%%%%%%%%%%%%%%%%%%%%%%%%%%%%%%%%%%%%%%%%%%%%%%%%
\subsection{\dimple の\texorpdfstring{$Y$}{Y}座標(傾き前)}
テーブル中心Pを原点としたとき、傾き前の\expandafterindex{\dimplekana iれつめjばんめ@\dimple$i$列目$j$番目}$i$列目$j$番目の\dimple の$Y$座標は、
%% label{eq:dPosYBefore}
\begin{alignat}{3}
  \notag
  \text{A, C面:}\quad
  && \mathcal D_{yij,\mathrm A} &= g_{ty}-\frac{d_i}2+(j-1)p_x\\
  \label{eq:dPosYBefore}
  \text{B面:}\quad
  && \mathcal D_{yi,\mathrm B} &= g_{ty}+\frac{w'_{\mathrm Bq+(i-1)p_z}}2\\
  \notag
  \text{D面:}\quad
  && \mathcal D_{yi,\mathrm D} &= g_{ty}-\frac{w'_{\mathrm Bq+(i-1)p_z}}2
\end{alignat}
B・D面については$j$に依らないことがわかる。



%%%%%%%%%%%%%%%%%%%%%%%%%%%%%%%%%%%%%%%%%%%%%%%%%%%%%%%%%%
%% subsection 5.3.3 %%%%%%%%%%%%%%%%%%%%%%%%%%%%%%%%%%%%%%
%%%%%%%%%%%%%%%%%%%%%%%%%%%%%%%%%%%%%%%%%%%%%%%%%%%%%%%%%%
\subsection{\dimple の\texorpdfstring{$Z$}{Z}座標(傾き前)}
テーブル中心Pを原点としたとき、傾き前の\expandafterindex{\dimplekana iれつめjばんめ@\dimple$i$列目$j$番目}$i$列目$j$番目の\dimple の$Z$座標は、
%% label{eq:dPosZBefore}
\begin{align}
  \label{eq:dPosZBefore}
  \text{A, B, C, D面:}\quad
  \mathcal D_{zi} = f_t'-q-(i-1)p_z
\end{align}
$Z$座標についてはどの面も$j$に依らないことがわかる。



%%%%%%%%%%%%%%%%%%%%%%%%%%%%%%%%%%%%%%%%%%%%%%%%%%%%%%%%%%
%% subsection 5.3.4 %%%%%%%%%%%%%%%%%%%%%%%%%%%%%%%%%%%%%%
%%%%%%%%%%%%%%%%%%%%%%%%%%%%%%%%%%%%%%%%%%%%%%%%%%%%%%%%%%
\subsection{傾き角}
\index{トップがわのAがわうちたんてん@トップ側のA側内端点}A側内面トップ端と、\expandafterindex{Aがわないめん(\dimplekana1れつめ)@A側内面(\dimple1列目)}A側内面のトップ端から$q$の位置との$x$方向の差は、
\begin{align*}
  \sqrt{\left(R_\mathrm c+\frac{w'_{\mathrm Aq}}2\right)^{\!\!2}-(f_\mathrm T-q)^2}
  -\sqrt{\left(R_\mathrm c+\frac{w'_{\mathrm A0}}2\right)^{\!\!2}-f_\mathrm T^2}
\end{align*}
で与えられる。
このとき、これが負になる場合は傾ける必要はなく、正となる場合のみ傾ける。
したがってその\index{かたむきかく(\dimplekana)@傾き角(\dimple)}傾き角$\phi$は、
%% label{eq:dKatamuki}
\begin{subequations}
\label{eq:dKatamuki}
\begin{alignat}{2}
  \text{正の場合:}&&\quad
  \tan\phi
  &= \frac{\displaystyle
           \sqrt{\left(R_\mathrm c+\frac{w'_{\mathrm Aq}}2\right)^{\!\!2}-(f_\mathrm T-q)^2}
           -\sqrt{\left(R_\mathrm c+\frac{w'_{\mathrm A0}}2\right)^{\!\!2}-f_\mathrm T^2}}q\\[8pt]
  \text{負の場合:}&&
  \phi
  &= 0
\end{alignat}
\end{subequations}
で与えられる。
%%%%%%%%%%%%%%%%%%%%%%%%%%%%%%%%%%%%%%%%%%%%%%%%%%%%%%%%%%
%% hosoku %%%%%%%%%%%%%%%%%%%%%%%%%%%%%%%%%%%%%%%%%%%%%%%%
%%%%%%%%%%%%%%%%%%%%%%%%%%%%%%%%%%%%%%%%%%%%%%%%%%%%%%%%%%
\begin{hosoku}
なお、これが負になるのは、
\begin{align*}
  & \left(R_\mathrm c+\frac{w'_{\mathrm Aq}}2\right)^{\!\!2}-(f_\mathrm T-q)^2
    < \left(R_\mathrm c+\frac{w'_{\mathrm A0}}2\right)^{\!\!2}-f_\mathrm T^2\\
  \longrightarrow~~
  & \frac{w_{\mathrm A0}-w_{\mathrm Aq}}2
    \left(2R_\mathrm c+\frac{w_{\mathrm A0}'+w_{\mathrm Aq}'}2\right)
    > q(2f_\mathrm T-q)
\end{align*}
である。
したがって、以下のような場合に生じる傾向があることがわかる。
\begin{enumerate}
\item \index{きょくりつ@曲率}曲率が小さい(湾曲$R_\mathrm c$が大きい)
\item \index{ないめんテーパ@内面テーパ}内面テーパがきつい($w_{\mathrm A0}-w_{\mathrm Aq}$が大きい)
\item \index{ないけい@内径}内径・\index{めっきまくあつ@めっき膜厚}めっき膜厚が大きい($w_\mathrm A'$が大きい)
\end{enumerate}
たとえば、曲率0 ($R = +\infty$)の\index{モールド}モールド、つまり(\index{がいけい(モールド)@外形(モールド)}外形が)まっすぐの\index{わんきょくのないモールド@湾曲のないモールド}湾曲のないモールドなどが、これに該当する。
\end{hosoku}
%%%%%%%%%%%%%%%%%%%%%%%%%%%%%%%%%%%%%%%%%%%%%%%%%%%%%%%%%%
%%%%%%%%%%%%%%%%%%%%%%%%%%%%%%%%%%%%%%%%%%%%%%%%%%%%%%%%%%
%%%%%%%%%%%%%%%%%%%%%%%%%%%%%%%%%%%%%%%%%%%%%%%%%%%%%%%%%%
%%%%%%%%%%%%%%%%%%%%%%%%%%%%%%%%%%%%%%%%%%%%%%%%%%%%%%%%%%
%% Column %%%%%%%%%%%%%%%%%%%%%%%%%%%%%%%%%%%%%%%%%%%%%%%%
%%%%%%%%%%%%%%%%%%%%%%%%%%%%%%%%%%%%%%%%%%%%%%%%%%%%%%%%%%
\begin{Column}{C側\dimple の傾き角}
\expandafterindex{Cがわ\dimplekana@C側\dimple}C側\dimple については傾斜が外側に向いているため、傾けなくとも\index{アンダーカット}アンダーカットの心配はない。
しかし、傾けたまま加工をすると形状が歪になってしまうため、\dimple の形状をより円に近い形にするためには傾いていないほうが望ましい。
また一方で、面によって傾ける傾けないを分けると、プログラムが複雑になる(\index{じょうけんぶんき@条件分岐}条件分岐が増える)要因にもなる。
そのためここでは、どの面の\dimple に対しても同じ角度$\phi$を用いて加工を行うことにする。
\tcbline*
なお、C面に対する\dimple の形状をできるだけよいものにするには、\index{Cがわないめんテーパ@C側内面テーパ}C側の内面テーパに基づいた角度を用いるほうが望ましい。
そのため、C側\dimple に対する\expandafterindex{かたむきかく(Cがわ\dimplekana)@傾き角(C側\dimple)}傾き角$\phi_\mathrm C$についても(1つの例として)与えておく。
具体的には、以下の2点を基準として角度$\phi_\mathrm C$を取ることとする。
\begin{enumerate}
\item[a)] C側内面の\dimple1列目(トップ端から$q$)の位置
\item[b)] C側内面の\dimple$m$列目(トップ端から$q+(m-1)p_z$)の位置
\end{enumerate}
C側内面のトップ端から$q$の位置と、C側内面のトップ端から$q+(m-1)p_z$の位置との$x$方向の差は、
\begin{align*}
  \sqrt{\left(R_\mathrm c-\frac{w'_{\mathrm Aq+(m-1)p_z}}2\right)^{\!\!2}
        -\left\{f_\mathrm T-q-(m-1)p_z\right\}^2}
  -\sqrt{\left(R_\mathrm c-\frac{w'_{\mathrm Aq}}2\right)^{\!\!2}-(f_\mathrm T-q)^2}
\end{align*}
これより、C側\dimple に対する傾き角$\phi_\mathrm C$ ($\phi_\mathrm C > 0$)は、
\begin{align*}
  \tan\phi_\mathrm C
  = \frac{\displaystyle
          \sqrt{\left(R_\mathrm c-\frac{w'_{\mathrm Aq+(m-1)p_z}}2\right)^{\!2}
                -\left\{f_\mathrm T-q-(m-1)p_z\right\}^2}
          -\sqrt{\left(R_\mathrm c-\frac{w'_{\mathrm Aq}}2\right)^{\!2}-(f_\mathrm T-q)^2}}
         {(m-1)p_z}
\end{align*}
で与えられる。
なお、前述の通り$w_{\mathrm Aq+(m-1)p_z}$は$\lambda_j \leqq q+(m-1)p_z < \lambda_{j+1}$に対する$w_{\mathrm Aj}$, $w_{\mathrm Aj+1}$の\index{かじゅうさんじゅつへいきん(ないけい)@加重算術平均(内径)}加重算術平均
\begin{align*}
  w_{\mathrm Aq+(m-1)p_z}
  = \frac{\{q+(m-1)p_z-\lambda_j\}w_{\mathrm Aj+1}+\{\lambda_{j+1}-q-(m-1)p_z\}w_{\mathrm Aj}}
         {\lambda_{j+1}-\lambda_j}
\end{align*}
であり、\index{ないけい@内径}内径として代用している。($w_{\mathrm Bq+(m-1)p_z}$についても同様)
\end{Column}
%%%%%%%%%%%%%%%%%%%%%%%%%%%%%%%%%%%%%%%%%%%%%%%%%%%%%%%%%%
%%%%%%%%%%%%%%%%%%%%%%%%%%%%%%%%%%%%%%%%%%%%%%%%%%%%%%%%%%
%%%%%%%%%%%%%%%%%%%%%%%%%%%%%%%%%%%%%%%%%%%%%%%%%%%%%%%%%%




\clearpage
%%%%%%%%%%%%%%%%%%%%%%%%%%%%%%%%%%%%%%%%%%%%%%%%%%%%%%%%%%
%% subsection 26.3.5 %%%%%%%%%%%%%%%%%%%%%%%%%%%%%%%%%%%%%
%%%%%%%%%%%%%%%%%%%%%%%%%%%%%%%%%%%%%%%%%%%%%%%%%%%%%%%%%%
\subsection{B, D面の\dimple の位置(傾き前)}
B, D側\dimple において、その$X$座標がA側内面に最も近いものは、$m-1$列目または$m$列目の1番目の\dimple である。
これらの$X$座標は\pageeqref{eq:dPosXBefore}よりそれぞれ、
\begin{align*}
  m-1\text{列目:}&\quad
  g_{tx}+\mathcal L_{m-1}+\frac{d_{m-1}}2\\
  m\text{列目:}&\quad
  g_{tx}+\mathcal L_m+\frac{d_m}2
\end{align*}
%%%%%%%%%%%%%%%%%%%%%%%%%%%%%%%%%%%%%%%%%%%%%%%%%%%%%%%%%%
%% hosoku %%%%%%%%%%%%%%%%%%%%%%%%%%%%%%%%%%%%%%%%%%%%%%%%
%%%%%%%%%%%%%%%%%%%%%%%%%%%%%%%%%%%%%%%%%%%%%%%%%%%%%%%%%%
\begin{hosoku}
$d_{m-1} > d_m$のときは$m-1$列目, $d_m > d_{m-1}$のときは$m$列目をみればよい。
\end{hosoku}
%%%%%%%%%%%%%%%%%%%%%%%%%%%%%%%%%%%%%%%%%%%%%%%%%%%%%%%%%%
%%%%%%%%%%%%%%%%%%%%%%%%%%%%%%%%%%%%%%%%%%%%%%%%%%%%%%%%%%
%%%%%%%%%%%%%%%%%%%%%%%%%%%%%%%%%%%%%%%%%%%%%%%%%%%%%%%%%%
A側内面のトップ端からの($X$方向の)距離は、\index{トップたんのACないけい@トップ端のAC内径}トップ端のAC側内径として$w'_{\mathrm A0}$を代用すると、それぞれ
\begin{align*}
  m-1\text{列目:}&\quad
  \frac{w'_{\mathrm A0}}2-\mathcal L_{m-1}-\frac{d_{m-1}}2\\
  m\text{列目:}&\quad
  \frac{w'_{\mathrm A0}}2-\mathcal L_m-\frac{d_m}2
\end{align*}
これらのいずれか小さいほうが\index{こうぐけい@工具径}工具径(半径)よりも小さければ、\index{モールド}モールドを傾けて加工をする必要があると判断できる
%% footnote %%%%%%%%%%%%%%%%%%%%%
\footnote{もちろん、いくらか余裕代をとる必要がある。}。
%%%%%%%%%%%%%%%%%%%%%%%%%%%%%%%%%
%%%%%%%%%%%%%%%%%%%%%%%%%%%%%%%%%%%%%%%%%%%%%%%%%%%%%%%%%%
%% Column %%%%%%%%%%%%%%%%%%%%%%%%%%%%%%%%%%%%%%%%%%%%%%%%
%%%%%%%%%%%%%%%%%%%%%%%%%%%%%%%%%%%%%%%%%%%%%%%%%%%%%%%%%%
\begin{Column}{B, D側\dimple 加工で考慮すべき点}
\paragraph*{工具径とシャンク径}
\index{アンダーカット}アンダーカットが生じるのは主に\index{トップがわのAがわうちたんてん@トップ側のA側内端点}(A側内面の)トップ端なので、実際には\index{こうぐけい@工具径}工具径(工具の切削する部分)ではなく\index{シャンクけい(Tスロットカッター)@シャンク径(Tスロットカッター)}シャンク径等(工具のトップ端に相当する箇所)でよい。
そのため工具径よりシャンク径のほうが小さい場合は、より広い範囲の(B, D面の)\dimple を傾けずに切削することが可能となる。
\tcbline*
\paragraph*{端面の削り代}
\dimple の測定・加工は、端面を切削する前に行う。
そのため測定・加工の際は、\index{ぜんけずりしろ(トップたんめん)@全削り代(トップ端面)}端面の削り代の分だけ大きい(長い)ことに注意する必要がある。
削り代の分だけ\index{わんきょく@湾曲}湾曲も加味する必要があり、特に\index{Aがわないめん(トップがわ)@A側内面(トップ側)}A側内面と工具とのアンダーカットに留意しなければならない。
\tcbline*
\paragraph*{その他のずれ}
\index{モールドのけいじょう@モールドの形状}モールドの形状は当然ながら\index{ずめん(モールド)@図面(モールド)}図面のものとは一致はしない。
特に\index{わんきょく@湾曲}湾曲や\index{にくあつ@肉厚}肉厚などの図面とのずれは、アンダーカットに大きく寄与するのでこれも注意する必要がある。
\end{Column}
%%%%%%%%%%%%%%%%%%%%%%%%%%%%%%%%%%%%%%%%%%%%%%%%%%%%%%%%%%
%%%%%%%%%%%%%%%%%%%%%%%%%%%%%%%%%%%%%%%%%%%%%%%%%%%%%%%%%%
%%%%%%%%%%%%%%%%%%%%%%%%%%%%%%%%%%%%%%%%%%%%%%%%%%%%%%%%%%





\clearpage
%%%%%%%%%%%%%%%%%%%%%%%%%%%%%%%%%%%%%%%%%%%%%%%%%%%%%%%%%%
%% section 7.4 %%%%%%%%%%%%%%%%%%%%%%%%%%%%%%%%%%%%%%%%%%%
%%%%%%%%%%%%%%%%%%%%%%%%%%%%%%%%%%%%%%%%%%%%%%%%%%%%%%%%%%
\modHeadsection{傾き後の\dimple}
機内での回転は\index{テーブルちゅうしん@テーブル中心}テーブル中心Pを\index{げんてんP@原点P}原点として行われる。
また\dimple の加工は\index{トップたんのないけいちゅうしん@トップ端の内径中心}トップ端における内径中心を基準にして切削を行う。
傾ける前の\index{トップたんのないけいちゅうしん@トップ端の内径中心}トップ端内径中心$g_t$の座標は実測により(Pを中心とした$XYZ$直交座標でいうところの)[$g_{tx}$, $g_{ty}$, $f_t'$]で与えられる
%% footnote %%%%%%%%%%%%%%%%%%%%%
\footnote{ここではこれをテーブル中心Pを原点とした座標値として取り扱っている。
しかし、計測では機械座標系の値として$g_t$が与えられる。
たとえば\DMname の場合、$g_t$は通常(今の場合はテーブル中心Pより負側に湾曲中心があることが多いので)負の値として得られることに注意。
(ここでの計測では$XY$成分のみであり、$Z$については計測しないことにも注意。)}。
%%%%%%%%%%%%%%%%%%%%%%%%%%%%%%%%%
このとき、テーブルを角度$-\phi$だけ傾けた後のトップ端内面中心の座標$g'_t$は
%% footnote %%%%%%%%%%%%%%%%%%%%%
\footnote{これらをワーク座標原点としてもよいし、ワーク座標原点$g_t$はそのままで各面ごとに傾けてもよい。
ここでは後者の方法で加工を行うものとする。}、
%%%%%%%%%%%%%%%%%%%%%%%%%%%%%%%%%
%% label{eq:afterPhiTCenterFromO}
\begin{align}
  \label{eq:afterPhiTCenterFromO}
  \left[
  \begin{array}{c}
    g_{tx}'\\
    g_{ty}'\\
    g_{tz}'
  \end{array}
  \right]
  =\left[
   \begin{array}{c}
     g_{tx}\cos\phi+f_t'\sin\phi\\
     g_{ty}\\
     -g_{tx}\sin\phi+f_t'\cos\phi
   \end{array}
   \right].
   \end{align}
同様に、$i$列目における(傾ける前の)湾曲中心の位置は、[$g_{tx}+\mathcal L_i$, $g_{ty}$, $f_t'-q-(i-1)p_z$]で与えられる
%% footnote %%%%%%%%%%%%%%%%%%%%%
\footnote{ここではトップ端における湾曲中心を、トップ端における内面中心と同一視している。}
%%%%%%%%%%%%%%%%%%%%%%%%%%%%%%%%%
ので、テーブルを角度$-\phi$だけ傾けた後の$i$列目における湾曲中心の位置は、
\begin{align*}
  \left[
  \begin{array}{c}
    (g_{tx}+\mathcal L_i)\cos\phi+\{f_t'-q-(i-1)p_z\}\sin\phi\\
    g_{ty}\\
    -(g_{tx}+\mathcal L_i)\sin\phi+\{f_t'-q-(i-1)p_z\}\cos\phi
  \end{array}
  \right].
\end{align*}
したがって、傾けた後のトップ端の湾曲中心と$i$列目に対する湾曲中心との差分は、
%% label{eq:afterPhidimpleCenterDistance}
\begin{align}
  \label{eq:afterPhidimpleCenterDistance}
  \left[
  \begin{array}{c}
    \mathcal L_i\cos\phi-\{q+(i-1)p_z\}\sin\phi\\
    0\\
    -\mathcal L_i\sin\phi-\{q+(i-1)p_z\}\cos\phi
  \end{array}
  \right].
\end{align}
%%%%%%%%%%%%%%%%%%%%%%%%%%%%%%%%%%%%%%%%%%%%%%%%%%%%%%%%%%
%% hosoku %%%%%%%%%%%%%%%%%%%%%%%%%%%%%%%%%%%%%%%%%%%%%%%%
%%%%%%%%%%%%%%%%%%%%%%%%%%%%%%%%%%%%%%%%%%%%%%%%%%%%%%%%%%
\begin{hosoku}
傾けた後の$i$列目に対する湾曲中心と$j$列目に対する湾曲中心との差分は、
\begin{align*}
  \left[
  \begin{array}{c}
    \mathcal L_{j,i}\cos\phi-(j-i)p_z\sin\phi\\
    0\\
    -\mathcal L_{j,i}\sin\phi-(j-i)p_z\cos\phi
  \end{array}
  \right].
\end{align*}
特に、$j = i+1$の場合は、
\begin{align*}
  \left[
  \begin{array}{c}
    \mathcal L_{i+1,i}\cos\phi-p_z\sin\phi\\
    0\\
    -\mathcal L_{i+1,i}\sin\phi-p_z\cos\phi
  \end{array}
  \right].
\end{align*}
\end{hosoku}
%%%%%%%%%%%%%%%%%%%%%%%%%%%%%%%%%%%%%%%%%%%%%%%%%%%%%%%%%%
%%%%%%%%%%%%%%%%%%%%%%%%%%%%%%%%%%%%%%%%%%%%%%%%%%%%%%%%%%
%%%%%%%%%%%%%%%%%%%%%%%%%%%%%%%%%%%%%%%%%%%%%%%%%%%%%%%%%%
%%%%%%%%%%%%%%%%%%%%%%%%%%%%%%%%%%%%%%%%%%%%%%%%%%%%%%%%%%
%% Column %%%%%%%%%%%%%%%%%%%%%%%%%%%%%%%%%%%%%%%%%%%%%%%%
%%%%%%%%%%%%%%%%%%%%%%%%%%%%%%%%%%%%%%%%%%%%%%%%%%%%%%%%%%
\begin{Column}{タッチセンサープローブ径の考慮:$XY$と$Z$方向の非対称性}
マシニング内の計測では\index{タッチセンサープローブ}タッチセンサープローブを用いる。
そのため、\index{タッチセンサープローブせんたんきゅう@タッチセンサープローブ先端球}タッチセンサープローブ先端球の径の大きさに対して考慮・補正しなければならない。
タッチセンサープローブの位置の基準については、以下のようにとるのが通常である。
\begin{enumerate}
\item $X$方向:基準はタッチセンサープローブ先端球の($X$方向の)中心
\item $Y$方向:基準はタッチセンサープローブ先端球の($Y$方向の)中心
\item $Z$方向:基準はタッチセンサープローブ先端球の($Z$方向の)先端
\end{enumerate}
したがって、$XY$方向と$Z$方向とでは\index{きじゅんてん@基準点(タッチセンサープローブ)}基準点が異なり非対称となっている。
今の場合、基準が非対称な$X$と$Z$が混合する移動(回転)であるが、あくまでもタッチセンサープローブの先端(上記の基準点)が回転後の位置にある、ということである。
そのため補正については(傾きに関係なく)$Z$方向に対してのみ径の半分だけ補正すればよい。
\end{Column}
%%%%%%%%%%%%%%%%%%%%%%%%%%%%%%%%%%%%%%%%%%%%%%%%%%%%%%%%%%
%%%%%%%%%%%%%%%%%%%%%%%%%%%%%%%%%%%%%%%%%%%%%%%%%%%%%%%%%%
%%%%%%%%%%%%%%%%%%%%%%%%%%%%%%%%%%%%%%%%%%%%%%%%%%%%%%%%%%


%%%%%%%%%%%%%%%%%%%%%%%%%%%%%%%%%%%%%%%%%%%%%%%%%%%%%%%%%%
%% subsection 5.4.1 %%%%%%%%%%%%%%%%%%%%%%%%%%%%%%%%%%%%%%
%%%%%%%%%%%%%%%%%%%%%%%%%%%%%%%%%%%%%%%%%%%%%%%%%%%%%%%%%%
\subsection{傾き後の\dimple(A, C面側)}
\expandafterindex{かたむきかく(\dimplekana)@傾き角(\dimple)}傾ける角度$\phi$は\pageeqref{eq:dKatamuki}で与えられる。
このとき、傾けた後のAおよびC面側に対する$i$列目$j$番目の\dimple の位置は、\pageeqref{eq:dPosXBefore}, \eqref{eq:dPosYBefore}, \pageeqref{eq:dPosZBefore}より、
\begin{alignat*}{3}
  \text{A面:}&~~&
  \left[
  \begin{array}{c}
    \mathcal D_{xij,\mathrm A}'\\
    \mathcal D_{yij,\mathrm A}'\\
    \mathcal D_{zij,\mathrm A}'
  \end{array}
  \right]
 &= \left[
    \begin{array}{c}
      \mathcal D_{xi,\mathrm A}\cos\phi+\mathcal D_{zi}\sin\phi\\
      \mathcal D_{yij,\mathrm A}\\
      -\mathcal D_{xi,\mathrm A}\sin\phi+\mathcal D_{zi}\cos\phi
    \end{array}
    \right],\\[2pt]
  \text{C面:}&~~&
  \left[
  \begin{array}{c}
    \mathcal D_{xij,\mathrm C}'\\
    \mathcal D_{yij,\mathrm C}'\\
    \mathcal D_{zij,\mathrm C}'
  \end{array}
  \right]
 &= \left[
    \begin{array}{c}
      \mathcal D_{xi,\mathrm C}\cos\phi+\mathcal D_{zi}\sin\phi\\
      \mathcal D_{yij,\mathrm A}\\
      -\mathcal D_{xi,\mathrm C}\sin\phi+\mathcal D_{zi}\cos\phi
    \end{array}
    \right].
\end{alignat*}
特に、各列の中央(各列の\expandafterindex{わんきょくちゅうしん(\dimplekana)@湾曲中心(\dimple)}湾曲中心)$[g_{tx}+\mathcal L_i, g_{ty}, f_t'-q-(i-1)p_z]$を原点としてみた場合の位置は、
\begin{align*}
  \left[
  \begin{array}{c}
    \displaystyle \pm\frac{w_{Aq+(i-1)p_z}'}2\cos\phi\\[6pt]
    \displaystyle -\frac{d_i}2+(j-1)p_x\\[6pt]
    \displaystyle \mp\frac{w_{Aq+(i-1)p_z}'}2\sin\phi
  \end{array}
  \right]\qquad
  %%%%%%%%
  \left(
  \text{複号}
  \left\{
  \begin{array}{rl}
    \!\text{上}\!\!\!& \text{: A面}\\
    \!\text{下}\!\!\!& \text{: C面}\\
  \end{array}
  \right.
  \right).
\end{align*}





\paragraph*{$j$方向の差分}\noindent
$Y$方向の隣同士の差分、すなわち$i$を固定したときの$j$番目と$j+1$番目の位置の差分は、
\begin{align*}
  \left[
  \begin{array}{c}
    0\\
    \mathcal D_{yi(j+1),\mathrm A}-\mathcal D_{yij,\mathrm A}\\
    0
  \end{array}
  \right]
  = \left[
    \begin{array}{c}
      0\\
      p_x\\
      0
    \end{array}
    \right]\ .
\end{align*}


\paragraph*{$i$方向の差分}\noindent
$Z$方向の隣同士の差分、すなわち$j$を固定したときの$i$番目と$i+1$番目の位置の差分については、
\begin{align*}
 &\left[
  \begin{array}{c}
    (\mathcal D_{x(i+1),\mathrm A}-\mathcal D_{xi,\mathrm A})\cos\phi
    +(\mathcal D_{z(i+1)}-\mathcal D_{zi})\sin\phi\\
    (\mathcal D_{y(i+1)j,\mathrm A}-\mathcal D_{yij,\mathrm A})\\
    (\mathcal D_{xi,\mathrm A}-\mathcal D_{x(i+1),\mathrm A})\sin\phi
    +(\mathcal D_{z(i+1)}-\mathcal D_{zi})\cos\phi
  \end{array}
  \right]\\
 &= \left[
    \begin{array}{c}
      \displaystyle
      \left(\mathcal L_{i+1, i}+\frac{w'_{\mathrm Aq+ip_z}-w'_{\mathrm Aq+(i-1)p_z}}2\right)\!\cos\phi
      -p_z\sin\phi\\[6pt]
      \displaystyle-\frac{d_{i+1}-d_i}2\\[6pt]
      \displaystyle
      -\left(\mathcal L_{i+1, i}+\frac{w'_{\mathrm Aq+ip_z}-w'_{\mathrm Aq+(i-1)p_z}}2\right)\!\sin\phi
      -p_z\cos\phi
    \end{array}
    \right]\ .
\end{align*}
C面に対しては、これの各々の内径$w_\mathrm A'$の符号を入れ換えたものとなる。
%%%%%%%%%%%%%%%%%%%%%%%%%%%%%%%%%%%%%%%%%%%%%%%%%%%%%%%%%%
%% hosoku %%%%%%%%%%%%%%%%%%%%%%%%%%%%%%%%%%%%%%%%%%%%%%%%
%%%%%%%%%%%%%%%%%%%%%%%%%%%%%%%%%%%%%%%%%%%%%%%%%%%%%%%%%%
\begin{hosoku}
$X$成分の差分の大きさが($\mathcal L_{i+1, i}$からみて)A面(の$\cos\phi$成分)のそれと同じであることがわかる。
これは(水平方向の)内径を$w_{\mathrm A\lambda}$等で代用したからであり、実際の長さは異なる(振分中心を除いて対称ではなく、C側のほうが長い)ことに注意。
\end{hosoku}
%%%%%%%%%%%%%%%%%%%%%%%%%%%%%%%%%%%%%%%%%%%%%%%%%%%%%%%%%%
%%%%%%%%%%%%%%%%%%%%%%%%%%%%%%%%%%%%%%%%%%%%%%%%%%%%%%%%%%
%%%%%%%%%%%%%%%%%%%%%%%%%%%%%%%%%%%%%%%%%%%%%%%%%%%%%%%%%%
%%%%%%%%%%%%%%%%%%%%%%%%%%%%%%%%%%%%%%%%%%%%%%%%%%%%%%%%%%
%% hosoku %%%%%%%%%%%%%%%%%%%%%%%%%%%%%%%%%%%%%%%%%%%%%%%%
%%%%%%%%%%%%%%%%%%%%%%%%%%%%%%%%%%%%%%%%%%%%%%%%%%%%%%%%%%
\begin{hosoku}[label=hosoku:generallyDimpleN]
\pageautoref{fn:generallyDimpleN}でも述べたように、たいていの場合は$|d_{i+1}-d_i|=p_x$であり、また$d_{i+2} = d_i$である。
\end{hosoku}
%%%%%%%%%%%%%%%%%%%%%%%%%%%%%%%%%%%%%%%%%%%%%%%%%%%%%%%%%%
%%%%%%%%%%%%%%%%%%%%%%%%%%%%%%%%%%%%%%%%%%%%%%%%%%%%%%%%%%
%%%%%%%%%%%%%%%%%%%%%%%%%%%%%%%%%%%%%%%%%%%%%%%%%%%%%%%%%%




%%%%%%%%%%%%%%%%%%%%%%%%%%%%%%%%%%%%%%%%%%%%%%%%%%%%%%%%%%
%% subsection 5.4.2 %%%%%%%%%%%%%%%%%%%%%%%%%%%%%%%%%%%%%%
%%%%%%%%%%%%%%%%%%%%%%%%%%%%%%%%%%%%%%%%%%%%%%%%%%%%%%%%%%
\subsection{傾き後の\dimple(B, D面側)}
傾けた後のBおよびD面側に対する$i$列目$j$番目の\dimple の位置は、A面側のときと同様に、
\begin{alignat*}{3}
  \text{B面:}&~~&
  \left[
    \begin{array}{c}
      \mathcal D_{xij,\mathrm B}'\\
      \mathcal D_{yij,\mathrm B}'\\
      \mathcal D_{zij,\mathrm B}'
    \end{array}
  \right]
 &= \left[
    \begin{array}{c}
      \mathcal D_{xij,\mathrm B}\cos\phi+\mathcal D_{zi}\sin\phi\\
      \mathcal D_{yi,\mathrm B}\\
      -\mathcal D_{xij,\mathrm B}\sin\phi+\mathcal D_{zi}\cos\phi
    \end{array}
    \right],\\[2pt]
  \text{D面:}&~~&
  \left[
    \begin{array}{c}
      \mathcal D_{xij,\mathrm D}'\\
      \mathcal D_{yij,\mathrm D}'\\
      \mathcal D_{zij,\mathrm D}'
    \end{array}
  \right]
 &= \left[
    \begin{array}{c}
      \mathcal D_{xij,\mathrm B}\cos\phi+\mathcal D_{zi}\sin\phi\\
      \mathcal D_{yi,\mathrm D}\\
      -\mathcal D_{xij,\mathrm B}\sin\phi+\mathcal D_{zi}\cos\phi
    \end{array}
    \right].
\end{alignat*}
特に、各列の中央(各列の湾曲中心)$[g_{tx}+\mathcal L_i, g_{ty}, f_t'-q-(i-1)p_z]$を原点としてみた場合の位置は、
\begin{align*}
  \left[
  \begin{array}{c}
    \displaystyle \left\{\frac{d_i}2-(j-1)p_z\right\}\cos\phi\\
    \displaystyle \pm\frac{w_{Bq+(i-1)p_z}'}2\\
    \displaystyle -\left\{\frac{d_i}2-(j-1)p_z\right\}\sin\phi
  \end{array}
  \right]\qquad
  %%%%%%%%
  \left(
  \text{複号}
  \left\{
  \begin{array}{rl}
    \!+\!\!\!& \text{: B面}\\
    \!-\!\!\!& \text{: D面}\\
  \end{array}
  \right.
  \right).
\end{align*}



\paragraph*{$j$方向の差分}\noindent
$Y$方向の隣同士の差分、すなわち$i$を固定したときの$j$番目と$j+1$番目の位置の差分は、
\begin{align*}
  \left[
  \begin{array}{c}
    \left(\mathcal D_{xi(j+1),\mathrm B}-\mathcal D_{xij,\mathrm B}\right)\cos\phi\\
    0\\
    -\left(\mathcal D_{xi(j+1),\mathrm B}-\mathcal D_{xij,\mathrm B}\right)\sin\phi
  \end{array}
  \right]
  = \left[
    \begin{array}{c}
      -p_x\cos\phi\\[6pt]
      0\\
      p_x\sin\phi
    \end{array}
    \right]\ .
\end{align*}


\paragraph*{$i$方向の差分}\noindent
B面に対する$Z$方向の隣同士の差分、すなわち$j$を固定したときの$i$番目と$i+1$番目の位置の差分については、
\begin{align*}
 &\left[
  \begin{array}{c}
    \left(\mathcal D_{x(i+1)j,\mathrm B}-\mathcal D_{xij,\mathrm B}\right)\cos\phi
    +\left(\mathcal D_{z(i+1)}-\mathcal D_{zi}\right)\sin\phi\\[3pt]
    \mathcal D_{yi+1,\mathrm B}-\mathcal D_{yi,\mathrm B}\\[3pt]
    -\left(\mathcal D_{x(i+1)j,\mathrm B}-\mathcal D_{xij,\mathrm B}\right)\sin\phi
    +\left(\mathcal D_{z(i+1)}-\mathcal D_{zi}\right)\cos\phi
  \end{array}
  \right]\\
 &= \left[
    \begin{array}{c}
      \displaystyle\left(\mathcal L_{i+1, i}+\frac{d_{i+1}-d_i}2\right)\!\cos\phi-p_z\sin\phi\\[10pt]
      \displaystyle\frac{w'_{\mathrm Bq+ip_z}-w'_{\mathrm Bq+(i-1)p_z}}2\\[8pt]
      \displaystyle-\left(\mathcal L_{i+1, i}+\frac{d_{i+1}-d_i}2\right)\!\cos\phi-p_z\cos\phi
    \end{array}
    \right]\ .
\end{align*}
D面に対しては、これの各々の内径$w_\mathrm B'$の符号(この場合$Y$成分の符号)を入れ換えたものとなる。





%%%%%%%%%%%%%%%%%%%%%%%%%%%%%%%%%%%%%%%%%%%%%%%%%%%%%%%%%%
%%           %%%%%%%%%%%%%%%%%%%%%%%%%%%%%%%%%%%%%%%%%%%%%
%% chapter 8 %%%%%%%%%%%%%%%%%%%%%%%%%%%%%%%%%%%%%%%%%%%%%
%%           %%%%%%%%%%%%%%%%%%%%%%%%%%%%%%%%%%%%%%%%%%%%%
%%%%%%%%%%%%%%%%%%%%%%%%%%%%%%%%%%%%%%%%%%%%%%%%%%%%%%%%%%
\modHeadchapter[loC]{通り芯の幾何}
%!TEX root = ../RfCPN.tex


\modHeadchapter[loColumn]{\CenterlineEndFaceDif の幾何}
トップ・ボトムの両方に\Outcut がある場合を考える。
通常、それぞれの\OutcutCenter は個別に決められはせず、片方の中心の位置を基準として、もう片方の中心が定められる。
これらの中心の位置の差(\textbf{\CenterlineEndFaceDif}%
%% footnote %%%%%%%%%%%%%%%%%%%%%
\footnote{通常、\CenterlineEndFaceDif(centerline)というのはその名の通り中心線を表すことが多い。
しかし、ここでは\TopOutcutCenter と\BottomOutcutCenter との位置の差を表す用語として「\CenterlineEndFaceDif」と呼んでいる。})
%%%%%%%%%%%%%%%%%%%%%%%%%%%%%%%%%
$T_x$, $T_y$ ($T_x \geq 0$)を機内で測定する際は、C面が工具側に向くようにテーブルを$\pm$90$^\circ$回転($B$軸回転)し、\index{タッチセンサープローブ}タッチセンサープローブを用いてそれぞれの\Outcut 部の$Z$座標および$Y$座標を見ることで測定する。

ここでは、この\expandafterindex{\yomiCenterlineEndFaceDif そくてい@\nameCenterlineEndFaceDif 測定}\nameCenterlineEndFaceDif の測定に必要な位置等について定量的に求める。
なお、\TableCenter Pを原点として考えることにする。
またC面が工具側に向くように$B$軸を(\verb|G91|にて)$\pm$90$^\circ$回転した状態であるとする
%% footnote %%%%%%%%%%%%%%%%%%%%%
\footnote{{\ttfamily G90}(絶対座標)の場合、\index{テーブル}テーブルを傾けて\AlocationLength を調整した場合はその\index{かたむきかく(ふりわけちょうせい)@傾き角(振分調整)}回転角$-\theta$を忘れないよう注意。}。
%%%%%%%%%%%%%%%%%%%%%%%%%%%%%%%%%



%%%%%%%%%%%%%%%%%%%%%%%%%%%%%%%%%%%%%%%%%%%%%%%%%%%%%%%%%%
%% section 27.1 %%%%%%%%%%%%%%%%%%%%%%%%%%%%%%%%%%%%%%%%%%
%%%%%%%%%%%%%%%%%%%%%%%%%%%%%%%%%%%%%%%%%%%%%%%%%%%%%%%%%%
\modHeadsection{\BottomOutcutCenter が基準の場合}
通常、\TopOutcutCenter は、\BottomOutcutCenter よりA面側($-Z$側)にある。
このとき、\KeywayPos$\kappa_p$および\BottomOutcutLength$h_\mathrm B$ ($h_\mathrm B > 0$)を用いると、ボトム側($-X$側)およびトップ側($+X$側)の\index{Cがわがいさくめん@C側外削面}C側外削面の中心
%% footnote %%%%%%%%%%%%%%%%%%%%%
\footnote{トップ側には\Keyway があるので、\TopOutcutLength は\KeywayPos$\kappa_p$とみなしている。}
%%%%%%%%%%%%%%%%%%%%%%%%%%%%%%%%%
は、それぞれ
%% footnote %%%%%%%%%%%%%%%%%%%%%
\footnote{通常、\CenterlineEndFaceDifBD は$T_y = 0$である。}
%%%%%%%%%%%%%%%%%%%%%%%%%%%%%%%%%
\begin{align}
  \label{eq:centerlineB}
  \text{ボトム側:}\quad
  \left[
    \begin{array}{c}
      \displaystyle -f_\mathrm B'+\frac{h_\mathrm B}2\\[5pt]
      \mathcal G_{\mathrm By}\\[3pt]
      \displaystyle \mathcal G_{\mathrm Bx}+\frac{\mathfrak W_\mathrm B}2
    \end{array}
    \right]~, \qquad
  \text{トップ側:}\quad
  \left[
    \begin{array}{c}
      \displaystyle f_\mathrm T'-\frac{\kappa_p}2\\[5pt]
      \mathcal G_{\mathrm By}-T_y\\[3pt]
      \displaystyle \mathcal G_{\mathrm Bx}-T_x+\frac{\mathfrak W_\mathrm T}2
    \end{array}
  \right].
\end{align}



%%%%%%%%%%%%%%%%%%%%%%%%%%%%%%%%%%%%%%%%%%%%%%%%%%%%%%%%%%
%% section 27.2 %%%%%%%%%%%%%%%%%%%%%%%%%%%%%%%%%%%%%%%%%%
%%%%%%%%%%%%%%%%%%%%%%%%%%%%%%%%%%%%%%%%%%%%%%%%%%%%%%%%%%
\modHeadsection{\TopOutcutCenter が基準の場合}
通常、\BottomOutcutCenter は、\TopOutcutCenter よりC面側($+Z$側)にある。
このとき、トップ側($+X$側)およびボトム側($-X$側)の\Outcut 部C面の中心は、それぞれ
\begin{align}
  \label{eq:centerlineT}
  \text{トップ側:}~~
  \left[
    \begin{array}{c}
      \displaystyle f_\mathrm T'-\frac{\kappa_p}2\\[5pt]
      \mathcal G_{\mathrm Ty}\\[3pt]
      \displaystyle \mathcal G_{\mathrm Tx}+\frac{\mathfrak W_\mathrm B}2
    \end{array}
    \right]~, \qquad
  \text{ボトム側:}~~
  \left[
    \begin{array}{c}
      \displaystyle -f_\mathrm B'+\frac{h_\mathrm B}2\\[5pt]
      \mathcal G_{\mathrm Ty}+T_y\\[3pt]
      \displaystyle \mathcal G_{\mathrm Tx}+T_x+\frac{\mathfrak W_\mathrm B}2
    \end{array}
  \right].
\end{align}

\clearpage
~\vfill
%%%%%%%%%%%%%%%%%%%%%%%%%%%%%%%%%%%%%%%%%%%%%%%%%%%%%%%%%%
%% Column %%%%%%%%%%%%%%%%%%%%%%%%%%%%%%%%%%%%%%%%%%%%%%%%
%%%%%%%%%%%%%%%%%%%%%%%%%%%%%%%%%%%%%%%%%%%%%%%%%%%%%%%%%%
\begin{Column}{C側面の測定}
通常、\OutcutMilling をせずに\CenterlineEndFaceDifAC の測定を行うことはない。
ただ、\index{NCプログラム}NCプログラムの\index{しうんてん@試運転}試運転などで動きをみるといった可能性はありうるので、\OutcutMilling を行っていない状態で\index{そくてい(\yomiCenterlineEndFaceDif)@測定(\nameCenterlineEndFaceDif)}測定する場合についても述べておく。
なお、ここでは\index{テーブル}テーブルを回転して\AlocationLength の調整を行った場合、かつ\BottomOutcut(\BottomOutcutAsideThickness)が基準の場合を考える。
このとき、測定するC側外面の位置は、\pageeqref{eq:tableTi}, \pageeqref{eq:tableBRi}より、
\begin{align*}
  \text{トップ側:}~~
  & \left[
    \begin{array}{c}
      \displaystyle f_\mathrm T'-\frac{\kappa_p}2\\[5pt]
      G_{\mathrm By}-T_y\\[3pt]
      \displaystyle
      -G_{\mathrm Tx}
      -\sqrt{R_\mathrm c^2-f_\mathrm T^2}
      +\sqrt{R_\mathrm c^2-\left(f_\mathrm T-\frac\kappa2\right)^2}
    \end{array}
    \right],\\
  \text{ボトム側:}~~
  & \left[
    \begin{array}{c}
      \displaystyle -f_\mathrm B'+\frac{h_\mathrm B}2\\[5pt]
      G_{\mathrm By}\\[3pt]
      \displaystyle
      G_{\mathrm Bx}
      -\sqrt{R_\mathrm c^2-f_\mathrm B^2}
      +\sqrt{R_\mathrm c^2-\left(f_\mathrm B-\frac{h_\mathrm B}2\right)^2}
    \end{array}
    \right].
\end{align*}
なお、これらの$Z$座標の差は以下で与えられる。
\begin{align*}
  \sqrt{R_\mathrm i^2-\left(f_\mathrm T-\frac{\kappa_p}2\right)^2}
  -\sqrt{R_\mathrm i^2-\left(f_\mathrm B-\frac{h_\mathrm B}2\right)^2}~.
\end{align*}
\end{Column}
%%%%%%%%%%%%%%%%%%%%%%%%%%%%%%%%%%%%%%%%%%%%%%%%%%%%%%%%%%
%%%%%%%%%%%%%%%%%%%%%%%%%%%%%%%%%%%%%%%%%%%%%%%%%%%%%%%%%%
%%%%%%%%%%%%%%%%%%%%%%%%%%%%%%%%%%%%%%%%%%%%%%%%%%%%%%%%%%




%%%%%%%%%%%%%%%%%%%%%%%%%%%%%%%%%%%%%%%%%%%%%%%%%%%%%%%%%%
%%            %%%%%%%%%%%%%%%%%%%%%%%%%%%%%%%%%%%%%%%%%%%%
%% Chapter 9  %%%%%%%%%%%%%%%%%%%%%%%%%%%%%%%%%%%%%%%%%%%%
%%            %%%%%%%%%%%%%%%%%%%%%%%%%%%%%%%%%%%%%%%%%%%%
%%%%%%%%%%%%%%%%%%%%%%%%%%%%%%%%%%%%%%%%%%%%%%%%%%%%%%%%%%
\modHeadchapter[lot]{内径の幾何}
%!TEX root = ../RPA_for_Creating_Program_Note.tex



%%%%%%%%%%%%%%%%%%%%%%%%%%%%%%%%%%%%%%%%%%%%%%%%%%%%%%%%%%
%% section 7.1 %%%%%%%%%%%%%%%%%%%%%%%%%%%%%%%%%%%%%%%%%%%
%%%%%%%%%%%%%%%%%%%%%%%%%%%%%%%%%%%%%%%%%%%%%%%%%%%%%%%%%%
\modHeadsection{eテーパの算定\TBW}
使用する鋼材の種類として、\index{C(たんそ)@C(炭素)}C, \index{Si(ケイそ)@Si(ケイ素)}Si, \index{Mn(マンガン)@Mn(マンガン)}Mn, \index{P(リン)@P(リン)}P, \index{S(いおう)@S(硫黄)}Sが含まれている場合を考える。
\index{JISきかく(こうしゅ)@JIS規格(鋼種)}JIS規格に基づいた鋼種を用いるものとすれば、その規格によってそれぞれの\index{かがくそせい(eテーパ)@化学組成(eテーパ)}化学組成の\index{がんゆうりょう(かがくそせい)@含有量(化学組成)}含有量も決定される。
それぞれの化学組成の含有量($\mathrm{wt}\%$)を$X_\mathrm C$, $X_\mathrm{Si}$, $X_\mathrm{Mn}$, $X_\mathrm P$, $X_\mathrm S$とし、またその\index{えいきょうけいすう(eテーパ)@影響係数(eテーパ)}影響係数を$k_\mathrm C$, $k_\mathrm{Si}$, $k_\mathrm{Mn}$, $k_\mathrm P$, $k_\mathrm S$とする。
また、その鋼材の\index{えきそうせんおんど@液相線温度}液相線温度を$T_\mathrm l$[$^\circ\mathrm C$], \index{こそうせんおんど@固相線温度}固相線温度を$T_\mathrm s$[$^\circ\mathrm C$]とする。
一般にこれらの温度は、ある\index{きじゅんおんど(eテーパ)@基準温度(eテーパ)}基準となる温度$T_0$に対して、
\begin{align*}
  T = T_0-\sum_i k_iX_i
\end{align*}
として与えられる。
今の場合だと、
\begin{align*}
  T_l
  &= 1536-78X_\mathrm C-7.6X_\mathrm{Si}-4.9X_\mathrm{Mn}-34.4X_\mathrm P-38X_\mathrm S~,\\
  T_s
  &= 1536-415.5X_\mathrm C-12.3X_\mathrm{Si}-6.8X_\mathrm{Mn}-124.5X_\mathrm P-183.9X_\mathrm S
\end{align*}
となることが知られている\cite{1986KO}。\\

\modcaptionof{table}{化学組成の含有量}
\begin{tabular}[t]{|c|c|c|c|c|c|c|}
  \hline
  鋼材(wt\%) & C & Si & Mn & P & S
  \\\hline
  \index{JISコード(がんゆうりょう)}JISコード(含有量) & 0.1 & 0.5 & 0.6 & 0.7 & 0.8
  \\\hline
\end{tabular}


\clearpage
%%%%%%%%%%%%%%%%%%%%%%%%%%%%%%%%%%%%%%%%%%%%%%%%%%%%%%%%%%
%% section 9.1 %%%%%%%%%%%%%%%%%%%%%%%%%%%%%%%%%%%%%%%%%%%
%%%%%%%%%%%%%%%%%%%%%%%%%%%%%%%%%%%%%%%%%%%%%%%%%%%%%%%%%%
\modHeadsection{内側湾曲の近似曲線\TBW}
内面は\index{テーパ(ないめん)@テーパ(内面)}テーパがついた形になっている。
テーパは\index{ないけいテーパひょう@内径テーパ表}内径テーパ表等の数値に基づき、\index{CAD}CADにて\index{アイソパラメトリックきょくせん@アイソパラメトリック曲線}アイソパラメトリック曲線として記述され、それを\index{CAM}CAMに設定し、\index{しんがね@芯金}芯金が作成されている。




%%%%%%%%%%%%%%%%%%%%%%%%%%%%%%%%%%%%%%%%%%%%%%%%%%%%%%%%%%
%%            %%%%%%%%%%%%%%%%%%%%%%%%%%%%%%%%%%%%%%%%%%%%
%% Chapter 9  %%%%%%%%%%%%%%%%%%%%%%%%%%%%%%%%%%%%%%%%%%%%
%%            %%%%%%%%%%%%%%%%%%%%%%%%%%%%%%%%%%%%%%%%%%%%
%%%%%%%%%%%%%%%%%%%%%%%%%%%%%%%%%%%%%%%%%%%%%%%%%%%%%%%%%%
\modHeadchapter{その他の幾何}
%!TEX root = ../RPA_for_Creating_Program_Note.tex



%%%%%%%%%%%%%%%%%%%%%%%%%%%%%%%%%%%%%%%%%%%%%%%%%%%%%%%%%%
%% section 31.1 %%%%%%%%%%%%%%%%%%%%%%%%%%%%%%%%%%%%%%%%%%
%%%%%%%%%%%%%%%%%%%%%%%%%%%%%%%%%%%%%%%%%%%%%%%%%%%%%%%%%%
\modHeadsection{ワーク固定用ボルト\TBW}
(to be written...)




\setlength{\parindent}{11pt}
\begin{appendices}
%%%%%%%%%%%%%%%%%%%%%%%%%%%%%%%%%%%%%%%%%%%%%%%%%%%%%%%%%
%%                %%%%%%%%%%%%%%%%%%%%%%%%%%%%%%%%%%%%%%%
%% Appendix       %%%%%%%%%%%%%%%%%%%%%%%%%%%%%%%%%%%%%%%
%% Part RfCPN_pAC %%%%%%%%%%%%%%%%%%%%%%%%%%%%%%%%%%%%%%%
%% Start          %%%%%%%%%%%%%%%%%%%%%%%%%%%%%%%%%%%%%%%
%%                %%%%%%%%%%%%%%%%%%%%%%%%%%%%%%%%%%%%%%%
%%%%%%%%%%%%%%%%%%%%%%%%%%%%%%%%%%%%%%%%%%%%%%%%%%%%%%%%%
\Appendixpart





%%%%%%%%%%%%%%%%%%%%%%%%%%%%%%%%%%%%%%%%%%%%%%%%%%%%%%%%%%
%%            %%%%%%%%%%%%%%%%%%%%%%%%%%%%%%%%%%%%%%%%%%%%
%% Appendix A %%%%%%%%%%%%%%%%%%%%%%%%%%%%%%%%%%%%%%%%%%%%
%%            %%%%%%%%%%%%%%%%%%%%%%%%%%%%%%%%%%%%%%%%%%%%
%%%%%%%%%%%%%%%%%%%%%%%%%%%%%%%%%%%%%%%%%%%%%%%%%%%%%%%%%%
\modHeadchapter{モールドとテーブルとの位置}
%!TEX root = ../RfCPN.tex


\modHeadchapter{\index{ワーク}ワークと\Table との位置}
\index{CAD}CADによる描画において、\TableCenter が原点(\index{ワールドげんてん@ワールド原点}ワールド原点)に置かれているとする。
ここで\index{ワーク}ワークを描画する際、\index{ワークのちゅうしん@ワークの中心}ワークの中心
%% footnote %%%%%%%%%%%%%%%%%%%%%
\footnote{$R_\mathrm c$に相当する点。}\relax
%%%%%%%%%%%%%%%%%%%%%%%%%%%%%%%%%
を\index{CAD}CAD上の\index{ワールドげんてん@ワールド原点}ワールド原点にして描くほうが都合のいいことがある。
このとき、\index{ワーク}ワークと\index{うけいた@受板}受板が接するように移動する必要がある。
鉛直方向(トップ-ボトム方向)においては$f_d$だけ動かせばよいが、水平方向の移動距離はあまり自明とはいいがたい。
\index{Cがわがいめん@C側外面}C側外面と\index{うけいためん@受板面}受板面との寸法を単純に測ると、(水平方向でなく)最短距離が測定されてしまう。
工夫により水平方向の距離を出すことも可能ではあるが、ここではその距離を定量的に求めておく。



%%%%%%%%%%%%%%%%%%%%%%%%%%%%%%%%%%%%%%%%%%%%%%%%%%%%%%%%%%
%% section B.01 %%%%%%%%%%%%%%%%%%%%%%%%%%%%%%%%%%%%%%%%%%
%%%%%%%%%%%%%%%%%%%%%%%%%%%%%%%%%%%%%%%%%%%%%%%%%%%%%%%%%%
\modHeadsection{\Spacer 取付前}
(\Spacer を取付る前の)\index{ワークのちゅうしん@ワークの中心}ワークの中心が\TableCenter Pに置かれている場合を考える。
\BottomSideReceiverPlate に接する\index{ワーク}ワークの点と、\TableCenter Pとは、実軸方向に
\begin{align*}
  R_\mathrm c-R_\mathrm i\cos\alpha_{\mathrm U_\mathrm B}
\end{align*}
だけ差がある。
したがって、\index{ワーク}ワークと受板が接する点の位置は実軸方向に、
\begin{align*}
  \Delta_x+\sqrt{R_\mathrm i'^2-\bar l^2}-R_\mathrm c+R_\mathrm i\cos\alpha_{\mathrm U_\mathrm B}\ .
\end{align*}
そのため\pageeqref{eq:afterUBcontact} ($\delta_s = 0$)より、\index{ワーク}ワークと(ボトム側の)受板は
\begin{align*}
  \Delta_x+\sqrt{R_\mathrm i'^2-\bar l^2}-R_\mathrm c
\end{align*}
だけ実軸方向に離れていることがわかる。
\pageeqref{eq:tableCenter}より、これは\TableCenter Pと\index{ワーク}ワークの\CenterCurvature$R_\mathrm c$との差であることがわかる。



\clearpage
%%%%%%%%%%%%%%%%%%%%%%%%%%%%%%%%%%%%%%%%%%%%%%%%%%%%%%%%%%
%% section B.02 %%%%%%%%%%%%%%%%%%%%%%%%%%%%%%%%%%%%%%%%%%
%%%%%%%%%%%%%%%%%%%%%%%%%%%%%%%%%%%%%%%%%%%%%%%%%%%%%%%%%%
\modHeadsection{\Spacer 取付後}
\expandafterindex{\yomiSpacerThickness@\nameSpacerThickness}厚さ$\delta_s\,(>0)$の\Spacer を取付けた場合、\index{ワーク}ワークと受板との接点および\TableCenter Pとは、実軸方向に
\begin{align*}
  R_\mathrm c-R_\mathrm i\cos\alpha'_{\mathrm U_\mathrm B}
\end{align*}
だけ差があるので、その実軸方向の位置は、
\begin{align*}
  \Delta_x+\sqrt{R_\mathrm i'^2-\bar l^2}-R_\mathrm c+R_\mathrm i\cos\alpha'_{\mathrm U_\mathrm B}\ .
\end{align*}
そのため\pageeqref{eq:afterUBcontact}より、\index{ワーク}ワークと(ボトム側の)受板は
\begin{align*}
  &  \Delta_x+\sqrt{R_\mathrm i'^2-\bar l^2}-R_\mathrm c+R_\mathrm i\cos\alpha'_{\mathrm U_\mathrm B}
     -\ab(R_\mathrm i'\cos\alpha_{\mathrm U_\mathrm B}+\rho\cos\alpha'_{\mathrm U_\mathrm B})\\
  &= \Delta_x-R_\mathrm c+R_\mathrm i'\cos\alpha'_{\mathrm U_\mathrm B}\\
  &= \Delta_x-R_\mathrm c
     -\frac{\delta_s}2+\sqrt{R_\mathrm i'^2-\frac{\delta_s^2+(2\bar l)^2}4}\frac{2\bar l}{\sqrt{\delta_s^2+(2\bar l)^2}}
\end{align*}
だけ実軸方向に離れていることがわかる。
\pageeqref{eq:tableCenter}および\pageeqref{eq:spacerMoveHdistance}より、これは\Spacer 取付け後の\index{モールドちゅうしん@モールド中心}モールド中心と\CenterCurvature$R_\mathrm c$との差であることがわかる。






%%%%%%%%%%%%%%%%%%%%%%%%%%%%%%%%%%%%%%%%%%%%%%%%%%%%%%%%%%
%%            %%%%%%%%%%%%%%%%%%%%%%%%%%%%%%%%%%%%%%%%%%%%
%% Appendix A %%%%%%%%%%%%%%%%%%%%%%%%%%%%%%%%%%%%%%%%%%%%
%%            %%%%%%%%%%%%%%%%%%%%%%%%%%%%%%%%%%%%%%%%%%%%
%%%%%%%%%%%%%%%%%%%%%%%%%%%%%%%%%%%%%%%%%%%%%%%%%%%%%%%%%%
\modHeadchapter[loC]{アイソパラメトリック曲線}
%!TEX root = ../RfCPN.tex


\modHeadchapter[loColumn]{曲線の近似とNURBS曲線}
%%%%%%%%%%%%%%%%%%%%%%%%%%%%%%%%%%%%%%%%%%%%%%%%%%%%%%%%%%
%% section E.2 %%%%%%%%%%%%%%%%%%%%%%%%%%%%%%%%%%%%%%%%%%%
%%%%%%%%%%%%%%%%%%%%%%%%%%%%%%%%%%%%%%%%%%%%%%%%%%%%%%%%%%
\modHeadsection{アイソパラメトリック曲線}
\index{アイソパラメトリックきょくせん@アイソパラメトリック曲線}アイソパラメトリック曲線は、その曲線が属する\index{パラメトリックサーフェス}パラメトリックサーフェスの形状に依存する。
一般に、パラメトリックサーフェスは、パラメタ$u$, $v$を用いて次のように表すことができる。
\begin{align*}
  S(u, v) &=
  \left[
  \begin{array}{c}
    x(u, v)\\
    y(u, v)\\
    z(u, v)
  \end{array}
  \right]
\end{align*}
ここで、$x(u,v)$, $y(u,v)$, $z(u,v)$は表面の各座標を定義する関数である。
アイソパラメトリック曲線は、この表面上の一定の$u$または$v$の値に対応する曲線を指す。
したがって、アイソパラメトリック曲線は次のように表すことができる。
\begin{align*}
  C(t) = S\big(u(t), v(t)\big)\ .
\end{align*}
ここで$u(t)$, $v(t)$はパラメタであり、どちらかが一定(例えば$u(t)=k$または$v(t)=k$)で、もう一方のパラメタは$t$に対して変化する。
たとえば$v(t) = k$ ($k = \text{const.}$)とすると、
\begin{align*}
  C(u) = S\big(u(t), k\big)
\end{align*}
と表すことができる。
これを改めて、$u$で記述される基底関数$f_i(u)$と制御点$P_i$を用いて
\begin{align*}
  C(u) = \sum_{i=1}^nf_i(u)P_i
\end{align*}
と表すことにする。
こうすることで、曲面$S(u, v)$の一部を切り出した曲線$C(u)$は、\index{きていかんすう(アイソパラメトリックきょくせん)@基底関数(アイソパラメトリック曲線)}基底関数$f_i(u)$と\index{せいぎょてん(アイソパラメトリックきょくせん)@制御点(アイソパラメトリック曲線)}制御点$P_i$の組み合わせで表現することができる。

ここではこのようなアイソパラメトリック曲線の1つの例として、\index{B-スプラインきょくせん@B-スプライン曲線}B-スプライン曲線を見ていくことにする。



\clearpage
%%%%%%%%%%%%%%%%%%%%%%%%%%%%%%%%%%%%%%%%%%%%%%%%%%%%%%%%%%
%% section E.2 %%%%%%%%%%%%%%%%%%%%%%%%%%%%%%%%%%%%%%%%%%%
%%%%%%%%%%%%%%%%%%%%%%%%%%%%%%%%%%%%%%%%%%%%%%%%%%%%%%%%%%
\modHeadsection{Bスプライン曲線}
\index{きていかんすう(B-スプラインきょくせん)@基底関数(B-スプライン曲線)}基底関数$f_i(u)$として、次のような再帰的に定義される$N_{i, p}(u)$を考える。
\begin{gather*}
  N_{i, p}(u)
  = \frac{u-u_i}{u_{i+p}-u_i}N_{i, p-1}(u)
    +\frac{u_{i+p+1}-u}{u_{i+p+1}-u_{i+1}}N_{i+1, p-1}(u)\\
  N_{i, 0}(u)
  = \left\{
    \begin{array}{l}
      1\quad \big(\text{if}~u_i \leq u < u_{i+1}\big)\\
      0\quad (\text{otherwise})
    \end{array}
    \right.
\end{gather*}
ここで、$u_i$は\index{ノットベクトル}ノットベクトルと呼ばれるベクトル$U$の成分である。
$N_{i, p}(u)$は$u_i \leq u < u_{i+p+1}$の範囲で定義され、この範囲外では$0$となる。
$p$は\index{じすう(B-スプラインきょくせん)@次数(B-スプライン曲線)}次数、また$p+1$は\index{かいすう(B-スプラインきょくせん)@階数(B-スプライン曲線)}階数(order)という。
$N_{i, p}(u)$の定義からわかるように、$N_{i, 0}(u)$は0次であり、$p$が増えるにつれて$u$の1次式がかかる形になるため、$N_{i, p}(u)$は$p$次多項式となる。
よって、この$N_{i, p}(u)$の線形和として与えられる\index{B-スプラインきょくせん@B-スプライン曲線}B-スプライン曲線も$p$次多項式となる。

パラメタ$u$が特定の区間$[u_j , u_{j+1})$にあるとき、0とならないのは$N_{j-p, p}(u)$から$N_{j, p}(u)$のみである。
よって、すべての$N_{i, p}$を足し合わせると、
\begin{align*}
  \sum_{i=1}^nN_{i, p}(u) = \sum_{i=j-p}^j\!\!N_{i, p}(u) = 1\ .
\end{align*}
これは基底関数$N_{i, p}(u)$を用いて与えられるB-スプライン曲線が、その\index{せいぎょてん(B-スプラインきょくせん)@制御点(B-スプライン曲線)}制御点$P_i$の\index{とつけつごう@凸結合}凸結合であることを示している。
実際、このときB-スプライン曲線は
\begin{align*}
  C(u) = \sum_{i=1}^nN_{i, p}(u)P_i\qquad\left(\sum_{i=1}^nN_{i, p}(u) =1\right)
\end{align*}
と表され、これは制御点$P_i$の\index{かじゅうさんじゅつへいきん(B-スプラインきょくせん)@加重算術平均(B-スプライン曲線)}加重算術平均(\index{ウェイトさんじゅつへいきん(B-スプラインきょくせん)@ウェイト算術平均(B-スプライン曲線)}ウェイト算術平均, \index{おもみつきさんじゅつへいきん(B-スプラインきょくせん)@重み付き算術平均(B-スプライン曲線)}重み付き算術平均)とみなすことができる。
つまり、一種の重心のようなものであり、B-スプライン曲線は常に制御点$P_i$に囲まれた領域に存在することを意味する。

~\vfill
%%%%%%%%%%%%%%%%%%%%%%%%%%%%%%%%%%%%%%%%%%%%%%%%%%%%%%%%%%
%% Column %%%%%%%%%%%%%%%%%%%%%%%%%%%%%%%%%%%%%%%%%%%%%%%%
%%%%%%%%%%%%%%%%%%%%%%%%%%%%%%%%%%%%%%%%%%%%%%%%%%%%%%%%%%
\begin{\Columnname}{凸結合}
凸結合とは\index{きかがく@幾何学}幾何学の用語であり、ある点集合に対してその点集合の中の任意の2点を結ぶ線分が全てその点集合の中に含まれる性質を指す。
具体的には、2つの点$P$, $Q$が与えられたとき、それらの間の任意の点$R$はパラメタ$t$を用いて次のように表すことができる。
\begin{align*}
  R = tP+(1-t)Q\qquad\big(0 \leq t \leq 1\big)\ .
\end{align*}
これは点$R$が点$P$と点$Q$の間に位置することを示し、また$t$の値によって点$R$は点$P$と点$Q$の間をなめらかに移動する。
したがって凸結合は、ある点集合の任意の2点を結ぶ線分が全てその点集合の中に含まれるという性質を示す。
すなわち、今の場合は曲線が制御点の範囲内に存在するという性質を示す。
\end{\Columnname}
%%%%%%%%%%%%%%%%%%%%%%%%%%%%%%%%%%%%%%%%%%%%%%%%%%%%%%%%%%
%%%%%%%%%%%%%%%%%%%%%%%%%%%%%%%%%%%%%%%%%%%%%%%%%%%%%%%%%%
%%%%%%%%%%%%%%%%%%%%%%%%%%%%%%%%%%%%%%%%%%%%%%%%%%%%%%%%%%


\clearpage
%%%%%%%%%%%%%%%%%%%%%%%%%%%%%%%%%%%%%%%%%%%%%%%%%%%%%%%%%%
%% subsection E.4.3 %%%%%%%%%%%%%%%%%%%%%%%%%%%%%%%%%%%%%%
%%%%%%%%%%%%%%%%%%%%%%%%%%%%%%%%%%%%%%%%%%%%%%%%%%%%%%%%%%
\subsection{非一様B-スプライン曲線}
\index{ノットかんかく@ノット間隔}ノット間隔$u_{j+1}-u_j$がすべて等しい場合は\index{いちよう(ノットベクトル)@一様(ノットベクトル)}一様(\index{uniform(ノットベクトル)}uniform)であるという。
一般には、\index{ノットベクトル}ノットベクトルは\index{ひいちよう(ノットベクトル)@非一様(ノットベクトル)}非一様(\index{non-uniform(ノットベクトル)}non-uniform)である。
ここでは一様である場合も含めて、\index{ひいちようB-スプラインきょくせん@非一様B-スプライン曲線}非一様B-スプライン曲線と呼ぶこととする。


%%%%%%%%%%%%%%%%%%%%%%%%%%%%%%%%%%%%%%%%%%%%%%%%%%%%%%%%%%
%% subsection E.4.3 %%%%%%%%%%%%%%%%%%%%%%%%%%%%%%%%%%%%%%
%%%%%%%%%%%%%%%%%%%%%%%%%%%%%%%%%%%%%%%%%%%%%%%%%%%%%%%%%%
\subsection{有理B-スプライン曲線}
\index{きていかんすう(B-スプラインきょくせん)@基底関数(B-スプライン曲線)}基底関数として、$N_{i, p}(u)$に対して\index{ウェイト(B-スプラインきょくせん)@ウェイト(B-スプライン曲線)}ウェイト(\index{おもみ(B-スプラインきょくせん)@重み(B-スプライン曲線)}重み)$w_i$\,($>0$)をつけた
\begin{align*}
  \frac{N_{i, p}(u)w_i}{\displaystyle\sum_{j=1}^nN_{j, p}(u)w_j}
\end{align*}
を考える。
これは\index{せいぎょてん(B-スプラインきょくせん)@制御点(B-スプライン曲線)}制御点$P_i$についての\index{かじゅうさんじゅつへいきん(B-スプラインきょくせん)@加重算術平均(B-スプライン曲線)}加重算術平均とみなすことができ、この基底関数の総和は1となることがわかる。
ウェイト$w_i$がすべて1の場合は\index{ひゆうり(きていかんすう)@非有理(基底関数)}非有理(\index{non-rational(きていかんすう)@non-rational(基底関数)}non-rational)であるという。
このときB-スプライン曲線は次のように表すことができる。
\begin{align*}
  C(u) = \sum_{i=1}^nN_{i, p}(u)P_i~~, \quad
  U = \left\{u_0, u_1, u_2, \cdots, u_{n+p+1}\right\}\ .
\end{align*}
一般には、B-スプライン曲線は\index{ゆうり(きていかんすう)@有理(基底関数)}有理(\index{rational(きていかんすう)@rational(基底関数)}rational)である。
ここでは非有理である場合を含めて、\index{ゆうりB-スプラインきょくせん@有理B-スプライン曲線}有理B-スプライン曲線と呼ぶこととする。


%%%%%%%%%%%%%%%%%%%%%%%%%%%%%%%%%%%%%%%%%%%%%%%%%%%%%%%%%%
%% subsection E.1.4 %%%%%%%%%%%%%%%%%%%%%%%%%%%%%%%%%%%%%%
%%%%%%%%%%%%%%%%%%%%%%%%%%%%%%%%%%%%%%%%%%%%%%%%%%%%%%%%%%
\subsection{非一様有理B-スプライン曲線(NURBS曲線)}
以上より、\index{ひいちようゆうりB-スプラインきょくせん@非一様有理B-スプライン曲線}非一様有理B-スプライン(\index{Non-Uniform Rational B-Splineきょくせん@Non-Uniform Rational B-Spline曲線}Non-Uniform Rational B-Spline)曲線は、次のように定義される。
\begin{align*}
  C(u) = \frac{\displaystyle\sum_{i=1}^nN_{i, p}(u)w_iP_i}{\displaystyle\sum_{j=1}^nN_{j, p}(u)w_j}~~,\quad
  U = \left\{u_0, u_1, u_2, \cdots, u_{n+p+1}\right\}\ .
\end{align*}
$U$は\index{ひいちようノットベクトル@非一様ノットベクトル}非一様ノットベクトルである。
これは\index{NURBSきょくせん@NURBS曲線}NURBS曲線と呼ばれる。

~\vfill
%%%%%%%%%%%%%%%%%%%%%%%%%%%%%%%%%%%%%%%%%%%%%%%%%%%%%%%%%%
%% Column %%%%%%%%%%%%%%%%%%%%%%%%%%%%%%%%%%%%%%%%%%%%%%%%
%%%%%%%%%%%%%%%%%%%%%%%%%%%%%%%%%%%%%%%%%%%%%%%%%%%%%%%%%%
\begin{\Columnname}{B-スプライン}
\index{スプライン}スプラインとは製図等に用いられる\index{じざいじょうぎ@自在定規}自在定規の一種で、しなやかで弾力のある細長い板を指す。
平面上の通過すべき点でたわみを支えると、それらを結ぶ滑らかな曲線が得られる。
これは\index{だんせいエネルギー(スプライン)@弾性エネルギー(スプライン)}弾性エネルギーを最小にする曲線であり、数学的には3次のスプライン曲線となる。

 また、B-スプラインのBはBasisを意味する。
これは基底関数の集合から構成されることを示している。

 つまりB-スプライン曲線とは、基底関数の集合から構成されるスプラインを表す曲線、というような意味である。
\end{\Columnname}
%%%%%%%%%%%%%%%%%%%%%%%%%%%%%%%%%%%%%%%%%%%%%%%%%%%%%%%%%%
%%%%%%%%%%%%%%%%%%%%%%%%%%%%%%%%%%%%%%%%%%%%%%%%%%%%%%%%%%
%%%%%%%%%%%%%%%%%%%%%%%%%%%%%%%%%%%%%%%%%%%%%%%%%%%%%%%%%%



\clearpage
%%%%%%%%%%%%%%%%%%%%%%%%%%%%%%%%%%%%%%%%%%%%%%%%%%%%%%%%%%
%% section E.2 %%%%%%%%%%%%%%%%%%%%%%%%%%%%%%%%%%%%%%%%%%%
%%%%%%%%%%%%%%%%%%%%%%%%%%%%%%%%%%%%%%%%%%%%%%%%%%%%%%%%%%
\modHeadsection{近似曲線}
一般的な\index{CADソフトウェア}CADソフトウェアでは、曲線の描画の際に\index{NURBSきょくせん@NURBS曲線}NURBS曲線が用いられることが多い。
改めてNURBS曲線を記述すると、
\begin{align*}
  C(u) = \frac{\displaystyle\sum_{i=1}^nN_{i, p}(u)w_iP_i}{\displaystyle\sum_{j=1}^nN_{j, p}(u)w_j}~~,\quad
  U = \left\{u_0, u_1, u_2, \cdots, u_{n+p+1}\right\}\ .
\end{align*}
このとき、\index{せいぎょてん(NURBSきょくせん)@制御点(NURBS曲線)}制御点$P_i$, \index{ウェイト(NURBSきょくせん)@ウェイト(NURBS曲線)}ウェイト$w_i$, \index{じすう(NURBSきょくせん)@次数(NURBS曲線)}次数$p$, \index{ノットベクトル}ノットベクトル$U$を決定することで、NURBS曲線が描画される。


%%%%%%%%%%%%%%%%%%%%%%%%%%%%%%%%%%%%%%%%%%%%%%%%%%%%%%%%%%
%% subsection E.4.3 %%%%%%%%%%%%%%%%%%%%%%%%%%%%%%%%%%%%%%
%%%%%%%%%%%%%%%%%%%%%%%%%%%%%%%%%%%%%%%%%%%%%%%%%%%%%%%%%%
\subsection{次数および制御点・ウェイトの個数}
$n$個の通過点$Q_j$\,($j = 1, 2, 3, \cdots, n$)および次数が$p$が与えられたとき、一般的なCADソフトウェア等では、制御点およびウェイトの数は$n$, ノットベクトル$U$の成分は$n+p+1$個で
\begin{align*}
  U = \bigg\{
      \underbrace{0, 0, \cdots, 0\bigg.}_{p+1},
      \frac1{n-p}, \frac2{n-p}, \cdots, \frac{n-p-1}{n-p},
      \underbrace{1, 1, \cdots, 1\bigg.}_{p+1}
      \bigg\}
\end{align*}
とされることが多い。
このように設定すると、最初の$p+1$個および最後の$p+1$個の\index{ノット}ノットが同じ値であるため、曲線の両端点が最初と最後の制御点に一致する。
またその間のノットが等間隔であるため、両端点の間の曲線が非有理B-スプライン曲線として描かれる。
%%%%%%%%%%%%%%%%%%%%%%%%%%%%%%%%%%%%%%%%%%%%%%%%%%%%%%%%%%
%% hosoku %%%%%%%%%%%%%%%%%%%%%%%%%%%%%%%%%%%%%%%%%%%%%%%%
%%%%%%%%%%%%%%%%%%%%%%%%%%%%%%%%%%%%%%%%%%%%%%%%%%%%%%%%%%
\begin{hosoku}
制御点$P_i$の数$n$は階数$p+1$以上の必要がある。
曲率まで曲線のなめらかさを制御するには階数4 ($p = 3$)で十分であり、一般的なCADソフトウェアでは階数4の曲線が多く用いられる。
\end{hosoku}
%%%%%%%%%%%%%%%%%%%%%%%%%%%%%%%%%%%%%%%%%%%%%%%%%%%%%%%%%%
%%%%%%%%%%%%%%%%%%%%%%%%%%%%%%%%%%%%%%%%%%%%%%%%%%%%%%%%%%
%%%%%%%%%%%%%%%%%%%%%%%%%%%%%%%%%%%%%%%%%%%%%%%%%%%%%%%%%%


%%%%%%%%%%%%%%%%%%%%%%%%%%%%%%%%%%%%%%%%%%%%%%%%%%%%%%%%%%
%% subsection E.4.3 %%%%%%%%%%%%%%%%%%%%%%%%%%%%%%%%%%%%%%
%%%%%%%%%%%%%%%%%%%%%%%%%%%%%%%%%%%%%%%%%%%%%%%%%%%%%%%%%%
\subsection{制御点およびウェイト}
一般的なCADソフトウェアでは、制御点やウェイトは反復的に計算される。
$n$個のウェイトについては初期値としてすべて$w_i = 1$\,($i = 1, 2, \cdots, n$)とし、$n$個の制御点$P_i$については初期値としてそれぞれ$n$個の通過点$Q_j$とする。
このとき、
\begin{align*}
  C(u) = \frac{\displaystyle\sum_{i=1}^nN_{i, p}(u)w_iP_i}{\displaystyle\sum_{j=1}^nN_{j, p}(u)w_j}
\end{align*}
として初期値に対するNURBS曲線をを得る。
そして、適切な最適化手法を用いて、通過点を通るような最もよく近似するNURBS曲線が与えられるように$w_i$や$P_i$を調整する。


\clearpage
%%%%%%%%%%%%%%%%%%%%%%%%%%%%%%%%%%%%%%%%%%%%%%%%%%%%%%%%%%
%% subsection E.4.3 %%%%%%%%%%%%%%%%%%%%%%%%%%%%%%%%%%%%%%
%%%%%%%%%%%%%%%%%%%%%%%%%%%%%%%%%%%%%%%%%%%%%%%%%%%%%%%%%%
\subsection{最小二乗法\TBW}
ここではその最適化手法として、\index{さいしょうにじょうほう@最小二乗法}最小二乗法を考えることにする。







%%%%%%%%%%%%%%%%%%%%%%%%%%%%%%%%%%%%%%%%%%%%%%%%%%%%%%%%%%
%%            %%%%%%%%%%%%%%%%%%%%%%%%%%%%%%%%%%%%%%%%%%%%
%% Appendix B %%%%%%%%%%%%%%%%%%%%%%%%%%%%%%%%%%%%%%%%%%%%
%%            %%%%%%%%%%%%%%%%%%%%%%%%%%%%%%%%%%%%%%%%%%%%
%%%%%%%%%%%%%%%%%%%%%%%%%%%%%%%%%%%%%%%%%%%%%%%%%%%%%%%%%%
\modHeadchapter{諸公式}
%!TEX root = ../RfCPN.tex


\modHeadchapter{諸公式}



%%%%%%%%%%%%%%%%%%%%%%%%%%%%%%%%%%%%%%%%%%%%%%%%%%%%%%%%%%
%% section F.1 %%%%%%%%%%%%%%%%%%%%%%%%%%%%%%%%%%%%%%%%%%%
%%%%%%%%%%%%%%%%%%%%%%%%%%%%%%%%%%%%%%%%%%%%%%%%%%%%%%%%%%
\modHeadsection{近似計算}
\begin{Formula}[label=formula:taylorexpansion]{テイラー展開(マクローリン展開)}
$f(x)$の$x = 0$に対する\index{テイラーてんかい@テイラー展開}テイラー展開(\index{マクローリンてんかい@マクローリン展開}マクローリン展開)は、以下で与えられる。
\begin{align*}
  f(x) = \sum_{n=1}^{\infty}\frac{f^{(n)}(0)}{n!}x^n\ .
\end{align*}
特に、
\begin{align*}
  (1+x)^\frac12 &= 1+\frac x2-\frac{x^2}8+\frac{x^3}{16}-\frac{5x^4}{128}+o\!\left(x^5\right)\\
  \frac{\cos x}{1+\cos^2x} &= \frac12-\frac{x^4}{16}+o\left(x^6\right)\\
  \frac{\sin x\cos x}{1+\cos^2x} &= \frac x2-\frac{x^3}{12}-\frac{7x^5}{120}+o\left(x^7\right)
\end{align*}
\end{Formula}



%%%%%%%%%%%%%%%%%%%%%%%%%%%%%%%%%%%%%%%%%%%%%%%%%%%%%%%%%%
%% section F.02 %%%%%%%%%%%%%%%%%%%%%%%%%%%%%%%%%%%%%%%%%%
%%%%%%%%%%%%%%%%%%%%%%%%%%%%%%%%%%%%%%%%%%%%%%%%%%%%%%%%%%
\modHeadsection{2直線の関係}
\begin{Formula}[label=formula:intersectionof2lines]{2直線の交点}
2つの直線$y = a_1x+b_1$, $y = a_2x+b_2$ ($a_1 \ne a_2$)の交点は、以下で与えられる。
\begin{align*}
  \left(
  \frac{b_1-b_2}{a_2-a_1}~,~
  \frac{a_1b_2-a_2b_1}{a_2-a_1}
  \right)\ .
\end{align*}
\end{Formula}

\begin{Formula}[label=formula:anblebetween2lines]{2直線のなす角度}
2つの直線$y = m_1x$, $y = m_2x$ ($m_1m_2 \ne -1$)に対して、$m_1 = \tan\alpha$, $m_2 = \tan\beta$とみなすことができる。
したがって、2直線のなす角度$\phi$ ($0 < \phi < \nicefrac\pi2$)は、以下で与えられる。
\begin{align*}
  \tan\phi = \tan(\beta-\alpha) = \frac{m_2-m_1}{1+m_1m_2}\ .
\end{align*}
\end{Formula}



%\clearpage
%%%%%%%%%%%%%%%%%%%%%%%%%%%%%%%%%%%%%%%%%%%%%%%%%%%%%%%%%%%
%%% section F.2 %%%%%%%%%%%%%%%%%%%%%%%%%%%%%%%%%%%%%%%%%%%
%%%%%%%%%%%%%%%%%%%%%%%%%%%%%%%%%%%%%%%%%%%%%%%%%%%%%%%%%%%
%\modHeadsection{2点間の距離}
%\begin{Formula}{点と直線間の距離}
%点($p$, $q$)と直線$ax+by+c=0$との距離$d$は、以下で与えられる。
%\begin{align*}
%  d = \frac{|ap+bq+c|}{\sqrt{a^2+b^2}}.
%\end{align*}
%\end{Formula}
%\begin{Formula}{直線上の点と直線間の距離}
%点$\boldsymbol p$を通り方向ベクトルが$\boldsymbol m$の直線L上の点と、点$\boldsymbol q$を通り方向ベクトルが$\boldsymbol m'$の直線$\mathrm L'$上の点は、それぞれパラメタ$t$, $t'$を用いて、
%\begin{align*}
%  \mathrm L: \boldsymbol p+t\boldsymbol m\ , \qquad
%  \mathrm L': \boldsymbol q+t'\boldsymbol m'
%\end{align*}
%で表される。
%このとき、L上の点の中で$\mathrm L'$に最も近づく点の位置$\boldsymbol k$は、以下で与えられる
%%% footnote %%%%%%%%%%%%%%%%%%%%%
%\footnote{2点間の距離の2乗$|\boldsymbol p-\boldsymbol q+t\boldsymbol m-t'\boldsymbol m'|^2$に対し、それぞれのパラメタ$t$, $t'$に関する微分が0となる。
%それらを連立して解けば$\boldsymbol k$, $\boldsymbol k'$が求まる。}。
%%%%%%%%%%%%%%%%%%%%%%%%%%%%%%%%%%
%$\mathrm L'$上の点の中でLに最も近づく点の位置$\boldsymbol k'$についても同様である。
%\begin{align*}
%  \boldsymbol k
%  = \boldsymbol p
%    +\frac{\left\{\boldsymbol m-(\boldsymbol m, \boldsymbol m')\boldsymbol m', \boldsymbol p-\boldsymbol q\right\}}
%          {1+\left(\boldsymbol m, \boldsymbol m'\right)^2}\boldsymbol m
%\end{align*}
%また、これらの差の大きさ$\big|\boldsymbol k-\boldsymbol k'\big|$から、2直線間の距離$d$が求まる。
%\end{Formula}





%%%%%%%%%%%%%%%%%%%%%%%%%%%%%%%%%%%%%%%%%%%%%%%%%%%%%%%%%%%%%%%%%%%%
%%          %%%%%%%%%%%%%%%%%%%%%%%%%%%%%%%%%%%%%%%%%%%%%%%%%%%%%%%%
%% NOTATION %%%%%%%%%%%%%%%%%%%%%%%%%%%%%%%%%%%%%%%%%%%%%%%%%%%%%%%%
%%          %%%%%%%%%%%%%%%%%%%%%%%%%%%%%%%%%%%%%%%%%%%%%%%%%%%%%%%%
%%%%%%%%%%%%%%%%%%%%%%%%%%%%%%%%%%%%%%%%%%%%%%%%%%%%%%%%%%%%%%%%%%%%
\modHeadchapter[lot]{表記 一覧}
{\small%!TEX root = RfCPN.tex


\modHeadchapter[lot]{表記 一覧}
ここでは\pageautoref{part:AC}で用いた変数等
%% footnote %%%%%%%%%%%%%%%%%%%%%
\footnote{一部、\pageautoref{part:NC}で用いているものも含む。}\relax
%%%%%%%%%%%%%%%%%%%%%%%%%%%%%%%%%
の表記を一覧にしてまとめておく。


%%%%%%%%%%%%%%%%%%%%%%%%%%%%%%%%%%%%%%%%%%%%%%%%%%%%%%%%%%
%% section G.1 %%%%%%%%%%%%%%%%%%%%%%%%%%%%%%%%%%%%%%%%%%
%%%%%%%%%%%%%%%%%%%%%%%%%%%%%%%%%%%%%%%%%%%%%%%%%%%%%%%%%%
\modHeadsection{数式に用いられる記号}

%%%%%%%%%%%%%%%%%%%%%%%%%%%%%%%%%%%%%%%%%%%%%%%%%%%%%%%%%%
%% captionof %%%%%%%%%%%%%%%%%%%%%%%%%%%%%%%%%%%%%%%%%%%%%
%%%%%%%%%%%%%%%%%%%%%%%%%%%%%%%%%%%%%%%%%%%%%%%%%%%%%%%%%%
\begin{multicollongtblr}{\CenterCurvature・\AlocationLength・\OuterDiameter}{cX[l]c}
記号 & 内容 & \Drawing\\
$R_\mathrm c$ & \CenterCurvatureRadius & ○\\
$R_\mathrm i$ & 内側(C面側)湾曲半径 & \\
$R_\mathrm o$ & 外側(A面側)湾曲半径 & \\
$f_\mathrm T$ & \TopAlocationLength(調整前) & ○\\
$f_\mathrm B$ & \BottomAlocationLength(調整前) &○\\
$f_\mathrm T'$ & \TopAlocationLength(調整後) &\\
$f_\mathrm B'$ & \BottomAlocationLength(調整後) &\\
$f_\mathrm d$ & $\displaystyle \frac{f_\mathrm B-f_\mathrm T}2$ &\\
$W_x$ & \ACOD & ○\\
$W_y$ & \BDOD & ○\\
\end{multicollongtblr}

%%%%%%%%%%%%%%%%%%%%%%%%%%%%%%%%%%%%%%%%%%%%%%%%%%%%%%%%%%
%% captionof %%%%%%%%%%%%%%%%%%%%%%%%%%%%%%%%%%%%%%%%%%%%%
%%%%%%%%%%%%%%%%%%%%%%%%%%%%%%%%%%%%%%%%%%%%%%%%%%%%%%%%%%
\begin{multicollongtblr}{\EndFacecut}{cX[l]c}
記号 & 内容 & \Drawing\\
$W_\mathrm T$ & \TopEndHorizontalOD & ○\\
$w_\mathrm T$ & \TopEndID(略称) & ○\\
$w_\mathrm B$ & \BottomEndID(略称) & ○\\
$w_x$ & \EndFaceACID(略称) & ○\\
$w_y$ & \EndFaceBDID(略称) & ○\\
$\delta_w$ & \EndFacecutMilling{} 輪郭調整用パラメタ &\\
$\tau_x$ & \ACThickness(略称) &\\
\end{multicollongtblr}

\clearpage
%%%%%%%%%%%%%%%%%%%%%%%%%%%%%%%%%%%%%%%%%%%%%%%%%%%%%%%%%%
%% captionof %%%%%%%%%%%%%%%%%%%%%%%%%%%%%%%%%%%%%%%%%%%%%
%%%%%%%%%%%%%%%%%%%%%%%%%%%%%%%%%%%%%%%%%%%%%%%%%%%%%%%%%%
\begin{multicollongtblr}{角度}{cX[l]c}
記号 & 内容 & \Drawing\\
$\theta$ & \AlocationAngle &\\
$\theta'$ & \EqualAlocationAngle &\\
$\theta_\mathrm T'$ & \EqualAlocationAngle(\nameTopEndFace{} 工具側) &\\
$\theta_\mathrm B'$ & \EqualAlocationAngle(\nameBottomEndFace{} 工具側) &\\
$\psi$ & スペーサによる$\mathrm U_\mathrm B$を中心とした傾き角 &\\
$\Omega$ & \CurvatureCenter 点Oを原点としたワークの回転角度 &\\
$\alpha_{\mathrm c}$ & \CurvatureCenter$O$と$\mathrm T_\mathrm c$の角度(鋭角) &\\
$\alpha_{\mathrm T_\mathrm i}$ & \CurvatureCenter$O$と$\mathrm T_\mathrm i$の角度(鋭角) &\\
$\alpha_{\mathrm T_\mathrm o}$ & \CurvatureCenter$O$と$\mathrm T_\mathrm o$の角度(鋭角) &\\
$\alpha_{\mathrm U_\mathrm T}$ & \CurvatureCenter$O$と$\mathrm U_\mathrm T$の角度(鋭角) &\\
$\alpha_{\mathrm U_\mathrm B}$ & \CurvatureCenter$O$と$\mathrm U_\mathrm B$の角度(鋭角) &\\
$\alpha_{\mathrm U_\mathrm T}'$ & \CurvatureCenter$O'$と$\mathrm U_\mathrm T'$の角度(鋭角) &\\
$\alpha_{\mathrm U_\mathrm B}'$ & \CurvatureCenter$O'$と$\mathrm U_\mathrm B'$の角度(鋭角) &\\
\end{multicollongtblr}

%\clearpage
%%%%%%%%%%%%%%%%%%%%%%%%%%%%%%%%%%%%%%%%%%%%%%%%%%%%%%%%%%
%% captionof %%%%%%%%%%%%%%%%%%%%%%%%%%%%%%%%%%%%%%%%%%%%%
%%%%%%%%%%%%%%%%%%%%%%%%%%%%%%%%%%%%%%%%%%%%%%%%%%%%%%%%%%
\begin{multicollongtblr}{\Outcut}{cX[l]c}
記号 & 内容 & \Drawing\\
$\mathfrak W_\mathrm T$ & \TopOutcutWidth(略称) &\\
$\mathfrak W_{\mathrm Tx}$ & \TopOutcutACWidth & ○\\
$\mathfrak W_\mathrm B$ & \BottomOutcutWidth(略称) &\\
$\mathfrak W_{\mathrm Bx}$ & \BottomOutcutACWidth & ○\\
$h_\mathrm T$ & \TopOutcutLength & ○\\
$h_\mathrm B$ & \BottomOutcutLength & ○\\
$\tau_\mathrm T$ & \TopAsideThickness(指定時) & ○\\
$\tau_\mathrm B$ & \BottomAsideThickness(指定時) & ○\\
$\mu$ & \PlatingThk & ○\\
\end{multicollongtblr}

\clearpage
%%%%%%%%%%%%%%%%%%%%%%%%%%%%%%%%%%%%%%%%%%%%%%%%%%%%%%%%%%
%% captionof %%%%%%%%%%%%%%%%%%%%%%%%%%%%%%%%%%%%%%%%%%%%%
%%%%%%%%%%%%%%%%%%%%%%%%%%%%%%%%%%%%%%%%%%%%%%%%%%%%%%%%%%
\begin{multicollongtblr}{\Keyway}{cX[l]c}
記号 & 内容 & \Drawing\\
$W_\mathrm M$ & \KeywayDiameter(略称) & ○\\
$W_{mx}$ & \KeywayACOD & ○\\
$\kappa_p$ & \KeywayPos & ○\\
$\kappa_w$ & \KeywayWidth & ○\\
$\bar\kappa_w$ & $\displaystyle\kappa_p+\frac{\kappa_w}2$ &\\
$\kappa_d$ & (\Drawing 上の)\AsideKeywayDepth & ○\\
$\kappa_d'$ & (測定上の)\AsideKeywayDepth &\\
$\kappa_s$ & \TopsideKeywayDepth &\\
$\kappa_l$ & \BottomsideKeywayDepth &\\
$\zeta$ & \KeywayDepthMeasurementAngle &\\
\end{multicollongtblr}

%\clearpage
%%%%%%%%%%%%%%%%%%%%%%%%%%%%%%%%%%%%%%%%%%%%%%%%%%%%%%%%%%
%% captionof %%%%%%%%%%%%%%%%%%%%%%%%%%%%%%%%%%%%%%%%%%%%%
%%%%%%%%%%%%%%%%%%%%%%%%%%%%%%%%%%%%%%%%%%%%%%%%%%%%%%%%%%
\begin{multicollongtblr}{\EndFaceCChamfer}{cX[l]c}
記号 & 内容 & \Drawing\\
$c_\mathrm{To}$ & \TopEndFaceOutCChamferLength & ○\\
$c_\mathrm{Ti}$ & \TopEndFaceInCChamferLength & ○\\
$c_\mathrm{Bo}$ & \BottomEndFaceOutCChamferLength & ○\\
$c_\mathrm{Bi}$ & \BottomEndFaceInCChamferLength & ○\\
$D_\mathrm e$ & \TaperEndMillTipDiameter(直径) & ○\\
$D_\mathrm r$ & \TaperEndMillReferenceDiameter &\\
$d_\mathrm e$ & テーパエンドミル先端平面および参照直径平面間の距離 &\\
$\xi_\mathrm e$ & \TaperEndMillAngle & ○\\
\end{multicollongtblr}

%\clearpage
%%%%%%%%%%%%%%%%%%%%%%%%%%%%%%%%%%%%%%%%%%%%%%%%%%%%%%%%%%
%% captionof %%%%%%%%%%%%%%%%%%%%%%%%%%%%%%%%%%%%%%%%%%%%%
%%%%%%%%%%%%%%%%%%%%%%%%%%%%%%%%%%%%%%%%%%%%%%%%%%%%%%%%%%
\begin{multicollongtblr}{\EndFaceRChamfer}{cX[l]c}
記号 & 内容 & \Drawing\\
$r_\mathrm{To}$ & \TopEndFaceOutRChamferRadius & ○\\
$r_\mathrm{Ti}$ & \TopEndFaceInRChamferRadius & ○\\
$r_\mathrm{Bo}$ & \BottomEndFaceOutRChamferRadius & ○\\
$r_\mathrm{Bi}$ & \BottomFaceInRChamferRadius & ○\\
\end{multicollongtblr}

\clearpage
%%%%%%%%%%%%%%%%%%%%%%%%%%%%%%%%%%%%%%%%%%%%%%%%%%%%%%%%%%
%% captionof %%%%%%%%%%%%%%%%%%%%%%%%%%%%%%%%%%%%%%%%%%%%%
%%%%%%%%%%%%%%%%%%%%%%%%%%%%%%%%%%%%%%%%%%%%%%%%%%%%%%%%%%
\begin{multicollongtblr}{\EndFaceBoring}{cX[l]c}
記号 & 内容 & \Drawing\\
$p_\beta$ & \EndFaceBoring のA面側からの距離 & ○\\
$w_\beta$ & \EndFaceBoringWidth & ○\\
$d_\beta$ & \EndFaceBoringDepth & ○\\
$R_\beta$ & \EndFaceBoringCornerR & ○\\
$h_\beta$ & \EndFaceBoringLength & ○\\
\end{multicollongtblr}

%\clearpage
%%%%%%%%%%%%%%%%%%%%%%%%%%%%%%%%%%%%%%%%%%%%%%%%%%%%%%%%%%
%% captionof %%%%%%%%%%%%%%%%%%%%%%%%%%%%%%%%%%%%%%%%%%%%%
%%%%%%%%%%%%%%%%%%%%%%%%%%%%%%%%%%%%%%%%%%%%%%%%%%%%%%%%%%
\begin{multicollongtblr}{\Dimple(主に傾き前)}{cX[l]c}
記号 & 内容 & \Drawing\\
$m$ & \DimpleRowNum(1面) & ○\\
$q$ & トップ端から\DimpleFirstRow までの距離 & ○\\
$p_x$ & \DimpleHorizontalPitch & ○\\
$p_z$ & \DimpleVerticalPitch & ○\\
$d_i$ & \DimpleIRowLength & ○\\
$d_\mathrm o$ & \DimpleOddRowLength(全て同じ場合) & ○\\
$d_\mathrm e$ & \DimpleEvenRowLength(全て同じ場合) & ○\\
$n_m$ & $m$列目の\DimpleNum(1面) & ○\\
$n_\mathrm o$ & \OddRowDimpleNum(全て同じ場合) & ○\\
$n_\mathrm e$ & \EvenRowDimpleNum(全て同じ場合) & ○\\
$\lambda_i$ & \IDTaperTable におけるトップ端からの$i$番目の距離 & ○\\
$w_{\mathrm Ai}$ & $\lambda_i$に対する\ACID & ○\\
$w_{\mathrm Bi}$ & $\lambda_i$に対する\BDID & ○\\
$w_{\mathrm A\lambda}$ & トップ端からの距離$\lambda$に対する\expandafterindex{きんじ\yomiACID@近似\nameACID}近似\nameACID &\\
$w_{\mathrm B\lambda}$ & トップ端からの距離$\lambda$に対する\expandafterindex{きんじ\yomiBDID@近似\nameBDID}近似\nameBDID &\\
$w_{\mathrm A\lambda}'$ & $w_{\mathrm A\lambda}+2\mu$ &\\
$w_{\mathrm B\lambda}'$ & $w_{\mathrm B\lambda}+2\mu$ &\\
$w_{\mathrm A}'$ & $w_{\mathrm A\lambda}'$の略記 &\\
$w_{\mathrm B}'$ & $w_{\mathrm B\lambda}'$の略記 &\\
$\phi$ & \DimpleAngle(鋭角) &\\
$\phi_\mathrm C$ & \CsideDimpleAngle(鋭角) &\\
$\mathcal R_\mathrm o$ & \AsideInnerCurvature(略称) &\\
$\mathcal R_\mathrm i$ & \CsideInnerCurvature(略称) &\\
\end{multicollongtblr}

\clearpage
%%%%%%%%%%%%%%%%%%%%%%%%%%%%%%%%%%%%%%%%%%%%%%%%%%%%%%%%%%
%% captionof %%%%%%%%%%%%%%%%%%%%%%%%%%%%%%%%%%%%%%%%%%%%%
%%%%%%%%%%%%%%%%%%%%%%%%%%%%%%%%%%%%%%%%%%%%%%%%%%%%%%%%%%
\begin{multicollongtblr}{\Dimple(主に傾き後)}{cX[l]c}
記号 & 内容 & \Drawing\\
$g_\mathrm T'$ & 傾き後のトップ端内面中心 &\\
$g_{\mathrm Tx}'$ & 傾き後のトップ端内面中心$X$ &\\
$g_{\mathrm Ty}'$ & 傾き後のトップ端内面中心$Y$ &\\
$g_{\mathrm TZ}'$ & 傾き後のトップ端内面中心$Z$ &\\
$\mathcal L_i$ & $i$列目の\CurvatureCenter と\TopCurvatureCenter との差$X$ &\\
$\mathcal L_{i,j}$ & $i$列目の\CurvatureCenter と$j$列目の\CurvatureCenter との差$X$ &\\
$\mathcal D_{xi,\mathrm A}$ & \AFaceDimpleIRowJ$X$(P原点・傾き前) &\\
$\mathcal D_{xi,\mathrm C}$ & \CFaceDimpleIRowJ$X$(P原点・傾き前) &\\
$\mathcal D_{xij,\mathrm B}$ & \BFaceDimpleIRowJ, \DFaceDimpleIRowJ$X$(P原点・傾き前) &\\
$\mathcal D_{yij,\mathrm A}$ & \AFaceDimpleIRowJ, \CFaceDimpleIRowJ$Y$(P原点・傾き前) &\\
$\mathcal D_{yi,\mathrm B}$ & \BFaceDimpleIRowJ$Y$(P原点・傾き前) &\\
$\mathcal D_{yi,\mathrm D}$ & \DFaceDimpleIRowJ$Y$(P原点・傾き前) &\\
$\mathcal D_{zi}$ & \DimpleIRowJ$Z$(P原点・傾き前) &\\
$\mathcal D_{xij,\mathrm A}'$ & \AFaceDimpleIRowJ$X$(P原点・傾き後) &\\
$\mathcal D_{yij,\mathrm A}'$ & \AFaceDimpleIRowJ$Y$(P原点・傾き後) &\\
$\mathcal D_{zij,\mathrm A}'$ & \AFaceDimpleIRowJ$Z$(P原点・傾き後) &\\
$\mathcal D_{xij,\mathrm C}'$ & \CFaceDimpleIRowJ$X$(P原点・傾き後) &\\
$\mathcal D_{yij,\mathrm C}'$ & \CFaceDimpleIRowJ$Y$(P原点・傾き後) &\\
$\mathcal D_{zij,\mathrm C}'$ & \CFaceDimpleIRowJ$Z$(P原点・傾き後) &\\
$\mathcal D_{xij,\mathrm B}'$ & \BFaceDimpleIRowJ$X$(P原点・傾き後) &\\
$\mathcal D_{yij,\mathrm B}'$ & \BFaceDimpleIRowJ$Y$(P原点・傾き後) &\\
$\mathcal D_{zij,\mathrm B}'$ & \BFaceDimpleIRowJ$Z$(P原点・傾き後) &\\
$\mathcal D_{xij,\mathrm D}'$ & \DFaceDimpleIRowJ$X$(P原点・傾き後) &\\
$\mathcal D_{yij,\mathrm D}'$ & \DFaceDimpleIRowJ$Y$(P原点・傾き後) &\\
$\mathcal D_{zij,\mathrm D}'$ & \DFaceDimpleIRowJ$Z$(P原点・傾き後) &\\
\end{multicollongtblr}

%\clearpage
%%%%%%%%%%%%%%%%%%%%%%%%%%%%%%%%%%%%%%%%%%%%%%%%%%%%%%%%%%
%% captionof %%%%%%%%%%%%%%%%%%%%%%%%%%%%%%%%%%%%%%%%%%%%%
%%%%%%%%%%%%%%%%%%%%%%%%%%%%%%%%%%%%%%%%%%%%%%%%%%%%%%%%%%
\begin{multicollongtblr}{\CenterlineEndFaceDif}{cX[l]c}
記号 & 内容 & \Drawing\\
$T_x$ & \CenterlineEndFaceDifAC & ○\\
$T_y$ & \CenterlineEndFaceDifBD & ○\\
\end{multicollongtblr}

\clearpage
\begin{multicollongtblr}{実測値(計算値含む)}{cX[l]c}
記号 & 内容 & \Drawing\\
$G_{\mathrm Tx}$ & 外側中心の$X$座標 &\\
$G_{\mathrm Ty}$ & 外側中心の$Y$座標 &\\
$\mathcal G_\mathrm B$ & ボトム外削中心$\mathfrak B_\mathrm c'$ &\\
$\mathcal G_{\mathrm Bx}$ & ボトム外削中心$\mathfrak B_\mathrm c'$の$X$座標 &\\
$\mathcal G_{\mathrm By}$ & ボトム外削中心$\mathfrak B_\mathrm c'$の$Y$座標 &\\
$\mathcal G_{\mathrm Tx}$ & トップ外削中心$\mathfrak T_\mathrm c'$の$X$座標 &\\
$G_{mx}$ & \KeywayCenter M$'$の$X$座標 &\\
$\mathcal G_m$ & \KeywayCenter におけるA側外面の$X$座標 &\\
$g_\mathrm T$ & \TopIDCenter &\\
$g_{\mathrm Tx}$ & \TopIDCenter の$X$座標 &\\
$g_{\mathrm Ty}$ & \TopIDCenter の$Y$座標 &\\
\end{multicollongtblr}

\begin{multicollongtblr}{\Jig}{cX[l]c}
記号 & 内容 & \Drawing\\
$l$ & \JigLength の半分 & ○\\
$\rho$ & \ReceiverPlateRadius & ○\\
$\sigma$ & \ReceiverPlateWidth & ○\\
$R_\mathrm i'$ & $R_i-\rho$ &\\
$\bar l$ & $\displaystyle l-\frac\sigma2$ &\\
$l'$ & $l+f_\mathrm d$ &\\
$\delta_\mathrm s$ & \AlocationLength 調整用\SpacerThickness &\\
$\Delta$ & $\mathrm U_\mathrm B$と\TableCenter Pとの水平距離 &\\
$\Delta'$ & Oを原点とした\TableCenter Pの水平距離 &\\
$\Delta_y$ & \Jig の\index{ワーク}ワークが乗る部分の$Y$座標 & ◯\\
\end{multicollongtblr}

%\clearpage
\begin{multicollongtblr}{工具}{cX[l]c}
記号 & 内容 & \Drawing\\
$\phi_\mathrm D$ & \EndFacecutMilling 用\FaceMillDC(直径) & ○\\
$\phi_\mathrm D'$ & \EndFacecutMilling 用\FaceMillDCX(直径) & ○\\
$d_\mathrm e$ & テーパエンドミル先端からの距離(参照直径用) &\\
$D_\mathrm r$ & \TaperEndMillReferenceDiameter &\\
$D_\mathrm e$ & \TaperEndMillTipDiameter(直径) & ○\\
$\xi_\mathrm e$ & \TaperEndMillAngle & ○\\
\end{multicollongtblr}

\clearpage
\begin{multicollongtblr}{eテーパ}{cX[l]c}
記号 & 内容 & 参照\\
$T_\mathrm l$ & \index{えきそうせんおんど@液相線温度}液相線温度 &\\
$T_\mathrm s$ & \index{こそうせんおんど@固相線温度}固相線温度 &\\
$T_0$ & \index{きじゅんおんど@基準温度}基準温度 & ○\\
$X_\mathrm C$ & 炭素Cの含有量 & ○\\
$X_\mathrm{Si}$ & ケイ素Siの含有量 & ○\\
$X_\mathrm{Mn}$ & マンガンMnの含有量 & ○\\
$X_\mathrm P$ & リンPの含有量 & ○\\
$X_\mathrm S$ & 硫黄Sの含有量 & ○\\
$k_\mathrm C$ & 炭素Cにおける影響係数 & ○\\
$k_\mathrm{Si}$ & ケイ素Siにおける影響係数 & ○\\
$k_\mathrm{Mn}$ & マンガンMnにおける影響係数 & ○\\
$k_\mathrm P$ & リンPにおける影響係数 & ○\\
$k_\mathrm S$ & 硫黄Sにおける影響係数 & ○\\
\end{multicollongtblr}

%\clearpage
\begin{multicollongtblr}{\index{アイソパラメトリックきょくせん@アイソパラメトリック曲線}アイソパラメトリック曲線}{cX[l]c}
記号 & 内容 & 参照\\
$S(u, v)$ & \index{パラメトリックサーフェス}パラメトリックサーフェス &\\
$u, v$ & \index{パラメタ(パラメトリックサーフェス)}パラメトリックサーフェスのパラメタ &\\
$C(t)$ & \index{アイソパラメトリックきょくせん@アイソパラメトリック曲線}アイソパラメトリック曲線 &\\
$f_i$ & \index{きていかんすう(アイソパラメトリックきょくせん)@基底関数(アイソパラメトリック曲線)}アイソパラメトリック曲線の基底関数 &\\
$P_i$ & \index{せいぎょてん(アイソパラメトリックきょくせん)@制御点(アイソパラメトリック曲線)}アイソパラメトリック曲線の制御点 &\\
$N_{i, p}$ & \index{きていかんすう(B-スプラインきょくせん)@基底関数(B-スプライン曲線)}B-スプライン曲線の基底関数 &\\
$U$ & \index{ノットベクトル}ノットベクトル &\\
$u_i$ & \index{ノットベクトルのせいぶん@ノットベクトルの成分}ノットベクトルの成分 &\\
$p$ & \index{じすう(きていかんすう)@次数(基底関数)}基底関数$N_{i, p}$の次数 &\\
$w_i$ & \index{ウェイト(きていかんすう)@ウェイト(基底関数)}基底関数$N_{i, p}$のウェイト &\\
\end{multicollongtblr}

\begin{multicollongtblr}{その他}{cX[l]c}
記号 & 内容 & 参照\\
$\delta x$ & \TableCenter と\JigCenter とのずれ$X$ &\\
$\delta z$ & \TableCenter と\JigCenter とのずれ$Z$ &\\
\end{multicollongtblr}



\clearpage
%%%%%%%%%%%%%%%%%%%%%%%%%%%%%%%%%%%%%%%%%%%%%%%%%%%%%%%%%%
%% section G.2 %%%%%%%%%%%%%%%%%%%%%%%%%%%%%%%%%%%%%%%%%%%
%%%%%%%%%%%%%%%%%%%%%%%%%%%%%%%%%%%%%%%%%%%%%%%%%%%%%%%%%%
\modHeadsection{点・位置を示す記号}

\begin{multicollongtblr}{位置}{cX[l]c}
記号 & 内容 & 参照\\
O & \CurvatureCenter 点 &\\
P & \TableCenter 点 &\\
T$_{R_\mathrm c}$ & \TopCurvatureCenter &\\
$\mathrm T_\mathrm i$ & トップ内側(C側)端点 &\\
$\mathrm T_\mathrm o$ & トップ外側(A側)端点 &\\
$\mathrm B_\mathrm i$ & ボトム内側(C側)端点 &\\
$\mathrm B_\mathrm o$ & ボトム外側(A側)端点 &\\
$\mathrm U_\mathrm T$ & トップ側受板中心 &\\
$\mathrm U_\mathrm B$ & ボトム側受板中心 &\\
$O'$ & $-\psi$傾き後の\CurvatureCenter 点 &\\
$\mathrm U_\mathrm T'$ & $-\psi$傾き後のトップ側受板中心 &\\
$\mathrm U_\mathrm B'$ & $-\psi$傾き後のボトム側受板中心 &\\
T$_{R_\mathrm c}'$ & 傾き後の\TopCurvatureCenter &\\
T$_\mathrm c'$ & 傾き後のトップ端における外側中心 &\\
$\mathrm T_\mathrm i'$ & 傾き後のトップ内側(C側)端点 &\\
$\mathrm T_\mathrm o'$ & 傾き後のトップ外側(A側)端点 &\\
B$_{R_\mathrm c}'$ & 傾き後の\BottomCurvatureCenter &\\
B$_\mathrm c'$ & 傾き後のボトム端における外側中心 &\\
B$_{R_\mathrm i}'$ & 傾き後のボトム内側(C側)端点 &\\
B$_{R_\mathrm o}'$ & 傾き後のボトム外側(A側)端点 &\\
$\mathfrak T_\mathrm c$ & \TopOutcutCenter &\\
$\mathfrak B_\mathrm c$ & \BottomOutcutCenter &\\
$\mathfrak T_\mathrm c'$ & \TopOutcutCenter(P原点) &\\
$\mathfrak B_\mathrm c'$ & \BottomOutcutCenter(P原点) &\\
$\mathfrak B_\mathrm o'$ & ボトム外削A面 &\\
b$_\mathrm c'$ & ボトム端内面中心 &\\
b$_\mathrm o'$ & ボトム端A側内面 &\\
t$_\mathrm o'$ & トップ端A側内面 &\\
M & \KeywayCenter &\\
M$'$ & 傾き後の\KeywayCenter &\\
\end{multicollongtblr}

\clearrightpage
}


\end{appendices}
%%%%%%%%%%%%%%%%%%%%%%%%%%%%%%%%%%%%%%%%%%%%%%%%%%%%%%%%%
%%          %%%%%%%%%%%%%%%%%%%%%%%%%%%%%%%%%%%%%%%%%%%%%
%% Appendix %%%%%%%%%%%%%%%%%%%%%%%%%%%%%%%%%%%%%%%%%%%%%
%% Part ACN %%%%%%%%%%%%%%%%%%%%%%%%%%%%%%%%%%%%%%%%%%%%%
%% End      %%%%%%%%%%%%%%%%%%%%%%%%%%%%%%%%%%%%%%%%%%%%%
%%          %%%%%%%%%%%%%%%%%%%%%%%%%%%%%%%%%%%%%%%%%%%%%
%%%%%%%%%%%%%%%%%%%%%%%%%%%%%%%%%%%%%%%%%%%%%%%%%%%%%%%%%
\addtocontents{toc}{\protect\end{tocBox}}
\clearrightpage

\addtocontents{toc}{\protect\end{tcolorbox}}
\addtocontents{toc}{\protect\cleardoublepage}% add page break




%%%%%%%%%%%%%%%%%%%%%%%%%%%%%%%%%%%%%%%%%%%%%%%%%%%%%%%%%
%%         %%%%%%%%%%%%%%%%%%%%%%%%%%%%%%%%%%%%%%%%%%%%%%
%%         %%%%%%%%%%%%%%%%%%%%%%%%%%%%%%%%%%%%%%%%%%%%%%
%% Part NC %%%%%%%%%%%%%%%%%%%%%%%%%%%%%%%%%%%%%%%%%%%%%%
%%         %%%%%%%%%%%%%%%%%%%%%%%%%%%%%%%%%%%%%%%%%%%%%%
%%         %%%%%%%%%%%%%%%%%%%%%%%%%%%%%%%%%%%%%%%%%%%%%%
%%%%%%%%%%%%%%%%%%%%%%%%%%%%%%%%%%%%%%%%%%%%%%%%%%%%%%%%%
\addtocontents{toc}{\protect\begin{tcolorbox}[parttocstyle={Contents}{\the\numexpr\value{part}+1\relax}]}
\tPart{解析計算に基づく数値解析}{概要}{%
\paragraph*{目標(なにがしたいか?)}
明細ごとに異なる寸法・形状を持つすべてのモールドに対し、マシニングでの加工に必要な\textbf{数値情報および条件分岐情報等が自動的に得られるシステム}を構築する。
\tcbline*
\paragraph*{手段(どうやって?)}
前段階で導出した\textbf{解析的な情報}を用いて、各明細における具体的な\textbf{数値的な情報}に変換するシステムの構築を試みる。
\tcbline*
\paragraph*{背景(なぜ?)}
一般に、個々の製品や工具等の寸法は異なり、明細ごとに固有の寸法・形状を持つ。
実際の加工(プログラムの作成)の際には、それらをすべて考慮して計算した\textbf{具体的な数値情報}を指定する必要がある。

 こうした数値情報は\textbf{明細ごとに膨大にある}が、現時点(\DMname 設置時点)において、こうした手続きは\textbf{明細ごとに手作業}で行われている
%% footnote %%%%%%%%%%%%%%%%%%%%%
\footnote{さらには、「\index{ずめん@図面}図面の作成後、それを見てプログラムを作成する」という明らかに異常な(奇妙な)事態が放置され続けている。}。
%%%%%%%%%%%%%%%%%%%%%%%%%%%%%%%%%

 したがって、こうした手続きのシステム化を行い、可能な限り\textbf{自動化}することが喫緊の課題である。
そうすることで、\textbf{危険を伴う作業の削減(安全性の向上)}や、\textbf{品質の低下の防止}に大きく寄与できることが期待される。
また副次的効果として、作業効率・人的資源・安全(security)・保守などのいずれの観点からみた\textbf{能率の低下の防止}にも大きく貢献することも自ずと期待される。
\tcbline*
\paragraph*{結論(どうなった?)}
各明細のモールドにおける固有数値情報の入力により、マシニングでの加工に必要な数値情報および条件分岐情報等が(人による手動計算を介することなく)自動的に得られるシステムを構築した。
}
%!TEX root = ./RPA_for_Creating_Program_Note.tex





%%%%%%%%%%%%%%%%%%%%%%%%%%%%%%%%%%%%%%%%%%%%%%%%%%%%%%%%%%
%%            %%%%%%%%%%%%%%%%%%%%%%%%%%%%%%%%%%%%%%%%%%%%
%% chapter 32 %%%%%%%%%%%%%%%%%%%%%%%%%%%%%%%%%%%%%%%%%%%%
%%            %%%%%%%%%%%%%%%%%%%%%%%%%%%%%%%%%%%%%%%%%%%%
%%%%%%%%%%%%%%%%%%%%%%%%%%%%%%%%%%%%%%%%%%%%%%%%%%%%%%%%%%
\modHeadchapter{数値計算}
%!TEX root = ./RPA_for_Creating_Program_Note.tex


基本的に、\index{すうちじょうほう@数値情報}数値情報については数値計算用の言語を用いて行うため、その詳細は別ドキュメントに譲る。
ここでは各明細用の\index{メインプログラム}メインプログラムの記述に際して、\index{すうちけいさん@数値計算}数値計算に必要な部分をピックアップする。
なお、ここでは主に\DMname について述べるため、\index{スペーサ}スペーサに関するものは省略する。


%%%%%%%%%%%%%%%%%%%%%%%%%%%%%%%%%%%%%%%%%%%%%%%%%%%%%%%%%%
%% section 30.1 %%%%%%%%%%%%%%%%%%%%%%%%%%%%%%%%%%%%%%%%%%
%%%%%%%%%%%%%%%%%%%%%%%%%%%%%%%%%%%%%%%%%%%%%%%%%%%%%%%%%%
\modHeadsection{振分長・張出長・均等振分角の数値情報}
各パラメータを以下とする。
\begin{align*}
  \varDelta_x' = \varDelta_x+\sqrt{R_\mathrm i'-\bar l^2}\ , \quad
  R_\mathrm i' = R_\mathrm c-\frac{W_x}2-\rho\ ,\quad
  \bar l = l-\frac\sigma2\ ,\quad
  f_d = \frac{f_\mathrm B-f_\mathrm T}2\ .
\end{align*}


%%%%%%%%%%%%%%%%%%%%%%%%%%%%%%%%%%%%%%%%%%%%%%%%%%%%%%%%%%
%% subsection 30.1.1 %%%%%%%%%%%%%%%%%%%%%%%%%%%%%%%%%%%%%
%%%%%%%%%%%%%%%%%%%%%%%%%%%%%%%%%%%%%%%%%%%%%%%%%%%%%%%%%%
\subsection{再振分長}
\index{テーブル}テーブルを$-\theta$だけ回転させて調整したトップ・ボトム側の\index{さいふりわけちょう@再振分長}振分長$f'_\mathrm T$, $f'_\mathrm B$は、\pageeqref{eq:saifuriwake}より、
\begin{align*}
  \text{トップ端:}\quad
  & \HLbox{f_\mathrm T' = f_\mathrm T+\varDelta_x'\!\sin\theta}\ ,\\
  \text{ボトム端:}\quad
  & \HLbox{f_\mathrm B' = (f_\mathrm T+f_\mathrm B)-f_\mathrm T'}\ .
\end{align*}


%%%%%%%%%%%%%%%%%%%%%%%%%%%%%%%%%%%%%%%%%%%%%%%%%%%%%%%%%%
%% subsection 30.1.2 %%%%%%%%%%%%%%%%%%%%%%%%%%%%%%%%%%%%%
%%%%%%%%%%%%%%%%%%%%%%%%%%%%%%%%%%%%%%%%%%%%%%%%%%%%%%%%%%
\subsection{再張出長}
テーブルを$-\theta$だけ回転させた後の\index{ジグ}ジグ(長さ$2l$)からの\index{さいはりだしちょう@再張出長}張出長に換算すると、それぞれ
\begin{align*}
  \text{トップ端:}\quad
  & \HLbox{f_\mathrm T'-l}\ ,\\
  \text{ボトム端:}\quad
  & \HLbox{f_\mathrm B'-l}\ .
\end{align*}


%%%%%%%%%%%%%%%%%%%%%%%%%%%%%%%%%%%%%%%%%%%%%%%%%%%%%%%%%%
%% subsection 30.1.3 %%%%%%%%%%%%%%%%%%%%%%%%%%%%%%%%%%%%%
%%%%%%%%%%%%%%%%%%%%%%%%%%%%%%%%%%%%%%%%%%%%%%%%%%%%%%%%%%
\subsection{均等振分角}
トップ側とボトム側の振分長が均等かつ平行になるときの\index{かたむきかく(ふりわけちょうせい)@傾き角(振分調整)}回転角$\theta'$は、\pageeqref{eq:saifuriwakeangle}より、
\begin{align*}
  \theta' = \HLbox{\sin^{-1}\frac{f_d}{\varDelta_x'}}\ .
\end{align*}



\clearpage
%%%%%%%%%%%%%%%%%%%%%%%%%%%%%%%%%%%%%%%%%%%%%%%%%%%%%%%%%%
%% section 30.2 %%%%%%%%%%%%%%%%%%%%%%%%%%%%%%%%%%%%%%%%%%
%%%%%%%%%%%%%%%%%%%%%%%%%%%%%%%%%%%%%%%%%%%%%%%%%%%%%%%%%%
\modHeadsection{外径中心・湾曲中心・内径中心の数値情報}


%%%%%%%%%%%%%%%%%%%%%%%%%%%%%%%%%%%%%%%%%%%%%%%%%%%%%%%%%%
%% subsection 30.2.1 %%%%%%%%%%%%%%%%%%%%%%%%%%%%%%%%%%%%%
%%%%%%%%%%%%%%%%%%%%%%%%%%%%%%%%%%%%%%%%%%%%%%%%%%%%%%%%%%
\subsection{外側中心\texorpdfstring{$Y$}{Y}}
\index{ジグ}ジグの底の$Y$座標を$\varDelta_y$とすると、\index{そとがわちゅうしんY@外側中心$Y$}外側中心$Y$座標は、
\begin{align*}
  \HLbox{\varDelta_y+\frac{W_y}2}\ .
\end{align*}


%%%%%%%%%%%%%%%%%%%%%%%%%%%%%%%%%%%%%%%%%%%%%%%%%%%%%%%%%%
%% subsection 30.2.1 %%%%%%%%%%%%%%%%%%%%%%%%%%%%%%%%%%%%%
%%%%%%%%%%%%%%%%%%%%%%%%%%%%%%%%%%%%%%%%%%%%%%%%%%%%%%%%%%
\subsection{端面の外側中心\texorpdfstring{$X$}{X}}
トップ端およびボトム端における($-\theta$回転後の)\index{そとがわちゅうしん(たんめん)@外側中心(端面)}外側中心の$X$位置は、\pageeqref{eq:tableTc}, \pageeqref{eq:tableBc}よりそれぞれ、
\begin{align*}
  \text{トップ側:}\quad
  & \HLbox{%
      \frac{\sqrt{R_\mathrm o^2-f_\mathrm T^2}+\sqrt{R_\mathrm i^2-f_\mathrm T^2}}2-\varDelta_x'\cos\theta%
    }~,\\
  \text{ボトム側:}\quad
  & \HLbox{%
      \varDelta_x'\cos\theta-\frac{\sqrt{R_\mathrm o^2-f_\mathrm B^2}+\sqrt{R_\mathrm i^2-f_\mathrm B^2}}2%
    }\ .
\end{align*}


%%%%%%%%%%%%%%%%%%%%%%%%%%%%%%%%%%%%%%%%%%%%%%%%%%%%%%%%%%
%% subsection 30.3.1 %%%%%%%%%%%%%%%%%%%%%%%%%%%%%%%%%%%%%
%%%%%%%%%%%%%%%%%%%%%%%%%%%%%%%%%%%%%%%%%%%%%%%%%%%%%%%%%%
\subsection{端面の湾曲中心\texorpdfstring{$X$}{X}}
トップ端およびボトム端における($-\theta$回転後の)\index{わんきょくちゅうしん(たんめん)@湾曲中心(端面)}湾曲中心の$X$値は、\pageeqref{eq:tableTRc}, \pageeqref{eq:tableBRc}よりそれぞれ、
\begin{align*}
  \text{トップ側:}\quad
  & \HLbox{\sqrt{R_\mathrm c^2-f_\mathrm T^2}-\varDelta_x'\!\cos\theta}~,\\
  \text{ボトム側:}\quad
  & \HLbox{\varDelta_x'\!\cos\theta-\sqrt{R_\mathrm c^2-f_\mathrm B^2}}~.
\end{align*}


%%%%%%%%%%%%%%%%%%%%%%%%%%%%%%%%%%%%%%%%%%%%%%%%%%%%%%%%%%
%% subsection 30.3.1 %%%%%%%%%%%%%%%%%%%%%%%%%%%%%%%%%%%%%
%%%%%%%%%%%%%%%%%%%%%%%%%%%%%%%%%%%%%%%%%%%%%%%%%%%%%%%%%%
\subsection{端面の内側中心}
(計算上の)\index{うちがわちゅうしん(たんめん)@内側中心(端面)}端面の内側中心は、端面の湾曲中心をもって代用してもよいものとする。



\clearpage
%%%%%%%%%%%%%%%%%%%%%%%%%%%%%%%%%%%%%%%%%%%%%%%%%%%%%%%%%%
%% section 30.3 %%%%%%%%%%%%%%%%%%%%%%%%%%%%%%%%%%%%%%%%%%
%%%%%%%%%%%%%%%%%%%%%%%%%%%%%%%%%%%%%%%%%%%%%%%%%%%%%%%%%%
\modHeadsection{外削の数値情報}


%%%%%%%%%%%%%%%%%%%%%%%%%%%%%%%%%%%%%%%%%%%%%%%%%%%%%%%%%%
%% subsection 9.3.1 %%%%%%%%%%%%%%%%%%%%%%%%%%%%%%%%%%%%%%
%%%%%%%%%%%%%%%%%%%%%%%%%%%%%%%%%%%%%%%%%%%%%%%%%%%%%%%%%%
\subsection{外削中心:ボトムA面肉厚基準の場合}
\index{テーブルちゅうしん@テーブル中心}テーブル中心\index{P(テーブルちゅうしん)@P(テーブル中心)}Pを\index{げんてんP@原点P}原点とした\index{ボトムがわのがいさくちゅうしん@ボトム側の外削中心}ボトム側外削中心$\mathfrak B_\mathrm c'$の(おおよその)$X$座標は、\pageeqref{eq:gaisakucenterBt}より、
\begin{align*}
  \HLbox{%
    \varDelta_x'\cos\theta
    -\frac{\sqrt{R_\mathrm o^2-f_\mathrm B^2}+\sqrt{R_\mathrm i^2-f_\mathrm B^2}}2
    -\frac{w_\mathrm B}2
    -\tau_\mathrm B
    +\frac{\mathfrak W_\mathrm B}2
  }\ .
\end{align*}
このとき、計測したA側内面b$_\mathrm o'$の$X$座標が\pageeqref{eq:gaisakucenterBr}となるように、原点$\mathfrak B_\mathrm c'$を定める。
\begin{align*}
  \HLbox{-\left(\frac{\mathfrak W_\mathrm B}2-\tau_\mathrm B+\mu\right)}.
\end{align*}
トップ側にも外削がある場合、計測で定めた$\mathfrak B_\mathrm c'$の$X$座標$\mathcal G_{\mathrm Bx}$および\index{とおりしん@通り芯}通り芯$T_x$を用いて\pageeqref{eq:BbasedTx}で与えられる。
\begin{align*}
  \HLbox{-\mathcal G_{Bx}+T_x}\ .
\end{align*}


%%%%%%%%%%%%%%%%%%%%%%%%%%%%%%%%%%%%%%%%%%%%%%%%%%%%%%%%%%
%% subsection 30.3.2 %%%%%%%%%%%%%%%%%%%%%%%%%%%%%%%%%%%%%
%%%%%%%%%%%%%%%%%%%%%%%%%%%%%%%%%%%%%%%%%%%%%%%%%%%%%%%%%%
\subsection{外削中心:トップA面肉厚基準の場合}
\index{テーブルちゅうしん@テーブル中心}テーブル中心\index{P(テーブルちゅうしん)@P(テーブル中心)}Pを\index{げんてんP@原点P}原点とした\index{トップがわのがいさくちゅうしん@トップ側の外削中心}トップ側外削中心$\mathfrak T_\mathrm c'$の(おおよその)$X$座標は、\pageeqref{eq:gaisakucenterTt}より、
\begin{align*}
  \HLbox{%
    \frac{\sqrt{R_\mathrm o^2-f_\mathrm T^2}+\sqrt{R_\mathrm i^2-f_\mathrm T^2}}2
    -\varDelta_x'\cos\theta
    +\frac{w_\mathrm T}2
    +\tau_\mathrm T
    -\frac{\mathfrak W_\mathrm T}2
  }\ .
\end{align*}
このとき、計測したA側内面t$_\mathrm o'$の$X$座標が\pageeqref{eq:gaisakucenterTr}となるように、原点$\mathfrak T_\mathrm c'$を定める。
\begin{align*}
  \HLbox{\frac{\mathfrak W_\mathrm T}2-\tau_\mathrm T+\mu}~.
\end{align*}
ボトム側にも外削がある場合、計測で定めた$\mathfrak T_\mathrm c'$の$X$座標$\mathcal G_{\mathrm Tx}$および\index{とおりしん@通り芯}通り芯$T_x$を用いて\pageeqref{eq:TbasedTx}で与えられる。
\begin{align*}
  \HLbox{-\mathcal G_{Tx}+T_x}
\end{align*}

%%%%%%%%%%%%%%%%%%%%%%%%%%%%%%%%%%%%%%%%%%%%%%%%%%%%%%%%%%
%% subsection 30.3.3 %%%%%%%%%%%%%%%%%%%%%%%%%%%%%%%%%%%%%
%%%%%%%%%%%%%%%%%%%%%%%%%%%%%%%%%%%%%%%%%%%%%%%%%%%%%%%%%%
\subsection{外削長}

%%%%%%%%%%%%%%%%%%%%%%%%%%%%%%%%%%%%%%%%%%%%%%%%%%%%%%%%%%
%% subsubsection 30.3.3.1 %%%%%%%%%%%%%%%%%%%%%%%%%%%%%%%%
%%%%%%%%%%%%%%%%%%%%%%%%%%%%%%%%%%%%%%%%%%%%%%%%%%%%%%%%%%
\subsubsection{ボトムの外削}
ボトム側の外削における\index{こうぐ@工具}工具の先端の$Z$座標は、\index{ボトムがわのがいさくちょう@ボトム側の外削長}ボトム側の外削長を$h_\mathrm B$として、
\begin{align*}
  \HLbox{f_\mathrm B'-h_\mathrm B}\ .
\end{align*}

\clearpage
%%%%%%%%%%%%%%%%%%%%%%%%%%%%%%%%%%%%%%%%%%%%%%%%%%%%%%%%%%
%% subsubsection 30.3.3.1 %%%%%%%%%%%%%%%%%%%%%%%%%%%%%%%%
%%%%%%%%%%%%%%%%%%%%%%%%%%%%%%%%%%%%%%%%%%%%%%%%%%%%%%%%%%
\subsubsection{トップの外削}
トップ側の外削における工具の先端の$Z$座標は、\index{トップがわのがいさくちょう@トップ側の外削長}トップ側の外削長, \index{みぞいち@溝位置}溝位置, \index{みぞはば@溝幅}溝幅をそれぞれ$h_\mathrm T$, $\kappa_p$, $\kappa_w$として、
\begin{alignat*}{3}
  & \HLbox{f_\mathrm T'-h_\mathrm T} & \quad & \Big(\text{if}~h_\mathrm T > \kappa_p+\kappa_w\Big)\\
  & \HLbox{f_\mathrm T'-\left(\kappa_p+1[\mathrm{mm}]\right)} & \quad  & \Big(\text{if}~h_\mathrm T = \kappa_p+\kappa_w\Big)
\end{alignat*}


%\clearpage
%%%%%%%%%%%%%%%%%%%%%%%%%%%%%%%%%%%%%%%%%%%%%%%%%%%%%%%%%%
%% subsection 30.3.4 %%%%%%%%%%%%%%%%%%%%%%%%%%%%%%%%%%%%%
%%%%%%%%%%%%%%%%%%%%%%%%%%%%%%%%%%%%%%%%%%%%%%%%%%%%%%%%%%
\subsection{湾曲に沿った外削\TBW}
(to be written...)


\clearpage
%%%%%%%%%%%%%%%%%%%%%%%%%%%%%%%%%%%%%%%%%%%%%%%%%%%%%%%%%%
%% section 30.4 %%%%%%%%%%%%%%%%%%%%%%%%%%%%%%%%%%%%%%%%%%
%%%%%%%%%%%%%%%%%%%%%%%%%%%%%%%%%%%%%%%%%%%%%%%%%%%%%%%%%%
\modHeadsection{溝の数値情報}

%%%%%%%%%%%%%%%%%%%%%%%%%%%%%%%%%%%%%%%%%%%%%%%%%%%%%%%%%%
%% subsection 30.4.1 %%%%%%%%%%%%%%%%%%%%%%%%%%%%%%%%%%%%%
%%%%%%%%%%%%%%%%%%%%%%%%%%%%%%%%%%%%%%%%%%%%%%%%%%%%%%%%%%
\subsection{溝中心\texorpdfstring{$Z$}{Z}}
\index{みぞいち@溝位置}溝位置$\kappa_p$および\index{みぞはば@溝幅}溝幅$\kappa_w$に対し、\index{テーブルちゅうしん@テーブル中心}テーブル中心\index{P(テーブルちゅうしん)@P(テーブル中心)}Pを\index{げんてんP@原点P}原点とした\index{みぞちゅうしん@溝中心}溝の中心M$'$の$Z$座標は、\pageeqref{eq:mizocenterZ}より
\begin{align*}
  \HLbox{f_\mathrm T'-\kappa_p-\frac{\kappa_w}2}\ .
\end{align*}

%%%%%%%%%%%%%%%%%%%%%%%%%%%%%%%%%%%%%%%%%%%%%%%%%%%%%%%%%%
%% subsection 30.4.2 %%%%%%%%%%%%%%%%%%%%%%%%%%%%%%%%%%%%%
%%%%%%%%%%%%%%%%%%%%%%%%%%%%%%%%%%%%%%%%%%%%%%%%%%%%%%%%%%
\subsection{湾曲中心が基準の場合}
\index{トップたんのそとがわちゅうしん@トップ端の外側中心}トップ端の外側中心T$_\mathrm c'$と溝中心M$'$との$X$方向の差は、\pageeqref{eq:difTopMizoCenter}より、
\begin{align*}
  \HLbox{%
    \sqrt{R_\mathrm c^2-\left(f_\mathrm T-\kappa_p-\frac{\kappa_w}2\right)^{\!2}}
    -\frac{\sqrt{R_\mathrm o^2-f_\mathrm T^2}+\sqrt{R_\mathrm i^2-f_\mathrm T^2}}2%
  }\ .
\end{align*}


%%%%%%%%%%%%%%%%%%%%%%%%%%%%%%%%%%%%%%%%%%%%%%%%%%%%%%%%%%
%% subsection 30.4.3 %%%%%%%%%%%%%%%%%%%%%%%%%%%%%%%%%%%%%
%%%%%%%%%%%%%%%%%%%%%%%%%%%%%%%%%%%%%%%%%%%%%%%%%%%%%%%%%%
\subsection{外削中心が基準の場合}
\index{みぞちゅうしん@溝中心}溝の中心は\index{トップがわのがいさくちゅうしん@トップ側の外削中心}トップ側の外削中心とする。


%%%%%%%%%%%%%%%%%%%%%%%%%%%%%%%%%%%%%%%%%%%%%%%%%%%%%%%%%%
%% subsection 30.4.4 %%%%%%%%%%%%%%%%%%%%%%%%%%%%%%%%%%%%%
%%%%%%%%%%%%%%%%%%%%%%%%%%%%%%%%%%%%%%%%%%%%%%%%%%%%%%%%%%
\subsection{A側溝深さ指定の場合}

%%%%%%%%%%%%%%%%%%%%%%%%%%%%%%%%%%%%%%%%%%%%%%%%%%%%%%%%%%
%% subsubsection 30.4.4.1 %%%%%%%%%%%%%%%%%%%%%%%%%%%%%%%%
%%%%%%%%%%%%%%%%%%%%%%%%%%%%%%%%%%%%%%%%%%%%%%%%%%%%%%%%%%
\subsubsection{外削のない場合}
\index{Aがわみぞふかさ@A側溝深さ}A側溝深さ$\kappa_d$は、その測定値$\kappa_d'$が\index{ずめん(モールド)@図面(モールド)}図面上の値となるように与えられるものとする。このとき\pageeqref{eq:keydepthDif1}より、
\begin{align*}
  \HLbox{%
    \kappa_d
    = \frac{2\kappa_d'-\kappa_w\sin\zeta}{1+\cos^2\zeta}\cos\zeta
      +\sqrt{R_\mathrm o^2-\left(f_\mathrm T-\kappa_p-\frac{\kappa_w}2\right)^{\!2}}
      -\sqrt{R_\mathrm o^2-\left(f_\mathrm T-\kappa_p\right)^2}%
  }\ .
\end{align*}
ここで$\zeta$は\pageeqref{eq:angleZeta}より、
\begin{align*}
  \HLbox{%
    \tan\zeta
    = \frac{\sqrt{R_\mathrm o^2-\left(f_\mathrm T-\kappa_p-\kappa_w\right)^2}
            -\sqrt{R_\mathrm o^2-\left(f_\mathrm T-\kappa_p\right)^2}}
           {\kappa_w}%
  }\ .
\end{align*}
A側溝深さ$\kappa_d$に対し、\index{みぞちゅうしん@溝中心}溝中心の位置の$X$座標は\pageeqref{eq:mizocenterA}より、
\begin{align*}
  \HLbox{%
    \sqrt{R_\mathrm o^2-\left(f_\mathrm T-\kappa_p-\frac{\kappa_w}2\right)^{\!2}}
    -\kappa_d
    -\frac{W_{mx}}2
    -\varDelta_x%
  }\ .
\end{align*}
また\index{Aがわがいめん(みぞはばちゅうしん)@A側外面(溝幅中心)}A側外面の\index{じっそくち@実測値}実測値を$\mathcal G_m$とすると、溝中心と$G_m$との$X$座標の差は、\pageeqref{eq:mizocenterAd}より、
\begin{align*}
  \HLbox{-\frac{W_{mx}}2-\kappa_d}\ .
\end{align*}

\clearpage
%%%%%%%%%%%%%%%%%%%%%%%%%%%%%%%%%%%%%%%%%%%%%%%%%%%%%%%%%%
%% subsubsection 30.4.4.2 %%%%%%%%%%%%%%%%%%%%%%%%%%%%%%%%
%%%%%%%%%%%%%%%%%%%%%%%%%%%%%%%%%%%%%%%%%%%%%%%%%%%%%%%%%%
\subsubsection{外削のある場合}
\index{Aがわみぞふかさ@A側溝深さ}A側溝深さ$\kappa_d$に対し、トップ外削$X$中心を$\mathcal G_{\mathrm Tx}$とすると、$\mathcal G_{\mathrm Tx}$と\index{みぞちゅうしん@溝中心}溝中心との$X$座標の差は、\pageeqref{eq:mizocenterAG}より、
\begin{align*}
  \HLbox{\frac{\mathfrak W_x}2-\kappa_d-\frac{W_{mx}}2}\ .
\end{align*}



\clearpage
%%%%%%%%%%%%%%%%%%%%%%%%%%%%%%%%%%%%%%%%%%%%%%%%%%%%%%%%%%
%% section 9.2 %%%%%%%%%%%%%%%%%%%%%%%%%%%%%%%%%%%%%%%%%%%
%%%%%%%%%%%%%%%%%%%%%%%%%%%%%%%%%%%%%%%%%%%%%%%%%%%%%%%%%%
\modHeadsection{端面の外側C面取の数値情報\TBW}
(to be written...)


%\clearpage
%%%%%%%%%%%%%%%%%%%%%%%%%%%%%%%%%%%%%%%%%%%%%%%%%%%%%%%%%%
%% section 9.2 %%%%%%%%%%%%%%%%%%%%%%%%%%%%%%%%%%%%%%%%%%%
%%%%%%%%%%%%%%%%%%%%%%%%%%%%%%%%%%%%%%%%%%%%%%%%%%%%%%%%%%
\modHeadsection{端面の内側C面取の数値情報\TBW}
(to be written...)



%%%%%%%%%%%%%%%%%%%%%%%%%%%%%%%%%%%%%%%%%%%%%%%%%%%%%%%%%%
%% section 9.2 %%%%%%%%%%%%%%%%%%%%%%%%%%%%%%%%%%%%%%%%%%%
%%%%%%%%%%%%%%%%%%%%%%%%%%%%%%%%%%%%%%%%%%%%%%%%%%%%%%%%%%
\modHeadsection{端面の外側R面取の数値情報\TBW}
(to be written...)


%\clearpage
%%%%%%%%%%%%%%%%%%%%%%%%%%%%%%%%%%%%%%%%%%%%%%%%%%%%%%%%%%
%% section 9.2 %%%%%%%%%%%%%%%%%%%%%%%%%%%%%%%%%%%%%%%%%%%
%%%%%%%%%%%%%%%%%%%%%%%%%%%%%%%%%%%%%%%%%%%%%%%%%%%%%%%%%%
\modHeadsection{端面の内側R面取の数値情報\TBW}
(to be written...)


%\clearpage
%%%%%%%%%%%%%%%%%%%%%%%%%%%%%%%%%%%%%%%%%%%%%%%%%%%%%%%%%%
%% section 9.2 %%%%%%%%%%%%%%%%%%%%%%%%%%%%%%%%%%%%%%%%%%%
%%%%%%%%%%%%%%%%%%%%%%%%%%%%%%%%%%%%%%%%%%%%%%%%%%%%%%%%%%
\modHeadsection{端面の座ぐりの数値情報\TBW}
(to be written...)


%\clearpage
%%%%%%%%%%%%%%%%%%%%%%%%%%%%%%%%%%%%%%%%%%%%%%%%%%%%%%%%%%
%% section 32.2 %%%%%%%%%%%%%%%%%%%%%%%%%%%%%%%%%%%%%%%%%%%
%%%%%%%%%%%%%%%%%%%%%%%%%%%%%%%%%%%%%%%%%%%%%%%%%%%%%%%%%%
\modHeadsection{\dimple の数値情報\TBW}
(to be written...)





%%%%%%%%%%%%%%%%%%%%%%%%%%%%%%%%%%%%%%%%%%%%%%%%%%%%%%%%%%
%%            %%%%%%%%%%%%%%%%%%%%%%%%%%%%%%%%%%%%%%%%%%%%
%% chapter 33 %%%%%%%%%%%%%%%%%%%%%%%%%%%%%%%%%%%%%%%%%%%%
%%            %%%%%%%%%%%%%%%%%%%%%%%%%%%%%%%%%%%%%%%%%%%%
%%%%%%%%%%%%%%%%%%%%%%%%%%%%%%%%%%%%%%%%%%%%%%%%%%%%%%%%%%
\modHeadchapter{代入する数値\TBW}
寸法・形状等の数値情報や条件分岐情報は、明細ごとに固有である。
そのため、その固有情報は入力する必要がある。
ここではそうした入力する必要のある情報をまとめておく。




\begin{appendices}
%%%%%%%%%%%%%%%%%%%%%%%%%%%%%%%%%%%%%%%%%%%%%%%%%%%%%%%%%
%%                %%%%%%%%%%%%%%%%%%%%%%%%%%%%%%%%%%%%%%%
%% Appendix       %%%%%%%%%%%%%%%%%%%%%%%%%%%%%%%%%%%%%%%
%% Numerical Calc %%%%%%%%%%%%%%%%%%%%%%%%%%%%%%%%%%%%%%%
%% Start          %%%%%%%%%%%%%%%%%%%%%%%%%%%%%%%%%%%%%%%
%%                %%%%%%%%%%%%%%%%%%%%%%%%%%%%%%%%%%%%%%%
%%%%%%%%%%%%%%%%%%%%%%%%%%%%%%%%%%%%%%%%%%%%%%%%%%%%%%%%%
%\Appendixpart


\end{appendices}
%%%%%%%%%%%%%%%%%%%%%%%%%%%%%%%%%%%%%%%%%%%%%%%%%%%%%%%%%
%%                %%%%%%%%%%%%%%%%%%%%%%%%%%%%%%%%%%%%%%%
%% Appendix       %%%%%%%%%%%%%%%%%%%%%%%%%%%%%%%%%%%%%%%
%% Numerical Calc %%%%%%%%%%%%%%%%%%%%%%%%%%%%%%%%%%%%%%%
%% End            %%%%%%%%%%%%%%%%%%%%%%%%%%%%%%%%%%%%%%%
%%                %%%%%%%%%%%%%%%%%%%%%%%%%%%%%%%%%%%%%%%
%%%%%%%%%%%%%%%%%%%%%%%%%%%%%%%%%%%%%%%%%%%%%%%%%%%%%%%%%

\addtocontents{toc}{\protect\end{tcolorbox}}
\addtocontents{toc}{\protect\cleardoublepage}% add page break




\PartSeparateline{lot}%
%%%%%%%%%%%%%%%%%%%%%%%%%%%%%%%%%%%%%%%%%%%%%%%%%%%%%%%%%
%%               %%%%%%%%%%%%%%%%%%%%%%%%%%%%%%%%%%%%%%%%
%%               %%%%%%%%%%%%%%%%%%%%%%%%%%%%%%%%%%%%%%%%
%% Part MANUAL   %%%%%%%%%%%%%%%%%%%%%%%%%%%%%%%%%%%%%%%%
%%               %%%%%%%%%%%%%%%%%%%%%%%%%%%%%%%%%%%%%%%%
%%               %%%%%%%%%%%%%%%%%%%%%%%%%%%%%%%%%%%%%%%%
%%%%%%%%%%%%%%%%%%%%%%%%%%%%%%%%%%%%%%%%%%%%%%%%%%%%%%%%%
\addtocontents{toc}{\protect\begin{tcolorbox}[parttocstyle={Contents}{\the\numexpr\value{part}+1\relax}]}
\tPart[lot,lol]{各工程用自動化プログラム 取扱説明\TBW}{概要}{%
\paragraph*{目標(なにがしたいか?)}
(to be written...)
\tcbline*
\paragraph*{手段(どうやって?)}
(to be written...)
\tcbline*
\paragraph*{背景(なぜ?)}
(to be written...)
\tcbline*
\paragraph*{結論(どうなった?)}
(to be written...)
}
%!TEX root = ./RPA_for_Creating_Program_Note.tex


\addtocontents{toc}{\protect\cleardoublepage}
\addtocontents{lot}{\protect\tcbline*}
%%%%%%%%%%%%%%%%%%%%%%%%%%%%%%%%%%%%%%%%%%%%%%%%%%%%%%%%%
%% Part MANUAL %%%%%%%%%%%%%%%%%%%%%%%%%%%%%%%%%%%%%%%%%%
%%%%%%%%%%%%%%%%%%%%%%%%%%%%%%%%%%%%%%%%%%%%%%%%%%%%%%%%%
\addtocontents{toc}{\protect\begin{tocBox}{\tmppartnum}}%
\tPart[lot,lol]{各工程用自動化プログラム 取扱説明\TBW}{概要}{%
\paragraph*{目標(なにがしたいか?)}
(to be written...)
\tcbline*
\paragraph*{手段(どうやって?)}
(to be written...)
\tcbline*
\paragraph*{背景(なぜ?)}
(to be written...)
\tcbline*
\paragraph*{結論(どうなった?)}
(to be written...)
}

%%%%%%%%%%%%%%%%%%%%%%%%%%%%%%%%%%%%%%%%%%%%%%%%%%%%%%%%%%
%% chapters %%%%%%%%%%%%%%%%%%%%%%%%%%%%%%%%%%%%%%%%%%%%%%
%%%%%%%%%%%%%%%%%%%%%%%%%%%%%%%%%%%%%%%%%%%%%%%%%%%%%%%%%%
%!TEX root = ./RPA_for_Creating_Program_Note.tex


\modHeadchapter[lot]{\TBW}



%%%%%%%%%%%%%%%%%%%%%%%%%%%%%%%%%%%%%%%%%%%%%%%%%%%%%%%%%%
%% section 2.1 %%%%%%%%%%%%%%%%%%%%%%%%%%%%%%%%%%%%%%%%%%%
%%%%%%%%%%%%%%%%%%%%%%%%%%%%%%%%%%%%%%%%%%%%%%%%%%%%%%%%%%
\modHeadsection{\TBW}
(to be written...)


%%%%%%%%%%%%%%%%%%%%%%%%%%%%%%%%%%%%%%%%%%%%%%%%%%%%%%%%%
%% Appendicies %%%%%%%%%%%%%%%%%%%%%%%%%%%%%%%%%%%%%%%%%%
%%%%%%%%%%%%%%%%%%%%%%%%%%%%%%%%%%%%%%%%%%%%%%%%%%%%%%%%%
\begin{appendices}
\Appendixpart
%!TEX root = ../RPA_for_Creating_Program_Note.tex


ここでは\DMname の主な\index{Gコード}Gコードおよび\index{Mコード}Mコードを記載する。
なお、使用の欄は、●は作成したプログラムで使用しているもの、◯はバンドルのプログラムのみで使用されているものを示す。



%%%%%%%%%%%%%%%%%%%%%%%%%%%%%%%%%%%%%%%%%%%%%%%%%%%%%%%%%%
%% section H.1 %%%%%%%%%%%%%%%%%%%%%%%%%%%%%%%%%%%%%%%%%%%
%%%%%%%%%%%%%%%%%%%%%%%%%%%%%%%%%%%%%%%%%%%%%%%%%%%%%%%%%%
\modHeadsection{主なGコード:\DMname}
詳細については、別冊の「Mycenter プログラミング説明書」の「基本Gコード一覧表」[P.121]を参照されたし
%% footnote %%%%%%%%%%%%%%%%%%%%%
\footnote{\index{こていサイクルGコード@固定サイクルGコード}固定サイクルGコードについては、「Mycenter-HX630G/800 取扱説明書」の「機能説明書」の「キタムラ固定サイクルGコード」[p.2]を参照。}。\\
%%%%%%%%%%%%%%%%%%%%%%%%%%%%%%%%%


%%%%%%%%%%%%%%%%%%%%%%%%%%%%%%%%%%%%%%%%%%%%%%%%%%%%%%%%%%
%% captionof %%%%%%%%%%%%%%%%%%%%%%%%%%%%%%%%%%%%%%%%%%%%%
%%%%%%%%%%%%%%%%%%%%%%%%%%%%%%%%%%%%%%%%%%%%%%%%%%%%%%%%%%
\begin{multicollongtblr}{主なGコード:\DMname}{ccX[l]c}
{\ttfamily G}番号 & グループ & 内容 & 使用\\
{\ttfamily G00} & 01 & 位置決め(早送り) & ●\\
{\ttfamily G01} & 01 & 直線補間 & ●\\
{\ttfamily G02} & 01 & 円弧補間・ヘリカル補間(右回り) & ●\\
{\ttfamily G03} & 01 & 円弧補間・ヘリカル補間(左回り) & ●\\
{\ttfamily G04} & 00 & ドウェル(待機時間) & ●\\
{\ttfamily G10} & 00 & プログラマブルデータ入力(パラメータ入力) & ●\\
{\ttfamily G15} & 18 & 極座標指令OFF & \\
{\ttfamily G17} & 02 & $XY$平面 選択 & ●\\
{\ttfamily G18} & 02 & $XZ$平面 選択 & \\
{\ttfamily G19} & 02 & $YZ$平面 選択 & \\
{\ttfamily G21} & 06 & メトリック指令 & \\
{\ttfamily G22} & 04 & 移動前ストロークチェック ON & ◯\\
{\ttfamily G23} & 04 & 移動前ストロークチェック OFF & \\
{\ttfamily G28} & 00 & 第1レファレンス点(機械原点)への復帰 & ●\\
{\ttfamily G29} & 00 & レファレンス点からの復帰 & \\
{\ttfamily G30} & 00 & 第2レファレンス点(工具交換位置)復帰 & ●\\
{\ttfamily G31} & 00 & スキップ & ●\\
{\ttfamily G40} & 07 & 工具径補正なし & ●\\
{\ttfamily G40.1} & 15 & 法線制御OFF & \\
{\ttfamily G41} & 07 & 工具径補正 進行方向の左 & ●\\
{\ttfamily G42} & 07 & 工具径補正 進行方向の右 & ●\\
{\ttfamily G43} & 08 & 工具長補正$+$ & ●\\
{\ttfamily G44} & 08 & 工具長補正$-$ & \\
{\ttfamily G49} & 08 & 工具長補正なし & ●\\
{\ttfamily G50} & 11 & スケーリングOFF & \\
{\ttfamily G50.1} & 19 & {\ttfamily G}指令ミラーイメージOFF & \\
{\ttfamily G53} & 00 & 機械座標系 選択 & ●\\
{\ttfamily G54} & 12 & ワーク座標系1 選択 & ●\\
{\ttfamily G54.1} & 12 & 拡張ワーク座標系 選択 & ◯\\
{\ttfamily G55} & 12 & ワーク座標系2 選択 & ●\\
{\ttfamily G56} & 12 & ワーク座標系3 選択 & ●\\
{\ttfamily G57} & 12 & ワーク座標系4 選択 & ●\\
{\ttfamily G58} & 12 & ワーク座標系5 選択 & \\
{\ttfamily G59} & 12 & ワーク座標系6 選択 & \\
{\ttfamily G61.1} & 13 & 高精度制御ON & \\
{\ttfamily G65} & 00 & プログラム(ユーザマクロ)の単純呼出し(引数指定可) & ●\\
{\ttfamily G67} & 14 & ユーザマクロ モーダル呼出し なし & \\
{\ttfamily G69} & 16 &  & \\
{\ttfamily G80} & 09 & 固定サイクルなし & ●\\
{\ttfamily G90} & 03 & 絶対値指令 & ●\\
{\ttfamily G91} & 03 & 増分値(相対値)指令 & ●\\
{\ttfamily G94} & 05 & 毎分送り(非同期送り) & \\
{\ttfamily G97} & 17 & 周速一定制御OFF & \\
{\ttfamily G98} & 10 & 固定サイクルイニシャルレベル復帰 &
\end{multicollongtblr}



\clearpage
%%%%%%%%%%%%%%%%%%%%%%%%%%%%%%%%%%%%%%%%%%%%%%%%%%%%%%%%%%
%% section H.2 %%%%%%%%%%%%%%%%%%%%%%%%%%%%%%%%%%%%%%%%%%%
%%%%%%%%%%%%%%%%%%%%%%%%%%%%%%%%%%%%%%%%%%%%%%%%%%%%%%%%%%
\modHeadsection{主なMコード:\DMname}
詳細については、別冊の「Mycenter-HX630G/800 取扱説明書」の「10.2 補助機能(M)(S)(T)」[p.287]または「Mycenter プログラミング説明書」の「8. Mコード」[P.84]を参照されたし。\\

%%%%%%%%%%%%%%%%%%%%%%%%%%%%%%%%%%%%%%%%%%%%%%%%%%%%%%%%%%
%% captionof %%%%%%%%%%%%%%%%%%%%%%%%%%%%%%%%%%%%%%%%%%%%%
%%%%%%%%%%%%%%%%%%%%%%%%%%%%%%%%%%%%%%%%%%%%%%%%%%%%%%%%%%
\begin{multicollongtblr}{主なMコード:\DMname}{cX[l]c}
{\ttfamily M}番号 & 内容 & 使用\\
{\ttfamily M00} & プログラムの一時停止 & ●\\
{\ttfamily M01} & プログラムの一時停止(オプショナル) &\\
{\ttfamily M02} & プログラムの終了 & ●\\
{\ttfamily M03} & 主軸回転起動(スピンドル正転) & ●\\
{\ttfamily M04} & 主軸回転起動(スピンドル逆転) & ◯\\
{\ttfamily M05} & 主軸回転停止 & ●\\
{\ttfamily M06} & 工具交換 & ●\\
{\ttfamily M07} & 切削油(クーラント)No.2(センタースルー)ON & ●\\
{\ttfamily M08} & 切削油(クーラント)No.1(ノズル)ON & ●\\
{\ttfamily M09} & 切削油(クーラント)およびエアブローOFF & ●\\
{\ttfamily M10} & 4軸(B軸)クランプ & ●\\
{\ttfamily M11} & 4軸(B軸)アンクランプ & ●\\
{\ttfamily M19} & 主軸定位置停止(オリエンテーション) & ●\\
{\ttfamily M28} & チップコンベアON & ●\\
{\ttfamily M29} & チップコンベアOFF & ●\\
{\ttfamily M30} & プログラムのリセットおよびリワインド & ●\\
{\ttfamily M32} & 工具長測定装置ON(カバー開閉・前進・上下動作含む) & ◯\\
{\ttfamily M33} & 工具長測定装置OFF(カバー開閉・前進・上下動作含む) & ◯\\
{\ttfamily M43} & 完了確認無し主軸回転起動(スピンドル正転) & \\
{\ttfamily M44} & 完了確認無し主軸回転起動(スピンドル逆転) & \\
{\ttfamily M48} & オーバーライド無視キャンセル & ◯\\
{\ttfamily M49} & オーバーライド無視 & ◯\\
{\ttfamily M60} & パレットの入換え & ●\\
{\ttfamily M61} & パレット旋回No.1側 & ◯\\
{\ttfamily M62} & パレット旋回No.2側 & ◯\\
{\ttfamily M70} & 工具スキップ & \\
{\ttfamily M71} & パレットNo.1搬入 & ●\\
{\ttfamily M72} & パレットNo.2搬入 & ●\\
{\ttfamily M73} & パレット上昇 & ◯\\
{\ttfamily M74} & パレット加工 & ◯\\
{\ttfamily M78} & パレットクランプ & ◯\\
{\ttfamily M79} & パレットアンクランプ & ◯\\
{\ttfamily M98} & プログラムの呼出し(引数指定不可) & ●\\
{\ttfamily M99} & プログラムの復帰 & ●\\
{\ttfamily M117} & タッチプローブセンサー電源スイッチ & ●\\
{\ttfamily M151} & スピンドルスルーエアーブロー(常時ON) & ●\\
{\ttfamily M219} & 完了確認なし主軸定位置停止(オリエンテーション) & \\
{\ttfamily M260} & パレット旋回速さ100\% & \\
{\ttfamily M261} & パレット旋回速さ50\% & \\
{\ttfamily M262} & パレット旋回速さ25\% & ◯\\
{\ttfamily M263} & パレット旋回速さ3\% & \\
\end{multicollongtblr}

%!TEX root = ../RPA_for_Creating_Program_Note.tex


ここでは\MMname の主な\index{Gコード}Gコードおよび\index{Mコード}Mコードを記載する。



%%%%%%%%%%%%%%%%%%%%%%%%%%%%%%%%%%%%%%%%%%%%%%%%%%%%%%%%%%
%% section H.1 %%%%%%%%%%%%%%%%%%%%%%%%%%%%%%%%%%%%%%%%%%%
%%%%%%%%%%%%%%%%%%%%%%%%%%%%%%%%%%%%%%%%%%%%%%%%%%%%%%%%%%
\modHeadsection{主なGコード:\MMname}
詳細については、別冊の「FANUC Series 16/18/160/180-MB」の「3 準備機能(G機能)」[P.30]を参照されたし。\\


%%%%%%%%%%%%%%%%%%%%%%%%%%%%%%%%%%%%%%%%%%%%%%%%%%%%%%%%%%
%% captionof %%%%%%%%%%%%%%%%%%%%%%%%%%%%%%%%%%%%%%%%%%%%%
%%%%%%%%%%%%%%%%%%%%%%%%%%%%%%%%%%%%%%%%%%%%%%%%%%%%%%%%%%
\begin{multicollongtblr}{主なGコード:\MMname}{ccX[l]c}
{\ttfamily G}番号 & グループ & 内容 & 使用\\
{\ttfamily G00} & 01 & 位置決め(早送り) & \\
{\ttfamily G01} & 01 & 直線補間 & \\
{\ttfamily G02} & 01 & 円弧補間・ヘリカル補間(右回り) & \\
{\ttfamily G02.2} & 01 & インボリュート補間 & \\
{\ttfamily G02.3} & 01 & 指数関数補間 & \\
{\ttfamily G03} & 01 & 円弧補間・ヘリカル補間(左回り) & \\
{\ttfamily G03.2} & 01 & インボリュート補間 & \\
{\ttfamily G03.3} & 01 & 指数関数補間 & \\
{\ttfamily G04} & 00 & ドウェル(待機時間), イグザクトストップ & \\
{\ttfamily G05} & 00 & 高速サイクル加工 & \\
{\ttfamily G07.1} & 00 & 円筒補間 & \\
{\ttfamily G08} & 00 & 先行制御 & \\
{\ttfamily G09} & 00 & イグザクトストップ & \\
{\ttfamily G10} & 00 & プログラマブルデータ入力(パラメータ入力) & \\
{\ttfamily G10.6} & 00 & 工具退避および復帰 & \\
{\ttfamily G11} & 00 & プログラマブルデータ入力モードキャンセル & \\
{\ttfamily G12.1} & 25 & 極座標補間モード & \\
{\ttfamily G13.1} & 25 & 極座標補間キャンセルモード & \\
{\ttfamily G15} & 17 & 極座標指令OFF & \\
{\ttfamily G16} & 17 & 極座標指令ON & \\
{\ttfamily G17} & 02 & $XY$平面 選択 & \\
{\ttfamily G18} & 02 & $XZ$平面 選択 & \\
{\ttfamily G19} & 02 & $YZ$平面 選択 & \\
{\ttfamily G20} & 06 & インチ指令 & \\
{\ttfamily G21} & 06 & メトリック指令 & \\
{\ttfamily G22} & 04 & ストアードストロークチェックON & \\
{\ttfamily G23} & 04 & ストアードストロークチェックOFF & \\
{\ttfamily G25} & 24 & 主軸速さ変動検出OFF & \\
{\ttfamily G26} & 24 & 主軸速さ変動検出ON & \\
{\ttfamily G27} & 00 & レファレンス点復帰確認 & \\
{\ttfamily G28} & 00 & レファレンス点(機械原点)への自動復帰 & \\
{\ttfamily G29} & 00 & レファレンス点からの自動復帰 & \\
{\ttfamily G30} & 00 & 第2レファレンス点(工具交換位置)への復帰 & \\
{\ttfamily G30.1} & 00 & フローティングレファレンス点への復帰 & \\
{\ttfamily G31} & 00 & スキップ & \\
{\ttfamily G33} & 01 & ねじ切り & \\
{\ttfamily G37} & 00 & 工具長自動測定 & \\
{\ttfamily G39} & 00 & コーナーオフセット円弧補間 & \\
{\ttfamily G40} & 07 & 工具径補正なし & \\
{\ttfamily G40.1} & 19 & 法線方向制御OFF & \\
{\ttfamily G41} & 07 & 工具径補正 進行方向の左 & \\
{\ttfamily G41.1} & 19 & 法線方向制御 左側ON & \\
{\ttfamily G42} & 07 & 工具径補正 進行方向の右 & \\
{\ttfamily G42.1} & 19 & 法線方向制御 右側ON & \\
{\ttfamily G43} & 08 & 工具長補正$+$ & \\
{\ttfamily G44} & 08 & 工具長補正$-$ & \\
{\ttfamily G45} & 00 & 工具位置オフセット 伸長 & \\
{\ttfamily G46} & 00 & 工具位置オフセット 縮小 & \\
{\ttfamily G47} & 00 & 工具位置オフセット 2倍伸長 & \\
{\ttfamily G48} & 00 & 工具位置オフセット 2倍縮小 & \\
{\ttfamily G49} & 08 & 工具長補正なし & \\
{\ttfamily G50} & 11 & スケーリングOFF & \\
{\ttfamily G50.1} & 22 & プログラマブルミラーイメージOFF & \\
{\ttfamily G51} & 11 & スケーリングON & \\
{\ttfamily G51.1} & 22 & プログラマブルミラーイメージON & \\
{\ttfamily G52} & 00 & ローカル座標系 選択 & \\
{\ttfamily G53} & 00 & 機械座標系 選択 & \\
{\ttfamily G54} & 14 & ワーク座標系1 選択 & \\
{\ttfamily G54.1} & 14 & 拡張ワーク座標系 選択 & \\
{\ttfamily G55} & 14 & ワーク座標系2 選択 & \\
{\ttfamily G56} & 14 & ワーク座標系3 選択 & \\
{\ttfamily G57} & 14 & ワーク座標系4 選択 & \\
{\ttfamily G58} & 14 & ワーク座標系5 選択 & \\
{\ttfamily G59} & 14 & ワーク座標系6 選択 & \\
{\ttfamily G60} & 00 & 一方向位置決め & \\
{\ttfamily G61} & 15 & イグザクトストップモード & \\
{\ttfamily G62} & 15 & 自動コーナーオーバーライド & \\
{\ttfamily G63} & 15 & タッピングモード & \\
{\ttfamily G64} & 15 & 切削モード & \\
{\ttfamily G65} &  & マクロ呼出し & \\
{\ttfamily G66} & 12 & マクロモーダル呼出し & \\
{\ttfamily G67} & 12 & マクロモーダル呼出し なし & \\
{\ttfamily G68} & 16 & 座標回転ON & \\
{\ttfamily G69} & 16 & 座標回転OFF & \\
{\ttfamily G72.1} & 00 & 回転コピー & \\
{\ttfamily G72.2} & 00 & 平行コピー & \\
{\ttfamily G73} & 09 & ペックドリリングサイクル & \\
{\ttfamily G74} & 09 & 逆タッピングサイクル & \\
{\ttfamily G75} & 01 & プランジ研削サイクル(研削盤用) & \\
{\ttfamily G76} & 09 & ファインボーリングサイクル & \\
{\ttfamily G77} & 01 & プランジ直接定寸研削サイクル(研削盤用) & \\
{\ttfamily G78} & 01 & 連続送り平研削サイクル(研削盤用) & \\
{\ttfamily G79} & 01 & 間欠送り平研削サイクル(研削盤用) & \\
{\ttfamily G80} & 09 & 固定サイクルなし & \\
{\ttfamily G81} & 09 & ドリルサイクル & \\
{\ttfamily G82} & 09 & ドリルサイクル & \\
{\ttfamily G83} & 09 & ペックドリリングサイクル & \\
{\ttfamily G84} & 09 & タッピングサイクル & \\
{\ttfamily G85} & 09 & ボーリングサイクル & \\
{\ttfamily G86} & 09 & ボーリングサイクル & \\
{\ttfamily G87} & 09 & バックボーリングサイクル & \\
{\ttfamily G88} & 09 & ボーリングサイクル & \\
{\ttfamily G89} & 09 & ボーリングサイクル & \\
{\ttfamily G90} & 03 & 絶対値指令 & \\
{\ttfamily G91} & 03 & 増分値(相対値)指令 & \\
{\ttfamily G92} & 00 & ワーク座標系の設定 & \\
{\ttfamily G92.1} & 00 & ワーク座標系プリセット & \\
{\ttfamily G94} & 05 & 毎分送り & \\
{\ttfamily G95} & 05 & 毎回転送り & \\
{\ttfamily G96} & 13 & 周速一定制御ON & \\
{\ttfamily G97} & 13 & 周速一定制御OFF & \\
{\ttfamily G98} & 10 & 固定サイクルイニシャルレベル復帰 & \\
{\ttfamily G99} & 10 & 固定サイクルR点レベル復帰 & \\
{\ttfamily G160} & 20 & インフィード制御機能OFF(研削盤用) & \\
{\ttfamily G161} & 20 & インフィード制御機能ON(研削盤用) & \\
\end{multicollongtblr}



\clearpage
%%%%%%%%%%%%%%%%%%%%%%%%%%%%%%%%%%%%%%%%%%%%%%%%%%%%%%%%%%
%% section H.2 %%%%%%%%%%%%%%%%%%%%%%%%%%%%%%%%%%%%%%%%%%%
%%%%%%%%%%%%%%%%%%%%%%%%%%%%%%%%%%%%%%%%%%%%%%%%%%%%%%%%%%
\modHeadsection{主なMコード:\MMname}

%%%%%%%%%%%%%%%%%%%%%%%%%%%%%%%%%%%%%%%%%%%%%%%%%%%%%%%%%%
%% captionof %%%%%%%%%%%%%%%%%%%%%%%%%%%%%%%%%%%%%%%%%%%%%
%%%%%%%%%%%%%%%%%%%%%%%%%%%%%%%%%%%%%%%%%%%%%%%%%%%%%%%%%%
\begin{multicollongtblr}{主なMコード:\MMname}{cX[l]c}
{\ttfamily M}番号 & 内容 & 使用\\
{\ttfamily M00} & プログラムの一時停止 & \\
{\ttfamily M01} & プログラムの一時停止(オプショナル) &\\
{\ttfamily M02} & プログラムの終了 & \\
{\ttfamily M98} & プログラムの呼出し(引数指定不可) & \\
{\ttfamily M99} & プログラムの復帰 & \\
{\ttfamily M392} & 扉 開 & \\
{\ttfamily M393} & 扉 閉 & \\
\end{multicollongtblr}

%!TEX root = ../RfCPN.tex
\setcounter{lstlisting}{0}


\modHeadchapter[lot,lol]{作成したNCプログラム\label{chap:createdNCprgDM}}



%%%%%%%%%%%%%%%%%%%%%%%%%%%%%%%%%%%%%%%%%%%%%%%%%%%%%%%%%%
%% section K.1 %%%%%%%%%%%%%%%%%%%%%%%%%%%%%%%%%%%%%%%%%%%
%%%%%%%%%%%%%%%%%%%%%%%%%%%%%%%%%%%%%%%%%%%%%%%%%%%%%%%%%%
\modHeadsection{作成したNCプログラム 一覧}


%%%%%%%%%%%%%%%%%%%%%%%%%%%%%%%%%%%%%%%%%%%%%%%%%%%%%%%%%%
%% subsection K.1.1 %%%%%%%%%%%%%%%%%%%%%%%%%%%%%%%%%%%%%%
%%%%%%%%%%%%%%%%%%%%%%%%%%%%%%%%%%%%%%%%%%%%%%%%%%%%%%%%%%
\subsection{メインプログラムの例 一覧}
\DMC において作成した\index{メインプログラム}メインプログラムは以下のとおりである。
%%%%%%%%%%%%%%%%%%%%%%%%%%%%%%%%%%%%%%%%%%%%%%%%%%%%%%%%%%
%% marker %%%%%%%%%%%%%%%%%%%%%%%%%%%%%%%%%%%%%%%%%%%%%%%%
%%%%%%%%%%%%%%%%%%%%%%%%%%%%%%%%%%%%%%%%%%%%%%%%%%%%%%%%%%
\begin{marker}
ここでいうメインプログラムとは、\DrawingNumber と同一の\index{プログラムばんごう@プログラム番号}プログラム番号のものを指す。
ただし、ここでは例として挙げているので、\pageautoref{subsec:notopenwork}に伴い、プログラム番号は本来のものから変更している。
\end{marker}
%%%%%%%%%%%%%%%%%%%%%%%%%%%%%%%%%%%%%%%%%%%%%%%%%%%%%%%%%%
%%%%%%%%%%%%%%%%%%%%%%%%%%%%%%%%%%%%%%%%%%%%%%%%%%%%%%%%%%
%%%%%%%%%%%%%%%%%%%%%%%%%%%%%%%%%%%%%%%%%%%%%%%%%%%%%%%%%%

\begin{multicollongtblr}{作成したNCプログラム一覧:メインプログラムの例}{ccX[l]}
{\ttfamily O}番号 & \SetCell{c}工程 & 使用prg\\
\SetCell[r=10]{c}
\MainExOne & 外側芯出し・幅 & \MYOThickness\MXIface\\
           & 内側芯出し・幅 & \MXIWidth\MYIWidth\\
           & \CenterlineEndFaceDif & \Mcenterline\\
           & \Dimple & \DLone\\
           & \EndFacecut & \KTanmenRight\\
           & \Outcut & \KGaisakuRLeft\\
           & \Keyway & \KMizoConerLeft\\
           & \EndFaceOutChamfer & \KSotoMentoriRLeft\\
           & \EndFaceInChamfer & \KUchiMentoriRLeft\\
           & その他 & \OpauseCheck\OsensorOn\OsensorOff\\
\hline
\SetCell[r=8]{c}
\MainExTwo & 外側芯出し・幅 & \MXOThickness\MYOThickness\MXOface\\
           & 内側芯出し・幅 & \MXIWidth\MYIWidth\\
           & \Dimple & \DLone\\
           & \EndFacecut & \KTanmenRight\\
           & \Keyway & \KMizoConerLeft\\
           & \EndFaceOutChamfer & \KSotoMentoriRLeft\\
           & \EndFaceInChamfer & \KUchiMentoriRLeft\\
           & その他 & \OpauseCheck\OsensorOn\OsensorOff\\
\end{multicollongtblr}


\clearpage
%%%%%%%%%%%%%%%%%%%%%%%%%%%%%%%%%%%%%%%%%%%%%%%%%%%%%%%%%%
%% subsection K.1.2 %%%%%%%%%%%%%%%%%%%%%%%%%%%%%%%%%%%%%%
%%%%%%%%%%%%%%%%%%%%%%%%%%%%%%%%%%%%%%%%%%%%%%%%%%%%%%%%%%
\subsection{サブプログラム 一覧}
\DMC において作成した\index{サブプログラム}サブプログラムは以下のとおりである
%%%%%%%%%%%%%%%%%%%%%%%%%%%%%%%%%%%%%%%%%%%%%%%%%%%%%%%%%%
%% marker %%%%%%%%%%%%%%%%%%%%%%%%%%%%%%%%%%%%%%%%%%%%%%%%
%%%%%%%%%%%%%%%%%%%%%%%%%%%%%%%%%%%%%%%%%%%%%%%%%%%%%%%%%%
\begin{marker}
ここでいうサブプログラムとは、\DrawingNumber と同一の\index{プログラムばんごう@プログラム番号}プログラム番号以外のものを指す。
\end{marker}
%%%%%%%%%%%%%%%%%%%%%%%%%%%%%%%%%%%%%%%%%%%%%%%%%%%%%%%%%%
%%%%%%%%%%%%%%%%%%%%%%%%%%%%%%%%%%%%%%%%%%%%%%%%%%%%%%%%%%
%%%%%%%%%%%%%%%%%%%%%%%%%%%%%%%%%%%%%%%%%%%%%%%%%%%%%%%%%%

\begin{multicollongtblr}{作成したNCプログラム一覧:芯出し・幅・\CenterlineEndFaceDif 測定}{cX[l]l}
{\ttfamily O}番号 & 内容 & 使用prg\\
\MXOThickness & 測定 両側 外側中心・幅$X$ & \OsensorOff\\
\MYOThickness & 測定 両側 外側中心・幅$Y$ & \OsensorOff\\
\MXOface      & 測定 片側 \KeywayCenter$X$(トップA側外面測定) & \OsensorOff\\
\MXIWidth     & 測定 両側 内側中心・幅$X$ & \OsensorOff\\
\MYIWidth     & 測定 両側 内側中心・幅$Y$ & \OsensorOff\\
\MXIface      & 測定 片側 \OutcutCenter$X$(C側内面方向測定) & \OsensorOff\\
\Mcenterline & 測定 片側 \CenterlineEndFaceDif(C側外削面 $Z$方向, B側外削面 $Y$方向測定) & \OsensorOff\\
\end{multicollongtblr}

\begin{multicollongtblr}{作成したNCプログラム一覧:\Dimple}{cX[l]l}
{\ttfamily O}番号 & 内容 & 使用prg\\
\DLone      & \Dimple :移動 各列の中心上 & \DLtwoAC\DLtwoBD\\
\DLtwoAC    & \Dimple :移動 AC面 列内の各\Dimple 上 & \DMLthreeAC\DKLthreeAC\\
\DLtwoBD    & \Dimple :移動 BC面 列内の各\Dimple 上 & \DMLthreeBD\DKLthreeBD\\
\DMLthreeAC & \Dimple :測定 AC内表面$X$ & \OsensorOff\\
\DMLthreeBD & \Dimple :測定 BD内表面$Y$ & \OsensorOff\\
\DKLthreeAC & \Dimple :加工 AC内表面$X$ & -\\
\DKLthreeBD & \Dimple :加工 BD内表面$Y$ & -\\
\end{multicollongtblr}

%\clearpage
\begin{multicollongtblr}{作成したNCプログラム一覧:加工(\Dimple 以外)}{cX[l]l}
{\ttfamily O}番号 & 内容 & 使用prg\\
\KTanmenRight      & 加工 \EndFacecut{} コーナーR 右回り1周 & \KOLeftAR\\
\KGaisakuRLeft     & 加工 \Outcut{} コーナーR 左回り1周 & \KOLeftAR\OpauseCheck\\
\KMizoConerLeft    & 加工 \Keyway{} 左回り1周 & \KOLeftAR\OpauseCheck\\
\KSotoMentoriRLeft & 加工 \EndFaceOutChamfer{} コーナーR 左回り1周 & \KOLeftAR\OpauseCheck\\
\KUchiMentoriRLeft & 加工 \EndFaceInChamfer{} コーナーR 左回り1周 & \KILeftAC\OpauseCheck\\
\KOLeftAR   & 外側 左回り1周 右上始まり & -\\
\KILeftAC   & 内側 左回り1周 中央上始まり & -\\
\end{multicollongtblr}

\clearpage
\begin{multicollongtblr}{作成したNCプログラム一覧:その他}{cX[l]l}
{\ttfamily O}番号 & 内容 & 使用prg\\
\OpauseCheck  & 移動・加工後確認用:90$^\circ$回転 扉前一時停止 & -\\
\OsensorOn    & \index{タッチセンサーでんげん@タッチセンサー電源}タッチセンサー電源ON & -\\
\OsensorOff   & \index{タッチセンサーでんげん@タッチセンサー電源}タッチセンサー電源OFF & -\\
\OwarmingupA  & \index{だんきうんてん@暖機運転}暖機運転 & \Owarmingup\\
\Owarmingup   & 暖機運転用サブプログラム & \\
\OtoolLengthA & \index{とうろくこうぐ@登録工具}登録工具 工具長自動測定 & \OtoolLength\\
\OtoolLength  & \index{こうぐちょう@工具長}工具長 自動測定用サブプログラム & -\\
\end{multicollongtblr}



\clearrightpage
%%%%%%%%%%%%%%%%%%%%%%%%%%%%%%%%%%%%%%%%%%%%%%%%%%%%%%%%%%
%% section G.1 %%%%%%%%%%%%%%%%%%%%%%%%%%%%%%%%%%%%%%%%%%%
%%%%%%%%%%%%%%%%%%%%%%%%%%%%%%%%%%%%%%%%%%%%%%%%%%%%%%%%%%
\modHeadsection{メインプログラムの例}
\index{メインプログラム}メインプログラムについては、個々の\index{めいさい(モールド)@明細(モールド)}明細の情報(社内機密情報)を含む。
そのため、\pageautoref{subsec:notopenwork}に則り、記載は代表的・典型的なものに留める。\\

%%%%%%%%%%%%%%%%%%%%%%%%%%%%%%%%%%%%%%%%%%%%%%%%%%%%%%%%%%
%% Prg. \MainExOne %%%%%%%%%%%%%%%%%%%%%%%%%%%%%%%%%%%%%%%
%%%%%%%%%%%%%%%%%%%%%%%%%%%%%%%%%%%%%%%%%%%%%%%%%%%%%%%%%%
\modcaptionof{lstlisting}{\MainExOne:\index{メインプログラム}メインプログラムの例1}
\inputminted{gcode}{../Created_NC_Programs/main_program_examples/\nameMainExOne}


\clearrightpage
%%%%%%%%%%%%%%%%%%%%%%%%%%%%%%%%%%%%%%%%%%%%%%%%%%%%%%%%%%
%% Prg. \MainExTwo %%%%%%%%%%%%%%%%%%%%%%%%%%%%%%%%%%%%%%%
%%%%%%%%%%%%%%%%%%%%%%%%%%%%%%%%%%%%%%%%%%%%%%%%%%%%%%%%%%
\modcaptionof{lstlisting}{\MainExTwo:\index{メインプログラム}メインプログラムの例2}
\lstinputlisting[style=Gcode-more]{../Created_NC_Programs/main_program_examples/\nameMainExTwo}



\clearrightpage
%%%%%%%%%%%%%%%%%%%%%%%%%%%%%%%%%%%%%%%%%%%%%%%%%%%%%%%%%%
%% section E.2 %%%%%%%%%%%%%%%%%%%%%%%%%%%%%%%%%%%%%%%%%%%
%%%%%%%%%%%%%%%%%%%%%%%%%%%%%%%%%%%%%%%%%%%%%%%%%%%%%%%%%%
\modHeadsection{測定用(\Dimple 除く)サブプログラム}


%%%%%%%%%%%%%%%%%%%%%%%%%%%%%%%%%%%%%%%%%%%%%%%%%%%%%%%%%%
%% subsection E.2.1 %%%%%%%%%%%%%%%%%%%%%%%%%%%%%%%%%%%%%%
%%%%%%%%%%%%%%%%%%%%%%%%%%%%%%%%%%%%%%%%%%%%%%%%%%%%%%%%%%
\subsection{\index{しんだしそくてい@芯出し測定}芯出し測定用サブプログラム}

%%%%%%%%%%%%%%%%%%%%%%%%%%%%%%%%%%%%%%%%%%%%%%%%%%%%%%%%%%
%% Prg. \MXOThickness %%%%%%%%%%%%%%%%%%%%%%%%%%%%%%%%%%%%
%%%%%%%%%%%%%%%%%%%%%%%%%%%%%%%%%%%%%%%%%%%%%%%%%%%%%%%%%%
\modcaptionof{lstlisting}{\MXOThickness\,:測定 両側 外側中心\texorpdfstring{$X$}{X}}
\lstinputlisting[style=Gcode-more]{../Created_NC_Programs/sub_programs/\nameMXOThickness.nc}

\clearrightpage
%%%%%%%%%%%%%%%%%%%%%%%%%%%%%%%%%%%%%%%%%%%%%%%%%%%%%%%%%%
%% Prg. \MYOThickness %%%%%%%%%%%%%%%%%%%%%%%%%%%%%%%%%%%%
%%%%%%%%%%%%%%%%%%%%%%%%%%%%%%%%%%%%%%%%%%%%%%%%%%%%%%%%%%
\modcaptionof{lstlisting}{\MYOThickness\,:測定 両側 外側中心\texorpdfstring{$Y$}{Y}}
\lstinputlisting[style=Gcode-more]{../Created_NC_Programs/sub_programs/\nameMYOThickness.nc}

\clearrightpage
%%%%%%%%%%%%%%%%%%%%%%%%%%%%%%%%%%%%%%%%%%%%%%%%%%%%%%%%%%
%% Prg. \MYOThickness %%%%%%%%%%%%%%%%%%%%%%%%%%%%%%%%%%%%
%%%%%%%%%%%%%%%%%%%%%%%%%%%%%%%%%%%%%%%%%%%%%%%%%%%%%%%%%%
\modcaptionof{lstlisting}{\MXOface\,:測定 片側 \KeywayCenter\texorpdfstring{$X$}{X}(A側外面側測定)}
\lstinputlisting[style=Gcode-more]{../Created_NC_Programs/sub_programs/\nameMXOface.nc}

\clearrightpage
%%%%%%%%%%%%%%%%%%%%%%%%%%%%%%%%%%%%%%%%%%%%%%%%%%%%%%%%%%
%% Prg. \MXIWidth %%%%%%%%%%%%%%%%%%%%%%%%%%%%%%%%%%%%%%%%
%%%%%%%%%%%%%%%%%%%%%%%%%%%%%%%%%%%%%%%%%%%%%%%%%%%%%%%%%%
\modcaptionof{lstlisting}{\MXIWidth\,:測定 両側 内側中心\texorpdfstring{$X$}{X}}
\lstinputlisting[style=Gcode-more]{../Created_NC_Programs/sub_programs/\nameMXIWidth.nc}

\clearrightpage
%%%%%%%%%%%%%%%%%%%%%%%%%%%%%%%%%%%%%%%%%%%%%%%%%%%%%%%%%%
%% Prg. \MYIWidth %%%%%%%%%%%%%%%%%%%%%%%%%%%%%%%%%%%%%%%%
%%%%%%%%%%%%%%%%%%%%%%%%%%%%%%%%%%%%%%%%%%%%%%%%%%%%%%%%%%
\modcaptionof{lstlisting}{\MYIWidth\,:測定 両側 内側中心\texorpdfstring{$Y$}{Y}}
\lstinputlisting[style=Gcode-more]{../Created_NC_Programs/sub_programs/\nameMYIWidth.nc}

\clearrightpage
%%%%%%%%%%%%%%%%%%%%%%%%%%%%%%%%%%%%%%%%%%%%%%%%%%%%%%%%%%
%% Prg. \MXface %%%%%%%%%%%%%%%%%%%%%%%%%%%%%%%%%%%%%%%%%%
%%%%%%%%%%%%%%%%%%%%%%%%%%%%%%%%%%%%%%%%%%%%%%%%%%%%%%%%%%
\modcaptionof{lstlisting}{\MXIface\,:測定 片側 \OutcutCenter\texorpdfstring{$X$}{X}(C側内面測定)}
\lstinputlisting[style=Gcode-more]{../Created_NC_Programs/sub_programs/\nameMXIface.nc}


\clearrightpage
%%%%%%%%%%%%%%%%%%%%%%%%%%%%%%%%%%%%%%%%%%%%%%%%%%%%%%%%%%
%% subsection E.2.1 %%%%%%%%%%%%%%%%%%%%%%%%%%%%%%%%%%%%%%
%%%%%%%%%%%%%%%%%%%%%%%%%%%%%%%%%%%%%%%%%%%%%%%%%%%%%%%%%%
\subsection{\expandafterindex{サブプログラム(\yomiCenterlineEndFaceDif)@サブプログラム(\nameCenterlineEndFaceDif)}\nameCenterlineEndFaceDif 測定用サブプログラム}

%%%%%%%%%%%%%%%%%%%%%%%%%%%%%%%%%%%%%%%%%%%%%%%%%%%%%%%%%%
%% Prg. \Mcenterline %%%%%%%%%%%%%%%%%%%%%%%%%%%%%%%%%%%%%
%%%%%%%%%%%%%%%%%%%%%%%%%%%%%%%%%%%%%%%%%%%%%%%%%%%%%%%%%%
\modcaptionof{lstlisting}{\Mcenterline\,:測定 片側 \CenterlineEndFaceDif(C側外削面$Z$, B側外削面$Y$測定)}
\lstinputlisting[style=Gcode-more]{../Created_NC_Programs/sub_programs/\nameMcenterline.nc}


\clearrightpage
%%%%%%%%%%%%%%%%%%%%%%%%%%%%%%%%%%%%%%%%%%%%%%%%%%%%%%%%%%
%% section E.3 %%%%%%%%%%%%%%%%%%%%%%%%%%%%%%%%%%%%%%%%%%%
%%%%%%%%%%%%%%%%%%%%%%%%%%%%%%%%%%%%%%%%%%%%%%%%%%%%%%%%%%
\modHeadsection{加工用(\Dimple 除く)サブプログラム}


%%%%%%%%%%%%%%%%%%%%%%%%%%%%%%%%%%%%%%%%%%%%%%%%%%%%%%%%%%
%% Prg. \KTanmenRight %%%%%%%%%%%%%%%%%%%%%%%%%%%%%%%%%%%%
%%%%%%%%%%%%%%%%%%%%%%%%%%%%%%%%%%%%%%%%%%%%%%%%%%%%%%%%%%
\modcaptionof{lstlisting}{\KTanmenRight\,:加工 \EndFacecut{} コーナーR 右回り1周}
\lstinputlisting[style=Gcode-more]{../Created_NC_Programs/sub_programs/\nameKTanmenRight.nc}


\clearrightpage
%%%%%%%%%%%%%%%%%%%%%%%%%%%%%%%%%%%%%%%%%%%%%%%%%%%%%%%%%%
%% Prg. \KGaisakuRLeft %%%%%%%%%%%%%%%%%%%%%%%%%%%%%%%%%%%
%%%%%%%%%%%%%%%%%%%%%%%%%%%%%%%%%%%%%%%%%%%%%%%%%%%%%%%%%%
\modcaptionof{lstlisting}{\KGaisakuRLeft\,:加工 \Outcut{} コーナーR 左回り1周}
\lstinputlisting[style=Gcode-more]{../Created_NC_Programs/sub_programs/\nameKGaisakuRLeft.nc}


\clearrightpage
%%%%%%%%%%%%%%%%%%%%%%%%%%%%%%%%%%%%%%%%%%%%%%%%%%%%%%%%%%
%% Prg. \KMizoConerLeft %%%%%%%%%%%%%%%%%%%%%%%%%%%%%%%%%%
%%%%%%%%%%%%%%%%%%%%%%%%%%%%%%%%%%%%%%%%%%%%%%%%%%%%%%%%%%
\modcaptionof{lstlisting}{\KMizoConerLeft\,:加工 \Keyway{} 左回り1周}
\lstinputlisting[style=Gcode-more]{../Created_NC_Programs/sub_programs/\nameKMizoConerLeft.nc}


\clearrightpage
%%%%%%%%%%%%%%%%%%%%%%%%%%%%%%%%%%%%%%%%%%%%%%%%%%%%%%%%%%
%% Prg. \KSotoMentoriRLeft %%%%%%%%%%%%%%%%%%%%%%%%%%%%%%%
%%%%%%%%%%%%%%%%%%%%%%%%%%%%%%%%%%%%%%%%%%%%%%%%%%%%%%%%%%
\modcaptionof{lstlisting}{\KSotoMentoriRLeft\,:加工 \EndFaceOutChamfer{} コーナーR 左回り1周}
\lstinputlisting[style=Gcode-more]{../Created_NC_Programs/sub_programs/\nameKSotoMentoriRLeft.nc}


\clearrightpage
%%%%%%%%%%%%%%%%%%%%%%%%%%%%%%%%%%%%%%%%%%%%%%%%%%%%%%%%%%
%% Prg. \KUchiMentoriRLeft %%%%%%%%%%%%%%%%%%%%%%%%%%%%%%%
%%%%%%%%%%%%%%%%%%%%%%%%%%%%%%%%%%%%%%%%%%%%%%%%%%%%%%%%%%
\modcaptionof{lstlisting}{\KUchiMentoriRLeft\,:加工 \EndFaceInChamfer{} コーナーR 左回り1周}
\lstinputlisting[style=Gcode-more]{../Created_NC_Programs/sub_programs/\nameKUchiMentoriRLeft.nc}


\clearrightpage
%%%%%%%%%%%%%%%%%%%%%%%%%%%%%%%%%%%%%%%%%%%%%%%%%%%%%%%%%%
%% Prg. \KUchiMentoriRLeft %%%%%%%%%%%%%%%%%%%%%%%%%%%%%%%
%%%%%%%%%%%%%%%%%%%%%%%%%%%%%%%%%%%%%%%%%%%%%%%%%%%%%%%%%%
\modcaptionof{lstlisting}{\KOLeftAR\,:外側 左回り1周 右上始まり}
\lstinputlisting[style=Gcode-more]{../Created_NC_Programs/sub_programs/\nameKOLeftAR.nc}


\clearrightpage
%%%%%%%%%%%%%%%%%%%%%%%%%%%%%%%%%%%%%%%%%%%%%%%%%%%%%%%%%%
%% Prg. \KUchiMentoriRLeft %%%%%%%%%%%%%%%%%%%%%%%%%%%%%%%
%%%%%%%%%%%%%%%%%%%%%%%%%%%%%%%%%%%%%%%%%%%%%%%%%%%%%%%%%%
\modcaptionof{lstlisting}{\KILeftAC\,:内側 左回り1周 中央上始まり}
\lstinputlisting[style=Gcode-more]{../Created_NC_Programs/sub_programs/\nameKILeftAC.nc}


\clearrightpage
%%%%%%%%%%%%%%%%%%%%%%%%%%%%%%%%%%%%%%%%%%%%%%%%%%%%%%%%%%
%% section E.4 %%%%%%%%%%%%%%%%%%%%%%%%%%%%%%%%%%%%%%%%%%%
%%%%%%%%%%%%%%%%%%%%%%%%%%%%%%%%%%%%%%%%%%%%%%%%%%%%%%%%%%
\modHeadsection{\Dimple 用 移動・測定・加工用サブプログラム}


%%%%%%%%%%%%%%%%%%%%%%%%%%%%%%%%%%%%%%%%%%%%%%%%%%%%%%%%%%
%% Prg. \DLone %%%%%%%%%%%%%%%%%%%%%%%%%%%%%%%%%%%%%%%%%%%
%%%%%%%%%%%%%%%%%%%%%%%%%%%%%%%%%%%%%%%%%%%%%%%%%%%%%%%%%%
\modcaptionof{lstlisting}{\DLone\,:移動 各\Dimple 列の中心上}
\lstinputlisting[style=Gcode-more]{../Created_NC_Programs/sub_programs/\nameDLone.nc}


\clearrightpage
%%%%%%%%%%%%%%%%%%%%%%%%%%%%%%%%%%%%%%%%%%%%%%%%%%%%%%%%%%
%% Prg. \DLtwoAC %%%%%%%%%%%%%%%%%%%%%%%%%%%%%%%%%%%%%%%%%
%%%%%%%%%%%%%%%%%%%%%%%%%%%%%%%%%%%%%%%%%%%%%%%%%%%%%%%%%%
\modcaptionof{lstlisting}{\DLtwoAC\,:移動 AC面 列内の各\Dimple 上}
\lstinputlisting[style=Gcode-more]{../Created_NC_Programs/sub_programs/\nameDLtwoAC.nc}


\clearrightpage
%%%%%%%%%%%%%%%%%%%%%%%%%%%%%%%%%%%%%%%%%%%%%%%%%%%%%%%%%%
%% Prg. \DLtwoBD %%%%%%%%%%%%%%%%%%%%%%%%%%%%%%%%%%%%%%%%%
%%%%%%%%%%%%%%%%%%%%%%%%%%%%%%%%%%%%%%%%%%%%%%%%%%%%%%%%%%
\modcaptionof{lstlisting}{\DLtwoBD\,:移動 BD面 列内の各\Dimple 上}
\lstinputlisting[style=Gcode-more]{../Created_NC_Programs/sub_programs/\nameDLtwoBD.nc}


\clearrightpage
%%%%%%%%%%%%%%%%%%%%%%%%%%%%%%%%%%%%%%%%%%%%%%%%%%%%%%%%%%
%% Prg. \DMLthreeAC %%%%%%%%%%%%%%%%%%%%%%%%%%%%%%%%%%%%%%
%%%%%%%%%%%%%%%%%%%%%%%%%%%%%%%%%%%%%%%%%%%%%%%%%%%%%%%%%%
\modcaptionof{lstlisting}{\DMLthreeAC\,:測定 AC面 \Dimple 位置\texorpdfstring{$X$}{X}}
\lstinputlisting[style=Gcode-more]{../Created_NC_Programs/sub_programs/\nameDMLthreeAC.nc}


\clearrightpage
%%%%%%%%%%%%%%%%%%%%%%%%%%%%%%%%%%%%%%%%%%%%%%%%%%%%%%%%%%
%% Prg. \DMLthreeBD %%%%%%%%%%%%%%%%%%%%%%%%%%%%%%%%%%%%%%
%%%%%%%%%%%%%%%%%%%%%%%%%%%%%%%%%%%%%%%%%%%%%%%%%%%%%%%%%%
\modcaptionof{lstlisting}{\DMLthreeBD\,:測定 BD面 \Dimple 位置\texorpdfstring{$Y$}{Y}}
\lstinputlisting[style=Gcode-more]{../Created_NC_Programs/sub_programs/\nameDMLthreeBD.nc}


\clearrightpage
%%%%%%%%%%%%%%%%%%%%%%%%%%%%%%%%%%%%%%%%%%%%%%%%%%%%%%%%%%
%% Prg. \DKLthreeAC %%%%%%%%%%%%%%%%%%%%%%%%%%%%%%%%%%%%%%
%%%%%%%%%%%%%%%%%%%%%%%%%%%%%%%%%%%%%%%%%%%%%%%%%%%%%%%%%%
\modcaptionof{lstlisting}{\DKLthreeAC\,:加工 AC面 \Dimple\texorpdfstring{$X$}{X}}
\lstinputlisting[style=Gcode-more]{../Created_NC_Programs/sub_programs/\nameDKLthreeAC.nc}


\clearrightpage
%%%%%%%%%%%%%%%%%%%%%%%%%%%%%%%%%%%%%%%%%%%%%%%%%%%%%%%%%%
%% Prg. \DKLthreeBD %%%%%%%%%%%%%%%%%%%%%%%%%%%%%%%%%%%%%%
%%%%%%%%%%%%%%%%%%%%%%%%%%%%%%%%%%%%%%%%%%%%%%%%%%%%%%%%%%
\modcaptionof{lstlisting}{\DKLthreeBD\,:加工 BD面 \Dimple\texorpdfstring{$Y$}{Y}}
\lstinputlisting[style=Gcode-more]{../Created_NC_Programs/sub_programs/\nameDKLthreeBD.nc}



\clearrightpage
%%%%%%%%%%%%%%%%%%%%%%%%%%%%%%%%%%%%%%%%%%%%%%%%%%%%%%%%%%
%% section I.5 %%%%%%%%%%%%%%%%%%%%%%%%%%%%%%%%%%%%%%%%%%%
%%%%%%%%%%%%%%%%%%%%%%%%%%%%%%%%%%%%%%%%%%%%%%%%%%%%%%%%%%
\modHeadsection{その他のサブプログラム}


%%%%%%%%%%%%%%%%%%%%%%%%%%%%%%%%%%%%%%%%%%%%%%%%%%%%%%%%%%
%% subsection I.5.1 %%%%%%%%%%%%%%%%%%%%%%%%%%%%%%%%%%%%%%
%%%%%%%%%%%%%%%%%%%%%%%%%%%%%%%%%%%%%%%%%%%%%%%%%%%%%%%%%%
\subsection{一時停止およびワーク確認}


%%%%%%%%%%%%%%%%%%%%%%%%%%%%%%%%%%%%%%%%%%%%%%%%%%%%%%%%%%
%% Prg. \OpauseCheck %%%%%%%%%%%%%%%%%%%%%%%%%%%%%%%%%%%%%
%%%%%%%%%%%%%%%%%%%%%%%%%%%%%%%%%%%%%%%%%%%%%%%%%%%%%%%%%%
\modcaptionof{lstlisting}{\OpauseCheck\,:90\texorpdfstring{$\boldsymbol{^\circ}$}{°}回転 扉前一時停止}
\lstinputlisting[style=Gcode-more]{../Created_NC_Programs/sub_programs/\nameOpauseCheck.nc}


\clearrightpage
%%%%%%%%%%%%%%%%%%%%%%%%%%%%%%%%%%%%%%%%%%%%%%%%%%%%%%%%%%
%% subsection I.5.2 %%%%%%%%%%%%%%%%%%%%%%%%%%%%%%%%%%%%%%
%%%%%%%%%%%%%%%%%%%%%%%%%%%%%%%%%%%%%%%%%%%%%%%%%%%%%%%%%%
\subsection{タッチセンサープローブ電源スイッチ}
%%%%%%%%%%%%%%%%%%%%%%%%%%%%%%%%%%%%%%%%%%%%%%%%%%%%%%%%%%
%% Prg. \OsensorOn %%%%%%%%%%%%%%%%%%%%%%%%%%%%%%%%%%%%%%%
%%%%%%%%%%%%%%%%%%%%%%%%%%%%%%%%%%%%%%%%%%%%%%%%%%%%%%%%%%
\modcaptionof{lstlisting}{\OsensorOn\,:タッチセンサー電源ON}
\lstinputlisting[style=Gcode-more]{../Created_NC_Programs/sub_programs/\nameOsensorOn.nc}


\clearpage
%%%%%%%%%%%%%%%%%%%%%%%%%%%%%%%%%%%%%%%%%%%%%%%%%%%%%%%%%%
%% Prg. \OsensorOff %%%%%%%%%%%%%%%%%%%%%%%%%%%%%%%%%%%%%%
%%%%%%%%%%%%%%%%%%%%%%%%%%%%%%%%%%%%%%%%%%%%%%%%%%%%%%%%%%
\modcaptionof{lstlisting}{\OsensorOff\,:タッチセンサー電源OFF}
\lstinputlisting[style=Gcode-more]{../Created_NC_Programs/sub_programs/\nameOsensorOff.nc}


\clearrightpage
%%%%%%%%%%%%%%%%%%%%%%%%%%%%%%%%%%%%%%%%%%%%%%%%%%%%%%%%%%
%% subsection I.5.2 %%%%%%%%%%%%%%%%%%%%%%%%%%%%%%%%%%%%%%
%%%%%%%%%%%%%%%%%%%%%%%%%%%%%%%%%%%%%%%%%%%%%%%%%%%%%%%%%%
\subsection{暖機運転}
%%%%%%%%%%%%%%%%%%%%%%%%%%%%%%%%%%%%%%%%%%%%%%%%%%%%%%%%%%
%% Prg. \OsensorOn %%%%%%%%%%%%%%%%%%%%%%%%%%%%%%%%%%%%%%%
%%%%%%%%%%%%%%%%%%%%%%%%%%%%%%%%%%%%%%%%%%%%%%%%%%%%%%%%%%
\modcaptionof{lstlisting}{\OwarmingupA\,:暖機運転}
\lstinputlisting[style=Gcode-more]{../Created_NC_Programs/sub_programs/\nameOwarmingupA.nc}


\clearrightpage
%%%%%%%%%%%%%%%%%%%%%%%%%%%%%%%%%%%%%%%%%%%%%%%%%%%%%%%%%%
%% Prg. \OsensorOn %%%%%%%%%%%%%%%%%%%%%%%%%%%%%%%%%%%%%%%
%%%%%%%%%%%%%%%%%%%%%%%%%%%%%%%%%%%%%%%%%%%%%%%%%%%%%%%%%%
\modcaptionof{lstlisting}{\Owarmingup\,:暖機運転用サブプログラム}
\lstinputlisting[style=Gcode-more]{../Created_NC_Programs/sub_programs/\nameOwarmingup.nc}


\clearrightpage
%%%%%%%%%%%%%%%%%%%%%%%%%%%%%%%%%%%%%%%%%%%%%%%%%%%%%%%%%%
%% subsection I.5.2 %%%%%%%%%%%%%%%%%%%%%%%%%%%%%%%%%%%%%%
%%%%%%%%%%%%%%%%%%%%%%%%%%%%%%%%%%%%%%%%%%%%%%%%%%%%%%%%%%
\subsection{工具長自動測定}
%%%%%%%%%%%%%%%%%%%%%%%%%%%%%%%%%%%%%%%%%%%%%%%%%%%%%%%%%%
%% Prg. \OsensorOn %%%%%%%%%%%%%%%%%%%%%%%%%%%%%%%%%%%%%%%
%%%%%%%%%%%%%%%%%%%%%%%%%%%%%%%%%%%%%%%%%%%%%%%%%%%%%%%%%%
\modcaptionof{lstlisting}{\OtoolLengthA\,:指定工具 工具長自動測定}
\lstinputlisting[style=Gcode-more]{../Created_NC_Programs/sub_programs/\nameOtoolLengthA.nc}


\clearrightpage
%%%%%%%%%%%%%%%%%%%%%%%%%%%%%%%%%%%%%%%%%%%%%%%%%%%%%%%%%%
%% Prg. \OsensorOn %%%%%%%%%%%%%%%%%%%%%%%%%%%%%%%%%%%%%%%
%%%%%%%%%%%%%%%%%%%%%%%%%%%%%%%%%%%%%%%%%%%%%%%%%%%%%%%%%%
\modcaptionof{lstlisting}{\OtoolLength\,:工具長 自動測定用サブプログラム}
\lstinputlisting[style=Gcode-more]{../Created_NC_Programs/sub_programs/\nameOtoolLength.nc}

%!TEX root = ../RPA_for_Creating_Program_Note.tex
\setcounter{lstlisting}{0}


\modHeadchapter[lot]{バンドルのG-codeプログラム}
ここでは機械設置時付属の\index{G-codeプログラム}G-codeプログラムについて記載する。
これらの\index{プログラム(G-code)}プログラムは、当社の\index{しょくむちょさくぶつ@職務著作物}職務著作物ではないため、\pageautoref{subsec:standardscopyrightsSubcontractor}に則り、その詳細の記載は割愛する。



%%%%%%%%%%%%%%%%%%%%%%%%%%%%%%%%%%%%%%%%%%%%%%%%%%%%%%%%%%
%% section L.1 %%%%%%%%%%%%%%%%%%%%%%%%%%%%%%%%%%%%%%%%%%%
%%%%%%%%%%%%%%%%%%%%%%%%%%%%%%%%%%%%%%%%%%%%%%%%%%%%%%%%%%
\modHeadsection{バンドルのプログラム 一覧}
\index{バンドルのプログラム}バンドルのプログラムは以下のとおりである。\\

\begin{multicollongtblr}{\Gprgbox{O7000}-\Gprgbox{O7313}}{cX[l]l}
{\ttfamily O}番号 & 内容 & 使用prg\\
\Gprgbox{O7000} & パラメータの設定 & -\\
\Gprgbox{O7100} & \index{こうぐちょうほせいち@工具長補正値}工具長補正値 自動計測 & \Gprgbox{O9100}\\
\Gprgbox{O7101} & 工具長オフセット量の修正 & \Gprgbox{O9101}\\
\Gprgbox{O7102} & 工具破損検出 & \Gprgbox{O9102}\\
\Gprgbox{O7103} & 自動工具径測定 & \Gprgbox{O9103}\\
\Gprgbox{O7300} & \ttNum502, \ttNum503の測定 & \Gprgbox{O9300}\\
\Gprgbox{O7301} &  & \Gprgbox{O9301}\\
\Gprgbox{O7302} &  & \Gprgbox{O9302}\\
\Gprgbox{O7303} &  & \Gprgbox{O9303}\\
\Gprgbox{O7310} &  & \Gprgbox{O9310}\\
\Gprgbox{O7311} &  & \Gprgbox{O9311}\\
\Gprgbox{O7312} &  & \Gprgbox{O9312}\\
\Gprgbox{O7313} & $Z$軸座標設定 & \Gprgbox{O9313}
\end{multicollongtblr}

\begin{multicollongtblr}{\Gprgbox{O8123}}{cX[l]l}
{\ttfamily O}番号 & 内容 & 使用prg\\
\Gprgbox{O8123} & 初期加工原点設定{\ttfamily G54}, {\ttfamily G55}, {\ttfamily G56}, {\ttfamily G57} & -\\
\Gprgbox{O8998} & パレット認識マクロ用 変数値 & -\\
\Gprgbox{O8999} & パレット入換え & -
\end{multicollongtblr}

\clearpage
\begin{multicollongtblr}{\Gprgbox{O9001}-\Gprgbox{O9393}}{cX[l]l}
{\ttfamily O}番号 & 内容 & 使用prg\\
\Gprgbox{O9001} & 自動パレット交換 & -\\
\Gprgbox{O9002} & \expandafterindex{No.1パレット(\yomiDMC)@No.1パレット(\nameDMC)}No.1パレット & \Gprgbox{O9001}\\
\Gprgbox{O9003} & \expandafterindex{No.2パレット(\yomiDMC)@No.2パレット(\nameDMC)}No.2パレット & \Gprgbox{O9001}\\
\Gprgbox{O9006} & 自動工具交換 & -\\
\Gprgbox{O9020} & パレット認識マクロ & \Gprgbox{O8998}\\
\Gprgbox{O9021} & No.1パレットの選択確認 & -\\
\Gprgbox{O9022} & No.2パレットの選択確認 & -\\
\Gprgbox{O9100} & \index{こうぐちょうほせいち@工具長補正値}工具長補正値 自動測定 & -\\
\Gprgbox{O9101} & 工具長オフセット量の修正 & -\\
\Gprgbox{O9102} & 工具破損検出 & -\\
\Gprgbox{O9103} & 自動工具径測定 & -\\
\Gprgbox{O9200} & パラメタ確認 & -\\
\Gprgbox{O9300} & \index{タッチセンサープローブのたおれ@タッチセンサープローブの倒れ}タッチセンサープローブの倒れの補正量の測定 & \Gprgbox{O9392}\Gprgbox{O9393}\\
\Gprgbox{O9301} & 自動$XYZ$端面芯出し & \Gprgbox{O9390}\Gprgbox{O9391}\Gprgbox{O9392}\Gprgbox{O9393}\\
\Gprgbox{O9302} & 自動内円芯出し & \Gprgbox{O9392}\Gprgbox{O9393}\\
\Gprgbox{O9303} & 自動外円芯出し & \Gprgbox{O9392}\Gprgbox{O9393}\\
\Gprgbox{O9310} & 手動$XY$端面芯出し & \Gprgbox{O9392}\Gprgbox{O9393}\\
\Gprgbox{O9311} & 手動内円芯出し & \Gprgbox{O9392}\Gprgbox{O9393}\\
\Gprgbox{O9312} & 手動外円芯出し & \Gprgbox{O9392}\Gprgbox{O9393}\\
\Gprgbox{O9313} & $Z$座標系設定 & \Gprgbox{O9392}\Gprgbox{O9393}\\
\Gprgbox{O9390} &  & \Gprgbox{O9392}\\
\Gprgbox{O9391} &  & \Gprgbox{O9392}\\
\Gprgbox{O9392} &  & -\\
\Gprgbox{O9393} &  & -
\end{multicollongtblr}



\clearpage
%%%%%%%%%%%%%%%%%%%%%%%%%%%%%%%%%%%%%%%%%%%%%%%%%%%%%%%%%%
%% section H.2 %%%%%%%%%%%%%%%%%%%%%%%%%%%%%%%%%%%%%%%%%%%
%%%%%%%%%%%%%%%%%%%%%%%%%%%%%%%%%%%%%%%%%%%%%%%%%%%%%%%%%%
\modHeadsection{使用コモン変数}

\begin{multicollongtblr}{使用コモン変数:\Gprgbox{O7000}-\Gprgbox{O7313}}{ccX[l]}
{\ttfamily O}番号 & 内容 & 使用prg\\
\Gprgbox{O7000} & LHS &
  \ttNum500, \ttNum501, \ttNum504, \ttNum505, \ttNum506, \ttNum507, \ttNum512, \ttNum513, \ttNum514, \ttNum516, \ttNum517, \ttNum518, \ttNum519, \ttNum523, \ttNum524\\
\Gprgbox{O7100} & - & -\\
\Gprgbox{O7101} & - & -\\
\Gprgbox{O7102} & - & -\\
\Gprgbox{O7103} & - & -\\
\Gprgbox{O7300} & - & -\\
\Gprgbox{O7301} & - & -\\
\Gprgbox{O7302} & - & -\\
\Gprgbox{O7303} & - & -\\
\Gprgbox{O7310} & - & -\\
\Gprgbox{O7311} & - & -\\
\Gprgbox{O7312} & - & -\\
\Gprgbox{O7313} & - & -\\
\end{multicollongtblr}

\begin{multicollongtblr}{使用コモン変数:\Gprgbox{O8123}-\Gprgbox{O8999}}{ccX[l]}
{\ttfamily O}番号 & 内容 & 使用prg\\
\Gprgbox{O8123} & - & -\\
\Gprgbox{O8998} & LHS & \ttNum100, \ttNum101\\
\Gprgbox{O8999} & - & -\\
\end{multicollongtblr}

\clearpage
\begin{multicollongtblr}{使用コモン変数:\Gprgbox{O9001}-\Gprgbox{O9200}}{ccX[l]}
{\ttfamily O}番号 & \SetCell[c=2]{l}使用コモン変数\\
\Gprgbox{O9001} & - & -\\
\Gprgbox{O9002} & - & -\\
\Gprgbox{O9003} & - & -\\
\SetCell[r=2]{h,c}
\Gprgbox{O9006} & LHS & \ttNum137, \ttNum138, \ttNum139\\
               & RHS & \ttNum137\\
\SetCell[r=2]{h,c}
\Gprgbox{O9020} & LHS & \ttNum149\\
               & RHS & \ttNum101, \ttNum102\\
\Gprgbox{O9021} & - & -\\
\Gprgbox{O9022} & - & -\\
\Gprgbox{O9100} & RHS & \ttNum504, \ttNum505, \ttNum507, \ttNum513, \ttNum514, \ttNum516, \ttNum517\\
\Gprgbox{O9101} & RHS & \ttNum504, \ttNum506, \ttNum513, \ttNum514, \ttNum516, \ttNum517, \ttNum518\\
\Gprgbox{O9102} & RHS & \ttNum504, \ttNum506, \ttNum507, \ttNum513, \ttNum514, \ttNum516, \ttNum517, \ttNum518\\
\Gprgbox{O9103} & RHS & \ttNum516, \ttNum517, \ttNum519, \ttNum523, \ttNum524\\
\SetCell[r=2]{h,c}
\Gprgbox{O9200} & LHS & \ttNum120, \ttNum121, \ttNum122, \ttNum123, \ttNum124, \ttNum125, \ttNum126, \ttNum130, \ttNum131, \ttNum132, \ttNum133\\
               & RHS & \ttNum120, \ttNum121, \ttNum122, \ttNum123, \ttNum124, \ttNum125, \ttNum126\\
\end{multicollongtblr}

\begin{multicollongtblr}{使用コモン変数:\Gprgbox{O9300}-\Gprgbox{O9393}}{ccX[l]}
{\ttfamily O}番号 & \SetCell[c=2]{l}使用コモン変数\\
\SetCell[r=2]{h,c}
\Gprgbox{O9300} & LHS & \ttNum502, \ttNum503\\
               & RHS & \ttNum500, \ttNum501, \ttNum512, \ttNum514\\
\Gprgbox{O9301} & RHS & \ttNum500, \ttNum501, \ttNum502, \ttNum512, \ttNum514\\
\Gprgbox{O9302} & RHS & \ttNum500, \ttNum501, \ttNum502, \ttNum503, \ttNum512, \ttNum514\\
\Gprgbox{O9303} & RHS & \ttNum500, \ttNum501, \ttNum502, \ttNum503, \ttNum512, \ttNum514\\
\Gprgbox{O9310} & RHS & \ttNum501, \ttNum502, \ttNum503, \ttNum512, \ttNum514\\
\Gprgbox{O9311} & RHS & \ttNum501, \ttNum502, \ttNum503, \ttNum514\\
\Gprgbox{O9312} & RHS & \ttNum501, \ttNum502, \ttNum503, \ttNum514\\
\Gprgbox{O9313} & RHS & \ttNum501, \ttNum509, \ttNum514\\
\Gprgbox{O9390} & RHS & \ttNum500, \ttNum501, \ttNum514\\
\Gprgbox{O9391} & RHS & \ttNum500, \ttNum501, \ttNum503, \ttNum514\\
\Gprgbox{O9392} & - & -\\
\Gprgbox{O9393} & - & -\\
\end{multicollongtblr}



\clearpage
%%%%%%%%%%%%%%%%%%%%%%%%%%%%%%%%%%%%%%%%%%%%%%%%%%%%%%%%%%
%% section H.2 %%%%%%%%%%%%%%%%%%%%%%%%%%%%%%%%%%%%%%%%%%%
%%%%%%%%%%%%%%%%%%%%%%%%%%%%%%%%%%%%%%%%%%%%%%%%%%%%%%%%%%
\modHeadsection{使用システム変数}

\begin{multicollongtblr}{使用システム変数:\Gprgbox{O7000}-\Gprgbox{O7313}}{ccX[l]}
{\ttfamily O}番号 & 内容 & 使用prg\\
\Gprgbox{O7000} & - & -\\
\Gprgbox{O7100} & - & -\\
\Gprgbox{O7101} & - & -\\
\Gprgbox{O7102} & - & -\\
\Gprgbox{O7103} & - & -\\
\Gprgbox{O7300} & - & -\\
\Gprgbox{O7301} & - & -\\
\Gprgbox{O7302} & - & -\\
\Gprgbox{O7303} & - & -\\
\Gprgbox{O7310} & - & -\\
\Gprgbox{O7311} & - & -\\
\Gprgbox{O7312} & - & -\\
\Gprgbox{O7313} & - & -\\
\end{multicollongtblr}

\begin{multicollongtblr}{使用システム変数:\Gprgbox{O8123}-\Gprgbox{O8999}}{ccX[l]}
{\ttfamily O}番号 & \SetCell[c=2]{l}使用システム変数\\
\Gprgbox{O8123} & RHS & \ttNum3000\\
\Gprgbox{O8998} & - & -\\
\Gprgbox{O8999} & - & -\\
\end{multicollongtblr}

\clearpage
\begin{multicollongtblr}{使用システム変数:\Gprgbox{O9001}-\Gprgbox{O9200}}{ccX[l]}
{\ttfamily O}番号 & \SetCell[c=2]{l}使用システム変数\\
\Gprgbox{O9001} & RHS & \ttNum1000, \ttNum1001, \ttNum1002, \ttNum1003, \ttNum1010, \ttNum1011, \ttNum1012, \ttNum1013, \ttNum1014, \ttNum1015, \ttNum3000, \ttNum3003, \ttNum4003\\
\Gprgbox{O9002} & RHS & \ttNum1000, \ttNum1001, \ttNum3000\\
\Gprgbox{O9003} & RHS & \ttNum1000, \ttNum1001, \ttNum3000\\
\Gprgbox{O9006} & RHS & \ttNum3000, \ttNum4003, \ttNum4107, \ttNum4111\\
\Gprgbox{O9020} & RHS & \ttNum1000, \ttNum1001, \ttNum3000\\
\Gprgbox{O9021} & RHS & \ttNum1000, \ttNum1001, \ttNum3000\\
\Gprgbox{O9022} & RHS & \ttNum1000, \ttNum1001, \ttNum3000\\
\SetCell[r=2]{h,c}
\Gprgbox{O9100} & LHS & \ttNum1000x,~\ttNum1100x~[x=1-200]\\
               & RHS & \ttNum3000, \ttNum4001, \ttNum4003, \ttNum4012, \ttNum4130, \ttNum5021, \ttNum5022, \ttNum5043, \ttNum5063, \ttNum5203, \ttNum52x3~[x=(1-6)*20], \ttNum70y3~[y=(1-96)*20]\\
\SetCell[r=2]{h,c}
\Gprgbox{O9101} & LHS & \ttNum1000x~[x=1-200]\\
               & RHS & \ttNum3000, \ttNum4001, \ttNum4003, \ttNum4111, \ttNum5021, \ttNum5022, \ttNum5023, \ttNum5043, \ttNum5063, \ttNum1000x, \ttNum1100x~[x=1-200]\\
\Gprgbox{O9102} & RHS & \ttNum3000, \ttNum4001, \ttNum4003, \ttNum4111, \ttNum5021, \ttNum5022, \ttNum5023, \ttNum5043, \ttNum5063, \ttNum1000x,~\ttNum1100x~[x=1-200]\\
\SetCell[r=2]{h,c}
\Gprgbox{O9103} & LHS & \ttNum1600x,~\ttNum1700x~[x=1-200]\\
               & RHS & \ttNum3000, \ttNum4001, \ttNum4003, \ttNum5021, \ttNum5022, \ttNum5041, \ttNum5042, \ttNum5043, \ttNum5061, \ttNum5062\\
\SetCell[r=2]{h,c}
\Gprgbox{O9200} & LHS & \ttNum3006, \ttNum100000, \ttNum100002\\
               & RHS & \ttNum100010\\
\end{multicollongtblr}

\clearpage
\begin{multicollongtblr}[white]{使用システム変数:\Gprgbox{O9300}-\Gprgbox{O9393}}{ccX[l]}
\ttfamily O番号 & \SetCell[c=2]{l}使用システム変数\\
\Gprgbox{O9300} & RHS & \ttNum1004, \ttNum1005, \ttNum3000, \ttNum4001, \ttNum4003, \ttNum5001, \ttNum5002, \ttNum5003, \ttNum5061, \ttNum5062\\
\SetCell[r=2]{h,c}
\Gprgbox{O9301} & LHS & \ttNum52x1~[x=(1-6)*20], \ttNum70y1~[y=(1-96)*20]\\
               & RHS & \ttNum1004, \ttNum1005, \ttNum3000, \ttNum4001, \ttNum4003, \ttNum4012, \ttNum4130, \ttNum5001, \ttNum5003, \ttNum5061, \ttNum52x1~[x=(1-6)*20], \ttNum70y1~[y=(1-96)*20]\\
\SetCell[r=2]{h,c}
\Gprgbox{O9302} & LHS & \ttNum52x1,~\ttNum52x2~[x=(1-6)*20], \ttNum70y1,~\ttNum70y2~[y=(1-96)*20]\\
               & RHS & \ttNum1004, \ttNum1005, \ttNum3000, \ttNum4001, \ttNum4003, \ttNum4012, \ttNum4130, \ttNum5001, \ttNum5002, \ttNum5003, \ttNum5061, \ttNum5062, \ttNum52x1,~\ttNum52x2~[x=(1-6)*20], \ttNum70y1,~\ttNum70y2~[y=(1-96)*20]\\
\SetCell[r=2]{h,c}
\Gprgbox{O9303} & LHS & \ttNum52x1,~\ttNum52x2~[x=(1-6)*20], \ttNum70y1,~\ttNum70y2~[y=(1-96)*20]\\
               & RHS & \ttNum1004, \ttNum1005, \ttNum3000, \ttNum4001, \ttNum4003, \ttNum4012, \ttNum4130, \ttNum5001, \ttNum5002, \ttNum5003, \ttNum5061, \ttNum5062, \ttNum52x1,~\ttNum52x2~[x=(1-6)*20], \ttNum70y1,~\ttNum70y2~[y=(1-96)*20]\\
\SetCell[r=2]{h,c}
\Gprgbox{O9310} & LHS & \ttNum52x1,~\ttNum52x2~[x=(1-6)*20], \ttNum70y1,~\ttNum70y2~[y=(1-96)*20]\\*
               & RHS & \ttNum1004, \ttNum1005, \ttNum3000, \ttNum4001, \ttNum4003, \ttNum4012, \ttNum4130, \ttNum5061, \ttNum5062, \ttNum52x1,~\ttNum52x2~[x=(1-6)*20], \ttNum70y1,~\ttNum70y2~[y=(1-96)*20]\\
\SetCell[r=2]{h,c}
\Gprgbox{O9311} & LHS & \ttNum52x1,~\ttNum52x2~[x=(1-6)*20], \ttNum70y1,~\ttNum70y2~[y=(1-96)*20]\\*
               & RHS & \ttNum1004, \ttNum1005, \ttNum3000, \ttNum4001, \ttNum4003, \ttNum4012, \ttNum4130, \ttNum5003, \ttNum5061, \ttNum5062, \ttNum52x1,~\ttNum52x2~[x=(1-6)*20], \ttNum70y1,~\ttNum70y2~[y=(1-96)*20]\\
\SetCell[r=2]{h,c}
\Gprgbox{O9312} & LHS & \ttNum52x1,~\ttNum52x2~[x=(1-6)*20], \ttNum70y1,~\ttNum70y2~[y=(1-96)*20]\\*
               & RHS & \ttNum1004, \ttNum1005, \ttNum3000, \ttNum4001, \ttNum4003, \ttNum4012, \ttNum4130, \ttNum5003, \ttNum5061, \ttNum5062, \ttNum52x1,~\ttNum52x2~[x=(1-6)*20], \ttNum70y1,~\ttNum70y2~[y=(1-96)*20]\\
\SetCell[r=2]{h,c}
\Gprgbox{O9313} & LHS & \ttNum52x3~[x=(1-6)*20], \ttNum70y3~[y=(1-96)*20]\\*
               & RHS & \ttNum1004, \ttNum1005, \ttNum3000, \ttNum4012, \ttNum4130, \ttNum5043, \ttNum5063, \ttNum52x3~[x=(1-6)*20], \ttNum70y3~[y=(1-96)*20]\\
\SetCell[r=2]{h,c}
\Gprgbox{O9390} & LHS & \ttNum52x3~[x=(1-6)*20], \ttNum70y3~[y=(1-96)*20]\\*
               & RHS & \ttNum1004, \ttNum3000, \ttNum4003, \ttNum4012, \ttNum4130, \ttNum5003, \ttNum5043, \ttNum5063, \ttNum52x3~[x=(1-6)*20], \ttNum70y3~[y=(1-96)*20]\\
\Gprgbox{O9391} & LHS & \ttNum52x2~[x=(1-6)*20], \ttNum70y2~[y=(1-96)*20]\\*
               & RHS & \ttNum3000, \ttNum4003, \ttNum4012, \ttNum4130, \ttNum5002, \ttNum5003, \ttNum5062\\
\Gprgbox{O9392} & RHS & \ttNum1004, \ttNum3000\\
\Gprgbox{O9393} & RHS & \ttNum1004, \ttNum3000\\
\end{multicollongtblr}

\end{appendices}

\addtocontents{toc}{\protect\end{tocBox}}
\clearrightpage

\addtocontents{toc}{\protect\end{tcolorbox}}
\addtocontents{toc}{\protect\cleardoublepage}% add page break




%%%%%%%%%%%%%%%%%%%%%%%%%%%%%%%%%%%%%%%%%%%%%%%%%%%%%%%%%
%%               %%%%%%%%%%%%%%%%%%%%%%%%%%%%%%%%%%%%%%%%
%%               %%%%%%%%%%%%%%%%%%%%%%%%%%%%%%%%%%%%%%%%
%% Part Work P   %%%%%%%%%%%%%%%%%%%%%%%%%%%%%%%%%%%%%%%%
%%               %%%%%%%%%%%%%%%%%%%%%%%%%%%%%%%%%%%%%%%%
%%               %%%%%%%%%%%%%%%%%%%%%%%%%%%%%%%%%%%%%%%%
%%%%%%%%%%%%%%%%%%%%%%%%%%%%%%%%%%%%%%%%%%%%%%%%%%%%%%%%%
\addtocontents{toc}{\protect\begin{tcolorbox}[parttocstyle={Contents}{\the\numexpr\value{part}+1\relax}]}
\tPart{作業手順:マシニングの操作\TBW}{概要}{%
\paragraph*{目標(なにがしたいか?)}
(to be written...)
\tcbline*
\paragraph*{手段(どうやって?)}
(to be written...)
\tcbline*
\paragraph*{背景(なぜ?)}
(to be written...)
\tcbline*
\paragraph*{結論(どうなった?)}
(to be written...)
}
%!TEX root = ./RPA_for_Creating_Program_Note.tex





%%%%%%%%%%%%%%%%%%%%%%%%%%%%%%%%%%%%%%%%%%%%%%%%%%%%%%%%%%
%%            %%%%%%%%%%%%%%%%%%%%%%%%%%%%%%%%%%%%%%%%%%%%
%% chapter    %%%%%%%%%%%%%%%%%%%%%%%%%%%%%%%%%%%%%%%%%%%%
%%            %%%%%%%%%%%%%%%%%%%%%%%%%%%%%%%%%%%%%%%%%%%%
%%%%%%%%%%%%%%%%%%%%%%%%%%%%%%%%%%%%%%%%%%%%%%%%%%%%%%%%%%
\modHeadchapter{作業手順\TBW}
(to be written ...)


%%%%%%%%%%%%%%%%%%%%%%%%%%%%%%%%%%%%%%%%%%%%%%%%%%%%%%%%%%
%% section D.1 %%%%%%%%%%%%%%%%%%%%%%%%%%%%%%%%%%%%%%%%%%%
%%%%%%%%%%%%%%%%%%%%%%%%%%%%%%%%%%%%%%%%%%%%%%%%%%%%%%%%%%
\modHeadsection{\TBW}
(to be written ...)




\begin{appendices}
%%%%%%%%%%%%%%%%%%%%%%%%%%%%%%%%%%%%%%%%%%%%%%%%%%%%%%%%%
%%                %%%%%%%%%%%%%%%%%%%%%%%%%%%%%%%%%%%%%%%
%% Appendix       %%%%%%%%%%%%%%%%%%%%%%%%%%%%%%%%%%%%%%%
%% Work Procedure %%%%%%%%%%%%%%%%%%%%%%%%%%%%%%%%%%%%%%%
%% Start          %%%%%%%%%%%%%%%%%%%%%%%%%%%%%%%%%%%%%%%
%%                %%%%%%%%%%%%%%%%%%%%%%%%%%%%%%%%%%%%%%%
%%%%%%%%%%%%%%%%%%%%%%%%%%%%%%%%%%%%%%%%%%%%%%%%%%%%%%%%%
%\Appendixpart


\end{appendices}
%%%%%%%%%%%%%%%%%%%%%%%%%%%%%%%%%%%%%%%%%%%%%%%%%%%%%%%%%
%%                %%%%%%%%%%%%%%%%%%%%%%%%%%%%%%%%%%%%%%%
%% Appendix       %%%%%%%%%%%%%%%%%%%%%%%%%%%%%%%%%%%%%%%
%% Work Procedure %%%%%%%%%%%%%%%%%%%%%%%%%%%%%%%%%%%%%%%
%% End            %%%%%%%%%%%%%%%%%%%%%%%%%%%%%%%%%%%%%%%
%%                %%%%%%%%%%%%%%%%%%%%%%%%%%%%%%%%%%%%%%%
%%%%%%%%%%%%%%%%%%%%%%%%%%%%%%%%%%%%%%%%%%%%%%%%%%%%%%%%%

\addtocontents{toc}{\protect\end{tcolorbox}}
\addtocontents{toc}{\protect\cleardoublepage}% add page break




%%%%%%%%%%%%%%%%%%%%%%%%%%%%%%%%%%%%%%%%%%%%%%%%%%%%%%%%%
%%               %%%%%%%%%%%%%%%%%%%%%%%%%%%%%%%%%%%%%%%%
%%               %%%%%%%%%%%%%%%%%%%%%%%%%%%%%%%%%%%%%%%%
%% Part RDB      %%%%%%%%%%%%%%%%%%%%%%%%%%%%%%%%%%%%%%%%
%%               %%%%%%%%%%%%%%%%%%%%%%%%%%%%%%%%%%%%%%%%
%%               %%%%%%%%%%%%%%%%%%%%%%%%%%%%%%%%%%%%%%%%
%%%%%%%%%%%%%%%%%%%%%%%%%%%%%%%%%%%%%%%%%%%%%%%%%%%%%%%%%
\addtocontents{toc}{\protect\begin{tcolorbox}[parttocstyle={Contents}{\the\numexpr\value{part}+1\relax}]}
\tPart{RDBSの構築\TBW}{概要}{%
\paragraph*{目標(なにがしたいか?)}
(to be written...)
\tcbline*
\paragraph*{手段(どうやって?)}
(to be written...)
\tcbline*
\paragraph*{背景(なぜ?)}
(to be written...)
\tcbline*
\paragraph*{結論(どうなった?)}
(to be written...)
}
%%!TEX root = ./RPA_for_Creating_Program_Note.tex





%%%%%%%%%%%%%%%%%%%%%%%%%%%%%%%%%%%%%%%%%%%%%%%%%%%%%%%%%%
%%            %%%%%%%%%%%%%%%%%%%%%%%%%%%%%%%%%%%%%%%%%%%%
%% chapter    %%%%%%%%%%%%%%%%%%%%%%%%%%%%%%%%%%%%%%%%%%%%
%%            %%%%%%%%%%%%%%%%%%%%%%%%%%%%%%%%%%%%%%%%%%%%
%%%%%%%%%%%%%%%%%%%%%%%%%%%%%%%%%%%%%%%%%%%%%%%%%%%%%%%%%%
\modHeadchapter{データベースの構造}



%%%%%%%%%%%%%%%%%%%%%%%%%%%%%%%%%%%%%%%%%%%%%%%%%%%%%%%%%%
%% section D.1 %%%%%%%%%%%%%%%%%%%%%%%%%%%%%%%%%%%%%%%%%%%
%%%%%%%%%%%%%%%%%%%%%%%%%%%%%%%%%%%%%%%%%%%%%%%%%%%%%%%%%%
\modHeadsection{列}



%%%%%%%%%%%%%%%%%%%%%%%%%%%%%%%%%%%%%%%%%%%%%%%%%%%%%%%%%%
%% section D.2 %%%%%%%%%%%%%%%%%%%%%%%%%%%%%%%%%%%%%%%%%%%
%%%%%%%%%%%%%%%%%%%%%%%%%%%%%%%%%%%%%%%%%%%%%%%%%%%%%%%%%%
\modHeadsection{}


%%%%%%%%%%%%%%%%%%%%%%%%%%%%%%%%%%%%%%%%%%%%%%%%%%%%%%%%%%
%%           %%%%%%%%%%%%%%%%%%%%%%%%%%%%%%%%%%%%%%%%%%%%%
%% chapter 2 %%%%%%%%%%%%%%%%%%%%%%%%%%%%%%%%%%%%%%%%%%%%%
%%           %%%%%%%%%%%%%%%%%%%%%%%%%%%%%%%%%%%%%%%%%%%%%
%%%%%%%%%%%%%%%%%%%%%%%%%%%%%%%%%%%%%%%%%%%%%%%%%%%%%%%%%%
\modHeadchapter{}



%%%%%%%%%%%%%%%%%%%%%%%%%%%%%%%%%%%%%%%%%%%%%%%%%%%%%%%%%%
%% section 1.1 %%%%%%%%%%%%%%%%%%%%%%%%%%%%%%%%%%%%%%%%%%%
%%%%%%%%%%%%%%%%%%%%%%%%%%%%%%%%%%%%%%%%%%%%%%%%%%%%%%%%%%
\modHeadsection{}



%%%%%%%%%%%%%%%%%%%%%%%%%%%%%%%%%%%%%%%%%%%%%%%%%%%%%%%%%%
%% section 1.2 %%%%%%%%%%%%%%%%%%%%%%%%%%%%%%%%%%%%%%%%%%%
%%%%%%%%%%%%%%%%%%%%%%%%%%%%%%%%%%%%%%%%%%%%%%%%%%%%%%%%%%
\modHeadsection{}




\addtocontents{toc}{\protect\end{tcolorbox}}
%\addtocontents{toc}{\protect\cleardoublepage}% add page break



%%%%%%%%%%%%%%%%%%%%%%%%%%%%%%%%%%%%%%%%%%%%%%%%%%%%%%%%%%
%%            %%%%%%%%%%%%%%%%%%%%%%%%%%%%%%%%%%%%%%%%%%%%
%%            %%%%%%%%%%%%%%%%%%%%%%%%%%%%%%%%%%%%%%%%%%%%
%% BACKMATTER %%%%%%%%%%%%%%%%%%%%%%%%%%%%%%%%%%%%%%%%%%%%
%%            %%%%%%%%%%%%%%%%%%%%%%%%%%%%%%%%%%%%%%%%%%%%
%%            %%%%%%%%%%%%%%%%%%%%%%%%%%%%%%%%%%%%%%%%%%%%
%%%%%%%%%%%%%%%%%%%%%%%%%%%%%%%%%%%%%%%%%%%%%%%%%%%%%%%%%%
\clearpage
\backmatter
\pagestyle{plainheadback}


%\PartSeparateline{toc}
\addtocontents{toc}{\protect\vspace*{10pt}}
%%%%%%%%%%%%%%%%%%%%%%%%%%%%%%%%%%%%%%%%%%%%%%%%%%%%%%%%%%%%%%%%%%%%
%%                 %%%%%%%%%%%%%%%%%%%%%%%%%%%%%%%%%%%%%%%%%%%%%%%%%
%% LIST OF FIGURES %%%%%%%%%%%%%%%%%%%%%%%%%%%%%%%%%%%%%%%%%%%%%%%%%
%%                 %%%%%%%%%%%%%%%%%%%%%%%%%%%%%%%%%%%%%%%%%%%%%%%%%
%%%%%%%%%%%%%%%%%%%%%%%%%%%%%%%%%%%%%%%%%%%%%%%%%%%%%%%%%%%%%%%%%%%%
\phantomsection
\addchaptertocentry{}{\listfigurename}{}%
\listoffigures\thispagestyle{plainheadback}


\clearpage
%%%%%%%%%%%%%%%%%%%%%%%%%%%%%%%%%%%%%%%%%%%%%%%%%%%%%%%%%%%%%%%%%%%%
%%                 %%%%%%%%%%%%%%%%%%%%%%%%%%%%%%%%%%%%%%%%%%%%%%%%%
%% LIST OF ColumnS %%%%%%%%%%%%%%%%%%%%%%%%%%%%%%%%%%%%%%%%%%%%%%%%%
%%                 %%%%%%%%%%%%%%%%%%%%%%%%%%%%%%%%%%%%%%%%%%%%%%%%%
%%%%%%%%%%%%%%%%%%%%%%%%%%%%%%%%%%%%%%%%%%%%%%%%%%%%%%%%%%%%%%%%%%%%
\begingroup
\let\cleardoublepage\relax
\phantomsection
\addchaptertocentry{}{\listofloCname}
\listoftoc{loC}\thispagestyle{plainheadback}
\let\cleardoublepage\tmpcleardoublepage
\endgroup



\clearpage
%%%%%%%%%%%%%%%%%%%%%%%%%%%%%%%%%%%%%%%%%%%%%%%%%%%%%%%%%%%%%%%%%%%%
%%              %%%%%%%%%%%%%%%%%%%%%%%%%%%%%%%%%%%%%%%%%%%%%%%%%%%%
%% BIBLIOGRAPHY %%%%%%%%%%%%%%%%%%%%%%%%%%%%%%%%%%%%%%%%%%%%%%%%%%%%
%%              %%%%%%%%%%%%%%%%%%%%%%%%%%%%%%%%%%%%%%%%%%%%%%%%%%%%
%%%%%%%%%%%%%%%%%%%%%%%%%%%%%%%%%%%%%%%%%%%%%%%%%%%%%%%%%%%%%%%%%%%%
\printbibheading
\begingroup
\let\cleardoublepage\relax  % \cleardoublepageを無効にする
\addchaptertocentry{}{\bibname}
{\footnotesize%
\printbibliography[type=article, heading=articles]%
\printbibliography[type=online, heading=online]%
\thispagestyle{plainheadback}}
\let\cleardoublepage\tmpcleardoublepage
\endgroup




\clearpage
%%%%%%%%%%%%%%%%%%%%%%%%%%%%%%%%%%%%%%%%%%%%%%%%%%%%%%%%%%%%%%%%%%%%
%%       %%%%%%%%%%%%%%%%%%%%%%%%%%%%%%%%%%%%%%%%%%%%%%%%%%%%%%%%%%%
%% index %%%%%%%%%%%%%%%%%%%%%%%%%%%%%%%%%%%%%%%%%%%%%%%%%%%%%%%%%%%
%%       %%%%%%%%%%%%%%%%%%%%%%%%%%%%%%%%%%%%%%%%%%%%%%%%%%%%%%%%%%%
%%%%%%%%%%%%%%%%%%%%%%%%%%%%%%%%%%%%%%%%%%%%%%%%%%%%%%%%%%%%%%%%%%%%
\begingroup
\let\cleardoublepage\relax  % \cleardoublepageを無効にする
\addchaptertocentry{}{\indexname}
{\small\printindex\thispagestyle{plainheadback}}
\let\cleardoublepage\tmpcleardoublepage
\endgroup



\thispagestyle{empty}~\vfill\hfill{\tiny Printed in Japan}

\end{document}