%!TEX root = ./RPA_for_Creating_Program_Note.tex





%%%%%%%%%%%%%%%%%%%%%%%%%%%%%%%%%%%%%%%%%%%%%%%%%%%%%%%%%%
%%            %%%%%%%%%%%%%%%%%%%%%%%%%%%%%%%%%%%%%%%%%%%%
%% chapter    %%%%%%%%%%%%%%%%%%%%%%%%%%%%%%%%%%%%%%%%%%%%
%%            %%%%%%%%%%%%%%%%%%%%%%%%%%%%%%%%%%%%%%%%%%%%
%%%%%%%%%%%%%%%%%%%%%%%%%%%%%%%%%%%%%%%%%%%%%%%%%%%%%%%%%%
\modHeadchapter{数値計算}
基本的に、\index{すうちじょうほう@数値情報}数値情報については数値計算用の言語を用いて行うため、その詳細は別ドキュメントに譲る。
ここではその\index{すうちけいさん@数値計算}数値計算に必要な部分をピックアップする。
なお、ここでは主に\DMname について述べるため、スペーサに関するものは省略する。


%%%%%%%%%%%%%%%%%%%%%%%%%%%%%%%%%%%%%%%%%%%%%%%%%%%%%%%%%%
%% section 9.1 %%%%%%%%%%%%%%%%%%%%%%%%%%%%%%%%%%%%%%%%%%%
%%%%%%%%%%%%%%%%%%%%%%%%%%%%%%%%%%%%%%%%%%%%%%%%%%%%%%%%%%
\modHeadsection{振分け}
テーブルを$-\theta$だけ回転させたときのトップ・ボトム側の\index{ふりわけちょう@振分長}振分長$f'_\mathrm T$, $f'_\mathrm B$は、\pageeqref{eq:saifuriwake}より、
\begin{align*}
  f_\mathrm T' = f_\mathrm T+\varDelta'\!\sin\theta~~, \quad
  f_\mathrm B' = (f_\mathrm T+f_\mathrm B)-f_\mathrm T'\ .
\end{align*}
ジグ(長さ$2l$)からの張出長に換算するならば、それぞれ
\begin{align*}
  f_\mathrm T+\varDelta'\!\sin\theta-l~~, \quad
  (f_\mathrm T+f_\mathrm B)-f_\mathrm T'-l\ .
\end{align*}
トップ側とボトム側の振分長が均等かつ平行になるときの\index{かたむきかく(ふりわけよう)@傾き角(振分け用)}回転角$\theta'$は、\pageeqref{eq:saifuriwakeangle}より、
\begin{align*}
  \sin\theta' = \frac{f_d}{\varDelta'}\ .
\end{align*}
ここで、
\begin{align*}
  \varDelta' = \varDelta+\sqrt{R_\mathrm i'-\bar l^2}\ , \quad
  R_\mathrm i' = R_\mathrm c-\frac{W_x}2-\rho\ ,\quad
  \bar l = l-\frac\sigma2\ ,\quad
  f_d = \frac{f_\mathrm B-f_\mathrm T}2\ .
\end{align*}



\clearpage
%%%%%%%%%%%%%%%%%%%%%%%%%%%%%%%%%%%%%%%%%%%%%%%%%%%%%%%%%%
%% section 9.2 %%%%%%%%%%%%%%%%%%%%%%%%%%%%%%%%%%%%%%%%%%%
%%%%%%%%%%%%%%%%%%%%%%%%%%%%%%%%%%%%%%%%%%%%%%%%%%%%%%%%%%
\modHeadsection{外径・湾曲中心}
\index{ジグ}ジグの底の$Y$座標を$\varDelta_y$とすると、外中心$Y$座標は、
\begin{align*}
  \varDelta_y+\frac{W_y}2\ .
\end{align*}
トップ端およびボトム端における($-\theta$回転後の)\index{そとがわちゅうしん@外側中心}外側中心の$X$位置は、\pageeqref{eq:tableTc}, \pageeqref{eq:tableBc}よりそれぞれ、
\begin{align*}
  \frac{\sqrt{R_\mathrm o^2-f_\mathrm T^2}+\sqrt{R_\mathrm i^2-f_\mathrm T^2}}2-\varDelta'\cos\theta~, \quad
  \varDelta'\cos\theta-\frac{\sqrt{R_\mathrm o^2-f_\mathrm B^2}+\sqrt{R_\mathrm i^2-f_\mathrm B^2}}2\ .
\end{align*}
トップ端およびボトム端における($-\theta$回転後の)\index{わんきょくちゅうしん@湾曲中心}湾曲中心の$X$値は、\pageeqref{eq:tableTRc}, \pageeqref{eq:tableBRc}よりそれぞれ、
\begin{align*}
  \sqrt{R_\mathrm c^2-f_\mathrm T^2}-\varDelta'\!\cos\theta~, \quad
  \varDelta'\!\cos\theta-\sqrt{R_\mathrm c^2-f_\mathrm B^2}~.
\end{align*}
また、(計算上の)\index{うちがわちゅうしん@内側中心}内側中心はこの湾曲中心を持って代用してもよいものとする。



\clearpage
%%%%%%%%%%%%%%%%%%%%%%%%%%%%%%%%%%%%%%%%%%%%%%%%%%%%%%%%%%
%% section 9.4 %%%%%%%%%%%%%%%%%%%%%%%%%%%%%%%%%%%%%%%%%%%
%%%%%%%%%%%%%%%%%%%%%%%%%%%%%%%%%%%%%%%%%%%%%%%%%%%%%%%%%%
\modHeadsection{外削}


%%%%%%%%%%%%%%%%%%%%%%%%%%%%%%%%%%%%%%%%%%%%%%%%%%%%%%%%%%
%% subsection 9.3.1 %%%%%%%%%%%%%%%%%%%%%%%%%%%%%%%%%%%%%%
%%%%%%%%%%%%%%%%%%%%%%%%%%%%%%%%%%%%%%%%%%%%%%%%%%%%%%%%%%
\subsection{外削中心:ボトムA面肉厚基準の場合}
\index{テーブルちゅうしん@テーブル中心}テーブル中心\index{P(てーぶるちゅうしん)@P(テーブル中心)}Pを原点としたボトム側外削径の中心$\mathfrak B_\mathrm c'$の(おおよその)$X$座標は、\pageeqref{eq:gaisakucenterBt}より、
\begin{align*}
  \varDelta'\cos\theta-\frac{\sqrt{R_\mathrm o^2-f_\mathrm B^2}+\sqrt{R_\mathrm i^2-f_\mathrm B^2}}2
  -\frac{w_\mathrm B}2-\tau_\mathrm B+\frac{\mathfrak W_\mathrm B}2\ .
\end{align*}
このとき、計測したA側内面b$_\mathrm o'$の$X$座標が\pageeqref{eq:gaisakucenterBr}となるように、原点$\mathfrak B_\mathrm c'$を定める。
\begin{align*}
  -\left(\frac{\mathfrak W_\mathrm B}2-\tau_\mathrm B+\mu\right).
\end{align*}
トップ側にも外削がある場合、計測で定めた$\mathfrak B_\mathrm c'$の$X$座標$\mathcal G_{\mathrm Bx}$および\index{とおりしん@通り芯}通り芯$T_x$を用いて\pageeqref{eq:BbasedTx}で与えられる。
\begin{align*}
  -\mathcal G_{Bx}+T_x\ .
\end{align*}


%%%%%%%%%%%%%%%%%%%%%%%%%%%%%%%%%%%%%%%%%%%%%%%%%%%%%%%%%%
%% subsection 9.3.2 %%%%%%%%%%%%%%%%%%%%%%%%%%%%%%%%%%%%%%
%%%%%%%%%%%%%%%%%%%%%%%%%%%%%%%%%%%%%%%%%%%%%%%%%%%%%%%%%%
\subsection{外削中心:トップA面肉厚基準の場合}
\index{テーブルちゅうしん@テーブル中心}テーブル中心\index{P(てーぶるちゅうしん)@P(テーブル中心)}Pを原点としたトップ側外削径の中心$\mathfrak T_\mathrm c'$の(おおよその)$X$座標は、\pageeqref{eq:gaisakucenterTt}より、
\begin{align*}
  \frac{\sqrt{R_\mathrm o^2-f_\mathrm T^2}+\sqrt{R_\mathrm i^2-f_\mathrm T^2}}2-\varDelta'\cos\theta
  +\frac{w_\mathrm T}2+\tau_\mathrm T-\frac{\mathfrak W_\mathrm T}2\ .
\end{align*}
このとき、計測したA側内面t$_\mathrm o'$の$X$座標が\pageeqref{eq:gaisakucenterTr}となるように、原点$\mathfrak T_\mathrm c'$を定める。
\begin{align*}
  \frac{\mathfrak W_\mathrm T}2-\tau_\mathrm T+\mu~.
\end{align*}
ボトム側にも外削がある場合、計測で定めた$\mathfrak T_\mathrm c'$の$X$座標$\mathcal G_{\mathrm Tx}$および\index{とおりしん@通り芯}通り芯$T_x$を用いて\pageeqref{eq:TbasedTx}で与えられる。
\begin{align*}
  -\mathcal G_{Tx}+T_x
\end{align*}


%%%%%%%%%%%%%%%%%%%%%%%%%%%%%%%%%%%%%%%%%%%%%%%%%%%%%%%%%%
%% subsection 9.3.2 %%%%%%%%%%%%%%%%%%%%%%%%%%%%%%%%%%%%%%
%%%%%%%%%%%%%%%%%%%%%%%%%%%%%%%%%%%%%%%%%%%%%%%%%%%%%%%%%%
\subsection{外削長}
\index{がいさくちょう@外削長}外削長に関しては、基本的には\index{ふりわけちょう@振分長}振分長から外削長を引いた位置に$Z$座標を合わせればよい。
ただし、トップ側において、外削長$h_\mathrm T$が\index{みぞいち@溝位置}溝位置$\kappa_p$と\index{みぞはば@溝幅}溝幅$\kappa_w$の和に一致する場合は、外削長を$\kappa_p+1$mmとして切削する。
つまり、
\begin{align*}
  h_\mathrm T = \kappa_p+\kappa_w \quad \longrightarrow \quad h_\mathrm T+1[\mathrm{mm}]
\end{align*}


%%%%%%%%%%%%%%%%%%%%%%%%%%%%%%%%%%%%%%%%%%%%%%%%%%%%%%%%%%
%% subsection 9.3.2 %%%%%%%%%%%%%%%%%%%%%%%%%%%%%%%%%%%%%%
%%%%%%%%%%%%%%%%%%%%%%%%%%%%%%%%%%%%%%%%%%%%%%%%%%%%%%%%%%
\subsection{湾曲に沿った外削\TBW}
(to be written...)


\clearpage
%%%%%%%%%%%%%%%%%%%%%%%%%%%%%%%%%%%%%%%%%%%%%%%%%%%%%%%%%%
%% section 9.4 %%%%%%%%%%%%%%%%%%%%%%%%%%%%%%%%%%%%%%%%%%%
%%%%%%%%%%%%%%%%%%%%%%%%%%%%%%%%%%%%%%%%%%%%%%%%%%%%%%%%%%
\modHeadsection{溝}
\index{みぞいち@溝位置}溝位置$\kappa_p$および\index{みぞはば@溝幅}溝幅$\kappa_w$に対し、\index{テーブルちゅうしん@テーブル中心}テーブル中心\index{P(てーぶるちゅうしん)@P(テーブル中心)}Pを原点とした\index{みぞちゅうしん@溝中心}溝の中心の$Z$座標は、\pageeqref{eq:mizocenterZ}より
\begin{align*}
  f_\mathrm T'-\kappa_p-\frac{\kappa_w}2\ .
\end{align*}


%%%%%%%%%%%%%%%%%%%%%%%%%%%%%%%%%%%%%%%%%%%%%%%%%%%%%%%%%%
%% subsection 9.4.1 %%%%%%%%%%%%%%%%%%%%%%%%%%%%%%%%%%%%%%
%%%%%%%%%%%%%%%%%%%%%%%%%%%%%%%%%%%%%%%%%%%%%%%%%%%%%%%%%%
\subsection{湾曲中心が基準の場合}
トップ端における湾曲中心T$_{R_\mathrm c}'$と溝中心M$'$との$X$方向の差は、\pageeqref{eq:difTopMizoCenter}より、
\begin{align*}
  \sqrt{R_\mathrm c^2-\left(f_\mathrm T-\kappa_p-\frac{\kappa_w}2\right)^{\!2}}
  -\sqrt{R_\mathrm c^2-f_\mathrm T^2}\ .
\end{align*}


%%%%%%%%%%%%%%%%%%%%%%%%%%%%%%%%%%%%%%%%%%%%%%%%%%%%%%%%%%
%% subsection 10.4.1 %%%%%%%%%%%%%%%%%%%%%%%%%%%%%%%%%%%%%
%%%%%%%%%%%%%%%%%%%%%%%%%%%%%%%%%%%%%%%%%%%%%%%%%%%%%%%%%%
\subsection{外削中心が基準の場合}
\index{みぞちゅうしん@溝中心}溝の中心は\index{トップがわのがいさくちゅうしん@トップ側の外削中心}トップ側の外削中心とする。


%%%%%%%%%%%%%%%%%%%%%%%%%%%%%%%%%%%%%%%%%%%%%%%%%%%%%%%%%%
%% subsection 10.4.1 %%%%%%%%%%%%%%%%%%%%%%%%%%%%%%%%%%%%%
%%%%%%%%%%%%%%%%%%%%%%%%%%%%%%%%%%%%%%%%%%%%%%%%%%%%%%%%%%
\subsection{A側溝深さ指定の場合}
\paragraph*{外削のない場合}
\index{Aがわみぞふかさ@A側溝深さ}A側溝深さ$\kappa_d$は、その測定値$\kappa_d'$が図面上の値となるように与えられるものとする。このとき\pageeqref{eq:keydepthDif1}より、
\begin{align*}
  \kappa_d
  &= \frac{2\kappa_d'-\kappa_w\sin\zeta}{1+\cos^2\zeta}\cos\zeta
     +\sqrt{R_\mathrm o^2-\left(f_\mathrm T-\kappa_p-\frac{\kappa_w}2\right)^{\!2}}
     -\sqrt{R_\mathrm o^2-\left(f_\mathrm T-\kappa_p\right)^2}\ .
\end{align*}
ここで$\zeta$は\pageeqref{eq:angleZeta}より、
\begin{align*}
  \tan\zeta
  = \frac{\sqrt{R_\mathrm o^2-\left(f_\mathrm T-\kappa_p-\kappa_w\right)^2}
          -\sqrt{R_\mathrm o^2-\left(f_\mathrm T-\kappa_p\right)^2}}
         {\kappa_w}\ .
\end{align*}
A側溝深さ$\kappa_d$に対し、\index{みぞちゅうしん@溝中心}溝中心の位置の$X$座標は\pageeqref{eq:mizocenterA}より、
\begin{align*}
  \sqrt{R_\mathrm o^2-\left(f_\mathrm T-\kappa_p-\frac{\kappa_w}2\right)^{\!2}}-\kappa_d-\frac{W_{mx}}2
  -\varDelta\ .
\end{align*}
また\index{Aがわがいめん@A側外面}A側外面の\index{じっそくち@実測値}実測値を$\mathcal G_m$とすると、\pageeqref{eq:mizocenterAd}より、
\begin{align*}
  \mathcal G_m-\frac{W_{mx}}2-\kappa_d\ .
\end{align*}


\paragraph*{外削のある場合}
\index{Aがわみぞふかさ@A側溝深さ}A側溝深さ$\kappa_d$に対し、トップ外削$X$中心を$\mathcal G_{\mathrm Tx}$とすると、\index{みぞちゅうしん@溝中心}溝中心の位置の$X$座標は\pageeqref{eq:mizocenterAG}より、
\begin{align*}
  \mathcal G_{\mathrm Tx}+\frac{\mathfrak W_x}2-\kappa_d-\frac{W_{mx}}2\ .
\end{align*}



\clearpage
%%%%%%%%%%%%%%%%%%%%%%%%%%%%%%%%%%%%%%%%%%%%%%%%%%%%%%%%%%
%% section 9.2 %%%%%%%%%%%%%%%%%%%%%%%%%%%%%%%%%%%%%%%%%%%
%%%%%%%%%%%%%%%%%%%%%%%%%%%%%%%%%%%%%%%%%%%%%%%%%%%%%%%%%%
\modHeadsection{外側C面取\TBW}
(to be written...)


%\clearpage
%%%%%%%%%%%%%%%%%%%%%%%%%%%%%%%%%%%%%%%%%%%%%%%%%%%%%%%%%%
%% section 9.2 %%%%%%%%%%%%%%%%%%%%%%%%%%%%%%%%%%%%%%%%%%%
%%%%%%%%%%%%%%%%%%%%%%%%%%%%%%%%%%%%%%%%%%%%%%%%%%%%%%%%%%
\modHeadsection{内側C面取\TBW}
(to be written...)


%\clearpage
%%%%%%%%%%%%%%%%%%%%%%%%%%%%%%%%%%%%%%%%%%%%%%%%%%%%%%%%%%
%% section 9.2 %%%%%%%%%%%%%%%%%%%%%%%%%%%%%%%%%%%%%%%%%%%
%%%%%%%%%%%%%%%%%%%%%%%%%%%%%%%%%%%%%%%%%%%%%%%%%%%%%%%%%%
\modHeadsection{座ぐり\TBW}
(to be written...)





%%%%%%%%%%%%%%%%%%%%%%%%%%%%%%%%%%%%%%%%%%%%%%%%%%%%%%%%%%
%%            %%%%%%%%%%%%%%%%%%%%%%%%%%%%%%%%%%%%%%%%%%%%
%% chapter    %%%%%%%%%%%%%%%%%%%%%%%%%%%%%%%%%%%%%%%%%%%%
%%            %%%%%%%%%%%%%%%%%%%%%%%%%%%%%%%%%%%%%%%%%%%%
%%%%%%%%%%%%%%%%%%%%%%%%%%%%%%%%%%%%%%%%%%%%%%%%%%%%%%%%%%
\modHeadchapter{代入する数値\TBW}
(to be written...)



%%%%%%%%%%%%%%%%%%%%%%%%%%%%%%%%%%%%%%%%%%%%%%%%%%%%%%%%%%
%%%%%%%%%%%%%%%%%%%%%%%%%%%%%%%%%%%%%%%%%%%%%%%%%%%%%%%%%%
%%%%%%%%%%%%%%%%%%%%%%%%%%%%%%%%%%%%%%%%%%%%%%%%%%%%%%%%%%
\begin{tcolorbox}[title={2023/07/28時点の\MMname 実測値}, fonttitle=\gtfamily\bfseries]
\begin{align*}
  \text{Bot ($B=0$)}
  \left\{
  \begin{array}{rl}
    X: & 97.790 \sim 99.930\\
    Y: & -823.850\\
    Z: & -634.620
  \end{array}
  \right.\quad
  \text{Top ($B=180.$)}
  \left\{
  \begin{array}{rl}
    X: & -97.980 \sim -99.570\\
    Y: & -823.780\\
    Z: & -634.720
  \end{array}
  \right.
\end{align*}\\
・$X$については、ジグの当たる点の凸部と端部($Z$方向は目分量)\\
・$Y$については、モールドの底が当たる面\\
・$Z$については、$X0$ $Y-850.$における、ジグとの接点\\
※これらの値に、\index{タッチセンサーせんたん@タッチセンサー先端}タッチセンサー先端球の半径を加減する必要がある
\end{tcolorbox}
%%%%%%%%%%%%%%%%%%%%%%%%%%%%%%%%%%%%%%%%%%%%%%%%%%%%%%%%%%
%%%%%%%%%%%%%%%%%%%%%%%%%%%%%%%%%%%%%%%%%%%%%%%%%%%%%%%%%%
%%%%%%%%%%%%%%%%%%%%%%%%%%%%%%%%%%%%%%%%%%%%%%%%%%%%%%%%%%




\begin{appendices}
%%%%%%%%%%%%%%%%%%%%%%%%%%%%%%%%%%%%%%%%%%%%%%%%%%%%%%%%%
%%                %%%%%%%%%%%%%%%%%%%%%%%%%%%%%%%%%%%%%%%
%% Appendix       %%%%%%%%%%%%%%%%%%%%%%%%%%%%%%%%%%%%%%%
%% Numerical Calc %%%%%%%%%%%%%%%%%%%%%%%%%%%%%%%%%%%%%%%
%% Start          %%%%%%%%%%%%%%%%%%%%%%%%%%%%%%%%%%%%%%%
%%                %%%%%%%%%%%%%%%%%%%%%%%%%%%%%%%%%%%%%%%
%%%%%%%%%%%%%%%%%%%%%%%%%%%%%%%%%%%%%%%%%%%%%%%%%%%%%%%%%
\Appendixpart




%%%%%%%%%%%%%%%%%%%%%%%%%%%%%%%%%%%%%%%%%%%%%%%%%%%%%%%%%%
%%            %%%%%%%%%%%%%%%%%%%%%%%%%%%%%%%%%%%%%%%%%%%%
%% Appendix F %%%%%%%%%%%%%%%%%%%%%%%%%%%%%%%%%%%%%%%%%%%%
%%            %%%%%%%%%%%%%%%%%%%%%%%%%%%%%%%%%%%%%%%%%%%%
%%%%%%%%%%%%%%%%%%%%%%%%%%%%%%%%%%%%%%%%%%%%%%%%%%%%%%%%%%
\modHeadchapter{作成したNCプログラム}
\addcontentsline{lol}{chapter}{\thechapter. \Chaptername}
%!TEX root = ../RPA_for_Creating_Program_Note.tex
\setcounter{lstlisting}{0}

ここでは具体的に作成したプログラムを記載する。


%%%%%%%%%%%%%%%%%%%%%%%%%%%%%%%%%%%%%%%%%%%%%%%%%%%%%%%%%%
%% section G.1 %%%%%%%%%%%%%%%%%%%%%%%%%%%%%%%%%%%%%%%%%%%
%%%%%%%%%%%%%%%%%%%%%%%%%%%%%%%%%%%%%%%%%%%%%%%%%%%%%%%%%%
\modHeadsection{メインプログラムの例}
\addtocontents{lol}{\protect\addvspace{3pt}}{}{}
\addcontentsline{lol}{section}{\numberline{\thesection}\Sectionname}
\index{メインプログラム}メインプログラムについては、個々の明細の情報(社内機密情報)を含む。
そのため、\pageautoref{subsec:notopenwork}に則り、記載は代表的・典型的なものに留める。\\

%%%%%%%%%%%%%%%%%%%%%%%%%%%%%%%%%%%%%%%%%%%%%%%%%%%%%%%%%%
%% Prg. \MXOThickness %%%%%%%%%%%%%%%%%%%%%%%%%%%%%%%%%%%%
%%%%%%%%%%%%%%%%%%%%%%%%%%%%%%%%%%%%%%%%%%%%%%%%%%%%%%%%%%
\modcaptionof{lstlisting}{\MainEx}
\lstinputlisting[style=Gcode-more]{../Mould_Machining_Programs/main_programs/\MainEx}


\clearpage
%%%%%%%%%%%%%%%%%%%%%%%%%%%%%%%%%%%%%%%%%%%%%%%%%%%%%%%%%%
%% section E.2 %%%%%%%%%%%%%%%%%%%%%%%%%%%%%%%%%%%%%%%%%%%
%%%%%%%%%%%%%%%%%%%%%%%%%%%%%%%%%%%%%%%%%%%%%%%%%%%%%%%%%%
\modHeadsection{測定用(\dimple 除く)サブプログラム}
\addtocontents{lol}{\protect\addvspace{3pt}}{}{}
\addcontentsline{lol}{section}{\numberline{\thesection}\Sectionname}

%%%%%%%%%%%%%%%%%%%%%%%%%%%%%%%%%%%%%%%%%%%%%%%%%%%%%%%%%%
%% Prg. \MXOThickness %%%%%%%%%%%%%%%%%%%%%%%%%%%%%%%%%%%%
%%%%%%%%%%%%%%%%%%%%%%%%%%%%%%%%%%%%%%%%%%%%%%%%%%%%%%%%%%
\modcaptionof{lstlisting}{\MXOThickness\,:測定 外側中心\texorpdfstring{$X$}{X}}
\lstinputlisting[style=Gcode-more]{../Mould_Machining_Programs/sub_programs/\MXOThickness}

\clearpage
%%%%%%%%%%%%%%%%%%%%%%%%%%%%%%%%%%%%%%%%%%%%%%%%%%%%%%%%%%
%% Prg. \MYOThickness %%%%%%%%%%%%%%%%%%%%%%%%%%%%%%%%%%%%
%%%%%%%%%%%%%%%%%%%%%%%%%%%%%%%%%%%%%%%%%%%%%%%%%%%%%%%%%%
\modcaptionof{lstlisting}{\MYOThickness\,:測定 外側中心\texorpdfstring{$Y$}{Y}}
\lstinputlisting[style=Gcode-more]{../Mould_Machining_Programs/sub_programs/\MYOThickness}


\clearpage
%%%%%%%%%%%%%%%%%%%%%%%%%%%%%%%%%%%%%%%%%%%%%%%%%%%%%%%%%%
%% Prg. \MXIWidth %%%%%%%%%%%%%%%%%%%%%%%%%%%%%%%%%%%%%%%%
%%%%%%%%%%%%%%%%%%%%%%%%%%%%%%%%%%%%%%%%%%%%%%%%%%%%%%%%%%
\modcaptionof{lstlisting}{\MXIWidth\,:測定 内側中心\texorpdfstring{$X$}{X}}
\lstinputlisting[style=Gcode-more]{../Mould_Machining_Programs/sub_programs/\MXIWidth}


\clearpage
%%%%%%%%%%%%%%%%%%%%%%%%%%%%%%%%%%%%%%%%%%%%%%%%%%%%%%%%%%
%% Prg. \MYIWidth %%%%%%%%%%%%%%%%%%%%%%%%%%%%%%%%%%%%%%%%
%%%%%%%%%%%%%%%%%%%%%%%%%%%%%%%%%%%%%%%%%%%%%%%%%%%%%%%%%%
\modcaptionof{lstlisting}{\MYIWidth\,:測定 内側中心\texorpdfstring{$Y$}{Y}}
\lstinputlisting[style=Gcode-more]{../Mould_Machining_Programs/sub_programs/\MYIWidth}


\clearpage
%%%%%%%%%%%%%%%%%%%%%%%%%%%%%%%%%%%%%%%%%%%%%%%%%%%%%%%%%%
%% Prg. \MXface %%%%%%%%%%%%%%%%%%%%%%%%%%%%%%%%%%%%%%%%%%
%%%%%%%%%%%%%%%%%%%%%%%%%%%%%%%%%%%%%%%%%%%%%%%%%%%%%%%%%%
\modcaptionof{lstlisting}{\MXIface\,:測定 外削中心 \texorpdfstring{$X$}{X} C面方向}
\lstinputlisting[style=Gcode-more]{../Mould_Machining_Programs/sub_programs/\MXIface}


\clearpage
%%%%%%%%%%%%%%%%%%%%%%%%%%%%%%%%%%%%%%%%%%%%%%%%%%%%%%%%%%
%% Prg. \MYcenterline %%%%%%%%%%%%%%%%%%%%%%%%%%%%%%%%%%%%
%%%%%%%%%%%%%%%%%%%%%%%%%%%%%%%%%%%%%%%%%%%%%%%%%%%%%%%%%%
\modcaptionof{lstlisting}{\MYcenterline\,:測定 通り芯\texorpdfstring{$Y$}{Y}}
\lstinputlisting[style=Gcode-more]{../Mould_Machining_Programs/sub_programs/\MYcenterline}


\clearpage
%%%%%%%%%%%%%%%%%%%%%%%%%%%%%%%%%%%%%%%%%%%%%%%%%%%%%%%%%%
%% Prg. \MXcenterline  %%%%%%%%%%%%%%%%%%%%%%%%%%%%%%%%%%%
%%%%%%%%%%%%%%%%%%%%%%%%%%%%%%%%%%%%%%%%%%%%%%%%%%%%%%%%%%
\modcaptionof{lstlisting}{\MXcenterline\,:測定 通り芯\texorpdfstring{$X$}{X}}
\lstinputlisting[style=Gcode-more]{../Mould_Machining_Programs/sub_programs/\MXcenterline}



\clearpage
%%%%%%%%%%%%%%%%%%%%%%%%%%%%%%%%%%%%%%%%%%%%%%%%%%%%%%%%%%
%% section E.3 %%%%%%%%%%%%%%%%%%%%%%%%%%%%%%%%%%%%%%%%%%%
%%%%%%%%%%%%%%%%%%%%%%%%%%%%%%%%%%%%%%%%%%%%%%%%%%%%%%%%%%
\modHeadsection{加工用(\dimple 除く)サブプログラム}
\addtocontents{lol}{\protect\addvspace{10pt}}{}{}
\addcontentsline{lol}{section}{\numberline{\thesection}\Sectionname}


%%%%%%%%%%%%%%%%%%%%%%%%%%%%%%%%%%%%%%%%%%%%%%%%%%%%%%%%%%
%% Prg. \KTanmenRight %%%%%%%%%%%%%%%%%%%%%%%%%%%%%%%%%%%%
%%%%%%%%%%%%%%%%%%%%%%%%%%%%%%%%%%%%%%%%%%%%%%%%%%%%%%%%%%
\modcaptionof{lstlisting}{\KTanmenRight\,:加工 端面用 コーナーR 右回り1周}
\lstinputlisting[style=Gcode-more]{../Mould_Machining_Programs/sub_programs/\KTanmenRight}


\clearpage
%%%%%%%%%%%%%%%%%%%%%%%%%%%%%%%%%%%%%%%%%%%%%%%%%%%%%%%%%%
%% Prg. \KGaisakuRLeft %%%%%%%%%%%%%%%%%%%%%%%%%%%%%%%%%%%
%%%%%%%%%%%%%%%%%%%%%%%%%%%%%%%%%%%%%%%%%%%%%%%%%%%%%%%%%%
\modcaptionof{lstlisting}{\KGaisakuRLeft\,:加工 外削用 コーナーR 左回り1周}
\lstinputlisting[style=Gcode-more]{../Mould_Machining_Programs/sub_programs/\KGaisakuRLeft}


\clearpage
%%%%%%%%%%%%%%%%%%%%%%%%%%%%%%%%%%%%%%%%%%%%%%%%%%%%%%%%%%
%% Prg. \KMizoConerLeft %%%%%%%%%%%%%%%%%%%%%%%%%%%%%%%%%%
%%%%%%%%%%%%%%%%%%%%%%%%%%%%%%%%%%%%%%%%%%%%%%%%%%%%%%%%%%
\modcaptionof{lstlisting}{\KMizoConerLeft\,:加工 溝用 左回り1周}
\lstinputlisting[style=Gcode-more]{../Mould_Machining_Programs/sub_programs/\KMizoConerLeft}


\clearpage
%%%%%%%%%%%%%%%%%%%%%%%%%%%%%%%%%%%%%%%%%%%%%%%%%%%%%%%%%%
%% Prg. \KSotoMentoriRLeft %%%%%%%%%%%%%%%%%%%%%%%%%%%%%%%
%%%%%%%%%%%%%%%%%%%%%%%%%%%%%%%%%%%%%%%%%%%%%%%%%%%%%%%%%%
\modcaptionof{lstlisting}{\KSotoMentoriRLeft\,:加工 外面取用 コーナーR 左回り1周}
\lstinputlisting[style=Gcode-more]{../Mould_Machining_Programs/sub_programs/\KSotoMentoriRLeft}


\clearpage
%%%%%%%%%%%%%%%%%%%%%%%%%%%%%%%%%%%%%%%%%%%%%%%%%%%%%%%%%%
%% Prg. \KUchiMentoriRLeft %%%%%%%%%%%%%%%%%%%%%%%%%%%%%%%
%%%%%%%%%%%%%%%%%%%%%%%%%%%%%%%%%%%%%%%%%%%%%%%%%%%%%%%%%%
\modcaptionof{lstlisting}{\KUchiMentoriRLeft\,:加工 内面取用 コーナーR 左回り1周}
\lstinputlisting[style=Gcode-more]{../Mould_Machining_Programs/sub_programs/\KUchiMentoriRLeft}


\clearpage
%%%%%%%%%%%%%%%%%%%%%%%%%%%%%%%%%%%%%%%%%%%%%%%%%%%%%%%%%%
%% Prg. \KUchiMentoriRLeft %%%%%%%%%%%%%%%%%%%%%%%%%%%%%%%
%%%%%%%%%%%%%%%%%%%%%%%%%%%%%%%%%%%%%%%%%%%%%%%%%%%%%%%%%%
\modcaptionof{lstlisting}{\KOLeftAR\,:外側 左回り1周 右上始まり}
\lstinputlisting[style=Gcode-more]{../Mould_Machining_Programs/sub_programs/\KOLeftAR}


\clearpage
%%%%%%%%%%%%%%%%%%%%%%%%%%%%%%%%%%%%%%%%%%%%%%%%%%%%%%%%%%
%% Prg. \KUchiMentoriRLeft %%%%%%%%%%%%%%%%%%%%%%%%%%%%%%%
%%%%%%%%%%%%%%%%%%%%%%%%%%%%%%%%%%%%%%%%%%%%%%%%%%%%%%%%%%
\modcaptionof{lstlisting}{\KILeftAC\,:内側 左回り1周 中央上始まり \texttt{G41}工具径補正あり}
\lstinputlisting[style=Gcode-more]{../Mould_Machining_Programs/sub_programs/\KILeftAC}


\clearpage
%%%%%%%%%%%%%%%%%%%%%%%%%%%%%%%%%%%%%%%%%%%%%%%%%%%%%%%%%%
%% section E.4 %%%%%%%%%%%%%%%%%%%%%%%%%%%%%%%%%%%%%%%%%%%
%%%%%%%%%%%%%%%%%%%%%%%%%%%%%%%%%%%%%%%%%%%%%%%%%%%%%%%%%%
\modHeadsection{\dimple 用 移動・測定・加工用サブプログラム}
\addtocontents{lol}{\protect\addvspace{10pt}}{}{}
\addcontentsline{lol}{section}{\numberline{\thesection}\Sectionname}


%%%%%%%%%%%%%%%%%%%%%%%%%%%%%%%%%%%%%%%%%%%%%%%%%%%%%%%%%%
%% Prg. \DLone %%%%%%%%%%%%%%%%%%%%%%%%%%%%%%%%%%%%%%%%%%%
%%%%%%%%%%%%%%%%%%%%%%%%%%%%%%%%%%%%%%%%%%%%%%%%%%%%%%%%%%
\modcaptionof{lstlisting}{\DLone\,:\dimple レベル1:移動 各列の中心上}
\lstinputlisting[style=Gcode-more]{../Mould_Machining_Programs/sub_programs/\DLone}


\clearpage
%%%%%%%%%%%%%%%%%%%%%%%%%%%%%%%%%%%%%%%%%%%%%%%%%%%%%%%%%%
%% Prg. \DLtwoAC %%%%%%%%%%%%%%%%%%%%%%%%%%%%%%%%%%%%%%%%%
%%%%%%%%%%%%%%%%%%%%%%%%%%%%%%%%%%%%%%%%%%%%%%%%%%%%%%%%%%
\modcaptionof{lstlisting}{\DLtwoAC\,:\dimple レベル2:移動 AC面 列内の各\dimple 上}
\lstinputlisting[style=Gcode-more]{../Mould_Machining_Programs/sub_programs/\DLtwoAC}


\clearpage
%%%%%%%%%%%%%%%%%%%%%%%%%%%%%%%%%%%%%%%%%%%%%%%%%%%%%%%%%%
%% Prg. \DLtwoBD %%%%%%%%%%%%%%%%%%%%%%%%%%%%%%%%%%%%%%%%%
%%%%%%%%%%%%%%%%%%%%%%%%%%%%%%%%%%%%%%%%%%%%%%%%%%%%%%%%%%
\modcaptionof{lstlisting}{\DLtwoBD\,:\dimple レベル2:移動 BC面 列内の各\dimple 上}
\lstinputlisting[style=Gcode-more]{../Mould_Machining_Programs/sub_programs/\DLtwoBD}


\clearpage
%%%%%%%%%%%%%%%%%%%%%%%%%%%%%%%%%%%%%%%%%%%%%%%%%%%%%%%%%%
%% Prg. \DMLthreeAC %%%%%%%%%%%%%%%%%%%%%%%%%%%%%%%%%%%%%%
%%%%%%%%%%%%%%%%%%%%%%%%%%%%%%%%%%%%%%%%%%%%%%%%%%%%%%%%%%
\modcaptionof{lstlisting}{\DMLthreeAC\,:\dimple レベル3:測定 AC内表面\texorpdfstring{$X$}{X}}
\lstinputlisting[style=Gcode-more]{../Mould_Machining_Programs/sub_programs/\DMLthreeAC}


\clearpage
%%%%%%%%%%%%%%%%%%%%%%%%%%%%%%%%%%%%%%%%%%%%%%%%%%%%%%%%%%
%% Prg. \DMLthreeBD %%%%%%%%%%%%%%%%%%%%%%%%%%%%%%%%%%%%%%
%%%%%%%%%%%%%%%%%%%%%%%%%%%%%%%%%%%%%%%%%%%%%%%%%%%%%%%%%%
\modcaptionof{lstlisting}{\DMLthreeBD\,:\dimple レベル3:測定 BD内表面\texorpdfstring{$Y$}{Y}}
\lstinputlisting[style=Gcode-more]{../Mould_Machining_Programs/sub_programs/\DMLthreeBD}


\clearpage
%%%%%%%%%%%%%%%%%%%%%%%%%%%%%%%%%%%%%%%%%%%%%%%%%%%%%%%%%%
%% Prg. \DKLthreeAC %%%%%%%%%%%%%%%%%%%%%%%%%%%%%%%%%%%%%%
%%%%%%%%%%%%%%%%%%%%%%%%%%%%%%%%%%%%%%%%%%%%%%%%%%%%%%%%%%
\modcaptionof{lstlisting}{\DKLthreeAC\,:\dimple レベル3:加工 AC内表面\texorpdfstring{$X$}{X}}
\lstinputlisting[style=Gcode-more]{../Mould_Machining_Programs/sub_programs/\DKLthreeAC}


\clearpage
%%%%%%%%%%%%%%%%%%%%%%%%%%%%%%%%%%%%%%%%%%%%%%%%%%%%%%%%%%
%% Prg. \DKLthreeBD %%%%%%%%%%%%%%%%%%%%%%%%%%%%%%%%%%%%%%
%%%%%%%%%%%%%%%%%%%%%%%%%%%%%%%%%%%%%%%%%%%%%%%%%%%%%%%%%%
\modcaptionof{lstlisting}{\DKLthreeBD\,:\dimple レベル3:加工 BD内表面\texorpdfstring{$Y$}{Y}}
\lstinputlisting[style=Gcode-more]{../Mould_Machining_Programs/sub_programs/\DKLthreeBD}



\clearpage
%%%%%%%%%%%%%%%%%%%%%%%%%%%%%%%%%%%%%%%%%%%%%%%%%%%%%%%%%%
%% section E.5 %%%%%%%%%%%%%%%%%%%%%%%%%%%%%%%%%%%%%%%%%%%
%%%%%%%%%%%%%%%%%%%%%%%%%%%%%%%%%%%%%%%%%%%%%%%%%%%%%%%%%%
\modHeadsection{その他のサブプログラム}
\addtocontents{lol}{\protect\addvspace{10pt}}{}{}
\addcontentsline{lol}{section}{\numberline{\thesection}\Sectionname}


%%%%%%%%%%%%%%%%%%%%%%%%%%%%%%%%%%%%%%%%%%%%%%%%%%%%%%%%%%
%% Prg. \OpauseCheck %%%%%%%%%%%%%%%%%%%%%%%%%%%%%%%%%%%%%
%%%%%%%%%%%%%%%%%%%%%%%%%%%%%%%%%%%%%%%%%%%%%%%%%%%%%%%%%%
\modcaptionof{lstlisting}{\OpauseCheck\,:加工後 確認用:\texorpdfstring{$\boldsymbol{90^\circ}$}{90°}回転 扉前一時停止}
\lstinputlisting[style=Gcode-more]{../Mould_Machining_Programs/sub_programs/\OpauseCheck}


\clearpage
%%%%%%%%%%%%%%%%%%%%%%%%%%%%%%%%%%%%%%%%%%%%%%%%%%%%%%%%%%
%% Prg. \OsensorOn %%%%%%%%%%%%%%%%%%%%%%%%%%%%%%%%%%%%%%%
%%%%%%%%%%%%%%%%%%%%%%%%%%%%%%%%%%%%%%%%%%%%%%%%%%%%%%%%%%
\modcaptionof{lstlisting}{\OsensorOn\,:タッチセンサー電源ON}
\lstinputlisting[style=Gcode-more]{../Mould_Machining_Programs/sub_programs/\OsensorOn}


\clearpage
%%%%%%%%%%%%%%%%%%%%%%%%%%%%%%%%%%%%%%%%%%%%%%%%%%%%%%%%%%
%% Prg. \OsensorOff %%%%%%%%%%%%%%%%%%%%%%%%%%%%%%%%%%%%%%
%%%%%%%%%%%%%%%%%%%%%%%%%%%%%%%%%%%%%%%%%%%%%%%%%%%%%%%%%%
\modcaptionof{lstlisting}{\OsensorOff\,:タッチセンサー電源OFF}
\lstinputlisting[style=Gcode-more]{../Mould_Machining_Programs/sub_programs/\OsensorOff}





%%%%%%%%%%%%%%%%%%%%%%%%%%%%%%%%%%%%%%%%%%%%%%%%%%%%%%%%%%
%%            %%%%%%%%%%%%%%%%%%%%%%%%%%%%%%%%%%%%%%%%%%%%
%% Appendix G %%%%%%%%%%%%%%%%%%%%%%%%%%%%%%%%%%%%%%%%%%%%
%%            %%%%%%%%%%%%%%%%%%%%%%%%%%%%%%%%%%%%%%%%%%%%
%%%%%%%%%%%%%%%%%%%%%%%%%%%%%%%%%%%%%%%%%%%%%%%%%%%%%%%%%%
\modHeadchapter{バンドルのNCプログラム\TBW}
\addcontentsline{lol}{chapter}{\thechapter. \Chaptername}
\addcontentsline{loC}{chapter}{\thechapter. \Chaptername}
%!TEX root = ../RPA_for_Creating_Program_Note.tex
\setcounter{lstlisting}{0}

ここでは機械設置時付属のG-codeプログラムについて記載する。
これらのプログラムは、当社または当社の従業員による著作物ではないため、\pageautoref{subsec:copyrightsSubcontractor}に則り、その詳細の記載は割愛する。



%%%%%%%%%%%%%%%%%%%%%%%%%%%%%%%%%%%%%%%%%%%%%%%%%%%%%%%%%%
%% section H.1 %%%%%%%%%%%%%%%%%%%%%%%%%%%%%%%%%%%%%%%%%%%
%%%%%%%%%%%%%%%%%%%%%%%%%%%%%%%%%%%%%%%%%%%%%%%%%%%%%%%%%%
\modHeadsection{バンドルのプログラム 一覧\TBW}
バンドルのプログラムは以下のとおりである。\\

\begin{3columnstable}{O7000-O7313\TBW}{番号}{内容\hspace*{0.6\textwidth}~}{使用prg}{Sc}{Sl}{Sc}
O7000 & パラメータの設定 &\\\hline
O7100 & 工具長補正値 自動計測 & O9100\\\hline
O7101 & 工具長オフセット量の修正 & O9101\\\hline
O7102 & 工具破損検出 & O9102\\\hline
O7103 & 自動工具径測定 & O9103\\\hline
O7300 & \ttNum502, \ttNum503の測定 & O9300\\\hline
O7301 &  & O9301\\\hline
O7302 &  & O9302\\\hline
O7303 &  & O9303\\\hline
O7310 &  & O9310\\\hline
O7311 &  & O9311\\\hline
O7312 &  & O9312\\\hline
O7313 & \texorpdfstring{$Z$}{Z}軸座標設定 & O9313
\end{3columnstable}

\begin{2columnstable}{O8123}{番号}{内容\hspace*{0.72\textwidth}~}{Sc}
O8123 & 初期加工原点設定 G54 G55 G56 G57
\end{2columnstable}

\clearpage
\begin{3columnstable}{O9001-O9393\TBW}{番号}{内容\hspace*{0.5\textwidth}~}{使用prg}{Sc}{Sl}{Sl}
O9001 & 自動パレット交換 &\\\hline
O9002 & No.1パレット & O9001\\\hline
O9003 & No.2パレット & O9001\\\hline
O9006 & 自動工具交換 &\\\hline
O9020 & パレット認識マクロ &\\\hline
O9021 & No.1パレットの選択確認 &\\\hline
O9022 & No.2パレットの選択確認 &\\\hline
O9100 & 工具長補正値 自動測定 &\\\hline
O9101 & 工具長オフセット量の修正 &\\\hline
O9102 & 工具破損検出 &\\\hline
O9103 & 自動工具径測定 &\\\hline
O9200 & パラメタ確認 &\\\hline
O9300 & プローブの倒れの補正量の測定 & O9392, O9393\\\hline
O9301 & 自動\texorpdfstring{$XYZ$}{XYZ}端面芯出し & O9390, O9391, O9392, O9393\\\hline
O9302 & 自動内円芯出し & O9392, O9393\\\hline
O9303 & 自動外円芯出し & O9392, O9393\\\hline
O9310 & 手動\texorpdfstring{$XY$}{XY}端面芯出し & O9392, O9393\\\hline
O9311 & 手動内円芯出し & O9392, O9393\\\hline
O9312 & 手動外円芯出し & O9392, O9393\\\hline
O9313 & \texorpdfstring{$Z$}{Z}座標系設定 & O9392, O9393\\\hline
O9390 &  & O9392\\\hline
O9391 &  & O9392\\\hline
O9392 &  &\\\hline
O9393 &  &
\end{3columnstable}



\clearpage
%%%%%%%%%%%%%%%%%%%%%%%%%%%%%%%%%%%%%%%%%%%%%%%%%%%%%%%%%%
%% section H.2 %%%%%%%%%%%%%%%%%%%%%%%%%%%%%%%%%%%%%%%%%%%
%%%%%%%%%%%%%%%%%%%%%%%%%%%%%%%%%%%%%%%%%%%%%%%%%%%%%%%%%%
\modHeadsection{使用変数\TBW}

\paragraph*{O7000}
コモン変数(LHS):
\ttNum500, \ttNum501, \ttNum504, \ttNum505, \ttNum506, \ttNum507, \ttNum512, \ttNum513, \ttNum514, \ttNum516, \ttNum517, \ttNum518, \ttNum519, \ttNum523, \ttNum524

\paragraph*{O9001}
システム変数:
\ttNum1001, \ttNum1002, \ttNum1003, \ttNum1010, \ttNum1011, \ttNum1012, \ttNum1013, \ttNum1014, \ttNum1015, \ttNum3003, \ttNum4003

\paragraph*{O9002, O9003}
システム変数: \ttNum1001

\paragraph*{O9006}
コモン変数(LHS): \ttNum137, \ttNum138, \ttNum139 \quad
システム変数: \ttNum4003, \ttNum4107, \ttNum4111

\paragraph*{O9020}
コモン変数(LHS): \ttNum149 \quad
コモン変数(RHS): \ttNum101, \ttNum102 \quad
システム変数: \ttNum1001

\paragraph*{O9021, O9022}
システム変数: \ttNum1001

%%%%%%%%%%%%%%%%%%%%%%%%%%%%%%%%%%%%%%%%%%%%%%%%%%%%%%%%%%
%% subsection E.2.3 %%%%%%%%%%%%%%%%%%%%%%%%%%%%%%%%%%%%%%
%%%%%%%%%%%%%%%%%%%%%%%%%%%%%%%%%%%%%%%%%%%%%%%%%%%%%%%%%%
\subsection{O9100-O9103: 自動工具長測定・工具破損検出}
\begin{hosoku}\small
コモン変数(RHS): \ttNum504, \ttNum505, \ttNum507, \ttNum513, \ttNum514, \ttNum516, \ttNum517\\
システム変数(RHS): \ttNum4001, \ttNum4003, \ttNum4012, \ttNum4130, \ttNum5021, \ttNum5022, \ttNum5043, \ttNum5063, \ttNum5203\\
システム変数(LHS): \ttNum100xx, \ttNum110xx
\end{hosoku}

\begin{hosoku}\small
コモン変数(RHS): \ttNum504, \ttNum506, \ttNum513, \ttNum514, \ttNum516, \ttNum517, \ttNum518\\
システム変数(RHS): \ttNum4001, \ttNum4003, \ttNum4111, \ttNum5021, \ttNum5022, \ttNum5023, \ttNum5043, \ttNum5063, \ttNum100xx, \ttNum110xx\\
システム変数(LHS): \ttNum100xx
\end{hosoku}

\begin{hosoku}\small
コモン変数(RHS): \ttNum504, \ttNum506, \ttNum507, \ttNum513, \ttNum514, \ttNum516, \ttNum517, \ttNum518\\
システム変数: \ttNum4001, \ttNum4003, \ttNum4111, \ttNum5021, \ttNum5022, \ttNum5023, \ttNum5043, \ttNum5063, \ttNum100xx, \ttNum110xx
\end{hosoku}

\begin{hosoku}\small
コモン変数(RHS): \ttNum516, \ttNum517, \ttNum519, \ttNum523, \ttNum524\\
システム変数(RHS): \ttNum4001, \ttNum4003, \ttNum5021, \ttNum5022, \ttNum5041, \ttNum5042, \ttNum5043, \ttNum5061, \ttNum5062\\
システム変数(LHS): \ttNum160xx, \ttNum170xx
\end{hosoku}

%%%%%%%%%%%%%%%%%%%%%%%%%%%%%%%%%%%%%%%%%%%%%%%%%%%%%%%%%%
%% subsection E.2.3 %%%%%%%%%%%%%%%%%%%%%%%%%%%%%%%%%%%%%%
%%%%%%%%%%%%%%%%%%%%%%%%%%%%%%%%%%%%%%%%%%%%%%%%%%%%%%%%%%
\subsection{O9200: パラメタ確認}
\begin{hosoku}\small
コモン変数(LHS): \ttNum120, \ttNum121, \ttNum122, \ttNum123, \ttNum124, \ttNum125, \ttNum126, \ttNum130, \ttNum131, \ttNum132, \ttNum133\\
コモン変数(RHS): \ttNum120, \ttNum121, \ttNum122, \ttNum123, \ttNum124, \ttNum125, \ttNum126\\
システム変数(LHS): \ttNum3006, \ttNum100000, \ttNum100002\\
システム変数(RHS): \ttNum100010
\end{hosoku}

%%%%%%%%%%%%%%%%%%%%%%%%%%%%%%%%%%%%%%%%%%%%%%%%%%%%%%%%%%
%% subsection E.2.3 %%%%%%%%%%%%%%%%%%%%%%%%%%%%%%%%%%%%%%
%%%%%%%%%%%%%%%%%%%%%%%%%%%%%%%%%%%%%%%%%%%%%%%%%%%%%%%%%%
\subsection{O9300-O9313: タッチセンサー自動芯出し}
\begin{hosoku}\small
コモン変数(LHS): \ttNum502, \ttNum503\\
コモン変数(RHS): \ttNum500, \ttNum501, \ttNum512, \ttNum514\\
システム変数(LHS): \ttNum1004, \ttNum1005, \ttNum4003\\
システム変数(RHS): \ttNum4001, \ttNum4003, \ttNum5001, \ttNum5002, \ttNum5003, \ttNum5061, \ttNum5062
\end{hosoku}

\begin{hosoku}\small
コモン変数(LHS): \\
コモン変数(RHS): \ttNum500, \ttNum501, \ttNum502, \ttNum512, \ttNum514\\
システム変数(LHS): \ttNum1004, \ttNum1005, \ttNum4012, \ttNum4130\\
システム変数(RHS): \ttNum4001, \ttNum4003, \ttNum5001, \ttNum5003, \ttNum5061
\end{hosoku}

\begin{hosoku}\small
コモン変数(RHS): \ttNum500, \ttNum501, \ttNum502, \ttNum503, \ttNum512, \ttNum514\\
システム変数(LHS): \ttNum1004, \ttNum1005\\
システム変数(RHS): \ttNum4001, \ttNum4003, \ttNum4012, \ttNum4130, \ttNum5001, \ttNum5002, \ttNum5003, \ttNum5061, \ttNum5062
\end{hosoku}

\begin{hosoku}\small
コモン変数(RHS): \ttNum500, \ttNum501, \ttNum502, \ttNum503, \ttNum514\\
システム変数: \ttNum1004, \ttNum1005, \ttNum4001, \ttNum4003, \ttNum4012, \ttNum4130, \ttNum5001, \ttNum5002, \ttNum5003, \ttNum5061, \ttNum5062
\end{hosoku}

\begin{hosoku}\small
コモン変数(RHS): \ttNum501, \ttNum502, \ttNum503, \ttNum514\\
システム変数: \ttNum1004, \ttNum1005, \ttNum4001, \ttNum4003, \ttNum4012, \ttNum4130, \ttNum5003, \ttNum5061, \ttNum5062
\end{hosoku}

\begin{hosoku}\small
コモン変数(RHS): \ttNum501, \ttNum502, \ttNum503, \ttNum514\\
システム変数: \ttNum1004, \ttNum1005, \ttNum4001, \ttNum4003, \ttNum4012, \ttNum4130, \ttNum5003, \ttNum5061, \ttNum5062
\end{hosoku}

%%%%%%%%%%%%%%%%%%%%%%%%%%%%%%%%%%%%%%%%%%%%%%%%%%%%%%%%%%
%% subsection E.2.3 %%%%%%%%%%%%%%%%%%%%%%%%%%%%%%%%%%%%%%
%%%%%%%%%%%%%%%%%%%%%%%%%%%%%%%%%%%%%%%%%%%%%%%%%%%%%%%%%%
\subsection{O9390-O9393}
\begin{hosoku}\small
コモン変数(RHS): \ttNum500, \ttNum501, \ttNum514\\
システム変数: \ttNum1004, \ttNum4003, \ttNum4012, \ttNum4130, \ttNum5003, \ttNum5043, \ttNum5063
\end{hosoku}

\begin{hosoku}\small
コモン変数(RHS): \ttNum500, \ttNum501, \ttNum503, \ttNum514\\
システム変数: \ttNum4003, \ttNum4012, \ttNum4130, \ttNum5002, \ttNum5003, \ttNum5062
\end{hosoku}

\begin{hosoku}\small
システム変数: \ttNum1004
\end{hosoku}

\begin{hosoku}\small
システム変数: \ttNum1004
\end{hosoku}



\end{appendices}
%%%%%%%%%%%%%%%%%%%%%%%%%%%%%%%%%%%%%%%%%%%%%%%%%%%%%%%%%
%%                %%%%%%%%%%%%%%%%%%%%%%%%%%%%%%%%%%%%%%%
%% Appendix       %%%%%%%%%%%%%%%%%%%%%%%%%%%%%%%%%%%%%%%
%% Numerical Calc %%%%%%%%%%%%%%%%%%%%%%%%%%%%%%%%%%%%%%%
%% End            %%%%%%%%%%%%%%%%%%%%%%%%%%%%%%%%%%%%%%%
%%                %%%%%%%%%%%%%%%%%%%%%%%%%%%%%%%%%%%%%%%
%%%%%%%%%%%%%%%%%%%%%%%%%%%%%%%%%%%%%%%%%%%%%%%%%%%%%%%%%
