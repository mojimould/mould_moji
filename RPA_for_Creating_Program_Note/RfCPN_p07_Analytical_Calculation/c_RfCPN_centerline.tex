%!TEX root = ../RPA_for_Creating_Program_Note.tex


\modHeadchapter[loC]{通り芯の幾何}
トップ・ボトムの両方に\index{がいさく@外削}外削がある場合を考える。
通常、それぞれの外削の中心は個別に決められはせず、片方の中心の位置を基準として、もう片方の中心が定められる。
これらの中心の位置の差(\textbf{通り芯}
%% footnote %%%%%%%%%%%%%%%%%%%%%
\footnote{通常、通り芯(centerline)というのはその名の通り中心線を表すことが多い。
しかし、ここでは\index{トップがわのがいさくちゅうしん@トップ側の外削中心}トップ側外削中心と\index{ボトムがわのがいさくちゅうしん@ボトム側の外削中心}ボトム側外削中心との位置の差を表す用語として「通り芯」と呼んでいる。})
%%%%%%%%%%%%%%%%%%%%%%%%%%%%%%%%%
$T_x$, $T_y$ ($T_x \geq 0$)を機内で測定する際は、C面が工具側に向くようにテーブルを$\pm$90$^\circ$回転($B$軸回転)し、\index{タッチセンサープローブ}タッチセンサープローブを用いてそれぞれの外削部の$Z$座標および$Y$座標を見ることで測定する。

ここでは、この通り芯の測定に必要な位置等について定量的に求める。
なお、\index{テーブルちゅうしん@テーブル中心}テーブルの中心Pを原点として考えることにする。
またC面が工具側に向くように$B$軸を(\verb|G91|にて)$\pm$90$^\circ$回転した状態であるとする
%% footnote %%%%%%%%%%%%%%%%%%%%%
\footnote{{\ttfamily G90}(絶対座標)の場合、\index{テーブル}テーブルを傾けて\index{ふりわけちょう@振分長}振分長を調整した場合はその\index{かたむきかく(ふりわけちょうせい)@傾き角(振分調整)}回転角$-\theta$を忘れないよう注意。}。
%%%%%%%%%%%%%%%%%%%%%%%%%%%%%%%%%



%%%%%%%%%%%%%%%%%%%%%%%%%%%%%%%%%%%%%%%%%%%%%%%%%%%%%%%%%%
%% section 27.1 %%%%%%%%%%%%%%%%%%%%%%%%%%%%%%%%%%%%%%%%%%
%%%%%%%%%%%%%%%%%%%%%%%%%%%%%%%%%%%%%%%%%%%%%%%%%%%%%%%%%%
\modHeadsection{ボトムの外削が基準の場合}
通常、トップ側の外削中心は、ボトム側の外削中心よりA面側($-Z$側)にある。
このとき、\KeywayPos$\kappa_p$および\BottomOutcutLength$h_\mathrm B$ ($h_\mathrm B > 0$)を用いると、ボトム側($-X$側)およびトップ側($+X$側)の\index{Cがわがいさくめん@C側外削面}C側外削面の中心
%% footnote %%%%%%%%%%%%%%%%%%%%%
\footnote{トップ側には\Keyway があるので、\TopOutcutLength は\KeywayPos$\kappa_p$とみなしている。}
%%%%%%%%%%%%%%%%%%%%%%%%%%%%%%%%%
は、それぞれ
%% footnote %%%%%%%%%%%%%%%%%%%%%
\footnote{通常、\index{Yほうこうの通り芯@$Y$方向の通り芯}$Y$方向の通り芯は$T_y = 0$である。}
%%%%%%%%%%%%%%%%%%%%%%%%%%%%%%%%%
\begin{align}
  \label{eq:centerlineB}
  \text{ボトム側:}\quad
  \left[
    \begin{array}{c}
      \displaystyle -f_\mathrm B'+\frac{h_\mathrm B}2\\[5pt]
      \mathcal G_{\mathrm By}\\[3pt]
      \displaystyle \mathcal G_{\mathrm Bx}+\frac{\mathfrak W_\mathrm B}2
    \end{array}
    \right]~, \qquad
  \text{トップ側:}\quad
  \left[
    \begin{array}{c}
      \displaystyle f_\mathrm T'-\frac{\kappa_p}2\\[5pt]
      \mathcal G_{\mathrm By}-T_y\\[3pt]
      \displaystyle \mathcal G_{\mathrm Bx}-T_x+\frac{\mathfrak W_\mathrm T}2
    \end{array}
  \right].
\end{align}



%%%%%%%%%%%%%%%%%%%%%%%%%%%%%%%%%%%%%%%%%%%%%%%%%%%%%%%%%%
%% section 27.2 %%%%%%%%%%%%%%%%%%%%%%%%%%%%%%%%%%%%%%%%%%
%%%%%%%%%%%%%%%%%%%%%%%%%%%%%%%%%%%%%%%%%%%%%%%%%%%%%%%%%%
\modHeadsection{トップの外削が基準の場合}
通常、ボトムの外削中心は、トップの外削中心よりC面側($+Z$側)にある。
このとき、トップ側($+X$側)およびボトム側($-X$側)の外削C面の中心は、それぞれ
\begin{align}
  \label{eq:centerlineT}
  \text{トップ側:}~~
  \left[
    \begin{array}{c}
      \displaystyle f_\mathrm T'-\frac{\kappa_p}2\\[5pt]
      \mathcal G_{\mathrm Ty}\\[3pt]
      \displaystyle \mathcal G_{\mathrm Tx}+\frac{\mathfrak W_\mathrm B}2
    \end{array}
    \right]~, \qquad
  \text{ボトム側:}~~
  \left[
    \begin{array}{c}
      \displaystyle -f_\mathrm B'+\frac{h_\mathrm B}2\\[5pt]
      \mathcal G_{\mathrm Ty}+T_y\\[3pt]
      \displaystyle \mathcal G_{\mathrm Tx}+T_x+\frac{\mathfrak W_\mathrm B}2
    \end{array}
  \right].
\end{align}

\clearpage
~\vfill
%%%%%%%%%%%%%%%%%%%%%%%%%%%%%%%%%%%%%%%%%%%%%%%%%%%%%%%%%%
%% Column %%%%%%%%%%%%%%%%%%%%%%%%%%%%%%%%%%%%%%%%%%%%%%%%
%%%%%%%%%%%%%%%%%%%%%%%%%%%%%%%%%%%%%%%%%%%%%%%%%%%%%%%%%%
\begin{Column}{C側面の測定}
通常、外削をせずに\index{Xほうこうのとおりしん@$X$方向の通り芯}通り芯の測定を行うことはない。
ただ、\index{プログラム(G-code)}プログラムの\index{しうんてん@試運転}試運転などで動きをみるといった可能性はありうるので、\index{がいさくかこう@外削加工}外削加工を行っていない状態で\index{そくてい(とおりしん)@測定(通り芯)}測定する場合についても述べておく。
なお、ここでは\index{テーブル}テーブルを回転して\index{ふりわけちょう@振分長}振分長の調整を行った場合、かつボトムの外削(\index{Aがわにくあつ@A側肉厚}A側肉厚)が基準の場合を考える。
このとき、測定するC側外面の位置は、\pageeqref{eq:tableTi}, \pageeqref{eq:tableBRi}より、
\begin{align*}
  \text{トップ側:}~~
  & \left[
    \begin{array}{c}
      \displaystyle f_\mathrm T'-\frac{\kappa_p}2\\[5pt]
      G_{\mathrm By}-T_y\\[3pt]
      \displaystyle
      -G_{\mathrm Tx}
      -\sqrt{R_\mathrm c^2-f_\mathrm T^2}
      +\sqrt{R_\mathrm c^2-\left(f_\mathrm T-\frac\kappa2\right)^2}
    \end{array}
    \right],\\
  \text{ボトム側:}~~
  & \left[
    \begin{array}{c}
      \displaystyle -f_\mathrm B'+\frac{h_\mathrm B}2\\[5pt]
      G_{\mathrm By}\\[3pt]
      \displaystyle
      G_{\mathrm Bx}
      -\sqrt{R_\mathrm c^2-f_\mathrm B^2}
      +\sqrt{R_\mathrm c^2-\left(f_\mathrm B-\frac{h_\mathrm B}2\right)^2}
    \end{array}
    \right].
\end{align*}
なお、これらの$Z$座標の差は以下で与えられる。
\begin{align*}
  \sqrt{R_\mathrm i^2-\left(f_\mathrm T-\frac{\kappa_p}2\right)^2}
  -\sqrt{R_\mathrm i^2-\left(f_\mathrm B-\frac{h_\mathrm B}2\right)^2}~.
\end{align*}
\end{Column}
%%%%%%%%%%%%%%%%%%%%%%%%%%%%%%%%%%%%%%%%%%%%%%%%%%%%%%%%%%
%%%%%%%%%%%%%%%%%%%%%%%%%%%%%%%%%%%%%%%%%%%%%%%%%%%%%%%%%%
%%%%%%%%%%%%%%%%%%%%%%%%%%%%%%%%%%%%%%%%%%%%%%%%%%%%%%%%%%
