%!TEX root = ../RPA_for_Creating_Program_Note.tex


\modHeadchapter[loC]{\Keyway の幾何}
ここでは主に、\textbf{\Keyway}に関する測定・加工に必要な\expandafterindex{きかてきせいしつ(\yomiKeyway)@幾何的性質(\nameKeyway)}幾何的性質を考える。



%%%%%%%%%%%%%%%%%%%%%%%%%%%%%%%%%%%%%%%%%%%%%%%%%%%%%%%%%%
%% section 22.1 %%%%%%%%%%%%%%%%%%%%%%%%%%%%%%%%%%%%%%%%%%
%%%%%%%%%%%%%%%%%%%%%%%%%%%%%%%%%%%%%%%%%%%%%%%%%%%%%%%%%%
\modHeadsection{\Keyway の基本事項}
\textbf{\Keyway}に関しては、その基準が以下のように与えられる場合が考えられる。
\begin{enumerate}[label=\sarrow]
\item \KeywayCenter Mが、ワークの\index{わんきょくちゅうしんせん@湾曲中心線}湾曲中心線上にある場合
\item \KeywayCenter Mが、(トップ側)\index{がいさくちゅうしん@外削中心}外削径の中心線上にある場合
\item \AsideKeywayDepth に指定がある場合
\end{enumerate}
なお、\KeywayDiameter を$W_\mathrm M$, \KeywayPos(端面から\Keyway までの長さ)・\KeywayWidth・\AsideKeywayDepth をそれぞれ$\kappa_p$, $\kappa_w$, $\kappa_d$とする。
このときいずれの場合も、$y$方向(機内における$Z$方向
%% footnote %%%%%%%%%%%%%%%%%%%%%
\footnote{計算上の$xy$座標($x$:実軸, $y$:虚軸)と、機内における$XZ$座標とが混在する形で話を進めているので注意されたし。})
%%%%%%%%%%%%%%%%%%%%%%%%%%%%%%%%%
の\index{せっさくはんい@切削範囲}切削範囲は、テーブル中心Pを\index{げんてん@原点}原点として、
\begin{align*}
  \big[f_\mathrm T'-(\kappa_p+\kappa_w)\ ,\ f_\mathrm T'-\kappa_p\big]
\end{align*}
であり、また\KeywayCenter M$'$の$y$座標($Z$座標)はこの切削範囲の中央
%% label{eq:mizocenterZ}
\begin{align}
  \label{eq:mizocenterZ}
  f_\mathrm T'-\bar\kappa_w \qquad
  \left(\bar\kappa_w \equiv \kappa_p+\frac{\kappa_w}2\right)
\end{align}
で与えられる。



\clearpage
%%%%%%%%%%%%%%%%%%%%%%%%%%%%%%%%%%%%%%%%%%%%%%%%%%%%%%%%%%
%% section 22.2 %%%%%%%%%%%%%%%%%%%%%%%%%%%%%%%%%%%%%%%%%%
%%%%%%%%%%%%%%%%%%%%%%%%%%%%%%%%%%%%%%%%%%%%%%%%%%%%%%%%%%
\modHeadsection{湾曲中心が基準の場合}
\index{トップたんのわんきょくちゅうしん@トップ端の湾曲中心}トップ端における湾曲中心T$_{R_\mathrm c}'$と\KeywayCenter M$'$との$x$方向の差は、
\begin{align*}
  \sqrt{R_\mathrm c^2-\left(f_\mathrm T-\bar\kappa_w\right)^2}
  -\sqrt{R_\mathrm c^2-f_\mathrm T^2}\ .
\end{align*}
実際の測定では、湾曲中心T$_{R_\mathrm c}'$ではなく\index{トップたんのそとがわちゅうしん@トップ端の外側中心}トップ端面の外側中心T$_\mathrm c'$が測定されるので、T$_\mathrm c'$とM$'$との$x$方向の差を考える必要がある。
すなわち、
%% label{eq:difTopMizoCenter}
\begin{align}
  \label{eq:difTopMizoCenter}
  \sqrt{R_\mathrm c^2-\left(f_\mathrm T-\bar\kappa_w\right)^2}
  -\frac{\sqrt{R_\mathrm o^2-f_\mathrm T^2}+\sqrt{R_\mathrm i^2-f_\mathrm T^2}}2\ .
\end{align}


%%%%%%%%%%%%%%%%%%%%%%%%%%%%%%%%%%%%%%%%%%%%%%%%%%%%%%%%%%
%% ubsection 23.2.1 %%%%%%%%%%%%%%%%%%%%%%%%%%%%%%%%%%%%%%
%%%%%%%%%%%%%%%%%%%%%%%%%%%%%%%%%%%%%%%%%%%%%%%%%%%%%%%%%%
\subsection{スペーサを用いた場合の\KeywayCenter(湾曲中心基準)}
\KeywayCenter M$'$がモールドの湾曲中心線上にある場合、テーブル中心Pを原点とした$x$座標は、\pageeqref{eq:spacerTRc}より、
\begin{align*}
  -\Delta+\sqrt{R_\mathrm c^2-(f_\mathrm T-\bar\kappa_w)^2}+\frac{\delta_\mathrm s}2
  -\sqrt{R_\mathrm i'^2-\frac{\delta_\mathrm s^2+(2\bar l)^2}4}\frac{2\bar l}{\sqrt{\delta_\mathrm s^2+\left(2\bar l\right)^2}}
\end{align*}
となる。
なお実際の作業では、簡単のため、トップ端面の外側中心T$_\mathrm c'$を測定し、それをトップ端面における湾曲中心T$_{R_\mathrm c}'$とみなして\KeywayCenter M$'$の位置を計算することが多い。
実測した外側中心の$X$・$Y$座標$G_{\mathrm Tx}$, $G_{\mathrm Ty}$を湾曲中心のそれとみなすと、機内における\KeywayCenter M$'$の位置は、テーブル中心Pを原点として、
\begin{align*}
  \left(
    G_{\mathrm Tx}
    +\sqrt{R_\mathrm c^2-(f_\mathrm T-\bar\kappa_w)^2}
    -\sqrt{R_\mathrm c^2-f_\mathrm T^2}\ ,\
    G_{\mathrm Ty}~,~
    f_\mathrm T'-\bar\kappa_w
  \right).
\end{align*}
湾曲中心とみなさずに正確に求めるなら、これに\pageeqref{eq:TRc-Tc}を引けばよい。
その場合の$X$座標は、
%% label{eq:Mreal}
\begin{align}
  \label{eq:Mreal}
  G_{\mathrm Tx}
  +\sqrt{R_\mathrm c^2-(f_\mathrm T-\bar\kappa_w)^2}
  -\frac{\sqrt{R_\mathrm o^2-f_\mathrm T^2}+\sqrt{R_\mathrm i^2-f_\mathrm T^2}}2\ .
\end{align}


%%%%%%%%%%%%%%%%%%%%%%%%%%%%%%%%%%%%%%%%%%%%%%%%%%%%%%%%%%
%% subsection 04.2.2 %%%%%%%%%%%%%%%%%%%%%%%%%%%%%%%%%%%%%
%%%%%%%%%%%%%%%%%%%%%%%%%%%%%%%%%%%%%%%%%%%%%%%%%%%%%%%%%%
\subsection{テーブルを傾けた場合の\KeywayCenter(湾曲中心基準)}
\KeywayCenter M$'$がモールドの湾曲中心線上にある場合、テーブル中心Pを原点とした$x$座標は、\pageeqref{eq:tableTRc}より、
\begin{align*}
  \sqrt{R_\mathrm c^2-(f_\mathrm T-\bar\kappa_w)^2}
  -\Delta'\cos\theta\ .
\end{align*}
実測した\index{そとがわちゅうしん@外側中心}外側中心の$X$, $Y$座標$G_{\mathrm Tx}$, $G_{\mathrm Ty}$を\index{わんきょくちゅうしん@湾曲中心}湾曲中心のそれとみなした場合とそうでない場合は、\pageeqref{eq:Mreal}で与えられる。



\clearpage
%%%%%%%%%%%%%%%%%%%%%%%%%%%%%%%%%%%%%%%%%%%%%%%%%%%%%%%%%%
%% section 04.2 %%%%%%%%%%%%%%%%%%%%%%%%%%%%%%%%%%%%%%%%%%
%%%%%%%%%%%%%%%%%%%%%%%%%%%%%%%%%%%%%%%%%%%%%%%%%%%%%%%%%%
\modHeadsection{外削径の中心が基準の場合}
\KeywayCenter M$'$がトップ外削径の中心線上にある場合、機内におけるその位置座標は、
\begin{align*}
  \left(
    -\mathcal G_{Bx}+T_x\ ,\
    \mathcal G_{By}\ ,\
    f_\mathrm T'-\bar\kappa_w
  \right) \qquad
  \text{または}\qquad
  \left(
    \mathcal G_{Bx}\ ,\
    \mathcal G_{By}\ ,\
    f_\mathrm T'-\bar\kappa_w
  \right).
\end{align*}
ただし、前者はボトムの外削を基準にした(ボトム基準の\CenterlineEndFaceDif がある)場合であり、後者はトップの外削を基準にした場合である。



%\clearpage
%%%%%%%%%%%%%%%%%%%%%%%%%%%%%%%%%%%%%%%%%%%%%%%%%%%%%%%%%%
%% section 04.3 %%%%%%%%%%%%%%%%%%%%%%%%%%%%%%%%%%%%%%%%%%
%%%%%%%%%%%%%%%%%%%%%%%%%%%%%%%%%%%%%%%%%%%%%%%%%%%%%%%%%%
\modHeadsection{\AsideKeywayDepth が基準の場合}


%%%%%%%%%%%%%%%%%%%%%%%%%%%%%%%%%%%%%%%%%%%%%%%%%%%%%%%%%%
%% subsection 4.3.1 %%%%%%%%%%%%%%%%%%%%%%%%%%%%%%%%%%%%%%
%%%%%%%%%%%%%%%%%%%%%%%%%%%%%%%%%%%%%%%%%%%%%%%%%%%%%%%%%%
\subsection{外削のない場合}
トップ側に外削がなく、\AsideKeywayDepth が指定されている場合を考える。
\KeywayCenter の位置の$X$座標は、テーブル中心Pを\index{げんてん@原点}原点として、
%% label{eq:mizocenterA}
\begin{align}
  \label{eq:mizocenterA}
  \sqrt{R_\mathrm o^2-(f_\mathrm T-\bar\kappa_w)^2}-\kappa_d-\frac{W_{mx}}2
  -\Delta'
\end{align}
で与えられる。
ここで、$W_{mx}$は\KeywayACOD を表す。
なお実際の作業では、モールドの\index{Aがわがいめん@A側外面}A側外面の\KeywayCenter に相当する箇所を直接測定し、その位置を基準として原点を割り出す。
トップ端面における中心の$X$座標(\index{じっそくち@実測値}実測値)$G_{\mathrm Tx}$がわかっている場合、\KeywayCenter の$X$座標$G_{mx}$は\pageeqref{eq:Mreal}で与えられ、また\KeywayCenter に対する\index{Aがわがいめん@A側外面}A側外面と$G_{\mathrm Tx}$との差(の絶対値)は、
%% label{eq:mizocenterA}
\begin{align}
  \label{eq:mizocenterAd}
  \frac{W_{mx}}2+\kappa_d
  +\sqrt{R_\mathrm c^2-(f_\mathrm T-\bar\kappa_w)^2}
  -\frac{\sqrt{R_\mathrm o^2-f_\mathrm T^2}+\sqrt{R_\mathrm i^2-f_\mathrm T^2}}2\ .
\end{align}


%%%%%%%%%%%%%%%%%%%%%%%%%%%%%%%%%%%%%%%%%%%%%%%%%%%%%%%%%%
%% subsection 4.3.2 %%%%%%%%%%%%%%%%%%%%%%%%%%%%%%%%%%%%%%
%%%%%%%%%%%%%%%%%%%%%%%%%%%%%%%%%%%%%%%%%%%%%%%%%%%%%%%%%%
\subsection{外削のある場合}
トップ側に外削があり、かつ\AsideKeywayDepth が指定されている場合を考える。
トップ側の外削中心の実測値を$\mathcal G_{\mathrm Tx}$とすると、\KeywayCenter$G_{mx}$との差($G_{mx}-\mathcal G_{\mathrm Tx}$)は
%% label{eq:mizocenterAG}
\begin{align}
  \label{eq:mizocenterAG}
  \frac{\mathfrak W_x}2-\kappa_d-\frac{W_{mx}}2\ .
\end{align}



\clearpage
%%%%%%%%%%%%%%%%%%%%%%%%%%%%%%%%%%%%%%%%%%%%%%%%%%%%%%%%%%
%% section 04.4 %%%%%%%%%%%%%%%%%%%%%%%%%%%%%%%%%%%%%%%%%%
%%%%%%%%%%%%%%%%%%%%%%%%%%%%%%%%%%%%%%%%%%%%%%%%%%%%%%%%%%
\modHeadsection{測定上の\KeywayDepth}
ここでは特に\AsideKeywayDepth$\kappa_d$に限って話を進める。
トップ側に外削がある場合、$\kappa_d$は\index{Aがわがいさくめん@A側外削面}A側外削面と\Keyway のA側面との差で与えられる。
しかし外削がない場合、つまり\index{Aがわがいめん@A側外面}A側外面に湾曲がある場合は、$\kappa_d$の値は自明ではない。


%%%%%%%%%%%%%%%%%%%%%%%%%%%%%%%%%%%%%%%%%%%%%%%%%%%%%%%%%%
%% subsection 4.4.1 %%%%%%%%%%%%%%%%%%%%%%%%%%%%%%%%%%%%%%
%%%%%%%%%%%%%%%%%%%%%%%%%%%%%%%%%%%%%%%%%%%%%%%%%%%%%%%%%%
\subsection{図面上の\KeywayDepth}
単純に考えると、図面上においてAsideKeywayDepth が$\kappa_d$のとき、これは\KeywayCenter の位置におけるA側外面から端面と水平な方向に$\kappa_d$という意味で与えられる。
このとき、\Keyway のA面側の水平方向の位置は、\pageeqref{eq:mizocenterA}より、
\begin{align*}
  \sqrt{R_\mathrm o^2-(f_\mathrm T-\bar\kappa_w)^2}-\kappa_d-\Delta'\ .
\end{align*}
\KeywayCenter の$X$座標$G_{mx}$が与えられている場合は、
\begin{align*}
  G_{mx}+\frac{W_{mx}}2
\end{align*}


%%%%%%%%%%%%%%%%%%%%%%%%%%%%%%%%%%%%%%%%%%%%%%%%%%%%%%%%%%
%% subsection 23.4.2 %%%%%%%%%%%%%%%%%%%%%%%%%%%%%%%%%%%%%
%%%%%%%%%%%%%%%%%%%%%%%%%%%%%%%%%%%%%%%%%%%%%%%%%%%%%%%%%%
\subsection{測定上の傾き}
通常、実測には\index{マイクロメータ}マイクロメータ(\index{デプスゲージ}デプスゲージ)が用いられる。
このとき、測定は\index{そくていき@測定器}測定器を湾曲に沿った形にして行われる。
したがって、その測定値は\KeywayWidth の両端に対する外面の傾斜だけ傾いた形で与えられる。
\KeywayWidth の両端に対する外面の$XZ$位置は、\KeywayCenter の$X$座標を$G_{mx}$として、
\begin{align*}
  \text{トップ側:}&~~
  \left(
  G_{mx}+\frac{W_x}2
  -\sqrt{R_\mathrm o^2-(f_\mathrm T-\bar\kappa_w)^2}
  +\sqrt{R_\mathrm o^2-(f_\mathrm T-\kappa_p)^2}~,~~
  f_\mathrm T-\kappa_p
  \right)\\
  \text{ボトム側:}&~~
  \left(
  G_{mx}+\frac{W_x}2
  +\sqrt{R_\mathrm o^2-(f_\mathrm T-\kappa_p-\kappa_w)^2}
  -\sqrt{R_\mathrm o^2-(f_\mathrm T-\bar\kappa_w)^2}~,~~
  f_\mathrm T-\kappa_p-\kappa_w
  \right)
\end{align*}
またその差は、
\begin{align*}
  \left(
  \sqrt{R_\mathrm o^2-(f_\mathrm T-\kappa_p)^2}
  -\sqrt{R_\mathrm o^2-(f_\mathrm T-\kappa_p-\kappa_w)^2}~,~~
  \kappa_w
  \right)
\end{align*}
したがって、測定の際の傾斜の角度$\zeta$ ($> 0$)は
%% label{eq:angleZeta}
\begin{align}
  \label{eq:angleZeta}
  \tan\zeta
  = \frac{\sqrt{R_\mathrm o^2-\left(f_\mathrm T-\kappa_p-\kappa_w\right)^2}
          -\sqrt{R_\mathrm o^2-\left(f_\mathrm T-\kappa_p\right)^2}}
         {\kappa_w}\ .
\end{align}


\clearpage
%%%%%%%%%%%%%%%%%%%%%%%%%%%%%%%%%%%%%%%%%%%%%%%%%%%%%%%%%%
%% subsection 4.4.3 %%%%%%%%%%%%%%%%%%%%%%%%%%%%%%%%%%%%%%
%%%%%%%%%%%%%%%%%%%%%%%%%%%%%%%%%%%%%%%%%%%%%%%%%%%%%%%%%%
\subsection{測定における\KeywayDepth の補正}%\label{subsec:keywayDepthDif}
測定の際は、測定器を\Keyway のトップ側およびボトム側に寄せる形で測定し、その平均値を測定値としている。
つまり、トップ側に寄っているときは測定器の針の付け根が\KeywayPos にあるところ、ボトム側に寄っているときは針の先端が\KeywayWidth ボトム側にあるところで測定を行っている。
この平均値を$\kappa_d'$とする。

トップ側およびボトム側の\AsideKeywayDepth をそれぞれ$\kappa_s$, $\kappa_l$とすると、
\begin{align*}
  \kappa_l-\kappa_s = \kappa_w\tan\zeta \quad,\qquad
  \kappa_d' = \frac{\kappa_l\cos\zeta+\kappa_s\sec\zeta}2
\end{align*}
であり、
\begin{align*}
  \kappa_l = \frac{2\kappa_d'\cos\zeta+\kappa_w\tan\zeta}{1+\cos^2\zeta}~~, \quad
  \kappa_s = \frac{2\kappa_d'-\kappa_w\sin\zeta}{1+\cos^2\zeta}\cos\zeta\ .
\end{align*}
%%%%%%%%%%%%%%%%%%%%%%%%%%%%%%%%%%%%%%%%%%%%%%%%%%%%%%%%%%
%% hosoku %%%%%%%%%%%%%%%%%%%%%%%%%%%%%%%%%%%%%%%%%%%%%%%%
%%%%%%%%%%%%%%%%%%%%%%%%%%%%%%%%%%%%%%%%%%%%%%%%%%%%%%%%%%
\begin{hosoku}
$\kappa_l = a+b$, $\kappa_s = a-b$とすると、
\begin{align*}
  a = \frac{\kappa_l+\kappa_s}2~~, \quad
  b = \frac{\kappa_l-\kappa_s}2\,\left(= \frac12\kappa_w\tan\zeta\right)
\end{align*}
であり、
\begin{align*}
  2\kappa_d' = a(\cos\zeta+\sec\zeta)+b(\cos\zeta-\sec\zeta)
  \quad\longrightarrow\quad
  a = \frac{2\kappa_d'\cos\zeta+b\sin^2\zeta}{1+\cos^2\zeta}\ .
\end{align*}
また、$\kappa_l$, $\kappa_s$はその平均値$a$からその差の半分$b$の和あるいは差として得られる。
\begin{align*}
  \kappa_l
  &= \frac{2\kappa_d'\cos\zeta+\frac{\kappa_w\tan\zeta}2\sin^2\zeta}{1+\cos^2\zeta}+\frac12\kappa_w\tan\zeta
   = \frac{2\kappa_d'\cos\zeta+\kappa_w\tan\zeta}{1+\cos^2\zeta}\ ,\\
  \kappa_s
  &= \frac{2\kappa_d'\cos\zeta+\frac{\kappa_w\tan\zeta}2\sin^2\zeta}{1+\cos^2\zeta}-\frac12\kappa_w\tan\zeta
   = \frac{2\kappa_d'-\kappa_w\sin\zeta}{1+\cos^2\zeta}\cos\zeta\ .
\end{align*}
\end{hosoku}
%%%%%%%%%%%%%%%%%%%%%%%%%%%%%%%%%%%%%%%%%%%%%%%%%%%%%%%%%%
%%%%%%%%%%%%%%%%%%%%%%%%%%%%%%%%%%%%%%%%%%%%%%%%%%%%%%%%%%
%%%%%%%%%%%%%%%%%%%%%%%%%%%%%%%%%%%%%%%%%%%%%%%%%%%%%%%%%%
$\kappa_d$および$\kappa_s$の差は
\begin{align*}
  \kappa_d-\kappa_s
  &= \sqrt{R_\mathrm o^2-(f_\mathrm T-\bar\kappa_w)^2}
     -\sqrt{R_\mathrm o^2-(f_\mathrm T-\kappa_p)^2}
\end{align*}
なので、
\begin{subequations}
%% label{eq:keydepthDif1}
\begin{align}
  \label{eq:keydepthDif1}
  \kappa_d
  &= \frac{2\kappa_d'-\kappa_w\sin\zeta}{1+\cos^2\zeta}\cos\zeta
     +\sqrt{R_\mathrm o^2-(f_\mathrm T-\bar\kappa_w)^2}
     -\sqrt{R_\mathrm o^2-(f_\mathrm T-\kappa_p)^2}
\end{align}
あるいは、
%% label{eq:keydepthDif2}
\begin{align}
  \label{eq:keydepthDif2}
  \kappa_d'
  &= \frac{1+\cos^2\zeta}{2\cos\zeta}
     \left\{
     \kappa_d
     -\sqrt{R_\mathrm o^2-(f_\mathrm T-\bar\kappa_w)^2}
     +\sqrt{R_\mathrm o^2-(f_\mathrm T-\kappa_p)^2}
     \right\}
     +\frac12\kappa_w\sin\zeta\ .
\end{align}
\end{subequations}
したがって、$\kappa_d$を図面上の数値とする場合は\pageeqref{eq:keydepthDif2}を、$\kappa_d'$を図面上の数値とする場合は、\pageeqref{eq:keydepthDif1}を用いて補正すればよい。

\clearpage
~\vfill
%%%%%%%%%%%%%%%%%%%%%%%%%%%%%%%%%%%%%%%%%%%%%%%%%%%%%%%%%%
%% Column %%%%%%%%%%%%%%%%%%%%%%%%%%%%%%%%%%%%%%%%%%%%%%%%
%%%%%%%%%%%%%%%%%%%%%%%%%%%%%%%%%%%%%%%%%%%%%%%%%%%%%%%%%%
\begin{Column}{測定における\KeywayDepth 補正の近似計算}
$R\to\infty$ ($R^{-1}\to0$)に対して、
\begin{align*}
  \tan\zeta \xlongrightarrow{R\to\infty} 0~~, \quad
  \kappa_d' \xlongrightarrow{R\to\infty} \kappa_d\ .
\end{align*}
もう少し詳しく見ると、\index{テイラーてんかい@テイラー展開}テイラー展開(\index{マクローリンてんかい@マクローリン展開}マクローリン展開)\pageautoref{formula:taylorexpansion}より、
\begin{align*}
  \tan\zeta
  = \frac{R_\mathrm o}{\kappa_w}
     \left\{
     \sqrt{1-\left(\frac{f_\mathrm T-\kappa_p-\kappa_w}{R_\mathrm o}\right)^2}
     -\sqrt{1-\left(\frac{f_\mathrm T-\kappa_p}{R_\mathrm o}\right)^2}
     \right\}
  = \frac{f_\mathrm T-\bar\kappa_w}{R_\mathrm o}+o\left(R_\mathrm o^{-3}\right)
\end{align*}
であり、これより、
\begin{align*}
  \sin\zeta = \frac{f_\mathrm T-\bar\kappa_w}{R_\mathrm o}+o\left(R_\mathrm o^{-3}\right)~~, \quad
  \cos\zeta = 1-\frac12\left(\frac{f_\mathrm T-\bar\kappa_w}{R_\mathrm o}\right)^2
              +o\left(R_\mathrm o^{-4}\right)
\end{align*}
したがって、
\begin{align*}
  \kappa_d
  &= \kappa_d'-\frac{\kappa_w}2\frac{f_\mathrm T-\bar\kappa_w}{R_\mathrm o}
     +R_\mathrm o
      \left\{
      \sqrt{1-\left(\frac{f_\mathrm T-\bar\kappa_w}{R_\mathrm o}\right)^2}
      -\sqrt{1-\left(\frac{f_\mathrm T-\kappa_p}{R_\mathrm o}\right)^2}
      \right\}\\
  &= \kappa_d'+\frac{\kappa_w^2}{8R_\mathrm o}
     +o\left(R_\mathrm o^{-3}\right)\ .
\end{align*}
これから、$\kappa_d > \kappa_d'$であることがわかる。
\end{Column}
%%%%%%%%%%%%%%%%%%%%%%%%%%%%%%%%%%%%%%%%%%%%%%%%%%%%%%%%%%
%%%%%%%%%%%%%%%%%%%%%%%%%%%%%%%%%%%%%%%%%%%%%%%%%%%%%%%%%%
%%%%%%%%%%%%%%%%%%%%%%%%%%%%%%%%%%%%%%%%%%%%%%%%%%%%%%%%%%
