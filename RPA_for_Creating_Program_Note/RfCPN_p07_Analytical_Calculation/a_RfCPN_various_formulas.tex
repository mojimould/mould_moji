%!TEX root = ../RPA_for_Creating_Program_Note.tex


\modHeadchapter{諸公式}



%%%%%%%%%%%%%%%%%%%%%%%%%%%%%%%%%%%%%%%%%%%%%%%%%%%%%%%%%%
%% section F.1 %%%%%%%%%%%%%%%%%%%%%%%%%%%%%%%%%%%%%%%%%%%
%%%%%%%%%%%%%%%%%%%%%%%%%%%%%%%%%%%%%%%%%%%%%%%%%%%%%%%%%%
\modHeadsection{近似計算}
\begin{Formula}[label=formula:taylorexpansion]{テイラー展開(マクローリン展開)}
$f(x)$の$x = 0$に対する\index{テイラーてんかい@テイラー展開}テイラー展開(\index{マクローリンてんかい@マクローリン展開}マクローリン展開)は、以下で与えられる。
\begin{align*}
  f(x) = \sum_{n=1}^{\infty}\frac{f^{(n)}(0)}{n!}x^n\ .
\end{align*}
特に、
\begin{align*}
  (1+x)^\frac12 &= 1+\frac x2-\frac{x^2}8+\frac{x^3}{16}-\frac{5x^4}{128}+o\!\left(x^5\right)\\
  \frac{\cos x}{1+\cos^2x} &= \frac12-\frac{x^4}{16}+o\left(x^6\right)\\
  \frac{\sin x\cos x}{1+\cos^2x} &= \frac x2-\frac{x^3}{12}-\frac{7x^5}{120}+o\left(x^7\right)
\end{align*}
\end{Formula}



\clearpage
%%%%%%%%%%%%%%%%%%%%%%%%%%%%%%%%%%%%%%%%%%%%%%%%%%%%%%%%%%
%% section F.2 %%%%%%%%%%%%%%%%%%%%%%%%%%%%%%%%%%%%%%%%%%%
%%%%%%%%%%%%%%%%%%%%%%%%%%%%%%%%%%%%%%%%%%%%%%%%%%%%%%%%%%
\modHeadsection{2点間の距離}
\begin{Formula}{点と直線間の距離}
点($p$, $q$)と直線$ax+by+c=0$との距離$d$は、以下で与えられる。
\begin{align*}
  d = \frac{|ap+bq+c|}{\sqrt{a^2+b^2}}.
\end{align*}
\end{Formula}
\begin{Formula}{直線上の点と直線間の距離}
点$\boldsymbol p$を通り方向ベクトルが$\boldsymbol m$の直線L上の点と、点$\boldsymbol q$を通り方向ベクトルが$\boldsymbol m'$の直線$\mathrm L'$上の点は、それぞれパラメータ$t$, $t'$を用いて、
\begin{align*}
  \mathrm L: \boldsymbol p+t\boldsymbol m\ , \qquad
  \mathrm L': \boldsymbol q+t'\boldsymbol m'
\end{align*}
で表される。
このとき、L上の点の中で$\mathrm L'$に最も近づく点の位置$\boldsymbol k$は、以下で与えられる
%% footnote %%%%%%%%%%%%%%%%%%%%%
\footnote{2点間の距離の2乗$|\boldsymbol p-\boldsymbol q+t\boldsymbol m-t'\boldsymbol m'|^2$に対し、それぞれのパラメータ$t$, $t'$に関する微分が0となる。
それらを連立して解けば$\boldsymbol k$, $\boldsymbol k'$が求まる。}。
%%%%%%%%%%%%%%%%%%%%%%%%%%%%%%%%%
$\mathrm L'$上の点の中でLに最も近づく点の位置$\boldsymbol k'$についても同様である。
\begin{align*}
  \boldsymbol k
  = \boldsymbol p
    +\frac{\left\{\boldsymbol m-(\boldsymbol m, \boldsymbol m')\boldsymbol m', \boldsymbol p-\boldsymbol q\right\}}
          {1+\left(\boldsymbol m, \boldsymbol m'\right)^2}\boldsymbol m
\end{align*}
また、これらの差の大きさ$\big|\boldsymbol k-\boldsymbol k'\big|$から、2直線間の距離$d$が求まる。
\end{Formula}
