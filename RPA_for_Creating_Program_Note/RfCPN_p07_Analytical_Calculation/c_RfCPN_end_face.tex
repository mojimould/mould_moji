%!TEX root = ../RPA_for_Creating_Program_Note.tex


\modHeadchapter[loC]{端面(外径)の幾何}
ここでは主に、\index{モールド}モールドの\index{たんめん@端面}\textbf{端面}に関する測定・加工に必要な\expandafterindex{きかがくてきせいしつ(たんめん)@幾何学的性質(端面)}幾何学的性質を考える。
ただし機内のモールドは、考えている側の端面が工具側に向いているものとする。



%%%%%%%%%%%%%%%%%%%%%%%%%%%%%%%%%%%%%%%%%%%%%%%%%%%%%%%%%%
%% section 2.1 %%%%%%%%%%%%%%%%%%%%%%%%%%%%%%%%%%%%%%%%%%%
%%%%%%%%%%%%%%%%%%%%%%%%%%%%%%%%%%%%%%%%%%%%%%%%%%%%%%%%%%
\modHeadsection{トップ側の端面}


%%%%%%%%%%%%%%%%%%%%%%%%%%%%%%%%%%%%%%%%%%%%%%%%%%%%%%%%%%
%% subsection 2.1.1 %%%%%%%%%%%%%%%%%%%%%%%%%%%%%%%%%%%%%%
%%%%%%%%%%%%%%%%%%%%%%%%%%%%%%%%%%%%%%%%%%%%%%%%%%%%%%%%%%
\subsection{トップ端面における湾曲中心の位置}

%%%%%%%%%%%%%%%%%%%%%%%%%%%%%%%%%%%%%%%%%%%%%%%%%%%%%%%%%%
%% subsubsection 2.1.1.1 %%%%%%%%%%%%%%%%%%%%%%%%%%%%%%%%%
%%%%%%%%%%%%%%%%%%%%%%%%%%%%%%%%%%%%%%%%%%%%%%%%%%%%%%%%%%
\subsubsection{スペーサを用いた場合のT\texorpdfstring{$_{R_\mathrm c}'$}{Rc'}}
\index{スペーサ}スペーサを取付けた後の\index{トップたんのわんきょくちゅうしん@トップ端の湾曲中心}トップ端の湾曲中心の位置T$_{R_\mathrm c}'$と、\index{テーブルちゅうしん@テーブル中心}テーブル中心Pとの$X$方向の差は、
\begin{align*}
  \left(
    R_\mathrm ce^{i\alpha_\mathrm c}
    -R_\mathrm i'e^{-i\alpha'_{\mathrm U_\mathrm B}}
    +R_\mathrm i'e^{-i\alpha_{\mathrm U_\mathrm B}}
  \right)
  -\Delta'
  = R_\mathrm ce^{i\alpha_\mathrm c}-R_\mathrm i'e^{-i\alpha'_{\mathrm U_\mathrm B}}-\Delta \qquad
    \left(\sin\alpha_\mathrm c = \frac{f_\mathrm T}{R_\mathrm c}\right)
\end{align*}
の実部を見ればよい。
したがって
%% footnote %%%%%%%%%%%%%%%%%%%%%
\footnote{この場合、トップ側が工具側に向いている。}、
%%%%%%%%%%%%%%%%%%%%%%%%%%%%%%%%%
%% label{eq:spacerTRc}
\begin{align}
  \notag
  &  R_\mathrm c\cos\alpha_\mathrm c-R_\mathrm i'\cos\alpha'_{\mathrm U_\mathrm B}-\Delta\\
  &= -\Delta+\sqrt{R_\mathrm c^2-f_\mathrm T^2}+\frac{\delta_\mathrm s}2
     -\sqrt{R_\mathrm i'^2-\frac{\delta_\mathrm s^2+(2\bar l)^2}4}
      \frac{2\bar l}{\sqrt{\delta_\mathrm s^2+\left(2\bar l\right)^2}}
     \label{eq:spacerTRc}
\end{align}
で与えられる
%% footnote %%%%%%%%%%%%%%%%%%%%%
\footnote{実際の作業では、この点を(\index{たんめんのわんきょくちゅうしん@端面の湾曲中心}端面の湾曲中心T$_{R_\mathrm c}\!$でなく)\index{たんめんのがいさくちゅうしん@端面の外削中心}端面の外側中心T$_\mathrm c$とみなすことが多い。}。
%%%%%%%%%%%%%%%%%%%%%%%%%%%%%%%%%


%%%%%%%%%%%%%%%%%%%%%%%%%%%%%%%%%%%%%%%%%%%%%%%%%%%%%%%%%%
%% subsubsection 2.1.1.2 %%%%%%%%%%%%%%%%%%%%%%%%%%%%%%%%%
%%%%%%%%%%%%%%%%%%%%%%%%%%%%%%%%%%%%%%%%%%%%%%%%%%%%%%%%%%
\subsubsection{テーブルを傾けた場合のT\texorpdfstring{$_{R_\mathrm c}'$}{Rc'}}
テーブルを傾けた後の\index{トップたんのわんきょくちゅうしん@トップ端の湾曲中心}トップ端の湾曲中心の位置T$_{R_\mathrm c}'$と、テーブル中心Pとの$X$方向の差は、
\begin{align*}
  \left(R_\mathrm ce^{i\alpha_\mathrm c}-\Delta'e^{-i\theta}+\Delta'\right)-\Delta'
  = R_\mathrm ce^{i\alpha_\mathrm c}-\Delta'e^{-i\theta}
\end{align*}
の実部を見ればよい。
すなわち、
%% label{eq:tableTRc}
\begin{align}
  \label{eq:tableTRc}
  R_\mathrm c\cos\alpha_\mathrm c-\Delta'\cos\theta
  = \sqrt{R_\mathrm c^2-f_\mathrm T^2}-\left(\Delta+\sqrt{R_i'^2-\bar l^2}\right)\cos\theta~.
\end{align}


\clearpage
%%%%%%%%%%%%%%%%%%%%%%%%%%%%%%%%%%%%%%%%%%%%%%%%%%%%%%%%%%
%% subsection 2.1.2 %%%%%%%%%%%%%%%%%%%%%%%%%%%%%%%%%%%%%%
%%%%%%%%%%%%%%%%%%%%%%%%%%%%%%%%%%%%%%%%%%%%%%%%%%%%%%%%%%
\subsection{トップ端面における外側中心の位置}
\index{トップたんのわんきょくちゅうしん@トップ端の湾曲中心}トップ端の湾曲中心T$_{R_\mathrm c}'$と\index{トップたんのがいけいちゅうしん@トップ端の外径中心}外径中心T$_\mathrm c'$との差は、以下で与えられる。
%% label{eq:TRc-Tc}
\begin{align}
  \label{eq:TRc-Tc}
  \sqrt{R_\mathrm c^2-f_\mathrm T^2}
  -\frac{\sqrt{R_\mathrm o^2-f_\mathrm T^2}+\sqrt{R_\mathrm i^2-f_\mathrm T^2}}2\ .
\end{align}
よって、\index{がいけいちゅうしん@外径中心}外径中心T$_\mathrm c'$の位置は、\index{わんきょくちゅうしん@湾曲中心}湾曲中心T$_{R_\mathrm c}'$から\pageeqref{eq:TRc-Tc}だけ加味すればよい。
以下では、外径中心T$_\mathrm c'$の位置を直接計算し、このことが整合していることを確かめる。

%%%%%%%%%%%%%%%%%%%%%%%%%%%%%%%%%%%%%%%%%%%%%%%%%%%%%%%%%%
%% subsubsection 2.1.2.1 %%%%%%%%%%%%%%%%%%%%%%%%%%%%%%%%%
%%%%%%%%%%%%%%%%%%%%%%%%%%%%%%%%%%%%%%%%%%%%%%%%%%%%%%%%%%
\subsubsection{スペーサを用いた場合のT\texorpdfstring{$_\mathrm c'$}{c'}}
同様にして、外面A・C側のトップ端点T$_\mathrm o'$, T$_\mathrm i'$の、\index{テーブルちゅうしん@テーブル中心}テーブル中心Pを原点とした場合の$X$座標はそれぞれ、
\begin{align*}
  \text{C側端点:}&
  -\Delta+\sqrt{R_\mathrm i^2-f_\mathrm T^2}+\frac{\delta_\mathrm s}2
  -\sqrt{R_\mathrm i'^2-\frac{\delta_\mathrm s^2+(2\bar l)^2}4}\frac{2\bar l}{\sqrt{\delta_\mathrm s^2+(2\bar l)^2}}\ ,\\
  \text{A側端点:}&
  -\Delta+\sqrt{R_\mathrm o^2-f_\mathrm T^2}+\frac{\delta_\mathrm s}2
  -\sqrt{R_\mathrm i'^2-\frac{\delta_\mathrm s^2+(2\bar l)^2}4}\frac{2\bar l}{\sqrt{\delta_\mathrm s^2+(2\bar l)^2}}\ .
\end{align*}
したがって、トップ端における外径中心T$_\mathrm c'$の$X$座標は、
%% label{eq:spacerTc}
\begin{align}
  \label{eq:spacerTc}
  -\Delta+\frac{\sqrt{R_\mathrm o^2-f_\mathrm T^2}+\sqrt{R_\mathrm i^2-f_\mathrm T^2}}2
  +\frac{\delta_\mathrm s}2
  -\sqrt{R_\mathrm i'^2-\frac{\delta_\mathrm s^2+(2\bar l)^2}4}
   \frac{2\bar l}{\sqrt{\delta_\mathrm s^2+(2\bar l)^2}}\ .
\end{align}
これより、\index{わんきょくちゅうしん@湾曲中心}湾曲中心T$_{R_\mathrm c}'$と\index{がいけいちゅうしん@外径中心}外径中心T$_\mathrm c'$との差は\pageeqref{eq:TRc-Tc}となることがわかる。

%%%%%%%%%%%%%%%%%%%%%%%%%%%%%%%%%%%%%%%%%%%%%%%%%%%%%%%%%%
%% subsubsection 2.1.2.2 %%%%%%%%%%%%%%%%%%%%%%%%%%%%%%%%%
%%%%%%%%%%%%%%%%%%%%%%%%%%%%%%%%%%%%%%%%%%%%%%%%%%%%%%%%%%
\subsubsection{テーブルを傾けた場合のT\texorpdfstring{$_\mathrm c'$}{c'}}
同様にして、外面A・C側のトップ端点T$_\mathrm o'$, T$_\mathrm i'$の、\index{テーブルちゅうしん@テーブル中心}テーブル中心Pを原点とした場合の$X$座標はそれぞれ、
\begin{subequations}
\begin{align}
%% label{eq:tableTi}
  \label{eq:tableTi}
  \text{C側端点:}&~
  \sqrt{R_\mathrm i^2-f_\mathrm T^2}-\Delta'\cos\theta\ ,\\
  \text{A側端点:}&~
  \sqrt{R_\mathrm o^2-f_\mathrm T^2}-\Delta'\cos\theta\ .
\end{align}
\end{subequations}
したがって、トップ端における(AC外径の)中点T$_\mathrm c'$の$X$座標は、
%% label{eq:tableTc}
\begin{align}
  \label{eq:tableTc}
  \frac{\sqrt{R_\mathrm o^2-f_\mathrm T^2}+\sqrt{R_\mathrm i^2-f_\mathrm T^2}}2
  -\Delta'\cos\theta\ .
\end{align}
これより、\index{わんきょくちゅうしん@湾曲中心}湾曲中心T$_{R_\mathrm c}'$と\index{がいけいちゅうしん@外径中心}外径中心T$_\mathrm c'$との差は\pageeqref{eq:TRc-Tc}となることがわかる。




\clearpage
%%%%%%%%%%%%%%%%%%%%%%%%%%%%%%%%%%%%%%%%%%%%%%%%%%%%%%%%%%
%% section 2.2 %%%%%%%%%%%%%%%%%%%%%%%%%%%%%%%%%%%%%%%%%%%
%%%%%%%%%%%%%%%%%%%%%%%%%%%%%%%%%%%%%%%%%%%%%%%%%%%%%%%%%%
\modHeadsection{ボトム側の端面}


%%%%%%%%%%%%%%%%%%%%%%%%%%%%%%%%%%%%%%%%%%%%%%%%%%%%%%%%%%
%% subsection 2.2.1 %%%%%%%%%%%%%%%%%%%%%%%%%%%%%%%%%%%%%%
%%%%%%%%%%%%%%%%%%%%%%%%%%%%%%%%%%%%%%%%%%%%%%%%%%%%%%%%%%
\subsection{ボトム端面における湾曲中心の位置}

%%%%%%%%%%%%%%%%%%%%%%%%%%%%%%%%%%%%%%%%%%%%%%%%%%%%%%%%%%
%% subsubsection 2.2.1.1 %%%%%%%%%%%%%%%%%%%%%%%%%%%%%%%%%
%%%%%%%%%%%%%%%%%%%%%%%%%%%%%%%%%%%%%%%%%%%%%%%%%%%%%%%%%%
\subsubsection{スペーサを用いた場合のB\texorpdfstring{$_{R_\mathrm c}'$}{Rc'}}
\index{スペーサ}スペーサ取付後の\index{ボトムたんのわんきょくちゅうしん@ボトム端の湾曲中心}ボトム端面における湾曲中心B$_{R_\mathrm c}'$と、\index{テーブルちゅうしん@テーブル中心}テーブル中心Pとの$X$方向の差は、トップ側の場合と同様に考えて
%% footnote %%%%%%%%%%%%%%%%%%%%%
\footnote{この場合は、ボトム側が工具側に向いている。}、
%%%%%%%%%%%%%%%%%%%%%%%%%%%%%%%%%
\begin{align*}
%  \label{eq:spacerBRc}
  \Delta-\sqrt{R_\mathrm c^2-f_\mathrm B^2}-\frac{\delta_\mathrm s}2
  +\sqrt{R_\mathrm i'^2-\frac{\delta_\mathrm s^2+(2\bar l)^2}4}\frac{2\bar l}{\sqrt{\delta_\mathrm s^2+(2\bar l)^2}}\ .
\end{align*}

%%%%%%%%%%%%%%%%%%%%%%%%%%%%%%%%%%%%%%%%%%%%%%%%%%%%%%%%%%
%% subsubsection 2.2.1.2 %%%%%%%%%%%%%%%%%%%%%%%%%%%%%%%%%
%%%%%%%%%%%%%%%%%%%%%%%%%%%%%%%%%%%%%%%%%%%%%%%%%%%%%%%%%%
\subsubsection{テーブルを傾けた場合のB\texorpdfstring{$_{R_\mathrm c}'$}{Rc'}}
テーブルを傾けた後のボトム端面における\index{わんきょくちゅうしん@湾曲中心}湾曲中心の位置B$_{R_\mathrm c}'$と、\index{テーブルちゅうしん@テーブル中心}テーブル中心Pとの$X$方向の差は、トップ側の場合と同様に考えて
\begin{align*}
  %\label{eq:tableBRc}
  \left(\Delta+\sqrt{R_i'^2-\bar l^2}\right)\cos\theta-\sqrt{R_\mathrm c^2-f_\mathrm B^2}~.
\end{align*}


%%%%%%%%%%%%%%%%%%%%%%%%%%%%%%%%%%%%%%%%%%%%%%%%%%%%%%%%%%
%% subsection 2.2.2 %%%%%%%%%%%%%%%%%%%%%%%%%%%%%%%%%%%%%%
%%%%%%%%%%%%%%%%%%%%%%%%%%%%%%%%%%%%%%%%%%%%%%%%%%%%%%%%%%
\subsection{ボトム端面における外側中心の位置}
\index{ボトムたんのわんきょくちゅうしん@ボトム端の湾曲中心}ボトム端における湾曲中心B$_{R_\mathrm c}'$と\index{ボトムたんのがいけいちゅうしん@ボトム端の外径中心}外径中心B$_\mathrm c'$との差は、以下で与えられる。
%% label{eq:BRc-Bc}
\begin{align}
  \label{eq:BRc-Bc}
  \sqrt{R_\mathrm c^2-f_\mathrm B^2}
  -\frac{\sqrt{R_\mathrm o^2-f_\mathrm B^2}+\sqrt{R_\mathrm i^2-f_\mathrm B^2}}2\ .
\end{align}
よって、\index{がいけいちゅうしん@外径中心}外径中心B$_\mathrm c'$の位置は、湾曲中心B$_{R_\mathrm c}'$から\pageeqref{eq:BRc-Bc}だけ加味すればよい。
以下では、外径中心B$_\mathrm c'$の位置を直接計算し、このことが整合していることを確かめる。

%%%%%%%%%%%%%%%%%%%%%%%%%%%%%%%%%%%%%%%%%%%%%%%%%%%%%%%%%%
%% subsubsection 2.2.2.1 %%%%%%%%%%%%%%%%%%%%%%%%%%%%%%%%%
%%%%%%%%%%%%%%%%%%%%%%%%%%%%%%%%%%%%%%%%%%%%%%%%%%%%%%%%%%
\subsubsection{スペーサを用いた場合のB\texorpdfstring{$_\mathrm c'$}{c'}}
外面A・C面側のボトム端点B$_{R_\mathrm o}'$, B$_{R_\mathrm i}'$の、\index{テーブルちゅうしん@テーブル中心}テーブル中心Pを原点とした場合の$X$座標はそれぞれ、
\begin{align*}
  \text{C側端点:}&~~
  \Delta-\sqrt{R_\mathrm i^2-f_\mathrm B^2}-\frac{\delta_\mathrm s}2+\sqrt{R_\mathrm i'^2-\frac{\delta_\mathrm s^2+(2\bar l)^2}4}\frac{2\bar l}{\sqrt{\delta_\mathrm s^2+(2\bar l)^2}}\ ,\\
  \text{A側端点:}&~~
  \Delta-\sqrt{R_\mathrm o^2-f_\mathrm B^2}-\frac{\delta_\mathrm s}2+\sqrt{R_\mathrm i'^2-\frac{\delta_\mathrm s^2+(2\bar l)^2}4}\frac{2\bar l}{\sqrt{\delta_\mathrm s^2+(2\bar l^2}}\ .
\end{align*}
したがって、ボトム端における(AC外径の)中点B$_\mathrm c'$の$X$座標は、
%% label{eq:spacerBc}
\begin{align}
  \label{eq:spacerBc}
  \Delta-\frac{\sqrt{R_\mathrm o^2-f_\mathrm B^2}+\sqrt{R_\mathrm i^2-f_\mathrm B^2}}2
  -\frac{\delta_\mathrm s}2+\sqrt{R_\mathrm i'^2-\frac{\delta_\mathrm s^2+(2\bar l)^2}4}\frac{2\bar l}{\sqrt{\delta_\mathrm s^2+(2\bar l)^2}}\ .
\end{align}
これより、\index{わんきょくちゅうしん@湾曲中心}湾曲中心B$_{R_\mathrm c}'$と\index{がいけいちゅうしん@外径中心}外径中心B$_\mathrm c'$との差は\pageeqref{eq:BRc-Bc}となることがわかる。

\clearpage
%%%%%%%%%%%%%%%%%%%%%%%%%%%%%%%%%%%%%%%%%%%%%%%%%%%%%%%%%%
%% subsubsection 2.2.2.2 %%%%%%%%%%%%%%%%%%%%%%%%%%%%%%%%%
%%%%%%%%%%%%%%%%%%%%%%%%%%%%%%%%%%%%%%%%%%%%%%%%%%%%%%%%%%
\subsubsection{テーブルを傾けた場合のB\texorpdfstring{$_\mathrm c'$}{c'}}
外面A・C面側のボトム端点B$_{R_\mathrm o}'$, B$_{R_\mathrm i}'$の、\index{テーブルちゅうしん@テーブル中心}テーブル中心Pを原点とした場合の$X$座標はそれぞれ、
\begin{subequations}
\begin{align}
  \label{eq:tableBRi}
  \text{C側端点:}&~~
  \Delta'\cos\theta-\sqrt{R_\mathrm i^2-f_\mathrm B^2}\ ,\\
  \text{A側端点:}&~~
  \Delta'\cos\theta-\sqrt{R_\mathrm o^2-f_\mathrm B^2}\ .
\end{align}
\end{subequations}
したがって、ボトム端における(AC外径の)中点B$_\mathrm c'$の$X$座標は、
%% label{eq:tableBc}
\begin{align}
  \label{eq:tableBc}
  \Delta'\cos\theta-\frac{\sqrt{R_\mathrm o^2-f_\mathrm B^2}+\sqrt{R_\mathrm i^2-f_\mathrm B^2}}2
\end{align}
これより、\index{わんきょくちゅうしん@湾曲中心}湾曲中心B$_{R_\mathrm c}'$と\index{がいけいちゅうしん@外径中心}外径中心B$_\mathrm c'$との差は\pageeqref{eq:BRc-Bc}となることがわかる。
\vfill
%%%%%%%%%%%%%%%%%%%%%%%%%%%%%%%%%%%%%%%%%%%%%%%%%%%%%%%%%%
%% Column %%%%%%%%%%%%%%%%%%%%%%%%%%%%%%%%%%%%%%%%%%%%%%%%
%%%%%%%%%%%%%%%%%%%%%%%%%%%%%%%%%%%%%%%%%%%%%%%%%%%%%%%%%%
\begin{Column}{端面における外径の近似計算}
\index{がいけい@外径}外径を$W_x$に対して、トップ端面部の水平方向の長さ$W_\mathrm T$は以下で与えられる。(ボトム端面部も同様)
\begin{align*}
  W_\mathrm T
  = \sqrt{\left(R+\frac{W_x}2\right)^2-f_\mathrm T^2}
    -\sqrt{\left(R-\frac{W_x}2\right)^2-f_\mathrm T^2}\ .
\end{align*}
\index{テイラーてんかい@テイラー展開}テイラー展開(\index{マクローリンてんかい@マクローリン展開}マクローリン展開)\pageautoref{formula:taylorexpansion}より、
\begin{align*}
  (1+x)^\frac12 = 1+\frac x2-\frac{x^2}8+\frac{x^3}{16}-\frac{5x^4}{128}+o\left(x^5\right)
\end{align*}
なので、
\begin{align*}
  & (1+x)^\frac12(1+y)^\frac12-(1-x)^\frac12(1-y)^\frac12\\
  &= x+y+\frac{(x+y)(x-y)^2}8-\frac{xy(x+y)\big\{5(x-y)^2+7xy\big\}}{128}+\cdots\ .
\end{align*}
したがって、
\begin{align*}
  x = \frac{\nicefrac{W_x}2+f_\mathrm T}R\ ,\quad y = \frac{\nicefrac{W_x}2-f_\mathrm T}R\quad
  \longrightarrow \quad
  x+y = \frac{W_x}R\ , \quad x-y = \frac{2f_\mathrm T}R
\end{align*}
とすると、
\begin{align*}
  W_\mathrm T
  = R\left\{(1+x)^\frac12(1+y)^\frac12-(1-x)^\frac12(1-y)^\frac12\right\}
  = W_x\left(1+\frac{f_\mathrm T^2}{2R^2}+\cdots\right)
\end{align*}
と近似できる。
また$W_\mathrm T > W_x$であり、$R\to\infty$ ($R^{-1}\to0$)のとき$W_\mathrm T = W_x$であることもわかる。
\end{Column}
%%%%%%%%%%%%%%%%%%%%%%%%%%%%%%%%%%%%%%%%%%%%%%%%%%%%%%%%%%
%%%%%%%%%%%%%%%%%%%%%%%%%%%%%%%%%%%%%%%%%%%%%%%%%%%%%%%%%%
%%%%%%%%%%%%%%%%%%%%%%%%%%%%%%%%%%%%%%%%%%%%%%%%%%%%%%%%%%



\clearpage
%%%%%%%%%%%%%%%%%%%%%%%%%%%%%%%%%%%%%%%%%%%%%%%%%%%%%%%%%%
%% section 22.3 %%%%%%%%%%%%%%%%%%%%%%%%%%%%%%%%%%%%%%%%%%
%%%%%%%%%%%%%%%%%%%%%%%%%%%%%%%%%%%%%%%%%%%%%%%%%%%%%%%%%%
\modHeadsection{端面加工の工具径補正}
端面の加工として、$X+$, $Y+$方向の角から始めて
%% footnote %%%%%%%%%%%%%%%%%%%%%
\footnote{\DMC の場合、\index{こうぐこうかんいち@工具交換位置}工具交換位置に近いので、このほうが移動距離が短くなる。}、
%%%%%%%%%%%%%%%%%%%%%%%%%%%%%%%%%
(工具から見て)左回りに加工する場合を考える。
このとき\index{かこうのけいろ@加工の経路}加工の経路として、単純に外径の値を指定すれば加工することは可能である。
しかしその場合、\index{こうぐ@工具}工具(\index{フェイスミル}フェイスミル)のほぼ中心に近い位置で切削する形になるので、工具に大きな負荷がかかることになる。
これを避けるために、理想的には、\index{ないけい@内径}内径$w_{\mathrm T, \mathrm B}$の外側に工具が沿う形で切削するのが望ましい。
つまり、端面の内径$w_y$\,(BD方向の$w_{\mathrm T}$または$w_{\mathrm B}$)を基準として、\index{こうぐはんけい@工具半径}工具半径分だけ(進行方向に対して右側に)補正をする形にすればよい。
ここでは誤差等を考慮して、内径から$\delta_w$だけ縮めた輪郭(の外側)に沿う形を考える。


%%%%%%%%%%%%%%%%%%%%%%%%%%%%%%%%%%%%%%%%%%%%%%%%%%%%%%%%%%
%% subsection 22.3.1 %%%%%%%%%%%%%%%%%%%%%%%%%%%%%%%%%%%%%
%%%%%%%%%%%%%%%%%%%%%%%%%%%%%%%%%%%%%%%%%%%%%%%%%%%%%%%%%%
\subsection{加工の開始可能範囲}
初めの位置は$X+$, $Y+$方向の角の右方向($X+$方向)に工具があるものとする。
工具の\index{はけい(フェイスミル)@刃径(フェイスミル)}刃径(直径)を$\phi_\mathrm D$, \index{さいだいはけい(フェイスミル)@最大刃径(フェイスミル)}最大刃径(直径)$\phi'_\mathrm D$とすると
%% footnote %%%%%%%%%%%%%%%%%%%%%
\footnote{通常、刃径は\index{DC(フェイスミルはけい)@DC(フェイスミル刃径)}DC、最大刃径は\index{DCX(フェイスミルさいだいはけい)@DCX(フェイスミル最大刃径)}DCXと表記され、それぞれ直径として与えられることが多い。}、
%%%%%%%%%%%%%%%%%%%%%%%%%%%%%%%%%
$Y$位置については、工具の中心が
\begin{align}
  \label{eq:tanmenKakouStartY}
  \frac{w_y}2-\delta_w+\frac{\phi_\mathrm D}2
\end{align}
にあればよい。
そのためここでは、まず$Y$方向に\index{ぜったいざひょう@絶対座標}絶対座標(\verb|G90|)
\begin{align*}
  \frac{w_y}2-\delta_w
\end{align*}
まで移動し、その後に工具半径分の補正量として$\nicefrac{\phi_\mathrm D}2$だけ$Y+$方向にずらす形で、左方向($X-$方向)に移動する場合を考える。


%%%%%%%%%%%%%%%%%%%%%%%%%%%%%%%%%%%%%%%%%%%%%%%%%%%%%%%%%%
%% subsection 20.3.1 %%%%%%%%%%%%%%%%%%%%%%%%%%%%%%%%%%%%%
%%%%%%%%%%%%%%%%%%%%%%%%%%%%%%%%%%%%%%%%%%%%%%%%%%%%%%%%%%
\subsection{工具径補正を用いる場合}
\index{こうぐけいほせい@工具径補正}工具径補正の機能\verb|G42|を用いる場合を考える。
このとき、動き始めは$Y$方向の補正分も加えて斜めに移動することになる。
ここで、
\begin{enumerate}
\item $X$, $Y$方向には同じ速さで動く
\item $Y$方向の移動がなくなるまで工具はモールドに触れない
\end{enumerate}
とすると、加工(移動)の開始位置の$X$座標は、
\begin{align*}
  \frac{W_x}2+\frac{\phi'_\mathrm D+\phi_\mathrm D}2
  = \frac{w_x}2+\tau_x+\frac{\phi'_\mathrm D+\phi_\mathrm D}2
\end{align*}
より右方向($X+$方向)であればよい。
なお、$W_x$, $\tau_x$はAC方向の\index{がいけい@外径}外径およびBD方向の\index{にくあつ@肉厚}肉厚である
%% footnote %%%%%%%%%%%%%%%%%%%%%
\footnote{ここでは話を単純化し、\PlatingThk や\index{へんにく@偏肉}偏肉は無視している。}。
%%%%%%%%%%%%%%%%%%%%%%%%%%%%%%%%%


\clearpage
%%%%%%%%%%%%%%%%%%%%%%%%%%%%%%%%%%%%%%%%%%%%%%%%%%%%%%%%%%
%% subsection 2.30.2 %%%%%%%%%%%%%%%%%%%%%%%%%%%%%%%%%%%%%
%%%%%%%%%%%%%%%%%%%%%%%%%%%%%%%%%%%%%%%%%%%%%%%%%%%%%%%%%%
\subsection{手動で補正を行う場合}
手動で補正する場合は、予め$Y$位置を\pageeqref{eq:tanmenKakouStartY}に移動しておいて、そのまま左方向($X-$方向)に移動すればよい。
よって、加工(移動)の開始位置の$X$座標は、
\begin{align*}
  \frac{W_x+\phi_\mathrm D}2 = \frac{w_x+\phi_\mathrm D}2+\tau_x
\end{align*}
より右方向($X+$方向)にあればよい
%% footnote %%%%%%%%%%%%%%%%%%%%%
\footnote{実際のプログラムでは、安全を考慮して$\nicefrac{w_x}2+\phi'_\mathrm D$としている。
この場合、
\begin{align*}
  \phi'_\mathrm D > \frac{\phi_\mathrm D}2+\tau_x
\end{align*}
である限り、衝突は生じないことになる。
一般に、$\tau_x < \nicefrac{\phi_\mathrm D}2$であるので、これは常に満たされる。}。
%%%%%%%%%%%%%%%%%%%%%%%%%%%%%%%%%
