%!TEX root = ../RPA_for_Creating_Program_Note.tex



ここでは主に、\index{テーパエンドミル}テーパエンドミルを用いた、端面外側および内側の\index{Cめんとり@C面取}\textbf{C面取}について考える。


%%%%%%%%%%%%%%%%%%%%%%%%%%%%%%%%%%%%%%%%%%%%%%%%%%%%%%%%%%
%% section 5.1 %%%%%%%%%%%%%%%%%%%%%%%%%%%%%%%%%%%%%%%%%%%
%%%%%%%%%%%%%%%%%%%%%%%%%%%%%%%%%%%%%%%%%%%%%%%%%%%%%%%%%%
\modHeadsection{C面取の寸法}
トップ端面における\index{そとがわCめんとり@外側C面取}外側C面取および\index{うちがわCめんとり@内側C面取}内側C面取の大きさをそれぞれ$c_\mathrm{To}$, $c_\mathrm{Ti}$とする。
ここで$c_\mathrm{To}$や$c_\mathrm{Ti}$は、端面に垂直な方向の距離とする。
このとき、片角$\xi_\mathrm e$\,($>0$)の\index{テーパエンドミル}テーパエンドミルに対して、C面取の$XY$方向の距離は、
\begin{align*}
  c_\mathrm{To}\tan\xi_\mathrm e\qquad\text{および}\qquad c_\mathrm{Ti}\tan\xi_\mathrm e
\end{align*}
で与えられる。
ボトム端面における外側C面取および内側C面取の大きさ$c_\mathrm{Bo}$, $c_\mathrm{Bi}$についても同様である。


%%%%%%%%%%%%%%%%%%%%%%%%%%%%%%%%%%%%%%%%%%%%%%%%%%%%%%%%%%
%% section 5.2 %%%%%%%%%%%%%%%%%%%%%%%%%%%%%%%%%%%%%%%%%%%
%%%%%%%%%%%%%%%%%%%%%%%%%%%%%%%%%%%%%%%%%%%%%%%%%%%%%%%%%%
\modHeadsection{工具の参照直径}
\index{テーパエンドミル}テーパエンドミルは、その名の通り\index{テーパ}テーパの付いた工具であり、先端が平坦になっているものも多い。
しかし、先端部分を\index{こうぐちょう@工具長}工具長として設定すると、その部分が段差となり\index{テーパかこう@テーパ加工}テーパ加工を適切に行うことができない。
そのため、先端部分から一定の距離$d_\mathrm e$だけずらした箇所を\index{こうぐちょう@工具長}工具長として設定し、またその箇所の直径(\index{さんしょうちょっけい@参照直径}\textbf{参照直径})$D_\mathrm r$を工具直径として補正を行うことが推奨される。

ここで、先端径(直径)
%% footnote %%%%%%%%%%%%%%%%%%%%%
\footnote{先端が平坦でなく尖っている場合は$D_\mathrm e = 0$とする。}
%%%%%%%%%%%%%%%%%%%%%%%%%%%%%%%%%
およびテーパの角度(片角)を$D_\mathrm e$, $\xi_\mathrm e$とすると、参照直径$D_\mathrm r$は
\begin{align*}
  D_\mathrm r = D_\mathrm e+2d_\mathrm e\tan\xi_\mathrm e
\end{align*}
で与えられる。
通常、\index{こうぐけいほせい@工具径補正}工具径補正は工具の半径を用いて行うので、工具径を
\begin{align*}
  \frac{D_\mathrm r}2 = \frac{D_\mathrm e}2+d_\mathrm e\tan\xi_\mathrm e
\end{align*}
として設定すればよい。
あるいは、先端径を基準としてそこから補正を行う形にする場合は、その差
\begin{align*}
  \frac{D_\mathrm r}2-\frac{D_\mathrm e}2 = d_\mathrm e\tan\xi_\mathrm e
\end{align*}
だけ補正すればよい。
%%%%%%%%%%%%%%%%%%%%%%%%%%%%%%%%%%%%%%%%%%%%%%%%%%%%%%%%%%
%% hosoku %%%%%%%%%%%%%%%%%%%%%%%%%%%%%%%%%%%%%%%%%%%%%%%%
%%%%%%%%%%%%%%%%%%%%%%%%%%%%%%%%%%%%%%%%%%%%%%%%%%%%%%%%%%
\begin{hosoku}
なお、$\xi_\mathrm e = \nicefrac\pi{12}$\,($15^\circ$), $\nicefrac\pi6$\,($30^\circ$), $\nicefrac\pi4$\,($45^\circ$)のとき、それぞれ
\begin{align*}
  \tan\frac\pi{12} = 2-\sqrt3\ , \quad
  \tan\frac\pi6 = \frac1{\sqrt3}\ , \quad
  \tan\frac\pi4 = 1\ .
\end{align*}
\end{hosoku}
%%%%%%%%%%%%%%%%%%%%%%%%%%%%%%%%%%%%%%%%%%%%%%%%%%%%%%%%%%
%%%%%%%%%%%%%%%%%%%%%%%%%%%%%%%%%%%%%%%%%%%%%%%%%%%%%%%%%%
%%%%%%%%%%%%%%%%%%%%%%%%%%%%%%%%%%%%%%%%%%%%%%%%%%%%%%%%%%


\clearpage
%%%%%%%%%%%%%%%%%%%%%%%%%%%%%%%%%%%%%%%%%%%%%%%%%%%%%%%%%%
%% section 24.2 %%%%%%%%%%%%%%%%%%%%%%%%%%%%%%%%%%%%%%%%%%
%%%%%%%%%%%%%%%%%%%%%%%%%%%%%%%%%%%%%%%%%%%%%%%%%%%%%%%%%%
\modHeadsection[中心座標\texorpdfstring{$X$}{X}の移動]{中心座標$X$の移動}
\index{がいさく@外削}外削があり、かつ外側の\index{Cめんとり@C面取}C面取の場合であれば、Cの大きさに依らず加工径の中心座標($XY$)は変わらない。
一方、外削のない場合は、中心の$X$座標はCの大きさに応じて湾曲中心線に沿って移動する。
このとき\index{たんめん@端面}端面と面取先端部との中心座標($X$)の差
%% footnote %%%%%%%%%%%%%%%%%%%%%
\footnote{どちらの場合も端面が工具側にある場合を考えている。}
%%%%%%%%%%%%%%%%%%%%%%%%%%%%%%%%%
は、
\begin{align*}
  \text{トップ側:}&~~
  \sqrt{R_\mathrm c^2-\left(f_\mathrm T-c_\mathrm{To}\right)^2}-\sqrt{R_\mathrm c^2-f_\mathrm T^2}\ ,\\
  \text{ボトム側:}&~~
  \sqrt{R_\mathrm c^2-f_\mathrm B^2}-\sqrt{R_\mathrm c^2-\left(f_\mathrm B-c_\mathrm{Bo}\right)^2}\ .
\end{align*}
これは内側C面取でも同様であり、
\begin{align*}
  \text{トップ側:}&~~
  \sqrt{R_\mathrm c^2-\left(f_\mathrm T-c_\mathrm{Ti}\right)^2}-\sqrt{R_\mathrm c^2-f_\mathrm T^2}\ ,\\
  \text{ボトム側:}&~~
  \sqrt{R_\mathrm c^2-f_\mathrm B^2}-\sqrt{R_\mathrm c^2-\left(f_\mathrm B-c_\mathrm{Bi}\right)^2}\ .
\end{align*}



