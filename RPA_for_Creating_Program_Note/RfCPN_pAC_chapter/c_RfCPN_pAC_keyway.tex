%!TEX root = ../RPA_for_Creating_Program_Note.tex



\index{みぞ@溝}溝に関しては、その基準が以下のように与えられる場合が考えられる。
\begin{enumerate}
\item \index{みぞけい@溝径}溝径の中心Mが、モールドの\index{わんきょくちゅうしんせん@湾曲中心線}湾曲中心線上にある場合
\item \index{みぞちゅうしん@溝中心}溝径の中心Mが、(トップ側)\index{がいさくちゅうしん@外削中心}外削径の中心線上にある場合
\item \index{Aがわみぞふかさ@A側溝深さ}A面側の\index{みぞふかさ@溝深さ}溝深さに指定がある場合
\end{enumerate}
なお、溝径を$W_\mathrm M$, \index{みぞいち@溝位置}溝位置(端面から溝までの長さ)・\index{みぞはば@溝幅}溝幅・A側溝深さをそれぞれ$\kappa_p$, $\kappa_w$, $\kappa_d$とする。
このときいずれの場合も、$y$方向(機内における$Z$方向
%% footnote %%%%%%%%%%%%%%%%%%%%%
\footnote{計算上の$xy$座標($x$:実軸, $y$:虚軸)と、機内における$XZ$座標とが混在する形で話を進めているので注意されたし。})
%%%%%%%%%%%%%%%%%%%%%%%%%%%%%%%%%
の\index{せっさくはんい@切削範囲}切削範囲は、テーブル中心Pを\index{げんてん@原点}原点として、
\begin{align*}
  \big[f_\mathrm T'-(\kappa_p+\kappa_w)\ ,\ f_\mathrm T'-\kappa_p\big]
\end{align*}
であり、また溝中心M$'$の$y$座標($Z$座標)はこの切削範囲の中央
%% label{eq:mizocenterZ}
\begin{align}
  \label{eq:mizocenterZ}
  f_\mathrm T'-\left(\kappa_p+\frac{\kappa_w}2\right)
\end{align}
で与えられる。



%%%%%%%%%%%%%%%%%%%%%%%%%%%%%%%%%%%%%%%%%%%%%%%%%%%%%%%%%%
%% section 4.1 %%%%%%%%%%%%%%%%%%%%%%%%%%%%%%%%%%%%%%%%%%%
%%%%%%%%%%%%%%%%%%%%%%%%%%%%%%%%%%%%%%%%%%%%%%%%%%%%%%%%%%
\modHeadsection{湾曲中心が基準の場合}
トップ端における湾曲中心T$_{R_\mathrm c}'$と溝中心M$'$との$x$方向の差は、
%% label{eq:difTopMizoCenter}
\begin{align}
  \label{eq:difTopMizoCenter}
  \sqrt{R_\mathrm c^2-\left(f_\mathrm T-\kappa_p-\frac{\kappa_w}2\right)^{\!2}}
  -\sqrt{R_\mathrm c^2-f_\mathrm T^2}
\end{align}
で与えられる。


%%%%%%%%%%%%%%%%%%%%%%%%%%%%%%%%%%%%%%%%%%%%%%%%%%%%%%%%%%
%% subsubsection 4.1.1 %%%%%%%%%%%%%%%%%%%%%%%%%%%%%%%%%%%
%%%%%%%%%%%%%%%%%%%%%%%%%%%%%%%%%%%%%%%%%%%%%%%%%%%%%%%%%%
\subsection{スペーサを用いた場合の溝中心(湾曲中心基準)}
溝中心M$'$がモールドの湾曲中心線上にある場合、テーブル中心Pを原点とした$x$座標は、\pageeqref{eq:spacerTRc}より、
\begin{align*}
  -\varDelta+\sqrt{R_\mathrm c^2-\left(f_\mathrm T-\kappa_p-\frac{\kappa_w}2\right)^{\!2}}+\frac\delta2
  -\sqrt{R_\mathrm i'^2-\frac{\delta^2+(2\bar l)^2}4}\frac{2\bar l}{\sqrt{\delta^2+\left(2\bar l\right)^2}}
\end{align*}
となる。
なお実際の作業では、簡単のため、トップ端面の外側中心T$_\mathrm c'$を測定し、それをトップ端面における湾曲中心T$_{R_\mathrm c}'$とみなして溝中心M$'$の位置を計算することが多い。
実測した外側中心の$X$・$Y$座標$G_{\mathrm Tx}$, $G_{\mathrm Ty}$を湾曲中心のそれとみなすと、機内における溝中心M$'$の位置は、テーブル中心Pを原点として、
%% label{eq:Mreal}
\begin{subequations}
  \label{eq:Mreal}
\begin{align}
  \left(
    G_{\mathrm Tx}
    +\sqrt{R_\mathrm c^2-\left(f_\mathrm T-\kappa_p-\frac{\kappa_w}2\right)^{\!2}}
    -\sqrt{R_\mathrm c^2-f_\mathrm T^2}\ ,\
    G_{\mathrm Ty}~,~
    f_\mathrm T'-\kappa_p-\frac{\kappa_w}2
  \right).
\end{align}
湾曲中心とみなさずに正確に求めるなら、これに\pageeqref{eq:TRc-Tc}を引けばよい。
その場合の$X$座標は、
\begin{align}
  G_{\mathrm Tx}
  +\sqrt{R_\mathrm c^2-\left(f_\mathrm T-\kappa_p-\frac{\kappa_w}2\right)^{\!2}}
  -\frac{\sqrt{R_\mathrm o^2-f_\mathrm T^2}+\sqrt{R_\mathrm i^2-f_\mathrm T^2}}2\ .
\end{align}
\end{subequations}


%%%%%%%%%%%%%%%%%%%%%%%%%%%%%%%%%%%%%%%%%%%%%%%%%%%%%%%%%%
%% subsubsection 4.1.2 %%%%%%%%%%%%%%%%%%%%%%%%%%%%%%%%%%%
%%%%%%%%%%%%%%%%%%%%%%%%%%%%%%%%%%%%%%%%%%%%%%%%%%%%%%%%%%
\subsection{テーブルを傾けた場合の溝中心(湾曲中心基準)}
\index{みぞちゅうしん@溝中心}溝中心M$'$がモールドの湾曲中心線上にある場合、テーブル中心Pを原点とした$x$座標は、\pageeqref{eq:tableTRc}より、
\begin{align*}
  \sqrt{R_\mathrm c^2-\left(f_\mathrm T-\kappa_p-\frac{\kappa_w}2\right)^{\!2}}
  -\varDelta'\cos\theta\ .
\end{align*}
実測した\index{そとがわちゅうしん@外側中心}外側中心の$X$・$Y$座標$G_{\mathrm Tx}$, $G_{\mathrm Ty}$を\index{わんきょくちゅうしん@湾曲中心}湾曲中心のそれとみなした場合とそうでない場合は、\pageeqref{eq:Mreal}で与えられる。




%\clearpage
%%%%%%%%%%%%%%%%%%%%%%%%%%%%%%%%%%%%%%%%%%%%%%%%%%%%%%%%%%
%% section 4.2 %%%%%%%%%%%%%%%%%%%%%%%%%%%%%%%%%%%%%%%%%%%
%%%%%%%%%%%%%%%%%%%%%%%%%%%%%%%%%%%%%%%%%%%%%%%%%%%%%%%%%%
\modHeadsection{外削径の中心が基準の場合}
\index{みぞちゅうしん@溝中心}溝中心M$'$がトップ外削径の中心線上にある場合、機内におけるその位置座標は、
\begin{align*}
  \left(
    -\mathcal G_{Bx}+T_x\ ,\
    \mathcal G_{By}\ ,\
    f_\mathrm T'-\kappa_p-\frac{\kappa_w}2
  \right) \qquad
  \text{または}\qquad
  \left(
    \mathcal G_{Bx}\ ,\
    \mathcal G_{By}\ ,\
    f_\mathrm T'-\kappa_p-\frac{\kappa_w}2
  \right).
\end{align*}
ただし、前者はボトムの外削を基準にした(ボトム基準の\index{とおりしん@通り芯}通り芯がある)場合であり、後者はトップの外削を基準にした場合である。



\clearpage
%%%%%%%%%%%%%%%%%%%%%%%%%%%%%%%%%%%%%%%%%%%%%%%%%%%%%%%%%%
%% section 4.3 %%%%%%%%%%%%%%%%%%%%%%%%%%%%%%%%%%%%%%%%%%%
%%%%%%%%%%%%%%%%%%%%%%%%%%%%%%%%%%%%%%%%%%%%%%%%%%%%%%%%%%
\modHeadsection{A面側の溝深さが基準の場合}


%%%%%%%%%%%%%%%%%%%%%%%%%%%%%%%%%%%%%%%%%%%%%%%%%%%%%%%%%%
%% subsection 4.3.1 %%%%%%%%%%%%%%%%%%%%%%%%%%%%%%%%%%%%%%
%%%%%%%%%%%%%%%%%%%%%%%%%%%%%%%%%%%%%%%%%%%%%%%%%%%%%%%%%%
\subsection{外削のない場合}
トップ側に外削がなく、\index{Aがわみぞふかさ@A側溝深さ}A側面からの溝深さが指定されている場合を考える。
\index{みぞちゅうしん@溝中心}溝中心の位置の$X$座標は、テーブル中心Pを\index{げんてん@原点}原点として、
%% label{eq:mizocenterA}
\begin{align}
  \label{eq:mizocenterA}
  \sqrt{R_\mathrm o^2-\left(f_\mathrm T-\kappa_p-\frac{\kappa_w}2\right)^{\!2}}-\kappa_d-\frac{W_{mx}}2
  -\varDelta'
\end{align}
で与えられる。
ここで、$W_{mx}$は溝のAC方向の径を表す。
なお実際の作業では、モールドの\index{Aがわがいめん@A側外面}A側外面の\index{みぞはばちゅうおう@溝幅中央}溝幅中央に相当する箇所を直接計測し、その位置を基準として原点を割り出す。
トップ端面における中心の$X$座標(\index{じっそくち@実測値}実測値)$G_{\mathrm Tx}$がわかっている場合、\index{みぞちゅうしん@溝中心}溝の中心の$X$座標$G_{mx}$は\pageeqref{eq:Mreal}で与えられ、また\index{みぞはばちゅうおう@溝幅中央}溝幅中央に対する\index{Aがわがいめん@A側外面}A側外面と$G_{\mathrm Tx}$との差(の絶対値)は、
%% label{eq:mizocenterA}
\begin{align}
  \label{eq:mizocenterAd}
  \frac{W_{mx}}2+\kappa_d
  +\sqrt{R_\mathrm c^2-\left(f_\mathrm T-\kappa_p-\frac{\kappa_w}2\right)^{\!2}}
  -\frac{\sqrt{R_\mathrm o^2-f_\mathrm T^2}+\sqrt{R_\mathrm i^2-f_\mathrm T^2}}2\ .
\end{align}


%%%%%%%%%%%%%%%%%%%%%%%%%%%%%%%%%%%%%%%%%%%%%%%%%%%%%%%%%%
%% subsection 4.3.2 %%%%%%%%%%%%%%%%%%%%%%%%%%%%%%%%%%%%%%
%%%%%%%%%%%%%%%%%%%%%%%%%%%%%%%%%%%%%%%%%%%%%%%%%%%%%%%%%%
\subsection{外削のある場合}
トップ側に外削があり、かつ\index{Aがわみぞふかさ@A側溝深さ}A側溝深さが指定されている場合を考える。
トップ側の外削中心の実測値を$\mathcal G_{\mathrm Tx}$とすると、\index{みぞちゅうしん@溝中心}溝中心$G_{mx}$との差($G_{mx}-\mathcal G_{\mathrm Tx}$)は
%% label{eq:mizocenterAG}
\begin{align}
  \label{eq:mizocenterAG}
  \frac{\mathfrak W_x}2-\kappa_d-\frac{W_{mx}}2\ .
\end{align}



\clearpage
%%%%%%%%%%%%%%%%%%%%%%%%%%%%%%%%%%%%%%%%%%%%%%%%%%%%%%%%%%
%% section 4.4 %%%%%%%%%%%%%%%%%%%%%%%%%%%%%%%%%%%%%%%%%%%
%%%%%%%%%%%%%%%%%%%%%%%%%%%%%%%%%%%%%%%%%%%%%%%%%%%%%%%%%%
\modHeadsection{測定上の溝深さ}
ここでは特に\index{Aがわみぞふかさ@A側溝深さ}A面側溝深さ$\kappa_d$に限って話を進める。
トップ側に外削がある場合、$\kappa_d$は\index{Aがわがいさくめん@A側外削面}A側外削面と溝のA側面との差で与えられる。
しかし外削がない場合、つまり\index{Aがわがいめん@A側外面}A側外面に湾曲がある場合は、$\kappa_d$の値は自明ではない。


%%%%%%%%%%%%%%%%%%%%%%%%%%%%%%%%%%%%%%%%%%%%%%%%%%%%%%%%%%
%% subsection 4.4.1 %%%%%%%%%%%%%%%%%%%%%%%%%%%%%%%%%%%%%%
%%%%%%%%%%%%%%%%%%%%%%%%%%%%%%%%%%%%%%%%%%%%%%%%%%%%%%%%%%
\subsection{図面上の溝深さ}
図面上においてA側の溝深さが$\kappa_d$のとき、これは溝幅の中央の位置におけるA側外面から端面と水平な方向に$\kappa_d$という意味で与えられる。
すなわち、溝のA面側の水平方向の位置は、\pageeqref{eq:mizocenterA}より、
\begin{align*}
  \sqrt{R_\mathrm o^2-\left(f_\mathrm T-\kappa_p-\frac{\kappa_w}2\right)^{\!2}}-\kappa_d-\varDelta'\ .
\end{align*}
溝中心の$X$座標$G_{mx}$が与えられている場合は、
\begin{align*}
  G_{mx}+\frac{W_{mx}}2
\end{align*}




%%%%%%%%%%%%%%%%%%%%%%%%%%%%%%%%%%%%%%%%%%%%%%%%%%%%%%%%%%
%% subsection 4.4.2 %%%%%%%%%%%%%%%%%%%%%%%%%%%%%%%%%%%%%%
%%%%%%%%%%%%%%%%%%%%%%%%%%%%%%%%%%%%%%%%%%%%%%%%%%%%%%%%%%
\subsection{測定上の溝深さ\TBW}
通常、実測には\index{マイクロメータ}マイクロメータ(\index{デプスゲージ}デプスゲージ)が用いられる。
このとき、計測は計測器を湾曲に沿った形にして行われる。
したがって、その計測値は\index{みぞはば@溝幅}溝幅の両端に対する外面の傾斜だけ傾いた形で与えられる。
溝幅の両端に対する外面の$XZ$位置は、溝中心の$X$座標を$G_{mx}$として、
\begin{align*}
  \text{トップ側:}&~
  \left(
  G_{mx}+\frac{W_x}2
  -\sqrt{R_\mathrm o^2-\left(f_\mathrm T-\kappa_p-\frac{\kappa_w}2\right)^{\!2}}
  +\sqrt{R_\mathrm o^2-\left(f_\mathrm T-\kappa_p\right)^2}~,~~
  f_\mathrm T-\kappa_p
  \right)\\
  \text{ボトム側:}&~
  \left(
  G_{mx}+\frac{W_x}2
  +\sqrt{R_\mathrm o^2-\left(f_\mathrm T-\kappa_p-\kappa_w\right)^2}
  -\sqrt{R_\mathrm o^2-\left(f_\mathrm T-\kappa_p-\frac{\kappa_w}2\right)^{\!2}}~,~~
  f_\mathrm T-\kappa_p-\kappa_w
  \right)
\end{align*}
またその差は、
\begin{align*}
  \left(
  \sqrt{R_\mathrm o^2-\left(f_\mathrm T-\kappa_p\right)^2}
  -\sqrt{R_\mathrm o^2-\left(f_\mathrm T-\kappa_p-\kappa_w\right)^2}~,~~
  \kappa_w
  \right)
\end{align*}
したがって、計測の際の傾斜の角度$\zeta$ ($> 0$)は
\begin{align*}
  \tan\zeta
  = \frac{\sqrt{R_\mathrm o^2-\left(f_\mathrm T-\kappa_p-\kappa_w\right)^2}
          -\sqrt{R_\mathrm o^2-\left(f_\mathrm T-\kappa_p\right)^2}}
         {\kappa_w}\ .
\end{align*}





