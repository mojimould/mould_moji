%!TEX root = ../RPA_for_Creating_Program_Note.tex



\index{みぞ@溝}\textbf{溝}に関しては、その基準が以下のように与えられる場合が考えられる。
\begin{enumerate}
\item \index{みぞけい@溝径}溝径の中心Mが、モールドの\index{わんきょくちゅうしんせん@湾曲中心線}湾曲中心線上にある場合
\item \index{みぞちゅうしん@溝中心}溝径の中心Mが、(トップ側)\index{がいさくちゅうしん@外削中心}外削径の中心線上にある場合
\item \index{Aがわみぞふかさ@A側溝深さ}A面側の\index{みぞふかさ@溝深さ}溝深さに指定がある場合
\end{enumerate}
なお、溝径を$W_\mathrm M$, \index{みぞいち@溝位置}溝位置(端面から溝までの長さ)・\index{みぞはば@溝幅}溝幅・A側溝深さをそれぞれ$\kappa_p$, $\kappa_w$, $\kappa_d$とする。
このときいずれの場合も、$y$方向(機内における$Z$方向
%% footnote %%%%%%%%%%%%%%%%%%%%%
\footnote{計算上の$xy$座標($x$:実軸, $y$:虚軸)と、機内における$XZ$座標とが混在する形で話を進めているので注意されたし。})
%%%%%%%%%%%%%%%%%%%%%%%%%%%%%%%%%
の\index{せっさくはんい@切削範囲}切削範囲は、テーブル中心Pを\index{げんてん@原点}原点として、
\begin{align*}
  \big[f_\mathrm T'-(\kappa_p+\kappa_w)\ ,\ f_\mathrm T'-\kappa_p\big]
\end{align*}
であり、また溝中心M$'$の$y$座標($Z$座標)はこの切削範囲の中央
%% label{eq:mizocenterZ}
\begin{align}
  \label{eq:mizocenterZ}
  f_\mathrm T'-\bar\kappa_w \qquad
  \left(\bar\kappa_w \equiv \kappa_p-\frac{\kappa_w}2\right)
\end{align}
で与えられる。



%%%%%%%%%%%%%%%%%%%%%%%%%%%%%%%%%%%%%%%%%%%%%%%%%%%%%%%%%%
%% section 4.1 %%%%%%%%%%%%%%%%%%%%%%%%%%%%%%%%%%%%%%%%%%%
%%%%%%%%%%%%%%%%%%%%%%%%%%%%%%%%%%%%%%%%%%%%%%%%%%%%%%%%%%
\modHeadsection{湾曲中心が基準の場合}
トップ端における湾曲中心T$_{R_\mathrm c}'$と溝中心M$'$との$x$方向の差は、
%% label{eq:difTopMizoCenter}
\begin{align}
  \label{eq:difTopMizoCenter}
  \sqrt{R_\mathrm c^2-\left(f_\mathrm T-\bar\kappa_w\right)^{\!2}}
  -\sqrt{R_\mathrm c^2-f_\mathrm T^2}
\end{align}
で与えられる。


%%%%%%%%%%%%%%%%%%%%%%%%%%%%%%%%%%%%%%%%%%%%%%%%%%%%%%%%%%
%% subsubsection 4.1.1 %%%%%%%%%%%%%%%%%%%%%%%%%%%%%%%%%%%
%%%%%%%%%%%%%%%%%%%%%%%%%%%%%%%%%%%%%%%%%%%%%%%%%%%%%%%%%%
\subsection{スペーサを用いた場合の溝中心(湾曲中心基準)}
溝中心M$'$がモールドの湾曲中心線上にある場合、テーブル中心Pを原点とした$x$座標は、\pageeqref{eq:spacerTRc}より、
\begin{align*}
  -\varDelta+\sqrt{R_\mathrm c^2-(f_\mathrm T-\bar\kappa_w)^2}+\frac\delta2
  -\sqrt{R_\mathrm i'^2-\frac{\delta^2+(2\bar l)^2}4}\frac{2\bar l}{\sqrt{\delta^2+\left(2\bar l\right)^2}}
\end{align*}
となる。
なお実際の作業では、簡単のため、トップ端面の外側中心T$_\mathrm c'$を測定し、それをトップ端面における湾曲中心T$_{R_\mathrm c}'$とみなして溝中心M$'$の位置を計算することが多い。
実測した外側中心の$X$・$Y$座標$G_{\mathrm Tx}$, $G_{\mathrm Ty}$を湾曲中心のそれとみなすと、機内における溝中心M$'$の位置は、テーブル中心Pを原点として、
%% label{eq:Mreal}
\begin{subequations}
  \label{eq:Mreal}
\begin{align}
  \left(
    G_{\mathrm Tx}
    +\sqrt{R_\mathrm c^2-(f_\mathrm T-\bar\kappa_w)^2}
    -\sqrt{R_\mathrm c^2-f_\mathrm T^2}\ ,\
    G_{\mathrm Ty}~,~
    f_\mathrm T'-\bar\kappa_w
  \right).
\end{align}
湾曲中心とみなさずに正確に求めるなら、これに\pageeqref{eq:TRc-Tc}を引けばよい。
その場合の$X$座標は、
\begin{align}
  G_{\mathrm Tx}
  +\sqrt{R_\mathrm c^2-(f_\mathrm T-\bar\kappa_w)^2}
  -\frac{\sqrt{R_\mathrm o^2-f_\mathrm T^2}+\sqrt{R_\mathrm i^2-f_\mathrm T^2}}2\ .
\end{align}
\end{subequations}


%%%%%%%%%%%%%%%%%%%%%%%%%%%%%%%%%%%%%%%%%%%%%%%%%%%%%%%%%%
%% subsubsection 4.1.2 %%%%%%%%%%%%%%%%%%%%%%%%%%%%%%%%%%%
%%%%%%%%%%%%%%%%%%%%%%%%%%%%%%%%%%%%%%%%%%%%%%%%%%%%%%%%%%
\subsection{テーブルを傾けた場合の溝中心(湾曲中心基準)}
\index{みぞちゅうしん@溝中心}溝中心M$'$がモールドの湾曲中心線上にある場合、テーブル中心Pを原点とした$x$座標は、\pageeqref{eq:tableTRc}より、
\begin{align*}
  \sqrt{R_\mathrm c^2-(f_\mathrm T-\bar\kappa_w)^2}
  -\varDelta'\cos\theta\ .
\end{align*}
実測した\index{そとがわちゅうしん@外側中心}外側中心の$X$・$Y$座標$G_{\mathrm Tx}$, $G_{\mathrm Ty}$を\index{わんきょくちゅうしん@湾曲中心}湾曲中心のそれとみなした場合とそうでない場合は、\pageeqref{eq:Mreal}で与えられる。




%\clearpage
%%%%%%%%%%%%%%%%%%%%%%%%%%%%%%%%%%%%%%%%%%%%%%%%%%%%%%%%%%
%% section 4.2 %%%%%%%%%%%%%%%%%%%%%%%%%%%%%%%%%%%%%%%%%%%
%%%%%%%%%%%%%%%%%%%%%%%%%%%%%%%%%%%%%%%%%%%%%%%%%%%%%%%%%%
\modHeadsection{外削径の中心が基準の場合}
\index{みぞちゅうしん@溝中心}溝中心M$'$がトップ外削径の中心線上にある場合、機内におけるその位置座標は、
\begin{align*}
  \left(
    -\mathcal G_{Bx}+T_x\ ,\
    \mathcal G_{By}\ ,\
    f_\mathrm T'-\bar\kappa_w
  \right) \qquad
  \text{または}\qquad
  \left(
    \mathcal G_{Bx}\ ,\
    \mathcal G_{By}\ ,\
    f_\mathrm T'-\bar\kappa_w
  \right).
\end{align*}
ただし、前者はボトムの外削を基準にした(ボトム基準の\index{とおりしん@通り芯}通り芯がある)場合であり、後者はトップの外削を基準にした場合である。



%\clearpage
%%%%%%%%%%%%%%%%%%%%%%%%%%%%%%%%%%%%%%%%%%%%%%%%%%%%%%%%%%
%% section 4.3 %%%%%%%%%%%%%%%%%%%%%%%%%%%%%%%%%%%%%%%%%%%
%%%%%%%%%%%%%%%%%%%%%%%%%%%%%%%%%%%%%%%%%%%%%%%%%%%%%%%%%%
\modHeadsection{A面側の溝深さが基準の場合}


%%%%%%%%%%%%%%%%%%%%%%%%%%%%%%%%%%%%%%%%%%%%%%%%%%%%%%%%%%
%% subsection 4.3.1 %%%%%%%%%%%%%%%%%%%%%%%%%%%%%%%%%%%%%%
%%%%%%%%%%%%%%%%%%%%%%%%%%%%%%%%%%%%%%%%%%%%%%%%%%%%%%%%%%
\subsection{外削のない場合}
トップ側に外削がなく、\index{Aがわみぞふかさ@A側溝深さ}A側面からの溝深さが指定されている場合を考える。
\index{みぞちゅうしん@溝中心}溝中心の位置の$X$座標は、テーブル中心Pを\index{げんてん@原点}原点として、
%% label{eq:mizocenterA}
\begin{align}
  \label{eq:mizocenterA}
  \sqrt{R_\mathrm o^2-(f_\mathrm T-\bar\kappa_w)^2}-\kappa_d-\frac{W_{mx}}2
  -\varDelta'
\end{align}
で与えられる。
ここで、$W_{mx}$は溝のAC方向の径を表す。
なお実際の作業では、モールドの\index{Aがわがいめん@A側外面}A側外面の\index{みぞはばちゅうおう@溝幅中央}溝幅中央に相当する箇所を直接計測し、その位置を基準として原点を割り出す。
トップ端面における中心の$X$座標(\index{じっそくち@実測値}実測値)$G_{\mathrm Tx}$がわかっている場合、\index{みぞちゅうしん@溝中心}溝の中心の$X$座標$G_{mx}$は\pageeqref{eq:Mreal}で与えられ、また\index{みぞはばちゅうおう@溝幅中央}溝幅中央に対する\index{Aがわがいめん@A側外面}A側外面と$G_{\mathrm Tx}$との差(の絶対値)は、
%% label{eq:mizocenterA}
\begin{align}
  \label{eq:mizocenterAd}
  \frac{W_{mx}}2+\kappa_d
  +\sqrt{R_\mathrm c^2-(f_\mathrm T-\bar\kappa_w)^2}
  -\frac{\sqrt{R_\mathrm o^2-f_\mathrm T^2}+\sqrt{R_\mathrm i^2-f_\mathrm T^2}}2\ .
\end{align}


%%%%%%%%%%%%%%%%%%%%%%%%%%%%%%%%%%%%%%%%%%%%%%%%%%%%%%%%%%
%% subsection 4.3.2 %%%%%%%%%%%%%%%%%%%%%%%%%%%%%%%%%%%%%%
%%%%%%%%%%%%%%%%%%%%%%%%%%%%%%%%%%%%%%%%%%%%%%%%%%%%%%%%%%
\subsection{外削のある場合}
トップ側に外削があり、かつ\index{Aがわみぞふかさ@A側溝深さ}A側溝深さが指定されている場合を考える。
トップ側の外削中心の実測値を$\mathcal G_{\mathrm Tx}$とすると、\index{みぞちゅうしん@溝中心}溝中心$G_{mx}$との差($G_{mx}-\mathcal G_{\mathrm Tx}$)は
%% label{eq:mizocenterAG}
\begin{align}
  \label{eq:mizocenterAG}
  \frac{\mathfrak W_x}2-\kappa_d-\frac{W_{mx}}2\ .
\end{align}



\clearpage
%%%%%%%%%%%%%%%%%%%%%%%%%%%%%%%%%%%%%%%%%%%%%%%%%%%%%%%%%%
%% section 4.4 %%%%%%%%%%%%%%%%%%%%%%%%%%%%%%%%%%%%%%%%%%%
%%%%%%%%%%%%%%%%%%%%%%%%%%%%%%%%%%%%%%%%%%%%%%%%%%%%%%%%%%
\modHeadsection{測定上の溝深さ}
ここでは特に\index{Aがわみぞふかさ@A側溝深さ}A面側溝深さ$\kappa_d$に限って話を進める。
トップ側に外削がある場合、$\kappa_d$は\index{Aがわがいさくめん@A側外削面}A側外削面と溝のA側面との差で与えられる。
しかし外削がない場合、つまり\index{Aがわがいめん@A側外面}A側外面に湾曲がある場合は、$\kappa_d$の値は自明ではない。


%%%%%%%%%%%%%%%%%%%%%%%%%%%%%%%%%%%%%%%%%%%%%%%%%%%%%%%%%%
%% subsection 4.4.1 %%%%%%%%%%%%%%%%%%%%%%%%%%%%%%%%%%%%%%
%%%%%%%%%%%%%%%%%%%%%%%%%%%%%%%%%%%%%%%%%%%%%%%%%%%%%%%%%%
\subsection{図面上の溝深さ}
単純に考えると、図面上においてA側の溝深さが$\kappa_d$のとき、これは溝幅の中央の位置におけるA側外面から端面と水平な方向に$\kappa_d$という意味で与えられる。
このとき、溝のA面側の水平方向の位置は、\pageeqref{eq:mizocenterA}より、
\begin{align*}
  \sqrt{R_\mathrm o^2-(f_\mathrm T-\bar\kappa_w)^2}-\kappa_d-\varDelta'\ .
\end{align*}
溝中心の$X$座標$G_{mx}$が与えられている場合は、
\begin{align*}
  G_{mx}+\frac{W_{mx}}2
\end{align*}



%%%%%%%%%%%%%%%%%%%%%%%%%%%%%%%%%%%%%%%%%%%%%%%%%%%%%%%%%%
%% subsection 4.4.2 %%%%%%%%%%%%%%%%%%%%%%%%%%%%%%%%%%%%%%
%%%%%%%%%%%%%%%%%%%%%%%%%%%%%%%%%%%%%%%%%%%%%%%%%%%%%%%%%%
\subsection{測定上の傾き}
通常、実測には\index{マイクロメータ}マイクロメータ(\index{デプスゲージ}デプスゲージ)が用いられる。
このとき、計測は\index{けいそくき@計測器}計測器を湾曲に沿った形にして行われる。
したがって、その計測値は\index{みぞはば@溝幅}溝幅の両端に対する外面の傾斜だけ傾いた形で与えられる。
溝幅の両端に対する外面の$XZ$位置は、溝中心の$X$座標を$G_{mx}$として、
\begin{align*}
  \text{トップ側:}&~~
  \left(
  G_{mx}+\frac{W_x}2
  -\sqrt{R_\mathrm o^2-(f_\mathrm T-\bar\kappa_w)^2}
  +\sqrt{R_\mathrm o^2-(f_\mathrm T-\kappa_p)^2}~,~~
  f_\mathrm T-\kappa_p
  \right)\\
  \text{ボトム側:}&~~
  \left(
  G_{mx}+\frac{W_x}2
  +\sqrt{R_\mathrm o^2-(f_\mathrm T-\kappa_p-\kappa_w)^2}
  -\sqrt{R_\mathrm o^2-(f_\mathrm T-\bar\kappa_w)^2}~,~~
  f_\mathrm T-\kappa_p-\kappa_w
  \right)
\end{align*}
またその差は、
\begin{align*}
  \left(
  \sqrt{R_\mathrm o^2-(f_\mathrm T-\kappa_p)^2}
  -\sqrt{R_\mathrm o^2-(f_\mathrm T-\kappa_p-\kappa_w)^2}~,~~
  \kappa_w
  \right)
\end{align*}
したがって、計測の際の傾斜の角度$\zeta$ ($> 0$)は
%% label{eq:angleZeta}
\begin{align}
  \label{eq:angleZeta}
  \tan\zeta
  = \frac{\sqrt{R_\mathrm o^2-\left(f_\mathrm T-\kappa_p-\kappa_w\right)^2}
          -\sqrt{R_\mathrm o^2-\left(f_\mathrm T-\kappa_p\right)^2}}
         {\kappa_w}\ .
\end{align}


\clearpage
%%%%%%%%%%%%%%%%%%%%%%%%%%%%%%%%%%%%%%%%%%%%%%%%%%%%%%%%%%
%% subsection 4.4.3 %%%%%%%%%%%%%%%%%%%%%%%%%%%%%%%%%%%%%%
%%%%%%%%%%%%%%%%%%%%%%%%%%%%%%%%%%%%%%%%%%%%%%%%%%%%%%%%%%
\subsection{測定における溝深さの補正\label{subsec:keywayDepthDif}}
測定の際は、計測器を溝のトップ側およびボトム側に寄せる形で測定し、その平均値を測定値としている。
つまり、トップ側に寄っているときは計測器の針の付け根が溝位置にあるところ、ボトム側に寄っているときは針の先端が溝幅ボトム側にあるところで測定を行っている。
この平均値を$\kappa_d'$とする。

トップ側およびボトム側の(A側)溝深さをそれぞれ$\kappa_s$, $\kappa_l$とすると、
\begin{align*}
  \kappa_l-\kappa_s = \kappa_w\tan\zeta \quad,\qquad
  \kappa_d' = \frac{\kappa_l\cos\zeta+\kappa_s\sec\zeta}2
\end{align*}
であり、
\begin{align*}
  \kappa_l = \frac{2\kappa_d'\cos\zeta+\kappa_w\tan\zeta}{1+\cos^2\zeta}~~, \quad
  \kappa_s = \frac{2\kappa_d'-\kappa_w\sin\zeta}{1+\cos^2\zeta}\cos\zeta\ .
\end{align*}
%%%%%%%%%%%%%%%%%%%%%%%%%%%%%%%%%%%%%%%%%%%%%%%%%%%%%%%%%%
%% hosoku %%%%%%%%%%%%%%%%%%%%%%%%%%%%%%%%%%%%%%%%%%%%%%%%
%%%%%%%%%%%%%%%%%%%%%%%%%%%%%%%%%%%%%%%%%%%%%%%%%%%%%%%%%%
\begin{hosoku}
$\kappa_l = a+b$, $\kappa_s = a-b$とすると、
\begin{align*}
  a = \frac{\kappa_l+\kappa_s}2~~, \quad
  b = \frac{\kappa_l-\kappa_s}2\,\left(= \frac12\kappa_w\tan\zeta\right)
\end{align*}
であり、
\begin{align*}
  2\kappa_d' = a(\cos\zeta+\sec\zeta)+b(\cos\zeta-\sec\zeta)
  \quad\longrightarrow\quad
  a = \frac{2\kappa_d'\cos\zeta+b\sin^2\zeta}{1+\cos^2\zeta}\ .
\end{align*}
また、$\kappa_l$, $\kappa_s$はその平均値$a$からその差の半分$b$の和あるいは差として得られる。
\begin{align*}
  \kappa_l
  &= \frac{2\kappa_d'\cos\zeta+\frac{\kappa_w\tan\zeta}2\sin^2\zeta}{1+\cos^2\zeta}+\frac12\kappa_w\tan\zeta\ ,\\
  \kappa_s
  &= \frac{2\kappa_d'\cos\zeta+\frac{\kappa_w\tan\zeta}2\sin^2\zeta}{1+\cos^2\zeta}-\frac12\kappa_w\tan\zeta\ .
\end{align*}
\end{hosoku}
%%%%%%%%%%%%%%%%%%%%%%%%%%%%%%%%%%%%%%%%%%%%%%%%%%%%%%%%%%
%%%%%%%%%%%%%%%%%%%%%%%%%%%%%%%%%%%%%%%%%%%%%%%%%%%%%%%%%%
%%%%%%%%%%%%%%%%%%%%%%%%%%%%%%%%%%%%%%%%%%%%%%%%%%%%%%%%%%
$\kappa_d$および$\kappa_s$の差は
\begin{align*}
  \kappa_d-\kappa_s
  &= \sqrt{R_\mathrm o^2-(f_\mathrm T-\bar\kappa_w)^2}
     -\sqrt{R_\mathrm o^2-(f_\mathrm T-\kappa_p)^2}
\end{align*}
なので、
\begin{subequations}
%% label{eq:keydepthDif1}
\begin{align}
  \label{eq:keydepthDif1}
  \kappa_d
  &= \frac{2\kappa_d'-\kappa_w\sin\zeta}{1+\cos^2\zeta}\cos\zeta
     +\sqrt{R_\mathrm o^2-(f_\mathrm T-\bar\kappa_w)^2}
     -\sqrt{R_\mathrm o^2-(f_\mathrm T-\kappa_p)^2}
\end{align}
あるいは、
%% label{eq:keydepthDif2}
\begin{align}
  \label{eq:keydepthDif2}
  \kappa_d'
  &= \frac{1+\cos^2\zeta}{2\cos\zeta}
     \left\{
     \kappa_d
     -\sqrt{R_\mathrm o^2-(f_\mathrm T-\bar\kappa_w)^2}
     +\sqrt{R_\mathrm o^2-(f_\mathrm T-\kappa_p)^2}
     \right\}
     +\frac12\kappa_w\sin\zeta\ .
\end{align}
\end{subequations}
したがって、$\kappa_d$を図面上の数値とする場合は\pageeqref{eq:keydepthDif2}を、$\kappa_d'$を図面上の数値とする場合は、\pageeqref{eq:keydepthDif1}を用いて補正すればよい。


\clearpage
~\vfill
%%%%%%%%%%%%%%%%%%%%%%%%%%%%%%%%%%%%%%%%%%%%%%%%%%%%%%%%%%
%% Column %%%%%%%%%%%%%%%%%%%%%%%%%%%%%%%%%%%%%%%%%%%%%%%%
%%%%%%%%%%%%%%%%%%%%%%%%%%%%%%%%%%%%%%%%%%%%%%%%%%%%%%%%%%
\begin{Column}{測定における溝深さ補正の近似計算}
$R\to\infty$ ($R^{-1}\to0$)に対して、
\begin{align*}
  \tan\zeta \xlongrightarrow{R\to\infty} 0~~, \quad
  \kappa_d' \xlongrightarrow{R\to\infty} \kappa_d\ .
\end{align*}
もう少し詳しく見ると、\index{テイラーてんかい@テイラー展開}テイラー展開(\index{マクローリンてんかい@マクローリン展開}マクローリン展開)\pageautoref{formula:taylorexpansion}より、
\begin{align*}
  \tan\zeta
  = \frac{R_\mathrm o}{\kappa_w}\!
     \left\{
     \sqrt{1-\left(\frac{f_\mathrm T-\kappa_p-\kappa_w}{R_\mathrm o}\right)^{\!\!2}}
     -\sqrt{1-\left(\frac{f_\mathrm T-\kappa_p}{R_\mathrm o}\right)^{\!\!2}}
     \right\}
  = \frac{f_\mathrm T-\bar\kappa_w}{R_\mathrm o}+o\!\left(R_\mathrm o^{-3}\right)
\end{align*}
であり、これより、
\begin{align*}
  \sin\zeta = \frac{f_\mathrm T-\bar\kappa_w}{R_\mathrm o}+o\!\left(R_\mathrm o^{-3}\right)~~, \quad
  \cos\zeta = 1-\frac12\!\left(\frac{f_\mathrm T-\bar\kappa_w}{R_\mathrm o}\right)^{\!\!2}
              +o\!\left(R_\mathrm o^{-4}\right)
\end{align*}
したがって、
\begin{align*}
  \kappa_d
  &= \kappa_d'-\frac{\kappa_w}2\frac{f_\mathrm T-\bar\kappa_w}{R_\mathrm o}
     +R_\mathrm o\!
      \left\{
      \sqrt{1-\left(\frac{f_\mathrm T-\bar\kappa_w}{R_\mathrm o}\right)^{\!\!2}}
      -\sqrt{1-\left(\frac{f_\mathrm T-\kappa_p}{R_\mathrm o}\right)^{\!\!2}}
      \right\}\\
  &= \kappa_d'+\frac{\kappa_w}2\frac{f_\mathrm T-\kappa_p}{R_\mathrm o}
     +o\!\left(R_\mathrm o^{-3}\right)\ .
\end{align*}
これから、$\kappa_d > \kappa_d'$であることがわかる。
\end{Column}
%%%%%%%%%%%%%%%%%%%%%%%%%%%%%%%%%%%%%%%%%%%%%%%%%%%%%%%%%%
%%%%%%%%%%%%%%%%%%%%%%%%%%%%%%%%%%%%%%%%%%%%%%%%%%%%%%%%%%
%%%%%%%%%%%%%%%%%%%%%%%%%%%%%%%%%%%%%%%%%%%%%%%%%%%%%%%%%%






