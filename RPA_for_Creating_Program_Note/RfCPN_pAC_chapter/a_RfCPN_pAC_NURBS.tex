%!TEX root = ../RPA_for_Creating_Program_Note.tex



アイソパラメトリック曲線は、その曲線が属するパラメトリックサーフェスの形状に依存する。
一般に、パラメトリックサーフェスは、パラメータ$u$, $v$を用いて次のように表すことができる。
\begin{align*}
  S(u, v) &=
  \left[
  \begin{array}{c}
    x(u, v)\\
    y(u, v)\\
    z(u, v)
  \end{array}
  \right]
\end{align*}
ここで、$x(u,v)$, $y(u,v)$, $z(u,v)$はサーフェスの各座標を定義する関数である。
アイソパラメトリック曲線は、このサーフェス上の一定の$u$または$v$の値に対応する曲線を指す。
したがって、アイソパラメトリック曲線は次のように表すことができる。
\begin{align*}
  C(t) = S\big(u(t), v(t)\big)\ .
\end{align*}
ここで$u(t)$, $v(t)$はパラメータであり、どちらかが一定(例えば$u(t)=k$または$v(t)=k$)で、もう一方のパラメータは$t$に対して変化する。
たとえば$v(t) = k$ ($k = \text{const.}$)とすると、
\begin{align*}
  C(u) = S\big(u(t), k\big)
\end{align*}
と表すことができる。
これを改めて$u$の関数として
\begin{align*}
  C(u) = \frac{\displaystyle\sum_{i=0}^nf_i(u)w_iP_i}{\displaystyle\sum_{i=0}^nf_i(u)w_i}
\end{align*}
と表すことにする。
こうすることで、曲面$S(u, v)$の一部を切り出した曲線$C(u)$は、基底関数$f_i(u)$と(重み$w_i$を付けた)制御点$P_i$の組み合わせで表現することができる。

主なアイソパラメトリック曲線として、ベジエ曲線・Bスプライン曲線・NURBS曲線が挙げられる。
Bスプライン曲線はベジエ曲線を一般化したものとみなすことができ、さらにNURBS曲線はBスプライン曲線を一般化したものとみなすことができる。
そのため、ここではNURBS曲線
%% footnote %%%%%%%%%%%%%%%%%%%%%
\footnote{NURBSはNon-Uniform Rational B-Spline(非一様有理Bスプライン)の略である。
}
%%%%%%%%%%%%%%%%%%%%%%%%%%%%%%%%%
について詳しく見ていくことにする。


\clearpage
%%%%%%%%%%%%%%%%%%%%%%%%%%%%%%%%%%%%%%%%%%%%%%%%%%%%%%%%%%
%% section E.1 %%%%%%%%%%%%%%%%%%%%%%%%%%%%%%%%%%%%%%%%%%%
%%%%%%%%%%%%%%%%%%%%%%%%%%%%%%%%%%%%%%%%%%%%%%%%%%%%%%%%%%
\modHeadsection{NURBS曲線\TBW}
~