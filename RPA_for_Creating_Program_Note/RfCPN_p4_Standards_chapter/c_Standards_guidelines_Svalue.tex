%!TEX root = ../RPA_for_Creating_Program_Note.tex


\modHeadchapter[lot]{工具の主軸回転数(Sコード値)}
%%%%%%%%%%%%%%%%%%%%%%%%%%%%%%%%%%%%%%%%%%%%%%%%%%%%%%%%%%
%% marker %%%%%%%%%%%%%%%%%%%%%%%%%%%%%%%%%%%%%%%%%%%%%%%%
%%%%%%%%%%%%%%%%%%%%%%%%%%%%%%%%%%%%%%%%%%%%%%%%%%%%%%%%%%
\begin{marker}
この章では工具の具体的な\index{しゅじくかいてんすう@主軸回転数}主軸回転数(\index{スピンドルかいてんすう@スピンドル回転数}スピンドル回転数)を記述している。
しかし、工具の主軸回転数の具体的な数値は、(ソフトウェアでなく)ハードウェアの標準に記載するほうが望ましい。
\end{marker}
%%%%%%%%%%%%%%%%%%%%%%%%%%%%%%%%%%%%%%%%%%%%%%%%%%%%%%%%%%
%%%%%%%%%%%%%%%%%%%%%%%%%%%%%%%%%%%%%%%%%%%%%%%%%%%%%%%%%%
%%%%%%%%%%%%%%%%%%%%%%%%%%%%%%%%%%%%%%%%%%%%%%%%%%%%%%%%%%


%%%%%%%%%%%%%%%%%%%%%%%%%%%%%%%%%%%%%%%%%%%%%%%%%%%%%%%%%%
%% section 07.1 %%%%%%%%%%%%%%%%%%%%%%%%%%%%%%%%%%%%%%%%%%
%%%%%%%%%%%%%%%%%%%%%%%%%%%%%%%%%%%%%%%%%%%%%%%%%%%%%%%%%%
\modHeadsection{主軸回転数の基本事項\TBW}
\begin{enumerate}
\item \index{でんげんとうにゅうじ(きかいほんたい)@電源投入時(機械本体)}電源投入時の\index{しゅじくかいてんすう@主軸回転数}主軸回転数:0
\item 設定可能な最小単位:1回転/min
\item \index{ハイギア(しゅじくかいてんすう)@ハイギア(主軸回転数)}ハイギア(high gear)と\index{ローギア(しゅじくかいてんすう)@ローギア(主軸回転数)}ローギア(low gear)の2種類がある
\item 回転数(設定値)2600以上でハイギアに、2599以下でローギアとなる
\item 設定可能最低値は35
\end{enumerate}
~\newline\noindent
\dateKouguRotation における主軸回転数の\index{せっていち(しゅじくかいてんすう)@設定値(主軸回転数)}設定値は、以下のとおりである。\\

\begin{multicollongtblr}{工具の主軸回転数設定値 一覧}{X[l]c}
加工の種類 & 回転数\\
端面加工 & 950\\
外削加工 & 930\\
溝加工 & 930\\
外面取加工 & 1800\\
内面取加工 & 1800\\
座ぐり加工 & 500\\
\dimple 加工 & 2000\\
\end{multicollongtblr}

