%!TEX root = ../RPA_for_Creating_Program_Note.tex



ここでは\DMname の加工システムで使用している\index{コモンへんすう@コモン変数}コモン変数について述べる。


%%%%%%%%%%%%%%%%%%%%%%%%%%%%%%%%%%%%%%%%%%%%%%%%%%%%%%%%%%
%% section 11.1 %%%%%%%%%%%%%%%%%%%%%%%%%%%%%%%%%%%%%%%%%%
%%%%%%%%%%%%%%%%%%%%%%%%%%%%%%%%%%%%%%%%%%%%%%%%%%%%%%%%%%
\modHeadsection{コモン変数の範囲}
\DMname で使用可能なコモン変数は以下のとおりである。
\begin{enumerate}
\item \ttNum100\,-\ttNum199
\item \ttNum400\,-\ttNum999
\item \ttNum900000\,-\ttNum907399
\end{enumerate}
%%%%%%%%%%%%%%%%%%%%%%%%%%%%%%%%%%%%%%%%%%%%%%%%%%%%%%%%%%
%% marker %%%%%%%%%%%%%%%%%%%%%%%%%%%%%%%%%%%%%%%%%%%%%%%%
%%%%%%%%%%%%%%%%%%%%%%%%%%%%%%%%%%%%%%%%%%%%%%%%%%%%%%%%%%
\begin{marker}
メーカーのマニュアルには\ttNum400\,-\ttNum999でなく\ttNum500\,-\ttNum999と記載されている誤植があるので注意
\end{marker}
%%%%%%%%%%%%%%%%%%%%%%%%%%%%%%%%%%%%%%%%%%%%%%%%%%%%%%%%%%
%%%%%%%%%%%%%%%%%%%%%%%%%%%%%%%%%%%%%%%%%%%%%%%%%%%%%%%%%%
%%%%%%%%%%%%%%%%%%%%%%%%%%%%%%%%%%%%%%%%%%%%%%%%%%%%%%%%%%



%%%%%%%%%%%%%%%%%%%%%%%%%%%%%%%%%%%%%%%%%%%%%%%%%%%%%%%%%%
%% section 18.2 %%%%%%%%%%%%%%%%%%%%%%%%%%%%%%%%%%%%%%%%%%
%%%%%%%%%%%%%%%%%%%%%%%%%%%%%%%%%%%%%%%%%%%%%%%%%%%%%%%%%%
\modHeadsection{\ttNum100\,-\ttNum199}
\ttNum100\,-\ttNum174については、(機械設置時の)\index{バンドルのプログラム}バンドルのプログラムやカスタマイズされた\index{Mコード}Mコードで既に使用されているものが多いため、基本的には(\index{RHS(コモンへんすう)@RHS(コモン変数)}RHSとしては)使用しないものとし、一時的なもの(\index{LHS(コモンへんすう)@LHS(コモン変数)}LHS)として扱うものとする。


%%%%%%%%%%%%%%%%%%%%%%%%%%%%%%%%%%%%%%%%%%%%%%%%%%%%%%%%%%
%% subsection 18.2.1 %%%%%%%%%%%%%%%%%%%%%%%%%%%%%%%%%%%%%
%%%%%%%%%%%%%%%%%%%%%%%%%%%%%%%%%%%%%%%%%%%%%%%%%%%%%%%%%%
\subsection{\ttNum100\,-\ttNum174:一時保存値}
\noindent\ttNum100\,-\ttNum109については、主に一時的な保存に用いるものとする。\\

\begin{multicollongtblr}[white]{\ttNum100\,-\ttNum109:一時保存値}{cX[l]}
変数 & 内容\\
\ttNum100 & 各工程 切削回数用 一時保存値(仕上げ前 全削り代$X$ or $Z$)\\
\ttNum101 & 各工程 切削回数用 一時保存値(仕上げ前 全削り代$Y$)\\
\ttNum102 & 各工程 切削回数用 一時保存値 (max[\ttNum100, \ttNum101])\\
\ttNum103 & 各工程 切削回数用 一時保存値(仕上げ前 切削回数)\\
\ttNum104 & 各工程 切削回数用 一時保存値(加工時 径$X$)\\
\ttNum105 & 各工程 切削回数用 一時保存値(加工時 径$Y$)\\
\ttNum106 & 各工程 切削回数用 一時保存値(仕上げ 切削回数)\\
\SetRow{unusingVariables}
\ttNum107 & (予備)\\
\SetRow{unusingVariables}
\ttNum108 & (予備)\\
\SetRow{unusingVariables}
\ttNum109 & (予備)\\
\end{multicollongtblr}


\clearpage
%%%%%%%%%%%%%%%%%%%%%%%%%%%%%%%%%%%%%%%%%%%%%%%%%%%%%%%%%%
%% subsection 18.2.2 %%%%%%%%%%%%%%%%%%%%%%%%%%%%%%%%%%%%%
%%%%%%%%%%%%%%%%%%%%%%%%%%%%%%%%%%%%%%%%%%%%%%%%%%%%%%%%%%
\subsection{\ttNum175\,-\ttNum199:各工程用 補助機能(使用頻度 低)}
\noindent\ttNum175\,-\ttNum199については、使用頻度が低いと思われる\index{ほじょきのう(コモンへんすう)@補助機能(コモン変数)}補助機能のものとする。\\

\begin{multicollongtblr}[white]{\ttNum175\,-\ttNum199:補助機能(使用頻度 低)}{cX[l]c}
変数 & 内容 & 初期値\\
\ttNum175 & 計測・加工 開始N番号 & 0\\
\SetRow{unusingVariables}
\ttNum176 & (不使用) &\\
\ttNum177 & 芯出し測定後 一時停止 (0:{\ttfamily M00}, 1:non-stop) & 1\\
\ttNum178 & \dimple 測定後 一時停止 (0:{\ttfamily M00}, 1:non-stop) & 1\\
\ttNum179 & トップ端面加工後 一時停止 (0:{\ttfamily M00}, 1:non-stop, 2:扉前{\ttfamily M00}) & 1\\
\ttNum180 & ボトム端面加工後 一時停止 (0:{\ttfamily M00}, 1:non-stop, 2:扉前{\ttfamily M00}) & 1\\
\SetRow{unusingVariables}
\ttNum181 & (不使用) &\\
\ttNum182 & トップ外削 仕上げ加工 追加回数 (上限3) & 0\\
\ttNum183 & 溝 仕上げ加工 追加回数 (上限3) & 0\\
\ttNum184 & トップ外面取 仕上げ加工 追加回数 (上限3) & 0\\
\ttNum185 & トップ内面取 仕上げ加工 追加回数 (上限3) & 0\\
\SetRow{unusingVariables}
\ttNum186\TBW & (座ぐり 仕上げ加工 追加回数用 予備) & 0\\
\SetRow{unusingVariables}
\ttNum187 & (不使用) &\\
\ttNum188 & ボトム外削 仕上げ加工 追加回数 (上限3) & 0\\
\ttNum189 & ボトム外面取 仕上げ加工 追加回数 (上限3) & 0\\
\ttNum190 & ボトム内面取 仕上げ加工 追加回数 (上限3) & 0\\
\SetRow{unusingVariables}
\ttNum191 & (不使用) &\\
\SetRow{unusingVariables}
\ttNum192 & (予備) &\\
\SetRow{unusingVariables}
\ttNum193 & (予備) &\\
\SetRow{unusingVariables}
\ttNum194 & (予備) &\\
\SetRow{unusingVariables}
\ttNum195 & (予備) &\\
\SetRow{unusingVariables}
\ttNum196 & (予備) &\\
\SetRow{unusingVariables}
\ttNum197 & (予備) &\\
\SetRow{unusingVariables}
\ttNum198 & (予備) &\\
\SetRow{unusingVariables}
\ttNum199 & (予備) &\\
\end{multicollongtblr}



\clearpage
%%%%%%%%%%%%%%%%%%%%%%%%%%%%%%%%%%%%%%%%%%%%%%%%%%%%%%%%%%
%% section 18.3 %%%%%%%%%%%%%%%%%%%%%%%%%%%%%%%%%%%%%%%%%%
%%%%%%%%%%%%%%%%%%%%%%%%%%%%%%%%%%%%%%%%%%%%%%%%%%%%%%%%%%
\modHeadsection{\ttNum400\,-\ttNum499:各工程用 補助機能(使用頻度 高)}
\ttNum400\,-\ttNum499については、使用頻度が高いと思われる\index{ほじょきのう(コモンへんすう)@補助機能(コモン変数)}補助機能のものとする。\\
ただし、\ttNum475\,-\ttNum499については、該当する\index{めいさい(モールド)@明細(モールド)}明細が少ないと思われる補助機能のものとする。


%\clearpage
%%%%%%%%%%%%%%%%%%%%%%%%%%%%%%%%%%%%%%%%%%%%%%%%%%%%%%%%%%
%% subsection 17.3.1 %%%%%%%%%%%%%%%%%%%%%%%%%%%%%%%%%%%%%
%%%%%%%%%%%%%%%%%%%%%%%%%%%%%%%%%%%%%%%%%%%%%%%%%%%%%%%%%%
\subsection{\ttNum400\,-\ttNum424:初期設定および調整}

\begin{multicollongtblr}[white]{\ttNum400\,-\ttNum424:補助機能(使用頻度 高)}{cX[l]c}
変数 & 内容 & 初期値\\
\ttNum400 & トップ端面 全削り代 &\\
\ttNum401 & ボトム端面 全削り代(0 or \ttNum0: \ttNum400) &\\
\ttNum402 & \expandafterindex{\dimplekana しそくてい@\dimple 試測定}\dimple 試測定, 通り芯測定(0:off, 1:on) & 0\\
\SetRow{unusingVariables}
\ttNum403 & (不使用) &\\
\ttNum404 & {\ttfamily G54}$X$:ボトム外$X$芯出し(両側・片側測定)測定位置$Z-$補正($X$自動補正) & 0\\
\ttNum405 & {\ttfamily G54}$Y$:ボトム外$Y$芯出し(両側測定)測定位置$Z-$補正 & 0\\
\ttNum406 & {\ttfamily G55}$X$:ボトム内$X$芯出し(両側測定)測定位置$Z-$補正($X$自動補正) & 0\\
\ttNum407 & {\ttfamily G55}$Y$:ボトム内$Y$芯出し(両側測定)測定位置$Z-$補正 & 0\\
\SetRow{unusingVariables}
\ttNum408 & (不使用) &\\
\ttNum409 & {\ttfamily G56}$X$:トップ外$X$芯出し(両側・片側測定)測定位置$Z-$補正($X$自動補正) & 0\\
\ttNum410 & {\ttfamily G56}$Y$:トップ外$Y$芯出し(両側測定)測定位置$Z-$補正 & 0\\
\ttNum411 & {\ttfamily G57}$X$:トップ内$X$芯出し(両側測定)測定位置$Z-$補正($X$自動補正) & 0\\
\ttNum412 & {\ttfamily G57}$Y$:トップ内$Y$芯出し(両側測定)測定位置$Z-$補正 & 0\\
\SetRow{unusingVariables}
\ttNum413 & (不使用) &\\
\ttNum414 & トップ外削 A面肉厚$+$補正(外削中心$X-$補正) & 0\\
\ttNum415 & トップ外削径(直径)$+$補正 & 0\\
\ttNum416 & トップ外削 仕上げ前 一時停止 (0:{\ttfamily M00}, 1:non-stop, 2:扉前{\ttfamily M00}) & 1\\
\ttNum417 & トップ外削 仕上げ後 一時停止 (0:{\ttfamily M00}, 1:non-stop, 2:扉前{\ttfamily M00}) & 1\\
\SetRow{unusingVariables}
\ttNum418 & (不使用) &\\
\ttNum419 & 溝位置$+$補正(溝幅不変) & 0\\
\ttNum420 & 溝幅$+$補正 & 0\\
\ttNum421 & 溝A面深さ$+$補正(溝径中心$X-$補正) & 0\\
\ttNum422 & 溝径(直径)$+$補正 & 0\\
\ttNum423 & 溝 仕上げ前 一時停止 (0:{\ttfamily M00}, 1:non-stop, 2:扉前{\ttfamily M00}) & 1\\
\ttNum424 & 溝 仕上げ後 一時停止 (0:{\ttfamily M00}, 1:non-stop, 2:扉前{\ttfamily M00}) & 1\\
\end{multicollongtblr}
%%%%%%%%%%%%%%%%%%%%%%%%%%%%%%%%%%%%%%%%%%%%%%%%%%%%%%%%%%
%% marker %%%%%%%%%%%%%%%%%%%%%%%%%%%%%%%%%%%%%%%%%%%%%%%%
%%%%%%%%%%%%%%%%%%%%%%%%%%%%%%%%%%%%%%%%%%%%%%%%%%%%%%%%%%
\begin{marker}
\index{がいさくちゅうしんそくてい@外削中心測定}外削中心測定\MXIface で\index{うちめんとり@内面取}内面取がある場合、測定位置$Z$は端面の$Z$位置でないことに注意
\end{marker}
%%%%%%%%%%%%%%%%%%%%%%%%%%%%%%%%%%%%%%%%%%%%%%%%%%%%%%%%%%
%%%%%%%%%%%%%%%%%%%%%%%%%%%%%%%%%%%%%%%%%%%%%%%%%%%%%%%%%%
%%%%%%%%%%%%%%%%%%%%%%%%%%%%%%%%%%%%%%%%%%%%%%%%%%%%%%%%%%


\clearpage
%%%%%%%%%%%%%%%%%%%%%%%%%%%%%%%%%%%%%%%%%%%%%%%%%%%%%%%%%%
%% subsection 18.3.3 %%%%%%%%%%%%%%%%%%%%%%%%%%%%%%%%%%%%%
%%%%%%%%%%%%%%%%%%%%%%%%%%%%%%%%%%%%%%%%%%%%%%%%%%%%%%%%%%
\subsection{\ttNum425\,-\ttNum449:初期設定および調整(続き)}

\begin{multicollongtblr}[white]{\ttNum425\,-\ttNum449:補助機能(使用頻度 高)続き}{cX[l]c}
変数 & 内容 & 初期値\\
\ttNum425 & トップ外面取$X+$補正 & 0\\
\ttNum426 & トップ外面取径(直径)$+$補正 & 0\\
\ttNum427 & トップ外面取 仕上げ前 一時停止 (0:{\ttfamily M00}, 1:non-stop, 2:扉前{\ttfamily M00}) & 1\\
\ttNum428 & トップ外面取 仕上げ後 一時停止 (0:{\ttfamily M00}, 1:non-stop, 2:扉前{\ttfamily M00}) & 1\\
\SetRow{unusingVariables}
\ttNum429 & (不使用) &\\
\ttNum430 & トップ内面取$X+$補正 & 0\\
\ttNum431 & (トップ内面取径用 予備) & 0\\
\ttNum432 & トップ内面取 仕上げ前 一時停止 (0:{\ttfamily M00}, 1:non-stop, 2:扉前{\ttfamily M00}) & 1\\
\ttNum433 & トップ内面取 仕上げ後 一時停止 (0:{\ttfamily M00}, 1:non-stop, 2:扉前{\ttfamily M00}) & 1\\
\SetRow{unusingVariables}
\ttNum434 & (不使用) &\\
\SetRow{unusingVariables}
\ttNum435 & (不使用) &\\
\ttNum436 & ボトム外削 A面肉厚$+$補正(外削中心$X+$補正) & 0\\
\ttNum437 & ボトム外削径(直径)$+$補正 & 0\\
\ttNum438 & ボトム外削 仕上げ前 一時停止 (0:{\ttfamily M00}, 1:non-stop, 2:扉前{\ttfamily M00}) & 1\\
\ttNum439 & ボトム外削 仕上げ後 一時停止 (0:{\ttfamily M00}, 1:non-stop, 2:扉前{\ttfamily M00}) & 1\\
\SetRow{unusingVariables}
\ttNum440 & (不使用) &\\
\ttNum441 & ボトム外面取$X+$補正 & 0\\
\ttNum442 & ボトム外面取径(直径)$+$補正 & 0\\
\ttNum443 & ボトム外面取 仕上げ前 一時停止 (0:{\ttfamily M00}, 1:non-stop, 2:扉前{\ttfamily M00}) & 1\\
\ttNum444 & ボトム外面取 仕上げ後 一時停止 (0:{\ttfamily M00}, 1:non-stop, 2:扉前{\ttfamily M00}) & 1\\
\SetRow{unusingVariables}
\ttNum445 & (不使用) &\\
\ttNum446 & ボトム内面取$X+$補正 & 0\\
\ttNum447 & (ボトム内面取径用 予備) & 0\\
\ttNum448 & ボトム内面取 仕上げ前 一時停止 (0:{\ttfamily M00}, 1:non-stop, 2:扉前{\ttfamily M00}) & 1\\
\ttNum449 & ボトム内面取 仕上げ後 一時停止 (0:{\ttfamily M00}, 1:non-stop, 2:扉前{\ttfamily M00}) & 1\\
\end{multicollongtblr}


\clearpage
%%%%%%%%%%%%%%%%%%%%%%%%%%%%%%%%%%%%%%%%%%%%%%%%%%%%%%%%%%
%% subsection 18.3.2 %%%%%%%%%%%%%%%%%%%%%%%%%%%%%%%%%%%%%
%%%%%%%%%%%%%%%%%%%%%%%%%%%%%%%%%%%%%%%%%%%%%%%%%%%%%%%%%%
\subsection{\ttNum450\,-\ttNum474:\dimple の深さ調整}
%%%%%%%%%%%%%%%%%%%%%%%%%%%%%%%%%%%%%%%%%%%%%%%%%%%%%%%%%%
%% marker %%%%%%%%%%%%%%%%%%%%%%%%%%%%%%%%%%%%%%%%%%%%%%%%
%%%%%%%%%%%%%%%%%%%%%%%%%%%%%%%%%%%%%%%%%%%%%%%%%%%%%%%%%%
\begin{marker}
\ttNum461-\ttNum464の値は2024/02/29時点のもの
\end{marker}
%%%%%%%%%%%%%%%%%%%%%%%%%%%%%%%%%%%%%%%%%%%%%%%%%%%%%%%%%%
%%%%%%%%%%%%%%%%%%%%%%%%%%%%%%%%%%%%%%%%%%%%%%%%%%%%%%%%%%
%%%%%%%%%%%%%%%%%%%%%%%%%%%%%%%%%%%%%%%%%%%%%%%%%%%%%%%%%%

\begin{multicollongtblr}[white]{\ttNum450\,-\ttNum474:\dimple の深さ調整}{cX[l]c}
変数 & 内容 & 初期値\\
\SetRow{unusingVariables}
\ttNum450 & (予備) &\\
\SetRow{unusingVariables}
\ttNum451 & (予備) &\\
\SetRow{unusingVariables}
\ttNum452 & (予備) &\\
\SetRow{unusingVariables}
\ttNum453 & (予備) &\\
\SetRow{unusingVariables}
\ttNum454 & (予備) &\\
\SetRow{unusingVariables}
\ttNum455 & (予備) &\\
\SetRow{unusingVariables}
\ttNum456 & (予備) &\\
\SetRow{unusingVariables}
\ttNum457 & (予備) &\\
\SetRow{unusingVariables}
\ttNum458 & (予備) &\\
\ttNum459 & \dimple 加工後 一時停止 (0:{\ttfamily M00}, 1:non-stop, 2:扉前{\ttfamily M00}) & 1\\
\SetRow{unusingVariables}
\ttNum460 & (不使用) &\\
\ttNum461 & 工具{\ttfamily T31}(Tスロット)A側\dimple~深さ補正値(深さに$+$補正) & 0.06\\
\ttNum462 & 工具{\ttfamily T31}(Tスロット)C側\dimple~深さ補正値(深さに$+$補正) & 0.03\\
\ttNum463 & 工具{\ttfamily T31}(Tスロット)B側\dimple~深さ補正値(深さに$+$補正) & 0.06\\
\ttNum464 & 工具{\ttfamily T31}(Tスロット)D側\dimple~深さ補正値(深さに$+$補正) & 0.03\\
\SetRow{unusingVariables}
\ttNum465 & (不使用) &\\
\ttNum466 & 工具{\ttfamily T32}(Tスロット)A側\dimple~深さ補正値(深さに$+$補正) &\\
\ttNum467 & 工具{\ttfamily T32}(Tスロット)C側\dimple~深さ補正値(深さに$+$補正) &\\
\ttNum468 & 工具{\ttfamily T32}(Tスロット)B側\dimple~深さ補正値(深さに$+$補正) &\\
\ttNum469 & 工具{\ttfamily T32}(Tスロット)D側\dimple~深さ補正値(深さに$+$補正) &\\
\SetRow{unusingVariables}
\ttNum470 & (不使用) &\\
\ttNum471 & 工具{\ttfamily T33}(Tスロット)A側\dimple~深さ補正値(深さに$+$補正) &\\
\ttNum472 & 工具{\ttfamily T33}(Tスロット)C側\dimple~深さ補正値(深さに$+$補正) &\\
\ttNum473 & 工具{\ttfamily T33}(Tスロット)B側\dimple~深さ補正値(深さに$+$補正) &\\
\ttNum474 & 工具{\ttfamily T33}(Tスロット)D側\dimple~深さ補正値(深さに$+$補正) &\\
\end{multicollongtblr}


\clearpage
%%%%%%%%%%%%%%%%%%%%%%%%%%%%%%%%%%%%%%%%%%%%%%%%%%%%%%%%%%
%% subsection 17.3.4 %%%%%%%%%%%%%%%%%%%%%%%%%%%%%%%%%%%%%
%%%%%%%%%%%%%%%%%%%%%%%%%%%%%%%%%%%%%%%%%%%%%%%%%%%%%%%%%%
\subsection{\ttNum475\,-\ttNum499}

\begin{multicollongtblr}[white]{\ttNum475\,-\ttNum499:補助機能(該当明細 少)}{cX[l]c}
変数 & 内容 & 初期値\\
\ttNum475 & 溝幅$Z$方向中央切削(3回加工)(0:off, 3:on) & 0\\
\SetRow{unusingVariables}
\ttNum476 & (不使用) &\\
\SetRow{unusingVariables}
\ttNum477\TBW & (トップ座ぐり$X+$補正用 予備) & 0\\
\SetRow{unusingVariables}
\ttNum478\TBW & (トップ座ぐり径 補正用 予備) & 0\\
\ttNum479 & トップ座ぐり 仕上げ前 一時停止 (0:{\ttfamily M00}, 1:non-stop, 2:扉前{\ttfamily M00}) & 1\\
\ttNum480 & トップ座ぐり 仕上げ後 一時停止 (0:{\ttfamily M00}, 1:non-stop, 2:扉前{\ttfamily M00}) & 1\\
\SetRow{unusingVariables}
\ttNum481 & (不使用) &\\
\SetRow{unusingVariables}
\ttNum482 & (予備) &\\
\SetRow{unusingVariables}
\ttNum483 & (予備) &\\
\SetRow{unusingVariables}
\ttNum484 & (予備) &\\
\SetRow{unusingVariables}
\ttNum485 & (予備) &\\
\SetRow{unusingVariables}
\ttNum486 & (予備) &\\
\SetRow{unusingVariables}
\ttNum487 & (予備) &\\
\SetRow{unusingVariables}
\ttNum488 & (予備) &\\
\SetRow{unusingVariables}
\ttNum489 & (予備) &\\
\SetRow{unusingVariables}
\ttNum490 & (予備) &\\
\SetRow{unusingVariables}
\ttNum491 & (予備) &\\
\SetRow{unusingVariables}
\ttNum492 & (予備) &\\
\SetRow{unusingVariables}
\ttNum493 & (予備) &\\
\SetRow{unusingVariables}
\ttNum494 & (予備) &\\
\SetRow{unusingVariables}
\ttNum495 & (予備) &\\
\SetRow{unusingVariables}
\ttNum496 & (予備) &\\
\SetRow{unusingVariables}
\ttNum497 & (予備) &\\
\SetRow{unusingVariables}
\ttNum498 & (予備) &\\
\SetRow{unusingVariables}
\ttNum499 & (予備) &\\
\end{multicollongtblr}



\clearpage
%%%%%%%%%%%%%%%%%%%%%%%%%%%%%%%%%%%%%%%%%%%%%%%%%%%%%%%%%%
%% section 18.4 %%%%%%%%%%%%%%%%%%%%%%%%%%%%%%%%%%%%%%%%%%
%%%%%%%%%%%%%%%%%%%%%%%%%%%%%%%%%%%%%%%%%%%%%%%%%%%%%%%%%%
\modHeadsection{\ttNum500\,-\ttNum574:バンドルプログラムの使用コモン変数}
\ttNum500\,-\ttNum574については、主に\index{バンドルのプログラム}バンドルのプログラム\prgbox{O910x}\prgbox{O93xx}で使用されているものである。
%%%%%%%%%%%%%%%%%%%%%%%%%%%%%%%%%%%%%%%%%%%%%%%%%%%%%%%%%%
%% marker %%%%%%%%%%%%%%%%%%%%%%%%%%%%%%%%%%%%%%%%%%%%%%%%
%%%%%%%%%%%%%%%%%%%%%%%%%%%%%%%%%%%%%%%%%%%%%%%%%%%%%%%%%%
\begin{marker}
これらの値は2023/09/26機械設置時に入力されていたもの
\end{marker}
%%%%%%%%%%%%%%%%%%%%%%%%%%%%%%%%%%%%%%%%%%%%%%%%%%%%%%%%%%
%%%%%%%%%%%%%%%%%%%%%%%%%%%%%%%%%%%%%%%%%%%%%%%%%%%%%%%%%%
%%%%%%%%%%%%%%%%%%%%%%%%%%%%%%%%%%%%%%%%%%%%%%%%%%%%%%%%%%

\begin{multicollongtblr}[white]{\ttNum500\,-\ttNum574:\prgbox{O910x}\prgbox{O93xx}用}{cX[l]c}
変数 & 内容 & 設定値\\
\ttNum500 & 芯ずれ許容差 \prgbox{O93xx} & 5\\
\ttNum501 & タッチセンサー信号遅れ補正 \prgbox{O93xx} & 0.040\\
\ttNum502 & タッチセンサープローブ中心$X$補正 \prgbox{O93xx} & -0.016507\\
\ttNum503 & タッチセンサープローブ中心$Y$補正 \prgbox{O93xx} & -0.068371\\
\ttNum504 & 測定距離 \prgbox{O910x} & 5\\
\ttNum505 & プローブ表面からプログラムの加工原点($Z$0)までの距離 \prgbox{O910x} & 785.529\\
\ttNum506 & 工具長の変化の許容差 \prgbox{O910x} & 1.0\\
\ttNum507 & 工具破損検出の許容差 \prgbox{O910x} & 1.0\\
\ttNum508 & (不明) & \ttNum0\\
\ttNum509 & $Z$座標系設定 \prgbox{O93xx} & 441.432\\
\ttNum510 & (不明) & 270\\
\ttNum511 & インチ/ミリ切替 \prgbox{O910x} & \ttNum0\\
\ttNum512 & タッチセンサープローブ半径$\mathrm{mm}$値 \prgbox{O93xx} & 5.0\\
\ttNum513 & 移動時用の送り速さ値 \prgbox{O910x} & 1000\\
\ttNum514 & スキップ({\ttfamily G31})測定時用 送り速さ値 \prgbox{O910x}\prgbox{O93xx} & 50\\
\ttNum515 & (不明) & \ttNum0\\
\ttNum516 & センサーの位置$X$座標 \prgbox{O910x} & -30.374\\
\ttNum517 & センサーの位置$Y$座標 \prgbox{O910x} & -913.761\\
\ttNum518 & センサーの位置$Z$座標 & -785.529\\
\ttNum519 & (不明) & 6\\
\ttNum520 & 拡張ワーク座標系 \prgbox{O910x} & 1861\\
\ttNum521 & (不明) & 0\\
\ttNum522 & (不明) & 0\\
\ttNum523 & アプローチ時用の送り速さ値 \prgbox{O910x} & 30\\
\ttNum524 & 測定時用の送り速さ値 \prgbox{O910x} & 3\\
$\cdots$ & (以下不明) & \ttNum0\\
\ttNum533 & (不明) & 0\\
\ttNum534 & (不明) & 0\\
$\cdots$ & (以下不明) & \ttNum0
\end{multicollongtblr}
%%%%%%%%%%%%%%%%%%%%%%%%%%%%%%%%%%%%%%%%%%%%%%%%%%%%%%%%%%
%% marker %%%%%%%%%%%%%%%%%%%%%%%%%%%%%%%%%%%%%%%%%%%%%%%%
%%%%%%%%%%%%%%%%%%%%%%%%%%%%%%%%%%%%%%%%%%%%%%%%%%%%%%%%%%
\begin{marker}
\ttNum505, \ttNum507, \ttNum516, \ttNum517は作成したプログラムでもRHSとして使用していることに注意
\end{marker}
%%%%%%%%%%%%%%%%%%%%%%%%%%%%%%%%%%%%%%%%%%%%%%%%%%%%%%%%%%
%%%%%%%%%%%%%%%%%%%%%%%%%%%%%%%%%%%%%%%%%%%%%%%%%%%%%%%%%%
%%%%%%%%%%%%%%%%%%%%%%%%%%%%%%%%%%%%%%%%%%%%%%%%%%%%%%%%%%



\clearpage
%%%%%%%%%%%%%%%%%%%%%%%%%%%%%%%%%%%%%%%%%%%%%%%%%%%%%%%%%%
%% section 11.5 %%%%%%%%%%%%%%%%%%%%%%%%%%%%%%%%%%%%%%%%%%
%%%%%%%%%%%%%%%%%%%%%%%%%%%%%%%%%%%%%%%%%%%%%%%%%%%%%%%%%%
\modHeadsection{\ttNum600\,-\ttNum699}


%%%%%%%%%%%%%%%%%%%%%%%%%%%%%%%%%%%%%%%%%%%%%%%%%%%%%%%%%%
%% subsection 11.5.1 %%%%%%%%%%%%%%%%%%%%%%%%%%%%%%%%%%%%%
%%%%%%%%%%%%%%%%%%%%%%%%%%%%%%%%%%%%%%%%%%%%%%%%%%%%%%%%%%
\subsection{\ttNum600\,-\ttNum624:ワークと工具間の距離の調整}
\ttNum600\,-\ttNum624については、主に\index{ワーク}ワークと工具間の距離に関するものとする。\\

\begin{multicollongtblr}[white]{\ttNum600\,-\ttNum624:ワークと工具間の距離の調整}{cX[l]c}
変数 & 内容 & 設定値\\
\ttNum600 & 工具 - 端面間 $Z$方向クリアランス平面間距離 & 100.0\\
\SetRow{unusingVariables}
\ttNum601 & (予備) &\\
\ttNum602 & タッチセンサー計測時の近付き量 & 10.0\\
\ttNum603 & タッチセンサー計測時の行過ぎ量 & 3.0\\
\SetRow{unusingVariables}
\ttNum604 & (予備) &\\
\ttNum605 & 外側加工面 法線方向クリアランス平面間距離 最小値 & 30.0\\
\ttNum606 & 内側加工面 法線方向クリアランス平面間距離 最小値 & 15.0\\
\SetRow{unusingVariables}
\ttNum607 & (予備) &\\
\ttNum608 & 端面加工用 内径輪郭径$-$補正量 & 5.0\\
\SetRow{unusingVariables}
\ttNum609 & (予備) &\\
\SetRow{unusingVariables}
\ttNum610 & (外削加工用予備) & \\
\SetRow{unusingVariables}
\ttNum611 & (予備) &\\
\SetRow{unusingVariables}
\ttNum612 & (溝加工用予備) & \\
\SetRow{unusingVariables}
\ttNum613 & (予備) &\\
\SetRow{unusingVariables}
\ttNum614 & (外面取加工用予備) & \\
\SetRow{unusingVariables}
\ttNum615 & (予備) &\\
\SetRow{unusingVariables}
\ttNum616 & (内面取加工用予備) &\\
\SetRow{unusingVariables}
\ttNum617 & (予備) &\\
\SetRow{unusingVariables}
\ttNum618 & (座ぐり加工用予備) &\\
\SetRow{unusingVariables}
\ttNum619 & (予備) &\\
\ttNum620 & \dimple 加工用 加工時近付き量 & 1.0 \\
\SetRow{unusingVariables}
\ttNum621 & (予備) &\\
\SetRow{unusingVariables}
\ttNum622 & (予備) &\\
\SetRow{unusingVariables}
\ttNum623 & (予備) &\\
\SetRow{unusingVariables}
\ttNum624 & (予備) &\\
\end{multicollongtblr}


\clearpage
%%%%%%%%%%%%%%%%%%%%%%%%%%%%%%%%%%%%%%%%%%%%%%%%%%%%%%%%%%
%% subsection 11.5.2 %%%%%%%%%%%%%%%%%%%%%%%%%%%%%%%%%%%%%
%%%%%%%%%%%%%%%%%%%%%%%%%%%%%%%%%%%%%%%%%%%%%%%%%%%%%%%%%%
\subsection{\ttNum625\,-\ttNum649:残り代および1回あたりの削り代}
\ttNum625\,-\ttNum649については、各加工の\index{のこりけずりしろ@残り削り代}残り削り代や\index{1かいあたりのけずりしろ@1回あたりの削り代}1回あたりの削り代に関するものとする。\\

\begin{multicollongtblr}[white]{\ttNum625\,-\ttNum649:残り代および1回あたりの削り代}{cX[l]c}
変数 & 内容 & 初期値\\
\ttNum625 & 端面加工1回あたりの$Z$方向削り代 & 4.0\\
\SetRow{unusingVariables}
\ttNum626 & (予備) & \\
\ttNum627 & 外削加工1回あたりの削り代(直径) & 2.0\\
\ttNum628 & 外削加工 仕上げ前 残り削り代(直径) & 1.0\\
\ttNum629 & 溝加工1回あたりの削り代(溝深さ) & 5.0\\
\ttNum630 & 溝加工 仕上げ前 残り削り代(直径) & 1.0\\
\ttNum631 & 外面取加工1回あたりの削り代(直径) & 2.000\\
\ttNum632 & 外面取加工 仕上げ前 残り削り代(直径) & 1.155\\
\ttNum633 & 内面取加工1回あたりの削り代(直径) & 2.387\\
\ttNum634 & 内面取加工 仕上げ前 残り削り代(直径) & 1.0\\
\SetRow{unusingVariables}
\ttNum635 & (座ぐり加工用予備) & \\
\SetRow{unusingVariables}
\ttNum636 & (座ぐり加工用予備) & \\
\SetRow{unusingVariables}
\ttNum637 & (予備) & \\
\SetRow{unusingVariables}
\ttNum638 & (予備) & \\
\SetRow{unusingVariables}
\ttNum639 & (予備) & \\
\SetRow{unusingVariables}
\ttNum640 & (予備) & \\
\SetRow{unusingVariables}
\ttNum641 & (予備) & \\
\SetRow{unusingVariables}
\ttNum642 & (予備) & \\
\SetRow{unusingVariables}
\ttNum643 & (予備) & \\
\SetRow{unusingVariables}
\ttNum644 & (予備) & \\
\SetRow{unusingVariables}
\ttNum645 & (予備) & \\
\SetRow{unusingVariables}
\ttNum646 & (予備) & \\
\SetRow{unusingVariables}
\ttNum647 & (予備) & \\
\SetRow{unusingVariables}
\ttNum648 & (予備) & \\
\SetRow{unusingVariables}
\ttNum649 & (予備) & \\
\end{multicollongtblr}


\clearpage
%%%%%%%%%%%%%%%%%%%%%%%%%%%%%%%%%%%%%%%%%%%%%%%%%%%%%%%%%%
%% subsection 17.5.3 %%%%%%%%%%%%%%%%%%%%%%%%%%%%%%%%%%%%%
%%%%%%%%%%%%%%%%%%%%%%%%%%%%%%%%%%%%%%%%%%%%%%%%%%%%%%%%%%
\subsection{\ttNum650\,-\ttNum674:工具の送り速さ}
\ttNum650\,-\ttNum674については、工具の\index{おくりはやさ@送り速さ}送り速さに関するものとする。\\

\begin{multicollongtblr}[white]{\ttNum650\,-\ttNum674:工具の送り速さ}{cX[l]c}
変数 & 内容 & 設定値\\
\ttNum650 & 早送り$Z$:{\ttfamily T50}アプローチ以外 & 15000\\
\ttNum651 & アプローチ・$XY$リトラクト:{\ttfamily T50}以外 & 6000\\
\ttNum652 & 早送り$XY$:{\ttfamily T50}, 早送り$Z$:{\ttfamily T50}アプローチ & 5400\\
\ttNum653 & アプローチ:{\ttfamily T50} & 1500\\
\ttNum654 & アプローチ:工具長測定 & 960\\
\ttNum655 & 測定時アプローチ:{\ttfamily T50}(両側測定) & 200\\
\ttNum656 & 測定時アプローチ:{\ttfamily T50}(片側測定) & 50\\
\SetRow{unusingVariables}
\ttNum657 & (不使用) &\\
\ttNum658 & 端面加工:直線 & 900\\
\ttNum659 & 端面加工:コーナー & 900\\
\ttNum660 & 外削加工:直線 & 400\\
\ttNum661 & 外削加工:コーナー & 400\\
\ttNum662 & 溝加工:直線 & 500\\
\ttNum663 & 溝加工:コーナー & 400\\
\ttNum664 & 外面取加工:直線 & 500\\
\ttNum665 & 外面取加工:コーナー & 400\\
\ttNum666 & 内面取加工:直線 & 400\\
\ttNum667 & 内面取加工:コーナー & 400\\
\ttNum668 & 座ぐり加工:直線 & 40\\
\ttNum669 & 座ぐり加工:コーナー & 40\\
\ttNum670 & \dimple 測定:表面アプローチ & 600\\
\ttNum671 & \dimple 加工:加工 & 100\\
\SetRow{unusingVariables}
\ttNum672 & (予備) & \\
\SetRow{unusingVariables}
\ttNum673 & (予備) & \\
\SetRow{unusingVariables}
\ttNum674 & (予備) & \\
\end{multicollongtblr}


\clearpage
%%%%%%%%%%%%%%%%%%%%%%%%%%%%%%%%%%%%%%%%%%%%%%%%%%%%%%%%%%
%% subsection 17.5.4 %%%%%%%%%%%%%%%%%%%%%%%%%%%%%%%%%%%%%
%%%%%%%%%%%%%%%%%%%%%%%%%%%%%%%%%%%%%%%%%%%%%%%%%%%%%%%%%%
\subsection{\ttNum675\,-\ttNum699:スピンドルの回転数(切削時)}
\ttNum675\,-\ttNum699については、切削時の工具の\index{しゅじくかいてんすう@主軸回転数}主軸回転数に関するものとする。\\

\begin{multicollongtblr}[white]{\ttNum675\,-\ttNum699:主軸回転数(切削時)}{cX[l]c}
変数 & 内容 & 設定値\\
\ttNum675 & 端面加工 & 950\\
\ttNum676 & 外削加工 & 930\\
\ttNum677 & 溝加工 & 930\\
\ttNum678 & 外面取加工 & 1800\\
\ttNum679 & 内面取加工 & 1800\\
\ttNum680 & 座ぐり加工 & 500\\
\ttNum681 & \dimple 加工 & 2000\\
\SetRow{unusingVariables}
\ttNum682 & (予備) & \\
\SetRow{unusingVariables}
\ttNum683 & (予備) & \\
\SetRow{unusingVariables}
\ttNum684 & (予備) & \\
\SetRow{unusingVariables}
\ttNum685 & (予備) & \\
\SetRow{unusingVariables}
\ttNum686 & (予備) & \\
\SetRow{unusingVariables}
\ttNum687 & (予備) & \\
\SetRow{unusingVariables}
\ttNum688 & (予備) & \\
\SetRow{unusingVariables}
\ttNum689 & (予備) & \\
\SetRow{unusingVariables}
\ttNum690 & (予備) & \\
\SetRow{unusingVariables}
\ttNum691 & (予備) & \\
\SetRow{unusingVariables}
\ttNum692 & (予備) & \\
\SetRow{unusingVariables}
\ttNum693 & (予備) & \\
\SetRow{unusingVariables}
\ttNum694 & (予備) & \\
\SetRow{unusingVariables}
\ttNum695 & (予備) & \\
\SetRow{unusingVariables}
\ttNum696 & (予備) & \\
\SetRow{unusingVariables}
\ttNum697 & (予備) & \\
\SetRow{unusingVariables}
\ttNum698 & (予備) & \\
\SetRow{unusingVariables}
\ttNum699 & (予備) & \\
\end{multicollongtblr}



\clearpage
%%%%%%%%%%%%%%%%%%%%%%%%%%%%%%%%%%%%%%%%%%%%%%%%%%%%%%%%%%
%% section 11.6 %%%%%%%%%%%%%%%%%%%%%%%%%%%%%%%%%%%%%%%%%%
%%%%%%%%%%%%%%%%%%%%%%%%%%%%%%%%%%%%%%%%%%%%%%%%%%%%%%%%%%
\modHeadsection{\ttNum700\,-\ttNum750:\dimple 用}
\ttNum700\,-\ttNum750については、主に\expandafterindex{\dimplekana そくていようサブプログラム@\dimple 測定用サブプログラム}\dimple 測定用サブプログラム\prgbox{O2x000x}で使用されるものとする。


%%%%%%%%%%%%%%%%%%%%%%%%%%%%%%%%%%%%%%%%%%%%%%%%%%%%%%%%%%
%% subsection 18.6.1 %%%%%%%%%%%%%%%%%%%%%%%%%%%%%%%%%%%%%
%%%%%%%%%%%%%%%%%%%%%%%%%%%%%%%%%%%%%%%%%%%%%%%%%%%%%%%%%%
\subsection{\ttNum700\,-\ttNum724:\dimple~\DLone 用}

\begin{multicollongtblr}[white]{\ttNum700\,-\ttNum724:\dimple~移動 \DLone 用}{cX[l]}
変数 & 内容\\
\SetRow{unusingVariables}
\ttNum700 & (予備)\\
\ttNum701 & プログラム読込み時のワーク座標系(\ttNum4012)\\
\ttNum702 & 工具別$Z$補正({\ttfamily T50}:\ttNum512, {\ttfamily T3}x:0)\\
\ttNum703 & 工具別$XY$補正({\ttfamily T50}:\ttNum512, {\ttfamily T3}x:\ttNum[2400+\ttNum4111]+\ttNum[2600+\ttNum4111])\\
\ttNum704 & 工具別移動{\ttfamily G\#} ({\ttfamily T50}:31, {\ttfamily T3}x:1)\\
\ttNum705 & テーブル中心からワーク座標(\ttNum701)原点までの$X$距離\\
\ttNum706 & 傾き後のトップ端面中心(機械座標)$X$ (\cf\pageeqref{eq:afterPhiTCenterFromO})\\
\ttNum707 & テーブル中心から傾き後のトップ端面中心までの$Z$距離 (\cf\pageeqref{eq:afterPhiTCenterFromO})\\
\ttNum708 & 傾き後トップ端中心(ブロックエンド)$X$座標(\ttNum5001)\\
\ttNum709 & 傾き後トップ端中心(ブロックエンド)$Z$座標(\ttNum5003)\\
\ttNum710 & テーブル中心から\dimple1列目までの$Z$距離$Z-q$\\
\ttNum711 & トップ端中心から\dimple1列目中心までの$X$距離(\cf\pageeqref{eq:dimpleCenterDistance})\\
\ttNum712 & 傾き後\dimple1列目中心$X$移動距離(\cf\pageeqref{eq:afterPhidimpleCenterDistance})\\
\ttNum713 & 傾き後\dimple1列目中心$Z$移動距離(\cf\pageeqref{eq:afterPhidimpleCenterDistance})\\
\ttNum714 & 傾き後\dimple1列目中心(ブロックエンド)$X$座標 (\ttNum5001)\\
\ttNum715 & 傾き後\dimple1列目中心(ブロックエンド)$Y$座標 (\ttNum5002)\\
\ttNum716 & 傾き後\dimple1列目中心(ブロックエンド)$Z$座標 (\ttNum5003)\\
\ttNum717\color{red}$^*$ & 各面 ループ用数値(1:A, 2:C, 3:B, 4:D)\\
\ttNum718 & BD内半径$-\text{\ttNum703}-\text{\ttNum602}$\\
\ttNum719 & (AC内半径$-\text{\ttNum703}-\text{\ttNum602})\cos\phi$\\
\SetRow{unusingVariables}
\ttNum720 & (予備)\\
\SetRow{unusingVariables}
\ttNum721 & (予備)\\
\SetRow{unusingVariables}
\ttNum722 & (予備)\\
\SetRow{unusingVariables}
\ttNum723 & (予備)\\
\SetRow{unusingVariables}
\ttNum724 & (予備)\\
\end{multicollongtblr}
%%%%%%%%%%%%%%%%%%%%%%%%%%%%%%%%%%%%%%%%%%%%%%%%%%%%%%%%%%
%% hosoku %%%%%%%%%%%%%%%%%%%%%%%%%%%%%%%%%%%%%%%%%%%%%%%%
%%%%%%%%%%%%%%%%%%%%%%%%%%%%%%%%%%%%%%%%%%%%%%%%%%%%%%%%%%
\begin{marker}
\ttNum717は\DLtwoAC および\DLtwoBD で\index{RHS(コモンへんすう)@RHS(コモン変数)}RHSとして使用していることに注意
\end{marker}
%%%%%%%%%%%%%%%%%%%%%%%%%%%%%%%%%%%%%%%%%%%%%%%%%%%%%%%%%%
%%%%%%%%%%%%%%%%%%%%%%%%%%%%%%%%%%%%%%%%%%%%%%%%%%%%%%%%%%
%%%%%%%%%%%%%%%%%%%%%%%%%%%%%%%%%%%%%%%%%%%%%%%%%%%%%%%%%%


\clearpage
%%%%%%%%%%%%%%%%%%%%%%%%%%%%%%%%%%%%%%%%%%%%%%%%%%%%%%%%%%
%% subsection 18.6.2 %%%%%%%%%%%%%%%%%%%%%%%%%%%%%%%%%%%%%
%%%%%%%%%%%%%%%%%%%%%%%%%%%%%%%%%%%%%%%%%%%%%%%%%%%%%%%%%%
\subsection{\ttNum725\,-\ttNum749:\dimple~\DLtwoAC\DLtwoBD 用}

\begin{multicollongtblr}[white]{\ttNum725\,-\ttNum744:\dimple~移動 \DLtwoAC\DLtwoBD 用}{cX[l]}
変数 & 内容\\
\ttNum725 & プログラム読込時ブロックエンド$Y$ or $X$ (\ttNum5002, \ttNum5001)\\
\ttNum726 & プログラム読込時ブロックエンド$Z$ (\ttNum5003)\\
\ttNum727 & \dimple~偶数列の列数\\
\ttNum728 & \dimple~偶数列(一列)の\dimple~数\\
\ttNum729 & \dimple~奇数列(一列)の\dimple~数\\
\ttNum730 & \dimple~現在の列の\dimple~数\\
\SetRow{unusingVariables}
\ttNum731 & (予備)\\
\SetRow{unusingVariables}
\ttNum732 & (予備)\\
\SetRow{unusingVariables}
\ttNum733 & (予備)\\
\SetRow{unusingVariables}
\ttNum734 & (予備)\\
\SetRow{unusingVariables}
\ttNum735 & (予備)\\
\SetRow{unusingVariables}
\ttNum736 & (予備)\\
\SetRow{unusingVariables}
\ttNum737 & (予備)\\
\SetRow{unusingVariables}
\ttNum738 & (予備)\\
\SetRow{unusingVariables}
\ttNum739 & (予備)\\
\SetRow{unusingVariables}
\ttNum740 & (予備)\\
\SetRow{unusingVariables}
\ttNum741 & (予備)\\
\SetRow{unusingVariables}
\ttNum742 & (予備)\\
\SetRow{unusingVariables}
\ttNum743 & (予備)\\
\SetRow{unusingVariables}
\ttNum744 & (予備)\\
\end{multicollongtblr}


%\clearpage
\begin{multicollongtblr}[white]{\ttNum745\,-\ttNum749:\dimple~測定~\DMLthreeAC\DMLthreeBD 用}{cX[l]}
変数 & 内容\\
\SetRow{unusingVariables}
\ttNum745 & (予備)\\
\SetRow{unusingVariables}
\ttNum746 & (予備)\\
\SetRow{unusingVariables}
\ttNum747 & (予備)\\
\ttNum748 & プログラム読込時ブロックエンド$X$ or $Y$ (\ttNum5001, \ttNum5002)\\
\ttNum749\color{red}$^*$ & \dimple~表面位置$X$ or $Y$測定値
\end{multicollongtblr}
%%%%%%%%%%%%%%%%%%%%%%%%%%%%%%%%%%%%%%%%%%%%%%%%%%%%%%%%%%
%% hosoku %%%%%%%%%%%%%%%%%%%%%%%%%%%%%%%%%%%%%%%%%%%%%%%%
%%%%%%%%%%%%%%%%%%%%%%%%%%%%%%%%%%%%%%%%%%%%%%%%%%%%%%%%%%
\begin{marker}
\ttNum749は\DLtwoAC および\DLtwoBD で\index{RHS(コモンへんすう)@RHS(コモン変数)}RHSとして使用していることに注意
\end{marker}
%%%%%%%%%%%%%%%%%%%%%%%%%%%%%%%%%%%%%%%%%%%%%%%%%%%%%%%%%%
%%%%%%%%%%%%%%%%%%%%%%%%%%%%%%%%%%%%%%%%%%%%%%%%%%%%%%%%%%
%%%%%%%%%%%%%%%%%%%%%%%%%%%%%%%%%%%%%%%%%%%%%%%%%%%%%%%%%%



\clearpage
%%%%%%%%%%%%%%%%%%%%%%%%%%%%%%%%%%%%%%%%%%%%%%%%%%%%%%%%%%
%% section 19.7 %%%%%%%%%%%%%%%%%%%%%%%%%%%%%%%%%%%%%%%%%%
%%%%%%%%%%%%%%%%%%%%%%%%%%%%%%%%%%%%%%%%%%%%%%%%%%%%%%%%%%
\modHeadsection{\ttNum900001\,-\ttNum900031, \ttNum900101\,-\ttNum900500:実測値}
\ttNum900001\,-\ttNum900500については、\index{じっそくち@実測値}実測値または計算値を格納する。


%%%%%%%%%%%%%%%%%%%%%%%%%%%%%%%%%%%%%%%%%%%%%%%%%%%%%%%%%%
%% subsection 19.7.1 %%%%%%%%%%%%%%%%%%%%%%%%%%%%%%%%%%%%%
%%%%%%%%%%%%%%%%%%%%%%%%%%%%%%%%%%%%%%%%%%%%%%%%%%%%%%%%%%
\subsection{\ttNum900001\,-\ttNum900049:\dimple 以外}

\begin{multicollongtblr}[white]{\ttNum900001\,-\ttNum900005:外中心$X$ 両側測定用 \MXOThickness}{cX[l]}
変数 & 内容\\
\ttNum900001 & $X$外中心測定 $-X$側測定値\\
\ttNum900002 & $X$外中心測定 $+X$側測定値\\
\ttNum900003 & $X$外中心測定値\\
\ttNum900004 & $X$外中心測定 厚さ測定値\\
\SetRow{unusingVariables}
\ttNum900005 & (予備)\\
\end{multicollongtblr}


\begin{multicollongtblr}[white]{\ttNum900006\,-\ttNum900010:外中心$Y$ 両側測定用 \MYOThickness}{cX[l]}
変数 & 内容\\
\ttNum900006 & $Y$外中心測定 $-Y$側測定値\\
\ttNum900007 & $Y$外中心測定 $+Y$側測定値\\
\ttNum900008 & $Y$外中心測定値\\
\ttNum900009 & $Y$外中心測定 厚さ測定値\\
\SetRow{unusingVariables}
\ttNum900010 & (予備)\\
\end{multicollongtblr}


%\clearpage
\begin{multicollongtblr}[white]{\ttNum900011\,-\ttNum900013:溝中心$X$ 片側測定用 \MXOface}{cX[l]}
変数 & 内容\\
\ttNum900011 & $X$溝中心測定 A側外面測定値\\
\SetRow{unusingVariables}
\ttNum900012 & (予備)\\
\SetRow{unusingVariables}
\ttNum900013 & (予備)\\
\end{multicollongtblr}


%\clearpage
\begin{multicollongtblr}[white]{\ttNum900014\,-\ttNum900018:内中心$X$ 両側測定用 \MXIWidth}{cX[l]}
変数 & 内容\\
\ttNum900014 & $X$内中心測定 $-X$側測定値\\
\ttNum900015 & $X$内中心測定 $+X$側測定値\\
\ttNum900016 & $X$内中心測定値\\
\ttNum900017 & $X$内中心測定 厚さ測定値\\
\SetRow{unusingVariables}
\ttNum900018 & (予備)\\
\end{multicollongtblr}


\clearpage
\begin{multicollongtblr}[white]{\ttNum900019\,-\ttNum900023:内中心$Y$ 両側測定用 \MYIWidth}{cX[l]}
変数 & 内容\\
\ttNum900019 & $Y$内中心測定 $-Y$側測定値\\
\ttNum900020 & $Y$内中心測定 $+Y$側測定値\\
\ttNum900021 & $Y$内中心測定値\\
\ttNum900022 & $Y$内中心測定 厚さ測定値\\
\SetRow{unusingVariables}
\ttNum900023 & (予備)\\
\end{multicollongtblr}


%\clearpage
\begin{multicollongtblr}[white]{\ttNum900024\,-\ttNum900026:外削中心$X$ 片側測定用 \MXIface}{cX[l]}
変数 & 内容\\
\ttNum900024 & $X$外削中心測定 内面測定値\\
\SetRow{unusingVariables}
\ttNum900025 & (予備)\\
\SetRow{unusingVariables}
\ttNum900026 & (予備)\\
\end{multicollongtblr}


%\clearpage
\begin{multicollongtblr}[white]{\ttNum900027\,-\ttNum900030:通り芯$Y$ 片側測定用 \MYcenterline}{cX[l]}
変数 & 内容\\
\ttNum900027 & $Y$通り芯 ボトム側測定値\\
\ttNum900028 & $Y$通り芯 トップ側測定値\\
\ttNum900029 & $Y$通り芯 測定値\\
\SetRow{unusingVariables}
\ttNum900030 & (予備)\\
\end{multicollongtblr}


\begin{multicollongtblr}[white]{\ttNum900031\,-\ttNum900034:通り芯$X$ 片側測定用 \MXcenterline}{cX[l]}
変数 & 内容\\
\ttNum900031 & $X$通り芯 トップ側測定値\\
\ttNum900032 & $X$通り芯 ボトム側測定値\\
\ttNum900033 & $X$通り芯 測定値\\
\SetRow{unusingVariables}
\ttNum900034 & (予備)\\
\end{multicollongtblr}


\clearpage
%%%%%%%%%%%%%%%%%%%%%%%%%%%%%%%%%%%%%%%%%%%%%%%%%%%%%%%%%%
%% subsection 18.7.2 %%%%%%%%%%%%%%%%%%%%%%%%%%%%%%%%%%%%%
%%%%%%%%%%%%%%%%%%%%%%%%%%%%%%%%%%%%%%%%%%%%%%%%%%%%%%%%%%
\subsection{\ttNum900101\,-\ttNum900500:\dimple}

%%%%%%%%%%%%%%%%%%%%%%%%%%%%%%%%%%%%%%%%%%%%%%%%%%%%%%%%%%
%% subsubsection 18.7.2.1 %%%%%%%%%%%%%%%%%%%%%%%%%%%%%%%%
%%%%%%%%%%%%%%%%%%%%%%%%%%%%%%%%%%%%%%%%%%%%%%%%%%%%%%%%%%
\subsubsection{\ttNum900101\,-\ttNum900300:\dimple AC面}

\begin{multicollongtblr}[white]{\ttNum900101\,-\ttNum900300:\dimple AC表面位置 測定値 \DMLthreeAC}{cX[l]}
変数 & 内容\\
\ttNum900101\,-\ttNum900200 & A側\dimple~表面位置$X$ 測定値\\
\ttNum900201\,-\ttNum900300 & C側\dimple~表面位置$X$ 測定値
\end{multicollongtblr}

%%%%%%%%%%%%%%%%%%%%%%%%%%%%%%%%%%%%%%%%%%%%%%%%%%%%%%%%%%
%% subsubsection 18.7.2.2 %%%%%%%%%%%%%%%%%%%%%%%%%%%%%%%%
%%%%%%%%%%%%%%%%%%%%%%%%%%%%%%%%%%%%%%%%%%%%%%%%%%%%%%%%%%
\subsubsection{\ttNum900301\,-\ttNum900500:\dimple BD面}

\begin{multicollongtblr}[white]{\ttNum900301\,-\ttNum900500:\dimple BD表面位置 測定値 \DMLthreeBD}{cX[l]}
変数 & 内容\\
\ttNum900301\,-\ttNum900400 & B側\dimple~表面位置$Y$ 測定値\\
\ttNum900401\,-\ttNum900500 & D側\dimple~表面位置$Y$ 測定値
\end{multicollongtblr}



\clearpage
%%%%%%%%%%%%%%%%%%%%%%%%%%%%%%%%%%%%%%%%%%%%%%%%%%%%%%%%%%
%% section 11.8 %%%%%%%%%%%%%%%%%%%%%%%%%%%%%%%%%%%%%%%%%%
%%%%%%%%%%%%%%%%%%%%%%%%%%%%%%%%%%%%%%%%%%%%%%%%%%%%%%%%%%
\modHeadsection{\ttNum901000\,-\ttNum901024:パレット・ジグ}
\ttNum901000\,-\ttNum901024については、主に\index{パレット}パレットや\index{ジグ}ジグに関するものとする。\\

\begin{multicollongtblr}[white]{\ttNum901000\,-\ttNum901024:主にパレット・ジグ}{cX[l]c}
変数 & 内容 & 設定値\\
\SetRow{unusingVariables}
\ttNum901000 & (予備) &\\
\ttNum901001 & パレット\ttNum1 ジグ中心機械座標$X$ & -550.019\\
\ttNum901002 & パレット\ttNum1 ジグ中心機械座標$Y$ & -739.555\\
\ttNum901003 & パレット\ttNum1 ジグ中心機械座標$Z$ & -1149.974\\
\ttNum901004 & パレット\ttNum1 ジグ中心機械座標$B$ & 0.073\\
\ttNum901005 & パレット\ttNum2 ジグ中心機械座標$X$ & -550.019\\
\ttNum901006 & パレット\ttNum2 ジグ中心機械座標$Y$ & -739.555\\
\ttNum901007 & パレット\ttNum2 ジグ中心機械座標$Z$ & -1149.974\\
\ttNum901008 & パレット\ttNum2 ジグ中心機械座標$B$ & 0.073\\
\ttNum901009 & 工具中心機械座標$C$ & 0\\
\SetRow{unusingVariables}
\ttNum901010 & (予備) &\\
\ttNum901011 & パレット\ttNum1 ジグ外側幅$2l$(機械座標系$B$0 $Z$方向) & 660.0\\
\ttNum901012 & パレット\ttNum1 ジグ内側幅(機械座標系$B$0 $Z$方向) & 410.0\\
\ttNum901013 & パレット\ttNum1 ジグ幅(機械座標系$B$0 $X$方向) & 455.0\\
\SetRow{unusingVariables}
\ttNum901014 & パレット\ttNum1 ジグ底面-固定用ボルト取付具間距離(高さ, $Y$方向) &\\
\SetRow{unusingVariables}
\ttNum901015 & パレット\ttNum1 ジグ底面-固定用ボルト取付具間距離(幅, 機械座標系$B$0 $X$方向) &\\
\SetRow{unusingVariables}
\ttNum901016 & (予備) &\\
\ttNum901017 & パレット\ttNum2 ジグ外側幅$2l$(機械座標系$B$0における$Z$方向) & 660.0\\
\ttNum901018 & パレット\ttNum2 ジグ内側幅(機械座標系$B$0における$Z$方向) & 410.0\\
\ttNum901019 & パレット\ttNum2 ジグ幅(機械座標系$B$0における$X$方向) & 455.0\\
\SetRow{unusingVariables}
\ttNum901020 & パレット\ttNum2 ジグ底面-固定用ボルト取付具間距離(高さ, $Y$方向) &\\
\SetRow{unusingVariables}
\ttNum901021 & パレット\ttNum2 ジグ底面-固定用ボルト取付具間距離(幅, 機械座標系$B$0 $X$方向) &\\
\SetRow{unusingVariables}
\ttNum901022 & (予備) &\\
\SetRow{unusingVariables}
\ttNum901023 & (予備) &\\
\SetRow{unusingVariables}
\ttNum901024 & (予備) &\\
\end{multicollongtblr}
%%%%%%%%%%%%%%%%%%%%%%%%%%%%%%%%%%%%%%%%%%%%%%%%%%%%%%%%%%
%% marker %%%%%%%%%%%%%%%%%%%%%%%%%%%%%%%%%%%%%%%%%%%%%%%%
%%%%%%%%%%%%%%%%%%%%%%%%%%%%%%%%%%%%%%%%%%%%%%%%%%%%%%%%%%
\begin{marker}
\index{ジグのちゅうしん@ジグの中心}ジグ中心\index{きかいざひょう@機械座標}機械座標については2023/09/26時点のもの。
\end{marker}
%%%%%%%%%%%%%%%%%%%%%%%%%%%%%%%%%%%%%%%%%%%%%%%%%%%%%%%%%%
%%%%%%%%%%%%%%%%%%%%%%%%%%%%%%%%%%%%%%%%%%%%%%%%%%%%%%%%%%
%%%%%%%%%%%%%%%%%%%%%%%%%%%%%%%%%%%%%%%%%%%%%%%%%%%%%%%%%%
\clearpage
%%%%%%%%%%%%%%%%%%%%%%%%%%%%%%%%%%%%%%%%%%%%%%%%%%%%%%%%%%
%% hosoku %%%%%%%%%%%%%%%%%%%%%%%%%%%%%%%%%%%%%%%%%%%%%%%%
%%%%%%%%%%%%%%%%%%%%%%%%%%%%%%%%%%%%%%%%%%%%%%%%%%%%%%%%%%
\begin{hosoku}
その他の(\index{ずめん(ジグ)@図面(ジグ)}図面上の)\index{すんぽう(ジグ)@寸法(ジグ)}寸法として、
\begin{enumerate}
\item テーブル中心 と C面側ジグ端 との水平距離:196.5
\item 受板の円の半径$\rho$:100
\item 受板の鉛直方向の幅$\sigma$:40
\item テーブル中心 と 受板の円の中心 との水平距離$\varDelta$:201.5
\item 受板の円の中心 と 受板の水平方向の底 との距離:70
\end{enumerate}
\end{hosoku}
%%%%%%%%%%%%%%%%%%%%%%%%%%%%%%%%%%%%%%%%%%%%%%%%%%%%%%%%%%
%%%%%%%%%%%%%%%%%%%%%%%%%%%%%%%%%%%%%%%%%%%%%%%%%%%%%%%%%%
%%%%%%%%%%%%%%%%%%%%%%%%%%%%%%%%%%%%%%%%%%%%%%%%%%%%%%%%%%



\clearpage
%%%%%%%%%%%%%%%%%%%%%%%%%%%%%%%%%%%%%%%%%%%%%%%%%%%%%%%%%%
%% section 17.9 %%%%%%%%%%%%%%%%%%%%%%%%%%%%%%%%%%%%%%%%%%
%%%%%%%%%%%%%%%%%%%%%%%%%%%%%%%%%%%%%%%%%%%%%%%%%%%%%%%%%%
\modHeadsection{\ttNum901050\,-\ttNum901074:タッチセンサーブローブ}
\ttNum901050\,-\ttNum901074については、主に工具{\ttfamily T50}\index{タッチセンサープローブ}タッチセンサープローブおよび\index{こうぐそくてい@工具測定}工具測定用\index{タッチセンサーブローブ(こうぐそくてい)@タッチセンサーブローブ(工具測定)}タッチセンサーブローブに関するものとする。\\

\begin{multicollongtblr}[white]{\ttNum901050\,-\ttNum901074:工具{\ttfamily T50}および工具測定用タッチセンサーブローブ}{cX[l]c}
変数 & 内容 & 初期値\\
\ttNum901050 & {\ttfamily T50}先端球 半径 & 5.0\\
\ttNum901051 & {\ttfamily T50}軸部分 半径 & 3.75\\
\SetRow{unusingVariables}
\ttNum901052 & (予備) &\\
\ttNum901053 & 信号遅れ補正(送り速さ50mm/min時) & 0.040\\
\ttNum901054 & {\ttfamily T50}中心$X$補正 & -0.016507\\
\ttNum901055 & {\ttfamily T50}中心$Y$補正 & -0.068371\\
\SetRow{unusingVariables}
\ttNum901056 & (予備) &\\
\ttNum901057 & 測定距離 & 5\\
\ttNum901058 & 工具測定用タッチセンサープローブ表面とプログラムの原点($Z$0)との距離 & 785.529\\
\SetRow{unusingVariables}
\ttNum901059 & (予備) &\\
\ttNum901060 & 工具破損検出の許容差 & 1.0\\
\SetRow{unusingVariables}
\ttNum901061 & (予備) &\\
\ttNum901062 & 工具測定用タッチセンサープローブの位置$X$ & -30.374\\
\ttNum901063 & 工具測定用タッチセンサープローブの位置$Y$ & -913.761\\
\ttNum901064 & 工具測定用タッチセンサープローブの位置$Z$ & -820.000\\
\SetRow{unusingVariables}
\ttNum901065 & (予備) &\\
\SetRow{unusingVariables}
\ttNum901066 & (予備) &\\
\SetRow{unusingVariables}
\ttNum901067 & (予備) &\\
\SetRow{unusingVariables}
\ttNum901068 & (予備) &\\
\SetRow{unusingVariables}
\ttNum901069 & (予備) &\\
\SetRow{unusingVariables}
\ttNum901070 & (予備) &\\
\SetRow{unusingVariables}
\ttNum901071 & (予備) &\\
\SetRow{unusingVariables}
\ttNum901072 & (予備) &\\
\SetRow{unusingVariables}
\ttNum901073 & (予備) &\\
\SetRow{unusingVariables}
\ttNum901074 & (予備) &\\
\end{multicollongtblr}



\clearpage
%%%%%%%%%%%%%%%%%%%%%%%%%%%%%%%%%%%%%%%%%%%%%%%%%%%%%%%%%%
%% section 11.9 %%%%%%%%%%%%%%%%%%%%%%%%%%%%%%%%%%%%%%%%%%
%%%%%%%%%%%%%%%%%%%%%%%%%%%%%%%%%%%%%%%%%%%%%%%%%%%%%%%%%%
\modHeadsection{\ttNum901100\,-\ttNum901149:工具}
\ttNum901100\,-\ttNum901149については、\index{こうぐ@工具}工具に関するもの(工具長や工具径およびその摩耗量を除く)とする。
%%%%%%%%%%%%%%%%%%%%%%%%%%%%%%%%%%%%%%%%%%%%%%%%%%%%%%%%%%
%% hosoku %%%%%%%%%%%%%%%%%%%%%%%%%%%%%%%%%%%%%%%%%%%%%%%%
%%%%%%%%%%%%%%%%%%%%%%%%%%%%%%%%%%%%%%%%%%%%%%%%%%%%%%%%%%
\begin{hosoku}
工具長・工具径・工具の摩耗量といったオフセットの値は、\pageautoref{sec:IV.A.2}を参照
\end{hosoku}
%%%%%%%%%%%%%%%%%%%%%%%%%%%%%%%%%%%%%%%%%%%%%%%%%%%%%%%%%%
%%%%%%%%%%%%%%%%%%%%%%%%%%%%%%%%%%%%%%%%%%%%%%%%%%%%%%%%%%
%%%%%%%%%%%%%%%%%%%%%%%%%%%%%%%%%%%%%%%%%%%%%%%%%%%%%%%%%%

%%%%%%%%%%%%%%%%%%%%%%%%%%%%%%%%%%%%%%%%%%%%%%%%%%%%%%%%%%
%% subsection 17.9.1 %%%%%%%%%%%%%%%%%%%%%%%%%%%%%%%%%%%%%
%%%%%%%%%%%%%%%%%%%%%%%%%%%%%%%%%%%%%%%%%%%%%%%%%%%%%%%%%%
\subsection{\ttNum901100\,-\ttNum901124}

\begin{multicollongtblr}[white]{\ttNum901100\,-\ttNum901124:工具({\ttfamily T50}除く)}{cX[l]c}
変数 & 内容 & 初期値\\
\SetRow{unusingVariables}
\ttNum901100 & (予備) &\\
\ttNum901101 & 工具{\ttfamily T02}(フェイスミル)最大刃径(直径)DCX公称値$\phi'_\mathrm D$ & 113.5\\
\SetRow{unusingVariables}
\ttNum901102 & (端面加工工具{\ttfamily T02}-{\ttfamily T05}用 予備) &\\
\SetRow{unusingVariables}
\ttNum901103 & (端面加工工具{\ttfamily T02}-{\ttfamily T05}用 予備) &\\
\SetRow{unusingVariables}
\ttNum901104 & (端面加工工具{\ttfamily T02}-{\ttfamily T05}用 予備) &\\
\ttNum901105 & 工具{\ttfamily T06}(サイドカッター)厚さ$t$ & 7.0\\
\SetRow{unusingVariables}
\ttNum901106 & (溝加工工具{\ttfamily T06}, {\ttfamily T07}用 予備) &\\
\SetRow{unusingVariables}
\ttNum901107 & (溝加工工具{\ttfamily T06}, {\ttfamily T07}用 予備) &\\
\SetRow{unusingVariables}
\ttNum901108 & (溝加工工具{\ttfamily T06}, {\ttfamily T07}用 予備) &\\
\ttNum901109 & 工具{\ttfamily T08}(サイドカッター)厚さ$t$ & 5.0\\
\SetRow{unusingVariables}
\ttNum901110 & (溝加工工具{\ttfamily T08}, {\ttfamily T10}用 予備) &\\
\SetRow{unusingVariables}
\ttNum901111 & (溝加工工具{\ttfamily T08}, {\ttfamily T10}用 予備) &\\
\SetRow{unusingVariables}
\ttNum901112 & (溝加工工具{\ttfamily T08}, {\ttfamily T10}用 予備) &\\
\ttNum901113 & 工具{\ttfamily T11}(テーパエンドミル)参照直径用 工具長補正値 & 2.0\\
\SetRow{unusingVariables}
\ttNum901114 & (工具{\ttfamily T11}用 予備) &\\
\ttNum901115 & 工具{\ttfamily T12}(テーパエンドミル)参照直径用 工具長補正値 & 2.0\\
\SetRow{unusingVariables}
\ttNum901116 & (工具{\ttfamily T12}用 予備) &\\
\ttNum901117 & 工具{\ttfamily T13}(テーパエンドミル)参照直径用 工具長補正値 & 2.0\\
\SetRow{unusingVariables}
\ttNum901118 & (面取加工工具{\ttfamily T13}-{\ttfamily T15}用 予備) &\\
\SetRow{unusingVariables}
\ttNum901119 & (面取加工工具{\ttfamily T13}-{\ttfamily T15}用 予備) &\\
\SetRow{unusingVariables}
\ttNum901120 & (面取加工工具{\ttfamily T13}-{\ttfamily T15}用 予備) &\\
\SetRow{unusingVariables}
\ttNum901121 & (外削加工工具用 予備) &\\
\SetRow{unusingVariables}
\ttNum901122 & (外削加工工具用 予備) &\\
\SetRow{unusingVariables}
\ttNum901123 & (外削加工工具用 予備) &\\
\SetRow{unusingVariables}
\ttNum901124 & (外削加工工具用 予備) &\\
\end{multicollongtblr}

\clearpage
%%%%%%%%%%%%%%%%%%%%%%%%%%%%%%%%%%%%%%%%%%%%%%%%%%%%%%%%%%
%% subsection 17.9.2 %%%%%%%%%%%%%%%%%%%%%%%%%%%%%%%%%%%%%
%%%%%%%%%%%%%%%%%%%%%%%%%%%%%%%%%%%%%%%%%%%%%%%%%%%%%%%%%%
\subsection{\ttNum901125\,-\ttNum901149}

\begin{multicollongtblr}[white]{\ttNum901124\,-\ttNum901149:工具({\ttfamily T50}除く)続き}{cX[l]c}
変数 & 内容 & 初期値\\
\SetRow{unusingVariables}
\ttNum901125 & (外削加工工具用 予備) &\\
\SetRow{unusingVariables}
\ttNum901126 & (外削加工工具用 予備) &\\
\ttNum901127 & 工具{\ttfamily T31}(Tスロットカッター)厚さ & 8.0\\
\ttNum901128 & 工具{\ttfamily T31}(Tスロットカッター)シャンク直径(公称値) & 25.0\\
\ttNum901129 & 工具{\ttfamily T32}(Tスロットカッター)厚さ & 8.0\\
\ttNum901130 & 工具{\ttfamily T32}(Tスロットカッター)シャンク直径(公称値) & 25.0\\
\SetRow{unusingVariables}
\ttNum901132 & (\dimple 加工工具用 予備) &\\
\SetRow{unusingVariables}
\ttNum901133 & (\dimple 加工工具用 予備) &\\
\SetRow{unusingVariables}
\ttNum901134 & (\dimple 加工工具用 予備) &\\
\SetRow{unusingVariables}
\ttNum901135 & (\dimple 加工工具用 予備) &\\
\SetRow{unusingVariables}
\ttNum901136 & (\dimple 加工工具用 予備) &\\
\SetRow{unusingVariables}
\ttNum901137 & (\dimple 加工工具用 予備) &\\
\SetRow{unusingVariables}
\ttNum901138 & (予備) &\\
\SetRow{unusingVariables}
\ttNum901139 & (予備) &\\
\SetRow{unusingVariables}
\ttNum901140 & (予備) &\\
\SetRow{unusingVariables}
\ttNum901141 & (予備) &\\
\SetRow{unusingVariables}
\ttNum901142 & (予備) &\\
\SetRow{unusingVariables}
\ttNum901143 & (予備) &\\
\SetRow{unusingVariables}
\ttNum901144 & (予備) &\\
\SetRow{unusingVariables}
\ttNum901145 & (予備) &\\
\SetRow{unusingVariables}
\ttNum901146 & (予備) &\\
\SetRow{unusingVariables}
\ttNum901147 & (予備) &\\
\SetRow{unusingVariables}
\ttNum901148 & (予備) &\\
\SetRow{unusingVariables}
\ttNum901149 & (予備) &\\
\end{multicollongtblr}



\clearpage
%%%%%%%%%%%%%%%%%%%%%%%%%%%%%%%%%%%%%%%%%%%%%%%%%%%%%%%%%%
%% section 18.10 %%%%%%%%%%%%%%%%%%%%%%%%%%%%%%%%%%%%%%%%%%
%%%%%%%%%%%%%%%%%%%%%%%%%%%%%%%%%%%%%%%%%%%%%%%%%%%%%%%%%%
\modHeadsection{未使用(使用可)のコモン変数}
この章の各々の表に載せていない\index{コモンへんすう(みしよう)@コモン変数(未使用)}コモン変数は(\dateUnusedVariables 時点において)未使用であり、新たに用いても問題ない(他のプログラムと競合しない)。
これらのコモン変数を、以下にまとめておく。
\begin{enumerate}
\item[-] \ttNum575-\ttNum599
\item[-] \ttNum750-\ttNum999
\item[-] \ttNum900050-\ttNum900100
\item[-] \ttNum900501-\ttNum900999
\item[-] \ttNum901025-\ttNum901049
\item[-] \ttNum901075-\ttNum901999
\item[-] \ttNum901150-\ttNum907399
\end{enumerate}


%\clearpage
\vfill
%%%%%%%%%%%%%%%%%%%%%%%%%%%%%%%%%%%%%%%%%%%%%%%%%%%%%%%%%%
%%%%%%%%%%%%%%%%%%%%%%%%%%%%%%%%%%%%%%%%%%%%%%%%%%%%%%%%%%
%%%%%%%%%%%%%%%%%%%%%%%%%%%%%%%%%%%%%%%%%%%%%%%%%%%%%%%%%%
\begin{tcolorbox}[title={2023/07/28時点の\MMname 実測値}, fonttitle=\gtfamily\bfseries]
\begin{align*}
  \text{Bot ($B=0$)}
  \left\{
  \begin{array}{rl}
    X: & 97.790 \sim 99.930\\
    Y: & -823.850\\
    Z: & -634.620
  \end{array}
  \right.\quad
  \text{Top ($B=180.$)}
  \left\{
  \begin{array}{rl}
    X: & -97.980 \sim -99.570\\
    Y: & -823.780\\
    Z: & -634.720
  \end{array}
  \right.
\end{align*}\\
・$X$については、ジグの当たる点の凸部と端部($Z$方向は目分量)\\
・$Y$については、モールドの底が当たる面\\
・$Z$については、$X0$ $Y-850.$における、ジグとの接点\\
※これらの値に、\index{タッチセンサープローブせんたんきゅう@タッチセンサープローブ先端球}タッチセンサー先端球の半径を加減する必要がある
\end{tcolorbox}
%%%%%%%%%%%%%%%%%%%%%%%%%%%%%%%%%%%%%%%%%%%%%%%%%%%%%%%%%%
%%%%%%%%%%%%%%%%%%%%%%%%%%%%%%%%%%%%%%%%%%%%%%%%%%%%%%%%%%
%%%%%%%%%%%%%%%%%%%%%%%%%%%%%%%%%%%%%%%%%%%%%%%%%%%%%%%%%%
