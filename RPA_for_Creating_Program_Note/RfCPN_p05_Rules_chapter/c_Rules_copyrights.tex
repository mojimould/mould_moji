%!TEX root = ../RPA_for_Creating_Program_Note.tex


\modHeadchapter{著作物およびその公表}
作成された\index{ちょさくぶつ@著作物}著作物をオンライン上に公表することで、その開発や\index{ほしゅ@保守}保守等における生産性が大きく向上することができる。
一方で、サーバ停止等によるリスクや公表による\index{じょうほうろうえい@情報漏洩}情報漏洩等の\index{セキュリティリスク}セキュリティリスクも考えられる。

こうしたことを踏まえ、ここでは作成されたソフトウェア関連の\index{ちょさくぶつ@著作物}著作物の取り扱いについて述べる。



%%%%%%%%%%%%%%%%%%%%%%%%%%%%%%%%%%%%%%%%%%%%%%%%%%%%%%%%%%
%% section 10.1 %%%%%%%%%%%%%%%%%%%%%%%%%%%%%%%%%%%%%%%%%%
%%%%%%%%%%%%%%%%%%%%%%%%%%%%%%%%%%%%%%%%%%%%%%%%%%%%%%%%%%
\modHeadsection{著作物および著作者}


%%%%%%%%%%%%%%%%%%%%%%%%%%%%%%%%%%%%%%%%%%%%%%%%%%%%%%%%%%
%% subsection 03.1.1 %%%%%%%%%%%%%%%%%%%%%%%%%%%%%%%%%%%%%
%%%%%%%%%%%%%%%%%%%%%%%%%%%%%%%%%%%%%%%%%%%%%%%%%%%%%%%%%%
\subsection{著作物}
著作物の定義は、\href{https://elaws.e-gov.go.jp/document?lawid=345AC0000000048#Mp-At_2}{著作権法第2条}第1項(著作物)\cite{online:eGovCopyrightLaw}に則る。
すなわち、「思想または感情を創作的に表現したものであって、文芸・学術・美術または音楽の範囲に属するもの」を著作物とする。

%%%%%%%%%%%%%%%%%%%%%%%%%%%%%%%%%%%%%%%%%%%%%%%%%%%%%%%%%%
%% subsubsection 03.1.1.1 %%%%%%%%%%%%%%%%%%%%%%%%%%%%%%%%
%%%%%%%%%%%%%%%%%%%%%%%%%%%%%%%%%%%%%%%%%%%%%%%%%%%%%%%%%%
\subsubsection{プログラムの著作物}
\index{プログラム(ちょさくぶつ)@プログラム(著作物)}プログラム
%% footnote %%%%%%%%%%%%%%%%%%%%%
\footnote{\href{https://elaws.e-gov.go.jp/document?lawid=345AC0000000048\#Mp-At_10}{著作権法第10条}第1項第9号(\index{プログラムのちょさくぶつ@プログラムの著作物}プログラムの著作物)\cite{online:eGovCopyrightLaw}より、プログラムは著作物の一種として明示的に規程されている。}
%%%%%%%%%%%%%%%%%%%%%%%%%%%%%%%%%
の定義は、\href{https://elaws.e-gov.go.jp/document?lawid=345AC0000000048\#Mp-At_2}{著作権法第2条}第1項第10号の2(プログラム)\cite{online:eGovCopyrightLaw}に則り、「電子計算機を機能させて一の結果を得ることができるようにこれに対する指令を組み合わせたものとして表現したもの」とする。

%%%%%%%%%%%%%%%%%%%%%%%%%%%%%%%%%%%%%%%%%%%%%%%%%%%%%%%%%%
%% subsubsection 03.1.1.1 %%%%%%%%%%%%%%%%%%%%%%%%%%%%%%%%
%%%%%%%%%%%%%%%%%%%%%%%%%%%%%%%%%%%%%%%%%%%%%%%%%%%%%%%%%%
\subsubsection{データベースの著作物}
\index{データベース(ちょさくぶつ)@データベース(著作物)}データベースの定義は、\href{https://elaws.e-gov.go.jp/document?lawid=345AC0000000048\#Mp-At_2}{著作権法第2条}第1項第10号の3(データベース)\cite{online:eGovCopyrightLaw}およびに則り、「論文・数値・図形その他の情報の集合物であって、それらの情報を電子計算機を用いて検索することができるように体系的に構成したもの」とする。

なお、\index{データベースのちょさくぶつ@データベースの著作物}データベースの著作物については、\href{https://elaws.e-gov.go.jp/document?lawid=345AC0000000048\#Mp-At_12_2}{著作権法第12条の2}(データベースの著作物)\cite{online:eGovCopyrightLaw}に則る。

%%%%%%%%%%%%%%%%%%%%%%%%%%%%%%%%%%%%%%%%%%%%%%%%%%%%%%%%%%
%% subsection 20.1.2 %%%%%%%%%%%%%%%%%%%%%%%%%%%%%%%%%%%%%
%%%%%%%%%%%%%%%%%%%%%%%%%%%%%%%%%%%%%%%%%%%%%%%%%%%%%%%%%%
\subsection{著作者}
\index{ちょさくしゃ@著作者}著作者の定義は、\href{https://elaws.e-gov.go.jp/document?lawid=345AC0000000048#Mp-At_2}{著作権法第2条}第2項(著作者)\cite{online:eGovCopyrightLaw}に則る。
すなわち、「著作物を創作する者」を著作者とする。



\clearpage
%%%%%%%%%%%%%%%%%%%%%%%%%%%%%%%%%%%%%%%%%%%%%%%%%%%%%%%%%%
%% section 20.2 %%%%%%%%%%%%%%%%%%%%%%%%%%%%%%%%%%%%%%%%%%
%%%%%%%%%%%%%%%%%%%%%%%%%%%%%%%%%%%%%%%%%%%%%%%%%%%%%%%%%%
\modHeadsection{著作物の著作権および著作権者}


%%%%%%%%%%%%%%%%%%%%%%%%%%%%%%%%%%%%%%%%%%%%%%%%%%%%%%%%%%
%% subsection 20.2.1 %%%%%%%%%%%%%%%%%%%%%%%%%%%%%%%%%%%%%
%%%%%%%%%%%%%%%%%%%%%%%%%%%%%%%%%%%%%%%%%%%%%%%%%%%%%%%%%%
\subsection{著作人格権}
\index{ちょさくじんかくけん@著作人格権}著作人格権については、\index{ちょさくけんほう@著作権法}\href{https://elaws.e-gov.go.jp/document?lawid=345AC0000000048\#Mp-At_59}{著作権法第59条}(\index{ちょさくじんかくけんのいっしんせんぞくせい@著作人格権の一身専属性}著作者人格権の一身専属性)\cite{online:eGovCopyrightLaw}に則る。
すなわち、すべての\index{ちょさくぶつ@著作物}著作物の\index{ちょさくじんかくけん@著作人格権}著作人格権は、その\index{ちょさくしゃ@著作者}著作者に帰属する。
また、原則として\index{ちょさくじんかくけんのこうし@著作人格権の行使}著作人格権の行使の判断・決定は、著作者に委れられるものとする。


%%%%%%%%%%%%%%%%%%%%%%%%%%%%%%%%%%%%%%%%%%%%%%%%%%%%%%%%%%
%% subsection 20.2.2 %%%%%%%%%%%%%%%%%%%%%%%%%%%%%%%%%%%%%
%%%%%%%%%%%%%%%%%%%%%%%%%%%%%%%%%%%%%%%%%%%%%%%%%%%%%%%%%%
\subsection{著作財産権}
当社の従業員により作成された著作物の\index{ちょさくざいさんけん@著作財産権}著作財産権については、\index{ちょさくけんほう@著作権法}\href{https://elaws.e-gov.go.jp/document?lawid=345AC0000000048#Mp-At_15}{著作権法第15条}(職務上作成する著作物の著作者)\cite{online:eGovCopyrightLaw}に則る。
すなわち、著作物が当社の指示による職務上作成された著作物(\index{しょくむちょさくぶつ@職務著作物}職務著作物)に該当する場合、その\index{ちょさくざいさんけん@著作財産権}著作財産権は当社に帰属する。

著作物が職務著作物に該当しない場合、その著作財産権は著作者個人に帰属する。



%\clearpage
%%%%%%%%%%%%%%%%%%%%%%%%%%%%%%%%%%%%%%%%%%%%%%%%%%%%%%%%%%
%% section 20.2 %%%%%%%%%%%%%%%%%%%%%%%%%%%%%%%%%%%%%%%%%%
%%%%%%%%%%%%%%%%%%%%%%%%%%%%%%%%%%%%%%%%%%%%%%%%%%%%%%%%%%
\modHeadsection{同一性保持権}
\index{どういつせいほじけん@同一性保持権}同一性保持権については、\index{ちょさくけんほう@著作権法}\href{https://elaws.e-gov.go.jp/document?lawid=345AC0000000048\#Mp-At_20}{著作権法第20条}(同一性保持権)に則る。
すなわち、著作者はその著作物およびその題号の\index{どういつせい(ちょさくぶつ)@同一性(著作物)}同一性を保持する権利を有し、その意に反してこれらの変更・切除その他の改変を受けないものとする。


%\clearpage
%%%%%%%%%%%%%%%%%%%%%%%%%%%%%%%%%%%%%%%%%%%%%%%%%%%%%%%%%%
%% section 20.2 %%%%%%%%%%%%%%%%%%%%%%%%%%%%%%%%%%%%%%%%%%
%%%%%%%%%%%%%%%%%%%%%%%%%%%%%%%%%%%%%%%%%%%%%%%%%%%%%%%%%%
\modHeadsection{職務著作物}
\index{しょくむちょさくぶつ@職務著作物}職務著作物については、\index{ちょさくけんほう@著作権法}\href{https://elaws.e-gov.go.jp/document?lawid=345AC0000000048\#Mp-At_15}{著作権法第15条}(職務上作成する著作物の著作者)に則る。
すなわち、作成された\index{ちょさくぶつ@著作物}著作物が以下の4つの要件をすべて満たす場合に限り、その著作物は\index{しょくむちょさくぶつ@職務著作物}職務著作物に該当する。
\begin{enumerate}[label=\Roman*., ref=\Roman*]
\item \textbf{当社の発意に基づくこと}\\
当社がある目的を持って構想した著作物の具体的な作成を従業員に命じることを意味する
\item \textbf{当社の業務に従事する者が職務上作成するものであること}\\
その著作物が従業員の通常の業務範囲内で作成されたものであれば、それは職務著作物に該当する可能性がある
\item\label{item:copyrightrule3} \textbf{当社の名義の下に公表するものであること}\\
その著作物が作成した業務従事者の名前で公表されれば職務著作物とは認められない
\item \textbf{作成の時における契約、勤務規則その他に別段の定めがないこと}\\
その著作物についての契約や勤務規則等に、別段の(適切な)定めがある場合は、その定めに従う
\end{enumerate}
ただし\ref{item:copyrightrule3}について、\index{プログラムのちょさくぶつ@プログラムの著作物}プログラムの著作物に関してはこの要件は必ずしも必要とはしない。



\clearpage
%%%%%%%%%%%%%%%%%%%%%%%%%%%%%%%%%%%%%%%%%%%%%%%%%%%%%%%%%%
%% section 20.2 %%%%%%%%%%%%%%%%%%%%%%%%%%%%%%%%%%%%%%%%%%
%%%%%%%%%%%%%%%%%%%%%%%%%%%%%%%%%%%%%%%%%%%%%%%%%%%%%%%%%%
\modHeadsection{著作物の管理}
当社の\index{しょくむちょさくぶつ@職務著作物}職務著作物について、当社(電子機器内を含む)に保管しているものは、そのすべてが以下の状態にあらなければならない。
\begin{enumerate}[label=\roman*)]
\item \textbf{業務と著作物の関連性の識別}\\
どの\index{ちょさくぶつ@著作物}著作物がどの業務に使用されているかを即座に識別できる状態
\item \textbf{複数の業務との関連性の識別}\\
複数の業務に関わる場合は、該当する業務について即座に識別できる状態
\item \textbf{著作権者の把握}\\
著作物のすべての\index{ちょさくじんかくけんしゃ@著作人格権者}著作人格権者およびすべての\index{ちょさくざいさんけんしゃ@著作財産権者}著作財産権者を把握できている状態
\item \textbf{公表・提示の状況の管理}\\
公表・提示の状況が即座に識別できる状態
\item \textbf{公表・保存・破棄等の基準設定}\\
公表・保存・破棄等に対して一定の適切な基準を設けられている状態
\item \textbf{著作物のバックアップ}\\
業務上重要な著作物について、定期的な\index{バックアップ}バックアップおよび\index{データふくげん@データ復元}データ復元のプロセスが設けられている状態
\end{enumerate}
また、当社の職務著作物でない著作物について、当社に保管しているものは、それが当社の職務著作物ではないことが判別できる状態にあらねばならない。



\clearpage
%%%%%%%%%%%%%%%%%%%%%%%%%%%%%%%%%%%%%%%%%%%%%%%%%%%%%%%%%%
%% section 20.4 %%%%%%%%%%%%%%%%%%%%%%%%%%%%%%%%%%%%%%%%%%
%%%%%%%%%%%%%%%%%%%%%%%%%%%%%%%%%%%%%%%%%%%%%%%%%%%%%%%%%%
\modHeadsection{著作物の公表}
著作物の公表・提示の権利については、\href{https://elaws.e-gov.go.jp/document?lawid=345AC0000000048\#Mp-At_18}{著作権法第18条}(\index{こうひょうけん@公表権}公表権)\cite{online:eGovCopyrightLaw}に則る。
すなわち、\index{ちょさくしゃ@著作者}著作者は、その著作物でまだ公表されていないもの(その同意を得ないで公表された著作物を含む)を公衆に提供し、または提示する権利を有する(当該著作物を原著作物とする\index{にじてきちょさくぶつ@二次的著作物}二次的著作物についても同様)。

なお、\index{ちょさくぶつのこうひょう@著作物の公表}著作物の公表については、\href{https://elaws.e-gov.go.jp/document?lawid=345AC0000000048\#Mp-At_4}{著作権法第4条}(著作物の公表)\cite{online:eGovCopyrightLaw}に則る。


%%%%%%%%%%%%%%%%%%%%%%%%%%%%%%%%%%%%%%%%%%%%%%%%%%%%%%%%%%
%% subsection 20.4.1 %%%%%%%%%%%%%%%%%%%%%%%%%%%%%%%%%%%%%
%%%%%%%%%%%%%%%%%%%%%%%%%%%%%%%%%%%%%%%%%%%%%%%%%%%%%%%%%%
\subsection{公表する著作物}

%%%%%%%%%%%%%%%%%%%%%%%%%%%%%%%%%%%%%%%%%%%%%%%%%%%%%%%%%%
%% subsubsection 20.4.2.1 %%%%%%%%%%%%%%%%%%%%%%%%%%%%%%%%
%%%%%%%%%%%%%%%%%%%%%%%%%%%%%%%%%%%%%%%%%%%%%%%%%%%%%%%%%%
\subsubsection{公表の判断および著作権者の同意}
著作物の公表・提示については、その業務ごとに判断する。
その判断の根拠として、生産性の向上や\index{セキュリティリスク}セキュリティリスク等が熟慮されなければならない。

ただし公表は、\href{https://elaws.e-gov.go.jp/document?lawid=345AC0000000048\#Mp-At_2}{著作権法第18条}に基づき、その著作物におけるすべての\index{ちょさくじんかくけんしゃ@著作人格権者}著作人格権の保有者(\index{ちょさくしゃ@著作者}著作者)およびすべての\index{ちょさくざいさんけんしゃ@著作財産権者}著作財産権の保有者の、全員の同意の下で行われることを前提とする。

なお、次節(非公表にする著作物)に該当する著作物に関してはその限りではない。

%%%%%%%%%%%%%%%%%%%%%%%%%%%%%%%%%%%%%%%%%%%%%%%%%%%%%%%%%%
%% subsubsection 20.4.2.2 %%%%%%%%%%%%%%%%%%%%%%%%%%%%%%%%
%%%%%%%%%%%%%%%%%%%%%%%%%%%%%%%%%%%%%%%%%%%%%%%%%%%%%%%%%%
\subsubsection{個人の著作権者}%\label{subsec:individualholder}
\index{ちょさくじんかくけんしゃ@著作人格権者}著作人格権および\index{ちょさくざいさんけんしゃ@著作財産権者}著作財産権の保有者が同一人物であり、かつ1人の個人のみである場合は、その個人は公表・提示する権利を有する。

%%%%%%%%%%%%%%%%%%%%%%%%%%%%%%%%%%%%%%%%%%%%%%%%%%%%%%%%%%
%% subsubsection 20.4.2.3 %%%%%%%%%%%%%%%%%%%%%%%%%%%%%%%%
%%%%%%%%%%%%%%%%%%%%%%%%%%%%%%%%%%%%%%%%%%%%%%%%%%%%%%%%%%
\subsubsection{データ保護とプライバシー}
公表の際は、\index{こじんじょうほうほごほう@個人情報保護法}\href{https://elaws.e-gov.go.jp/document?lawid=415AC0000000057}{個人情報の保護に関する法律}(個人情報保護法)\cite{online:eGovPersonalInfoProtectionLaw}に基づいて、\index{データほご@データ保護}データ保護・\index{プライバシーほご@プライバシー保護}プライバシー保護に十分に配慮しなければならない。


%%%%%%%%%%%%%%%%%%%%%%%%%%%%%%%%%%%%%%%%%%%%%%%%%%%%%%%%%%
%% subsection 20.4.2 %%%%%%%%%%%%%%%%%%%%%%%%%%%%%%%%%%%%%
%%%%%%%%%%%%%%%%%%%%%%%%%%%%%%%%%%%%%%%%%%%%%%%%%%%%%%%%%%
\subsection{非公表にする著作物}

%%%%%%%%%%%%%%%%%%%%%%%%%%%%%%%%%%%%%%%%%%%%%%%%%%%%%%%%%%
%% subsubsection 20.4.2.1 %%%%%%%%%%%%%%%%%%%%%%%%%%%%%%%%
%%%%%%%%%%%%%%%%%%%%%%%%%%%%%%%%%%%%%%%%%%%%%%%%%%%%%%%%%%
\subsubsection{機密情報の保護および公表の範囲}%\label{subsec:openwork}
\index{きみつじこう@機密事項}機密事項を含む著作物については、原則として公表しない。
あるいは、公表する場合でも(個々のものすべてではなく)代表的・典型的なものに留めるものとする。

%%%%%%%%%%%%%%%%%%%%%%%%%%%%%%%%%%%%%%%%%%%%%%%%%%%%%%%%%%
%% subsubsection 20.4.2.2 %%%%%%%%%%%%%%%%%%%%%%%%%%%%%%%%
%%%%%%%%%%%%%%%%%%%%%%%%%%%%%%%%%%%%%%%%%%%%%%%%%%%%%%%%%%
\subsubsection{外部作成の著作物および著作権者の同意}%\label{subsec:tulescopyrightsSubcontractor}
著作権者(特に著作財産権者)が当社の従業員または当社自体ではない著作物については、原則として公表しない。
公表する場合は、すべての著作人格権者およびすべての著作財産権者の同意の下に行われるものとする。

