%!TEX root = ../RPA_for_Creating_Program_Note.tex

ここでは\index{G-code}G-codeを記述する際や、\index{ずめん@図面}図面・3Dモデルの描画をする際に必要となる、\index{すんぽう@寸法}寸法や\index{こうさ@公差}公差等の取り扱いについて触れる。

ただし、特記事項等がある場合は、それを優先するものとする。
以下では主にそうした特別な記述のない、いわゆる一般的な場合について記載する。

なお、以降で述べる水平方向とは、端面のAC方向のことを指す。


%%%%%%%%%%%%%%%%%%%%%%%%%%%%%%%%%%%%%%%%%%%%%%%%%%%%%%%%%%
%% section 13.1 %%%%%%%%%%%%%%%%%%%%%%%%%%%%%%%%%%%%%%%%%%
%%%%%%%%%%%%%%%%%%%%%%%%%%%%%%%%%%%%%%%%%%%%%%%%%%%%%%%%%%
\modHeadsection{基本事項}

%%%%%%%%%%%%%%%%%%%%%%%%%%%%%%%%%%%%%%%%%%%%%%%%%%%%%%%%%%
%% subsection 13.1.1 %%%%%%%%%%%%%%%%%%%%%%%%%%%%%%%%%%%%%
%%%%%%%%%%%%%%%%%%%%%%%%%%%%%%%%%%%%%%%%%%%%%%%%%%%%%%%%%%
\subsection{寸法公差の取扱い}
全般的に、\index{すんぽうこうさ@寸法公差}寸法公差がある場合、\index{+こうさ@$+$公差}$+$公差と\index{-こうさ@$-$公差}$-$公差の中央(平均)を見るものとする。
ただし、\index{ないめんテーパひょう@内面テーパ表}内面テーパ表を見る際は、この限りではない。

たとえば、$100^{+0.5}_{\phantom -0}$であれば、100.25とみなす。

%%%%%%%%%%%%%%%%%%%%%%%%%%%%%%%%%%%%%%%%%%%%%%%%%%%%%%%%%%
%% subsection 13.1.2 %%%%%%%%%%%%%%%%%%%%%%%%%%%%%%%%%%%%%
%%%%%%%%%%%%%%%%%%%%%%%%%%%%%%%%%%%%%%%%%%%%%%%%%%%%%%%%%%
\subsection{寸法の優先度}
公差のある寸法と公差のない寸法(\index{かっこすんぽう@括弧寸法}括弧寸法含む)とが共存して記載されている場合、公差のある寸法を優先する。

たとえば、2つの線の寸法がそれぞれ$12^{+0.1}_{\phantom -0}$, $4.05$と記述されていて、かつその和に相当する部分の寸法が16と記述されている場合は、16.10とみなす。


%%%%%%%%%%%%%%%%%%%%%%%%%%%%%%%%%%%%%%%%%%%%%%%%%%%%%%%%%%
%% section 13.2 %%%%%%%%%%%%%%%%%%%%%%%%%%%%%%%%%%%%%%%%%%
%%%%%%%%%%%%%%%%%%%%%%%%%%%%%%%%%%%%%%%%%%%%%%%%%%%%%%%%%%
\modHeadsection{全長・振分長}

%%%%%%%%%%%%%%%%%%%%%%%%%%%%%%%%%%%%%%%%%%%%%%%%%%%%%%%%%%
%% subsection 13.2.1 %%%%%%%%%%%%%%%%%%%%%%%%%%%%%%%%%%%%%
%%%%%%%%%%%%%%%%%%%%%%%%%%%%%%%%%%%%%%%%%%%%%%%%%%%%%%%%%%
\subsection{全長と振分長の公差の関係}
\index{ふりわけちょう@振分長}振分長の公差については、\index{ぜんちょう@全長}全長の公差を\index{トップふりわけちょう@トップ振分長}トップ振分長と\index{ボトムふりわけちょう@ボトム振分長}ボトム振分長とで等分配する。

たとえば、全長が$1000^{\phantom +0}_{-1.0}$でトップ振分長が200であれば、全長の公差分$-0.5$を等分配し、それぞれ$-0.25$, $-0.25$とする。
つまり、トップ振分長は199.75, ボトム振分長は799.75とする
%% footnote %%%%%%%%%%%%%%%%%%%%%
\footnote{\index{ふりわけちゅうしん@振分中心}振分中心からのずれとして考えると、振分長に依らず等分配するのが自然、と捉えることができる。}。
%%%%%%%%%%%%%%%%%%%%%%%%%%%%%%%%%

%%%%%%%%%%%%%%%%%%%%%%%%%%%%%%%%%%%%%%%%%%%%%%%%%%%%%%%%%%
%% subsection 13.2.2 %%%%%%%%%%%%%%%%%%%%%%%%%%%%%%%%%%%%%
%%%%%%%%%%%%%%%%%%%%%%%%%%%%%%%%%%%%%%%%%%%%%%%%%%%%%%%%%%
\subsection{振分長が括弧寸法の場合}
片方の振分長が括弧寸法の場合は、全長の公差をそのまま括弧寸法に割り当てる。

たとえば、全長が$1000^{\phantom +0}_{-1.0}$でトップ振分長が200, ボトム振分長が(800)であれば、トップ振分長は200, ボトム振分長は799.5とする。



\clearpage
%%%%%%%%%%%%%%%%%%%%%%%%%%%%%%%%%%%%%%%%%%%%%%%%%%%%%%%%%%
%% section 13.3 %%%%%%%%%%%%%%%%%%%%%%%%%%%%%%%%%%%%%%%%%%
%%%%%%%%%%%%%%%%%%%%%%%%%%%%%%%%%%%%%%%%%%%%%%%%%%%%%%%%%%
\modHeadsection{外径}
\index{ちゅうしんわんきょく@中心湾曲}中心湾曲を$R_\mathrm c$, トップ振分長を$f_\mathrm T$, 外径を$W_x$とすると、トップ端面部の水平方向の長さ$W_\mathrm T$は以下で与えられる。(ボトム端面部も同様)
\begin{align*}
  W_\mathrm T
  = \sqrt{\left(R+\frac{W_x}2\right)^{\!2}-f_\mathrm T^2}
    -\sqrt{\left(R-\frac{W_x}2\right)^{\!2}-f_\mathrm T^2}\ .
\end{align*}
ただし、$(\nicefrac{f_\mathrm T}{R_\mathrm c})^2$が公差に対して十分小さい場合は、端面部の水平方向の長さ$W_\mathrm T$は\index{がいけい@外径}外径$W_x$とみなしてもよいものとする。


%%%%%%%%%%%%%%%%%%%%%%%%%%%%%%%%%%%%%%%%%%%%%%%%%%%%%%%%%%
%% section 04.4 %%%%%%%%%%%%%%%%%%%%%%%%%%%%%%%%%%%%%%%%%%
%%%%%%%%%%%%%%%%%%%%%%%%%%%%%%%%%%%%%%%%%%%%%%%%%%%%%%%%%%
\modHeadsection{内径}

%%%%%%%%%%%%%%%%%%%%%%%%%%%%%%%%%%%%%%%%%%%%%%%%%%%%%%%%%%
%% subsection 04.4.1 %%%%%%%%%%%%%%%%%%%%%%%%%%%%%%%%%%%%%
%%%%%%%%%%%%%%%%%%%%%%%%%%%%%%%%%%%%%%%%%%%%%%%%%%%%%%%%%%
\subsection{内面テーパ表の公差}
\index{ないめんテーパひょう@内面テーパ表}内面テーパ表を参照する際は、\index{ぜんちょう@全長}全長の\index{こうさ@公差}公差は考慮しないものとする。
また、トップ端からの距離のピッチも、同様に公差は考慮しないものとする。

たとえば、全長が$800^{+0.5}_{\phantom -0}$, トップ振分長が400, ピッチが25である場合を考える。
このとき、トップ端は振分中心から400の位置にあり、ピッチは25であるものとし、両端についてはそれを適宜延長して調整する。

%%%%%%%%%%%%%%%%%%%%%%%%%%%%%%%%%%%%%%%%%%%%%%%%%%%%%%%%%%
%% subsection 04.4.1 %%%%%%%%%%%%%%%%%%%%%%%%%%%%%%%%%%%%%
%%%%%%%%%%%%%%%%%%%%%%%%%%%%%%%%%%%%%%%%%%%%%%%%%%%%%%%%%%
\subsection{内面テーパ表にない内径}
内面テーパ表におけるトップ端からの距離$\lambda_i$ ($i = 0, 1, 2, ...$), それに対するAC側内径$w_{\mathrm Ai}$に対し、トップ端から$\lambda$の位置にある内径$w_{\mathrm A\lambda}$は、
\begin{align*}
  w_{\mathrm A\lambda}
  = \frac{(\lambda-\lambda_i)w_{\mathrm Ai+1}+(\lambda_{i+1}-\lambda)w_{\mathrm Ai}}{\lambda_{i+1}-\lambda_i}
  \qquad
  \Big(\lambda_i \leqq \lambda < \lambda_{i+1}\Big)
\end{align*}
とみなしてもよいものとする。
BD側内径$w_{\mathrm B\lambda}$についても同様である。

%%%%%%%%%%%%%%%%%%%%%%%%%%%%%%%%%%%%%%%%%%%%%%%%%%%%%%%%%%
%% subsection 04.4.2 %%%%%%%%%%%%%%%%%%%%%%%%%%%%%%%%%%%%%
%%%%%%%%%%%%%%%%%%%%%%%%%%%%%%%%%%%%%%%%%%%%%%%%%%%%%%%%%%
\subsection{水平方向の内径}
トップ端から$\lambda$の位置における内径を$w_\lambda$は、中心湾曲線上のトップ端から$\lambda$の位置における水平方向の内径とみなしてよいものとする。



\clearpage
%%%%%%%%%%%%%%%%%%%%%%%%%%%%%%%%%%%%%%%%%%%%%%%%%%%%%%%%%%
%% section 13.5 %%%%%%%%%%%%%%%%%%%%%%%%%%%%%%%%%%%%%%%%%%
%%%%%%%%%%%%%%%%%%%%%%%%%%%%%%%%%%%%%%%%%%%%%%%%%%%%%%%%%%
\modHeadsection{外削\TBW}
(to be written...)


%%%%%%%%%%%%%%%%%%%%%%%%%%%%%%%%%%%%%%%%%%%%%%%%%%%%%%%%%%
%% section 13.6 %%%%%%%%%%%%%%%%%%%%%%%%%%%%%%%%%%%%%%%%%%
%%%%%%%%%%%%%%%%%%%%%%%%%%%%%%%%%%%%%%%%%%%%%%%%%%%%%%%%%%
\modHeadsection{溝}
\index{Aがわみぞふかさ@A側溝深さ}A側溝深さが\index{こうさ@公差}公差のある寸法で、かつトップ側に外削のない場合、\pageautoref{subsec:keywayDepthDif}における$\kappa_d'$を図面上の値とする。
したがって\pageeqref{eq:keydepthDif1}より、溝幅中央における溝A側面とA側外面との距離$\kappa_d$を
\begin{align*}
  \kappa_d
  &= \frac{2\kappa_d'-\kappa_w\sin\zeta}{1+\cos^2\zeta}\cos\zeta
     +\sqrt{R_\mathrm o^2-\left(f_\mathrm T-\kappa_p-\frac{\kappa_w}2\right)^{\!2}}
     -\sqrt{R_\mathrm o^2-\left(f_\mathrm T-\kappa_p\right)^2}
\end{align*}
として扱う。
表記等の詳細については\pageautoref{subsec:keywayDepthDif}を参照。



%%%%%%%%%%%%%%%%%%%%%%%%%%%%%%%%%%%%%%%%%%%%%%%%%%%%%%%%%%
%% section 13.5 %%%%%%%%%%%%%%%%%%%%%%%%%%%%%%%%%%%%%%%%%%
%%%%%%%%%%%%%%%%%%%%%%%%%%%%%%%%%%%%%%%%%%%%%%%%%%%%%%%%%%
\modHeadsection{面取\TBW}
(to be written about $d_\mathrm e$ and $D_\mathrm r$)

%%%%%%%%%%%%%%%%%%%%%%%%%%%%%%%%%%%%%%%%%%%%%%%%%%%%%%%%%%
%% subsection 04.4.2 %%%%%%%%%%%%%%%%%%%%%%%%%%%%%%%%%%%%%
%%%%%%%%%%%%%%%%%%%%%%%%%%%%%%%%%%%%%%%%%%%%%%%%%%%%%%%%%%
\subsection{外側C面取}



