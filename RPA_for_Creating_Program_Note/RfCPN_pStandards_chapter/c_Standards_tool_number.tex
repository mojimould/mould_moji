%!TEX root = ../RPA_for_Creating_Program_Note.tex


ここでは\DMname で使用する\index{こうぐばんごう@工具番号}工具番号(\index{Tコード}Tコード値)について記述する。


%%%%%%%%%%%%%%%%%%%%%%%%%%%%%%%%%%%%%%%%%%%%%%%%%%%%%%%%%%
%% section 05.1 %%%%%%%%%%%%%%%%%%%%%%%%%%%%%%%%%%%%%%%%%%
%%%%%%%%%%%%%%%%%%%%%%%%%%%%%%%%%%%%%%%%%%%%%%%%%%%%%%%%%%
\modHeadsection{基本事項}
\begin{enumerate}[label=\Roman*), ref=\Roman*)]
\item 工具番号01は空とする
\item 工具番号02-05は\index{たんめんかこう@端面加工}端面加工用の工具とする
\item 工具番号06-09は\index{みぞかこう@溝加工}溝加工用の工具とする
\item 工具番号11-15は\index{めんとりかこう@面取加工}面取加工用の工具とする
\item 工具番号16-20は\index{がいさくかこう@外削加工}外削加工用の工具とする
\item 工具番号31-35は\expandafterindex{\dimplekana かこう@\dimple 加工}\dimple 加工用の工具とする
\item 工具番号36-40は\dimple 加工用の工具(\index{アングルヘッド}アングルヘッド)とする
\item 工具番号41-45は\index{にがしみぞかこう@逃し溝加工}逃し溝加工用の工具とする
\item 工具番号49, 50は測定用の工具とする
\end{enumerate}



\clearpage
%%%%%%%%%%%%%%%%%%%%%%%%%%%%%%%%%%%%%%%%%%%%%%%%%%%%%%%%%%
%% section 13.2 %%%%%%%%%%%%%%%%%%%%%%%%%%%%%%%%%%%%%%%%%%
%%%%%%%%%%%%%%%%%%%%%%%%%%%%%%%%%%%%%%%%%%%%%%%%%%%%%%%%%%
\modHeadsection{登録工具}
\addtocontents{lot}{\protect\addvspace{3pt}}{}{}
\addcontentsline{lot}{section}{\numberline{\thesection}\Sectionname}
\dateTourokuKougu 時点において\index{とうろくこうぐ@登録工具}登録されている工具は以下の通りである\\
\begin{2columnstable}{\DMname の登録工具}{Tコード}{使用工具}
\verb|T01| & 空\\\hline
\hline
\verb|T02| & $\phi100$端面加工用フェイスミル\\\hline
\hline
\verb|T06| & $\phi100\times t7$溝加工用サイドカッター\\\hline
\verb|T07| & $\phi100\times t5$溝加工用サイドカッター\\\hline
\hline
\verb|T11| & $\phi4.0\times 15^\circ$面取加工用テーパーエンドミル\\\hline
\verb|T12| & $\phi0.8\times 30^\circ$面取加工用テーパーエンドミル\\\hline
\verb|T13| & $\phi2.0\times 45^\circ$面取加工用テーパーエンドミル\\\hline
\hline
\verb|T16| & $\phi20$外削加工用スクエアエンドミル\\\hline
\hline
\verb|T31| & $\phi40\times R20$\dimple 加工用Tスロットカッター\\\hline
\verb|T32| & $\phi40\times R6.6$\dimple 加工用Tスロットカッター\\\hline
\verb|T33| & $\phi30\times R15$\dimple 加工用Tスロットカッター\\\hline
\hline
\verb|T36| & 10.0in\dimple 加工用アングルヘッド\\\hline
\hline
\verb|T41| & 15.5in逃し溝加工用アングルヘッド\\\hline
\hline
\verb|T50| & $\phi10\times200$(延長100)測定用タッチセンサープローブ
\end{2columnstable}
