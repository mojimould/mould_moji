%!TEX root = ../RPA_for_Creating_Program_Note.tex



ここでは\DMname におけるアラーム発生システム変数\hx\ttNum3000について記述する。


%%%%%%%%%%%%%%%%%%%%%%%%%%%%%%%%%%%%%%%%%%%%%%%%%%%%%%%%%%
%% section 17.1 %%%%%%%%%%%%%%%%%%%%%%%%%%%%%%%%%%%%%%%%%%
%%%%%%%%%%%%%%%%%%%%%%%%%%%%%%%%%%%%%%%%%%%%%%%%%%%%%%%%%%
\modHeadsection{アラームの分類:\DMname\TBW}
\DMname で使用される\index{アラームへんすう@アラーム変数}アラーム変数\hx\ttNum3000の値は、以下のように分類する。\\

%%%%%%%%%%%%%%%%%%%%%%%%%%%%%%%%%%%%%%%%%%%%%%%%%%%%%%%%%%
%% Alerm Classification %%%%%%%%%%%%%%%%%%%%%%%%%%%%%%%%%%
%%%%%%%%%%%%%%%%%%%%%%%%%%%%%%%%%%%%%%%%%%%%%%%%%%%%%%%%%%
\begin{3columnstable}{\DMname のアラーム番号の分類\TBW}{番号}{メッセージ}{異常の内容}{c}{Sl}{l}
001 & pallet alarm & パレット\\\hline
002 & touch sensor alarm & センサー(\verb|T50|)\\\hline
100 & work coordinate is not assigned & ワーク座標\\\hline
200 & argument is not assigned & 引数\\
\end{3columnstable}

%%%%%%%%%%%%%%%%%%%%%%%%%%%%%%%%%%%%%%%%%%%%%%%%%%%%%%%%%%
%% Alerm Classification %%%%%%%%%%%%%%%%%%%%%%%%%%%%%%%%%%
%%%%%%%%%%%%%%%%%%%%%%%%%%%%%%%%%%%%%%%%%%%%%%%%%%%%%%%%%%
\begin{2columnstable}{\DMname のアラーム番号の分類(バンドルのプログラム)}{番号}{メッセージ}{Sc}
001 & Pallet Alarm\\\hline
002 & Pallet Alarm\\\hline
003 & Pallet Alarm\\\hline
007 & Tool-Life-Check\\\hline
010 & Macro Pallet Check Alarm\\\hline
011 & No Program Selected\\\hline
012 & Wrong Pallet Alarm\\\hline
121 & Argument Is Not Assigned\\\hline
140 & A-Axis-Is-Not-Command\\\hline
141 & Tool-Measure-Alarm, Tool-Check, Measured-Value-Is-Over\\\hline
142 & Tool-Brake\\\hline
143 & Check-X,Y-Axis-Command-Value, Work-Coordinate-Setting-Error\\\hline
145 & MP10/MP12/MP60-Low-Battery\\\hline
146 & MP10/MP12/MP60-Alarm
\end{2columnstable}
