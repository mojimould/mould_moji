%!TEX root = ../RPA_for_Creating_Program_Note.tex



ここでは\DMname の加工システムで使用している\index{コモンへんすう@コモン変数}コモン変数について述べる。


%%%%%%%%%%%%%%%%%%%%%%%%%%%%%%%%%%%%%%%%%%%%%%%%%%%%%%%%%%
%% section 11.1 %%%%%%%%%%%%%%%%%%%%%%%%%%%%%%%%%%%%%%%%%%
%%%%%%%%%%%%%%%%%%%%%%%%%%%%%%%%%%%%%%%%%%%%%%%%%%%%%%%%%%
\modHeadsection{コモン変数の範囲\TBW}
\DMname で使用可能なコモン変数は以下のとおりである。
\begin{enumerate}
\item \ttNum100\,-\ttNum199
\item \ttNum400\,-\ttNum999
\item \ttNum900000\,-\ttNum907399
\end{enumerate}



%%%%%%%%%%%%%%%%%%%%%%%%%%%%%%%%%%%%%%%%%%%%%%%%%%%%%%%%%%
%% section 18.2 %%%%%%%%%%%%%%%%%%%%%%%%%%%%%%%%%%%%%%%%%%
%%%%%%%%%%%%%%%%%%%%%%%%%%%%%%%%%%%%%%%%%%%%%%%%%%%%%%%%%%
\modHeadsection{\ttNum100\,-\ttNum199}
\addtocontents{lot}{\protect\addvspace{3pt}}{}{}
\addcontentsline{lot}{section}{\numberline{\thesection}\Sectionname}


%%%%%%%%%%%%%%%%%%%%%%%%%%%%%%%%%%%%%%%%%%%%%%%%%%%%%%%%%%
%% subsection 18.2.1 %%%%%%%%%%%%%%%%%%%%%%%%%%%%%%%%%%%%%
%%%%%%%%%%%%%%%%%%%%%%%%%%%%%%%%%%%%%%%%%%%%%%%%%%%%%%%%%%
\subsection{\ttNum100\,-\ttNum174:一時保存値}
\ttNum100\,-\ttNum174については、(機械設置時の)\index{バンドルのプログラム}バンドルのプログラムで既に使用されているものが多いため、基本的には(\index{RHS}RHSとしては)使用しないものとし、一時的なもの(\index{LHS}LHS)として扱うものとする。
\newline


%\clearpage
\noindent\ttNum100\,-\ttNum110については、主に一時的な保存に用いるものとする。\\

\modcaptionof{table}{\ttNum100\,-\ttNum110:一時保存値}
\begin{twoCtable}{}
\ttNum100 & 各工程 切削回数用 一時保存値(仕上げ前 全削り代$X$ or $Z$)\\\hline
\ttNum101 & 各工程 切削回数用 一時保存値(仕上げ前 全削り代$Y$)\\\hline
\ttNum102 & 各工程 切削回数用 一時保存値 (max[\ttNum100, \ttNum101])\\\hline
\ttNum103 & 各工程 切削回数用 一時保存値(仕上げ前 切削回数)\\\hline
\ttNum104 & 各工程 切削回数用 一時保存値(加工時 径$X$)\\\hline
\ttNum105 & 各工程 切削回数用 一時保存値(加工時 径$Y$)\\\hline
\ttNum106 & 各工程 切削回数用 一時保存値(仕上げ 切削回数)\\\hline
\ttNum107 & (以下 予備)\\
\end{twoCtable}


\clearpage
%%%%%%%%%%%%%%%%%%%%%%%%%%%%%%%%%%%%%%%%%%%%%%%%%%%%%%%%%%
%% subsection 18.2.2 %%%%%%%%%%%%%%%%%%%%%%%%%%%%%%%%%%%%%
%%%%%%%%%%%%%%%%%%%%%%%%%%%%%%%%%%%%%%%%%%%%%%%%%%%%%%%%%%
\subsection{\ttNum175\,-\ttNum199:各工程後一時停止}
\noindent\ttNum175\,-\ttNum199については、主に各工程後の確認のためのものとする。\\

\modcaptionof{table}{\ttNum175\,-\ttNum199:各工程後一時停止}
\begin{twoCtable}{}
\ttNum175 & (以下 予備)\\\hline
$\vdots$ & \qquad$\vdots$\\\hline
\ttNum183 & (不使用)\\\hline
\ttNum184 & 芯出し測定後 一時停止 (0:non-stop, 1: \verb|M00|) & 0\\\hline
\ttNum185 & (不使用)\\\hline
\ttNum186 & \dimple 測定後 一時停止 (0:non-stop, 1: \verb|M00|) & 0\\\hline
\ttNum187 & \dimple 加工後 一時停止 (0:non-stop, 1: \verb|M00|, 2:\verb|O900003|) & 0\\\hline
\ttNum188 & (不使用)\\\hline
\ttNum189 & トップ端面加工後 一時停止 (0:non-stop, 1: \verb|M00|, 2:\verb|O900003|) & 0\\\hline
\ttNum190 & トップ外削加工後 一時停止 (0:non-stop, 1: \verb|M00|, 2:\verb|O900003|) & 0\\\hline
\ttNum191 & トップ溝加工後 一時停止 (0:non-stop, 1: \verb|M00|, 2:\verb|O900003|) & 0\\\hline
\ttNum192 & トップ外面取加工後 一時停止 (0:non-stop, 1: \verb|M00|, 2:\verb|O900003|) & 0\\\hline
\ttNum193 & トップ内面取加工後 一時停止 (0:non-stop, 1: \verb|M00|, 2:\verb|O900003|) & 0\\\hline
\ttNum194 & トップ座ぐり加工後 一時停止 (0:non-stop, 1: \verb|M00|, 2:\verb|O900003|) & 0\\\hline
\ttNum195 & (不使用)\\\hline
\ttNum196 & ボトム端面加工後 一時停止 (0:non-stop, 1: \verb|M00|, 2:\verb|O900003|) & 0\\\hline
\ttNum197 & ボトム外削加工後 一時停止 (0:non-stop, 1: \verb|M00|, 2:\verb|O900003|) & 0\\\hline
\ttNum198 & ボトム外面取加工後 一時停止 (0:non-stop, 1: \verb|M00|, 2:\verb|O900003|) & 0\\\hline
\ttNum199 & ボトム内面取加工後 一時停止 (0:non-stop, 1: \verb|M00|, 2:\verb|O900003|) & 0
\end{twoCtable}



\clearpage
%%%%%%%%%%%%%%%%%%%%%%%%%%%%%%%%%%%%%%%%%%%%%%%%%%%%%%%%%%
%% section 18.3 %%%%%%%%%%%%%%%%%%%%%%%%%%%%%%%%%%%%%%%%%%
%%%%%%%%%%%%%%%%%%%%%%%%%%%%%%%%%%%%%%%%%%%%%%%%%%%%%%%%%%
\modHeadsection{\ttNum400\,-\ttNum474:加工時の調整}
\addtocontents{lot}{\protect\addvspace{3pt}}{}{}
\addcontentsline{lot}{section}{\numberline{\thesection}\Sectionname}
\ttNum400\,-\ttNum474については、\index{さぎょうしゃ@作業者}作業者が入力・変更することが想定されるものとする。


%\clearpage
%%%%%%%%%%%%%%%%%%%%%%%%%%%%%%%%%%%%%%%%%%%%%%%%%%%%%%%%%%
%% subsection 18.3.1 %%%%%%%%%%%%%%%%%%%%%%%%%%%%%%%%%%%%%
%%%%%%%%%%%%%%%%%%%%%%%%%%%%%%%%%%%%%%%%%%%%%%%%%%%%%%%%%%
\subsection{\ttNum400\,-\ttNum424}
\modcaptionof{table}{\ttNum400\,-\ttNum404:初期設定}
\begin{twoCtable}{}
\ttNum400 & トップ端面 全削り代\\\hline
\ttNum401 & ボトム端面 全削り代\\\hline
\ttNum402 & 計測・加工 開始N番号 & 0\\\hline
\ttNum403 & 通り芯測定(0:off, 1:on) & 0\\\hline
\ttNum404 & (不使用)
\end{twoCtable}

\modcaptionof{table}{\ttNum405\,-\ttNum414:測定時の調整(\dimple 除く)}
\begin{twoCtable}{}
\ttNum405 & ボトム外$X$芯出し(両側・片側測定)測定位置$Z+$補正($X$自動補正) & 0\\\hline
\ttNum406 & ボトム外$Y$芯出し(両側測定)測定位置$Z+$補正 & 0\\\hline
\ttNum407 & ボトム内$X$芯出し(両側測定)測定位置$Z+$補正($X$自動補正) & 0\\\hline
\ttNum408 & ボトム内$Y$芯出し(両側測定)測定位置$Z+$補正 & 0\\\hline
\ttNum409 & (不使用)\\\hline
\ttNum410 & トップ外$X$芯出し(両側・片側測定)測定位置$Z+$補正($X$自動補正) & 0\\\hline
\ttNum411 & トップ外$Y$芯出し(両側測定)測定位置$Z+$補正 & 0\\\hline
\ttNum412 & トップ内$X$芯出し(両側測定)測定位置$Z+$補正($X$自動補正) & 0\\\hline
\ttNum413 & トップ内$Y$芯出し(両側測定)測定位置$Z+$補正 & 0\\\hline
\ttNum414 & (不使用)
\end{twoCtable}


%\clearpage
\modcaptionof{table}{\ttNum415\,-\ttNum424:加工時の調整(端面・\dimple 除く)}
\begin{twoCtable}{}
\ttNum415 & トップ外削 A面肉厚$+$補正(外削中心$X-$補正) & 0\\\hline
\ttNum416 & トップ外削 仕上げ前 一時停止 (0:non-stop, 1:\verb|M00|, 2:\verb|P90003|) & 0\\\hline
\ttNum417 & トップ外削 仕上げ加工 追加回数 (上限3) & 0\\\hline
\ttNum418 & (不使用)\\\hline
\ttNum419 & 溝位置$+$補正 & 0\\\hline
\ttNum420 & 溝幅$+$補正 & 0\\\hline
\ttNum421 & 溝A面深さ$+$補正(溝径中心$X-$補正) & 0\\\hline
\ttNum422 & 溝幅$Z$方向中央切削(3回加工)(0:off, 3:on) & 0\\\hline
\ttNum423 & 溝 仕上げ前 一時停止 (0:non-stop, 1:\verb|M00|, 2:\verb|P90003|) & 0\\\hline
\ttNum424 & 溝 仕上げ加工 追加回数 (上限3) & 0
\end{twoCtable}

\clearpage
%%%%%%%%%%%%%%%%%%%%%%%%%%%%%%%%%%%%%%%%%%%%%%%%%%%%%%%%%%
%% subsection 18.3.3 %%%%%%%%%%%%%%%%%%%%%%%%%%%%%%%%%%%%%
%%%%%%%%%%%%%%%%%%%%%%%%%%%%%%%%%%%%%%%%%%%%%%%%%%%%%%%%%%
\subsection{\ttNum425\,-\ttNum449\TBW}

\modcaptionof{table}{\ttNum425\,-\ttNum449:加工時の調整(続き)\TBW}
\begin{twoCtable}{}
\ttNum425 & トップ外面取$X+$補正 & 0\\\hline
\ttNum426 & トップ外面取 仕上げ前 一時停止 (0:non-stop, 1:\verb|M00|, 2:\verb|P90003|) & 0\\\hline
\ttNum427 & トップ外面取 仕上げ加工 追加回数 (上限3) & 0\\\hline
\ttNum428 & (不使用)\\\hline
\ttNum429 & トップ内面取$X+$補正 & 0\\\hline
\ttNum430 & トップ内面取 仕上げ前 一時停止 (0:non-stop, 1:\verb|M00|, 2:\verb|P90003|) & 0\\\hline
\ttNum431 & トップ内面取 仕上げ加工 追加回数 (上限3) & 0\\\hline
\ttNum432 & (不使用)\\\hline
\ttNum433\TBW & (座ぐり$X+$補正) &\\\hline
\ttNum434\TBW & (座ぐり 仕上げ前 一時停止 (0:non-stop, 1:\verb|M00|, 2:\verb|P90003|)) &\\\hline
\ttNum435\TBW & (座ぐり 仕上げ加工 追加回数 (上限3)) &\\\hline
\ttNum436 & (不使用)\\\hline
\ttNum437 & ボトム外削 A面肉厚$+$補正(外削中心$X+$補正) & 0\\\hline
\ttNum438 & ボトム外削 仕上げ前 一時停止 (0:non-stop, 1:\verb|M00|, 2:\verb|P90003|) & 0\\\hline
\ttNum439 & ボトム外削 仕上げ加工 追加回数 (上限3) & 0\\\hline
\ttNum440 & (不使用)\\\hline
\ttNum441 & ボトム外面取$X+$補正 & 0\\\hline
\ttNum442 & ボトム外面取 仕上げ前 一時停止 (0:non-stop, 1:\verb|M00|, 2:\verb|P90003|) & 0\\\hline
\ttNum443 & ボトム外面取 仕上げ加工 追加回数 (上限3) & 0\\\hline
\ttNum444 & (不使用)\\\hline
\ttNum445 & ボトム内面取$X+$補正 & 0\\\hline
\ttNum446 & ボトム内面取 仕上げ前 一時停止 (0:non-stop, 1:\verb|M00|, 2:\verb|P90003|) & 0\\\hline
\ttNum447 & ボトム内面取 仕上げ加工 追加回数 (上限3) & 0\\\hline
\ttNum448 & (不使用)\\\hline
\ttNum449 & (予備)
\end{twoCtable}


\clearpage
%%%%%%%%%%%%%%%%%%%%%%%%%%%%%%%%%%%%%%%%%%%%%%%%%%%%%%%%%%
%% subsection 18.3.2 %%%%%%%%%%%%%%%%%%%%%%%%%%%%%%%%%%%%%
%%%%%%%%%%%%%%%%%%%%%%%%%%%%%%%%%%%%%%%%%%%%%%%%%%%%%%%%%%
\subsection{\ttNum450\,-\ttNum474}
\modcaptionof{table}{\ttNum450\,-\ttNum474:\dimple 深さの調整}
\begin{twoCtable}{}
\ttNum450 & 工具\verb|T31|(Tスロット)A側\dimple~深さ補正値(深さに$+$補正) & 0.06\\\hline
\ttNum451 & 工具\verb|T31|(Tスロット)C側\dimple~深さ補正値(深さに$+$補正) & 0.03\\\hline
\ttNum452 & 工具\verb|T31|(Tスロット)B側\dimple~深さ補正値(深さに$+$補正) & 0.06\\\hline
\ttNum453 & 工具\verb|T31|(Tスロット)D側\dimple~深さ補正値(深さに$+$補正) & 0.03\\\hline
\ttNum454 & (不使用)\\\hline
\ttNum455 & 工具\verb|T32|(Tスロット)A側\dimple~深さ補正値(深さに$+$補正)\\\hline
\ttNum456 & 工具\verb|T32|(Tスロット)C側\dimple~深さ補正値(深さに$+$補正)\\\hline
\ttNum457 & 工具\verb|T32|(Tスロット)B側\dimple~深さ補正値(深さに$+$補正)\\\hline
\ttNum458 & 工具\verb|T32|(Tスロット)D側\dimple~深さ補正値(深さに$+$補正)\\\hline
\ttNum459 & (不使用)\\\hline
\ttNum460 & 工具\verb|T33|(Tスロット)A側\dimple~深さ補正値(深さに$+$補正)\\\hline
\ttNum461 & 工具\verb|T33|(Tスロット)C側\dimple~深さ補正値(深さに$+$補正)\\\hline
\ttNum462 & 工具\verb|T33|(Tスロット)B側\dimple~深さ補正値(深さに$+$補正)\\\hline
\ttNum463 & 工具\verb|T33|(Tスロット)D側\dimple~深さ補正値(深さに$+$補正)\\\hline
\ttNum464 & (不使用)\\\hline
& (以下予備)
\end{twoCtable}
%%%%%%%%%%%%%%%%%%%%%%%%%%%%%%%%%%%%%%%%%%%%%%%%%%%%%%%%%%
%% hosoku %%%%%%%%%%%%%%%%%%%%%%%%%%%%%%%%%%%%%%%%%%%%%%%%
%%%%%%%%%%%%%%%%%%%%%%%%%%%%%%%%%%%%%%%%%%%%%%%%%%%%%%%%%%
\begin{hosoku}
\ttNum450-\ttNum453の値は2024/01/16時点のもの。
\end{hosoku}
%%%%%%%%%%%%%%%%%%%%%%%%%%%%%%%%%%%%%%%%%%%%%%%%%%%%%%%%%%
%%%%%%%%%%%%%%%%%%%%%%%%%%%%%%%%%%%%%%%%%%%%%%%%%%%%%%%%%%
%%%%%%%%%%%%%%%%%%%%%%%%%%%%%%%%%%%%%%%%%%%%%%%%%%%%%%%%%%
%%%%%%%%%%%%%%%%%%%%%%%%%%%%%%%%%%%%%%%%%%%%%%%%%%%%%%%%%%
%% hosoku %%%%%%%%%%%%%%%%%%%%%%%%%%%%%%%%%%%%%%%%%%%%%%%%
%%%%%%%%%%%%%%%%%%%%%%%%%%%%%%%%%%%%%%%%%%%%%%%%%%%%%%%%%%
\begin{hosoku}
\ttNum475-\ttNum499については(\customtoday 時点において)未使用。
\end{hosoku}
%%%%%%%%%%%%%%%%%%%%%%%%%%%%%%%%%%%%%%%%%%%%%%%%%%%%%%%%%%
%%%%%%%%%%%%%%%%%%%%%%%%%%%%%%%%%%%%%%%%%%%%%%%%%%%%%%%%%%
%%%%%%%%%%%%%%%%%%%%%%%%%%%%%%%%%%%%%%%%%%%%%%%%%%%%%%%%%%



\clearpage
%%%%%%%%%%%%%%%%%%%%%%%%%%%%%%%%%%%%%%%%%%%%%%%%%%%%%%%%%%
%% section 18.4 %%%%%%%%%%%%%%%%%%%%%%%%%%%%%%%%%%%%%%%%%%
%%%%%%%%%%%%%%%%%%%%%%%%%%%%%%%%%%%%%%%%%%%%%%%%%%%%%%%%%%
\modHeadsection{\ttNum500\,-\ttNum599:バンドルプログラムの使用コモン変数}
\addtocontents{lot}{\protect\addvspace{3pt}}{}{}
\addcontentsline{lot}{section}{\numberline{\thesection}\Sectionname}
\ttNum500\,-\ttNum599については、主に\index{バンドルのプログラム}バンドルのプログラム(\index{O910x}O910xおよび\index{O93xx}O93xx)で使用されるものとする。\\

\modcaptionof{table}{\ttNum500\,-\ttNum599:O910x, O93xx用}
\begin{twoCtable}{}
\ttNum500 & 芯ずれ許容差 (O93xx) & 5.000\\\hline
\ttNum501 & タッチセンサー信号遅れ補正 (O93xx) & 0.040\\\hline
\ttNum502 & タッチセンサープローブ中心$X$補正 (O93xx) & -0.016507\\\hline
\ttNum503 & タッチセンサープローブ中心$Y$補正 (O93xx) & -0.068371\\\hline
\ttNum504 & 測定距離 (O910x) & 5.000\\\hline
\ttNum505 & プローブ表面からプログラムの加工原点($Z$0)までの距離 (O910x) & 785.529\\\hline
\ttNum506 & 工具長の変化の許容差 (O910x) & 1.000\\\hline
\ttNum507 & 工具破損検出の許容差 (O910x) & 1.000\\\hline
\ttNum508 & (不明) & \ttNum0\\\hline
\ttNum509 & Z座標系設定 (O93xx)& 441.432\\\hline
\ttNum510 & (不明) & 270.000\\\hline
\ttNum511 & インチ/ミリ切替 (O910x) & \ttNum0\\\hline
\ttNum512 & タッチセンサープローブ半径$\mathrm{mm}$値 (O93xx) & 5.000\\\hline
\ttNum513 & 移動時用の送り速さ値 (O910x) & 1000.000\\\hline
\ttNum514 & スキップ(\verb|G31|)測定時用 送り速さ値 (O910x, O93xx) & 50.000\\\hline
\ttNum515 & (不明) & \ttNum0\\\hline
\ttNum516 & センサーの位置$X$座標 (O910x) & -30.374\\\hline
\ttNum517 & センサーの位置$Y$座標 (O910x) & -913.761\\\hline
\ttNum518 & センサーの位置$Z$座標 (O910x) & -785.529\\\hline
\ttNum519 & (不明) & 6.000\\\hline
\ttNum520 & 拡張ワーク座標系 (O910x) & 1861.000\\\hline
\ttNum521 & (不明) & 0.000\\\hline
\ttNum522 & (不明) & 0.000\\\hline
\ttNum523 & アプローチ時用の送り速さ値 (O910x) & 30.000\\\hline
\ttNum524 & 測定時用の送り速さ値 (O910x) & 3.000\\\hline
$\vdots$ & (以下不明) & \ttNum0\\\hline
\ttNum533 & (不明) & 0.000\\\hline
\ttNum534 & (不明) & 0.000\\\hline
$\vdots$ & (以下不明) & \ttNum0
\end{twoCtable}
%%%%%%%%%%%%%%%%%%%%%%%%%%%%%%%%%%%%%%%%%%%%%%%%%%%%%%%%%%
%% hosoku %%%%%%%%%%%%%%%%%%%%%%%%%%%%%%%%%%%%%%%%%%%%%%%%
%%%%%%%%%%%%%%%%%%%%%%%%%%%%%%%%%%%%%%%%%%%%%%%%%%%%%%%%%%
\begin{hosoku}
これらの値は2023/09/26時点のもの。
\end{hosoku}
%%%%%%%%%%%%%%%%%%%%%%%%%%%%%%%%%%%%%%%%%%%%%%%%%%%%%%%%%%
%%%%%%%%%%%%%%%%%%%%%%%%%%%%%%%%%%%%%%%%%%%%%%%%%%%%%%%%%%
%%%%%%%%%%%%%%%%%%%%%%%%%%%%%%%%%%%%%%%%%%%%%%%%%%%%%%%%%%



\clearpage
%%%%%%%%%%%%%%%%%%%%%%%%%%%%%%%%%%%%%%%%%%%%%%%%%%%%%%%%%%
%% section 11.5 %%%%%%%%%%%%%%%%%%%%%%%%%%%%%%%%%%%%%%%%%%
%%%%%%%%%%%%%%%%%%%%%%%%%%%%%%%%%%%%%%%%%%%%%%%%%%%%%%%%%%
\modHeadsection{\ttNum600\,-\ttNum674}
\addtocontents{lot}{\protect\addvspace{3pt}}{}{}
\addcontentsline{lot}{section}{\numberline{\thesection}\Sectionname}

%%%%%%%%%%%%%%%%%%%%%%%%%%%%%%%%%%%%%%%%%%%%%%%%%%%%%%%%%%
%% subsection 11.5.1 %%%%%%%%%%%%%%%%%%%%%%%%%%%%%%%%%%%%%
%%%%%%%%%%%%%%%%%%%%%%%%%%%%%%%%%%%%%%%%%%%%%%%%%%%%%%%%%%
\subsection{\ttNum600\,-\ttNum649:ワークと工具間の距離の調整}
\ttNum600\,-\ttNum649については、主に製品と工具間の距離に関するものとする。\\

\modcaptionof{table}{\ttNum600\,-\ttNum649:ワークと工具間の距離の調整}
\begin{twoCtable}{}
\ttNum600 & (予備)\\\hline
\ttNum601 & 工具 - 端面間 $Z$方向クリアランス平面間距離 & 100.0\\\hline
\ttNum602 & (予備)\\\hline
\ttNum603 & タッチセンサー計測時の近付き量 & 10.0\\\hline
\ttNum604 & タッチセンサー計測時の行過ぎ量 & 5.0\\\hline
\ttNum605 & (予備)\\\hline
\ttNum606 & (予備)\\\hline
\ttNum607 & 外側加工面 法線方向クリアランス平面間距離 最小値 & 30.0\\\hline
\ttNum608 & 内側加工面 法線方向クリアランス平面間距離 最小値 & 15.0\\\hline
\ttNum609 & (予備)\\\hline
\ttNum610 & 端面加工用 内径輪郭径$-$補正量 & 5.0\\\hline
\ttNum611 & (予備)\\\hline
\ttNum612 & (外削加工用予備) \\\hline
\ttNum613 & (予備)\\\hline
\ttNum614 & 溝加工用 BD溝加工のみ時 AC方向クリアランス平面間距離 & \verb|GE| \ttNum607*1.5\\\hline
\ttNum615 & (予備)\\\hline
\ttNum616 & (外面取加工用予備) \\\hline
\ttNum617 & (予備)\\\hline
\ttNum618 & (内面取加工用予備) \\\hline
\ttNum617 & (予備)\\\hline
\ttNum618 & (座ぐり加工用予備) \\\hline
\ttNum619 & (予備)\\\hline
\ttNum620 & \dimple 加工用 加工近付き量 & 1 \\\hline
& (以下予備)
\end{twoCtable}
%%%%%%%%%%%%%%%%%%%%%%%%%%%%%%%%%%%%%%%%%%%%%%%%%%%%%%%%%%
%% hosoku %%%%%%%%%%%%%%%%%%%%%%%%%%%%%%%%%%%%%%%%%%%%%%%%
%%%%%%%%%%%%%%%%%%%%%%%%%%%%%%%%%%%%%%%%%%%%%%%%%%%%%%%%%%
\begin{hosoku}
\ttNum614について、正確には$\max[\ttNum607, \text{{\small(溝工具半径)}}]\times1.5$.
\end{hosoku}
%%%%%%%%%%%%%%%%%%%%%%%%%%%%%%%%%%%%%%%%%%%%%%%%%%%%%%%%%%
%%%%%%%%%%%%%%%%%%%%%%%%%%%%%%%%%%%%%%%%%%%%%%%%%%%%%%%%%%
%%%%%%%%%%%%%%%%%%%%%%%%%%%%%%%%%%%%%%%%%%%%%%%%%%%%%%%%%%



\clearpage
%%%%%%%%%%%%%%%%%%%%%%%%%%%%%%%%%%%%%%%%%%%%%%%%%%%%%%%%%%
%% subsection 11.5.2 %%%%%%%%%%%%%%%%%%%%%%%%%%%%%%%%%%%%%
%%%%%%%%%%%%%%%%%%%%%%%%%%%%%%%%%%%%%%%%%%%%%%%%%%%%%%%%%%
\subsection{\ttNum650\,-\ttNum674:残り代および1回あたりの削り代}
\ttNum650\,-\ttNum674については、各加工の残り代や1回あたりの削り代に関するものとする。\\

\modcaptionof{table}{\ttNum650\,-\ttNum674:残り代および1回あたりの削り代}
\begin{twoCtable}{}
\ttNum650 & 端面加工1回あたりの$Z$方向削り代 & 4.0\\\hline
\ttNum651 & (予備) & \\\hline
\ttNum652 & 外削加工1回あたりの削り代(直径) & 2.0\\\hline
\ttNum653 & 外削加工 仕上げ前 残り削り代(直径) & 1.0\\\hline
\ttNum654 & 溝加工1回あたりの削り代(溝深さ) & 5.0\\\hline
\ttNum655 & 溝加工 仕上げ前 残り削り代(直径) & 1.0\\\hline
\ttNum656 & 外面取加工1回あたりの削り代(直径) & 2.0\\\hline
\ttNum657 & 外面取加工 仕上げ前 残り削り代(直径) & 1.0\\\hline
\ttNum658 & 内面取加工1回あたりの削り代(直径) & 2.0\\\hline
\ttNum659 & 内面取加工 仕上げ前 残り削り代(直径) & 1.0\\\hline
\ttNum660 & (座ぐり加工用予備) & \\\hline
\ttNum661 & (座ぐり加工用予備) & \\\hline
& (以下予備)
\end{twoCtable}
%%%%%%%%%%%%%%%%%%%%%%%%%%%%%%%%%%%%%%%%%%%%%%%%%%%%%%%%%%
%% hosoku %%%%%%%%%%%%%%%%%%%%%%%%%%%%%%%%%%%%%%%%%%%%%%%%
%%%%%%%%%%%%%%%%%%%%%%%%%%%%%%%%%%%%%%%%%%%%%%%%%%%%%%%%%%
\begin{hosoku}
\ttNum675-\ttNum699については(\customtoday 時点において)未使用。
\end{hosoku}
%%%%%%%%%%%%%%%%%%%%%%%%%%%%%%%%%%%%%%%%%%%%%%%%%%%%%%%%%%
%%%%%%%%%%%%%%%%%%%%%%%%%%%%%%%%%%%%%%%%%%%%%%%%%%%%%%%%%%
%%%%%%%%%%%%%%%%%%%%%%%%%%%%%%%%%%%%%%%%%%%%%%%%%%%%%%%%%%



\clearpage
%%%%%%%%%%%%%%%%%%%%%%%%%%%%%%%%%%%%%%%%%%%%%%%%%%%%%%%%%%
%% section 11.6 %%%%%%%%%%%%%%%%%%%%%%%%%%%%%%%%%%%%%%%%%%
%%%%%%%%%%%%%%%%%%%%%%%%%%%%%%%%%%%%%%%%%%%%%%%%%%%%%%%%%%
\modHeadsection{\ttNum700\,-\ttNum774:\dimple\TBW}
\addtocontents{lot}{\protect\addvspace{3pt}}{}{}
\addcontentsline{lot}{section}{\numberline{\thesection}\Sectionname}
\ttNum700\,-\ttNum774については、主に\expandafterindex{\dimplekana そくていようサブプログラム@\dimple 測定用サブプログラム}\dimple 用サブプログラム(O2x000x)で使用されるものとする。\\

\modcaptionof{table}{\ttNum700\,-\ttNum725:\dimple レベル1サブプログラム用 (\DLone)}
\begin{twoCtable}{}
\ttNum700 & (予備)\\\hline
\ttNum701 & プログラム読込み時の座標系(\ttNum4012)\\\hline
\ttNum702 & 工具別$Z$補正(\verb|T50|:\ttNum512, \verb|T3|x:0)\\\hline
\ttNum703 & 工具別$XY$補正(\verb|T50|:\ttNum512, \verb|T3|x:\ttNum[2400+\ttNum4111]+\ttNum[2600+\ttNum4111])\\\hline
\ttNum704 & 工具別移動\verb|G#| (\verb|T50|:31, \verb|T3|x:1)\\\hline
\ttNum705 & テーブル中心からワーク座標(\ttNum701)原点までの$X$距離\\\hline
\ttNum706 & 傾き後のトップ端面中心(機械座標)$X$ (\cf\pageeqref{eq:afterPhiTCenterFromO})\\\hline
\ttNum707 & テーブル中心から傾き後のトップ端面中心までの$Z$距離 (\cf\pageeqref{eq:afterPhiTCenterFromO})\\\hline
\ttNum708 & 傾き後トップ端中心(ブロックエンド)$X$座標(\ttNum5001)\\\hline
\ttNum709 & 傾き後トップ端中心(ブロックエンド)$Z$座標(\ttNum5003)\\\hline
\ttNum710 & テーブル中心から\dimple1列目までの$Z$距離$Z-q$\\\hline
\ttNum711 & トップ端中心から\dimple1列目中心までの$X$距離(\cf\pageeqref{eq:dimpleCenterDistance})\\\hline
\ttNum712 & 傾き後\dimple1列目中心$X$移動距離(\cf\pageeqref{eq:afterPhidimpleCenterDistance})\\\hline
\ttNum713 & 傾き後\dimple1列目中心$Z$移動距離(\cf\pageeqref{eq:afterPhidimpleCenterDistance})\\\hline
\ttNum714 & 傾き後\dimple1列目中心(ブロックエンド)$X$座標 (\ttNum5001)\\\hline
\ttNum715 & 傾き後\dimple1列目中心(ブロックエンド)$Y$座標 (\ttNum5002)\\\hline
\ttNum716 & 傾き後\dimple1列目中心(ブロックエンド)$Z$座標 (\ttNum5003)\\\hline
\ttNum717 & 各面用ループ番号(1:A, 2:C, 3:B, 4:D)\\\hline
\ttNum718 & BD内半径$-\text{\ttNum703}-10$\\\hline
\ttNum719 & (AC内半径$-\text{\ttNum703}-10)\cos\phi$\\\hline
& (以下予備)
\end{twoCtable}


%\clearpage
%%%%%%%%%%%%%%%%%%%%%%%%%%%%%%%%%%%%%%%%%%%%%%%%%%%%%%%%%%
%% common variables %%%%%%%%%%%%%%%%%%%%%%%%%%%%%%%%%%%%%%
%%%%%%%%%%%%%%%%%%%%%%%%%%%%%%%%%%%%%%%%%%%%%%%%%%%%%%%%%%
\modcaptionof{table}{\ttNum725\,-\ttNum749:\dimple~レベル2サブプログラム用 (\DLtwoAC, \DLtwoBD)}
\begin{twoCtable}{}
\ttNum725 & プログラム読込時ブロックエンド$Y$ or $X$ (\ttNum5002, \ttNum5001)\\\hline
\ttNum726 & プログラム読込時ブロックエンド$Z$ (\ttNum5003)\\\hline
\ttNum727 & \dimple~偶数列の列数\\\hline
\ttNum728 & \dimple~偶数列(一列)の\dimple~数\\\hline
\ttNum729 & \dimple~奇数列(一列)の\dimple~数\\\hline
\ttNum730 & \dimple~現在の列の\dimple~数\\\hline
& (以下予備)
\end{twoCtable}


\clearpage
%%%%%%%%%%%%%%%%%%%%%%%%%%%%%%%%%%%%%%%%%%%%%%%%%%%%%%%%%%
%% common variables %%%%%%%%%%%%%%%%%%%%%%%%%%%%%%%%%%%%%%
%%%%%%%%%%%%%%%%%%%%%%%%%%%%%%%%%%%%%%%%%%%%%%%%%%%%%%%%%%
\modcaptionof{table}{\ttNum750\,-\ttNum774:\dimple~測定用 (\DMLthreeAC, \DMLthreeBD)\TBW}
\begin{twoCtable}{}
\ttNum750 & プログラム読込時ブロックエンド$X$ or $Y$ (\ttNum5001, \ttNum5002)\\\hline
\ttNum751 & \dimple~表面位置$X$ or $Y$測定値\\\hline
& (以下予備)
\end{twoCtable}
%%%%%%%%%%%%%%%%%%%%%%%%%%%%%%%%%%%%%%%%%%%%%%%%%%%%%%%%%%
%% hosoku %%%%%%%%%%%%%%%%%%%%%%%%%%%%%%%%%%%%%%%%%%%%%%%%
%%%%%%%%%%%%%%%%%%%%%%%%%%%%%%%%%%%%%%%%%%%%%%%%%%%%%%%%%%
\begin{hosoku}
\ttNum775-\ttNum899については(\customtoday 時点において)未使用。
\end{hosoku}
%%%%%%%%%%%%%%%%%%%%%%%%%%%%%%%%%%%%%%%%%%%%%%%%%%%%%%%%%%
%%%%%%%%%%%%%%%%%%%%%%%%%%%%%%%%%%%%%%%%%%%%%%%%%%%%%%%%%%
%%%%%%%%%%%%%%%%%%%%%%%%%%%%%%%%%%%%%%%%%%%%%%%%%%%%%%%%%%



\clearpage
%%%%%%%%%%%%%%%%%%%%%%%%%%%%%%%%%%%%%%%%%%%%%%%%%%%%%%%%%%
%% section 11.7 %%%%%%%%%%%%%%%%%%%%%%%%%%%%%%%%%%%%%%%%%%
%%%%%%%%%%%%%%%%%%%%%%%%%%%%%%%%%%%%%%%%%%%%%%%%%%%%%%%%%%
\modHeadsection{\ttNum900001\,-\ttNum900031, \ttNum900101\,-\ttNum900500:実測値\TBW}
\addtocontents{lot}{\protect\addvspace{3pt}}{}{}
\addcontentsline{lot}{section}{\numberline{\thesection}\Sectionname}
\ttNum900001\,-\ttNum900031, \ttNum900101\,-\ttNum900500については、主に\index{じっそくち@実測値}実測値を格納する。\\

\modcaptionof{table}{\ttNum900001\,-\ttNum900005:外中心$X$測定用 (\MXOThickness)}
\begin{twoCtable}{}
\ttNum900001 & $X$外中心測定 $-X$側測定値\\\hline
\ttNum900002 & $X$外中心測定 $+X$側測定値\\\hline
\ttNum900003 & $X$外中心測定値\\\hline
\ttNum900004 & $X$外中心測定 厚さ測定値\\\hline
\ttNum900005 & (予備)\\
\end{twoCtable}


\modcaptionof{table}{\ttNum900006\,-\ttNum900010:外中心$Y$測定用 (\MYOThickness)}
\begin{twoCtable}{}
\ttNum900006 & $Y$外中心測定 $-Y$側測定値\\\hline
\ttNum900007 & $Y$外中心測定 $+Y$側測定値\\\hline
\ttNum900008 & $Y$外中心測定値\\\hline
\ttNum900009 & $Y$外中心測定 厚さ測定値\\\hline
\ttNum900010 & (予備)\\
\end{twoCtable}


%\clearpage
\modcaptionof{table}{\ttNum900011\,-\ttNum900013:溝中心$X$測定用 (\MXOface)}
\begin{twoCtable}{}
\ttNum900011 & $X$溝中心測定 A側外面測定値\\\hline
\ttNum900012 & (予備)\\\hline
\ttNum900013 & (予備)\\
\end{twoCtable}


%\clearpage
\modcaptionof{table}{\ttNum900014\,-\ttNum900018:内中心$X$測定用 (\MXIWidth)}
\begin{twoCtable}{}
\ttNum900014 & $X$内中心測定 $-X$側測定値\\\hline
\ttNum900015 & $X$内中心測定 $+X$側測定値\\\hline
\ttNum900016 & $X$内中心測定値\\\hline
\ttNum900017 & $X$内中心測定 厚さ測定値\\\hline
\ttNum900018 & (予備)\\
\end{twoCtable}


%\clearpage
\modcaptionof{table}{\ttNum900019\,-\ttNum900023:内中心$Y$測定用 (\MYIWidth)}
\begin{twoCtable}{}
\ttNum900019 & $Y$内中心測定 $-Y$側測定値\\\hline
\ttNum900020 & $Y$内中心測定 $+Y$側測定値\\\hline
\ttNum900021 & $Y$内中心測定値\\\hline
\ttNum900022 & $Y$内中心測定 厚さ測定値\\\hline
\ttNum900023 & (予備)\\
\end{twoCtable}


\clearpage
\modcaptionof{table}{\ttNum900024\,-\ttNum900026:外削中心$X$測定用 (\MXIface)}
\begin{twoCtable}{}
\ttNum900024 & $X$外削中心測定 内面測定値\\\hline
\ttNum900025 & (予備)\\\hline
\ttNum900026 & (予備)\\
\end{twoCtable}


%\clearpage
\modcaptionof{table}{\ttNum900027\,-\ttNum900030:通り芯$X$測定用 (\MYcenterline)}
\begin{twoCtable}{}
\ttNum900027 & $Y$通り芯 ボトム側測定値\\\hline
\ttNum900028 & $Y$通り芯 トップ側測定値\\\hline
\ttNum900029 & $Y$通り芯 測定値\\\hline
\ttNum900030 & (予備)\\
\end{twoCtable}


\modcaptionof{table}{\ttNum900031\,-\ttNum900034:通り芯$X$測定用 (\MXcenterline)}
\begin{twoCtable}{}
\ttNum900031 & $X$通り芯 トップ側測定値\\\hline
\ttNum900032 & $X$通り芯 ボトム側測定値\\\hline
\ttNum900033 & $X$通り芯 測定値\\\hline
\ttNum900034 & (予備)\\
\end{twoCtable}


%\clearpage
\modcaptionof{table}{\ttNum900101\,-\ttNum900500:\dimple 用 表面位置測定値 (\DMLthreeAC, \DMLthreeBD)}
\begin{twoCtable}{}
\ttNum900101\,-\ttNum900200 & A側\dimple~深さ測定値(Tスロット)\\\hline
\ttNum900201\,-\ttNum900300 & C側\dimple~深さ測定値(Tスロット)\\\hline
\ttNum900301\,-\ttNum900400 & B側\dimple~深さ測定値(Tスロット)\\\hline
\ttNum900401\,-\ttNum900500 & D側\dimple~深さ測定値(Tスロット)
\end{twoCtable}



\clearpage
%%%%%%%%%%%%%%%%%%%%%%%%%%%%%%%%%%%%%%%%%%%%%%%%%%%%%%%%%%
%% section 11.8 %%%%%%%%%%%%%%%%%%%%%%%%%%%%%%%%%%%%%%%%%%
%%%%%%%%%%%%%%%%%%%%%%%%%%%%%%%%%%%%%%%%%%%%%%%%%%%%%%%%%%
\modHeadsection{\ttNum901000\,-\ttNum901099:パレット・ジグ}
\addtocontents{lot}{\protect\addvspace{3pt}}{}{}
\addcontentsline{lot}{section}{\numberline{\thesection}\Sectionname}
\ttNum901000\,-\ttNum901099については、主に\index{パレット}パレットや\index{ジグ}ジグに関するものとする。\\

\modcaptionof{table}{\ttNum901000\,-\ttNum901099:主にパレット・ジグ}
\begin{twoCtable}{}
\ttNum901000 & (予備)\\\hline
\ttNum901001 & パレット\ttNum1 ジグ中心機械座標$X$ & -550.019\\\hline
\ttNum901002 & パレット\ttNum1 ジグ中心機械座標$Y$ & -740\\\hline
\ttNum901003 & パレット\ttNum1 ジグ中心機械座標$Z$ & -1149.974\\\hline
\ttNum901004 & パレット\ttNum1 ジグ中心機械座標$B$ & -0.073\\\hline
\ttNum901005 & パレット\ttNum2 ジグ中心機械座標$X$ & -550.019\\\hline
\ttNum901006 & パレット\ttNum2 ジグ中心機械座標$Y$ & -740\\\hline
\ttNum901007 & パレット\ttNum2 ジグ中心機械座標$Z$ & -1149.974\\\hline
\ttNum901008 & パレット\ttNum2 ジグ中心機械座標$B$ & -0.073\\\hline
\ttNum901009 & 工具中心機械座標$C$ & 0\\\hline
\ttNum901010 & (予備)\\\hline
\hline
\ttNum901011 & パレット\ttNum1 ジグ外側幅$2l$(機械座標系$B$0における$Z$方向) & 660\\\hline
\ttNum901012 & パレット\ttNum1 ジグ内側幅(機械座標系$B$0における$Z$方向) & 410\\\hline
\ttNum901013 & パレット\ttNum1 ジグ幅(機械座標系$B$0における$X$方向) & 455\\\hline
\ttNum901014 & (予備)\\\hline
\ttNum901015 & パレット\ttNum2 ジグ外側幅$2l$(機械座標系$B$0における$Z$方向) & 660\\\hline
\ttNum901016 & パレット\ttNum2 ジグ内側幅(機械座標系$B$0における$Z$方向) & 410\\\hline
\ttNum901017 & パレット\ttNum2 ジグ幅(機械座標系$B$0における$X$方向) & 455\\\hline
& (以下予備)
\end{twoCtable}
%%%%%%%%%%%%%%%%%%%%%%%%%%%%%%%%%%%%%%%%%%%%%%%%%%%%%%%%%%
%% hosoku %%%%%%%%%%%%%%%%%%%%%%%%%%%%%%%%%%%%%%%%%%%%%%%%
%%%%%%%%%%%%%%%%%%%%%%%%%%%%%%%%%%%%%%%%%%%%%%%%%%%%%%%%%%
\begin{hosoku}
\index{ジグのちゅうしん@ジグの中心}ジグ中心\index{きかいざひょう@機械座標}機械座標については2023/09/26時点のものである。
その他の(\index{ずめん@図面}図面上の)\index{すんぽう@寸法}寸法として、
\begin{enumerate}
\item テーブル中心 と C面側ジグ端 との水平距離:196.5
\item 受板の円の半径$\rho$:100
\item 受板の鉛直方向の幅$\sigma$:40
\item テーブル中心 と 受板の円の中心 との水平距離$\varDelta$:201.5
\item 受板の円の中心 と 受板の水平方向の底 との距離:70
\end{enumerate}
\end{hosoku}
%%%%%%%%%%%%%%%%%%%%%%%%%%%%%%%%%%%%%%%%%%%%%%%%%%%%%%%%%%
%%%%%%%%%%%%%%%%%%%%%%%%%%%%%%%%%%%%%%%%%%%%%%%%%%%%%%%%%%
%%%%%%%%%%%%%%%%%%%%%%%%%%%%%%%%%%%%%%%%%%%%%%%%%%%%%%%%%%



\clearpage
%%%%%%%%%%%%%%%%%%%%%%%%%%%%%%%%%%%%%%%%%%%%%%%%%%%%%%%%%%
%% section 11.9 %%%%%%%%%%%%%%%%%%%%%%%%%%%%%%%%%%%%%%%%%%
%%%%%%%%%%%%%%%%%%%%%%%%%%%%%%%%%%%%%%%%%%%%%%%%%%%%%%%%%%
\modHeadsection{\ttNum901100\,-\ttNum901199:工具}
\addtocontents{lot}{\protect\addvspace{3pt}}{}{}
\addcontentsline{lot}{section}{\numberline{\thesection}\Sectionname}
\ttNum901100\,-\ttNum901199については、\index{こうぐ@工具}工具に関するもの(工具長や工具径およびその摩耗量を除く)とする。\\

\modcaptionof{table}{\ttNum901100\,-\ttNum901199:工具}
\begin{twoCtable}{}
\ttNum901100 & (予備)\\\hline
\hline
\ttNum901101 & 工具\verb|T02|(フェイスミル)最大刃径(直径)DCX公称値$\phi'_\mathrm D$ & 113.5\\\hline
\ttNum901102 & (端面加工工具用予備)\\\hline
\ttNum901103 & (端面加工工具用予備)\\\hline
\ttNum901104 & (端面加工工具用予備)\\\hline
\hline
\ttNum901105 & 工具\verb|T06|(サイドカッター)厚さ$t$ & 7.0\\\hline
\ttNum901106 & (溝加工工具用予備)\\\hline
\ttNum901107 & (溝加工工具用予備)\\\hline
\ttNum901108 & (溝加工工具用予備)\\\hline
\ttNum901109 & 工具\verb|T08|(サイドカッター)厚さ$t$ & 5.0\\\hline
\ttNum901110 & (溝加工工具用予備)\\\hline
\ttNum901111 & (溝加工工具用予備)\\\hline
\ttNum901112 & (溝加工工具用予備)\\\hline
\hline
\ttNum901113 & (以下 C面取り加工工具用予備)\\\hline
$\vdots$ & \qquad$\vdots$\\\hline
\hline
\ttNum901121 & (以下 外削加工工具用予備)\\\hline
$\vdots$ & \qquad$\vdots$\\\hline
\hline
\ttNum901127 & \verb|T31| Tスロットカッターシャンク直径(公称値) & 25\\\hline
\ttNum901128 & (以下 \dimple 加工工具用予備)\\\hline
$\vdots$ & \qquad$\vdots$
\end{twoCtable}
%%%%%%%%%%%%%%%%%%%%%%%%%%%%%%%%%%%%%%%%%%%%%%%%%%%%%%%%%%
%% hosoku %%%%%%%%%%%%%%%%%%%%%%%%%%%%%%%%%%%%%%%%%%%%%%%%
%%%%%%%%%%%%%%%%%%%%%%%%%%%%%%%%%%%%%%%%%%%%%%%%%%%%%%%%%%
\begin{hosoku}
\index{タッチセンサープローブのじく@タッチセンサープローブの軸}タッチセンサープローブの軸の半径:3.75
\end{hosoku}
%%%%%%%%%%%%%%%%%%%%%%%%%%%%%%%%%%%%%%%%%%%%%%%%%%%%%%%%%%
%%%%%%%%%%%%%%%%%%%%%%%%%%%%%%%%%%%%%%%%%%%%%%%%%%%%%%%%%%
%%%%%%%%%%%%%%%%%%%%%%%%%%%%%%%%%%%%%%%%%%%%%%%%%%%%%%%%%%
%%%%%%%%%%%%%%%%%%%%%%%%%%%%%%%%%%%%%%%%%%%%%%%%%%%%%%%%%%
%% hosoku %%%%%%%%%%%%%%%%%%%%%%%%%%%%%%%%%%%%%%%%%%%%%%%%
%%%%%%%%%%%%%%%%%%%%%%%%%%%%%%%%%%%%%%%%%%%%%%%%%%%%%%%%%%
\begin{hosoku}
工具長・工具径・摩耗量といったオフセットの値は、\pageautoref{sec:A.2}を参照。
\end{hosoku}
%%%%%%%%%%%%%%%%%%%%%%%%%%%%%%%%%%%%%%%%%%%%%%%%%%%%%%%%%%
%%%%%%%%%%%%%%%%%%%%%%%%%%%%%%%%%%%%%%%%%%%%%%%%%%%%%%%%%%
%%%%%%%%%%%%%%%%%%%%%%%%%%%%%%%%%%%%%%%%%%%%%%%%%%%%%%%%%%

\clearpage
~\vfill
%%%%%%%%%%%%%%%%%%%%%%%%%%%%%%%%%%%%%%%%%%%%%%%%%%%%%%%%%%
%%%%%%%%%%%%%%%%%%%%%%%%%%%%%%%%%%%%%%%%%%%%%%%%%%%%%%%%%%
%%%%%%%%%%%%%%%%%%%%%%%%%%%%%%%%%%%%%%%%%%%%%%%%%%%%%%%%%%
\begin{tcolorbox}[title={2023/07/28時点の\MMname 実測値}, fonttitle=\gtfamily\bfseries]
\begin{align*}
  \text{Bot ($B=0$)}
  \left\{
  \begin{array}{rl}
    X: & 97.790 \sim 99.930\\
    Y: & -823.850\\
    Z: & -634.620
  \end{array}
  \right.\quad
  \text{Top ($B=180.$)}
  \left\{
  \begin{array}{rl}
    X: & -97.980 \sim -99.570\\
    Y: & -823.780\\
    Z: & -634.720
  \end{array}
  \right.
\end{align*}\\
・$X$については、ジグの当たる点の凸部と端部($Z$方向は目分量)\\
・$Y$については、モールドの底が当たる面\\
・$Z$については、$X0$ $Y-850.$における、ジグとの接点\\
※これらの値に、\index{タッチセンサーせんたん@タッチセンサー先端}タッチセンサー先端球の半径を加減する必要がある
\end{tcolorbox}
%%%%%%%%%%%%%%%%%%%%%%%%%%%%%%%%%%%%%%%%%%%%%%%%%%%%%%%%%%
%%%%%%%%%%%%%%%%%%%%%%%%%%%%%%%%%%%%%%%%%%%%%%%%%%%%%%%%%%
%%%%%%%%%%%%%%%%%%%%%%%%%%%%%%%%%%%%%%%%%%%%%%%%%%%%%%%%%%
