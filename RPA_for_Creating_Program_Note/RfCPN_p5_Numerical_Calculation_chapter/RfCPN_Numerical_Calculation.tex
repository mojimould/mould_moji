%!TEX root = ./RPA_for_Creating_Program_Note.tex


基本的に、\index{すうちじょうほう@数値情報}数値情報については数値計算用の言語を用いて行うため、その詳細は別ドキュメントに譲る。
ここでは各明細用の\index{メインプログラム}メインプログラムの記述に際して、\index{すうちけいさん@数値計算}数値計算に必要な部分をピックアップする。
なお、ここでは主に\DMname について述べるため、\index{スペーサ}スペーサに関するものは省略する。


%%%%%%%%%%%%%%%%%%%%%%%%%%%%%%%%%%%%%%%%%%%%%%%%%%%%%%%%%%
%% section 30.1 %%%%%%%%%%%%%%%%%%%%%%%%%%%%%%%%%%%%%%%%%%
%%%%%%%%%%%%%%%%%%%%%%%%%%%%%%%%%%%%%%%%%%%%%%%%%%%%%%%%%%
\modHeadsection{振分長・張出長・均等振分角}
各パラメータを以下とする。
\begin{align*}
  \varDelta_x' = \varDelta_x+\sqrt{R_\mathrm i'-\bar l^2}\ , \quad
  R_\mathrm i' = R_\mathrm c-\frac{W_x}2-\rho\ ,\quad
  \bar l = l-\frac\sigma2\ ,\quad
  f_d = \frac{f_\mathrm B-f_\mathrm T}2\ .
\end{align*}


%%%%%%%%%%%%%%%%%%%%%%%%%%%%%%%%%%%%%%%%%%%%%%%%%%%%%%%%%%
%% subsection 30.1.1 %%%%%%%%%%%%%%%%%%%%%%%%%%%%%%%%%%%%%
%%%%%%%%%%%%%%%%%%%%%%%%%%%%%%%%%%%%%%%%%%%%%%%%%%%%%%%%%%
\subsection{振分長}
テーブルを$-\theta$だけ回転させて調整したトップ・ボトム側の\index{ふりわけちょう@振分長}振分長$f'_\mathrm T$, $f'_\mathrm B$は、\pageeqref{eq:saifuriwake}より、
\begin{align*}
  f_\mathrm T' = f_\mathrm T+\varDelta_x'\!\sin\theta~~, \quad
  f_\mathrm B' = (f_\mathrm T+f_\mathrm B)-f_\mathrm T'\ .
\end{align*}


%%%%%%%%%%%%%%%%%%%%%%%%%%%%%%%%%%%%%%%%%%%%%%%%%%%%%%%%%%
%% subsection 30.1.2 %%%%%%%%%%%%%%%%%%%%%%%%%%%%%%%%%%%%%
%%%%%%%%%%%%%%%%%%%%%%%%%%%%%%%%%%%%%%%%%%%%%%%%%%%%%%%%%%
\subsection{張出長}
ジグ(長さ$2l$)からの\index{はりだしちょう@張出長}張出長に換算すると、それぞれ
\begin{align*}
  f_\mathrm T'-l~~, \quad
  f_\mathrm B'-l\ .
\end{align*}


%%%%%%%%%%%%%%%%%%%%%%%%%%%%%%%%%%%%%%%%%%%%%%%%%%%%%%%%%%
%% subsection 30.1.3 %%%%%%%%%%%%%%%%%%%%%%%%%%%%%%%%%%%%%
%%%%%%%%%%%%%%%%%%%%%%%%%%%%%%%%%%%%%%%%%%%%%%%%%%%%%%%%%%
\subsection{均等振分角}
トップ側とボトム側の振分長が均等かつ平行になるときの\index{かたむきかく(ふりわけちょうせい)@傾き角(振分調整)}回転角$\theta'$は、\pageeqref{eq:saifuriwakeangle}より、
\begin{align*}
  \sin\theta' = \frac{f_d}{\varDelta_x'}\ .
\end{align*}



\clearpage
%%%%%%%%%%%%%%%%%%%%%%%%%%%%%%%%%%%%%%%%%%%%%%%%%%%%%%%%%%
%% section 30.2 %%%%%%%%%%%%%%%%%%%%%%%%%%%%%%%%%%%%%%%%%%
%%%%%%%%%%%%%%%%%%%%%%%%%%%%%%%%%%%%%%%%%%%%%%%%%%%%%%%%%%
\modHeadsection{外径中心・湾曲中心・内径中心}


%%%%%%%%%%%%%%%%%%%%%%%%%%%%%%%%%%%%%%%%%%%%%%%%%%%%%%%%%%
%% subsection 30.2.1 %%%%%%%%%%%%%%%%%%%%%%%%%%%%%%%%%%%%%
%%%%%%%%%%%%%%%%%%%%%%%%%%%%%%%%%%%%%%%%%%%%%%%%%%%%%%%%%%
\subsection{外側中心\texorpdfstring{$Y$}{Y}}
\index{ジグ}ジグの底の$Y$座標を$\varDelta_y$とすると、外中心$Y$座標は、
\begin{align*}
  \varDelta_y+\frac{W_y}2\ .
\end{align*}


%%%%%%%%%%%%%%%%%%%%%%%%%%%%%%%%%%%%%%%%%%%%%%%%%%%%%%%%%%
%% subsection 30.2.1 %%%%%%%%%%%%%%%%%%%%%%%%%%%%%%%%%%%%%
%%%%%%%%%%%%%%%%%%%%%%%%%%%%%%%%%%%%%%%%%%%%%%%%%%%%%%%%%%
\subsection{端面の外側中心\texorpdfstring{$X$}{X}}
トップ端およびボトム端における($-\theta$回転後の)\index{そとがわちゅうしん@外側中心}外側中心の$X$位置は、\pageeqref{eq:tableTc}, \pageeqref{eq:tableBc}よりそれぞれ、
\begin{align*}
  \text{トップ端:}\quad
  & \frac{\sqrt{R_\mathrm o^2-f_\mathrm T^2}+\sqrt{R_\mathrm i^2-f_\mathrm T^2}}2-\varDelta_x'\cos\theta~,\\
  \text{ボトム端:}\quad
  &\varDelta_x'\cos\theta-\frac{\sqrt{R_\mathrm o^2-f_\mathrm B^2}+\sqrt{R_\mathrm i^2-f_\mathrm B^2}}2\ .
\end{align*}


%%%%%%%%%%%%%%%%%%%%%%%%%%%%%%%%%%%%%%%%%%%%%%%%%%%%%%%%%%
%% subsection 30.3.1 %%%%%%%%%%%%%%%%%%%%%%%%%%%%%%%%%%%%%
%%%%%%%%%%%%%%%%%%%%%%%%%%%%%%%%%%%%%%%%%%%%%%%%%%%%%%%%%%
\subsection{端面の湾曲中心\texorpdfstring{$X$}{X}}
トップ端およびボトム端における($-\theta$回転後の)\index{わんきょくちゅうしん@湾曲中心}湾曲中心の$X$値は、\pageeqref{eq:tableTRc}, \pageeqref{eq:tableBRc}よりそれぞれ、
\begin{align*}
  \text{トップ端:}\quad
  & \sqrt{R_\mathrm c^2-f_\mathrm T^2}-\varDelta_x'\!\cos\theta~,\\
  \text{ボトム端:}\quad
  &\varDelta_x'\!\cos\theta-\sqrt{R_\mathrm c^2-f_\mathrm B^2}~.
\end{align*}


%%%%%%%%%%%%%%%%%%%%%%%%%%%%%%%%%%%%%%%%%%%%%%%%%%%%%%%%%%
%% subsection 30.3.1 %%%%%%%%%%%%%%%%%%%%%%%%%%%%%%%%%%%%%
%%%%%%%%%%%%%%%%%%%%%%%%%%%%%%%%%%%%%%%%%%%%%%%%%%%%%%%%%%
\subsection{端面の内側中心}
(計算上の)\index{うちがわちゅうしん@内側中心}内側中心は、湾曲中心を持って代用してもよいものとする。



\clearpage
%%%%%%%%%%%%%%%%%%%%%%%%%%%%%%%%%%%%%%%%%%%%%%%%%%%%%%%%%%
%% section 30.3 %%%%%%%%%%%%%%%%%%%%%%%%%%%%%%%%%%%%%%%%%%
%%%%%%%%%%%%%%%%%%%%%%%%%%%%%%%%%%%%%%%%%%%%%%%%%%%%%%%%%%
\modHeadsection{外削}


%%%%%%%%%%%%%%%%%%%%%%%%%%%%%%%%%%%%%%%%%%%%%%%%%%%%%%%%%%
%% subsection 9.3.1 %%%%%%%%%%%%%%%%%%%%%%%%%%%%%%%%%%%%%%
%%%%%%%%%%%%%%%%%%%%%%%%%%%%%%%%%%%%%%%%%%%%%%%%%%%%%%%%%%
\subsection{外削中心:ボトムA面肉厚基準の場合}
\index{テーブルちゅうしん@テーブル中心}テーブル中心\index{P(てーぶるちゅうしん)@P(テーブル中心)}Pを原点としたボトム側外削径の中心$\mathfrak B_\mathrm c'$の(おおよその)$X$座標は、\pageeqref{eq:gaisakucenterBt}より、
\begin{align*}
  \varDelta_x'\cos\theta-\frac{\sqrt{R_\mathrm o^2-f_\mathrm B^2}+\sqrt{R_\mathrm i^2-f_\mathrm B^2}}2
  -\frac{w_\mathrm B}2-\tau_\mathrm B+\frac{\mathfrak W_\mathrm B}2\ .
\end{align*}
このとき、計測したA側内面b$_\mathrm o'$の$X$座標が\pageeqref{eq:gaisakucenterBr}となるように、原点$\mathfrak B_\mathrm c'$を定める。
\begin{align*}
  -\left(\frac{\mathfrak W_\mathrm B}2-\tau_\mathrm B+\mu\right).
\end{align*}
トップ側にも外削がある場合、計測で定めた$\mathfrak B_\mathrm c'$の$X$座標$\mathcal G_{\mathrm Bx}$および\index{とおりしん@通り芯}通り芯$T_x$を用いて\pageeqref{eq:BbasedTx}で与えられる。
\begin{align*}
  -\mathcal G_{Bx}+T_x\ .
\end{align*}


%%%%%%%%%%%%%%%%%%%%%%%%%%%%%%%%%%%%%%%%%%%%%%%%%%%%%%%%%%
%% subsection 30.3.2 %%%%%%%%%%%%%%%%%%%%%%%%%%%%%%%%%%%%%
%%%%%%%%%%%%%%%%%%%%%%%%%%%%%%%%%%%%%%%%%%%%%%%%%%%%%%%%%%
\subsection{外削中心:トップA面肉厚基準の場合}
\index{テーブルちゅうしん@テーブル中心}テーブル中心\index{P(てーぶるちゅうしん)@P(テーブル中心)}Pを原点としたトップ側外削径の中心$\mathfrak T_\mathrm c'$の(おおよその)$X$座標は、\pageeqref{eq:gaisakucenterTt}より、
\begin{align*}
  \frac{\sqrt{R_\mathrm o^2-f_\mathrm T^2}+\sqrt{R_\mathrm i^2-f_\mathrm T^2}}2-\varDelta_x'\cos\theta
  +\frac{w_\mathrm T}2+\tau_\mathrm T-\frac{\mathfrak W_\mathrm T}2\ .
\end{align*}
このとき、計測したA側内面t$_\mathrm o'$の$X$座標が\pageeqref{eq:gaisakucenterTr}となるように、原点$\mathfrak T_\mathrm c'$を定める。
\begin{align*}
  \frac{\mathfrak W_\mathrm T}2-\tau_\mathrm T+\mu~.
\end{align*}
ボトム側にも外削がある場合、計測で定めた$\mathfrak T_\mathrm c'$の$X$座標$\mathcal G_{\mathrm Tx}$および\index{とおりしん@通り芯}通り芯$T_x$を用いて\pageeqref{eq:TbasedTx}で与えられる。
\begin{align*}
  -\mathcal G_{Tx}+T_x
\end{align*}

%%%%%%%%%%%%%%%%%%%%%%%%%%%%%%%%%%%%%%%%%%%%%%%%%%%%%%%%%%
%% subsection 30.3.3 %%%%%%%%%%%%%%%%%%%%%%%%%%%%%%%%%%%%%
%%%%%%%%%%%%%%%%%%%%%%%%%%%%%%%%%%%%%%%%%%%%%%%%%%%%%%%%%%
\subsection{外削長}

%%%%%%%%%%%%%%%%%%%%%%%%%%%%%%%%%%%%%%%%%%%%%%%%%%%%%%%%%%
%% subsubsection 30.3.3.1 %%%%%%%%%%%%%%%%%%%%%%%%%%%%%%%%
%%%%%%%%%%%%%%%%%%%%%%%%%%%%%%%%%%%%%%%%%%%%%%%%%%%%%%%%%%
\subsubsection{ボトムの外削}
ボトム側の外削における工具の先端の$Z$座標は、ボトム側の\index{がいさくちょう@外削長}外削長を$h_\mathrm B$として、
\begin{align*}
  f_\mathrm B'-h_\mathrm B\ .
\end{align*}

\clearpage
%%%%%%%%%%%%%%%%%%%%%%%%%%%%%%%%%%%%%%%%%%%%%%%%%%%%%%%%%%
%% subsubsection 30.3.3.1 %%%%%%%%%%%%%%%%%%%%%%%%%%%%%%%%
%%%%%%%%%%%%%%%%%%%%%%%%%%%%%%%%%%%%%%%%%%%%%%%%%%%%%%%%%%
\subsubsection{トップの外削}
トップ側の外削における工具の先端の$Z$座標は、トップ側の\index{がいさくちょう@外削長}外削長, \index{みぞいち@溝位置}溝位置, \index{みぞはば@溝幅}溝幅をそれぞれ$h_\mathrm T$, $\kappa_p$, $\kappa_w$として、
\begin{alignat*}{3}
  & f_\mathrm T'-h_\mathrm T & \quad & \Big(h_\mathrm T > \kappa_p+\kappa_w\Big)\\
  & f_\mathrm T'-\left(\kappa_p+1[\mathrm{mm}]\right) & \quad  & \Big(h_\mathrm T = \kappa_p+\kappa_w\Big)
\end{alignat*}


%\clearpage
%%%%%%%%%%%%%%%%%%%%%%%%%%%%%%%%%%%%%%%%%%%%%%%%%%%%%%%%%%
%% subsection 30.3.4 %%%%%%%%%%%%%%%%%%%%%%%%%%%%%%%%%%%%%
%%%%%%%%%%%%%%%%%%%%%%%%%%%%%%%%%%%%%%%%%%%%%%%%%%%%%%%%%%
\subsection{湾曲に沿った外削\TBW}
(to be written...)


\clearpage
%%%%%%%%%%%%%%%%%%%%%%%%%%%%%%%%%%%%%%%%%%%%%%%%%%%%%%%%%%
%% section 30.4 %%%%%%%%%%%%%%%%%%%%%%%%%%%%%%%%%%%%%%%%%%
%%%%%%%%%%%%%%%%%%%%%%%%%%%%%%%%%%%%%%%%%%%%%%%%%%%%%%%%%%
\modHeadsection{溝}

%%%%%%%%%%%%%%%%%%%%%%%%%%%%%%%%%%%%%%%%%%%%%%%%%%%%%%%%%%
%% subsection 30.4.1 %%%%%%%%%%%%%%%%%%%%%%%%%%%%%%%%%%%%%
%%%%%%%%%%%%%%%%%%%%%%%%%%%%%%%%%%%%%%%%%%%%%%%%%%%%%%%%%%
\subsection{溝中心\texorpdfstring{$Z$}{Z}}
\index{みぞいち@溝位置}溝位置$\kappa_p$および\index{みぞはば@溝幅}溝幅$\kappa_w$に対し、\index{テーブルちゅうしん@テーブル中心}テーブル中心\index{P(てーぶるちゅうしん)@P(テーブル中心)}Pを原点とした\index{みぞちゅうしん@溝中心}溝の中心の$Z$座標は、\pageeqref{eq:mizocenterZ}より
\begin{align*}
  f_\mathrm T'-\kappa_p-\frac{\kappa_w}2\ .
\end{align*}

%%%%%%%%%%%%%%%%%%%%%%%%%%%%%%%%%%%%%%%%%%%%%%%%%%%%%%%%%%
%% subsection 30.4.2 %%%%%%%%%%%%%%%%%%%%%%%%%%%%%%%%%%%%%
%%%%%%%%%%%%%%%%%%%%%%%%%%%%%%%%%%%%%%%%%%%%%%%%%%%%%%%%%%
\subsection{湾曲中心が基準の場合}
トップ端における湾曲中心T$_{R_\mathrm c}'$と溝中心M$'$との$X$方向の差は、\pageeqref{eq:difTopMizoCenter}より、
\begin{align*}
  \sqrt{R_\mathrm c^2-\left(f_\mathrm T-\kappa_p-\frac{\kappa_w}2\right)^{\!2}}
  -\sqrt{R_\mathrm c^2-f_\mathrm T^2}\ .
\end{align*}


%%%%%%%%%%%%%%%%%%%%%%%%%%%%%%%%%%%%%%%%%%%%%%%%%%%%%%%%%%
%% subsection 30.4.3 %%%%%%%%%%%%%%%%%%%%%%%%%%%%%%%%%%%%%
%%%%%%%%%%%%%%%%%%%%%%%%%%%%%%%%%%%%%%%%%%%%%%%%%%%%%%%%%%
\subsection{外削中心が基準の場合}
\index{みぞちゅうしん@溝中心}溝の中心は\index{トップがわのがいさくちゅうしん@トップ側の外削中心}トップ側の外削中心とする。


%%%%%%%%%%%%%%%%%%%%%%%%%%%%%%%%%%%%%%%%%%%%%%%%%%%%%%%%%%
%% subsection 30.4.4 %%%%%%%%%%%%%%%%%%%%%%%%%%%%%%%%%%%%%
%%%%%%%%%%%%%%%%%%%%%%%%%%%%%%%%%%%%%%%%%%%%%%%%%%%%%%%%%%
\subsection{A側溝深さ指定の場合}

%%%%%%%%%%%%%%%%%%%%%%%%%%%%%%%%%%%%%%%%%%%%%%%%%%%%%%%%%%
%% subsubsection 30.4.4.1 %%%%%%%%%%%%%%%%%%%%%%%%%%%%%%%%
%%%%%%%%%%%%%%%%%%%%%%%%%%%%%%%%%%%%%%%%%%%%%%%%%%%%%%%%%%
\subsubsection{外削のない場合}
\index{Aがわみぞふかさ@A側溝深さ}A側溝深さ$\kappa_d$は、その測定値$\kappa_d'$が図面上の値となるように与えられるものとする。このとき\pageeqref{eq:keydepthDif1}より、
\begin{align*}
  \kappa_d
  &= \frac{2\kappa_d'-\kappa_w\sin\zeta}{1+\cos^2\zeta}\cos\zeta
     +\sqrt{R_\mathrm o^2-\left(f_\mathrm T-\kappa_p-\frac{\kappa_w}2\right)^{\!2}}
     -\sqrt{R_\mathrm o^2-\left(f_\mathrm T-\kappa_p\right)^2}\ .
\end{align*}
ここで$\zeta$は\pageeqref{eq:angleZeta}より、
\begin{align*}
  \tan\zeta
  = \frac{\sqrt{R_\mathrm o^2-\left(f_\mathrm T-\kappa_p-\kappa_w\right)^2}
          -\sqrt{R_\mathrm o^2-\left(f_\mathrm T-\kappa_p\right)^2}}
         {\kappa_w}\ .
\end{align*}
A側溝深さ$\kappa_d$に対し、\index{みぞちゅうしん@溝中心}溝中心の位置の$X$座標は\pageeqref{eq:mizocenterA}より、
\begin{align*}
  \sqrt{R_\mathrm o^2-\left(f_\mathrm T-\kappa_p-\frac{\kappa_w}2\right)^{\!2}}-\kappa_d-\frac{W_{mx}}2
  -\varDelta_x\ .
\end{align*}
また\index{Aがわがいめん@A側外面}A側外面の\index{じっそくち@実測値}実測値を$\mathcal G_m$とすると、\pageeqref{eq:mizocenterAd}より、
\begin{align*}
  \mathcal G_m-\frac{W_{mx}}2-\kappa_d\ .
\end{align*}

%\clearpage
%%%%%%%%%%%%%%%%%%%%%%%%%%%%%%%%%%%%%%%%%%%%%%%%%%%%%%%%%%
%% subsubsection 30.4.4.2 %%%%%%%%%%%%%%%%%%%%%%%%%%%%%%%%
%%%%%%%%%%%%%%%%%%%%%%%%%%%%%%%%%%%%%%%%%%%%%%%%%%%%%%%%%%
\subsubsection{外削のある場合}
\index{Aがわみぞふかさ@A側溝深さ}A側溝深さ$\kappa_d$に対し、トップ外削$X$中心を$\mathcal G_{\mathrm Tx}$とすると、\index{みぞちゅうしん@溝中心}溝中心の位置の$X$座標は\pageeqref{eq:mizocenterAG}より、
\begin{align*}
  \mathcal G_{\mathrm Tx}+\frac{\mathfrak W_x}2-\kappa_d-\frac{W_{mx}}2\ .
\end{align*}



\clearpage
%%%%%%%%%%%%%%%%%%%%%%%%%%%%%%%%%%%%%%%%%%%%%%%%%%%%%%%%%%
%% section 9.2 %%%%%%%%%%%%%%%%%%%%%%%%%%%%%%%%%%%%%%%%%%%
%%%%%%%%%%%%%%%%%%%%%%%%%%%%%%%%%%%%%%%%%%%%%%%%%%%%%%%%%%
\modHeadsection{外側C面取\TBW}
(to be written...)


%\clearpage
%%%%%%%%%%%%%%%%%%%%%%%%%%%%%%%%%%%%%%%%%%%%%%%%%%%%%%%%%%
%% section 9.2 %%%%%%%%%%%%%%%%%%%%%%%%%%%%%%%%%%%%%%%%%%%
%%%%%%%%%%%%%%%%%%%%%%%%%%%%%%%%%%%%%%%%%%%%%%%%%%%%%%%%%%
\modHeadsection{内側C面取\TBW}
(to be written...)



%%%%%%%%%%%%%%%%%%%%%%%%%%%%%%%%%%%%%%%%%%%%%%%%%%%%%%%%%%
%% section 9.2 %%%%%%%%%%%%%%%%%%%%%%%%%%%%%%%%%%%%%%%%%%%
%%%%%%%%%%%%%%%%%%%%%%%%%%%%%%%%%%%%%%%%%%%%%%%%%%%%%%%%%%
\modHeadsection{外側R面取\TBW}
(to be written...)


%\clearpage
%%%%%%%%%%%%%%%%%%%%%%%%%%%%%%%%%%%%%%%%%%%%%%%%%%%%%%%%%%
%% section 9.2 %%%%%%%%%%%%%%%%%%%%%%%%%%%%%%%%%%%%%%%%%%%
%%%%%%%%%%%%%%%%%%%%%%%%%%%%%%%%%%%%%%%%%%%%%%%%%%%%%%%%%%
\modHeadsection{内側R面取\TBW}
(to be written...)


%\clearpage
%%%%%%%%%%%%%%%%%%%%%%%%%%%%%%%%%%%%%%%%%%%%%%%%%%%%%%%%%%
%% section 9.2 %%%%%%%%%%%%%%%%%%%%%%%%%%%%%%%%%%%%%%%%%%%
%%%%%%%%%%%%%%%%%%%%%%%%%%%%%%%%%%%%%%%%%%%%%%%%%%%%%%%%%%
\modHeadsection{座ぐり\TBW}
(to be written...)


%\clearpage
%%%%%%%%%%%%%%%%%%%%%%%%%%%%%%%%%%%%%%%%%%%%%%%%%%%%%%%%%%
%% section 32.2 %%%%%%%%%%%%%%%%%%%%%%%%%%%%%%%%%%%%%%%%%%%
%%%%%%%%%%%%%%%%%%%%%%%%%%%%%%%%%%%%%%%%%%%%%%%%%%%%%%%%%%
\modHeadsection{\dimple\TBW}
(to be written...)
