%!TEX root = ../RPA_for_Creating_Program_Note.tex


ここでは主に内面の\expandafterindex{\dimplekana@\dimple}\textbf{\dimple}に関する計測・加工に必要な幾何学的性質を考える。

なお、\dimple の加工は\MMname で行うことはできず、\DMname のみで行う。
また\DMname では、\index{ふりわけちょう@振分長}振分長の調整について\index{スペーサ}スペーサを用いた方法は行わず、\index{テーブル}テーブルの回転を用いた方法のみで行う方針である。
したがって、スペーサを用いた方法の場合は考慮する必要がない。
そのため以降では、(\dimple に関する計測・加工については)テーブルを$-\theta$だけ回転した場合についてのみを考えることにする。



%%%%%%%%%%%%%%%%%%%%%%%%%%%%%%%%%%%%%%%%%%%%%%%%%%%%%%%%%%
%% section 6.1 %%%%%%%%%%%%%%%%%%%%%%%%%%%%%%%%%%%%%%%%%%%
%%%%%%%%%%%%%%%%%%%%%%%%%%%%%%%%%%%%%%%%%%%%%%%%%%%%%%%%%%
\modHeadsection{\dimple の表記法}
初めに、\dimple に関する\expandafterindex{ひょうき(\dimple)@表記(\dimple)}表記を簡単にまとめておく。
なお\dimple はトップ側にあるため、トップ側が工具側に向いているものとして話を進める。
%%%%%%%%%%%%%%%%%%%%%%%%%%%%%%%%%%%%%%%%%%%%%%%%%%%%%%%%%%
%% tcolorbox %%%%%%%%%%%%%%%%%%%%%%%%%%%%%%%%%%%%%%%%%%%%%
%%%%%%%%%%%%%%%%%%%%%%%%%%%%%%%%%%%%%%%%%%%%%%%%%%%%%%%%%%
\begin{tcolorbox}[title={\dimple に関する表記法}, fonttitle=\gtfamily\bfseries, breakable, enhanced jigsaw]
\begin{enumerate}
\item
\subparagraph*{列の数えかた}
\dimple が$m$列あるとき、トップ側から順に1列目, 2列目, …,$m$列目のように数える。

\item
\subparagraph*{列内の個数の数えかた}
各々の\expandafterindex{れつ(\dimple)@列(\dimple)}列の\dimple の\expandafterindex{こすう(\dimple)@個数(\dimple)}は、AC面側については工具側からみて下から順に、BD面については工具側からみて右から順に1つ目,2つ目,…のように数える。

\item
\subparagraph*{\dimple の寸法}
\expandafterindex{トップたんと\dimple1れつめとのきょり@トップ端と\dimple1列目との距離}トップ端から1列目までの距離を$q$, 鉛直・水平方向の\expandafterindex{ピッチ(\dimplekana)@ピッチ(\dimple)}ピッチをそれぞれ$p_z$, $p_x$とし、\expandafterindex{iれつめのながさ(\dimplekana)@$i$列目の長さ(\dimple)}$i$列目の長さをそれぞれ$d_i$とする。

特に、\expandafterindex{きすうれつめのながさ(\dimplekana)@奇数列目の長さ(\dimple)}奇数列目の長さが全て同じ場合はその長さを$d_\mathrm o$, \expandafterindex{ぐうすうれつめのながさ(\dimplekana)@偶数列目の長さ(\dimple)}偶数列目の長さが全て同じ場合はその長さを$d_\mathrm e$とも表記する。
(\pageautoref{fn:generallyDimpleN}および\pageautoref{hosoku:generallyDimpleN}参照)

\item
\subparagraph*{内径テーパ表の寸法}
\index{ないけいテーパひょう@内径テーパ表}内径テーパ表における\index{トップたんからのきょり(ないけいテーパひょう)@トップ端からの距離(内径テーパ表)}トップ端からの距離を$\lambda_i$ ($i = 0$, $1$, $2$, $\cdots$), それに対するAC・BD側\index{ないけい(ないけいテーパひょう)@内径(内径テーパ表)}内径をそれぞれ$w_{\mathrm Ai}$, $w_{\mathrm Bi}$とする。
(\pageautoref{hosoku:example4taper}参照)

\item
\subparagraph*{内径の(近似)寸法}
トップ端から$\lambda$の位置のAC内径を$w_{\mathrm A\lambda}$と表す。
このとき$w_{\mathrm A\lambda}$は、$\lambda_j \leq \lambda < \lambda_{j+1}$に対する$w_{\mathrm Aj}$, $w_{\mathrm Aj+1}$の\index{かじゅうさんじゅつへいきん(ないけい)@加重算術平均(内径)}加重算術平均(\index{ウェイトさんじゅつへいきん(ないけい)@ウェイト算術平均(内径)}ウェイト算術平均・\index{おもみつきさんじゅつへいきん(ないけい)@重み付き算術平均(内径)}重み付き算術平均)
\begin{align*}
  w_{\mathrm A\lambda}
  = \frac{(\lambda-\lambda_j)w_{\mathrm Aj+1}+(\lambda_{j+1}-\lambda)w_{\mathrm Aj}}{\lambda_{j+1}-\lambda_j}
  \qquad
  \Big(\lambda_j \leq \lambda < \lambda_{j+1}\Big)
\end{align*}
とみなすことにする。($w_{\mathrm B\lambda}$についても同様)

\item
\subparagraph*{めっき厚を含めた内径の(近似)寸法}
\index{めっきまくあつ@めっき膜厚}めっき膜厚$\mu$を考慮したAC・BD内径$w'_{\mathrm A\lambda}$, $w'_{\mathrm B\lambda}$をそれぞれ以下のように表す。
\begin{align*}
  w'_{\mathrm A\lambda} \equiv w_{\mathrm A\lambda}+2\mu~, \quad
  w'_{\mathrm B\lambda} \equiv w_{\mathrm B\lambda}+2\mu~.
\end{align*}
\end{enumerate}
\end{tcolorbox}\noindent
%%%%%%%%%%%%%%%%%%%%%%%%%%%%%%%%%%%%%%%%%%%%%%%%%%%%%%%%%%
%%%%%%%%%%%%%%%%%%%%%%%%%%%%%%%%%%%%%%%%%%%%%%%%%%%%%%%%%%
%%%%%%%%%%%%%%%%%%%%%%%%%%%%%%%%%%%%%%%%%%%%%%%%%%%%%%%%%%
このとき$m$列目の\dimple の個数$n_m$は、$n_m = \nicefrac{d_m}{p_x}+1$となる
%% footnote %%%%%%%%%%%%%%%%%%%%%
\footnote{\label{fn:generallyDimpleN}%
たいていの場合、\expandafterindex{きすうれつのこすう(\dimplekana)@奇数列の個数(\dimple)}奇数列の個数は全て同じ数$n_\mathrm o$であり、\expandafterindex{ぐうすうれつのこすう(\dimplekana)@偶数列の個数(\dimple)}偶数列の個数も全て同じ$n_\mathrm e$である。
また$|n_\mathrm o-n_\mathrm d| = 1$である。}。
%%%%%%%%%%%%%%%%%%%%%%%%%%%%%%%%%
%%%%%%%%%%%%%%%%%%%%%%%%%%%%%%%%%%%%%%%%%%%%%%%%%%%%%%%%%%
%% hosoku %%%%%%%%%%%%%%%%%%%%%%%%%%%%%%%%%%%%%%%%%%%%%%%%
%%%%%%%%%%%%%%%%%%%%%%%%%%%%%%%%%%%%%%%%%%%%%%%%%%%%%%%%%%
\begin{hosoku}[label=hosoku:example4taper]
たとえば内径テーパ表の値が25mm\index{ピッチ(ないけいテーパひょう)@ピッチ(内径テーパ表)}ピッチの場合、$\lambda_0=0$, $\lambda_1=25$, $\lambda_2=50$, $\cdots$とし、それぞれのACおよびBD側\index{ないけい(ないけいテーパひょう)@内径(内径テーパ表)}内径を$w_{\mathrm A0}$, $w_{\mathrm A1}$, $w_{\mathrm A2}$, $\cdots$および$w_{\mathrm B0}$, $w_{\mathrm B1}$, $w_{\mathrm B2}$, $\cdots$とする、という意味である。
ここでは離散値である$\lambda_i$を、連続値$\lambda$に(近似的に)置きかえている。
実際、たとえば$\lambda = \lambda_j$のとき$w_{\mathrm Aj} = w_{\mathrm A\lambda}$となることがわかる。
\end{hosoku}\relax
%%%%%%%%%%%%%%%%%%%%%%%%%%%%%%%%%%%%%%%%%%%%%%%%%%%%%%%%%%
%%%%%%%%%%%%%%%%%%%%%%%%%%%%%%%%%%%%%%%%%%%%%%%%%%%%%%%%%%
%%%%%%%%%%%%%%%%%%%%%%%%%%%%%%%%%%%%%%%%%%%%%%%%%%%%%%%%%%
%%%%%%%%%%%%%%%%%%%%%%%%%%%%%%%%%%%%%%%%%%%%%%%%%%%%%%%%%%
%% hosoku %%%%%%%%%%%%%%%%%%%%%%%%%%%%%%%%%%%%%%%%%%%%%%%%
%%%%%%%%%%%%%%%%%%%%%%%%%%%%%%%%%%%%%%%%%%%%%%%%%%%%%%%%%%
\begin{hosoku}
内径テーパ表の\index{ピッチ(ないけいテーパひょう)@ピッチ(内径テーパ表)}ピッチ$\lambda_{i+1}-\lambda_i$は常に一定の場合が多い。
$\lambda_{i+1}-\lambda_i$が$i$について常に一定であれば、$\lambda_j \leq z < \lambda_{j+1}$となる$j$は、
\begin{align*}
  j = z \bDiv (\lambda_{i+1}-\lambda_i) = \left\lfloor\frac z{\lambda_{i+1}-\lambda_i}\right\rfloor
\end{align*}
のように表すことができる。
\end{hosoku}
%%%%%%%%%%%%%%%%%%%%%%%%%%%%%%%%%%%%%%%%%%%%%%%%%%%%%%%%%%
%%%%%%%%%%%%%%%%%%%%%%%%%%%%%%%%%%%%%%%%%%%%%%%%%%%%%%%%%%
%%%%%%%%%%%%%%%%%%%%%%%%%%%%%%%%%%%%%%%%%%%%%%%%%%%%%%%%%%
\vfill
%%%%%%%%%%%%%%%%%%%%%%%%%%%%%%%%%%%%%%%%%%%%%%%%%%%%%%%%%%
%% Column %%%%%%%%%%%%%%%%%%%%%%%%%%%%%%%%%%%%%%%%%%%%%%%%
%%%%%%%%%%%%%%%%%%%%%%%%%%%%%%%%%%%%%%%%%%%%%%%%%%%%%%%%%%
\begin{Column}{商$\boldsymbol{\bDiv}$と余り$\boldsymbol{\bmod}$とガウス括弧$\boldsymbol{\lfloor\,\rfloor}$}
\renewcommand\theequation{c\thechapter.\arabic{equation}}
\setcounter{equation}{0}
\paragraph*{$\boldsymbol\bDiv$と$\boldsymbol\bmod$}
割り算の余りを表す記号としては\index{mod(あまり)@$\bmod$(余り)}$\bmod$が広く使われる。
商を表す記号は一般的な数学のテキスト等ではあまり用いられないが、プログラミング言語等では\index{div(しょう)@$\bDiv$(商)}$\bDiv$を用いられることがある。
これに倣って、ここでは商には$\bDiv$, 余りには$\bmod$を用いている。

 一般に、実数$a$, $b$ ($b\neq0$)に対して$a = bq+r$ ($0 \leq r < |b|$)を満たす整数$q$を\index{しょう(div)@商($\bDiv$)}商、$r$を\index{あまり(mod)@余り($\bmod$)}余りと呼び、このとき$a \bDiv b = q$および$a \bmod b = r$のように表される。
なお、ここでは簡単のため、$q \geq 0$として考えることにする。
\tcbline*
\paragraph*{ガウス括弧}
\index{ガウスかっこ@ガウス括弧}$\lfloor x\rfloor$は、$x \in R$ に対して$x$を超えない最大の整数。
簡単にいうと、($x > 0$の場合は)小数点以下を切り捨てた整数部分を表す。
\index{ガウスきごう@ガウス記号}ガウス記号, \index{ゆかかんすう@床関数}床関数(floor function)などとも呼ばれる。
\end{Column}



\clearpage
%%%%%%%%%%%%%%%%%%%%%%%%%%%%%%%%%%%%%%%%%%%%%%%%%%%%%%%%%%
%% section 6.2 %%%%%%%%%%%%%%%%%%%%%%%%%%%%%%%%%%%%%%%%%%%
%%%%%%%%%%%%%%%%%%%%%%%%%%%%%%%%%%%%%%%%%%%%%%%%%%%%%%%%%%
\modHeadsection{\dimple 加工の基本方針}
\dimple の加工における留意事項の1つに、モールドの内面(特にトップ端)と工具が接触してしまう\index{アンダーカット}アンダーカットというものがある
%% footnote %%%%%%%%%%%%%%%%%%%%%
\footnote{\dimple の測定・加工ではとりわけアンダーカットが生じやすい、という意味である。
その他の計測・加工についても当然アンダーカットは十分に生じうる。}。
%%%%%%%%%%%%%%%%%%%%%%%%%%%%%%%%%
特にA面側は工具へ向かう方向に\index{わんきょく@湾曲}湾曲があるため、アンダーカットが生じやすい。
そこで、アンダーカットを避けつつ加工ができるようにするため、モールドをいくらか(湾曲と反対側に)傾けて加工を行う。
その\expandafterindex{かたむきかく(\dimplekana)@傾き角(\dimple)}傾き角$\phi$ ($0 \leq \phi < \nicefrac\pi2$)について、ここでは次の2点を基準に考えることにする。
\begin{tcolorbox}[title=A面の\dimple, fonttitle=\gtfamily\bfseries]
\begin{enumerate}
\item[a)] A側内面のトップ端点
\item[b)] A側内面の\dimple1列目(トップ端から$q$)の位置
\end{enumerate}
\end{tcolorbox}\noindent
この2点を通る直線と鉛直方向との角度を、傾き角$-\phi$とする
%% footnote %%%%%%%%%%%%%%%%%%%%%
\footnote{振分長の調整に用いたテーブルの\index{かたむきかく(ふりわけちょうせい)@傾き角(振分調整)}傾き角$\theta$と混同しないように注意。}。
%%%%%%%%%%%%%%%%%%%%%%%%%%%%%%%%%
なお、\index{トップたんのACないけい@トップ端のAC内径}トップ端のAC内径は$w'_{\mathrm A0}$で代用してもよいものとする。
このとき$\phi > 0$となる(C面側に傾く)場合は$\phi$だけ傾けて加工を行う。
一方、$\phi \leq 0$となる(A面側に傾く)場合は、そもそもアンダーカットが生じないので、傾けずにそのまま加工を行うものとする。
%%%%%%%%%%%%%%%%%%%%%%%%%%%%%%%%%%%%%%%%%%%%%%%%%%%%%%%%%%
%% hosoku %%%%%%%%%%%%%%%%%%%%%%%%%%%%%%%%%%%%%%%%%%%%%%%%
%%%%%%%%%%%%%%%%%%%%%%%%%%%%%%%%%%%%%%%%%%%%%%%%%%%%%%%%%%
\begin{hosoku}
ここでは\dimple の工具として、\index{Tスロットカッター}Tスロットカッターを考えている。
しかし、当然ながら\index{こうぐけい@工具径}工具径は有限であるため、いくら適切に傾けたところで限界はある。
ここではその限界として、A側内面のトップ端の$X$座標と、それと最も$X$座標が近い\dimple との($X$方向の)距離を算出する。
そしてそれを工具径と比べることで、どこまでの範囲を加工するかを決定する。
加工できない部分に\dimple がある場合は、別の工具(\index{アングルヘッド}アングルヘッド)を使用して加工を行う。
\end{hosoku}
%%%%%%%%%%%%%%%%%%%%%%%%%%%%%%%%%%%%%%%%%%%%%%%%%%%%%%%%%%
%%%%%%%%%%%%%%%%%%%%%%%%%%%%%%%%%%%%%%%%%%%%%%%%%%%%%%%%%%
%%%%%%%%%%%%%%%%%%%%%%%%%%%%%%%%%%%%%%%%%%%%%%%%%%%%%%%%%%
%%%%%%%%%%%%%%%%%%%%%%%%%%%%%%%%%%%%%%%%%%%%%%%%%%%%%%%%%%
%% Column %%%%%%%%%%%%%%%%%%%%%%%%%%%%%%%%%%%%%%%%%%%%%%%%
%%%%%%%%%%%%%%%%%%%%%%%%%%%%%%%%%%%%%%%%%%%%%%%%%%%%%%%%%%
\begin{Column}{曲率と傾き}
内面A側・C側の\index{わんきょく(ないめん)@湾曲(内面)}湾曲をそれぞれ$\mathcal R_\mathrm o$, $\mathcal R_\mathrm i$とすると、\index{きょくりつ(ないめん)@曲率(内面)}曲率はそれぞれ$\mathcal R_\mathrm o^{-1} < R_\mathrm c^{-1} < \mathcal R_\mathrm i^{-1}$である。
そのため、(トップ側の)A側の$\mathcal R_\mathrm o$を基準にするとより緩やかに、C側の$\mathcal R_\mathrm i$を基準にするとよりきつく傾くことになる。
また、トップ端から($Z$方向に)遠い点を基準にするとより緩やかに、近い点を基準にするとよりきつく傾くことになる。
\end{Column}
%%%%%%%%%%%%%%%%%%%%%%%%%%%%%%%%%%%%%%%%%%%%%%%%%%%%%%%%%%
%%%%%%%%%%%%%%%%%%%%%%%%%%%%%%%%%%%%%%%%%%%%%%%%%%%%%%%%%%
%%%%%%%%%%%%%%%%%%%%%%%%%%%%%%%%%%%%%%%%%%%%%%%%%%%%%%%%%%

以下ではこの傾き角$\phi$と、回転後の\dimple や内面の位置を定量的に与えることを試みる。



\clearpage
%%%%%%%%%%%%%%%%%%%%%%%%%%%%%%%%%%%%%%%%%%%%%%%%%%%%%%%%%6
%% section 6.3 %%%%%%%%%%%%%%%%%%%%%%%%%%%%%%%%%%%%%%%%%%%
%%%%%%%%%%%%%%%%%%%%%%%%%%%%%%%%%%%%%%%%%%%%%%%%%%%%%%%%%%
\modHeadsection{\dimple の位置と傾き角(傾き前)}
\pageeqref{eq:tableTRc}より、テーブルを$-\theta$傾けて振分長の調整を行った場合、テーブル中心Pを原点とした\index{ちゅうしんわんきょくせん@中心湾曲線}中心湾曲線のトップ端における$X$座標は、
\begin{align*}
  R_\mathrm c\cos\alpha_\mathrm c-\Delta'\cos\theta = \sqrt{R_\mathrm c^2-f_\mathrm T^2}-\Delta'\cos\theta
\end{align*}
で与えられる。
これは\index{タッチセンサープローブ}タッチセンサープローブによる\expandafterindex{そくていかいしてん(\dimplekana)@測定開始点(\dimple)}測定の開始点として用いることができる。
一方で、それ以外の作業では、\index{トップたんのないけいちゅうしん@トップ端の内径中心}トップ端における内径の中心座標$g_t$を直接測定するので、それを用いることにする
%% footnote %%%%%%%%%%%%%%%%%%%%%
\footnote{これは\index{ちゅうしんわんきょくせん@中心湾曲線}中心湾曲線上にない点であるが、\index{こうさ@公差}公差の範囲内であるものとして、ここではこれで代用する。}。
%%%%%%%%%%%%%%%%%%%%%%%%%%%%%%%%%
よって、テーブル中心Pを原点とした場合における、\expandafterindex{\dimplekana1れつめ@\dimple1列目}\dimple1列目中央の(だいたいの)位置
%% footnote %%%%%%%%%%%%%%%%%%%%%
\footnote{$w_{\mathrm Aq}$, $w_{\mathrm Bq}$は\index{わんきょくちゅうしん@湾曲中心}湾曲中心\index{O(わんきょくちゅうしん)@O(湾曲中心)}O(0, 0)方向への長さであるため正確ではないことに注意。}\relax
%%%%%%%%%%%%%%%%%%%%%%%%%%%%%%%%%
は、次で与えられる。
\begin{align*}
\begin{array}{rl}
  \text{A面($+X$方向):}
  & \displaystyle
    \left(
      g_{tx}+\mathcal L_0+\frac{w'_{\mathrm Aq}}2~,~
      g_{ty}~,~
      f_t'-q
    \right),\\[12pt]
  \text{C面($-X$方向):}
  & \displaystyle
    \left(
      g_{tx}+\mathcal L_0-\frac{w'_{\mathrm Aq}}2~,~
      g_{ty}~,~
      f_t'-q
    \right),\\[12pt]
  \text{B面($+Y$方向):}
  & \displaystyle
    \left(
      g_{tx}+\mathcal L_0~,~
      g_{ty}+\frac{w'_{\mathrm Bq}}2~,~
      f_t'-q
    \right),\\[12pt]
  \text{D面($-Y$方向):}
  & \displaystyle
    \left(
      g_{tx}+\mathcal L_0~,~
      g_{ty}-\frac{w'_{\mathrm Bq}}2~,~
      f_t'-q
    \right).
\end{array}
\end{align*}
ここで、\expandafterindex{\dimplekana iれつめ@\dimple$i$列目}$i$列目の湾曲中心と\index{トップたんのわんきょくちゅうしん@トップ端の湾曲中心}トップ端の湾曲中心との$X$座標の差を、
%% label{eq:}
\begin{align}
  \label{eq:dimpleCenterDistance}
  \mathcal L_i
  \equiv \sqrt{R_\mathrm c^2-\left\{f_\mathrm T-q-(i-1)p_z\right\}^2}-\sqrt{R_\mathrm c^2-f_\mathrm T^2}
\end{align}
と表した。
なお、$i$列目の湾曲中心と$j$列目の湾曲中心との$X$座標の差を
\begin{align*}
  \mathcal L_{i,j}
  \equiv \mathcal L_i-\mathcal L_j
  = \sqrt{R_\mathrm c^2-\left(f_\mathrm T-q-(i-1)p_z\right)^2}
    -\sqrt{R_\mathrm c^2-\left\{f_\mathrm T-q-(j-1)p_z\right\}^2}
\end{align*}
と表すことにする。


%%%%%%%%%%%%%%%%%%%%%%%%%%%%%%%%%%%%%%%%%%%%%%%%%%%%%%%%%%
%% subsection 5.3.1 %%%%%%%%%%%%%%%%%%%%%%%%%%%%%%%%%%%%%%
%%%%%%%%%%%%%%%%%%%%%%%%%%%%%%%%%%%%%%%%%%%%%%%%%%%%%%%%%%
\subsection{\dimple の\texorpdfstring{$X$}{X}座標(傾き前)}
テーブル中心Pを原点としたとき、傾き前の\expandafterindex{\dimplekana iれつめjばんめ@\dimple$i$列目$j$番目}$i$列目$j$番目の\dimple の$X$座標は、
%% label{eq:dPosXBefore}
\begin{align}
  \notag
  \text{A面:}\quad
  \mathcal D_{xi,\mathrm A}
  &= g_{tx}+\mathcal L_i+\frac{w'_{\mathrm Aq+(i-1)p_z}}2\\
  \label{eq:dPosXBefore}
  \text{C面:}\quad
  \mathcal D_{xi,\mathrm C}
  &= g_{tx}+\mathcal L_i-\frac{w'_{\mathrm Aq+(i-1)p_z}}2\\
  \notag
  \text{B, D面:}\quad
  \mathcal D_{xij,\mathrm B}
  &= g_{tx}+\mathcal L_i+\frac{d_i}2-(j-1)p_x
\end{align}
なお、A・C面については$j$に依らないことがわかる。
そのため、たとえば$\mathcal D_{xij,\mathrm A}$ではなく、$\mathcal D_{xi,\mathrm A}$のように表記している。


\clearpage
%%%%%%%%%%%%%%%%%%%%%%%%%%%%%%%%%%%%%%%%%%%%%%%%%%%%%%%%%%
%% subsection 5.3.2 %%%%%%%%%%%%%%%%%%%%%%%%%%%%%%%%%%%%%%
%%%%%%%%%%%%%%%%%%%%%%%%%%%%%%%%%%%%%%%%%%%%%%%%%%%%%%%%%%
\subsection{\dimple の\texorpdfstring{$Y$}{Y}座標(傾き前)}
テーブル中心Pを原点としたとき、傾き前の\expandafterindex{\dimplekana iれつめjばんめ@\dimple$i$列目$j$番目}$i$列目$j$番目の\dimple の$Y$座標は、
%% label{eq:dPosYBefore}
\begin{alignat}{3}
  \notag
  \text{A, C面:}\quad
  && \mathcal D_{yij,\mathrm A} &= g_{ty}-\frac{d_i}2+(j-1)p_x\\
  \label{eq:dPosYBefore}
  \text{B面:}\quad
  && \mathcal D_{yi,\mathrm B} &= g_{ty}+\frac{w'_{\mathrm Bq+(i-1)p_z}}2\\
  \notag
  \text{D面:}\quad
  && \mathcal D_{yi,\mathrm D} &= g_{ty}-\frac{w'_{\mathrm Bq+(i-1)p_z}}2
\end{alignat}
B・D面については$j$に依らないことがわかる。


%%%%%%%%%%%%%%%%%%%%%%%%%%%%%%%%%%%%%%%%%%%%%%%%%%%%%%%%%%
%% subsection 5.3.3 %%%%%%%%%%%%%%%%%%%%%%%%%%%%%%%%%%%%%%
%%%%%%%%%%%%%%%%%%%%%%%%%%%%%%%%%%%%%%%%%%%%%%%%%%%%%%%%%%
\subsection{\dimple の\texorpdfstring{$Z$}{Z}座標(傾き前)}
テーブル中心Pを原点としたとき、傾き前の\expandafterindex{\dimplekana iれつめjばんめ@\dimple$i$列目$j$番目}$i$列目$j$番目の\dimple の$Z$座標は、
%% label{eq:dPosZBefore}
\begin{align}
  \label{eq:dPosZBefore}
  \text{A, B, C, D面:}\quad
  \mathcal D_{zi} = f_t'-q-(i-1)p_z
\end{align}
$Z$座標についてはどの面も$j$に依らないことがわかる。


%%%%%%%%%%%%%%%%%%%%%%%%%%%%%%%%%%%%%%%%%%%%%%%%%%%%%%%%%%
%% subsection 5.3.4 %%%%%%%%%%%%%%%%%%%%%%%%%%%%%%%%%%%%%%
%%%%%%%%%%%%%%%%%%%%%%%%%%%%%%%%%%%%%%%%%%%%%%%%%%%%%%%%%%
\subsection{傾き角}
\index{トップがわのAがわうちたんてん@トップ側のA側内端点}A側内面トップ端と、\expandafterindex{Aがわないめん(\dimplekana1れつめ)@A側内面(\dimple1列目)}A側内面のトップ端から$q$の位置との$x$方向の差は、
\begin{align*}
  \sqrt{\left(R_\mathrm c+\frac{w'_{\mathrm Aq}}2\right)^2-(f_\mathrm T-q)^2}
  -\sqrt{\left(R_\mathrm c+\frac{w'_{\mathrm A0}}2\right)^2-f_\mathrm T^2}
\end{align*}
で与えられる。
このとき、これが負になる場合は傾ける必要はなく、正となる場合のみ傾ける。
したがってその\index{かたむきかく(\dimplekana)@傾き角(\dimple)}傾き角$\phi$は、
%% label{eq:dKatamuki}
\begin{subequations}
\label{eq:dKatamuki}
\begin{alignat}{2}
  \text{正の場合:}&&\quad
  \tan\phi
  &= \frac{\displaystyle
           \sqrt{\left(R_\mathrm c+\frac{w'_{\mathrm Aq}}2\right)^2-(f_\mathrm T-q)^2}
           -\sqrt{\left(R_\mathrm c+\frac{w'_{\mathrm A0}}2\right)^2-f_\mathrm T^2}}q\\[8pt]
  \text{負の場合:}&&
  \phi
  &= 0
\end{alignat}
\end{subequations}
で与えられる。
%%%%%%%%%%%%%%%%%%%%%%%%%%%%%%%%%%%%%%%%%%%%%%%%%%%%%%%%%%
%% Column %%%%%%%%%%%%%%%%%%%%%%%%%%%%%%%%%%%%%%%%%%%%%%%%
%%%%%%%%%%%%%%%%%%%%%%%%%%%%%%%%%%%%%%%%%%%%%%%%%%%%%%%%%%
\begin{Column}{傾き角が負となるモールド}
$\phi < 0$となるのは、
\begin{align*}
  & \left(R_\mathrm c+\frac{w'_{\mathrm Aq}}2\right)^2-(f_\mathrm T-q)^2
    < \left(R_\mathrm c+\frac{w'_{\mathrm A0}}2\right)^2-f_\mathrm T^2\\
  \longrightarrow~~
  & \frac{w_{\mathrm A0}-w_{\mathrm Aq}}2
    \left(2R_\mathrm c+\frac{w_{\mathrm A0}'+w_{\mathrm Aq}'}2\right)
    > q(2f_\mathrm T-q)
\end{align*}
の場合である。
したがって、以下のような場合に生じる傾向があることがわかる。
\begin{enumerate}
\item \index{きょくりつ@曲率}曲率が小さい(湾曲$R_\mathrm c$が大きい)
\item \index{ないめんテーパ@内面テーパ}内面テーパがきつい($w_{\mathrm A0}-w_{\mathrm Aq}$が大きい)
\item \index{ないけい@内径}内径・\index{めっきまくあつ@めっき膜厚}めっき膜厚が大きい($w_\mathrm A'$が大きい)
\end{enumerate}
たとえば、曲率0 ($R = +\infty$)の\index{モールド}モールド、つまり(\index{がいけい(モールド)@外形(モールド)}外形が)まっすぐの\index{わんきょくのないモールド@湾曲のないモールド}湾曲のないモールドなどが、これに該当する。
\end{Column}
%%%%%%%%%%%%%%%%%%%%%%%%%%%%%%%%%%%%%%%%%%%%%%%%%%%%%%%%%%
%%%%%%%%%%%%%%%%%%%%%%%%%%%%%%%%%%%%%%%%%%%%%%%%%%%%%%%%%%
%%%%%%%%%%%%%%%%%%%%%%%%%%%%%%%%%%%%%%%%%%%%%%%%%%%%%%%%%%
%%%%%%%%%%%%%%%%%%%%%%%%%%%%%%%%%%%%%%%%%%%%%%%%%%%%%%%%%%
%% Column %%%%%%%%%%%%%%%%%%%%%%%%%%%%%%%%%%%%%%%%%%%%%%%%
%%%%%%%%%%%%%%%%%%%%%%%%%%%%%%%%%%%%%%%%%%%%%%%%%%%%%%%%%%
\begin{Column}{C側\dimple の傾き角}
\expandafterindex{Cがわ\dimplekana@C側\dimple}C側\dimple については傾斜が外側に向いているため、傾けなくとも\index{アンダーカット}アンダーカットの心配はない。
しかし、傾けたまま加工をすると形状が歪になってしまうため、\dimple の形状をより円に近い形にするためには傾いていないほうが望ましい。
また一方で、面によって傾ける傾けないを分けると、プログラムが複雑になる(\index{じょうけんぶんき@条件分岐}条件分岐が増える)要因にもなる。
そのためここでは、どの面の\dimple に対しても同じ角度$\phi$を用いて加工を行うことにする。
\tcbline*
なお、C面に対する\dimple の形状をできるだけよいものにするには、\index{Cがわないめんテーパ@C側内面テーパ}C側の内面テーパに基づいた角度を用いるほうが望ましい。
そのため、C側\dimple に対する\expandafterindex{かたむきかく(Cがわ\dimplekana)@傾き角(C側\dimple)}傾き角$\phi_\mathrm C$についても(1つの例として)与えておく。
具体的には、以下の2点を基準として角度$\phi_\mathrm C$を取ることとする。
\begin{enumerate}
\item[a)] C側内面の\dimple1列目(トップ端から$q$)の位置
\item[b)] C側内面の\dimple$m$列目(トップ端から$q+(m-1)p_z$)の位置
\end{enumerate}
C側内面のトップ端から$q$の位置と、C側内面のトップ端から$q+(m-1)p_z$の位置との$x$方向の差は、
\begin{align*}
  \sqrt{\left(R_\mathrm c-\frac{w'_{\mathrm Aq+(m-1)p_z}}2\right)^2
        -\left\{f_\mathrm T-q-(m-1)p_z\right\}^2}
  -\sqrt{\left(R_\mathrm c-\frac{w'_{\mathrm Aq}}2\right)^2-(f_\mathrm T-q)^2}
\end{align*}
これより、C側\dimple に対する傾き角$\phi_\mathrm C$ ($\phi_\mathrm C > 0$)は、
\begin{align*}
  \tan\phi_\mathrm C
  = \frac{\displaystyle
          \sqrt{\left(R_\mathrm c-\frac{w'_{\mathrm Aq+(m-1)p_z}}2\right)^2
                -\left\{f_\mathrm T-q-(m-1)p_z\right\}^2}
          -\sqrt{\left(R_\mathrm c-\frac{w'_{\mathrm Aq}}2\right)^2-(f_\mathrm T-q)^2}}
         {(m-1)p_z}
\end{align*}
で与えられる。
なお、前述の通り$w_{\mathrm Aq+(m-1)p_z}$は$\lambda_j \leq q+(m-1)p_z < \lambda_{j+1}$に対する$w_{\mathrm Aj}$, $w_{\mathrm Aj+1}$の\index{かじゅうさんじゅつへいきん(ないけい)@加重算術平均(内径)}加重算術平均
\begin{align*}
  w_{\mathrm Aq+(m-1)p_z}
  = \frac{\{q+(m-1)p_z-\lambda_j\}w_{\mathrm Aj+1}+\{\lambda_{j+1}-q-(m-1)p_z\}w_{\mathrm Aj}}
         {\lambda_{j+1}-\lambda_j}
\end{align*}
であり、\index{ないけい@内径}内径として代用している。($w_{\mathrm Bq+(m-1)p_z}$についても同様)
\end{Column}
%%%%%%%%%%%%%%%%%%%%%%%%%%%%%%%%%%%%%%%%%%%%%%%%%%%%%%%%%%
%%%%%%%%%%%%%%%%%%%%%%%%%%%%%%%%%%%%%%%%%%%%%%%%%%%%%%%%%%
%%%%%%%%%%%%%%%%%%%%%%%%%%%%%%%%%%%%%%%%%%%%%%%%%%%%%%%%%%


\clearpage
%%%%%%%%%%%%%%%%%%%%%%%%%%%%%%%%%%%%%%%%%%%%%%%%%%%%%%%%%%
%% subsection 26.3.5 %%%%%%%%%%%%%%%%%%%%%%%%%%%%%%%%%%%%%
%%%%%%%%%%%%%%%%%%%%%%%%%%%%%%%%%%%%%%%%%%%%%%%%%%%%%%%%%%
\subsection{B, D面の\dimple の位置(傾き前)}
B, D側\dimple において、その$X$座標がA側内面に最も近いものは、$m-1$列目または$m$列目の1番目の\dimple である。
これらの$X$座標は\pageeqref{eq:dPosXBefore}よりそれぞれ、
\begin{align*}
  m-1\text{列目:}&\quad
  g_{tx}+\mathcal L_{m-1}+\frac{d_{m-1}}2\\
  m\text{列目:}&\quad
  g_{tx}+\mathcal L_m+\frac{d_m}2
\end{align*}
%%%%%%%%%%%%%%%%%%%%%%%%%%%%%%%%%%%%%%%%%%%%%%%%%%%%%%%%%%
%% hosoku %%%%%%%%%%%%%%%%%%%%%%%%%%%%%%%%%%%%%%%%%%%%%%%%
%%%%%%%%%%%%%%%%%%%%%%%%%%%%%%%%%%%%%%%%%%%%%%%%%%%%%%%%%%
\begin{hosoku}
$d_{m-1} > d_m$のときは$m-1$列目, $d_m > d_{m-1}$のときは$m$列目をみればよい。
\end{hosoku}
%%%%%%%%%%%%%%%%%%%%%%%%%%%%%%%%%%%%%%%%%%%%%%%%%%%%%%%%%%
%%%%%%%%%%%%%%%%%%%%%%%%%%%%%%%%%%%%%%%%%%%%%%%%%%%%%%%%%%
%%%%%%%%%%%%%%%%%%%%%%%%%%%%%%%%%%%%%%%%%%%%%%%%%%%%%%%%%%
A側内面のトップ端からの($X$方向の)距離は、\index{トップたんのACないけい@トップ端のAC内径}トップ端のAC側内径として$w'_{\mathrm A0}$を代用すると、それぞれ
\begin{align*}
  m-1\text{列目:}&\quad
  \frac{w'_{\mathrm A0}}2-\mathcal L_{m-1}-\frac{d_{m-1}}2\\
  m\text{列目:}&\quad
  \frac{w'_{\mathrm A0}}2-\mathcal L_m-\frac{d_m}2
\end{align*}
これらのいずれか小さいほうが\index{こうぐけい@工具径}工具径(半径)よりも小さければ、\index{モールド}モールドを傾けて加工をする必要があると判断できる
%% footnote %%%%%%%%%%%%%%%%%%%%%
\footnote{もちろん、いくらか余裕代をとる必要がある。}。
%%%%%%%%%%%%%%%%%%%%%%%%%%%%%%%%%
\vfill
%%%%%%%%%%%%%%%%%%%%%%%%%%%%%%%%%%%%%%%%%%%%%%%%%%%%%%%%%%
%% Column %%%%%%%%%%%%%%%%%%%%%%%%%%%%%%%%%%%%%%%%%%%%%%%%
%%%%%%%%%%%%%%%%%%%%%%%%%%%%%%%%%%%%%%%%%%%%%%%%%%%%%%%%%%
\begin{Column}{B, D側\dimple 加工で考慮すべき点}
\paragraph*{工具径とシャンク径}
\index{アンダーカット}アンダーカットが生じるのは主に\index{トップがわのAがわうちたんてん@トップ側のA側内端点}(A側内面の)トップ端なので、実際には\index{こうぐけい@工具径}工具径(工具の切削する部分)ではなく\index{シャンクけい(Tスロットカッター)@シャンク径(Tスロットカッター)}シャンク径等(工具のトップ端に相当する箇所)でよい。
そのため工具径よりシャンク径のほうが小さい場合は、より広い範囲の(B, D面の)\dimple を傾けずに切削することが可能となる。
\tcbline*
\paragraph*{端面の削り代}
\dimple の測定・加工は、端面を切削する前に行う。
そのため測定・加工の際は、\index{ぜんけずりしろ(トップたんめん)@全削り代(トップ端面)}端面の削り代の分だけ大きい(長い)ことに注意する必要がある。
削り代の分だけ\index{わんきょく@湾曲}湾曲も加味する必要があり、特に\index{Aがわないめん(トップがわ)@A側内面(トップ側)}A側内面と工具とのアンダーカットに留意しなければならない。
\tcbline*
\paragraph*{その他のずれ}
\index{モールドのけいじょう@モールドの形状}モールドの形状は当然ながら\index{ずめん(モールド)@図面(モールド)}図面のものとは一致はしない。
特に\index{わんきょく@湾曲}湾曲や\index{にくあつ@肉厚}肉厚などの図面とのずれは、アンダーカットに大きく寄与するのでこれも注意する必要がある。
\end{Column}
%%%%%%%%%%%%%%%%%%%%%%%%%%%%%%%%%%%%%%%%%%%%%%%%%%%%%%%%%%
%%%%%%%%%%%%%%%%%%%%%%%%%%%%%%%%%%%%%%%%%%%%%%%%%%%%%%%%%%
%%%%%%%%%%%%%%%%%%%%%%%%%%%%%%%%%%%%%%%%%%%%%%%%%%%%%%%%%%



\clearpage
%%%%%%%%%%%%%%%%%%%%%%%%%%%%%%%%%%%%%%%%%%%%%%%%%%%%%%%%%%
%% section 7.4 %%%%%%%%%%%%%%%%%%%%%%%%%%%%%%%%%%%%%%%%%%%
%%%%%%%%%%%%%%%%%%%%%%%%%%%%%%%%%%%%%%%%%%%%%%%%%%%%%%%%%%
\modHeadsection{傾き後の\dimple}
機内での回転は\index{テーブルちゅうしん@テーブル中心}テーブル中心Pを\index{げんてんP@原点P}原点として行われる。
また\dimple の加工は\index{トップたんのないけいちゅうしん@トップ端の内径中心}トップ端における内径中心を基準にして切削を行う。
傾ける前の\index{トップたんのないけいちゅうしん@トップ端の内径中心}トップ端内径中心$g_t$の座標は実測により(Pを中心とした$XYZ$直交座標でいうところの)[$g_{tx}$, $g_{ty}$, $f_t'$]で与えられる
%% footnote %%%%%%%%%%%%%%%%%%%%%
\footnote{ここではこれをテーブル中心Pを原点とした座標値として取り扱っている。
しかし、計測では機械座標系の値として$g_t$が与えられる。
たとえば\DMname の場合、$g_t$は通常(今の場合はテーブル中心Pより負側に湾曲中心があることが多いので)負の値として得られることに注意。
(ここでの計測では$XY$成分のみであり、$Z$については計測しないことにも注意。)}。
%%%%%%%%%%%%%%%%%%%%%%%%%%%%%%%%%
このとき、テーブルを角度$-\phi$だけ傾けた後のトップ端内面中心の座標$g'_t$は
%% footnote %%%%%%%%%%%%%%%%%%%%%
\footnote{これらをワーク座標原点としてもよいし、ワーク座標原点$g_t$はそのままで各面ごとに傾けてもよい。
ここでは後者の方法で加工を行うものとする。}、
%%%%%%%%%%%%%%%%%%%%%%%%%%%%%%%%%
%% label{eq:afterPhiTCenterFromO}
\begin{align}
  \label{eq:afterPhiTCenterFromO}
  \left[
  \begin{array}{c}
    g_{tx}'\\
    g_{ty}'\\
    g_{tz}'
  \end{array}
  \right]
  =\left[
   \begin{array}{c}
     g_{tx}\cos\phi+f_t'\sin\phi\\
     g_{ty}\\
     -g_{tx}\sin\phi+f_t'\cos\phi
   \end{array}
   \right].
   \end{align}
同様に、$i$列目における(傾ける前の)湾曲中心の位置は、[$g_{tx}+\mathcal L_i$, $g_{ty}$, $f_t'-q-(i-1)p_z$]で与えられる
%% footnote %%%%%%%%%%%%%%%%%%%%%
\footnote{ここではトップ端における湾曲中心を、トップ端における内径中心と同一視している。}
%%%%%%%%%%%%%%%%%%%%%%%%%%%%%%%%%
ので、テーブルを角度$-\phi$だけ傾けた後の$i$列目における湾曲中心の位置は、
\begin{align*}
  \left[
  \begin{array}{c}
    (g_{tx}+\mathcal L_i)\cos\phi+\{f_t'-q-(i-1)p_z\}\sin\phi\\
    g_{ty}\\
    -(g_{tx}+\mathcal L_i)\sin\phi+\{f_t'-q-(i-1)p_z\}\cos\phi
  \end{array}
  \right].
\end{align*}
したがって、傾けた後のトップ端の湾曲中心と$i$列目に対する湾曲中心との差分は、
%% label{eq:afterPhidimpleCenterDistance}
\begin{align}
  \label{eq:afterPhidimpleCenterDistance}
  \left[
  \begin{array}{c}
    \mathcal L_i\cos\phi-\{q+(i-1)p_z\}\sin\phi\\
    0\\
    -\mathcal L_i\sin\phi-\{q+(i-1)p_z\}\cos\phi
  \end{array}
  \right].
\end{align}
%%%%%%%%%%%%%%%%%%%%%%%%%%%%%%%%%%%%%%%%%%%%%%%%%%%%%%%%%%
%% hosoku %%%%%%%%%%%%%%%%%%%%%%%%%%%%%%%%%%%%%%%%%%%%%%%%
%%%%%%%%%%%%%%%%%%%%%%%%%%%%%%%%%%%%%%%%%%%%%%%%%%%%%%%%%%
\begin{hosoku}
傾けた後の$i$列目に対する湾曲中心と$j$列目に対する湾曲中心との差分は、
\begin{align*}
  \left[
  \begin{array}{c}
    \mathcal L_{j,i}\cos\phi-(j-i)p_z\sin\phi\\
    0\\
    -\mathcal L_{j,i}\sin\phi-(j-i)p_z\cos\phi
  \end{array}
  \right].
\end{align*}
特に、$j = i+1$の場合は、
\begin{align*}
  \left[
  \begin{array}{c}
    \mathcal L_{i+1,i}\cos\phi-p_z\sin\phi\\
    0\\
    -\mathcal L_{i+1,i}\sin\phi-p_z\cos\phi
  \end{array}
  \right].
\end{align*}
\end{hosoku}
%%%%%%%%%%%%%%%%%%%%%%%%%%%%%%%%%%%%%%%%%%%%%%%%%%%%%%%%%%
%%%%%%%%%%%%%%%%%%%%%%%%%%%%%%%%%%%%%%%%%%%%%%%%%%%%%%%%%%
%%%%%%%%%%%%%%%%%%%%%%%%%%%%%%%%%%%%%%%%%%%%%%%%%%%%%%%%%%

\clearpage
%%%%%%%%%%%%%%%%%%%%%%%%%%%%%%%%%%%%%%%%%%%%%%%%%%%%%%%%%%
%% Column %%%%%%%%%%%%%%%%%%%%%%%%%%%%%%%%%%%%%%%%%%%%%%%%
%%%%%%%%%%%%%%%%%%%%%%%%%%%%%%%%%%%%%%%%%%%%%%%%%%%%%%%%%%
\begin{Column}{タッチセンサープローブ径の考慮:$XY$と$Z$方向の非対称性}
マシニング内の計測では\index{タッチセンサープローブ}タッチセンサープローブを用いる。
そのため、\index{タッチセンサープローブせんたんきゅう@タッチセンサープローブ先端球}タッチセンサープローブ先端球の径の大きさに対して考慮・補正しなければならない。
タッチセンサープローブの位置の基準については、以下のようにとるのが通常である。
\begin{enumerate}
\item $X$方向:基準はタッチセンサープローブ先端球の($X$方向の)中心
\item $Y$方向:基準はタッチセンサープローブ先端球の($Y$方向の)中心
\item $Z$方向:基準はタッチセンサープローブ先端球の($Z$方向の)先端
\end{enumerate}
したがって、$XY$方向と$Z$方向とでは\index{きじゅんてん@基準点(タッチセンサープローブ)}基準点が異なり非対称となっている。
今の場合、基準が非対称な$X$と$Z$が混合する移動(回転)であるが、あくまでもタッチセンサープローブの先端(上記の基準点)が回転後の位置にある、ということである。
そのため補正については(傾きに関係なく)$Z$方向に対してのみ径の半分だけ補正すればよい。
\end{Column}
%%%%%%%%%%%%%%%%%%%%%%%%%%%%%%%%%%%%%%%%%%%%%%%%%%%%%%%%%%
%%%%%%%%%%%%%%%%%%%%%%%%%%%%%%%%%%%%%%%%%%%%%%%%%%%%%%%%%%
%%%%%%%%%%%%%%%%%%%%%%%%%%%%%%%%%%%%%%%%%%%%%%%%%%%%%%%%%%


%%%%%%%%%%%%%%%%%%%%%%%%%%%%%%%%%%%%%%%%%%%%%%%%%%%%%%%%%%
%% subsection 5.4.1 %%%%%%%%%%%%%%%%%%%%%%%%%%%%%%%%%%%%%%
%%%%%%%%%%%%%%%%%%%%%%%%%%%%%%%%%%%%%%%%%%%%%%%%%%%%%%%%%%
\subsection{傾き後の\dimple(A, C面側)}
\expandafterindex{かたむきかく(\dimplekana)@傾き角(\dimple)}傾ける角度$\phi$は\pageeqref{eq:dKatamuki}で与えられる。
このとき、傾けた後のAおよびC面側に対する$i$列目$j$番目の\dimple の位置は、\pageeqref{eq:dPosXBefore}, \eqref{eq:dPosYBefore}, \pageeqref{eq:dPosZBefore}より、
\begin{alignat*}{3}
  \text{A面:}&~~&
  \left[
  \begin{array}{c}
    \mathcal D_{xij,\mathrm A}'\\
    \mathcal D_{yij,\mathrm A}'\\
    \mathcal D_{zij,\mathrm A}'
  \end{array}
  \right]
 &= \left[
    \begin{array}{c}
      \mathcal D_{xi,\mathrm A}\cos\phi+\mathcal D_{zi}\sin\phi\\
      \mathcal D_{yij,\mathrm A}\\
      -\mathcal D_{xi,\mathrm A}\sin\phi+\mathcal D_{zi}\cos\phi
    \end{array}
    \right],\\[2pt]
  \text{C面:}&~~&
  \left[
  \begin{array}{c}
    \mathcal D_{xij,\mathrm C}'\\
    \mathcal D_{yij,\mathrm C}'\\
    \mathcal D_{zij,\mathrm C}'
  \end{array}
  \right]
 &= \left[
    \begin{array}{c}
      \mathcal D_{xi,\mathrm C}\cos\phi+\mathcal D_{zi}\sin\phi\\
      \mathcal D_{yij,\mathrm A}\\
      -\mathcal D_{xi,\mathrm C}\sin\phi+\mathcal D_{zi}\cos\phi
    \end{array}
    \right].
\end{alignat*}
特に、各列の中央(各列の\expandafterindex{わんきょくちゅうしん(\dimplekana)@湾曲中心(\dimple)}湾曲中心)$[g_{tx}+\mathcal L_i, g_{ty}, f_t'-q-(i-1)p_z]$を原点としてみた場合の位置は、
\begin{align*}
  \left[
  \begin{array}{c}
    \displaystyle \pm\frac{w_{Aq+(i-1)p_z}'}2\cos\phi\\[6pt]
    \displaystyle -\frac{d_i}2+(j-1)p_x\\[6pt]
    \displaystyle \mp\frac{w_{Aq+(i-1)p_z}'}2\sin\phi
  \end{array}
  \right]\qquad
  %%%%%%%%
  \left(
  \text{複号}
  \left\{
  \begin{array}{rl}
    \!\text{上}\!\!\!\!& \text{: A面}\\
    \!\text{下}\!\!\!\!& \text{: C面}\\
  \end{array}
  \right.
  \right).
\end{align*}





\paragraph*{$j$方向の差分}
$Y$方向の隣同士の差分、すなわち$i$を固定したときの$j$番目と$j+1$番目の位置の差分は、
\begin{align*}
  \left[
  \begin{array}{c}
    0\\
    \mathcal D_{yi(j+1),\mathrm A}-\mathcal D_{yij,\mathrm A}\\
    0
  \end{array}
  \right]
  = \left[
    \begin{array}{c}
      0\\
      p_x\\
      0
    \end{array}
    \right]\ .
\end{align*}


\clearpage
\paragraph*{$i$方向の差分}
$Z$方向の隣同士の差分、すなわち$j$を固定したときの$i$番目と$i+1$番目の位置の差分については、
\begin{align*}
 &\left[
  \begin{array}{c}
    (\mathcal D_{x(i+1),\mathrm A}-\mathcal D_{xi,\mathrm A})\cos\phi
    +(\mathcal D_{z(i+1)}-\mathcal D_{zi})\sin\phi\\
    (\mathcal D_{y(i+1)j,\mathrm A}-\mathcal D_{yij,\mathrm A})\\
    (\mathcal D_{xi,\mathrm A}-\mathcal D_{x(i+1),\mathrm A})\sin\phi
    +(\mathcal D_{z(i+1)}-\mathcal D_{zi})\cos\phi
  \end{array}
  \right]\\
 &= \left[
    \begin{array}{c}
      \displaystyle
      \left(\mathcal L_{i+1, i}+\frac{w'_{\mathrm Aq+ip_z}-w'_{\mathrm Aq+(i-1)p_z}}2\right)\cos\phi
      -p_z\sin\phi\\[6pt]
      \displaystyle-\frac{d_{i+1}-d_i}2\\[6pt]
      \displaystyle
      -\left(\mathcal L_{i+1, i}+\frac{w'_{\mathrm Aq+ip_z}-w'_{\mathrm Aq+(i-1)p_z}}2\right)\sin\phi
      -p_z\cos\phi
    \end{array}
    \right]\ .
\end{align*}
C面に対しては、これの各々の内径$w_\mathrm A'$の符号を入れ換えたものとなる。
%%%%%%%%%%%%%%%%%%%%%%%%%%%%%%%%%%%%%%%%%%%%%%%%%%%%%%%%%%
%% hosoku %%%%%%%%%%%%%%%%%%%%%%%%%%%%%%%%%%%%%%%%%%%%%%%%
%%%%%%%%%%%%%%%%%%%%%%%%%%%%%%%%%%%%%%%%%%%%%%%%%%%%%%%%%%
\begin{hosoku}
$X$成分の差分の大きさが($\mathcal L_{i+1, i}$からみて)A面(の$\cos\phi$成分)のそれと同じであることがわかる。
これは(水平方向の)内径を$w_{\mathrm A\lambda}$等で代用したからであり、実際の長さは異なる(振分中心を除いて対称ではなく、C側のほうが長い)ことに注意。
\end{hosoku}
%%%%%%%%%%%%%%%%%%%%%%%%%%%%%%%%%%%%%%%%%%%%%%%%%%%%%%%%%%
%%%%%%%%%%%%%%%%%%%%%%%%%%%%%%%%%%%%%%%%%%%%%%%%%%%%%%%%%%
%%%%%%%%%%%%%%%%%%%%%%%%%%%%%%%%%%%%%%%%%%%%%%%%%%%%%%%%%%
%%%%%%%%%%%%%%%%%%%%%%%%%%%%%%%%%%%%%%%%%%%%%%%%%%%%%%%%%%
%% hosoku %%%%%%%%%%%%%%%%%%%%%%%%%%%%%%%%%%%%%%%%%%%%%%%%
%%%%%%%%%%%%%%%%%%%%%%%%%%%%%%%%%%%%%%%%%%%%%%%%%%%%%%%%%%
\begin{hosoku}[label=hosoku:generallyDimpleN]
\pageautoref{fn:generallyDimpleN}でも述べたように、たいていの場合は$|d_{i+1}-d_i|=p_x$であり、また$d_{i+2} = d_i$である。
\end{hosoku}
%%%%%%%%%%%%%%%%%%%%%%%%%%%%%%%%%%%%%%%%%%%%%%%%%%%%%%%%%%
%%%%%%%%%%%%%%%%%%%%%%%%%%%%%%%%%%%%%%%%%%%%%%%%%%%%%%%%%%
%%%%%%%%%%%%%%%%%%%%%%%%%%%%%%%%%%%%%%%%%%%%%%%%%%%%%%%%%%


%%%%%%%%%%%%%%%%%%%%%%%%%%%%%%%%%%%%%%%%%%%%%%%%%%%%%%%%%%
%% subsection 5.4.2 %%%%%%%%%%%%%%%%%%%%%%%%%%%%%%%%%%%%%%
%%%%%%%%%%%%%%%%%%%%%%%%%%%%%%%%%%%%%%%%%%%%%%%%%%%%%%%%%%
\subsection{傾き後の\dimple(B, D面側)}
傾けた後のBおよびD面側に対する$i$列目$j$番目の\dimple の位置は、A面側のときと同様に、
\begin{alignat*}{3}
  \text{B面:}&~~&
  \left[
    \begin{array}{c}
      \mathcal D_{xij,\mathrm B}'\\
      \mathcal D_{yij,\mathrm B}'\\
      \mathcal D_{zij,\mathrm B}'
    \end{array}
  \right]
 &= \left[
    \begin{array}{c}
      \mathcal D_{xij,\mathrm B}\cos\phi+\mathcal D_{zi}\sin\phi\\
      \mathcal D_{yi,\mathrm B}\\
      -\mathcal D_{xij,\mathrm B}\sin\phi+\mathcal D_{zi}\cos\phi
    \end{array}
    \right],\\[2pt]
  \text{D面:}&~~&
  \left[
    \begin{array}{c}
      \mathcal D_{xij,\mathrm D}'\\
      \mathcal D_{yij,\mathrm D}'\\
      \mathcal D_{zij,\mathrm D}'
    \end{array}
  \right]
 &= \left[
    \begin{array}{c}
      \mathcal D_{xij,\mathrm B}\cos\phi+\mathcal D_{zi}\sin\phi\\
      \mathcal D_{yi,\mathrm D}\\
      -\mathcal D_{xij,\mathrm B}\sin\phi+\mathcal D_{zi}\cos\phi
    \end{array}
    \right].
\end{alignat*}
特に、各列の中央(各列の湾曲中心)$[g_{tx}+\mathcal L_i, g_{ty}, f_t'-q-(i-1)p_z]$を原点としてみた場合の位置は、
\begin{align*}
  \left[
  \begin{array}{c}
    \displaystyle \left\{\frac{d_i}2-(j-1)p_z\right\}\cos\phi\\
    \displaystyle \pm\frac{w_{Bq+(i-1)p_z}'}2\\
    \displaystyle -\left\{\frac{d_i}2-(j-1)p_z\right\}\sin\phi
  \end{array}
  \right]\qquad
  %%%%%%%%
  \left(
  \text{複号}
  \left\{
  \begin{array}{rl}
    \!+\!\!\!\!& \text{: B面}\\
    \!-\!\!\!\!& \text{: D面}\\
  \end{array}
  \right.
  \right).
\end{align*}

\paragraph*{$j$方向の差分}\noindent
$Y$方向の隣同士の差分、すなわち$i$を固定したときの$j$番目と$j+1$番目の位置の差分は、
\begin{align*}
  \left[
  \begin{array}{c}
    \left(\mathcal D_{xi(j+1),\mathrm B}-\mathcal D_{xij,\mathrm B}\right)\cos\phi\\
    0\\
    -\left(\mathcal D_{xi(j+1),\mathrm B}-\mathcal D_{xij,\mathrm B}\right)\sin\phi
  \end{array}
  \right]
  = \left[
    \begin{array}{c}
      -p_x\cos\phi\\[6pt]
      0\\
      p_x\sin\phi
    \end{array}
    \right]\ .
\end{align*}

\paragraph*{$i$方向の差分}\noindent
B面に対する$Z$方向の隣同士の差分、すなわち$j$を固定したときの$i$番目と$i+1$番目の位置の差分については、
\begin{align*}
 &\left[
  \begin{array}{c}
    \left(\mathcal D_{x(i+1)j,\mathrm B}-\mathcal D_{xij,\mathrm B}\right)\cos\phi
    +\left(\mathcal D_{z(i+1)}-\mathcal D_{zi}\right)\sin\phi\\[3pt]
    \mathcal D_{yi+1,\mathrm B}-\mathcal D_{yi,\mathrm B}\\[3pt]
    -\left(\mathcal D_{x(i+1)j,\mathrm B}-\mathcal D_{xij,\mathrm B}\right)\sin\phi
    +\left(\mathcal D_{z(i+1)}-\mathcal D_{zi}\right)\cos\phi
  \end{array}
  \right]\\
 &= \left[
    \begin{array}{c}
      \displaystyle\left(\mathcal L_{i+1, i}+\frac{d_{i+1}-d_i}2\right)\cos\phi-p_z\sin\phi\\[10pt]
      \displaystyle\frac{w'_{\mathrm Bq+ip_z}-w'_{\mathrm Bq+(i-1)p_z}}2\\[8pt]
      \displaystyle-\left(\mathcal L_{i+1, i}+\frac{d_{i+1}-d_i}2\right)\cos\phi-p_z\cos\phi
    \end{array}
    \right]\ .
\end{align*}
D面に対しては、これの各々の内径$w_\mathrm B'$の符号(この場合$Y$成分の符号)を入れ換えたものとなる。
