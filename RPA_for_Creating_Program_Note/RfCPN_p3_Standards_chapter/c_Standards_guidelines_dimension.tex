%!TEX root = ../RPA_for_Creating_Program_Note.tex

ここではプログラムを記述する際や、\index{ずめん@図面}図面・3Dモデルの描画をする際に必要となる、\index{すんぽう@寸法}寸法や\index{こうさ@公差}公差等の取り扱いについて触れる。

なお、以降で述べる水平方向とは、端面のAC方向のことを指す。



%%%%%%%%%%%%%%%%%%%%%%%%%%%%%%%%%%%%%%%%%%%%%%%%%%%%%%%%%%
%% section 13.1 %%%%%%%%%%%%%%%%%%%%%%%%%%%%%%%%%%%%%%%%%%
%%%%%%%%%%%%%%%%%%%%%%%%%%%%%%%%%%%%%%%%%%%%%%%%%%%%%%%%%%
\modHeadsection{基本事項}


%%%%%%%%%%%%%%%%%%%%%%%%%%%%%%%%%%%%%%%%%%%%%%%%%%%%%%%%%%
%% subsection 13.1.1 %%%%%%%%%%%%%%%%%%%%%%%%%%%%%%%%%%%%%
%%%%%%%%%%%%%%%%%%%%%%%%%%%%%%%%%%%%%%%%%%%%%%%%%%%%%%%%%%
\subsection{寸法公差の取扱い}
全般的に、\index{すんぽうこうさ@寸法公差}寸法公差がある場合、\index{+こうさ@$+$公差}$+$公差と\index{-こうさ@$-$公差}$-$公差の中央(平均)を見るものとする。
ただし、\index{ないめんテーパひょう@内面テーパ表}内面テーパ表を見る際は、この限りではない。

たとえば、$100^{+0.5}_{\phantom -0}$であれば、100.25とみなす。


%%%%%%%%%%%%%%%%%%%%%%%%%%%%%%%%%%%%%%%%%%%%%%%%%%%%%%%%%%
%% subsection 13.1.2 %%%%%%%%%%%%%%%%%%%%%%%%%%%%%%%%%%%%%
%%%%%%%%%%%%%%%%%%%%%%%%%%%%%%%%%%%%%%%%%%%%%%%%%%%%%%%%%%
\subsection{寸法の優先度}
公差のある寸法と公差のない寸法(\index{かっこすんぽう@括弧寸法}括弧寸法含む)とが共存して記載されている場合、公差のある寸法を優先する。

たとえば、2つの線の寸法がそれぞれ$12^{+0.1}_{\phantom -0}$, $4.05$と記述されていて、かつその和に相当する部分の寸法が16と記述されている場合は、16.10とみなす。

ただし、特記事項等がある場合は、それを優先するものとする。



\clearpage
%%%%%%%%%%%%%%%%%%%%%%%%%%%%%%%%%%%%%%%%%%%%%%%%%%%%%%%%%%
%% section 11.2 %%%%%%%%%%%%%%%%%%%%%%%%%%%%%%%%%%%%%%%%%%
%%%%%%%%%%%%%%%%%%%%%%%%%%%%%%%%%%%%%%%%%%%%%%%%%%%%%%%%%%
\modHeadsection{全長・振分長}


%%%%%%%%%%%%%%%%%%%%%%%%%%%%%%%%%%%%%%%%%%%%%%%%%%%%%%%%%%
%% subsection 11.2.1 %%%%%%%%%%%%%%%%%%%%%%%%%%%%%%%%%%%%%
%%%%%%%%%%%%%%%%%%%%%%%%%%%%%%%%%%%%%%%%%%%%%%%%%%%%%%%%%%
\subsection{全長と振分長の公差の関係}
\index{ふりわけちょう@振分長}振分長の\index{こうさ@公差}公差については、\index{ぜんちょう@全長}全長の公差を\index{トップふりわけちょう@トップ振分長}トップ振分長と\index{ボトムふりわけちょう@ボトム振分長}ボトム振分長とで等分配する。

たとえば、全長が$1000^{\phantom +0}_{-1.0}$でトップ振分長が200であれば、全長の公差分$-0.5$を等分配し、それぞれ$-0.25$, $-0.25$とする。
つまり、トップ振分長は199.75, ボトム振分長は799.75とする
%% footnote %%%%%%%%%%%%%%%%%%%%%
\footnote{\index{ふりわけちゅうしん@振分中心}振分中心からのずれとして考えると、振分長に依らず等分配するのが自然、と捉えることができる。}。
%%%%%%%%%%%%%%%%%%%%%%%%%%%%%%%%%


%%%%%%%%%%%%%%%%%%%%%%%%%%%%%%%%%%%%%%%%%%%%%%%%%%%%%%%%%%
%% subsection 11.2.2 %%%%%%%%%%%%%%%%%%%%%%%%%%%%%%%%%%%%%
%%%%%%%%%%%%%%%%%%%%%%%%%%%%%%%%%%%%%%%%%%%%%%%%%%%%%%%%%%
\subsection{振分長が括弧寸法の場合}
片方の振分長が\index{かっこすんぽう@括弧寸法}括弧寸法の場合は、全長の公差をそのまま括弧寸法に割り当てる。

たとえば、全長が$1000^{\phantom +0}_{-1.0}$でトップ振分長が200, ボトム振分長が(800)であれば、トップ振分長は200, ボトム振分長は799.5とする。


%%%%%%%%%%%%%%%%%%%%%%%%%%%%%%%%%%%%%%%%%%%%%%%%%%%%%%%%%%
%% subsection 11.2.3 %%%%%%%%%%%%%%%%%%%%%%%%%%%%%%%%%%%%%
%%%%%%%%%%%%%%%%%%%%%%%%%%%%%%%%%%%%%%%%%%%%%%%%%%%%%%%%%%
\subsection{振分の調整}
\index{ふりわけちょう@振分長}振分長の調整を行う場合は、\index{スペーサ}スペーサまたは\index{テーブルかいてん(ふりわけちょうせい)@テーブル回転(振分調整)}テーブル回転のどちらかを用いて行うものとする。

%%%%%%%%%%%%%%%%%%%%%%%%%%%%%%%%%%%%%%%%%%%%%%%%%%%%%%%%%%
%% subsubsection 01.1.2.3 %%%%%%%%%%%%%%%%%%%%%%%%%%%%%%%%
%%%%%%%%%%%%%%%%%%%%%%%%%%%%%%%%%%%%%%%%%%%%%%%%%%%%%%%%%%
\subsubsection{スペーサによる調整}
スペーサは原則としてジグのトップ側の受板に設置する。
厚さ$\delta_s$の\index{スペーサ}スペーサによる調整を行う場合、もともとのトップ振分長$f_\mathrm T$に対し、トップ側の\index{ふりわけちょう@振分長}振分長(\index{さいふりわけちょう@再振分長}再振分長)$f'_\mathrm T$を以下のように調整する。
\begin{align*}
  f'_\mathrm T
  = f_\mathrm T
    +\sqrt{R_\mathrm i'^2-\frac{\delta_s^2+(2\bar l)^2}4}\frac{\delta_s}{\sqrt{\delta_s^2+(2\bar l)^2}}\qquad
    \left(R_\mathrm i' = R_\mathrm c-\frac{W_x}2-\rho~,~~\bar l = l-\frac\sigma2\right).
\end{align*}
$R_\textrm c$, $W_x$, $l$, $\rho$, $\sigma$はそれぞれ中心湾曲半径, AC方向の外径, ジグ幅の半分, 受板の半径, 受板の幅を示す。

%%%%%%%%%%%%%%%%%%%%%%%%%%%%%%%%%%%%%%%%%%%%%%%%%%%%%%%%%%
%% subsubsection 01.1.2.3 %%%%%%%%%%%%%%%%%%%%%%%%%%%%%%%%
%%%%%%%%%%%%%%%%%%%%%%%%%%%%%%%%%%%%%%%%%%%%%%%%%%%%%%%%%%
\subsubsection{テーブル回転による調整}
角度$\theta$だけテーブル回転をして調整を行う場合は、もともとのトップ振分長$f_\mathrm T$に対し、トップ側の\index{ふりわけちょう@振分長}振分長(\index{さいふりわけちょう@再振分長}再振分長)$f'_\mathrm T$を以下のように調整する。
\begin{align*}
  f_\mathrm T'
  = f_\mathrm T+\left(\varDelta+\sqrt{R_\mathrm i'-\bar l^2}\right)\sin\theta\qquad
    \left(R_\mathrm i' = R_\mathrm c-\frac{W_x}2-\rho~,~~\bar l = l-\frac\sigma2\right).
\end{align*}
$R_\textrm c$, $W_x$, $l$, $\rho$, $\sigma$, $\varDelta$はそれぞれ中心湾曲半径, AC方向の外径, ジグ幅の半分, 受板の半径, 受板の幅, 受板中心とテーブル中心との水平距離を示す。


\clearpage
%%%%%%%%%%%%%%%%%%%%%%%%%%%%%%%%%%%%%%%%%%%%%%%%%%%%%%%%%%
%% section 13.3 %%%%%%%%%%%%%%%%%%%%%%%%%%%%%%%%%%%%%%%%%%
%%%%%%%%%%%%%%%%%%%%%%%%%%%%%%%%%%%%%%%%%%%%%%%%%%%%%%%%%%
\modHeadsection{外径}
\index{ちゅうしんわんきょく@中心湾曲}中心湾曲を$R_\mathrm c$, トップ振分長を$f_\mathrm T$, 外径を$W_x$とすると、トップ端面部の水平方向の長さ$W_\mathrm T$は以下で与えられる。(ボトム端面部も同様)
\begin{align*}
  W_\mathrm T
  = \sqrt{\left(R_\mathrm c+\frac{W_x}2\right)^{\!2}-f_\mathrm T^2}
    -\sqrt{\left(R_\mathrm c-\frac{W_x}2\right)^{\!2}-f_\mathrm T^2}\ .
\end{align*}
なお、$(\nicefrac{f_\mathrm T}{R_\mathrm c})^2$が十分小さい場合は、$W_\mathrm T$は
\begin{align*}
  W_\mathrm T
  = W_x\!\left(1+\frac{f_\mathrm T^2}{2R^2}\right)
\end{align*}
とみなしてもよいものとする。


%%%%%%%%%%%%%%%%%%%%%%%%%%%%%%%%%%%%%%%%%%%%%%%%%%%%%%%%%%
%% section 04.4 %%%%%%%%%%%%%%%%%%%%%%%%%%%%%%%%%%%%%%%%%%
%%%%%%%%%%%%%%%%%%%%%%%%%%%%%%%%%%%%%%%%%%%%%%%%%%%%%%%%%%
\modHeadsection{内径}

%%%%%%%%%%%%%%%%%%%%%%%%%%%%%%%%%%%%%%%%%%%%%%%%%%%%%%%%%%
%% subsection 04.4.1 %%%%%%%%%%%%%%%%%%%%%%%%%%%%%%%%%%%%%
%%%%%%%%%%%%%%%%%%%%%%%%%%%%%%%%%%%%%%%%%%%%%%%%%%%%%%%%%%
\subsection{内面テーパ表の公差}
\index{ないめんテーパひょう@内面テーパ表}内面テーパ表を参照する際は、\index{ぜんちょう@全長}全長の\index{こうさ@公差}公差は考慮しないものとする。
また、トップ端からの距離のピッチも、同様に公差は考慮しないものとする。

たとえば、全長が$800^{+0.5}_{\phantom -0}$, トップ振分長が400, ピッチが25である場合を考える。
このとき、トップ端は振分中心から400の位置にあり、ピッチは25であるものとし、両端についてはそれを適宜延長して調整する。

%%%%%%%%%%%%%%%%%%%%%%%%%%%%%%%%%%%%%%%%%%%%%%%%%%%%%%%%%%
%% subsection 04.4.1 %%%%%%%%%%%%%%%%%%%%%%%%%%%%%%%%%%%%%
%%%%%%%%%%%%%%%%%%%%%%%%%%%%%%%%%%%%%%%%%%%%%%%%%%%%%%%%%%
\subsection{内面テーパ表にない内径}
内面テーパ表におけるトップ端からの距離$\lambda_i$ ($i = 0, 1, 2, ...$), それに対するAC側内径$w_{\mathrm Ai}$に対し、トップ端から$\lambda$の位置にある内径$w_{\mathrm A\lambda}$は、
\begin{align*}
  w_{\mathrm A\lambda}
  = \frac{(\lambda-\lambda_i)w_{\mathrm Ai+1}+(\lambda_{i+1}-\lambda)w_{\mathrm Ai}}{\lambda_{i+1}-\lambda_i}
  \qquad
  \Big(\lambda_i \leqq \lambda < \lambda_{i+1}\Big)
\end{align*}
とみなしてもよいものとする。
BD側内径$w_{\mathrm B\lambda}$についても同様である。

%%%%%%%%%%%%%%%%%%%%%%%%%%%%%%%%%%%%%%%%%%%%%%%%%%%%%%%%%%
%% subsection 11.4.3 %%%%%%%%%%%%%%%%%%%%%%%%%%%%%%%%%%%%%
%%%%%%%%%%%%%%%%%%%%%%%%%%%%%%%%%%%%%%%%%%%%%%%%%%%%%%%%%%
\subsection{水平方向の内径}
トップ端から$\lambda$の位置における内径を$w_\lambda$は、中心湾曲線上のトップ端から$\lambda$の位置における水平方向の内径とみなしてよいものとする。

%%%%%%%%%%%%%%%%%%%%%%%%%%%%%%%%%%%%%%%%%%%%%%%%%%%%%%%%%%
%% subsection 11.4.4 %%%%%%%%%%%%%%%%%%%%%%%%%%%%%%%%%%%%%
%%%%%%%%%%%%%%%%%%%%%%%%%%%%%%%%%%%%%%%%%%%%%%%%%%%%%%%%%%
\subsection{めっき膜厚の考慮\TBW}
(to be written ...)


\clearpage
%%%%%%%%%%%%%%%%%%%%%%%%%%%%%%%%%%%%%%%%%%%%%%%%%%%%%%%%%%
%% section 13.5 %%%%%%%%%%%%%%%%%%%%%%%%%%%%%%%%%%%%%%%%%%
%%%%%%%%%%%%%%%%%%%%%%%%%%%%%%%%%%%%%%%%%%%%%%%%%%%%%%%%%%
\modHeadsection{外削}

%%%%%%%%%%%%%%%%%%%%%%%%%%%%%%%%%%%%%%%%%%%%%%%%%%%%%%%%%%
%% subsection 11.5.1 %%%%%%%%%%%%%%%%%%%%%%%%%%%%%%%%%%%%%
%%%%%%%%%%%%%%%%%%%%%%%%%%%%%%%%%%%%%%%%%%%%%%%%%%%%%%%%%%
\subsection{外削長}
外削長の寸法は、端面に垂直な方向の値とする。
またトップ側の外削長については、溝幅の寸法も含むものとする。
このとき、外削長$h_\mathrm T$が、溝位置$\kappa_p$と溝幅$\kappa_w$の和に等しい場合は、
\begin{align*}
  h_\mathrm T = \kappa_p+1[\text{mm}]
\end{align*}
とみなして加工を行うものとする。

%%%%%%%%%%%%%%%%%%%%%%%%%%%%%%%%%%%%%%%%%%%%%%%%%%%%%%%%%%
%% subsection 11.5.2 %%%%%%%%%%%%%%%%%%%%%%%%%%%%%%%%%%%%%
%%%%%%%%%%%%%%%%%%%%%%%%%%%%%%%%%%%%%%%%%%%%%%%%%%%%%%%%%%
\subsection{湾曲に沿った外削\TBW}
(to be written ...)


%%%%%%%%%%%%%%%%%%%%%%%%%%%%%%%%%%%%%%%%%%%%%%%%%%%%%%%%%%
%% section 11.6 %%%%%%%%%%%%%%%%%%%%%%%%%%%%%%%%%%%%%%%%%%
%%%%%%%%%%%%%%%%%%%%%%%%%%%%%%%%%%%%%%%%%%%%%%%%%%%%%%%%%%
\modHeadsection{溝}

%%%%%%%%%%%%%%%%%%%%%%%%%%%%%%%%%%%%%%%%%%%%%%%%%%%%%%%%%%
%% subsection 11.6.2 %%%%%%%%%%%%%%%%%%%%%%%%%%%%%%%%%%%%%
%%%%%%%%%%%%%%%%%%%%%%%%%%%%%%%%%%%%%%%%%%%%%%%%%%%%%%%%%%
\subsection{溝位置および溝幅}
トップ端から垂直方向に、溝のトップ側の端までの距離を溝位置とする。
また、同様の方向に、溝のトップ側の端から溝のボトム側の端までの距離を溝幅とする。


%%%%%%%%%%%%%%%%%%%%%%%%%%%%%%%%%%%%%%%%%%%%%%%%%%%%%%%%%%
%% subsection 11.6.2 %%%%%%%%%%%%%%%%%%%%%%%%%%%%%%%%%%%%%
%%%%%%%%%%%%%%%%%%%%%%%%%%%%%%%%%%%%%%%%%%%%%%%%%%%%%%%%%%
\subsection{溝深さ}
トップ側に外削がなく、かつ\index{Aがわみぞふかさ@A側溝深さ}A側溝深さが\index{こうさ@公差}公差のある寸法$\kappa_d'$を持つ場合、溝幅中央における溝A側面とA側外面との距離$\kappa_d$は以下のものとみなす。
\begin{align*}
  \kappa_d
  &= \frac{2\kappa_d'-\kappa_w\sin\zeta}{1+\cos^2\zeta}\cos\zeta
     +\sqrt{R_\mathrm o^2-\left(f_\mathrm T-\kappa_p-\frac{\kappa_w}2\right)^{\!2}}
     -\sqrt{R_\mathrm o^2-\left(f_\mathrm T-\kappa_p\right)^2}\ ,\\
  \tan\zeta
  &= \frac{\sqrt{R_\mathrm o^2-\left(f_\mathrm T-\kappa_p-\kappa_w\right)^2}
           -\sqrt{R_\mathrm o^2-\left(f_\mathrm T-\kappa_p\right)^2}}
          {\kappa_w}\quad
     \left(R_\mathrm o = R_\mathrm c+\frac{W_x}2\right)
\end{align*}
$R_\mathrm c$, $W_x$, $f_\mathrm T$, $\kappa_p$, $\kappa_w$はそれぞれ中心湾曲半径, 外径, トップ振分長, 溝位置, 溝幅を示す。
なお、$(\nicefrac{f_\mathrm T}{R_\mathrm c})^2$が十分小さい場合は、
\begin{align*}
  \kappa_d
  &= \kappa_d'+\frac{\kappa_w^2}{8R_\mathrm o}
\end{align*}
とみなしてもよいものとする。


%%%%%%%%%%%%%%%%%%%%%%%%%%%%%%%%%%%%%%%%%%%%%%%%%%%%%%%%%%
%% subsection 11.6.2 %%%%%%%%%%%%%%%%%%%%%%%%%%%%%%%%%%%%%
%%%%%%%%%%%%%%%%%%%%%%%%%%%%%%%%%%%%%%%%%%%%%%%%%%%%%%%%%%
\subsection{溝中央\TBW}
(to be written ...)



\clearpage
%%%%%%%%%%%%%%%%%%%%%%%%%%%%%%%%%%%%%%%%%%%%%%%%%%%%%%%%%%
%% section 11.7 %%%%%%%%%%%%%%%%%%%%%%%%%%%%%%%%%%%%%%%%%%
%%%%%%%%%%%%%%%%%%%%%%%%%%%%%%%%%%%%%%%%%%%%%%%%%%%%%%%%%%
\modHeadsection{面取}


%%%%%%%%%%%%%%%%%%%%%%%%%%%%%%%%%%%%%%%%%%%%%%%%%%%%%%%%%%
%% subsection 11.6.2 %%%%%%%%%%%%%%%%%%%%%%%%%%%%%%%%%%%%%
%%%%%%%%%%%%%%%%%%%%%%%%%%%%%%%%%%%%%%%%%%%%%%%%%%%%%%%%%%
\subsection{端面のC面取の寸法}
端面の\index{そとがわCめんとり@外側C面取}外側C面取の寸法$c_\mathrm o$ならびに\index{うちがわCめんとり@内側C面取}内側C面取の寸法$c_\mathrm i$は、端面に垂直な方向の距離とみなす。
このとき、\index{かたかく(テーパエンドミル)@片角(テーパエンドミル)}片角が$\xi_\mathrm e$の\index{テーパエンドミル}テーパエンドミルに対して、\index{Cめんとり@C面取}C面取の$XY$方向の寸法は、$c_\mathrm o\tan\xi_\mathrm e$および$c_\mathrm i\tan\xi_\mathrm e$で与えられる。


%%%%%%%%%%%%%%%%%%%%%%%%%%%%%%%%%%%%%%%%%%%%%%%%%%%%%%%%%%
%% subsection 11.6.2 %%%%%%%%%%%%%%%%%%%%%%%%%%%%%%%%%%%%%
%%%%%%%%%%%%%%%%%%%%%%%%%%%%%%%%%%%%%%%%%%%%%%%%%%%%%%%%%%
\subsection{工具径補正}
\index{せんたんけい(テーパエンドミル)@先端径(テーパエンドミル)}先端径(直径)および先端の\index{かたかく(テーパエンドミル)@片角(テーパエンドミル)}片角がそれぞれ$D_\mathrm e$, $\xi_\mathrm e$の\index{テーパエンドミル}テーパエンドミルに対し、先端部分から軸方向に一定の距離$d_\mathrm e$を定める。
このとき、該当する工具の\index{こうぐけいほせいち@工具径補正値}工具径補正値を$\nicefrac{D_\mathrm e}2$として設定し、\index{こうぐけいまもうりょう@工具径摩耗量}工具径摩耗量を$d_\mathrm e\tan\xi_\mathrm e$として設定を行うものとする。


%%%%%%%%%%%%%%%%%%%%%%%%%%%%%%%%%%%%%%%%%%%%%%%%%%%%%%%%%%
%% subsection 04.4.2 %%%%%%%%%%%%%%%%%%%%%%%%%%%%%%%%%%%%%
%%%%%%%%%%%%%%%%%%%%%%%%%%%%%%%%%%%%%%%%%%%%%%%%%%%%%%%%%%
\subsection{端面の外側C面取}
外削のない場合は、加工径の中心座標$X$をトップ側・ボトム側のそれぞれに対して以下だけ補正する。
\begin{align*}
  \text{トップ側:}&~~
  \sqrt{R_\mathrm c^2-\left(f_\mathrm T-c_\mathrm o\right)^2}-\sqrt{R_\mathrm c^2-f_\mathrm T^2}\ ,\\
  \text{ボトム側:}&~~
  \sqrt{R_\mathrm c^2-f_\mathrm B^2}-\sqrt{R_\mathrm c^2-\left(f_\mathrm B-c_\mathrm o\right)^2}\ .
\end{align*}
ここで$c_\mathrm o$, $R_\mathrm c$, $f_\mathrm T$, $f_\mathrm B$はそれぞれ外側C面取の大きさ, 中心湾曲半径, トップ振分長, ボトム振分長を示す。


%%%%%%%%%%%%%%%%%%%%%%%%%%%%%%%%%%%%%%%%%%%%%%%%%%%%%%%%%%
%% subsection 04.4.2 %%%%%%%%%%%%%%%%%%%%%%%%%%%%%%%%%%%%%
%%%%%%%%%%%%%%%%%%%%%%%%%%%%%%%%%%%%%%%%%%%%%%%%%%%%%%%%%%
\subsection{端面の内側C面取}
加工径の中心座標$X$をトップ側・ボトム側のそれぞれに対して以下だけ補正する。
\begin{align*}
  \text{トップ側:}&~~
  \sqrt{R_\mathrm c^2-\left(f_\mathrm T-c_\mathrm i\right)^2}-\sqrt{R_\mathrm c^2-f_\mathrm T^2}\ ,\\
  \text{ボトム側:}&~~
  \sqrt{R_\mathrm c^2-f_\mathrm B^2}-\sqrt{R_\mathrm c^2-\left(f_\mathrm B-c_\mathrm i\right)^2}\ .
\end{align*}
ここで$c_\mathrm i$, $R_\mathrm c$, $f_\mathrm T$, $f_\mathrm B$はそれぞれ内側C面取の大きさ, 中心湾曲半径, トップ振分長, ボトム振分長を示す。


%%%%%%%%%%%%%%%%%%%%%%%%%%%%%%%%%%%%%%%%%%%%%%%%%%%%%%%%%%
%% subsection 04.4.2 %%%%%%%%%%%%%%%%%%%%%%%%%%%%%%%%%%%%%
%%%%%%%%%%%%%%%%%%%%%%%%%%%%%%%%%%%%%%%%%%%%%%%%%%%%%%%%%%
\subsection{外側R面取\TBW}
(to be written ...)


%%%%%%%%%%%%%%%%%%%%%%%%%%%%%%%%%%%%%%%%%%%%%%%%%%%%%%%%%%
%% subsection 04.4.2 %%%%%%%%%%%%%%%%%%%%%%%%%%%%%%%%%%%%%
%%%%%%%%%%%%%%%%%%%%%%%%%%%%%%%%%%%%%%%%%%%%%%%%%%%%%%%%%%
\subsection{内側R面取\TBW}
(to be written ...)


