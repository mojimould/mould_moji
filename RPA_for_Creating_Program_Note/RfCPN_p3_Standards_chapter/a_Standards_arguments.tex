%!TEX root = ../RPA_for_Creating_Program_Note.tex


%%%%%%%%%%%%%%%%%%%%%%%%%%%%%%%%%%%%%%%%%%%%%%%%%%%%%%%%%%
%% section F.1 %%%%%%%%%%%%%%%%%%%%%%%%%%%%%%%%%%%%%%%%%%%
%%%%%%%%%%%%%%%%%%%%%%%%%%%%%%%%%%%%%%%%%%%%%%%%%%%%%%%%%%
\modHeadsection{引数の指定}
\index{ひきすう@引数}引数の指定の仕方は2通りある。
ここではそれらを\index{ひきすうしてい@引数指定}引数指定Iおよび引数指定IIと呼ぶことにする。


%%%%%%%%%%%%%%%%%%%%%%%%%%%%%%%%%%%%%%%%%%%%%%%%%%%%%%%%%%
%% subsection 10.4.1 %%%%%%%%%%%%%%%%%%%%%%%%%%%%%%%%%%%%%
%%%%%%%%%%%%%%%%%%%%%%%%%%%%%%%%%%%%%%%%%%%%%%%%%%%%%%%%%%
\subsection{引数指定I}
\index{ひきすうしていI@引数指定I}引数指定Iでは、\index{ひきすうアドレス@引数アドレス}引数アドレスとしてA-Zのアルファベットをそれぞれ1回ずつ用いることができる。
そのため使用できる引数の数は、アルファベットの数(26個)である。
ただし、通常はG, L, N, O, Pは用いることができず、実質的に21個が使用可能な数である。


%%%%%%%%%%%%%%%%%%%%%%%%%%%%%%%%%%%%%%%%%%%%%%%%%%%%%%%%%%
%% subsection 10.4.1 %%%%%%%%%%%%%%%%%%%%%%%%%%%%%%%%%%%%%
%%%%%%%%%%%%%%%%%%%%%%%%%%%%%%%%%%%%%%%%%%%%%%%%%%%%%%%%%%
\subsection{引数指定II}
\index{ひきすうしていII@引数指定II}引数指定IIでは、引数アドレスとしてA, B, Cを1回とI, J, Kを10組まで用いることができる。
そのため使用できる引数の数は、33個である。

A, B, Cにはそれぞれ\ttNum1, \ttNum2, \ttNum3が割り当てられ、I, J, Kは入力の順序でマクロの\ttNum4から順番に\ttNum33まで割り当てられる。


\clearpage
%%%%%%%%%%%%%%%%%%%%%%%%%%%%%%%%%%%%%%%%%%%%%%%%%%%%%%%%%%
%% section F.2 %%%%%%%%%%%%%%%%%%%%%%%%%%%%%%%%%%%%%%%%%%%
%%%%%%%%%%%%%%%%%%%%%%%%%%%%%%%%%%%%%%%%%%%%%%%%%%%%%%%%%%
\modHeadsection{引数アドレスとそのローカル変数}
原則として、引数の数が22個以上必要でない限り、\DMname では引数指定Iを用いるものとする。


%%%%%%%%%%%%%%%%%%%%%%%%%%%%%%%%%%%%%%%%%%%%%%%%%%%%%%%%%%
%% subsection C.2.1 %%%%%%%%%%%%%%%%%%%%%%%%%%%%%%%%%%%%%%
%%%%%%%%%%%%%%%%%%%%%%%%%%%%%%%%%%%%%%%%%%%%%%%%%%%%%%%%%%
\subsection{引数指定I 一覧}
\index{ひきすうアドレス(ひきすうしていI)@引数アドレス(引数指定I)}引数指定Iの引数アドレスとそれに対応する\index{ローカルへんすう(ひきすう)@ローカル変数(引数)}ローカル変数は以下の通りである。\\
\noindent%
\begin{minipage}[t]{0.666\textwidth}
%%%%%%%%%%%%%%%%%%%%%%%%%%%%%%%%%%%%%%%%%%%%%%%%%%%%%%%%%%
%% captionof %%%%%%%%%%%%%%%%%%%%%%%%%%%%%%%%%%%%%%%%%%%%%
%%%%%%%%%%%%%%%%%%%%%%%%%%%%%%%%%%%%%%%%%%%%%%%%%%%%%%%%%%
\begin{twocolbreaktblr}{引数指定I}{cc|[3pt, white]|cc|[3pt, white]|cc|[3pt, white]|cc}
\cmidrule[r=0]{1-2}\cmidrule[lr=0]{3-4}\cmidrule[lr=0]{5-6}\cmidrule[l=0]{7-8}
記号 & 変数 & 記号 & 変数 & 記号 & 変数 & 記号 & 変数\\
\cmidrule[r=0]{1-2}\cmidrule[lr=0]{3-4}\cmidrule[lr=0]{5-6}\cmidrule[l=0]{7-8}
A & \ttfamily\#01 & H & \ttfamily\#11 & R & \ttfamily\#18 & X & \ttfamily\#24\\
\cmidrule[r=0]{1-2}\cmidrule[lr=0]{3-4}\cmidrule[lr=0]{5-6}\cmidrule[l=0]{7-8}
B & \ttfamily\#02 & I & \ttfamily\#04 & S & \ttfamily\#19 & Y & \ttfamily\#25\\
\cmidrule[r=0]{1-2}\cmidrule[lr=0]{3-4}\cmidrule[lr=0]{5-6}\cmidrule[l=0]{7-8}
C & \ttfamily\#03 & J & \ttfamily\#05 & T & \ttfamily\#20 & Z & \ttfamily\#26\\
\cmidrule[r=0]{1-2}\cmidrule[lr=0]{3-4}\cmidrule[lr=0]{5-6}\cmidrule[l=0]{7-8}
D & \ttfamily\#07 & K & \ttfamily\#06 & U & \ttfamily\#21\\
\cmidrule[r=0]{1-2}\cmidrule[lr=0]{3-4}\cmidrule[lr=0]{5-6}\cmidrule[l=0]{7-8}
E & \ttfamily\#08 & M & \ttfamily\#13 & V & \ttfamily\#22\\
\cmidrule[r=0]{1-2}\cmidrule[lr=0]{3-4}\cmidrule[lr=0]{5-6}\cmidrule[l=0]{7-8}
F & \ttfamily\#09 & Q & \ttfamily\#17 & W & \ttfamily\#23\\
\cmidrule[r=0]{1-2}\cmidrule[lr=0]{3-4}\cmidrule[lr=0]{5-6}\cmidrule[l=0]{7-8}
\end{twocolbreaktblr}%
\end{minipage}%
\begin{minipage}[t]{0.333\textwidth}
%%%%%%%%%%%%%%%%%%%%%%%%%%%%%%%%%%%%%%%%%%%%%%%%%%%%%%%%%%
%% captionof %%%%%%%%%%%%%%%%%%%%%%%%%%%%%%%%%%%%%%%%%%%%%
%%%%%%%%%%%%%%%%%%%%%%%%%%%%%%%%%%%%%%%%%%%%%%%%%%%%%%%%%%
\begin{multicollongtblr}{通常指定不可な引数}{cc}
記号 & 変数\\
G & \ttfamily\#10\\
L & \ttfamily\#12\\
N & \ttfamily\#14\\
O & \ttfamily\#15\\
P & \ttfamily\#16\\
\end{multicollongtblr}%
\end{minipage}


%%%%%%%%%%%%%%%%%%%%%%%%%%%%%%%%%%%%%%%%%%%%%%%%%%%%%%%%%%
%% subsection C.2.2 %%%%%%%%%%%%%%%%%%%%%%%%%%%%%%%%%%%%%%
%%%%%%%%%%%%%%%%%%%%%%%%%%%%%%%%%%%%%%%%%%%%%%%%%%%%%%%%%%
\subsection{引数指定II 一覧}
\index{ひきすうアドレス(ひきすうしていII)@引数アドレス(引数指定II)}引数指定IIの引数アドレスとそれに対応するローカル変数は以下の通りである。
なお、I, J, Kの添字は便宜上記述したものであり、実際のプログラムにはすべて同じI, J, Kの記号を用いる。\\
%%%%%%%%%%%%%%%%%%%%%%%%%%%%%%%%%%%%%%%%%%%%%%%%%%%%%%%%%%
%% captionof %%%%%%%%%%%%%%%%%%%%%%%%%%%%%%%%%%%%%%%%%%%%%
%%%%%%%%%%%%%%%%%%%%%%%%%%%%%%%%%%%%%%%%%%%%%%%%%%%%%%%%%%
\begin{twocolbreaktblr}{引数指定II}{cc|[3pt, white]|cc|[3pt, white]|cc|[3pt, white]|cc}
\cmidrule[r=0]{1-2}\cmidrule[lr=0]{3-4}\cmidrule[lr=0]{5-6}\cmidrule[l=0]{7-8}
記号 & 変数 & 記号 & 変数 & 記号 & 変数 & 記号 & 変数\\
\cmidrule[r=0]{1-2}\cmidrule[lr=0]{3-4}\cmidrule[lr=0]{5-6}\cmidrule[l=0]{7-8}
A & \ttfamily\#01 & I$_3$ & \ttfamily\#10 & I$_6$ & \ttfamily\#19 & I$_9$ & \ttfamily\#28\\
\cmidrule[r=0]{1-2}\cmidrule[lr=0]{3-4}\cmidrule[lr=0]{5-6}\cmidrule[l=0]{7-8}
B & \ttfamily\#02 & J$_3$ & \ttfamily\#11 & J$_6$ & \ttfamily\#20 & J$_9$ & \ttfamily\#29\\
\cmidrule[r=0]{1-2}\cmidrule[lr=0]{3-4}\cmidrule[lr=0]{5-6}\cmidrule[l=0]{7-8}
C & \ttfamily\#03 & K$_3$ & \ttfamily\#12 & K$_6$ & \ttfamily\#21 & K$_9$ & \ttfamily\#30\\
\cmidrule[r=0]{1-2}\cmidrule[lr=0]{3-4}\cmidrule[lr=0]{5-6}\cmidrule[l=0]{7-8}
I$_1$ & \ttfamily\#04 & I$_4$ & \ttfamily\#13 & I$_7$ & \ttfamily\#22 & I$_{10}$ & \ttfamily\#31\\
\cmidrule[r=0]{1-2}\cmidrule[lr=0]{3-4}\cmidrule[lr=0]{5-6}\cmidrule[l=0]{7-8}
J$_1$ & \ttfamily\#05 & J$_4$ & \ttfamily\#14 & J$_7$ & \ttfamily\#23 & J$_{10}$ & \ttfamily\#32\\
\cmidrule[r=0]{1-2}\cmidrule[lr=0]{3-4}\cmidrule[lr=0]{5-6}\cmidrule[l=0]{7-8}
K$_1$ & \ttfamily\#06 & K$_4$ & \ttfamily\#15 & K$_7$ & \ttfamily\#24 & K$_{10}$ & \ttfamily\#33\\
\cmidrule[r=0]{1-2}\cmidrule[lr=0]{3-4}\cmidrule[lr=0]{5-6}\cmidrule[l=0]{7-8}
I$_2$ & \ttfamily\#07 & I$_5$ & \ttfamily\#16 & I$_8$ & \ttfamily\#25\\
\cmidrule[r=0]{1-2}\cmidrule[lr=0]{3-4}\cmidrule[lr=0]{5-6}\cmidrule[l=0]{7-8}
J$_2$ & \ttfamily\#08 & J$_5$ & \ttfamily\#17 & J$_8$ & \ttfamily\#26\\
\cmidrule[r=0]{1-2}\cmidrule[lr=0]{3-4}\cmidrule[lr=0]{5-6}\cmidrule[l=0]{7-8}
K$_2$ & \ttfamily\#09 & K$_5$ & \ttfamily\#18 & K$_8$ & \ttfamily\#27\\
\cmidrule[r=0]{1-2}\cmidrule[lr=0]{3-4}\cmidrule[lr=0]{5-6}\cmidrule[l=0]{7-8}
\end{twocolbreaktblr}%

