%!TEX root = ../RPA_for_Creating_Program_Note.tex


%%%%%%%%%%%%%%%%%%%%%%%%%%%%%%%%%%%%%%%%%%%%%%%%%%%%%%%%%%
%% section F.1 %%%%%%%%%%%%%%%%%%%%%%%%%%%%%%%%%%%%%%%%%%%
%%%%%%%%%%%%%%%%%%%%%%%%%%%%%%%%%%%%%%%%%%%%%%%%%%%%%%%%%%
\modHeadsection{引数の指定}
\index{ひきすう@引数}引数の指定の仕方は2通りある。
ここではそれらを\index{ひきすうしてい@引数指定}引数指定Iおよび引数指定IIと呼ぶことにする。


%%%%%%%%%%%%%%%%%%%%%%%%%%%%%%%%%%%%%%%%%%%%%%%%%%%%%%%%%%
%% subsection 10.4.1 %%%%%%%%%%%%%%%%%%%%%%%%%%%%%%%%%%%%%
%%%%%%%%%%%%%%%%%%%%%%%%%%%%%%%%%%%%%%%%%%%%%%%%%%%%%%%%%%
\subsection{引数指定I}
\index{ひきすうしていI@引数指定I}引数指定Iでは、引数アドレスとしてA-Zのアルファベットをそれぞれ1回ずつ用いることができる。
そのため使用できる引数の数は、アルファベットの数(26個)である。
ただし、通常はG, L, N, O, Pは用いることができず、実質的に21個が使用可能な数である。


%%%%%%%%%%%%%%%%%%%%%%%%%%%%%%%%%%%%%%%%%%%%%%%%%%%%%%%%%%
%% subsection 10.4.1 %%%%%%%%%%%%%%%%%%%%%%%%%%%%%%%%%%%%%
%%%%%%%%%%%%%%%%%%%%%%%%%%%%%%%%%%%%%%%%%%%%%%%%%%%%%%%%%%
\subsection{引数指定II}
\index{ひきすうしていII@引数指定II}引数指定IIでは、引数アドレスとしてA, B, Cを1回とI, J, Kを10組まで用いることができる。
そのため使用できる引数の数は、33個である。

A, B, Cにはそれぞれ\ttNum1, \ttNum2, \ttNum3が割り当てられ、I, J, Kは入力の順序でマクロの\ttNum4から順番に\ttNum33まで割り当てられる。


\clearpage
%%%%%%%%%%%%%%%%%%%%%%%%%%%%%%%%%%%%%%%%%%%%%%%%%%%%%%%%%%
%% section F.2 %%%%%%%%%%%%%%%%%%%%%%%%%%%%%%%%%%%%%%%%%%%
%%%%%%%%%%%%%%%%%%%%%%%%%%%%%%%%%%%%%%%%%%%%%%%%%%%%%%%%%%
\modHeadsection{引数アドレスとローカル変数}
\addtocontents{lot}{\protect\addvspace{3pt}}{}{}
\addcontentsline{lot}{section}{\numberline{\thesection}\Sectionname}
\DMname では基本的に引数指定Iを用いるものとする。
引数指定Iのほうが直感的でわかりやすく、また変数の数も21個あれば十分賄えるからである。\\

%%%%%%%%%%%%%%%%%%%%%%%%%%%%%%%%%%%%%%%%%%%%%%%%%%%%%%%%%%
%% captionof %%%%%%%%%%%%%%%%%%%%%%%%%%%%%%%%%%%%%%%%%%%%%
%%%%%%%%%%%%%%%%%%%%%%%%%%%%%%%%%%%%%%%%%%%%%%%%%%%%%%%%%%
\begin{2columnstable}{引数指定I\TBW}{記号}{ローカル変数\hspace*{0.2\textwidth}~}{Sc}
A & \verb|#01|\\\hline
B & \verb|#02|\\\hline
C & \verb|#03|\\\hline
D & \verb|#07|\\\hline
E & \verb|#08|\\\hline
F & \verb|#09|\\\hline
H & \verb|#11|\\\hline
I & \verb|#04|\\\hline
J & \verb|#05|\\\hline
K & \verb|#06|\\\hline
M & \verb|#13|\\\hline
Q & \verb|#17|\\\hline
R & \verb|#18|\\\hline
S & \verb|#19|\\\hline
T & \verb|#20|\\\hline
U & \verb|#21|\\\hline
V & \verb|#22|\\\hline
W & \verb|#23|\\\hline
X & \verb|#24|\\\hline
Y & \verb|#25|\\\hline
Z & \verb|#26|
\end{2columnstable}


%\clearpage
%%%%%%%%%%%%%%%%%%%%%%%%%%%%%%%%%%%%%%%%%%%%%%%%%%%%%%%%%%
%% captionof %%%%%%%%%%%%%%%%%%%%%%%%%%%%%%%%%%%%%%%%%%%%%
%%%%%%%%%%%%%%%%%%%%%%%%%%%%%%%%%%%%%%%%%%%%%%%%%%%%%%%%%%
\begin{2columnstable}{通常は使用できない引数\TBW}{記号}{ローカル変数\hspace*{0.2\textwidth}~}{Sc}
G & \verb|#10|\\\hline
L & \verb|#12|\\\hline
N & \verb|#14|\\\hline
O & \verb|#15|\\\hline
P & \verb|#16|
\end{2columnstable}


