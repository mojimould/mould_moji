%!TEX root = ../RPA_for_Creating_Program_Note.tex



ここでは\DMname の加工システムで使用している\index{コモンへんすう@コモン変数}コモン変数について述べる。


%%%%%%%%%%%%%%%%%%%%%%%%%%%%%%%%%%%%%%%%%%%%%%%%%%%%%%%%%%
%% section 11.1 %%%%%%%%%%%%%%%%%%%%%%%%%%%%%%%%%%%%%%%%%%
%%%%%%%%%%%%%%%%%%%%%%%%%%%%%%%%%%%%%%%%%%%%%%%%%%%%%%%%%%
\modHeadsection{コモン変数の範囲\TBW}
\DMname で使用可能なコモン変数は以下のとおりである。
\begin{enumerate}
\item \ttNum100\,-\ttNum199
\item \ttNum400\,-\ttNum999
\item \ttNum900000\,-\ttNum907399
\end{enumerate}



%%%%%%%%%%%%%%%%%%%%%%%%%%%%%%%%%%%%%%%%%%%%%%%%%%%%%%%%%%
%% section 18.2 %%%%%%%%%%%%%%%%%%%%%%%%%%%%%%%%%%%%%%%%%%
%%%%%%%%%%%%%%%%%%%%%%%%%%%%%%%%%%%%%%%%%%%%%%%%%%%%%%%%%%
\modHeadsection{\ttNum100\,-\ttNum199}
\addtocontents{lot}{\protect\addvspace{3pt}}{}{}
\addcontentsline{lot}{section}{\numberline{\thesection}\Sectionname}


%%%%%%%%%%%%%%%%%%%%%%%%%%%%%%%%%%%%%%%%%%%%%%%%%%%%%%%%%%
%% subsection 18.2.1 %%%%%%%%%%%%%%%%%%%%%%%%%%%%%%%%%%%%%
%%%%%%%%%%%%%%%%%%%%%%%%%%%%%%%%%%%%%%%%%%%%%%%%%%%%%%%%%%
\subsection{\ttNum100\,-\ttNum174:一時保存値}
\ttNum100\,-\ttNum174については、(機械設置時の)\index{バンドルのプログラム}バンドルのプログラムで既に使用されているものが多いため、基本的には(\index{RHS}RHSとしては)使用しないものとし、一時的なもの(\index{LHS}LHS)として扱うものとする。
\newline


%\clearpage
\noindent\ttNum100\,-\ttNum110については、主に一時的な保存に用いるものとする。\\

\begin{2commonvariables}{\ttNum100\,-\ttNum110:一時保存値}{番号}{内容}
\ttNum100 & 各工程 切削回数用 一時保存値(仕上げ前 全削り代$X$ or $Z$)\\\hline
\ttNum101 & 各工程 切削回数用 一時保存値(仕上げ前 全削り代$Y$)\\\hline
\ttNum102 & 各工程 切削回数用 一時保存値 (max[\ttNum100, \ttNum101])\\\hline
\ttNum103 & 各工程 切削回数用 一時保存値(仕上げ前 切削回数)\\\hline
\ttNum104 & 各工程 切削回数用 一時保存値(加工時 径$X$)\\\hline
\ttNum105 & 各工程 切削回数用 一時保存値(加工時 径$Y$)\\\hline
\ttNum106 & 各工程 切削回数用 一時保存値(仕上げ 切削回数)\\\hline
\rowcolor{unusingVariables}
$\cdots$ & (以下 予備)\\
\end{2commonvariables}


\clearpage
%%%%%%%%%%%%%%%%%%%%%%%%%%%%%%%%%%%%%%%%%%%%%%%%%%%%%%%%%%
%% subsection 18.2.2 %%%%%%%%%%%%%%%%%%%%%%%%%%%%%%%%%%%%%
%%%%%%%%%%%%%%%%%%%%%%%%%%%%%%%%%%%%%%%%%%%%%%%%%%%%%%%%%%
\subsection{\ttNum175\,-\ttNum199:各工程後一時停止}
\noindent\ttNum175\,-\ttNum199については、主に各工程後の確認のためのものとする。\\

\begin{3commonvariables}{\ttNum175\,-\ttNum199:各工程後一時停止}{番号}{内容}{初期値}
\rowcolor{unusingVariables}
\ttNum175 & (以下 予備) &\\\hline
\rowcolor{unusingVariables}
$\cdots$ & \qquad$\cdots$ &\\\hline
\rowcolor{unusingVariables}
\ttNum183 & (不使用) &\\\hline
\ttNum184 & 芯出し測定後 一時停止 (0:non-stop, 1: \verb|M00|) & 0\\\hline
\rowcolor{unusingVariables}
\ttNum185 & (不使用) &\\\hline
\ttNum186 & \dimple 測定後 一時停止 (0:non-stop, 1: \verb|M00|) & 0\\\hline
\ttNum187 & \dimple 加工後 一時停止 (0:non-stop, 1: \verb|M00|, 2:\verb|O900003|) & 0\\\hline
\rowcolor{unusingVariables}
\ttNum188 & (不使用) &\\\hline
\ttNum189 & トップ端面加工後 一時停止 (0:non-stop, 1: \verb|M00|, 2:\verb|O900003|) & 0\\\hline
\ttNum190 & トップ外削加工後 一時停止 (0:non-stop, 1: \verb|M00|, 2:\verb|O900003|) & 0\\\hline
\ttNum191 & トップ溝加工後 一時停止 (0:non-stop, 1: \verb|M00|, 2:\verb|O900003|) & 0\\\hline
\ttNum192 & トップ外面取加工後 一時停止 (0:non-stop, 1: \verb|M00|, 2:\verb|O900003|) & 0\\\hline
\ttNum193 & トップ内面取加工後 一時停止 (0:non-stop, 1: \verb|M00|, 2:\verb|O900003|) & 0\\\hline
\ttNum194 & トップ座ぐり加工後 一時停止 (0:non-stop, 1: \verb|M00|, 2:\verb|O900003|) & 0\\\hline
\rowcolor{unusingVariables}
\ttNum195 & (不使用) &\\\hline
\ttNum196 & ボトム端面加工後 一時停止 (0:non-stop, 1: \verb|M00|, 2:\verb|O900003|) & 0\\\hline
\ttNum197 & ボトム外削加工後 一時停止 (0:non-stop, 1: \verb|M00|, 2:\verb|O900003|) & 0\\\hline
\ttNum198 & ボトム外面取加工後 一時停止 (0:non-stop, 1: \verb|M00|, 2:\verb|O900003|) & 0\\\hline
\ttNum199 & ボトム内面取加工後 一時停止 (0:non-stop, 1: \verb|M00|, 2:\verb|O900003|) & 0
\end{3commonvariables}



\clearpage
%%%%%%%%%%%%%%%%%%%%%%%%%%%%%%%%%%%%%%%%%%%%%%%%%%%%%%%%%%
%% section 18.3 %%%%%%%%%%%%%%%%%%%%%%%%%%%%%%%%%%%%%%%%%%
%%%%%%%%%%%%%%%%%%%%%%%%%%%%%%%%%%%%%%%%%%%%%%%%%%%%%%%%%%
\modHeadsection{\ttNum400\,-\ttNum474:加工時の調整}
\addtocontents{lot}{\protect\addvspace{3pt}}{}{}
\addcontentsline{lot}{section}{\numberline{\thesection}\Sectionname}
\ttNum400\,-\ttNum474については、\index{さぎょうしゃ@作業者}作業者が入力・変更することが想定されるものとする。


%\clearpage
%%%%%%%%%%%%%%%%%%%%%%%%%%%%%%%%%%%%%%%%%%%%%%%%%%%%%%%%%%
%% subsection 18.3.1 %%%%%%%%%%%%%%%%%%%%%%%%%%%%%%%%%%%%%
%%%%%%%%%%%%%%%%%%%%%%%%%%%%%%%%%%%%%%%%%%%%%%%%%%%%%%%%%%
\subsection{\ttNum400\,-\ttNum424}

\begin{3commonvariables}{\ttNum400\,-\ttNum404:初期設定}{番号}{内容}{初期値}
\ttNum400 & トップ端面 全削り代 &\\\hline
\ttNum401 & ボトム端面 全削り代(0 or \ttNum0: \ttNum400) &\\\hline
\ttNum402 & 計測・加工 開始N番号 & 0\\\hline
\ttNum403 & 通り芯測定(0:off, 1:on) & 0\\\hline
\rowcolor{unusingVariables}
\ttNum404 & (不使用) &
\end{3commonvariables}

\begin{3commonvariables}{\ttNum405\,-\ttNum414:測定時の調整(\dimple 除く)}{番号}{内容}{初期値}
\ttNum405 & (\verb|G54|$X$)ボトム外$X$芯出し(両側・片側測定)測定位置$Z-$補正($X$自動補正) & 0\\\hline
\ttNum406 & (\verb|G54|$Y$)ボトム外$Y$芯出し(両側測定)測定位置$Z-$補正 & 0\\\hline
\ttNum407 & (\verb|G55|$X$)ボトム内$X$芯出し(両側測定)測定位置$Z-$補正($X$自動補正) & 0\\\hline
\ttNum408 & (\verb|G55|$Y$)ボトム内$Y$芯出し(両側測定)測定位置$Z-$補正 & 0\\\hline
\rowcolor{unusingVariables}
\ttNum409 & (不使用) &\\\hline
\ttNum410 & (\verb|G56|$X$)トップ外$X$芯出し(両側・片側測定)測定位置$Z-$補正($X$自動補正) & 0\\\hline
\ttNum411 & (\verb|G56|$Y$)トップ外$Y$芯出し(両側測定)測定位置$Z-$補正 & 0\\\hline
\ttNum412 & (\verb|G57|$X$)トップ内$X$芯出し(両側測定)測定位置$Z-$補正($X$自動補正) & 0\\\hline
\ttNum413 & (\verb|G57|$Y$)トップ内$Y$芯出し(両側測定)測定位置$Z-$補正 & 0\\\hline
\rowcolor{unusingVariables}
\ttNum414 & (不使用) &
\end{3commonvariables}
%%%%%%%%%%%%%%%%%%%%%%%%%%%%%%%%%%%%%%%%%%%%%%%%%%%%%%%%%%
%% hosoku %%%%%%%%%%%%%%%%%%%%%%%%%%%%%%%%%%%%%%%%%%%%%%%%
%%%%%%%%%%%%%%%%%%%%%%%%%%%%%%%%%%%%%%%%%%%%%%%%%%%%%%%%%%
\begin{hosoku}
\ttNum405-\ttNum413は、端面の$Z$位置からの$-$補正。
\end{hosoku}
%%%%%%%%%%%%%%%%%%%%%%%%%%%%%%%%%%%%%%%%%%%%%%%%%%%%%%%%%%
%%%%%%%%%%%%%%%%%%%%%%%%%%%%%%%%%%%%%%%%%%%%%%%%%%%%%%%%%%
%%%%%%%%%%%%%%%%%%%%%%%%%%%%%%%%%%%%%%%%%%%%%%%%%%%%%%%%%%


\clearpage
\begin{3commonvariables}{\ttNum415\,-\ttNum424:加工時の調整(端面・\dimple 除く)}{番号}{内容}{初期値}
\ttNum415 & トップ外削 A面肉厚$+$補正(外削中心$X-$補正) & 0\\\hline
\ttNum416 & トップ外削 仕上げ前 一時停止 (0:non-stop, 1:\verb|M00|, 2:扉前\verb|M00|) & 0\\\hline
\ttNum417 & トップ外削 仕上げ加工 追加回数 (上限3) & 0\\\hline
\rowcolor{unusingVariables}
\ttNum418 & (不使用) &\\\hline
\ttNum419 & 溝位置$+$補正 & 0\\\hline
\ttNum420 & 溝幅$+$補正 & 0\\\hline
\ttNum421 & 溝A面深さ$+$補正(溝径中心$X-$補正) & 0\\\hline
\ttNum422 & 溝幅$Z$方向中央切削(3回加工)(0:off, 3:on) & 0\\\hline
\ttNum423 & 溝 仕上げ前 一時停止 (0:non-stop, 1:\verb|M00|, 2:扉前\verb|M00|) & 0\\\hline
\ttNum424 & 溝 仕上げ加工 追加回数 (上限3) & 0
\end{3commonvariables}

\clearpage
%%%%%%%%%%%%%%%%%%%%%%%%%%%%%%%%%%%%%%%%%%%%%%%%%%%%%%%%%%
%% subsection 18.3.3 %%%%%%%%%%%%%%%%%%%%%%%%%%%%%%%%%%%%%
%%%%%%%%%%%%%%%%%%%%%%%%%%%%%%%%%%%%%%%%%%%%%%%%%%%%%%%%%%
\subsection{\ttNum425\,-\ttNum449}

\begin{3commonvariables}{\ttNum425\,-\ttNum449:加工時の調整(続き)}{番号}{内容}{初期値}
\ttNum425 & トップ外面取$X+$補正 & 0\\\hline
\ttNum426 & トップ外面取 仕上げ前 一時停止 (0:non-stop, 1:\verb|M00|, 2:扉前\verb|M00|) & 0\\\hline
\ttNum427 & トップ外面取 仕上げ加工 追加回数 (上限3) & 0\\\hline
\rowcolor{unusingVariables}
\ttNum428 & (不使用) &\\\hline
\ttNum429 & トップ内面取$X+$補正 & 0\\\hline
\ttNum430 & トップ内面取 仕上げ前 一時停止 (0:non-stop, 1:\verb|M00|, 2:扉前\verb|M00|) & 0\\\hline
\ttNum431 & トップ内面取 仕上げ加工 追加回数 (上限3) & 0\\\hline
\rowcolor{unusingVariables}
\ttNum432 & (不使用) &\\\hline
\rowcolor{unusingVariables}
\ttNum433 & (座ぐり$X+$補正用 予備) &\\\hline
\rowcolor{unusingVariables}
\ttNum434 & (座ぐり 仕上げ前 一時停止用 予備) &\\\hline
\rowcolor{unusingVariables}
\ttNum435 & (座ぐり 仕上げ加工 追加回数用 予備) &\\\hline
\rowcolor{unusingVariables}
\ttNum436 & (不使用) &\\\hline
\ttNum437 & ボトム外削 A面肉厚$+$補正(外削中心$X+$補正) & 0\\\hline
\ttNum438 & ボトム外削 仕上げ前 一時停止 (0:non-stop, 1:\verb|M00|, 2:扉前\verb|M00|) & 0\\\hline
\ttNum439 & ボトム外削 仕上げ加工 追加回数 (上限3) & 0\\\hline
\rowcolor{unusingVariables}
\ttNum440 & (不使用) &\\\hline
\ttNum441 & ボトム外面取$X+$補正 & 0\\\hline
\ttNum442 & ボトム外面取 仕上げ前 一時停止 (0:non-stop, 1:\verb|M00|, 2:扉前\verb|M00|) & 0\\\hline
\ttNum443 & ボトム外面取 仕上げ加工 追加回数 (上限3) & 0\\\hline
\rowcolor{unusingVariables}
\ttNum444 & (不使用) &\\\hline
\ttNum445 & ボトム内面取$X+$補正 & 0\\\hline
\ttNum446 & ボトム内面取 仕上げ前 一時停止 (0:non-stop, 1:\verb|M00|, 2:扉前\verb|M00|) & 0\\\hline
\ttNum447 & ボトム内面取 仕上げ加工 追加回数 (上限3) & 0\\\hline
\rowcolor{unusingVariables}
\ttNum448 & (不使用) &\\\hline
\rowcolor{unusingVariables}
\ttNum449 & (予備) &
\end{3commonvariables}


\clearpage
%%%%%%%%%%%%%%%%%%%%%%%%%%%%%%%%%%%%%%%%%%%%%%%%%%%%%%%%%%
%% subsection 18.3.2 %%%%%%%%%%%%%%%%%%%%%%%%%%%%%%%%%%%%%
%%%%%%%%%%%%%%%%%%%%%%%%%%%%%%%%%%%%%%%%%%%%%%%%%%%%%%%%%%
\subsection{\ttNum450\,-\ttNum474}
%%%%%%%%%%%%%%%%%%%%%%%%%%%%%%%%%%%%%%%%%%%%%%%%%%%%%%%%%%
%% hosoku %%%%%%%%%%%%%%%%%%%%%%%%%%%%%%%%%%%%%%%%%%%%%%%%
%%%%%%%%%%%%%%%%%%%%%%%%%%%%%%%%%%%%%%%%%%%%%%%%%%%%%%%%%%
\begin{hosoku}
\ttNum450-\ttNum453の値は2024/01/16時点のもの。
\end{hosoku}
%%%%%%%%%%%%%%%%%%%%%%%%%%%%%%%%%%%%%%%%%%%%%%%%%%%%%%%%%%
%%%%%%%%%%%%%%%%%%%%%%%%%%%%%%%%%%%%%%%%%%%%%%%%%%%%%%%%%%
%%%%%%%%%%%%%%%%%%%%%%%%%%%%%%%%%%%%%%%%%%%%%%%%%%%%%%%%%%

\begin{3commonvariables}{\ttNum450\,-\ttNum474:\dimple 深さの調整}{番号}{内容}{設定値}
\ttNum450 & 工具\verb|T31|(Tスロット)A側\dimple~深さ補正値(深さに$+$補正) & 0.06\\\hline
\ttNum451 & 工具\verb|T31|(Tスロット)C側\dimple~深さ補正値(深さに$+$補正) & 0.03\\\hline
\ttNum452 & 工具\verb|T31|(Tスロット)B側\dimple~深さ補正値(深さに$+$補正) & 0.06\\\hline
\ttNum453 & 工具\verb|T31|(Tスロット)D側\dimple~深さ補正値(深さに$+$補正) & 0.03\\\hline
\rowcolor{unusingVariables}
\ttNum454 & (不使用) &\\\hline
\ttNum455 & 工具\verb|T32|(Tスロット)A側\dimple~深さ補正値(深さに$+$補正) &\\\hline
\ttNum456 & 工具\verb|T32|(Tスロット)C側\dimple~深さ補正値(深さに$+$補正) &\\\hline
\ttNum457 & 工具\verb|T32|(Tスロット)B側\dimple~深さ補正値(深さに$+$補正) &\\\hline
\ttNum458 & 工具\verb|T32|(Tスロット)D側\dimple~深さ補正値(深さに$+$補正) &\\\hline
\rowcolor{unusingVariables}
\ttNum459 & (不使用) &\\\hline
\ttNum460 & 工具\verb|T33|(Tスロット)A側\dimple~深さ補正値(深さに$+$補正) &\\\hline
\ttNum461 & 工具\verb|T33|(Tスロット)C側\dimple~深さ補正値(深さに$+$補正) &\\\hline
\ttNum462 & 工具\verb|T33|(Tスロット)B側\dimple~深さ補正値(深さに$+$補正) &\\\hline
\ttNum463 & 工具\verb|T33|(Tスロット)D側\dimple~深さ補正値(深さに$+$補正) &\\\hline
\rowcolor{unusingVariables}
\ttNum464 & (不使用) &\\\hline
\rowcolor{unusingVariables}
$\cdots$ & (以下予備) &
\end{3commonvariables}



\clearpage
%%%%%%%%%%%%%%%%%%%%%%%%%%%%%%%%%%%%%%%%%%%%%%%%%%%%%%%%%%
%% section 18.4 %%%%%%%%%%%%%%%%%%%%%%%%%%%%%%%%%%%%%%%%%%
%%%%%%%%%%%%%%%%%%%%%%%%%%%%%%%%%%%%%%%%%%%%%%%%%%%%%%%%%%
\modHeadsection{\ttNum500\,-\ttNum599:バンドルプログラムの使用コモン変数}
\addtocontents{lot}{\protect\addvspace{3pt}}{}{}
\addcontentsline{lot}{section}{\numberline{\thesection}\Sectionname}
\ttNum500\,-\ttNum599については、\index{バンドルのプログラム}バンドルのプログラム(O910xおよびO93xx)で使用されており、作成したプログラムには用いていない。
%%%%%%%%%%%%%%%%%%%%%%%%%%%%%%%%%%%%%%%%%%%%%%%%%%%%%%%%%%
%% hosoku %%%%%%%%%%%%%%%%%%%%%%%%%%%%%%%%%%%%%%%%%%%%%%%%
%%%%%%%%%%%%%%%%%%%%%%%%%%%%%%%%%%%%%%%%%%%%%%%%%%%%%%%%%%
\begin{hosoku}
これらの値は2023/09/26時点のもの。
\end{hosoku}
%%%%%%%%%%%%%%%%%%%%%%%%%%%%%%%%%%%%%%%%%%%%%%%%%%%%%%%%%%
%%%%%%%%%%%%%%%%%%%%%%%%%%%%%%%%%%%%%%%%%%%%%%%%%%%%%%%%%%
%%%%%%%%%%%%%%%%%%%%%%%%%%%%%%%%%%%%%%%%%%%%%%%%%%%%%%%%%%

\begin{3commonvariables}{\ttNum500\,-\ttNum524:O910x, O93xx用}{番号}{内容\hspace*{0.71\textwidth}}{設定値}
\ttNum500 & 芯ずれ許容差 (O93xx) & 5.000\\\hline
\ttNum501 & タッチセンサー信号遅れ補正 (O93xx) & 0.040\\\hline
\ttNum502 & タッチセンサープローブ中心$X$補正 (O93xx) & -0.016507\\\hline
\ttNum503 & タッチセンサープローブ中心$Y$補正 (O93xx) & -0.068371\\\hline
\ttNum504 & 測定距離 (O910x) & 5.000\\\hline
\ttNum505 & プローブ表面からプログラムの加工原点($Z$0)までの距離 (O910x) & 785.529\\\hline
\ttNum506 & 工具長の変化の許容差 (O910x) & 1.000\\\hline
\ttNum507 & 工具破損検出の許容差 (O910x) & 1.000\\\hline
\ttNum508 & (不明) & \ttNum0\\\hline
\ttNum509 & Z座標系設定 (O93xx)& 441.432\\\hline
\ttNum510 & (不明) & 270.000\\\hline
\ttNum511 & インチ/ミリ切替 (O910x) & \ttNum0\\\hline
\ttNum512 & タッチセンサープローブ半径$\mathrm{mm}$値 (O93xx) & 5.000\\\hline
\ttNum513 & 移動時用の送り速さ値 (O910x) & 1000.000\\\hline
\ttNum514 & スキップ(\verb|G31|)測定時用 送り速さ値 (O910x, O93xx) & 50.000\\\hline
\ttNum515 & (不明) & \ttNum0\\\hline
\ttNum516 & センサーの位置$X$座標 (O910x) & -30.374\\\hline
\ttNum517 & センサーの位置$Y$座標 (O910x) & -913.761\\\hline
\ttNum518 & センサーの位置$Z$座標 (O910x) & -785.529\\\hline
\ttNum519 & (不明) & 6.000\\\hline
\ttNum520 & 拡張ワーク座標系 (O910x) & 1861.000\\\hline
\ttNum521 & (不明) & 0.000\\\hline
\ttNum522 & (不明) & 0.000\\\hline
\ttNum523 & アプローチ時用の送り速さ値 (O910x) & 30.000\\\hline
\ttNum524 & 測定時用の送り速さ値 (O910x) & 3.000\\\hline
\end{3commonvariables}

\clearpage
\begin{3commonvariables}{\ttNum525\,-\ttNum599:O910x, O93xx用(続き)}{番号}{内容\hspace*{0.71\textwidth}}{設定値}
$\cdots$ & (以下不明) & \ttNum0\\\hline
\ttNum533 & (不明) & 0.000\\\hline
\ttNum534 & (不明) & 0.000\\\hline
$\cdots$ & (以下不明) & \ttNum0
\end{3commonvariables}



\clearpage
%%%%%%%%%%%%%%%%%%%%%%%%%%%%%%%%%%%%%%%%%%%%%%%%%%%%%%%%%%
%% section 11.5 %%%%%%%%%%%%%%%%%%%%%%%%%%%%%%%%%%%%%%%%%%
%%%%%%%%%%%%%%%%%%%%%%%%%%%%%%%%%%%%%%%%%%%%%%%%%%%%%%%%%%
\modHeadsection{\ttNum600\,-\ttNum699}
\addtocontents{lot}{\protect\addvspace{3pt}}{}{}
\addcontentsline{lot}{section}{\numberline{\thesection}\Sectionname}

%%%%%%%%%%%%%%%%%%%%%%%%%%%%%%%%%%%%%%%%%%%%%%%%%%%%%%%%%%
%% subsection 11.5.1 %%%%%%%%%%%%%%%%%%%%%%%%%%%%%%%%%%%%%
%%%%%%%%%%%%%%%%%%%%%%%%%%%%%%%%%%%%%%%%%%%%%%%%%%%%%%%%%%
\subsection{\ttNum600\,-\ttNum624:ワークと工具間の距離の調整}
\ttNum600\,-\ttNum624については、主に製品と工具間の距離に関するものとする。\\

\begin{3commonvariables}{\ttNum600\,-\ttNum624:ワークと工具間の距離の調整}{番号}{内容}{設定値}
\rowcolor{unusingVariables}
\ttNum600 & (予備) &\\\hline
\ttNum601 & 工具 - 端面間 $Z$方向クリアランス平面間距離 & 100.0\\\hline
\rowcolor{unusingVariables}
\ttNum602 & (予備) &\\\hline
\ttNum603 & タッチセンサー計測時の近付き量 & 9.0\\\hline
\ttNum604 & タッチセンサー計測時の行過ぎ量 & 3.0\\\hline
\rowcolor{unusingVariables}
\ttNum605 & (予備) &\\\hline
\rowcolor{unusingVariables}
\ttNum606 & (予備) &\\\hline
\ttNum607 & 外側加工面 法線方向クリアランス平面間距離 最小値 & 30.0\\\hline
\ttNum608 & 内側加工面 法線方向クリアランス平面間距離 最小値 & 15.0\\\hline
\rowcolor{unusingVariables}
\ttNum609 & (予備) &\\\hline
\ttNum610 & 端面加工用 内径輪郭径$-$補正量 & 5.0\\\hline
\rowcolor{unusingVariables}
\ttNum611 & (予備) &\\\hline
\rowcolor{unusingVariables}
\ttNum612 & (外削加工用予備) & \\\hline
\rowcolor{unusingVariables}
\ttNum613 & (予備) &\\\hline
\ttNum614 & 溝加工用 BD溝加工のみ時 AC方向クリアランス平面間距離 & \verb|GE| \ttNum607*1.5\\\hline
\rowcolor{unusingVariables}
\ttNum615 & (予備) &\\\hline
\rowcolor{unusingVariables}
\ttNum616 & (外面取加工用予備) & \\\hline
\rowcolor{unusingVariables}
\ttNum617 & (予備) &\\\hline
\rowcolor{unusingVariables}
\ttNum618 & (内面取加工用予備) &\\\hline
\rowcolor{unusingVariables}
\ttNum617 & (予備) &\\\hline
\rowcolor{unusingVariables}
\ttNum618 & (座ぐり加工用予備) &\\\hline
\rowcolor{unusingVariables}
\ttNum619 & (予備) &\\\hline
\ttNum620 & \dimple 加工用 加工近付き量 & 1.0 \\\hline
\rowcolor{unusingVariables}
$\cdots$ & (以下予備) &
\end{3commonvariables}
%%%%%%%%%%%%%%%%%%%%%%%%%%%%%%%%%%%%%%%%%%%%%%%%%%%%%%%%%%
%% hosoku %%%%%%%%%%%%%%%%%%%%%%%%%%%%%%%%%%%%%%%%%%%%%%%%
%%%%%%%%%%%%%%%%%%%%%%%%%%%%%%%%%%%%%%%%%%%%%%%%%%%%%%%%%%
\begin{hosoku}
\ttNum614について、正確には$\max[\ttNum607, \text{{\small(溝工具半径)}}]\times1.5$.
\end{hosoku}
%%%%%%%%%%%%%%%%%%%%%%%%%%%%%%%%%%%%%%%%%%%%%%%%%%%%%%%%%%
%%%%%%%%%%%%%%%%%%%%%%%%%%%%%%%%%%%%%%%%%%%%%%%%%%%%%%%%%%



\clearpage
%%%%%%%%%%%%%%%%%%%%%%%%%%%%%%%%%%%%%%%%%%%%%%%%%%%%%%%%%%
%% subsection 11.5.2 %%%%%%%%%%%%%%%%%%%%%%%%%%%%%%%%%%%%%
%%%%%%%%%%%%%%%%%%%%%%%%%%%%%%%%%%%%%%%%%%%%%%%%%%%%%%%%%%
\subsection{\ttNum650\,-\ttNum674:残り代および1回あたりの削り代}
\ttNum650\,-\ttNum674については、各加工の残り代や1回あたりの削り代に関するものとする。\\

\begin{3commonvariables}{\ttNum650\,-\ttNum674:残り代および1回あたりの削り代}{番号}{内容}{設定値}
\ttNum650 & 端面加工1回あたりの$Z$方向削り代 & 4.0\\\hline
\rowcolor{unusingVariables}
\ttNum651 & (予備) & \\\hline
\ttNum652 & 外削加工1回あたりの削り代(直径) & 2.0\\\hline
\ttNum653 & 外削加工 仕上げ前 残り削り代(直径) & 1.0\\\hline
\ttNum654 & 溝加工1回あたりの削り代(溝深さ) & 5.0\\\hline
\ttNum655 & 溝加工 仕上げ前 残り削り代(直径) & 1.0\\\hline
\ttNum656 & 外面取加工1回あたりの削り代(直径) & 2.0\\\hline
\ttNum657 & 外面取加工 仕上げ前 残り削り代(直径) & 1.0\\\hline
\ttNum658 & 内面取加工1回あたりの削り代(直径) & 2.0\\\hline
\ttNum659 & 内面取加工 仕上げ前 残り削り代(直径) & 1.0\\\hline
\rowcolor{unusingVariables}
\ttNum660 & (座ぐり加工用予備) & \\\hline
\rowcolor{unusingVariables}
\ttNum661 & (座ぐり加工用予備) & \\\hline
\rowcolor{unusingVariables}
$\cdots$ & (以下予備) &
\end{3commonvariables}


\clearpage
%%%%%%%%%%%%%%%%%%%%%%%%%%%%%%%%%%%%%%%%%%%%%%%%%%%%%%%%%%
%% subsection 11.5.2 %%%%%%%%%%%%%%%%%%%%%%%%%%%%%%%%%%%%%
%%%%%%%%%%%%%%%%%%%%%%%%%%%%%%%%%%%%%%%%%%%%%%%%%%%%%%%%%%
\subsection{\ttNum675\,-\ttNum699:工具の送り速さ}
\ttNum675\,-\ttNum699については、工具の送り速さに関するものとする。\\

\begin{3commonvariables}{\ttNum675\,-\ttNum699:工具の送り速さ}{番号}{内容}{設定値}
\ttNum675 & $Z$最大送り速さ値(タッチセンサープローブ除く) & 12000\\\hline
\ttNum676 & 最大送り速さ値(タッチセンサープローブ) & 5400\\\hline
\ttNum677 & アプローチ・リトラクト送り速さ値(タッチセンサープローブ除く) & 6000\\\hline
\ttNum678 & アプローチ・リトラクト送り速さ値(タッチセンサープローブ) & 1500\\\hline
\ttNum679 & アプローチ送り速さ値(片側測定時) & 50\\\hline
\ttNum680 & アプローチ送り速さ値(両側測定時) & 200\\\hline
\rowcolor{unusingVariables}
\ttNum681 & (予備) &\\\hline
\ttNum682 & 端面加工 直線 & 900\\\hline
\ttNum683 & 端面加工 コーナー & 900\\\hline
\ttNum684 & 外削加工 直線 & 400\\\hline
\ttNum685 & 外削加工 コーナー & 400\\\hline
\ttNum686 & 溝加工 直線 & 500\\\hline
\ttNum687 & 溝加工 コーナー & 400\\\hline
\ttNum688 & 外面取加工 直線 & 500\\\hline
\ttNum689 & 外面取加工 コーナー & 400\\\hline
\ttNum690 & 内面取加工 直線 & 400\\\hline
\ttNum691 & 内面取加工 コーナー & 400\\\hline
\rowcolor{unusingVariables}
\ttNum692 & (座ぐり加工 直線用 予備) & \\\hline
\rowcolor{unusingVariables}
\ttNum693 & (座ぐり加工 コーナー用 予備) & \\\hline
\ttNum694 & \dimple 表面アプローチ & 540\\\hline
\ttNum695 & \dimple 加工  & 100\\\hline
\rowcolor{unusingVariables}
$\cdots$ & (以下予備) &
\end{3commonvariables}



\clearpage
%%%%%%%%%%%%%%%%%%%%%%%%%%%%%%%%%%%%%%%%%%%%%%%%%%%%%%%%%%
%% section 11.6 %%%%%%%%%%%%%%%%%%%%%%%%%%%%%%%%%%%%%%%%%%
%%%%%%%%%%%%%%%%%%%%%%%%%%%%%%%%%%%%%%%%%%%%%%%%%%%%%%%%%%
\modHeadsection{\ttNum700\,-\ttNum750:\dimple}
\addtocontents{lot}{\protect\addvspace{3pt}}{}{}
\addcontentsline{lot}{section}{\numberline{\thesection}\Sectionname}
\ttNum700\,-\ttNum750については、主に\expandafterindex{\dimplekana そくていようサブプログラム@\dimple 測定用サブプログラム}\dimple 用サブプログラム(O2x000x)で使用されるものとする。


%%%%%%%%%%%%%%%%%%%%%%%%%%%%%%%%%%%%%%%%%%%%%%%%%%%%%%%%%%
%% subsection 18.6.1 %%%%%%%%%%%%%%%%%%%%%%%%%%%%%%%%%%%%%
%%%%%%%%%%%%%%%%%%%%%%%%%%%%%%%%%%%%%%%%%%%%%%%%%%%%%%%%%%
\subsection{\ttNum700\,-\ttNum724:内面溝レベル1サブプログラム用}

\begin{2commonvariables}{\ttNum700\,-\ttNum724:\dimple~レベル1サブプログラム用 (\DLone)}{番号}{内容}
\rowcolor{unusingVariables}
\ttNum700 & (予備)\\\hline
\ttNum701 & プログラム読込み時のワーク座標系(\ttNum4012)\\\hline
\ttNum702 & 工具別$Z$補正(\verb|T50|:\ttNum512, \verb|T3|x:0)\\\hline
\ttNum703 & 工具別$XY$補正(\verb|T50|:\ttNum512, \verb|T3|x:\ttNum[2400+\ttNum4111]+\ttNum[2600+\ttNum4111])\\\hline
\ttNum704 & 工具別移動\verb|G#| (\verb|T50|:31, \verb|T3|x:1)\\\hline
\ttNum705 & テーブル中心からワーク座標(\ttNum701)原点までの$X$距離\\\hline
\ttNum706 & 傾き後のトップ端面中心(機械座標)$X$ (\cf\pageeqref{eq:afterPhiTCenterFromO})\\\hline
\ttNum707 & テーブル中心から傾き後のトップ端面中心までの$Z$距離 (\cf\pageeqref{eq:afterPhiTCenterFromO})\\\hline
\ttNum708 & 傾き後トップ端中心(ブロックエンド)$X$座標(\ttNum5001)\\\hline
\ttNum709 & 傾き後トップ端中心(ブロックエンド)$Z$座標(\ttNum5003)\\\hline
\ttNum710 & テーブル中心から\dimple1列目までの$Z$距離$Z-q$\\\hline
\ttNum711 & トップ端中心から\dimple1列目中心までの$X$距離(\cf\pageeqref{eq:dimpleCenterDistance})\\\hline
\ttNum712 & 傾き後\dimple1列目中心$X$移動距離(\cf\pageeqref{eq:afterPhidimpleCenterDistance})\\\hline
\ttNum713 & 傾き後\dimple1列目中心$Z$移動距離(\cf\pageeqref{eq:afterPhidimpleCenterDistance})\\\hline
\ttNum714 & 傾き後\dimple1列目中心(ブロックエンド)$X$座標 (\ttNum5001)\\\hline
\ttNum715 & 傾き後\dimple1列目中心(ブロックエンド)$Y$座標 (\ttNum5002)\\\hline
\ttNum716 & 傾き後\dimple1列目中心(ブロックエンド)$Z$座標 (\ttNum5003)\\\hline
\ttNum717\color{red}$^*$ & 各面 ループ用数値(1:A, 2:C, 3:B, 4:D)\\\hline
\ttNum718 & BD内半径$-\text{\ttNum703}-10$\\\hline
\ttNum719 & (AC内半径$-\text{\ttNum703}-10)\cos\phi$\\\hline
\rowcolor{unusingVariables}
$\cdots$ & (以下予備)
\end{2commonvariables}
%%%%%%%%%%%%%%%%%%%%%%%%%%%%%%%%%%%%%%%%%%%%%%%%%%%%%%%%%%
%% hosoku %%%%%%%%%%%%%%%%%%%%%%%%%%%%%%%%%%%%%%%%%%%%%%%%
%%%%%%%%%%%%%%%%%%%%%%%%%%%%%%%%%%%%%%%%%%%%%%%%%%%%%%%%%%
\begin{hosoku}
\ttNum717はレベル2サブプログラム(O220001, O220002)でRHSとして使用していることに注意。
\end{hosoku}
%%%%%%%%%%%%%%%%%%%%%%%%%%%%%%%%%%%%%%%%%%%%%%%%%%%%%%%%%%
%%%%%%%%%%%%%%%%%%%%%%%%%%%%%%%%%%%%%%%%%%%%%%%%%%%%%%%%%%
%%%%%%%%%%%%%%%%%%%%%%%%%%%%%%%%%%%%%%%%%%%%%%%%%%%%%%%%%%


\clearpage
%%%%%%%%%%%%%%%%%%%%%%%%%%%%%%%%%%%%%%%%%%%%%%%%%%%%%%%%%%
%% subsection 18.6.2 %%%%%%%%%%%%%%%%%%%%%%%%%%%%%%%%%%%%%
%%%%%%%%%%%%%%%%%%%%%%%%%%%%%%%%%%%%%%%%%%%%%%%%%%%%%%%%%%
\subsection{\ttNum725\,-\ttNum749:内面溝レベル2, 3サブプログラム用}

\begin{2commonvariables}{\ttNum725\,-\ttNum744:\dimple~レベル2サブプログラム用 (\DLtwoAC, \DLtwoBD)}{番号}{内容}
\ttNum725 & プログラム読込時ブロックエンド$Y$ or $X$ (\ttNum5002, \ttNum5001)\\\hline
\ttNum726 & プログラム読込時ブロックエンド$Z$ (\ttNum5003)\\\hline
\ttNum727 & \dimple~偶数列の列数\\\hline
\ttNum728 & \dimple~偶数列(一列)の\dimple~数\\\hline
\ttNum729 & \dimple~奇数列(一列)の\dimple~数\\\hline
\ttNum730 & \dimple~現在の列の\dimple~数\\\hline
\rowcolor{unusingVariables}
$\cdots$ & (以下予備)
\end{2commonvariables}


%\clearpage
\begin{2commonvariables}{\ttNum745\,-\ttNum749:\dimple~測定用 (\DMLthreeAC, \DMLthreeBD)}{番号}{内容}
\rowcolor{unusingVariables}
\ttNum745 & (以下 予備)\\\hline
\rowcolor{unusingVariables}
$\cdots$ & \qquad$\cdots$\\\hline
\ttNum748 & プログラム読込時ブロックエンド$X$ or $Y$ (\ttNum5001, \ttNum5002)\\\hline
\ttNum749\color{red}$^*$ & \dimple~表面位置$X$ or $Y$測定値
\end{2commonvariables}
%%%%%%%%%%%%%%%%%%%%%%%%%%%%%%%%%%%%%%%%%%%%%%%%%%%%%%%%%%
%% hosoku %%%%%%%%%%%%%%%%%%%%%%%%%%%%%%%%%%%%%%%%%%%%%%%%
%%%%%%%%%%%%%%%%%%%%%%%%%%%%%%%%%%%%%%%%%%%%%%%%%%%%%%%%%%
\begin{hosoku}
\ttNum749はレベル2サブプログラム(O220001, O220002)でRHSとして使用していることに注意。
\end{hosoku}
%%%%%%%%%%%%%%%%%%%%%%%%%%%%%%%%%%%%%%%%%%%%%%%%%%%%%%%%%%
%%%%%%%%%%%%%%%%%%%%%%%%%%%%%%%%%%%%%%%%%%%%%%%%%%%%%%%%%%
%%%%%%%%%%%%%%%%%%%%%%%%%%%%%%%%%%%%%%%%%%%%%%%%%%%%%%%%%%



\clearpage
%%%%%%%%%%%%%%%%%%%%%%%%%%%%%%%%%%%%%%%%%%%%%%%%%%%%%%%%%%
%% section 19.7 %%%%%%%%%%%%%%%%%%%%%%%%%%%%%%%%%%%%%%%%%%
%%%%%%%%%%%%%%%%%%%%%%%%%%%%%%%%%%%%%%%%%%%%%%%%%%%%%%%%%%
\modHeadsection{\ttNum900001\,-\ttNum900031, \ttNum900101\,-\ttNum900500:実測値\TBW}
\addtocontents{lot}{\protect\addvspace{3pt}}{}{}
\addcontentsline{lot}{section}{\numberline{\thesection}\Sectionname}
\ttNum900001\,-\ttNum900500については、\index{じっそくち@実測値}実測値または計算値を格納する。

%%%%%%%%%%%%%%%%%%%%%%%%%%%%%%%%%%%%%%%%%%%%%%%%%%%%%%%%%%
%% subsection 19.7.1 %%%%%%%%%%%%%%%%%%%%%%%%%%%%%%%%%%%%%
%%%%%%%%%%%%%%%%%%%%%%%%%%%%%%%%%%%%%%%%%%%%%%%%%%%%%%%%%%
\subsection{\ttNum900001\,-\ttNum900049:\dimple 以外}

\begin{2commonvariables}{\ttNum900001\,-\ttNum900005:外中心$X$ 両側測定用 (\MXOThickness)}{番号}{内容}
\ttNum900001 & $X$外中心測定 $-X$側測定値\\\hline
\ttNum900002 & $X$外中心測定 $+X$側測定値\\\hline
\ttNum900003 & $X$外中心測定値\\\hline
\ttNum900004 & $X$外中心測定 厚さ測定値\\\hline
\rowcolor{unusingVariables}
\ttNum900005 & (予備)\\
\end{2commonvariables}


\begin{2commonvariables}{\ttNum900006\,-\ttNum900010:外中心$Y$ 両側測定用 (\MYOThickness)}{番号}{内容}
\ttNum900006 & $Y$外中心測定 $-Y$側測定値\\\hline
\ttNum900007 & $Y$外中心測定 $+Y$側測定値\\\hline
\ttNum900008 & $Y$外中心測定値\\\hline
\ttNum900009 & $Y$外中心測定 厚さ測定値\\\hline
\rowcolor{unusingVariables}
\ttNum900010 & (予備)\\
\end{2commonvariables}


%\clearpage
\begin{2commonvariables}{\ttNum900011\,-\ttNum900013:溝中心$X$ 片側測定用 (\MXOface)}{番号}{内容}
\ttNum900011 & $X$溝中心測定 A側外面測定値\\\hline
\rowcolor{unusingVariables}
\ttNum900012 & (予備)\\\hline
\rowcolor{unusingVariables}
\ttNum900013 & (予備)\\
\end{2commonvariables}


%\clearpage
\begin{2commonvariables}{\ttNum900014\,-\ttNum900018:内中心$X$ 両側測定用 (\MXIWidth)}{番号}{内容}
\ttNum900014 & $X$内中心測定 $-X$側測定値\\\hline
\ttNum900015 & $X$内中心測定 $+X$側測定値\\\hline
\ttNum900016 & $X$内中心測定値\\\hline
\ttNum900017 & $X$内中心測定 厚さ測定値\\\hline
\rowcolor{unusingVariables}
\ttNum900018 & (予備)\\
\end{2commonvariables}


\clearpage
\begin{2commonvariables}{\ttNum900019\,-\ttNum900023:内中心$Y$ 両側測定用 (\MYIWidth)}{番号}{内容}
\ttNum900019 & $Y$内中心測定 $-Y$側測定値\\\hline
\ttNum900020 & $Y$内中心測定 $+Y$側測定値\\\hline
\ttNum900021 & $Y$内中心測定値\\\hline
\ttNum900022 & $Y$内中心測定 厚さ測定値\\\hline
\rowcolor{unusingVariables}
\ttNum900023 & (予備)\\
\end{2commonvariables}


%\clearpage
\begin{2commonvariables}{\ttNum900024\,-\ttNum900026:外削中心$X$ 片側測定用 (\MXIface)}{番号}{内容}
\ttNum900024 & $X$外削中心測定 内面測定値\\\hline
\rowcolor{unusingVariables}
\ttNum900025 & (予備)\\\hline
\rowcolor{unusingVariables}
\ttNum900026 & (予備)\\
\end{2commonvariables}


%\clearpage
\begin{2commonvariables}{\ttNum900027\,-\ttNum900030:通り芯$Y$ 片側測定用 (\MYcenterline)}{番号}{内容}
\ttNum900027 & $Y$通り芯 ボトム側測定値\\\hline
\ttNum900028 & $Y$通り芯 トップ側測定値\\\hline
\ttNum900029 & $Y$通り芯 測定値\\\hline
\rowcolor{unusingVariables}
\ttNum900030 & (予備)\\
\end{2commonvariables}


\begin{2commonvariables}{\ttNum900031\,-\ttNum900034:通り芯$X$ 片側測定用 (\MXcenterline)}{番号}{内容}
\ttNum900031 & $X$通り芯 トップ側測定値\\\hline
\ttNum900032 & $X$通り芯 ボトム側測定値\\\hline
\ttNum900033 & $X$通り芯 測定値\\\hline
\rowcolor{unusingVariables}
\ttNum900034 & (予備)\\
\end{2commonvariables}


\clearpage
%%%%%%%%%%%%%%%%%%%%%%%%%%%%%%%%%%%%%%%%%%%%%%%%%%%%%%%%%%
%% subsection 18.7.2 %%%%%%%%%%%%%%%%%%%%%%%%%%%%%%%%%%%%%
%%%%%%%%%%%%%%%%%%%%%%%%%%%%%%%%%%%%%%%%%%%%%%%%%%%%%%%%%%
\subsection{\ttNum900101\,-\ttNum900500:\dimple}

%%%%%%%%%%%%%%%%%%%%%%%%%%%%%%%%%%%%%%%%%%%%%%%%%%%%%%%%%%
%% subsubsection 18.7.2.1 %%%%%%%%%%%%%%%%%%%%%%%%%%%%%%%%
%%%%%%%%%%%%%%%%%%%%%%%%%%%%%%%%%%%%%%%%%%%%%%%%%%%%%%%%%%
\subsubsection{\ttNum900101\,-\ttNum900300:\dimple AC面}

\begin{2commonvariables}{\ttNum900101\,-\ttNum900300:\dimple AC表面位置 測定値 (\DMLthreeAC)}{番号}{内容}
\ttNum900101\,-\ttNum900200 & A側\dimple~表面位置$X$ 測定値\\\hline
\ttNum900201\,-\ttNum900300 & C側\dimple~表面位置$X$ 測定値
\end{2commonvariables}

%%%%%%%%%%%%%%%%%%%%%%%%%%%%%%%%%%%%%%%%%%%%%%%%%%%%%%%%%%
%% subsubsection 18.7.2.2 %%%%%%%%%%%%%%%%%%%%%%%%%%%%%%%%
%%%%%%%%%%%%%%%%%%%%%%%%%%%%%%%%%%%%%%%%%%%%%%%%%%%%%%%%%%
\subsubsection{\ttNum900301\,-\ttNum900500:\dimple BD面}

\begin{2commonvariables}{\ttNum900301\,-\ttNum900500:\dimple BD表面位置 測定値 (\DMLthreeBD)}{番号}{内容}
\ttNum900301\,-\ttNum900400 & B側\dimple~表面位置$Y$ 測定値\\\hline
\ttNum900401\,-\ttNum900500 & D側\dimple~表面位置$Y$ 測定値
\end{2commonvariables}



\clearpage
%%%%%%%%%%%%%%%%%%%%%%%%%%%%%%%%%%%%%%%%%%%%%%%%%%%%%%%%%%
%% section 11.8 %%%%%%%%%%%%%%%%%%%%%%%%%%%%%%%%%%%%%%%%%%
%%%%%%%%%%%%%%%%%%%%%%%%%%%%%%%%%%%%%%%%%%%%%%%%%%%%%%%%%%
\modHeadsection{\ttNum901000\,-\ttNum901024:パレット・ジグ}
\addtocontents{lot}{\protect\addvspace{3pt}}{}{}
\addcontentsline{lot}{section}{\numberline{\thesection}\Sectionname}
\ttNum901000\,-\ttNum901024については、主に\index{パレット}パレットや\index{ジグ}ジグに関するものとする。\\

\begin{3commonvariables}{\ttNum901000\,-\ttNum901024:主にパレット・ジグ}{番号}{内容}{設定値}
\rowcolor{unusingVariables}
\ttNum901000 & (予備) &\\\hline
\ttNum901001 & パレット\ttNum1 ジグ中心機械座標$X$ & -550.019\\\hline
\ttNum901002 & パレット\ttNum1 ジグ中心機械座標$Y$ & -740.0\\\hline
\ttNum901003 & パレット\ttNum1 ジグ中心機械座標$Z$ & -1149.974\\\hline
\ttNum901004 & パレット\ttNum1 ジグ中心機械座標$B$ & -0.073\\\hline
\ttNum901005 & パレット\ttNum2 ジグ中心機械座標$X$ & -550.019\\\hline
\ttNum901006 & パレット\ttNum2 ジグ中心機械座標$Y$ & -740.0\\\hline
\ttNum901007 & パレット\ttNum2 ジグ中心機械座標$Z$ & -1149.974\\\hline
\ttNum901008 & パレット\ttNum2 ジグ中心機械座標$B$ & -0.073\\\hline
\ttNum901009 & 工具中心機械座標$C$ & 0\\\hline
\rowcolor{unusingVariables}
\ttNum901010 & (予備) &\\\hline
\hline
\ttNum901011 & パレット\ttNum1 ジグ外側幅$2l$(機械座標系$B$0における$Z$方向) & 660.0\\\hline
\ttNum901012 & パレット\ttNum1 ジグ内側幅(機械座標系$B$0における$Z$方向) & 410.0\\\hline
\ttNum901013 & パレット\ttNum1 ジグ幅(機械座標系$B$0における$X$方向) & 455.0\\\hline
\rowcolor{unusingVariables}
\ttNum901014 & (予備) &\\\hline
\ttNum901015 & パレット\ttNum2 ジグ外側幅$2l$(機械座標系$B$0における$Z$方向) & 660.0\\\hline
\ttNum901016 & パレット\ttNum2 ジグ内側幅(機械座標系$B$0における$Z$方向) & 410.0\\\hline
\ttNum901017 & パレット\ttNum2 ジグ幅(機械座標系$B$0における$X$方向) & 455.0\\\hline
\rowcolor{unusingVariables}
$\cdots$ & (以下予備) &
\end{3commonvariables}
%%%%%%%%%%%%%%%%%%%%%%%%%%%%%%%%%%%%%%%%%%%%%%%%%%%%%%%%%%
%% hosoku %%%%%%%%%%%%%%%%%%%%%%%%%%%%%%%%%%%%%%%%%%%%%%%%
%%%%%%%%%%%%%%%%%%%%%%%%%%%%%%%%%%%%%%%%%%%%%%%%%%%%%%%%%%
\begin{hosoku}
\index{ジグのちゅうしん@ジグの中心}ジグ中心\index{きかいざひょう@機械座標}機械座標については2023/09/26時点のもの。
その他の(\index{ずめん@図面}図面上の)\index{すんぽう@寸法}寸法として、
\begin{enumerate}
\item テーブル中心 と C面側ジグ端 との水平距離:196.5
\item 受板の円の半径$\rho$:100
\item 受板の鉛直方向の幅$\sigma$:40
\item テーブル中心 と 受板の円の中心 との水平距離$\varDelta$:201.5
\item 受板の円の中心 と 受板の水平方向の底 との距離:70
\end{enumerate}
\end{hosoku}
%%%%%%%%%%%%%%%%%%%%%%%%%%%%%%%%%%%%%%%%%%%%%%%%%%%%%%%%%%
%%%%%%%%%%%%%%%%%%%%%%%%%%%%%%%%%%%%%%%%%%%%%%%%%%%%%%%%%%
%%%%%%%%%%%%%%%%%%%%%%%%%%%%%%%%%%%%%%%%%%%%%%%%%%%%%%%%%%



\clearpage
%%%%%%%%%%%%%%%%%%%%%%%%%%%%%%%%%%%%%%%%%%%%%%%%%%%%%%%%%%
%% section 11.9 %%%%%%%%%%%%%%%%%%%%%%%%%%%%%%%%%%%%%%%%%%
%%%%%%%%%%%%%%%%%%%%%%%%%%%%%%%%%%%%%%%%%%%%%%%%%%%%%%%%%%
\modHeadsection{\ttNum901100\,-\ttNum901149:工具}
\addtocontents{lot}{\protect\addvspace{3pt}}{}{}
\addcontentsline{lot}{section}{\numberline{\thesection}\Sectionname}
\ttNum901100\,-\ttNum901149については、\index{こうぐ@工具}工具に関するもの(工具長や工具径およびその摩耗量を除く)とする。\\

\begin{3commonvariables}{\ttNum901100\,-\ttNum901124:工具}{番号}{内容\hspace*{0.71\textwidth}}{設定値}
\rowcolor{unusingVariables}
\ttNum901100 & (予備) &\\\hline
\ttNum901101 & 工具\verb|T02|(フェイスミル)最大刃径(直径)DCX公称値$\phi'_\mathrm D$ & 113.5\\\hline
\rowcolor{unusingVariables}
$\cdots$ & (以下 端面加工工具\verb|T02|-\verb|T05|用 予備) &\\\hline
\ttNum901105 & 工具\verb|T06|(サイドカッター)厚さ$t$ & 7.0\\\hline
\rowcolor{unusingVariables}
$\cdots$ & (以下 溝加工工具\verb|T06|, \verb|T07|用 予備) &\\\hline
\ttNum901109 & 工具\verb|T08|(サイドカッター)厚さ$t$ & 5.0\\\hline
\rowcolor{unusingVariables}
$\cdots$ & (以下 溝加工工具\verb|T08|, \verb|T10|用 予備) &\\\hline
\ttNum901113 & 工具\verb|T11|(テーパエンドミル)参照直径用 工具長補正値 & 2.0\\\hline
\rowcolor{unusingVariables}
\ttNum901114 & (工具\verb|T11|用 予備) &\\\hline
\ttNum901115 & 工具\verb|T12|(テーパエンドミル)参照直径用 工具長補正値 & 2.0\\\hline
\rowcolor{unusingVariables}
\ttNum901116 & (工具\verb|T12|用 予備) &\\\hline
\ttNum901117 & 工具\verb|T13|(テーパエンドミル)参照直径用 工具長補正値 & 2.0\\\hline
\rowcolor{unusingVariables}
$\cdots$ & (以下 溝加工工具\verb|T13|-\verb|T15|用 予備) &\\\hline
\rowcolor{unusingVariables}
\ttNum901121 & (以下 外削加工工具用 予備) &\\\hline
\end{3commonvariables}

%\clearpage
\begin{3commonvariables}{\ttNum901125\,-\ttNum901149:工具(続き)}{番号}{内容\hspace*{0.71\textwidth}}{設定値}
\rowcolor{unusingVariables}
$\cdots$ & (以下 外削加工工具用 予備) &\\\hline
\ttNum901127 & 工具\verb|T31|(Tスロットカッター)厚さ & 8.0\\\hline
\ttNum901128 & 工具\verb|T31|(Tスロットカッター)シャンク直径(公称値) & 25.0\\\hline
\ttNum901129 & 工具\verb|T32|(Tスロットカッター)厚さ & 8.0\\\hline
\ttNum901130 & 工具\verb|T32|(Tスロットカッター)シャンク直径(公称値) & 25.0\\\hline
\rowcolor{unusingVariables}
$\cdots$ & (以下 \dimple 加工工具用 予備) &\\
\end{3commonvariables}
%%%%%%%%%%%%%%%%%%%%%%%%%%%%%%%%%%%%%%%%%%%%%%%%%%%%%%%%%%
%% hosoku %%%%%%%%%%%%%%%%%%%%%%%%%%%%%%%%%%%%%%%%%%%%%%%%
%%%%%%%%%%%%%%%%%%%%%%%%%%%%%%%%%%%%%%%%%%%%%%%%%%%%%%%%%%
\begin{hosoku}
\index{タッチセンサープローブのじく@タッチセンサープローブの軸}タッチセンサープローブの軸の半径:3.75
\end{hosoku}
%%%%%%%%%%%%%%%%%%%%%%%%%%%%%%%%%%%%%%%%%%%%%%%%%%%%%%%%%%
%%%%%%%%%%%%%%%%%%%%%%%%%%%%%%%%%%%%%%%%%%%%%%%%%%%%%%%%%%
%%%%%%%%%%%%%%%%%%%%%%%%%%%%%%%%%%%%%%%%%%%%%%%%%%%%%%%%%%
%%%%%%%%%%%%%%%%%%%%%%%%%%%%%%%%%%%%%%%%%%%%%%%%%%%%%%%%%%
%% hosoku %%%%%%%%%%%%%%%%%%%%%%%%%%%%%%%%%%%%%%%%%%%%%%%%
%%%%%%%%%%%%%%%%%%%%%%%%%%%%%%%%%%%%%%%%%%%%%%%%%%%%%%%%%%
\begin{hosoku}
工具長・工具径・工具の摩耗量といったオフセットの値は、\pageautoref{sec:A.2}を参照。
\end{hosoku}
%%%%%%%%%%%%%%%%%%%%%%%%%%%%%%%%%%%%%%%%%%%%%%%%%%%%%%%%%%
%%%%%%%%%%%%%%%%%%%%%%%%%%%%%%%%%%%%%%%%%%%%%%%%%%%%%%%%%%
%%%%%%%%%%%%%%%%%%%%%%%%%%%%%%%%%%%%%%%%%%%%%%%%%%%%%%%%%%



\clearpage
%%%%%%%%%%%%%%%%%%%%%%%%%%%%%%%%%%%%%%%%%%%%%%%%%%%%%%%%%%
%% section 18.10 %%%%%%%%%%%%%%%%%%%%%%%%%%%%%%%%%%%%%%%%%%
%%%%%%%%%%%%%%%%%%%%%%%%%%%%%%%%%%%%%%%%%%%%%%%%%%%%%%%%%%
\modHeadsection{未使用(使用可)のコモン変数}
\addtocontents{lot}{\protect\addvspace{3pt}}{}{}
\addcontentsline{lot}{section}{\numberline{\thesection}\Sectionname}
この章の各々の表に載せていないコモン変数は(\dateUnusedVariables 時点において)未使用であり、新たに用いても問題ない(他のプログラムと競合しない)。
これらのコモン変数を、以下にまとめておく。
\begin{enumerate}
\item \ttNum475-\ttNum499
\item \ttNum625-\ttNum649
\item \ttNum750-\ttNum999
\item \ttNum900050-\ttNum900100
\item \ttNum900501-\ttNum900999
\item \ttNum901025-\ttNum901999
\item \ttNum901150-\ttNum907399
\end{enumerate}


%\clearpage
\vfill
%%%%%%%%%%%%%%%%%%%%%%%%%%%%%%%%%%%%%%%%%%%%%%%%%%%%%%%%%%
%%%%%%%%%%%%%%%%%%%%%%%%%%%%%%%%%%%%%%%%%%%%%%%%%%%%%%%%%%
%%%%%%%%%%%%%%%%%%%%%%%%%%%%%%%%%%%%%%%%%%%%%%%%%%%%%%%%%%
\begin{tcolorbox}[title={2023/07/28時点の\MMname 実測値}, fonttitle=\gtfamily\bfseries]
\begin{align*}
  \text{Bot ($B=0$)}
  \left\{
  \begin{array}{rl}
    X: & 97.790 \sim 99.930\\
    Y: & -823.850\\
    Z: & -634.620
  \end{array}
  \right.\quad
  \text{Top ($B=180.$)}
  \left\{
  \begin{array}{rl}
    X: & -97.980 \sim -99.570\\
    Y: & -823.780\\
    Z: & -634.720
  \end{array}
  \right.
\end{align*}\\
・$X$については、ジグの当たる点の凸部と端部($Z$方向は目分量)\\
・$Y$については、モールドの底が当たる面\\
・$Z$については、$X0$ $Y-850.$における、ジグとの接点\\
※これらの値に、\index{タッチセンサーせんたん@タッチセンサー先端}タッチセンサー先端球の半径を加減する必要がある
\end{tcolorbox}
%%%%%%%%%%%%%%%%%%%%%%%%%%%%%%%%%%%%%%%%%%%%%%%%%%%%%%%%%%
%%%%%%%%%%%%%%%%%%%%%%%%%%%%%%%%%%%%%%%%%%%%%%%%%%%%%%%%%%
%%%%%%%%%%%%%%%%%%%%%%%%%%%%%%%%%%%%%%%%%%%%%%%%%%%%%%%%%%
