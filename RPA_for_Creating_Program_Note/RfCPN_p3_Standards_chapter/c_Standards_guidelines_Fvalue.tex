%!TEX root = ../RPA_for_Creating_Program_Note.tex



工具の移動には、主に\verb|G00|, \verb|G01|, \verb|G02|, \verb|G03|, \verb|G31|が用いられる。
一般に、
\begin{enumerate}
\item \verb|G00|は主に早送りに使用されることが多い
\item \verb|G01|は直線状に移動し、主に切削の際の送りに使用されることが多い
\item \verb|G02|, \verb|G03|は円弧状に移動し、主に切削の際の送りに使用されることが多い
\item \verb|G31|は主に測定の際のスキップ機能に伴って使用されることが多い
\end{enumerate}
\verb|G00|は\index{さいだいおくりはやさ@最大送り速さ}最大送り速さで移動し、その速さは(Fコード値では)指定することはできない。
\verb|G01|, \verb|G02|, \verb|G03|, \verb|G31|はその送りの速さをFコード値で指定することができる。

なお、\verb|G28|, \verb|G30|では\verb|G00|によって移動する。
また、\verb|G31|はすべての工具で指定することができるが、スキップ機能を有するものはタッチセンサープローブのみである。
%%%%%%%%%%%%%%%%%%%%%%%%%%%%%%%%%%%%%%%%%%%%%%%%%%%%%%%%%%
%% hosoku %%%%%%%%%%%%%%%%%%%%%%%%%%%%%%%%%%%%%%%%%%%%%%%%
%%%%%%%%%%%%%%%%%%%%%%%%%%%%%%%%%%%%%%%%%%%%%%%%%%%%%%%%%%
\begin{hosoku}
この章では工具の具体的な送り速さ値を記述している。
しかし、工具の送り速さ値の具体的な数値は、(ソフトウェアでなく)ハードウェアの標準に記載するほうが望ましい。
\end{hosoku}
%%%%%%%%%%%%%%%%%%%%%%%%%%%%%%%%%%%%%%%%%%%%%%%%%%%%%%%%%%
%%%%%%%%%%%%%%%%%%%%%%%%%%%%%%%%%%%%%%%%%%%%%%%%%%%%%%%%%%
%%%%%%%%%%%%%%%%%%%%%%%%%%%%%%%%%%%%%%%%%%%%%%%%%%%%%%%%%%



%%%%%%%%%%%%%%%%%%%%%%%%%%%%%%%%%%%%%%%%%%%%%%%%%%%%%%%%%%
%% section 07.1 %%%%%%%%%%%%%%%%%%%%%%%%%%%%%%%%%%%%%%%%%%
%%%%%%%%%%%%%%%%%%%%%%%%%%%%%%%%%%%%%%%%%%%%%%%%%%%%%%%%%%
\modHeadsection{基本事項}
\begin{enumerate}
\item $X$, $Y$, $Z$方向の早送り速さ\verb|G00|のデフォルト値:60000mm/min
\item $B$方向の早送り速さ\verb|G00|のデフォルト値:12000$^\circ$/min ($\sim 33.33$回転/min)
\item $X$, $Y$, $Z$方向の指定できる送り速さの範囲:0~60000mm/min
\item 設定可能な最小単位:0.1mm/min
\item \verb|G01|, \verb|G02|, \verb|G03|, \verb|G31|を使用する際は、その速さ値(Fコード値)を省略せずに記述する
\end{enumerate}



\clearpage
%%%%%%%%%%%%%%%%%%%%%%%%%%%%%%%%%%%%%%%%%%%%%%%%%%%%%%%%%%
%% section 07.2 %%%%%%%%%%%%%%%%%%%%%%%%%%%%%%%%%%%%%%%%%%
%%%%%%%%%%%%%%%%%%%%%%%%%%%%%%%%%%%%%%%%%%%%%%%%%%%%%%%%%%
\modHeadsection{タッチセンサープローブ}
\DMname では、全長の長い\index{タッチセンサープローブ}タッチセンサープローブを用いる。
したがって、速さを大きくして移動をすると、その加速度によってセンサーが反応してしまったり、\index{タッチセンサー}タッチセンサーそのものに大きな負担がかかる。
そのため、タッチセンサーの速さに関しては他の工具よりも低めに設定する。
\begin{enumerate}[label=\Roman*., ref=\Roman*]
\item \label{item:FDTS} 原則として、\verb|G00|を用いた移動はしない
\item \ref{item:FDTS}に伴い、\verb|G28|, \verb|G30|を(直接的に)用いた移動はしない
\item \verb|G01|を位置決め(早送り)として用いるものとし、速さはF5400以下とする
\item ワークへの\index{アプローチ}アプローチの際は、\verb|G31|を用いるものとし、速さはF1500以下とする
\item 計測の際の\index{スキップ}スキップ(\verb|G31|)の速さは、計測の仕方に応じて以下のものとする
  \begin{enumerate}
  \item \index{しんごうおくれほせい@信号遅れ補正}信号遅れ補正を考慮する必要があるような場合は、速さはF50とする
  \item 信号遅れ補正を考慮する必要がない場合は、速さはF50以上300以下とする
  \end{enumerate}
\item ワークからの\index{リトラクト}リトラクトの際は\verb|G01|を用いるものとし、速さはF5400以下とする。
\end{enumerate}
ただし$Z$方向の移動はタッチセンサーではなくテーブルが移動するため、アプローチを除いて次節(タッチセンサー以外の工具)にしたがうものとする。

なお、\verb|G31|で送る場合は、センサーの電源を入れて行うこと。
電源が入っていなければ移動せず、その点がブロックエンドとなる。



\clearpage
%%%%%%%%%%%%%%%%%%%%%%%%%%%%%%%%%%%%%%%%%%%%%%%%%%%%%%%%%%
%% section 10.2 %%%%%%%%%%%%%%%%%%%%%%%%%%%%%%%%%%%%%%%%%%
%%%%%%%%%%%%%%%%%%%%%%%%%%%%%%%%%%%%%%%%%%%%%%%%%%%%%%%%%%
\modHeadsection{タッチセンサー以外の工具}
\DMname では、$Z$方向については\index{テーブル}テーブル(\index{パレット}パレット)が移動する。
したがって、$Z$方向の速さを大きくして移動すると、その加速度によってテーブル上の\index{ジグ}ジグが動いてしまう恐れがある。
そのため、$Z$方向の速さに関しては$X$, $Y$方向の速さより低めに設定する。
\begin{enumerate}[label=\Roman*., ref=\Roman*]
\item $X$, $Y$方向の移動は、\verb|G00|を用いてもよい
\item $Z$方向の移動は、原則として、\verb|G00|を用いた移動はしない
\item $Z$方向の送り速さは、F12000以下とする
\item ワークへの\index{アプローチ}アプローチの際は、\verb|G01|を用いるものとし、速さはF6000以下とする
\item 加工の際は、それぞれの加工や状況に応じて適切なFコード値を設定する
\item ワークからの\index{リトラクト}リトラクトの際は\verb|G01|を用いるものとし、速さはF12000以下とする。
\end{enumerate}


~\newline\noindent
\dateKouguSpeed における速さの設定値は、以下のとおりである。\\
\addtocontents{lot}{\protect\addvspace{3pt}}{}{}
\addcontentsline{lot}{section}{\numberline{\thesection}\Sectionname}

\begin{3columnstable}{工具の速さ 一覧\TBW}{内容}{速さ値}{備考}{Sl}{Sc}{Sc}
最大送り$XY$(タッチセンサープローブ以外) & 60000 & \verb|G00|\\\hline
最大送り$Z$(タッチセンサープローブ以外)  & 12000 &\\\hline
最大送り$XYZ$(タッチセンサープローブ)   & 5400 & \\\hline
アプローチ(タッチセンサープローブ以外)   & 6000 & \\\hline
アプローチ(タッチセンサープローブ)      & 1500 & \verb|G31|\\\hline
アプローチ(片側測定)                 & 50 & \verb|G31|\\\hline
アプローチ(両側測定)                 & 200 & \verb|G31|\\\hline
端面加工(直線)                     & 900 & \\\hline
端面加工(コーナー)                  & 900 &\\\hline
外削加工(直線)                     & 400 &\\\hline
外削加工(コーナー)                  & 400 &\\\hline
溝加工(直線)                       & 500 &\\\hline
溝加工(コーナー)                    & 400 &\\\hline
外面取加工(直線)                    & 500 &\\\hline
外面取加工(コーナー)                 & 400 &\\\hline
内面取加工(直線)                    & 400 &\\\hline
内面取加工(コーナー)                 & 400 &\\\hline
座ぐり加工(直線)\TBW                &  & \\\hline
座ぐり加工(コーナー)\TBW             &  &\\\hline
\dimple 加工(表面アプローチ)         & 540 &\\\hline
\dimple 加工(加工)                 & 100 &
\end{3columnstable}


