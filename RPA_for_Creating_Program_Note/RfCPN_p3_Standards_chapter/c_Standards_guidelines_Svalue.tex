%!TEX root = ../RPA_for_Creating_Program_Note.tex


%%%%%%%%%%%%%%%%%%%%%%%%%%%%%%%%%%%%%%%%%%%%%%%%%%%%%%%%%%
%% hosoku %%%%%%%%%%%%%%%%%%%%%%%%%%%%%%%%%%%%%%%%%%%%%%%%
%%%%%%%%%%%%%%%%%%%%%%%%%%%%%%%%%%%%%%%%%%%%%%%%%%%%%%%%%%
\begin{hosoku}
この章では工具の具体的な\index{しゅじくかいてんすう@主軸回転数}主軸回転数(\index{スピンドルかいてんすう@スピンドル回転数}スピンドル回転数)を記述している。
しかし、工具の主軸回転数の具体的な数値は、(ソフトウェアでなく)ハードウェアの標準に記載するほうが望ましい。
\end{hosoku}
%%%%%%%%%%%%%%%%%%%%%%%%%%%%%%%%%%%%%%%%%%%%%%%%%%%%%%%%%%
%%%%%%%%%%%%%%%%%%%%%%%%%%%%%%%%%%%%%%%%%%%%%%%%%%%%%%%%%%
%%%%%%%%%%%%%%%%%%%%%%%%%%%%%%%%%%%%%%%%%%%%%%%%%%%%%%%%%%


%%%%%%%%%%%%%%%%%%%%%%%%%%%%%%%%%%%%%%%%%%%%%%%%%%%%%%%%%%
%% section 07.1 %%%%%%%%%%%%%%%%%%%%%%%%%%%%%%%%%%%%%%%%%%
%%%%%%%%%%%%%%%%%%%%%%%%%%%%%%%%%%%%%%%%%%%%%%%%%%%%%%%%%%
\modHeadsection{スピンドル回転数の基本事項\TBW}
\begin{enumerate}
\item \index{でんげんとうにゅうじ(きかいほんたい)@電源投入時(機械本体)}電源投入時の\index{しゅじくかいてんすう@主軸回転数}主軸回転数:0
\item 設定可能な最小単位:1回転/min
\end{enumerate}
~\newline\noindent
\dateKouguRotation における主軸回転数の\index{せっていち(スピンドルかいてんすう)@設定値(スピンドル回転数)}設定値は、以下のとおりである。\\

\begin{2columnstable}{工具のスピンドル回転数 一覧}{|Sl|Sc|}{加工}{回転数}
端面加工 & 950\\\hline
外削加工 & 930\\\hline
溝加工 & 930\\\hline
外面取加工 & 1800\\\hline
内面取加工 & 1800\\\hline
座ぐり加工 & 500\\\hline
\dimple 加工 & 2000
\end{2columnstable}

