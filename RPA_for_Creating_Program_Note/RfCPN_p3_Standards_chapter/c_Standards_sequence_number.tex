%!TEX root = ../RPA_for_Creating_Program_Note.tex



一般に、\index{シーケンスばんごう@シーケンス番号}シーケンス番号は(重複していなければ)自由に付けて問題ない。
しかしこれに一定のルールを与えておくことで、プログラムの
\begin{enumerate}
\item どの部分で何が行われているか
\item どの部分でエラーが起きているか
\item 途中から稼働する場合、どのブロックから始めればよいか
\end{enumerate}
など、作業や管理をする際に効率よく制御することが可能になる。

そこで、ここでは\index{シーケンスばんごう@シーケンス番号}シーケンス番号(\index{Nコード}Nコード)についての規則を与える。



%%%%%%%%%%%%%%%%%%%%%%%%%%%%%%%%%%%%%%%%%%%%%%%%%%%%%%%%%%
%% section 15.1 %%%%%%%%%%%%%%%%%%%%%%%%%%%%%%%%%%%%%%%%%%
%%%%%%%%%%%%%%%%%%%%%%%%%%%%%%%%%%%%%%%%%%%%%%%%%%%%%%%%%%
\modHeadsection{シーケンス番号の基本事項}
\begin{enumerate}[label=\Roman*), ref=\Roman*)]
\item シーケンス番号は3桁とし、0埋めする
\item プログラムの始まりのシーケンス番号は\verb|N001|とする
\item 原則として、シーケンス番号は昇順とし、特に1桁目は連番とする
\end{enumerate}


%%%%%%%%%%%%%%%%%%%%%%%%%%%%%%%%%%%%%%%%%%%%%%%%%%%%%%%%%%
%% section 15.2 %%%%%%%%%%%%%%%%%%%%%%%%%%%%%%%%%%%%%%%%%%
%%%%%%%%%%%%%%%%%%%%%%%%%%%%%%%%%%%%%%%%%%%%%%%%%%%%%%%%%%
\modHeadsection{サブプログラム}
\DMname においては、原則としてサブプログラムは始めから実行されるものであり、途中の部分から実行されることはない。
そのため、\index{サブプログラム}サブプログラムについてはシーケンス番号は記述の順に(概ね\index{ブロック}ブロックごとに)連番とする。

なお、\index{エラー}エラー検出時に関するシーケンス番号、およびプログラム終了に関するシーケンス番号については、以降で述べるメインプログラムのもの(\autoref{subsec:sequenceNerror}, \pageautoref{subsec:sequenceNprgEnd})と同様とする。



\clearpage
%%%%%%%%%%%%%%%%%%%%%%%%%%%%%%%%%%%%%%%%%%%%%%%%%%%%%%%%%%
%% section 15.3 %%%%%%%%%%%%%%%%%%%%%%%%%%%%%%%%%%%%%%%%%%
%%%%%%%%%%%%%%%%%%%%%%%%%%%%%%%%%%%%%%%%%%%%%%%%%%%%%%%%%%
\modHeadsection{メインプログラム}
\DMname において、\index{メインプログラム}メインプログラム
%% footnote %%%%%%%%%%%%%%%%%%%%%
\footnote{ここでいうメインプログラムとは、下5桁が\index{ずめんばんごう@図面番号}図面番号と一致するものを指す。}
%%%%%%%%%%%%%%%%%%%%%%%%%%%%%%%%%
は\index{さぎょうしゃ@作業者}作業者が実際に設定を変更したり、途中の箇所から始めたりし得る。
そのため、メインプログラムについては各作業(計測・加工)ごとにシーケンス番号を割り振ることにする。


%%%%%%%%%%%%%%%%%%%%%%%%%%%%%%%%%%%%%%%%%%%%%%%%%%%%%%%%%%
%% subsection 15.3.1 %%%%%%%%%%%%%%%%%%%%%%%%%%%%%%%%%%%%%
%%%%%%%%%%%%%%%%%%%%%%%%%%%%%%%%%%%%%%%%%%%%%%%%%%%%%%%%%%
\subsection{N100:計測(\dimple ・逃し溝以外)}
\index{タッチセンサー}タッチセンサーを用いた計測(\dimple ・逃し溝を除く)を行う工程のシーケンス番号は100番台とする。
これには以下の\index{こうてい@工程}工程が含まれ、これらは2桁目の番号で区別される。
\begin{enumerate}
\item[100:] 芯出し計測
\item[650:] 通り芯$X$計測
\item[660:] 通り芯$Y$計測
\end{enumerate}


%\clearpage
%%%%%%%%%%%%%%%%%%%%%%%%%%%%%%%%%%%%%%%%%%%%%%%%%%%%%%%%%%
%% subsection 15.3.2 %%%%%%%%%%%%%%%%%%%%%%%%%%%%%%%%%%%%%
%%%%%%%%%%%%%%%%%%%%%%%%%%%%%%%%%%%%%%%%%%%%%%%%%%%%%%%%%%
\subsection{N200:計測(\dimple)}
\expandafterindex{\dimplekana@\dimple}\dimple および\index{にがしみぞ@逃し溝}逃し溝に関する\index{タッチセンサープローブ}タッチセンサープローブを用いた計測を行う工程のシーケンス番号は200番台とする。
これには以下の工程が含まれ、これらは2桁目の番号で区別される。
\begin{enumerate}
\item[200:] \dimple 計測
\item[250:] 逃し溝計測
\end{enumerate}


%%%%%%%%%%%%%%%%%%%%%%%%%%%%%%%%%%%%%%%%%%%%%%%%%%%%%%%%%%
%% subsection 14.2.1 %%%%%%%%%%%%%%%%%%%%%%%%%%%%%%%%%%%%%
%%%%%%%%%%%%%%%%%%%%%%%%%%%%%%%%%%%%%%%%%%%%%%%%%%%%%%%%%%
\subsection{N300:\dimple ・逃し溝加工}
\dimple および逃し溝加工を行う工程のシーケンス番号は300番台とする。
これには以下の工程が含まれ、これらは2桁目の番号で区別される。
\begin{enumerate}
\item[300:] \dimple 加工
\item[350:] 逃し溝加工
\end{enumerate}


%%%%%%%%%%%%%%%%%%%%%%%%%%%%%%%%%%%%%%%%%%%%%%%%%%%%%%%%%%
%% subsection 14.2.1 %%%%%%%%%%%%%%%%%%%%%%%%%%%%%%%%%%%%%
%%%%%%%%%%%%%%%%%%%%%%%%%%%%%%%%%%%%%%%%%%%%%%%%%%%%%%%%%%
\subsection{N400:トップ側の加工}
トップ側の加工を行う工程のシーケンス番号は400番台とする。
これには以下の工程が含まれ、これらは2桁目の番号で区別される。
\begin{enumerate}
\item[400:] トップ端面加工
\item[410:] トップ外削加工
\item[420:] 溝加工
\item[430:] トップ外面取加工
\item[440:] トップ内面取加工
\item[450:] 座ぐり加工
\end{enumerate}


\clearpage
%%%%%%%%%%%%%%%%%%%%%%%%%%%%%%%%%%%%%%%%%%%%%%%%%%%%%%%%%%
%% subsection 14.2.1 %%%%%%%%%%%%%%%%%%%%%%%%%%%%%%%%%%%%%
%%%%%%%%%%%%%%%%%%%%%%%%%%%%%%%%%%%%%%%%%%%%%%%%%%%%%%%%%%
\subsection{N500:ボトム側の加工}
ボトム側の加工を行う工程のシーケンス番号は500番台とする。
これには以下の工程が含まれ、これらは2桁目の番号で区別される。
\begin{enumerate}
\item[500:] ボトム端面加工
\item[510:] ボトム外削加工
\item[530:] ボトム外面取加工
\item[540:] ボトム内面取加工
\end{enumerate}


%\clearpage
%%%%%%%%%%%%%%%%%%%%%%%%%%%%%%%%%%%%%%%%%%%%%%%%%%%%%%%%%%
%% subsection 14.2.1 %%%%%%%%%%%%%%%%%%%%%%%%%%%%%%%%%%%%%
%%%%%%%%%%%%%%%%%%%%%%%%%%%%%%%%%%%%%%%%%%%%%%%%%%%%%%%%%%
\subsection{N800:エラー\label{subsec:sequenceNerror}\TBW}
\index{エラー}エラー検出時に\index{ジャンプ}ジャンプするシーケンス番号は800番台とする。
エラーの種類(システム変数\ttNum3000の値)に応じて以下のように分類し、(概ね)\index{ブロック}プロックごとに連番とする。
\begin{enumerate}
\item[800:] \verb|#3000=121 (Argument is not assigned)|
\item[810:] \verb|#3000=...|
\item[820:] \verb|#3000=1 (Pallet Alarm)|,\\
            \verb|#3000=145 (Sensor-Low-Battery)|, \verb|#3000=146 (Sensor-Alarm)|
\end{enumerate}


%%%%%%%%%%%%%%%%%%%%%%%%%%%%%%%%%%%%%%%%%%%%%%%%%%%%%%%%%%
%% subsection 14.2.1 %%%%%%%%%%%%%%%%%%%%%%%%%%%%%%%%%%%%%
%%%%%%%%%%%%%%%%%%%%%%%%%%%%%%%%%%%%%%%%%%%%%%%%%%%%%%%%%%
\subsection{N990:プログラムの終了\label{subsec:sequenceNprgEnd}}
\index{プログラムしゅうりょう@プログラム終了}プログラムを終了する工程のシーケンス番号は990番台とする。
特に、プログラムの終了はN999とする。



\clearpage
\noindent
改めて上記の\index{シーケンスばんごういちらん@シーケンス番号一覧}シーケンス番号を一覧にしておく。\\

%%%%%%%%%%%%%%%%%%%%%%%%%%%%%%%%%%%%%%%%%%%%%%%%%%%%%%%%%%
%% sequence numbers %%%%%%%%%%%%%%%%%%%%%%%%%%%%%%%%%%%%%%
%%%%%%%%%%%%%%%%%%%%%%%%%%%%%%%%%%%%%%%%%%%%%%%%%%%%%%%%%%
\begin{2columnstable}{シーケンス番号 一覧(メインプログラム)\TBW}{|Sc|Sl|}{N番号}{内容}
\verb|N001| & プログラムの始まり\\\hline
\hline
\verb|N10x| & タッチセンサー計測(芯出し)\\\hline
\hline
\verb|N20x| & タッチセンサー計測(\dimple)\\\hline
\verb|N25x| & タッチセンサー計測(逃し溝)\\\hline
\hline
\verb|N30x| & \dimple 加工\\\hline
\verb|N35x| & 逃し溝加工\\\hline
\hline
\verb|N40x| & トップ端面加工\\\hline
\verb|N41x| & トップ外削加工\\\hline
\verb|N42x| & 溝加工\\\hline
\verb|N43x| & トップ外面取加工\\\hline
\verb|N44x| & トップ内面取加工\\\hline
\verb|N45x| & 座ぐり加工\\\hline
\hline
\verb|N500| & ボトム端面加工\\\hline
\verb|N51x| & ボトム外削加工\\\hline
\verb|N52x| & ボトム外面取加工\\\hline
\verb|N53x| & ボトム内面取加工\\\hline
\hline
\verb|N65x| & タッチセンサー計測(通り芯$X$)\\\hline
\verb|N66x| & タッチセンサー計測(通り芯$Y$)\\\hline
\hline
\verb|N80x| & 引数によるエラー\\\hline
\verb|N81x|\TBW & \\\hline
\verb|N82x| & パレットまたはタッチセンサーによるエラー\\\hline
\hline
\verb|N99x| & プログラム終了の工程\\\hline
\verb|N999| & プログラム終了(\verb|M02|または\verb|M30|または\verb|M99|)
\end{2columnstable}


