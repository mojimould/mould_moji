%!TEX root = ../RPA_for_Creating_Program_Note.tex



マシニングセンタの精度を保つためには、\index{せっちかんきょう@設置環境}設置環境を整える必要がある。
ここではその目安を与える。



%%%%%%%%%%%%%%%%%%%%%%%%%%%%%%%%%%%%%%%%%%%%%%%%%%%%%%%%%%
%% section 11.1 %%%%%%%%%%%%%%%%%%%%%%%%%%%%%%%%%%%%%%%%%%
%%%%%%%%%%%%%%%%%%%%%%%%%%%%%%%%%%%%%%%%%%%%%%%%%%%%%%%%%%
\modHeadsection{設置箇所における基本事項}
マシニングセンタを設置する場所としては以下のような場所を選ぶものとする。
\begin{enumerate}
\item 日光が直接当たらない
\item 温風や冷風が直接当たらない
\item 基礎の地耐力が10t/m$^2$以上である
\end{enumerate}



%%%%%%%%%%%%%%%%%%%%%%%%%%%%%%%%%%%%%%%%%%%%%%%%%%%%%%%%%%
%% section 11.1 %%%%%%%%%%%%%%%%%%%%%%%%%%%%%%%%%%%%%%%%%%
%%%%%%%%%%%%%%%%%%%%%%%%%%%%%%%%%%%%%%%%%%%%%%%%%%%%%%%%%%
\modHeadsection{設置の条件}
\DMname についてメーカーが指定する主な\index{せっちじょうけん(きかいほんたい)@設置条件(機械本体)}設置条件は以下のとおりである。
詳細については\expandafterindex{オペレーションマニュアル(\DMname)@オペレーションマニュアル(\DMname)}オペレーションマニュアルを参照されたし。\\

\begin{multicollongtblr}{機械据付要件}{l X[l]}
項目 & 内容\\
室温 & $20\pm1^\circ$C\\
室温の温度変化 & $0.5^\circ$C/day以内\\
温度勾配 & $0.2^\circ$C/h以内\\
床面より5m間での上下間の室温差 & $0.7^\circ$C以内\\
基礎床面温度と室温の差 & $0.5^\circ$C以内\\
湿度 & $60\pm5$\%\\
切削油の温度 & $\pm2^\circ$C以内\\
周囲の振動による機械への影響 & $0.1\mu$m以内\\
機械と天井との間隔 & $1.5$m以上\\
床面の平面度 & $\pm5$mm(推奨値)\\
\end{multicollongtblr}

