%!TEX root = ./RPA_for_Creating_Program_Note.tex


基本的に、\index{すうちじょうほう@数値情報}数値情報については数値計算用の言語を用いて行うため、その詳細は別ドキュメントに譲る。
ここでは各明細用の\index{メインプログラム}メインプログラムの記述に際して、\index{すうちけいさん@数値計算}数値計算に必要な部分をピックアップする。
なお、ここでは主に\DMname について述べるため、\index{スペーサ}スペーサに関するものは省略する。


%%%%%%%%%%%%%%%%%%%%%%%%%%%%%%%%%%%%%%%%%%%%%%%%%%%%%%%%%%
%% section 30.1 %%%%%%%%%%%%%%%%%%%%%%%%%%%%%%%%%%%%%%%%%%
%%%%%%%%%%%%%%%%%%%%%%%%%%%%%%%%%%%%%%%%%%%%%%%%%%%%%%%%%%
\modHeadsection{再振分長・再張出長・均等振分角の数値情報}
各パラメータを以下とする。
\begin{align*}
  \varDelta_x' = \varDelta_x+\sqrt{R_\mathrm i'-\bar l^2}\ , \quad
  R_\mathrm i' = R_\mathrm c-\frac{W_x}2-\rho\ ,\quad
  \bar l = l-\frac\sigma2\ ,\quad
  f_d = \frac{f_\mathrm B-f_\mathrm T}2\ .
\end{align*}


%%%%%%%%%%%%%%%%%%%%%%%%%%%%%%%%%%%%%%%%%%%%%%%%%%%%%%%%%%
%% subsection 30.1.1 %%%%%%%%%%%%%%%%%%%%%%%%%%%%%%%%%%%%%
%%%%%%%%%%%%%%%%%%%%%%%%%%%%%%%%%%%%%%%%%%%%%%%%%%%%%%%%%%
\subsection{再振分長}
\index{テーブル}テーブルを$-\theta$だけ回転させて調整したトップ・ボトム側の\index{さいふりわけちょう@再振分長}振分長$f'_\mathrm T$, $f'_\mathrm B$は、\pageeqref{eq:saifuriwake}より、
\begin{align*}
  \text{トップ側:}\quad
  & \HLbox{f_\mathrm T' = f_\mathrm T+\varDelta_x'\!\sin\theta}\ ,\\
  \text{ボトム側:}\quad
  & \HLbox{f_\mathrm B' = (f_\mathrm T+f_\mathrm B)-f_\mathrm T'}\ .
\end{align*}


%%%%%%%%%%%%%%%%%%%%%%%%%%%%%%%%%%%%%%%%%%%%%%%%%%%%%%%%%%
%% subsection 30.1.2 %%%%%%%%%%%%%%%%%%%%%%%%%%%%%%%%%%%%%
%%%%%%%%%%%%%%%%%%%%%%%%%%%%%%%%%%%%%%%%%%%%%%%%%%%%%%%%%%
\subsection{再張出長}
テーブルを$-\theta$だけ回転させた後の\index{ジグ}ジグ(長さ$2l$)からの\index{さいはりだしちょう@再張出長}張出長に換算すると、それぞれ
\begin{align*}
  \text{トップ側:}\quad
  & \HLbox{f_\mathrm T'-l}\ ,\\
  \text{ボトム側:}\quad
  & \HLbox{f_\mathrm B'-l}\ .
\end{align*}


%%%%%%%%%%%%%%%%%%%%%%%%%%%%%%%%%%%%%%%%%%%%%%%%%%%%%%%%%%
%% subsection 30.1.3 %%%%%%%%%%%%%%%%%%%%%%%%%%%%%%%%%%%%%
%%%%%%%%%%%%%%%%%%%%%%%%%%%%%%%%%%%%%%%%%%%%%%%%%%%%%%%%%%
\subsection{均等振分角}
$f'_\mathrm T$および$f'_\mathrm B$が均等になり、かつトップ側およびボトム側の\index{たんめん@端面}端面が$X$軸方向に平行になるときの\index{かたむきかく(ふりわけちょうせい)@傾き角(振分調整)}回転角$\theta_\mathrm T'$, $\theta_\mathrm B'$は、\pageeqref{eq:saifuriwakeangle}よりそれぞれ、
\begin{align*}
  \text{トップ側:}\quad
  & \HLbox{\theta_\mathrm T' = -\sin^{-1}\frac{f_d}{\varDelta_x'}}\ ,\\
  \text{ボトム側:}\quad
  & \HLbox{\theta_\mathrm B' = \pi-\sin^{-1}\frac{f_d}{\varDelta_x'}}\ .
\end{align*}



\clearpage
%%%%%%%%%%%%%%%%%%%%%%%%%%%%%%%%%%%%%%%%%%%%%%%%%%%%%%%%%%
%% section 30.2 %%%%%%%%%%%%%%%%%%%%%%%%%%%%%%%%%%%%%%%%%%
%%%%%%%%%%%%%%%%%%%%%%%%%%%%%%%%%%%%%%%%%%%%%%%%%%%%%%%%%%
\modHeadsection{原点設定の数値情報}


%%%%%%%%%%%%%%%%%%%%%%%%%%%%%%%%%%%%%%%%%%%%%%%%%%%%%%%%%%
%% subsection 30.2.1 %%%%%%%%%%%%%%%%%%%%%%%%%%%%%%%%%%%%%
%%%%%%%%%%%%%%%%%%%%%%%%%%%%%%%%%%%%%%%%%%%%%%%%%%%%%%%%%%
\subsection{ボトム側の外側中心\texorpdfstring{$X$}{X}}

%%%%%%%%%%%%%%%%%%%%%%%%%%%%%%%%%%%%%%%%%%%%%%%%%%%%%%%%%%
%% subsubsection 30.2.1.1 %%%%%%%%%%%%%%%%%%%%%%%%%%%%%%%%
%%%%%%%%%%%%%%%%%%%%%%%%%%%%%%%%%%%%%%%%%%%%%%%%%%%%%%%%%%
\subsubsection{ボトム端の外側中心\texorpdfstring{$X$}{X}}
\index{テーブルちゅうしん@テーブル中心}テーブル中心\index{P(テーブルちゅうしん)@P(テーブル中心)}Pを\index{げんてんP@原点P}原点とした、($-\theta$回転後の)\index{ボトムたんのそとがわちゅうしん@ボトム端の外側中心}ボトム端の外径中心の$X$位置は、\pageeqref{eq:tableBc}より、
\begin{align*}
  \HLbox{%
    \varDelta_x'\cos\theta
    -\frac{\sqrt{R_\mathrm o^2-f_\mathrm B^2}+\sqrt{R_\mathrm i^2-f_\mathrm B^2}}2%
  }\ .
\end{align*}

%%%%%%%%%%%%%%%%%%%%%%%%%%%%%%%%%%%%%%%%%%%%%%%%%%%%%%%%%%
%% subsubsection 30.2.1.2 %%%%%%%%%%%%%%%%%%%%%%%%%%%%%%%%
%%%%%%%%%%%%%%%%%%%%%%%%%%%%%%%%%%%%%%%%%%%%%%%%%%%%%%%%%%
\subsubsection{ボトム側の外削中心\texorpdfstring{$X$}{X}(ボトムA側肉厚基準)}
\index{テーブルちゅうしん@テーブル中心}テーブル中心\index{P(テーブルちゅうしん)@P(テーブル中心)}Pを\index{げんてんP@原点P}原点とした、($-\theta$回転後の)\index{ボトムAがわにくあつ@ボトムA側肉厚}ボトムA側肉厚を\index{きじゅん(ボトムAがわにくあつ)@基準(ボトムA側肉厚)}基準とした\index{ボトムがわのがいさくちゅうしん@ボトム側の外削中心}ボトム側の外削中心$\mathfrak B_\mathrm c'$の(おおよその)$X$座標は、\pageeqref{eq:gaisakucenterBt}より、
\begin{align*}
  \HLbox{%
    \varDelta_x'\cos\theta
    -\frac{\sqrt{R_\mathrm o^2-f_\mathrm B^2}+\sqrt{R_\mathrm i^2-f_\mathrm B^2}}2
    -\frac{w_\mathrm B}2
    -\tau_\mathrm B
    +\frac{\mathfrak W_\mathrm B}2
  }\ .
\end{align*}

%%%%%%%%%%%%%%%%%%%%%%%%%%%%%%%%%%%%%%%%%%%%%%%%%%%%%%%%%%
%% subsubsection 30.2.1.2 %%%%%%%%%%%%%%%%%%%%%%%%%%%%%%%%
%%%%%%%%%%%%%%%%%%%%%%%%%%%%%%%%%%%%%%%%%%%%%%%%%%%%%%%%%%
\subsubsection{ボトム側の外削中心\texorpdfstring{$X$}{X}(トップA側肉厚基準)}
\index{テーブルちゅうしん@テーブル中心}テーブル中心\index{P(テーブルちゅうしん)@P(テーブル中心)}Pを\index{げんてんP@原点P}原点とした、($-\theta$回転後の)\index{トップAがわにくあつ@トップA側肉厚}トップA側肉厚を\index{きじゅん(トップAがわにくあつ)@基準(トップA側肉厚)}基準とした\index{トップがわのがいさくちゅうしん@トップ側の外削中心}ボトム側の外削中心$\mathfrak B_\mathrm c'$の(おおよその)$X$座標は、\pageeqref{eq:gaisakucenterTt}より、
\begin{align*}
  \HLbox{%
    -\left(
      \frac{\sqrt{R_\mathrm o^2-f_\mathrm T^2}+\sqrt{R_\mathrm i^2-f_\mathrm T^2}}2
      -\varDelta_x'\cos\theta
      +\frac{w_\mathrm T}2
      +\tau_\mathrm T
      -\frac{\mathfrak W_\mathrm T}2
    \right)
    +T_x
  }\ .
\end{align*}


%%%%%%%%%%%%%%%%%%%%%%%%%%%%%%%%%%%%%%%%%%%%%%%%%%%%%%%%%%
%% subsection 30.2.2 %%%%%%%%%%%%%%%%%%%%%%%%%%%%%%%%%%%%%
%%%%%%%%%%%%%%%%%%%%%%%%%%%%%%%%%%%%%%%%%%%%%%%%%%%%%%%%%%
\subsection{ボトム側の内側中心\texorpdfstring{$X$}{X}}

%%%%%%%%%%%%%%%%%%%%%%%%%%%%%%%%%%%%%%%%%%%%%%%%%%%%%%%%%%
%% subsubsection 30.2.2.1 %%%%%%%%%%%%%%%%%%%%%%%%%%%%%%%%
%%%%%%%%%%%%%%%%%%%%%%%%%%%%%%%%%%%%%%%%%%%%%%%%%%%%%%%%%%
\subsubsection{ボトム端の湾曲中心\texorpdfstring{$X$}{X}}
\index{テーブルちゅうしん@テーブル中心}テーブル中心\index{P(テーブルちゅうしん)@P(テーブル中心)}Pを\index{げんてんP@原点P}原点とした、($-\theta$回転後の)\index{ボトムたんのわんきょくちゅうしん@ボトム端の湾曲中心}ボトム端の湾曲中心の$X$値は、\pageeqref{eq:tableBRc}より、
\begin{align*}
  \HLbox{\varDelta_x'\!\cos\theta-\sqrt{R_\mathrm c^2-f_\mathrm B^2}}~.
\end{align*}

%%%%%%%%%%%%%%%%%%%%%%%%%%%%%%%%%%%%%%%%%%%%%%%%%%%%%%%%%%
%% subsubsection 30.2.2.2 %%%%%%%%%%%%%%%%%%%%%%%%%%%%%%%%
%%%%%%%%%%%%%%%%%%%%%%%%%%%%%%%%%%%%%%%%%%%%%%%%%%%%%%%%%%
\subsubsection{ボトム端の内側中心\texorpdfstring{$X$}{X}}
\index{テーブルちゅうしん@テーブル中心}テーブル中心\index{P(テーブルちゅうしん)@P(テーブル中心)}Pを\index{げんてんP@原点P}原点とした、($-\theta$回転後の)\index{ボトムたんのうちがわちゅうしん@ボトム端の内側中心}ボトム端の内側中心は、ボトム端の湾曲中心をもって代用してもよいものとする。


\clearpage
%%%%%%%%%%%%%%%%%%%%%%%%%%%%%%%%%%%%%%%%%%%%%%%%%%%%%%%%%%
%% subsection 30.2.3 %%%%%%%%%%%%%%%%%%%%%%%%%%%%%%%%%%%%%
%%%%%%%%%%%%%%%%%%%%%%%%%%%%%%%%%%%%%%%%%%%%%%%%%%%%%%%%%%
\subsection{トップ側の外側中心\texorpdfstring{$X$}{X}}

%%%%%%%%%%%%%%%%%%%%%%%%%%%%%%%%%%%%%%%%%%%%%%%%%%%%%%%%%%
%% subsubsection 30.2.3.1 %%%%%%%%%%%%%%%%%%%%%%%%%%%%%%%%
%%%%%%%%%%%%%%%%%%%%%%%%%%%%%%%%%%%%%%%%%%%%%%%%%%%%%%%%%%
\subsubsection{トップ端の外側中心\texorpdfstring{$X$}{X}}
\index{テーブルちゅうしん@テーブル中心}テーブル中心\index{P(テーブルちゅうしん)@P(テーブル中心)}Pを\index{げんてんP@原点P}原点とした、($-\theta$回転後の\index{トップたんのそとがわちゅうしん@トップ端の外側中心}外側中心の$X$位置は、\pageeqref{eq:tableTc}より、
\begin{align*}
  \HLbox{%
    \frac{\sqrt{R_\mathrm o^2-f_\mathrm T^2}+\sqrt{R_\mathrm i^2-f_\mathrm T^2}}2-\varDelta_x'\cos\theta%
  }~.
\end{align*}

%%%%%%%%%%%%%%%%%%%%%%%%%%%%%%%%%%%%%%%%%%%%%%%%%%%%%%%%%%
%% subsubsection 30.2.3.2 %%%%%%%%%%%%%%%%%%%%%%%%%%%%%%%%
%%%%%%%%%%%%%%%%%%%%%%%%%%%%%%%%%%%%%%%%%%%%%%%%%%%%%%%%%%
\subsubsection{トップ側の外削中心\texorpdfstring{$X$}{X}(ボトムA側肉厚基準)}
\index{テーブルちゅうしん@テーブル中心}テーブル中心\index{P(テーブルちゅうしん)@P(テーブル中心)}Pを\index{げんてんP@原点P}原点とした、($-\theta$回転後の)\index{ボトムAがわにくあつ@ボトムA側肉厚}ボトムA側肉厚を\index{きじゅん(ボトムAがわにくあつ)@基準(ボトムA側肉厚)}基準とした\index{トップがわのがいさくちゅうしん@トップ側の外削中心}トップ側の外削中心$\mathfrak T_\mathrm c'$の(おおよその)$X$座標は、\pageeqref{eq:gaisakucenterBt}より、
\begin{align*}
  \HLbox{%
    -\left(
      \varDelta_x'\cos\theta
      -\frac{\sqrt{R_\mathrm o^2-f_\mathrm B^2}+\sqrt{R_\mathrm i^2-f_\mathrm B^2}}2
      -\frac{w_\mathrm B}2
      -\tau_\mathrm B
      +\frac{\mathfrak W_\mathrm B}2
    \right)
    +T_x
  }\ .
\end{align*}

%%%%%%%%%%%%%%%%%%%%%%%%%%%%%%%%%%%%%%%%%%%%%%%%%%%%%%%%%%
%% subsubsection 30.2.3.3 %%%%%%%%%%%%%%%%%%%%%%%%%%%%%%%%
%%%%%%%%%%%%%%%%%%%%%%%%%%%%%%%%%%%%%%%%%%%%%%%%%%%%%%%%%%
\subsubsection{トップ側の外削中心\texorpdfstring{$X$}{X}(トップA側肉厚基準)}
\index{テーブルちゅうしん@テーブル中心}テーブル中心\index{P(テーブルちゅうしん)@P(テーブル中心)}Pを\index{げんてんP@原点P}原点とした、($-\theta$回転後の)\index{トップAがわにくあつ@トップA側肉厚}トップA側肉厚を\index{きじゅん(トップAがわにくあつ)@基準(トップA側肉厚)}基準とした\index{トップがわのがいさくちゅうしん@トップ側の外削中心}トップ側の外削中心$\mathfrak T_\mathrm c'$の(おおよその)$X$座標は、\pageeqref{eq:gaisakucenterTt}より、
\begin{align*}
  \HLbox{%
    \frac{\sqrt{R_\mathrm o^2-f_\mathrm T^2}+\sqrt{R_\mathrm i^2-f_\mathrm T^2}}2
    -\varDelta_x'\cos\theta
    +\frac{w_\mathrm T}2
    +\tau_\mathrm T
    -\frac{\mathfrak W_\mathrm T}2
  }\ .
\end{align*}

%%%%%%%%%%%%%%%%%%%%%%%%%%%%%%%%%%%%%%%%%%%%%%%%%%%%%%%%%%
%% subsubsection 30.2.3.4 %%%%%%%%%%%%%%%%%%%%%%%%%%%%%%%%
%%%%%%%%%%%%%%%%%%%%%%%%%%%%%%%%%%%%%%%%%%%%%%%%%%%%%%%%%%
\subsubsection{溝中心\texorpdfstring{$X$}{X}(A側溝深さ基準・外削なし)}
\index{みぞいち@溝位置}溝位置$\kappa_p$, \index{みぞはば@溝幅}溝幅$\kappa_w$, \index{Aがわみぞふかさ@A側溝深さ}A側溝深さ$\kappa_d'$, \index{みぞACけい@溝AC径}溝AC径$W_{mx}$に対し、\index{みぞちゅうしん@溝中心}溝中心M$'$の$X$座標は\pageeqref{eq:mizocenterA}より、
\begin{gather*}
  \HLbox{%
    \sqrt{R_\mathrm o^2-\left(f_\mathrm T-\kappa_p-\frac{\kappa_w}2\right)^{\!2}}
    -\kappa_d
    -\frac{W_{mx}}2
    -\varDelta_x%
  }\\[9pt]
  \left(
  \kappa_d
  = \frac{2\kappa_d'-\kappa_w\sin\zeta}{1+\cos^2\zeta}\cos\zeta
    +\sqrt{R_\mathrm o^2-\left(f_\mathrm T-\kappa_p-\frac{\kappa_w}2\right)^{\!2}}
    -\sqrt{R_\mathrm o^2-\left(f_\mathrm T-\kappa_p\right)^2}
  \right).
\end{gather*}


\clearpage
%%%%%%%%%%%%%%%%%%%%%%%%%%%%%%%%%%%%%%%%%%%%%%%%%%%%%%%%%%
%% subsection 30.2.4 %%%%%%%%%%%%%%%%%%%%%%%%%%%%%%%%%%%%%
%%%%%%%%%%%%%%%%%%%%%%%%%%%%%%%%%%%%%%%%%%%%%%%%%%%%%%%%%%
\subsection{トップ側の内側中心\texorpdfstring{$X$}{X}}

%%%%%%%%%%%%%%%%%%%%%%%%%%%%%%%%%%%%%%%%%%%%%%%%%%%%%%%%%%
%% subsubsection 30.4.1.1 %%%%%%%%%%%%%%%%%%%%%%%%%%%%%%%%
%%%%%%%%%%%%%%%%%%%%%%%%%%%%%%%%%%%%%%%%%%%%%%%%%%%%%%%%%%
\subsubsection{トップ端の湾曲中心\texorpdfstring{$X$}{X}}
\index{テーブルちゅうしん@テーブル中心}テーブル中心\index{P(テーブルちゅうしん)@P(テーブル中心)}Pを\index{げんてんP@原点P}原点とした、($-\theta$回転後の)\index{トップたんのわんきょくちゅうしん@トップ端の湾曲中心}トップ端の湾曲中心の$X$値は、\pageeqref{eq:tableTRc}より、
\begin{align*}
  \HLbox{\sqrt{R_\mathrm c^2-f_\mathrm T^2}-\varDelta_x'\!\cos\theta}~.
\end{align*}

%%%%%%%%%%%%%%%%%%%%%%%%%%%%%%%%%%%%%%%%%%%%%%%%%%%%%%%%%%
%% subsubsection 30.4.1.2 %%%%%%%%%%%%%%%%%%%%%%%%%%%%%%%%
%%%%%%%%%%%%%%%%%%%%%%%%%%%%%%%%%%%%%%%%%%%%%%%%%%%%%%%%%%
\subsubsection{トップ端の内側中心\texorpdfstring{$X$}{X}}
\index{テーブルちゅうしん@テーブル中心}テーブル中心\index{P(テーブルちゅうしん)@P(テーブル中心)}Pを\index{げんてんP@原点P}原点とした、($-\theta$回転後の)\index{トップたんのうちがわちゅうしん@トップ端の内側中心}トップ端の内側中心は、トップ端の湾曲中心をもって代用してもよいものとする。


%%%%%%%%%%%%%%%%%%%%%%%%%%%%%%%%%%%%%%%%%%%%%%%%%%%%%%%%%%
%% subsection 30.2.1 %%%%%%%%%%%%%%%%%%%%%%%%%%%%%%%%%%%%%
%%%%%%%%%%%%%%%%%%%%%%%%%%%%%%%%%%%%%%%%%%%%%%%%%%%%%%%%%%
\subsection{外側中心・内側中心\texorpdfstring{$Y$}{Y}}
\index{ジグ}ジグの底の$Y$座標を$\varDelta_y$とすると、\index{そとがわちゅうしんY@外側中心$Y$}外側中心および\index{うちがわちゅうしんY@内側中心$Y$}内側中心$Y$座標は、
\begin{align*}
  \HLbox{\varDelta_y+\frac{W_y}2}\ .
\end{align*}
