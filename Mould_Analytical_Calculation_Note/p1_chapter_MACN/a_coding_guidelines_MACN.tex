%!TEX root = ../Mould_Analytical_Calculation_Note.tex

ここでは\DMname に関して、プログラムの構成や番号付けの規則を記載する。
なお、ここに記載しているものは正式なルールではなく、だいたいの目安・方針である。


%%%%%%%%%%%%%%%%%%%%%%%%%%%%%%%%%%%%%%%%%%%%%%%%%%%%%%%%%%
%% section C.1 %%%%%%%%%%%%%%%%%%%%%%%%%%%%%%%%%%%%%%%%%%%
%%%%%%%%%%%%%%%%%%%%%%%%%%%%%%%%%%%%%%%%%%%%%%%%%%%%%%%%%%
\section{プログラムの構成}
\begin{enumerate}
\item 個々の製品の計測・加工に対するプログラムをメインプログラムとする
\item 製品の各々の部分の計測・加工に対するプログラムをサブプログラムとして、メインプログラムに挿入する
\item サブプログラムの種類(番号)は、以下のように内容で分ける
  \begin{enumerate}
  \item 内面溝・逃し溝以外の計測に対するプログラム
  \item 内面溝の計測に対するプログラム
  \item 逃し溝の計測に対するプログラム
  \item 内面溝・逃し溝以外の加工に対するプログラム
  \item 内面溝の加工に対するプログラム
  \item 逃し溝の加工に対するプログラム
  \item その他、製品の加工には直接的には関係ないプログラム
  \end{enumerate}
\end{enumerate}



%%%%%%%%%%%%%%%%%%%%%%%%%%%%%%%%%%%%%%%%%%%%%%%%%%%%%%%%%%
%% section C.2 %%%%%%%%%%%%%%%%%%%%%%%%%%%%%%%%%%%%%%%%%%%
%%%%%%%%%%%%%%%%%%%%%%%%%%%%%%%%%%%%%%%%%%%%%%%%%%%%%%%%%%
\section{プログラムの番号付け}
\begin{enumerate}
\item プログラム番号には半角数字のみを用いる
\item プログラム番号には8桁の数字を用いる(ただし、左側0埋めの有無は問わない)
\item プログラム番号は右詰めとする(左側0埋めの有無は問わない)
\item 製品の図面番号(番号部分)とメインプログラム番号は同じものとする
%% footnote %%%%%%%%%%%%%%%%%%%%%
\footnote{このルールだと、バンドルのプログラム(O7xxx, O8xxx, O9xxx)と重複するおそれがある。
これについてはそうした問題に直面したときにその都度に対応するものとする。
基本的には、バンドルのプログラムを(可能であれば)変更する方針とする。}
%%%%%%%%%%%%%%%%%%%%%%%%%%%%%%%%%
\item 測定(内面溝・逃し溝を除く)に関するものは6桁目を1とする
\item 測定(内面溝)に関するものは6桁目を2とする
\item 測定(逃し溝)の測定に関するものは6桁目を3とする
\item 加工(内面溝・逃し溝を除く)に関するものは6桁目を5とする
\item 加工(内面溝)に関するものは6桁目を6とする
\item 加工(逃し溝)に関するものは6桁目を7とする
\item その他、計測・加工に直接関しないものは6桁目を9とする
\item 計測・加工の両方に同じもの用いるものは番号の若いほうに合わせる
\end{enumerate}
