%!TEX root = ../Mould_Analytical_Calculation_Note.tex

ここでは\DMname における製品の加工・測定に関する、プログラムの構成や番号付けの規則を記載する
%% footnote %%%%%%%%%%%%%%%%%%%%%
\footnote{工具長の測定やジグの測定など、製品の加工とは直接関係しないプログラムについてはこの限りではない。}。
%%%%%%%%%%%%%%%%%%%%%%%%%%%%%%%%%
なお、ここに記載しているものは正式なルールではなく、だいたいの目安・方針である。


%%%%%%%%%%%%%%%%%%%%%%%%%%%%%%%%%%%%%%%%%%%%%%%%%%%%%%%%%%
%% section C.1 %%%%%%%%%%%%%%%%%%%%%%%%%%%%%%%%%%%%%%%%%%%
%%%%%%%%%%%%%%%%%%%%%%%%%%%%%%%%%%%%%%%%%%%%%%%%%%%%%%%%%%
\section{プログラムの構成}
\begin{enumerate}
\item 個々の製品の計測・加工に対するプログラムはメインプログラムとする
\item 製品の各々の部分の計測・加工に対するプログラムはサブプログラムとして、メインプログラムに挿入する
\item サブプログラムの種類(番号)は、以下のように内容で分ける
  \begin{enumerate}
  \item 内面溝・逃し溝以外の計測に対するプログラム
  \item 内面溝の計測に対するプログラム
  \item 逃し溝の計測に対するプログラム
  \item 内面溝・逃し溝以外の加工に対するプログラム
  \item 内面溝の加工に対するプログラム
  \item 逃し溝の加工に対するプログラム
  \item その他、間接的な用途に使用するプログラム
  \end{enumerate}
\end{enumerate}



%%%%%%%%%%%%%%%%%%%%%%%%%%%%%%%%%%%%%%%%%%%%%%%%%%%%%%%%%%
%% section C.2 %%%%%%%%%%%%%%%%%%%%%%%%%%%%%%%%%%%%%%%%%%%
%%%%%%%%%%%%%%%%%%%%%%%%%%%%%%%%%%%%%%%%%%%%%%%%%%%%%%%%%%
\section{プログラムの番号付け}
\begin{enumerate}
\item プログラム番号には半角数字のみを用いる
\item プログラム番号には8桁の数字を用いる(ただし、左側0埋めの有無は問わない)
\item プログラム番号は右詰めとする(左側0埋めの有無は問わない)
\end{enumerate}
これをふまえ、メインプログラムとサブプログラムではそれぞれの以下のように番号付けを行う。



%%%%%%%%%%%%%%%%%%%%%%%%%%%%%%%%%%%%%%%%%%%%%%%%%%%%%%%%%%
%% subsection C.2.1 %%%%%%%%%%%%%%%%%%%%%%%%%%%%%%%%%%%%%%
%%%%%%%%%%%%%%%%%%%%%%%%%%%%%%%%%%%%%%%%%%%%%%%%%%%%%%%%%%
\subsection{メインプログラム}
\begin{enumerate}
\item 製品の図面番号(番号部分)とメインプログラム番号は同じものとする
%% footnote %%%%%%%%%%%%%%%%%%%%%
\footnote{この規則だと、バンドルのプログラム(O7xxx, O8xxx, O9xxx)と重複する恐れがある。
これについてはそうした問題に直面したときにその都度に対応するものとする。
基本的には、バンドルのプログラムを(可能であれば)変更する方針とする。}
%%%%%%%%%%%%%%%%%%%%%%%%%%%%%%%%%
\end{enumerate}



%%%%%%%%%%%%%%%%%%%%%%%%%%%%%%%%%%%%%%%%%%%%%%%%%%%%%%%%%%
%% subsection C.2.2 %%%%%%%%%%%%%%%%%%%%%%%%%%%%%%%%%%%%%%
%%%%%%%%%%%%%%%%%%%%%%%%%%%%%%%%%%%%%%%%%%%%%%%%%%%%%%%%%%
\subsection{サブプログラム}
\begin{enumerate}
\item 測定(内面溝・逃し溝を除く)に関するものは6桁目を1とする
\item 測定(内面溝)に関するものは6桁目を2とする
\item 測定(逃し溝)の測定に関するものは6桁目を3とする
\item 加工(内面溝・逃し溝を除く)に関するものは6桁目を5とする
\item 加工(内面溝)に関するものは6桁目を6とする
\item 加工(逃し溝)に関するものは6桁目を7とする
\item その他、計測・加工に直接関しないものは6桁目を9とする
\item 計測・加工の両方に同じもの用いるものは番号の若いほうに合わせる
\end{enumerate}



%%%%%%%%%%%%%%%%%%%%%%%%%%%%%%%%%%%%%%%%%%%%%%%%%%%%%%%%%%
%% section C.3 %%%%%%%%%%%%%%%%%%%%%%%%%%%%%%%%%%%%%%%%%%%
%%%%%%%%%%%%%%%%%%%%%%%%%%%%%%%%%%%%%%%%%%%%%%%%%%%%%%%%%%
\section{工具の速さ(Fコード値)}
\DMname では、全長の長いタッチセンサーを用いる。
したがって、速さを大きくして移動をすると、その慣性によってセンサーが反応してしまったり、タッチセンサーそのものに大きな負担がかかる。
そのため、タッチセンサーの速さに関しては他の工具よりF値を低めに設定するものとする。



%%%%%%%%%%%%%%%%%%%%%%%%%%%%%%%%%%%%%%%%%%%%%%%%%%%%%%%%%%
%% subsection C.3.1 %%%%%%%%%%%%%%%%%%%%%%%%%%%%%%%%%%%%%%
%%%%%%%%%%%%%%%%%%%%%%%%%%%%%%%%%%%%%%%%%%%%%%%%%%%%%%%%%%
\subsection{タッチセンサー}
\begin{enumerate}
\item 原則として、G00は使用しない
\item G01を位置決め(早送り)として用いるものとし、速さはF5400以下とする
\item ワークへのアプローチの際は、G31を用いるものとし、速さはF1500以下とする
\item 計測の際のスキップ(G31)の速さは、計測の仕方に応じて以下のものとする
  \begin{enumerate}
  \item 信号遅れ補正を考慮する必要があるような場合は、速さはF50とする
  \item 信号遅れ補正を考慮する必要がない場合は、速さはF50以上300以下とする
  \end{enumerate}
\item 測定直後、ワークから離れる際は、G01を用いて、速さはF3600以下とする
\end{enumerate}




%%%%%%%%%%%%%%%%%%%%%%%%%%%%%%%%%%%%%%%%%%%%%%%%%%%%%%%%%%
%% subsection C.3.2 %%%%%%%%%%%%%%%%%%%%%%%%%%%%%%%%%%%%%%
%%%%%%%%%%%%%%%%%%%%%%%%%%%%%%%%%%%%%%%%%%%%%%%%%%%%%%%%%%
\subsection{タッチセンサー以外の工具}
\begin{enumerate}
\item G00(位置決め・早送り)は、速さはF10800以下とする
\item ワークへのアプローチの際は、G01を用いるものとし、速さはF5400以下とする
\item 加工の際は、それぞれの加工に応じた適切な速さ値を設定する
\end{enumerate}
