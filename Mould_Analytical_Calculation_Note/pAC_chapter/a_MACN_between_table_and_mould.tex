%!TEX root = ../Mould_Analytical_Calculation_Note.tex



\index{CAD}CADによる描画において、テーブルの回転中心が原点(\index{ワールドげんてん@ワールド原点}ワールド原点)に置かれているとする。
ここでモールドを描画する際、\index{もーるどのちゅうしん@モールドの中心}モールドの中心
%% footnote %%%%%%%%%%%%%%%%%%%%%
\footnote{$R_\mathrm c$に相当する点。}\relax
%%%%%%%%%%%%%%%%%%%%%%%%%%%%%%%%%
をCAD上の原点(ワールド原点)にして描くほうが都合のいいことがある。
このとき、モールドと\index{うけいた@受板}受板が接するように移動する必要がある。
鉛直方向(トップ-ボトム方向)においては$f_d$だけ動かせばよいが、水平方向の移動距離はあまり自明とはいいがたい。
\index{Cがわがいめん@C側外面}C側モールド面と受板面との寸法を単純に測ると、(水平方向でなく)最短距離が測定されてしまう。
工夫により水平方向の距離を出すことも可能ではあるが、ここではその距離を定量的に求めておく。



%%%%%%%%%%%%%%%%%%%%%%%%%%%%%%%%%%%%%%%%%%%%%%%%%%%%%%%%%%
%% section A.1 %%%%%%%%%%%%%%%%%%%%%%%%%%%%%%%%%%%%%%%%%%%
%%%%%%%%%%%%%%%%%%%%%%%%%%%%%%%%%%%%%%%%%%%%%%%%%%%%%%%%%%
\modHeadsection{スペーサ取付前}
(スペーサを取付る前の)モールドの中心が\index{テーブルちゅうしん@テーブル中心}テーブル中心Pに置かれている場合を考える。
ボトム側の受板に接するモールドの点と、テーブル中心Pとは、実軸方向に
\begin{align*}
  R_\mathrm c-R_\mathrm i\cos\alpha_{\mathrm U_\mathrm B}
\end{align*}
だけ差がある。
したがって、モールドの受板と接する点の位置は実軸方向に、
\begin{align*}
  \varDelta+\sqrt{R_\mathrm i'^2-\bar l^2}-R_\mathrm c+R_\mathrm i\cos\alpha_{\mathrm U_\mathrm B}\ .
\end{align*}
そのため\pageeqref{eq:afterUBcontact} ($\delta = 0$)より、モールドと(ボトム側の)受板は
\begin{align*}
  \varDelta+\sqrt{R_\mathrm i'^2-\bar l^2}-R_\mathrm c
\end{align*}
だけ実軸方向に離れていることがわかる。
\pageeqref{eq:tableCenter}より、これはテーブル中心Pとモールドの中心湾曲$R_\mathrm c$との差であることがわかる。



%%%%%%%%%%%%%%%%%%%%%%%%%%%%%%%%%%%%%%%%%%%%%%%%%%%%%%%%%%
%% section A.2 %%%%%%%%%%%%%%%%%%%%%%%%%%%%%%%%%%%%%%%%%%%
%%%%%%%%%%%%%%%%%%%%%%%%%%%%%%%%%%%%%%%%%%%%%%%%%%%%%%%%%%
\modHeadsection{スペーサ取付後}
\index{スペーサあつ@スペーサ厚}厚さ$\delta\,(>0)$のスペーサを取付けた場合、モールドの受板と接する点とテーブル中心Pとは、実軸方向に
\begin{align*}
  R_\mathrm c-R_\mathrm i\cos\alpha'_{\mathrm U_\mathrm B}
\end{align*}
だけ差があるので、その実軸方向の位置は、
\begin{align*}
  \varDelta+\sqrt{R_\mathrm i'^2-\bar l^2}-R_\mathrm c+R_\mathrm i\cos\alpha'_{\mathrm U_\mathrm B}\ .
\end{align*}
そのため\pageeqref{eq:afterUBcontact}より、モールドと(ボトム側の)受板は
\begin{align*}
  &  \varDelta+\sqrt{R_\mathrm i'^2-\bar l^2}-R_\mathrm c+R_\mathrm i\cos\alpha'_{\mathrm U_\mathrm B}
     -\left(R_\mathrm i'\cos\alpha_{\mathrm U_\mathrm B}+\rho\cos\alpha'_{\mathrm U_\mathrm B}\right)\\
  &= \varDelta-R_\mathrm c+R_\mathrm i'\cos\alpha'_{\mathrm U_\mathrm B}\\
  &= \varDelta-R_\mathrm c
     -\frac\delta2+\sqrt{R_\mathrm i'^2-\frac{\delta^2+(2\bar l)^2}4}\frac{2\bar l}{\sqrt{\delta^2+(2\bar l)^2}}
\end{align*}
だけ実軸方向に離れていることがわかる。
\pageeqref{eq:tableCenter}および\pageeqref{eq:spacerMoveHdistance}より、これはスペーサ取付け後のモールド中心とモールドの中心湾曲$R_\mathrm c$との差であることがわかる。
