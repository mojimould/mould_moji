%!TEX root = ../Mould_Analytical_Calculation_Note.tex



%%%%%%%%%%%%%%%%%%%%%%%%%%%%%%%%%%%%%%%%%%%%%%%%%%%%%%%%%%
%% section 10.1 %%%%%%%%%%%%%%%%%%%%%%%%%%%%%%%%%%%%%%%%%%
%%%%%%%%%%%%%%%%%%%%%%%%%%%%%%%%%%%%%%%%%%%%%%%%%%%%%%%%%%
\modHeadsection{タッチセンサー}
\DMname では、全長の長い\index{タッチセンサープローブ}タッチセンサープローブを用いる。
したがって、速さを大きくして移動をすると、その加速度によってセンサーが反応してしまったり、\index{タッチセンサー}タッチセンサーそのものに大きな負担がかかる。
そのため、タッチセンサーの速さに関しては他の工具より\index{Fコード@Fコード}Fコード値を低めに設定するものとする。
\begin{enumerate}[label=\Roman*., ref=\Roman*]
\item \label{item:FDTS} 原則として、\verb|G00|を用いた移動はしない
\item \ref{item:FDTS}に伴い、\verb|G28|, \verb|G30|を(直接的に)用いた移動はしない
\item \verb|G01|を位置決め(早送り)として用いるものとし、速さはF5400以下とする
\item ワークへの\index{アプローチ}アプローチの際は、\verb|G31|を用いるものとし、速さはF1500以下とする
\item 計測の際の\index{スキップ}スキップ(\verb|G31|)の速さは、計測の仕方に応じて以下のものとする
  \begin{enumerate}
  \item \index{しんごうおくれほせい@信号遅れ補正}信号遅れ補正を考慮する必要があるような場合は、速さはF50とする
  \item 信号遅れ補正を考慮する必要がない場合は、速さはF50以上300以下とする
  \end{enumerate}
\item 測定直後、ワークから離れる際は、\verb|G01|を用いる
\end{enumerate}



%%%%%%%%%%%%%%%%%%%%%%%%%%%%%%%%%%%%%%%%%%%%%%%%%%%%%%%%%%
%% section 10.2 %%%%%%%%%%%%%%%%%%%%%%%%%%%%%%%%%%%%%%%%%%
%%%%%%%%%%%%%%%%%%%%%%%%%%%%%%%%%%%%%%%%%%%%%%%%%%%%%%%%%%
\modHeadsection{タッチセンサー以外の工具}
\DMname では、$Z$方向についてはテーブルが移動する。
したがって、$Z$方向の速さを大きくして移動すると、その加速度によってテーブル上のジグが動いてしまう恐れがある。
そのため、$Z$方向の速さに関しては$X$, $Y$方向の速さよりFコード値を低めに設定するものとする。
\begin{enumerate}[label=\Roman*., ref=\Roman*]
\item \verb|G00|(位置決め・早送り)は、$X$, $Y$方向の速さはF10800以下、$Z$方向の速さはF7200以下とする
\item ワークへのアプローチの際は、\verb|G01|を用いるものとし、速さはF5400以下とする
\item 加工の際は、それぞれの加工や状況に応じて適切なFコード値を設定する
\end{enumerate}
