%!TEX root = ../Mould_Analytical_Calculation_Note.tex



一般に、\index{シーケンスばんごう@シーケンス番号}シーケンス番号は(重複していなければ)自由に付けて問題ない。
しかしこれに一定のルールを与えておくことで、プログラムの
\begin{enumerate}
\item どの部分で何が行われているか
\item どの部分でエラーが起きたか
\item 途中から稼働する場合、どの番号から始めればよいか
\end{enumerate}
など、作業や管理をする際に効率よく制御することが可能になる。

そこで、ここではシーケンス番号(\index{Nコード}Nコード)についての規則を与える。


%%%%%%%%%%%%%%%%%%%%%%%%%%%%%%%%%%%%%%%%%%%%%%%%%%%%%%%%%%
%% section 14.1 %%%%%%%%%%%%%%%%%%%%%%%%%%%%%%%%%%%%%%%%%%
%%%%%%%%%%%%%%%%%%%%%%%%%%%%%%%%%%%%%%%%%%%%%%%%%%%%%%%%%%
\modHeadsection{サブプログラム}
\DMname においては、原則としてサブプログラムは始めから実行されるものであり、途中の部分から実行されることはない。
そのため、サブプログラムについてはシーケンス番号は記述の順に(概ねブロックごとに)連番とする。


%%%%%%%%%%%%%%%%%%%%%%%%%%%%%%%%%%%%%%%%%%%%%%%%%%%%%%%%%%
%% section 14.2 %%%%%%%%%%%%%%%%%%%%%%%%%%%%%%%%%%%%%%%%%%
%%%%%%%%%%%%%%%%%%%%%%%%%%%%%%%%%%%%%%%%%%%%%%%%%%%%%%%%%%
\modHeadsection{メインプログラム}
\DMname において、\index{メインプログラム}メインプログラム
%% footnote %%%%%%%%%%%%%%%%%%%%%
\footnote{ここでいうメインプログラムとは、下5桁が図面番号と一致するものを指す。}
%%%%%%%%%%%%%%%%%%%%%%%%%%%%%%%%%
は作業者が実際に設定を変更したり、途中の箇所から始めたりし得るものである。
そのため、メインプログラムについては各作業(計測・加工)ごとにシーケンス番号を割り振ることにする。


%%%%%%%%%%%%%%%%%%%%%%%%%%%%%%%%%%%%%%%%%%%%%%%%%%%%%%%%%%
%% subsection 14.2.1 %%%%%%%%%%%%%%%%%%%%%%%%%%%%%%%%%%%%%
%%%%%%%%%%%%%%%%%%%%%%%%%%%%%%%%%%%%%%%%%%%%%%%%%%%%%%%%%%
\subsection{N001:全過程}
メインプログラムの始まりは、N001とする。
つまり、N001から始めると、そのメインプログラムの全ての過程が実行される。


%%%%%%%%%%%%%%%%%%%%%%%%%%%%%%%%%%%%%%%%%%%%%%%%%%%%%%%%%%
%% subsection 14.2.1 %%%%%%%%%%%%%%%%%%%%%%%%%%%%%%%%%%%%%
%%%%%%%%%%%%%%%%%%%%%%%%%%%%%%%%%%%%%%%%%%%%%%%%%%%%%%%%%%
\subsection{N100:計測(内面溝・逃し溝以外)}
タッチセンサーを用いた計測(内面溝・逃し溝を除く)を行う過程のシーケンス番号は100番台とする。
これには以下の過程が含まれ、これらは2桁目の番号で区別される。
\begin{enumerate}
\item[100:] 芯出し計測
\item[150:] 通り芯$X$計測
\item[160:] 通り芯$Y$計測
\end{enumerate}


%%%%%%%%%%%%%%%%%%%%%%%%%%%%%%%%%%%%%%%%%%%%%%%%%%%%%%%%%%
%% subsection 14.2.1 %%%%%%%%%%%%%%%%%%%%%%%%%%%%%%%%%%%%%
%%%%%%%%%%%%%%%%%%%%%%%%%%%%%%%%%%%%%%%%%%%%%%%%%%%%%%%%%%
\subsection{N200:計測(内面溝)}
内面溝および逃し溝に関するタッチセンサーを用いた計測を行う過程のシーケンス番号は200番台とする。
これには以下の過程が含まれ、これらは2桁目の番号で区別される。
\begin{enumerate}
\item[200:] 内面溝計測
\item[250:] 逃し溝計測
\end{enumerate}


%%%%%%%%%%%%%%%%%%%%%%%%%%%%%%%%%%%%%%%%%%%%%%%%%%%%%%%%%%
%% subsection 14.2.1 %%%%%%%%%%%%%%%%%%%%%%%%%%%%%%%%%%%%%
%%%%%%%%%%%%%%%%%%%%%%%%%%%%%%%%%%%%%%%%%%%%%%%%%%%%%%%%%%
\subsection{N300:内面溝・逃し溝加工}
内面溝および逃し溝加工を行う過程のシーケンス番号は300番台とする。
これには以下の過程が含まれ、これらは2桁目の番号で区別される。
\begin{enumerate}
\item[300:] 内面溝加工
\item[350:] 逃し溝加工
\end{enumerate}


%%%%%%%%%%%%%%%%%%%%%%%%%%%%%%%%%%%%%%%%%%%%%%%%%%%%%%%%%%
%% subsection 14.2.1 %%%%%%%%%%%%%%%%%%%%%%%%%%%%%%%%%%%%%
%%%%%%%%%%%%%%%%%%%%%%%%%%%%%%%%%%%%%%%%%%%%%%%%%%%%%%%%%%
\subsection{N400:トップ側の加工}
トップ側の加工を行う過程のシーケンス番号は400番台とする。
これには以下の過程が含まれ、これらは2桁目の番号で区別される。
\begin{enumerate}
\item[400:] トップ端面加工
\item[410:] トップ外削加工
\item[420:] 溝加工
\item[430:] トップ外面取加工
\item[440:] トップ内面取加工
\item[450:] 座ぐり加工
\end{enumerate}


%%%%%%%%%%%%%%%%%%%%%%%%%%%%%%%%%%%%%%%%%%%%%%%%%%%%%%%%%%
%% subsection 14.2.1 %%%%%%%%%%%%%%%%%%%%%%%%%%%%%%%%%%%%%
%%%%%%%%%%%%%%%%%%%%%%%%%%%%%%%%%%%%%%%%%%%%%%%%%%%%%%%%%%
\subsection{N500:ボトム側の加工}
ボトム側の加工を行う過程のシーケンス番号は500番台とする。
これには以下の過程が含まれ、これらは2桁目の番号で区別される。
\begin{enumerate}
\item[500:] ボトム端面加工
\item[510:] ボトム外削加工
\item[520:] ボトム外面取加工
\item[530:] ボトム内面取加工
\end{enumerate}


%%%%%%%%%%%%%%%%%%%%%%%%%%%%%%%%%%%%%%%%%%%%%%%%%%%%%%%%%%
%% subsection 14.2.1 %%%%%%%%%%%%%%%%%%%%%%%%%%%%%%%%%%%%%
%%%%%%%%%%%%%%%%%%%%%%%%%%%%%%%%%%%%%%%%%%%%%%%%%%%%%%%%%%
\subsection{N800:エラー}
エラー検出時にジャンプするシーケンス番号は800番台とする。
これには以下の過程が含まれ、これらは2桁目の番号で区別される。
\begin{enumerate}
\item[800:] 入力変数(引数)そのものによる誤り
\item[810:] 入力変数(引数)の計算による誤り
\item[820:] パレットまたはタッチセンサーによるエラー
\end{enumerate}


%%%%%%%%%%%%%%%%%%%%%%%%%%%%%%%%%%%%%%%%%%%%%%%%%%%%%%%%%%
%% subsection 14.2.1 %%%%%%%%%%%%%%%%%%%%%%%%%%%%%%%%%%%%%
%%%%%%%%%%%%%%%%%%%%%%%%%%%%%%%%%%%%%%%%%%%%%%%%%%%%%%%%%%
\subsection{その他}
プログラム終了の過程のシーケンス番号は990番台とする。
特に、プログラムの終了はN999とする。\\


\pagebreak\noindent
改めて上記のシーケンス番号を一覧にしておく。\\
%%%%%%%%%%%%%%%%%%%%%%%%%%%%%%%%%%%%%%%%%%%%%%%%%%%%%%%%%%
%% sequence numbers %%%%%%%%%%%%%%%%%%%%%%%%%%%%%%%%%%%%%%
%%%%%%%%%%%%%%%%%%%%%%%%%%%%%%%%%%%%%%%%%%%%%%%%%%%%%%%%%%
\addtocontents{lot}{\protect\addvspace{3pt}}{}{}
\addcontentsline{lot}{section}{\numberline{\thesection}\Sectionname}
\modcaptionof{table}{シーケンス番号 一覧}
\begin{twoCtable}{}
N001 & プログラムの始まり\\\hline
\hline
N10x & タッチセンサー計測(芯出し)\\\hline
N15x & タッチセンサー計測(通り芯$X$)\\\hline
N16x & タッチセンサー計測(通り芯$Y$)\\\hline
\hline
N20x & タッチセンサー計測(内面溝)\\\hline
N25x & タッチセンサー計測(逃し溝)\\\hline
\hline
N30x & 内面溝加工\\\hline
N35x & 逃し溝加工\\\hline
\hline
N40x & トップ端面加工\\\hline
N41x & トップ外削加工\\\hline
N42x & 溝加工\\\hline
N43x & トップ外面取加工\\\hline
N44x & トップ内面取加工\\\hline
N45x & 座ぐり加工\\\hline
\hline
N500 & ボトム端面加工\\\hline
N51x & ボトム外削加工\\\hline
N52x & ボトム外面取加工\\\hline
N53x & ボトム内面取加工\\\hline
\hline
N80x & 入力変数(引数)そのものによる誤り\\\hline
N81x & 入力変数(引数)の計算による誤り\\\hline
N82x & パレットまたはタッチセンサーによるエラー\\\hline
\hline
N99x & プログラム終了の過程\\\hline
N999 & プログラム終了(\verb|M30|または\verb|M99|)
\end{twoCtable}


