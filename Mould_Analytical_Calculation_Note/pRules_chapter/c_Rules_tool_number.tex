%!TEX root = ../Mould_Analytical_Calculation_Note.tex



ここでは\DMname で使用する工具番号について記述する。
\modHeadsection{基本事項}
\begin{enumerate}[label=\Roman*), ref=\Roman*)]
\item 工具番号01は空とする
\item 工具番号02-05は端面加工用工具(フェイスミル)とする
\item 工具番号06-09は溝加工用工具(サイドカッター)とする
\item 工具番号11-15は面取加工用工具(テーパーエンドミル)とする
\item 工具番号16-20は外削加工用工具(スクエアエンドミル)とする
\item 工具番号31-35は内面溝加工用工具(Tスロットカッター)とする
\item 工具番号36-40は内面溝加工用工具(アングルヘッド)とする
\item 工具番号41-45は逃し溝加工用工具(アングルヘッド)とする
\item 工具番号49, 50は測定用工具(タッチセンサープローブ)とする
\end{enumerate}



%%%%%%%%%%%%%%%%%%%%%%%%%%%%%%%%%%%%%%%%%%%%%%%%%%%%%%%%%%
%% section 13.1 %%%%%%%%%%%%%%%%%%%%%%%%%%%%%%%%%%%%%%%%%%
%%%%%%%%%%%%%%%%%%%%%%%%%%%%%%%%%%%%%%%%%%%%%%%%%%%%%%%%%%
\modHeadsection{登録工具}
\addtocontents{lot}{\protect\addvspace{3pt}}{}{}
\addcontentsline{lot}{section}{\numberline{\thesection}\Sectionname}
\customtodayap 時点において登録されている工具は以下の通りである\\
%%%%%%%%%%%%%%%%%%%%%%%%%%%%%%%%%%%%%%%%%%%%%%%%%%%%%%%%%%
%% common variables %%%%%%%%%%%%%%%%%%%%%%%%%%%%%%%%%%%%%%
%%%%%%%%%%%%%%%%%%%%%%%%%%%%%%%%%%%%%%%%%%%%%%%%%%%%%%%%%%
\modcaptionof{table}{登録工具}
\begin{twoCtable}{}
\verb|T01| & 空\\\hline
\hline
\verb|T02| & $\phi100$端面加工用フェイスミル\\\hline
\hline
\verb|T06| & $\phi100\times t7$溝加工用サイドカッター\\\hline
\verb|T07| & $\phi100\times t5$溝加工用サイドカッター\\\hline
\hline
\verb|T11| & $\phi4.0\times 15^\circ$面取加工用テーパーエンドミル\\\hline
\verb|T12| & $\phi0.8\times 30^\circ$面取加工用テーパーエンドミル\\\hline
\verb|T13| & $\phi2.0\times 45^\circ$面取加工用テーパーエンドミル\\\hline
\hline
\verb|T16| & $\phi20$外削加工用スクエアエンドミル\\\hline
\hline
\verb|T31| & $\phi40\times R20$内面溝加工用Tスロットカッター\\\hline
\verb|T32| & $\phi40\times R6.6$内面溝加工用Tスロットカッター\\\hline
\verb|T33| & $\phi30\times R15$内面溝加工用Tスロットカッター\\\hline
\hline
\verb|T36| & 10.0in内面溝加工用アングルヘッド\\\hline
\hline
\verb|T41| & 15.5in逃し溝加工用アングルヘッド\\\hline
\hline
\verb|T50| & $\phi10\times200$(延長100)測定用タッチセンサープローブ
\end{twoCtable}
