%!TEX root = ../Mould_Analytical_Calculation_Note.tex

\DMname で取り決めている\index{コモンへんすう@コモン変数}コモン変数について、以下に挙げる。



%%%%%%%%%%%%%%%%%%%%%%%%%%%%%%%%%%%%%%%%%%%%%%%%%%%%%%%%%%
%% section 15.1 %%%%%%%%%%%%%%%%%%%%%%%%%%%%%%%%%%%%%%%%%%
%%%%%%%%%%%%%%%%%%%%%%%%%%%%%%%%%%%%%%%%%%%%%%%%%%%%%%%%%%
\modHeadsection{コモン変数 (\ttNum100\,-\ttNum199)}
\addtocontents{lot}{\protect\addvspace{3pt}}{}{}
\addcontentsline{lot}{section}{\numberline{\thesection}\Sectionname}
\ttNum100\,-\ttNum199については、(機械設置時の)\index{バンドルのプログラム}バンドルのプログラムで既に使用されているものが多いため、基本的には(\index{RHS}RHSとしては)使用しないものとする。
使用する場合は、一時的なもの(\index{LHS}LHS)として扱うものとする。



\clearpage
%%%%%%%%%%%%%%%%%%%%%%%%%%%%%%%%%%%%%%%%%%%%%%%%%%%%%%%%%%
%% section 15.2 %%%%%%%%%%%%%%%%%%%%%%%%%%%%%%%%%%%%%%%%%%
%%%%%%%%%%%%%%%%%%%%%%%%%%%%%%%%%%%%%%%%%%%%%%%%%%%%%%%%%%
\modHeadsection{コモン変数 (\ttNum400\,-\ttNum499)}
\addtocontents{lot}{\protect\addvspace{3pt}}{}{}
\addcontentsline{lot}{section}{\numberline{\thesection}\Sectionname}
%%%%%%%%%%%%%%%%%%%%%%%%%%%%%%%%%%%%%%%%%%%%%%%%%%%%%%%%%%
%% common variables %%%%%%%%%%%%%%%%%%%%%%%%%%%%%%%%%%%%%%
%%%%%%%%%%%%%%%%%%%%%%%%%%%%%%%%%%%%%%%%%%%%%%%%%%%%%%%%%%
\ttNum400\,-\ttNum474については、主に作業者が入力・変更することが想定されるものとする。\\

\modcaptionof{table}{コモン変数 (\ttNum400\,-\ttNum449)}
\begin{twoCtable}{}
\ttNum400 & 計測・加工 開始N番号\\\hline
\hline
\ttNum401 & (予備)\\\hline
\hline
\ttNum402 & トップ端面 全削り代\\\hline
\ttNum403 & ボトム端面 全削り代\\\hline
\ttNum404 & (予備)\\\hline
\ttNum405 & 通り芯測定(0:off, 1:on)\\\hline
\ttNum406 & ボトム外削 A面肉厚$+$補正(\verb|G54|ワーク座標系補正)\\\hline
\ttNum407 & ボトム外削 仕上げ前stop (0:non-stop, 1:外削径1mm残し\verb|M00|)\\\hline
\ttNum408 & ボトム外削 仕上げ加工回数 (上限5)\\\hline
\ttNum409 & トップ外削 A面肉厚$+$補正(\verb|G56|ワーク座標系補正)\\\hline
\ttNum410 & トップ外削 仕上げ前stop (0:non-stop, 1:外削径1mm残し\verb|M00|)\\\hline
\ttNum411 & トップ外削 仕上げ加工回数 (上限5)\\\hline
\ttNum412 & (予備)\\\hline
\ttNum413 & 溝位置$+$補正\\\hline
\ttNum414 & 溝幅$+$補正\\\hline
\ttNum415 & 溝A面深さ$+$補正(\verb|G56|ワーク座標系補正)\\\hline
\ttNum416 & 溝 仕上げ前stop (0:non-stop, 1:溝径1mm残し\verb|M00|)\\\hline
\ttNum417 & 溝 仕上げ加工回数 (上限5)\\\hline
\ttNum418 & (予備)\\\hline
\ttNum419 & トップ外面取$X+$補正(\verb|G56|ワーク座標系補正)\\\hline
\ttNum420 & トップ外面取 仕上げ前stop (0:non-stop, 1:$Z+1$mm手前\verb|M00|)\\\hline
\ttNum421 & トップ外面取 仕上げ加工回数 (上限5)\\\hline
\ttNum422 & (予備)\\\hline
\ttNum423 & トップ内面取$X+$補正(\verb|G57|ワーク座標系補正)\\\hline
\ttNum424 & トップ内面取 仕上げ前stop (0:non-stop, 1:$Z+1$mm手前\verb|M00|)\\\hline
\ttNum425 & トップ内面取 仕上げ加工回数 (上限5)\\\hline
\ttNum426 & (予備)\\\hline
\ttNum427 & ボトム外面取$X+$補正(\verb|G54|ワーク座標系補正)\\\hline
\ttNum428 & ボトム外面取 仕上げ前stop (0:non-stop, 1:$Z+1$mm手前\verb|M00|)\\\hline
\ttNum429 & ボトム外面取 仕上げ加工回数 (上限5)\\\hline
\ttNum430 & (予備)\\\hline
\ttNum431 & ボトム内面取$X+$補正(\verb|G55|ワーク座標系補正)\\\hline
\ttNum432 & ボトム内面取 仕上げ前stop (0:non-stop, 1:$Z+1$mm手前\verb|M00|)\\\hline
\ttNum433 & ボトム内面取 仕上げ加工回数 (上限5)\\\hline
& (以下予備)
\end{twoCtable}


%\clearpage
%%%%%%%%%%%%%%%%%%%%%%%%%%%%%%%%%%%%%%%%%%%%%%%%%%%%%%%%%%
%% common variables %%%%%%%%%%%%%%%%%%%%%%%%%%%%%%%%%%%%%%
%%%%%%%%%%%%%%%%%%%%%%%%%%%%%%%%%%%%%%%%%%%%%%%%%%%%%%%%%%
\modcaptionof{table}{コモン変数 (\ttNum450\,-\ttNum474)}
\begin{twoCtable}{}
\ttNum450 & (予備)\\\hline
\ttNum451 & 工具\verb|T31|(Tスロット)A側内面溝 深さ補正値(深さに$+$補正)\\\hline
\ttNum452 & 工具\verb|T31|(Tスロット)C側内面溝 深さ補正値(深さに$+$補正)\\\hline
\ttNum453 & 工具\verb|T31|(Tスロット)B側内面溝 深さ補正値(深さに$+$補正)\\\hline
\ttNum454 & 工具\verb|T31|(Tスロット)D側内面溝 深さ補正値(深さに$+$補正)\\\hline
\ttNum455 & (予備)\\\hline
\hline
\ttNum456 & 工具\verb|T32|(Tスロット)A側内面溝 深さ補正値(深さに$+$補正)\\\hline
\ttNum457 & 工具\verb|T32|(Tスロット)C側内面溝 深さ補正値(深さに$+$補正)\\\hline
\ttNum458 & 工具\verb|T32|(Tスロット)B側内面溝 深さ補正値(深さに$+$補正)\\\hline
\ttNum459 & 工具\verb|T32|(Tスロット)D側内面溝 深さ補正値(深さに$+$補正)\\\hline
\ttNum460 & (予備)\\\hline
\hline
\ttNum461 & 工具\verb|T33|(Tスロット)A側内面溝 深さ補正値(深さに$+$補正)\\\hline
\ttNum462 & 工具\verb|T33|(Tスロット)C側内面溝 深さ補正値(深さに$+$補正)\\\hline
\ttNum463 & 工具\verb|T33|(Tスロット)B側内面溝 深さ補正値(深さに$+$補正)\\\hline
\ttNum464 & 工具\verb|T33|(Tスロット)D側内面溝 深さ補正値(深さに$+$補正)\\\hline
& (以下予備)
\end{twoCtable}


\ttNum475\,-\ttNum499については、その他の調整に用いるものとする。\\
%%%%%%%%%%%%%%%%%%%%%%%%%%%%%%%%%%%%%%%%%%%%%%%%%%%%%%%%%%
%% common variables %%%%%%%%%%%%%%%%%%%%%%%%%%%%%%%%%%%%%%
%%%%%%%%%%%%%%%%%%%%%%%%%%%%%%%%%%%%%%%%%%%%%%%%%%%%%%%%%%
\modcaptionof{table}{コモン変数 (\ttNum475\,-\ttNum499)}
\begin{twoCtable}{}
\ttNum475 & (予備)\\\hline
\ttNum491 & 端面$Z$方向クリアランス平面間距離\\\hline
\ttNum492 & 外側加工面 法線方向クリアランス平面間距離 最小値\\\hline
\ttNum493 & 内側加工面 法線方向クリアランス平面間距離 最小値\\\hline
\ttNum494 & (予備)\\\hline
\ttNum495 & 端面加工 1回あたりの$Z$方向削り代\\\hline
\ttNum496 & 端面 切削回数 (FUP[\ttNum402\,/\ttNum495] or FUP[\ttNum403\,/\ttNum495])\\\hline
\ttNum497 & (予備)\\\hline
\ttNum498 & (予備)\\\hline
\ttNum499 & (予備)
\end{twoCtable}
%%%%%%%%%%%%%%%%%%%%%%%%%%%%%%%%%%%%%%%%%%%%%%%%%%%%%%%%%%
%% hosoku %%%%%%%%%%%%%%%%%%%%%%%%%%%%%%%%%%%%%%%%%%%%%%%%
%%%%%%%%%%%%%%%%%%%%%%%%%%%%%%%%%%%%%%%%%%%%%%%%%%%%%%%%%%
\begin{hosoku}
2023/12/28時点における設定値は、
\begin{align*}
  \ttNum491 = 100.0~, \quad \ttNum492 = 30.0~, \quad \ttNum493 = 15.0
\end{align*}
\end{hosoku}
%%%%%%%%%%%%%%%%%%%%%%%%%%%%%%%%%%%%%%%%%%%%%%%%%%%%%%%%%%
%%%%%%%%%%%%%%%%%%%%%%%%%%%%%%%%%%%%%%%%%%%%%%%%%%%%%%%%%%
%%%%%%%%%%%%%%%%%%%%%%%%%%%%%%%%%%%%%%%%%%%%%%%%%%%%%%%%%%



\clearpage
%%%%%%%%%%%%%%%%%%%%%%%%%%%%%%%%%%%%%%%%%%%%%%%%%%%%%%%%%%
%% section 13.1 %%%%%%%%%%%%%%%%%%%%%%%%%%%%%%%%%%%%%%%%%%
%%%%%%%%%%%%%%%%%%%%%%%%%%%%%%%%%%%%%%%%%%%%%%%%%%%%%%%%%%
\modHeadsection{コモン変数 (\ttNum500\,-\ttNum599)}
\addtocontents{lot}{\protect\addvspace{3pt}}{}{}
\addcontentsline{lot}{section}{\numberline{\thesection}\Sectionname}
\ttNum500\,-\ttNum599については、主にO910xおよびO93xxで使用されるものとする。\\
%%%%%%%%%%%%%%%%%%%%%%%%%%%%%%%%%%%%%%%%%%%%%%%%%%%%%%%%%%
%% common variables %%%%%%%%%%%%%%%%%%%%%%%%%%%%%%%%%%%%%%
%%%%%%%%%%%%%%%%%%%%%%%%%%%%%%%%%%%%%%%%%%%%%%%%%%%%%%%%%%
\modcaptionof{table}{コモン変数 (\ttNum500\,-\ttNum524)}
\begin{twoCtable}{}
\ttNum500 & 芯ずれ許容差 (O93xx)\\\hline
\ttNum501 & タッチセンサー信号遅れ補正 (O93xx)\\\hline
\ttNum502 & タッチセンサープローブ中心$X$補正 (O93xx)\\\hline
\ttNum503 & タッチセンサープローブ中心$Y$補正 (O93xx)\\\hline
\ttNum504 & 測定距離 (O910x)\\\hline
\ttNum505 & プローブ表面からプログラムの加工原点($Z$0)までの距離 (O910x)\\\hline
\ttNum506 & 工具長の変化の許容差 (O910x)\\\hline
\ttNum507 & 工具破損検出の許容差 (O910x)\\\hline
\ttNum509 & Z座標系設定 (O93xx)\\\hline
\ttNum511 & インチ/ミリ切替 (O910x)\\\hline
\ttNum512 & タッチセンサープローブ半径$\mathrm{mm}$値 (O93xx)\\\hline
\ttNum513 & 移動時用の送り速さ値 (O910x)\\\hline
\ttNum514 & スキップ(\verb|G31|)測定時用 送り速さ値 (O910x, O93xx)\\\hline
\ttNum516 & センサーの位置$X$座標 (O910x)\\\hline
\ttNum517 & センサーの位置$Y$座標 (O910x)\\\hline
\ttNum518 & センサーの位置$Z$座標 (O910x)\\\hline
\ttNum520 & 拡張ワーク座標系 (O910x)\\\hline
\ttNum523 & アプローチ時用の送り速さ値 (O910x)\\\hline
\ttNum524 & 測定時用の送り速さ値 (O910x)
\end{twoCtable}
%%%%%%%%%%%%%%%%%%%%%%%%%%%%%%%%%%%%%%%%%%%%%%%%%%%%%%%%%%
%% hosoku %%%%%%%%%%%%%%%%%%%%%%%%%%%%%%%%%%%%%%%%%%%%%%%%
%%%%%%%%%%%%%%%%%%%%%%%%%%%%%%%%%%%%%%%%%%%%%%%%%%%%%%%%%%
\begin{hosoku}
2023/12/28時点における設定値は、
\begin{align*}
  \ttNum512 = 10.0~, \quad \ttNum514 = 50.0
\end{align*}
\end{hosoku}
%%%%%%%%%%%%%%%%%%%%%%%%%%%%%%%%%%%%%%%%%%%%%%%%%%%%%%%%%%
%%%%%%%%%%%%%%%%%%%%%%%%%%%%%%%%%%%%%%%%%%%%%%%%%%%%%%%%%%
%%%%%%%%%%%%%%%%%%%%%%%%%%%%%%%%%%%%%%%%%%%%%%%%%%%%%%%%%%



\clearpage
%%%%%%%%%%%%%%%%%%%%%%%%%%%%%%%%%%%%%%%%%%%%%%%%%%%%%%%%%%
%% section 13.1 %%%%%%%%%%%%%%%%%%%%%%%%%%%%%%%%%%%%%%%%%%
%%%%%%%%%%%%%%%%%%%%%%%%%%%%%%%%%%%%%%%%%%%%%%%%%%%%%%%%%%
\modHeadsection{コモン変数 (\ttNum600\,-\ttNum699)}
\addtocontents{lot}{\protect\addvspace{3pt}}{}{}
\addcontentsline{lot}{section}{\numberline{\thesection}\Sectionname}
\ttNum600\,-\ttNum674については、主に\index{こうぐ@工具}工具に関するものとする。\\
%%%%%%%%%%%%%%%%%%%%%%%%%%%%%%%%%%%%%%%%%%%%%%%%%%%%%%%%%%
%% common variables %%%%%%%%%%%%%%%%%%%%%%%%%%%%%%%%%%%%%%
%%%%%%%%%%%%%%%%%%%%%%%%%%%%%%%%%%%%%%%%%%%%%%%%%%%%%%%%%%
\modcaptionof{table}{コモン変数 (\ttNum600\,-\ttNum674)}
\begin{twoCtable}{}
\ttNum600 & (予備)\\\hline
\hline
\ttNum601 & 工具\verb|T02|(フェイスミル)最大刃径(直径)DCX公称値$\phi'_\mathrm D$\\\hline
\ttNum602 & (工具\verb|T02|用予備)\\\hline
\ttNum603 & (工具\verb|T03|フェイスミル用予備)\\\hline
\ttNum604 & (工具\verb|T03|フェイスミル用予備)\\\hline
\hline
\ttNum605 & 工具\verb|T06|(サイドカッター)厚さ$t$\\\hline
\ttNum606 & (工具\verb|T06|用予備)\\\hline
\ttNum607 & (工具\verb|T07|サイドカッター用予備)\\\hline
\ttNum608 & (工具\verb|T07|サイドカッター用予備)\\\hline
\ttNum609 & 工具\verb|T08|(サイドカッター)厚さ$t$\\\hline
\ttNum610 & (工具\verb|T08|用予備)\\\hline
\ttNum611 & (工具\verb|T09|サイドカッター用予備)\\\hline
\ttNum612 & (工具\verb|T09|サイドカッター用予備)\\\hline
\hline
\ttNum613 & (工具\verb|T11|テーパエンドミル用予備)\\\hline
\ttNum614 & (工具\verb|T11|テーパエンドミル用予備)\\\hline
\ttNum615 & (工具\verb|T12|テーパエンドミル用予備)\\\hline
\ttNum616 & (工具\verb|T12|テーパエンドミル用予備)\\\hline
\ttNum617 & (工具\verb|T13|テーパエンドミル用予備)\\\hline
\ttNum618 & (工具\verb|T13|テーパエンドミル用予備)\\\hline
\ttNum619 & (工具\verb|T14|テーパエンドミル用予備)\\\hline
\ttNum620 & (工具\verb|T14|テーパエンドミル用予備)\\\hline
\hline
\ttNum621 & (工具\verb|T16|スクエアエンドミル用予備)\\\hline
\ttNum622 & (工具\verb|T16|スクエアエンドミル用予備)\\\hline
\ttNum623 & (工具\verb|T17|スクエアエンドミル用予備)\\\hline
\ttNum624 & (工具\verb|T17|スクエアエンドミル用予備)\\\hline
\ttNum625 & (工具\verb|T18|スクエアエンドミル用予備)\\\hline
\ttNum626 & (工具\verb|T18|スクエアエンドミル用予備)\\\hline
\hline
\ttNum627 & (工具\verb|T31| Tスロットカッター用予備)\\\hline
\ttNum628 & (工具\verb|T31| Tスロットカッター用予備)\\\hline
\ttNum629 & (工具\verb|T32| Tスロットカッター用予備)\\\hline
\ttNum630 & (工具\verb|T32| Tスロットカッター用予備)\\\hline
\ttNum631 & (工具\verb|T33| Tスロットカッター用予備)\\\hline
\ttNum632 & (工具\verb|T33| Tスロットカッター用予備)\\\hline
\ttNum633 & (工具\verb|T34| Tスロットカッター用予備)\\\hline
\ttNum634 & (工具\verb|T34| Tスロットカッター用予備)\\\hline
& (以下予備)
\end{twoCtable}
%%%%%%%%%%%%%%%%%%%%%%%%%%%%%%%%%%%%%%%%%%%%%%%%%%%%%%%%%%
%% hosoku %%%%%%%%%%%%%%%%%%%%%%%%%%%%%%%%%%%%%%%%%%%%%%%%
%%%%%%%%%%%%%%%%%%%%%%%%%%%%%%%%%%%%%%%%%%%%%%%%%%%%%%%%%%
\begin{hosoku}
2023/12/28時点における設定値は、
\begin{align*}
  \ttNum601 = 113.5~, \quad \ttNum605 = 7.0~, \quad \ttNum609 = 5.0
\end{align*}
\end{hosoku}\\
%%%%%%%%%%%%%%%%%%%%%%%%%%%%%%%%%%%%%%%%%%%%%%%%%%%%%%%%%%
%%%%%%%%%%%%%%%%%%%%%%%%%%%%%%%%%%%%%%%%%%%%%%%%%%%%%%%%%%
%%%%%%%%%%%%%%%%%%%%%%%%%%%%%%%%%%%%%%%%%%%%%%%%%%%%%%%%%%


%\clearpage
\ttNum675\,-\ttNum699については、主に各々の製品明細に固有のものとする。\\
%%%%%%%%%%%%%%%%%%%%%%%%%%%%%%%%%%%%%%%%%%%%%%%%%%%%%%%%%%
%% common variables %%%%%%%%%%%%%%%%%%%%%%%%%%%%%%%%%%%%%%
%%%%%%%%%%%%%%%%%%%%%%%%%%%%%%%%%%%%%%%%%%%%%%%%%%%%%%%%%%
\modcaptionof{table}{コモン変数 (\ttNum675\,-\ttNum699)}
\begin{twoCtable}{}
\ttNum675 & 中心湾曲$R_\mathrm c$ ($\nicefrac1{R_\mathrm c} = 0$の場合は\ttNum0)\\\hline
\ttNum676 & 振分長調整用 傾け角度$-\theta[\deg]$\\\hline
\ttNum677 & 内面溝用 傾け角度$-\phi[\deg]$\\\hline
\ttNum678 & 内面めっき膜厚$\mu$\\\hline
& (以下予備)
\end{twoCtable}


\clearpage
%%%%%%%%%%%%%%%%%%%%%%%%%%%%%%%%%%%%%%%%%%%%%%%%%%%%%%%%%%
%% section 13.1 %%%%%%%%%%%%%%%%%%%%%%%%%%%%%%%%%%%%%%%%%%
%%%%%%%%%%%%%%%%%%%%%%%%%%%%%%%%%%%%%%%%%%%%%%%%%%%%%%%%%%
\modHeadsection{コモン変数 (\ttNum700\,-\ttNum799)}
\addtocontents{lot}{\protect\addvspace{3pt}}{}{}
\addcontentsline{lot}{section}{\numberline{\thesection}\Sectionname}
\ttNum700\,-\ttNum799については、主に内面溝用サブプログラム(O2x000x, O5x000x)で使用されるものとする\\
%%%%%%%%%%%%%%%%%%%%%%%%%%%%%%%%%%%%%%%%%%%%%%%%%%%%%%%%%%
%% common variables %%%%%%%%%%%%%%%%%%%%%%%%%%%%%%%%%%%%%%
%%%%%%%%%%%%%%%%%%%%%%%%%%%%%%%%%%%%%%%%%%%%%%%%%%%%%%%%%%
\modcaptionof{table}{\ttNum700\,-\ttNum733}
\ttNum700\,-\ttNum733については、主に\DLone で使用されるものとする。
\begin{twoCtable}{}
\ttNum700 & (予備)\\\hline
\ttNum701 & プログラム読込み時の座標系(\ttNum4012)\\\hline
\ttNum702 & 工具別$Z$補正(\verb|T50|:\ttNum512, \verb|T3|x:0)\\\hline
\ttNum703 & 工具別$XY$補正(\verb|T50|:\ttNum512, \verb|T3|x:\ttNum[2400+\ttNum4111]+\ttNum[2600+\ttNum4111])\\\hline
\ttNum704 & 工具別移動\verb|G#| (\verb|T50|:31, \verb|T3|x:1)\\\hline
\ttNum705 & テーブル中心からワーク座標(\ttNum701)原点までの$X$距離\\\hline
\ttNum706 & 傾き後のトップ端面中心(機械座標)$X$ (\cf\pageeqref{eq:afterPhiTCenterFromO})\\\hline
\ttNum707 & テーブル中心から傾き後のトップ端面中心までの$Z$距離 (\cf\pageeqref{eq:afterPhiTCenterFromO})\\\hline
\ttNum708 & 傾き後トップ端中心(ブロックエンド)$X$座標(\ttNum5001)\\\hline
\ttNum709 & 傾き後トップ端中心(ブロックエンド)$Z$座標(\ttNum5003)\\\hline
\ttNum710 & テーブル中心から内面溝1列目までの$Z$距離$Z-q$\\\hline
\ttNum711 & トップ端中心から内面溝1列目中心までの$X$距離(\cf\pageeqref{eq:dimpleCenterDistance})\\\hline
\ttNum712 & 傾き後内面溝1列目中心$X$移動距離(\cf\pageeqref{eq:afterPhidimpleCenterDistance})\\\hline
\ttNum713 & 傾き後内面溝1列目中心$Z$移動距離(\cf\pageeqref{eq:afterPhidimpleCenterDistance})\\\hline
\ttNum714 & 傾き後内面溝1列目中心(ブロックエンド)$X$座標 (\ttNum5001)\\\hline
\ttNum715 & 傾き後内面溝1列目中心(ブロックエンド)$Y$座標 (\ttNum5002)\\\hline
\ttNum716 & 傾き後内面溝1列目中心(ブロックエンド)$Z$座標 (\ttNum5003)\\\hline
\ttNum717 & 各面用ループ番号(1:A, 2:C, 3:B, 4:D)\\\hline
\ttNum718 & BD内半径$-\text{\ttNum703}-10$\\\hline
\ttNum719 & (AC内半径$-\text{\ttNum703}-10)\cos\phi$\\\hline
& (以下予備)
\end{twoCtable}



%\clearpage
%%%%%%%%%%%%%%%%%%%%%%%%%%%%%%%%%%%%%%%%%%%%%%%%%%%%%%%%%%
%% common variables %%%%%%%%%%%%%%%%%%%%%%%%%%%%%%%%%%%%%%
%%%%%%%%%%%%%%%%%%%%%%%%%%%%%%%%%%%%%%%%%%%%%%%%%%%%%%%%%%
\modcaptionof{table}{\ttNum734\,-\ttNum766}
\ttNum734\,-\ttNum766については、主に\DLtwoAC, \DLtwoBD で使用されるものとする。
\begin{twoCtable}{}
\ttNum734 & プログラム読込時ブロックエンド$Y$ or $X$ (\ttNum5002, \ttNum5001)\\\hline
\ttNum735 & プログラム読込時ブロックエンド$Z$ (\ttNum5003)\\\hline
\ttNum736 & 内面溝 偶数列の列数\\\hline
\ttNum737 & 内面溝 偶数列(一列)の内面溝数\\\hline
\ttNum738 & 内面溝 奇数列(一列)の内面溝数\\\hline
\ttNum739 & 内面溝 現在の列の内面溝数\\\hline
& (以下予備)
\end{twoCtable}



\clearpage
%%%%%%%%%%%%%%%%%%%%%%%%%%%%%%%%%%%%%%%%%%%%%%%%%%%%%%%%%%
%% common variables %%%%%%%%%%%%%%%%%%%%%%%%%%%%%%%%%%%%%%
%%%%%%%%%%%%%%%%%%%%%%%%%%%%%%%%%%%%%%%%%%%%%%%%%%%%%%%%%%
\modcaptionof{table}{\ttNum767\,-\ttNum799}
\ttNum767\,-\ttNum799については、主に\DMLthreeAC, \DMLthreeBD, \DKLthreeAC, \DKLthreeBD で使用されるものとする。
\begin{twoCtable}{}
\ttNum767 & プログラム読込時ブロックエンド$X$ or $Y$ (\ttNum5001, \ttNum5002)\\\hline
\ttNum768 & 内面溝 表面位置$X$ or $Y$測定値\\\hline
& (以下予備)
\end{twoCtable}



\clearpage
%%%%%%%%%%%%%%%%%%%%%%%%%%%%%%%%%%%%%%%%%%%%%%%%%%%%%%%%%%
%% section 13.1 %%%%%%%%%%%%%%%%%%%%%%%%%%%%%%%%%%%%%%%%%%
%%%%%%%%%%%%%%%%%%%%%%%%%%%%%%%%%%%%%%%%%%%%%%%%%%%%%%%%%%
\modHeadsection{コモン変数 (\ttNum900001\,-\ttNum900031, \ttNum900101\,-\ttNum900500)}
\addtocontents{lot}{\protect\addvspace{3pt}}{}{}
\addcontentsline{lot}{section}{\numberline{\thesection}\Sectionname}
\ttNum900001\,-\ttNum900031, \ttNum900101\,-\ttNum900500については、主に実測値を格納する。\\
%%%%%%%%%%%%%%%%%%%%%%%%%%%%%%%%%%%%%%%%%%%%%%%%%%%%%%%%%%
%% common variables %%%%%%%%%%%%%%%%%%%%%%%%%%%%%%%%%%%%%%
%%%%%%%%%%%%%%%%%%%%%%%%%%%%%%%%%%%%%%%%%%%%%%%%%%%%%%%%%%
\modcaptionof{table}{\ttNum900001\,-\ttNum900005}
\ttNum900001\,-\ttNum900005については、主に\MXOThickness で使用されるものとする。
\begin{twoCtable}{}
\ttNum900001 & $X$外中心測定 $-X$側測定値\\\hline
\ttNum900002 & $X$外中心測定 $+X$側測定値\\\hline
\ttNum900003 & $X$外中心測定値\\\hline
\ttNum900004 & $X$外中心測定 厚さ測定値\\\hline
\ttNum900005 & (予備)\\
\end{twoCtable}



%%%%%%%%%%%%%%%%%%%%%%%%%%%%%%%%%%%%%%%%%%%%%%%%%%%%%%%%%%
%% common variables %%%%%%%%%%%%%%%%%%%%%%%%%%%%%%%%%%%%%%
%%%%%%%%%%%%%%%%%%%%%%%%%%%%%%%%%%%%%%%%%%%%%%%%%%%%%%%%%%
\modcaptionof{table}{\ttNum900006\,-\ttNum900010}
\ttNum900006\,-\ttNum900010については、主に\MYOThickness で使用されるものとする。
\begin{twoCtable}{}
\ttNum900006 & $Y$外中心測定 $-Y$側測定値\\\hline
\ttNum900007 & $Y$外中心測定 $+Y$側測定値\\\hline
\ttNum900008 & $Y$外中心測定値\\\hline
\ttNum900009 & $Y$外中心測定 厚さ測定値\\\hline
\ttNum900010 & (予備)\\
\end{twoCtable}


%\clearpage
%%%%%%%%%%%%%%%%%%%%%%%%%%%%%%%%%%%%%%%%%%%%%%%%%%%%%%%%%%
%% common variables %%%%%%%%%%%%%%%%%%%%%%%%%%%%%%%%%%%%%%
%%%%%%%%%%%%%%%%%%%%%%%%%%%%%%%%%%%%%%%%%%%%%%%%%%%%%%%%%%
\modcaptionof{table}{\ttNum900011\,-\ttNum900015}
\ttNum900011\,-\ttNum900015については、主に\MXIWidth で使用されるものとする。
\begin{twoCtable}{}
\ttNum900011 & $X$内中心測定 $-X$側測定値\\\hline
\ttNum900012 & $X$内中心測定 $+X$側測定値\\\hline
\ttNum900013 & $X$内中心測定値\\\hline
\ttNum900014 & $X$内中心測定 厚さ測定値\\\hline
\ttNum900015 & (予備)\\
\end{twoCtable}


%\clearpage
%%%%%%%%%%%%%%%%%%%%%%%%%%%%%%%%%%%%%%%%%%%%%%%%%%%%%%%%%%
%% common variables %%%%%%%%%%%%%%%%%%%%%%%%%%%%%%%%%%%%%%
%%%%%%%%%%%%%%%%%%%%%%%%%%%%%%%%%%%%%%%%%%%%%%%%%%%%%%%%%%
\modcaptionof{table}{\ttNum900016\,-\ttNum900020}
\ttNum900016\,-\ttNum900020については、主に\MYIWidth で使用されるものとする。
\begin{twoCtable}{}
\ttNum900016 & $Y$内中心測定 $-Y$側測定値\\\hline
\ttNum900017 & $Y$内中心測定 $+Y$側測定値\\\hline
\ttNum900018 & $Y$内中心測定値\\\hline
\ttNum900019 & $Y$内中心測定 厚さ測定値\\\hline
\ttNum900020 & (予備)\\
\end{twoCtable}


\clearpage
%%%%%%%%%%%%%%%%%%%%%%%%%%%%%%%%%%%%%%%%%%%%%%%%%%%%%%%%%%
%% common variables %%%%%%%%%%%%%%%%%%%%%%%%%%%%%%%%%%%%%%
%%%%%%%%%%%%%%%%%%%%%%%%%%%%%%%%%%%%%%%%%%%%%%%%%%%%%%%%%%
\modcaptionof{table}{\ttNum900021\,-\ttNum900023}
\ttNum900021\,-\ttNum900023については、主に\MXface で使用されるものとする。
\begin{twoCtable}{}
\ttNum900021 & $X$外削中心測定 内面測定値\\\hline
\ttNum900022 & (予備)\\\hline
\ttNum900023 & (予備)\\
\end{twoCtable}


%\clearpage
%%%%%%%%%%%%%%%%%%%%%%%%%%%%%%%%%%%%%%%%%%%%%%%%%%%%%%%%%%
%% common variables %%%%%%%%%%%%%%%%%%%%%%%%%%%%%%%%%%%%%%
%%%%%%%%%%%%%%%%%%%%%%%%%%%%%%%%%%%%%%%%%%%%%%%%%%%%%%%%%%
\modcaptionof{table}{\ttNum900024\,-\ttNum900027}
\ttNum900024\,-\ttNum900027については、主に\MYcenterline で使用されるものとする。
\begin{twoCtable}{}
\ttNum900024 & $Y$通り芯 ボトム側測定値\\\hline
\ttNum900025 & $Y$通り芯 トップ側測定値\\\hline
\ttNum900026 & $Y$通り芯 測定値\\\hline
\ttNum900027 & (予備)\\
\end{twoCtable}



%%%%%%%%%%%%%%%%%%%%%%%%%%%%%%%%%%%%%%%%%%%%%%%%%%%%%%%%%%
%% common variables %%%%%%%%%%%%%%%%%%%%%%%%%%%%%%%%%%%%%%
%%%%%%%%%%%%%%%%%%%%%%%%%%%%%%%%%%%%%%%%%%%%%%%%%%%%%%%%%%
\modcaptionof{table}{\ttNum900028\,-\ttNum900031}
\ttNum900028\,-\ttNum900031については、主に\MXcenterline で使用されるものとする。
\begin{twoCtable}{}
\ttNum900028 & $X$通り芯 トップ側測定値\\\hline
\ttNum900029 & $X$通り芯 ボトム側測定値\\\hline
\ttNum900030 & $X$通り芯 測定値\\\hline
\ttNum900031 & (予備)\\
\end{twoCtable}



%\clearpage
%%%%%%%%%%%%%%%%%%%%%%%%%%%%%%%%%%%%%%%%%%%%%%%%%%%%%%%%%%
%% common variables %%%%%%%%%%%%%%%%%%%%%%%%%%%%%%%%%%%%%%
%%%%%%%%%%%%%%%%%%%%%%%%%%%%%%%%%%%%%%%%%%%%%%%%%%%%%%%%%%
\modcaptionof{table}{\ttNum900101\,-\ttNum900500}
\ttNum900101\,-\ttNum900500については、主に\DMLthreeAC, \DMLthreeBD で使用されるものとする。
\begin{twoCtable}{}
\ttNum900101\,-\ttNum900200 & A側内面溝 深さ測定値(Tスロット)\\\hline
\ttNum900201\,-\ttNum900300 & C側内面溝 深さ測定値(Tスロット)\\\hline
\ttNum900301\,-\ttNum900400 & B側内面溝 深さ測定値(Tスロット)\\\hline
\ttNum900401\,-\ttNum900500 & D側内面溝 深さ測定値(Tスロット)
\end{twoCtable}



\clearpage
%%%%%%%%%%%%%%%%%%%%%%%%%%%%%%%%%%%%%%%%%%%%%%%%%%%%%%%%%%
%% section 13.1 %%%%%%%%%%%%%%%%%%%%%%%%%%%%%%%%%%%%%%%%%%
%%%%%%%%%%%%%%%%%%%%%%%%%%%%%%%%%%%%%%%%%%%%%%%%%%%%%%%%%%
\modHeadsection{コモン変数 (\ttNum901000\,-\ttNum901099)}
\addtocontents{lot}{\protect\addvspace{3pt}}{}{}
\addcontentsline{lot}{section}{\numberline{\thesection}\Sectionname}
\ttNum901000\,-\ttNum901099については、主にパレットやジグに関するものとする。\\
%%%%%%%%%%%%%%%%%%%%%%%%%%%%%%%%%%%%%%%%%%%%%%%%%%%%%%%%%%
%% common variables %%%%%%%%%%%%%%%%%%%%%%%%%%%%%%%%%%%%%%
%%%%%%%%%%%%%%%%%%%%%%%%%%%%%%%%%%%%%%%%%%%%%%%%%%%%%%%%%%
\modcaptionof{table}{コモン変数 (\ttNum901000\,-\ttNum901099)}
\begin{twoCtable}{}
\ttNum901000 & (予備)\\\hline
\ttNum901001 & パレット\ttNum1 ジグ中心機械座標$X$\\\hline
\ttNum901002 & パレット\ttNum1 ジグ中心機械座標$Y$\\\hline
\ttNum901003 & パレット\ttNum1 ジグ中心機械座標$Z$\\\hline
\ttNum901004 & パレット\ttNum1 ジグ中心機械座標$B$\\\hline
\ttNum901005 & パレット\ttNum2 ジグ中心機械座標$X$\\\hline
\ttNum901006 & パレット\ttNum2 ジグ中心機械座標$Y$\\\hline
\ttNum901007 & パレット\ttNum2 ジグ中心機械座標$Z$\\\hline
\ttNum901008 & パレット\ttNum2 ジグ中心機械座標$B$\\\hline
\ttNum901009 & 工具中心機械座標$C$\\\hline
\ttNum901010 & (予備)\\\hline
\hline
\ttNum901011 & パレット\ttNum1 ジグ外側幅$2l$(機械座標系$B$0における$Z$方向)\\\hline
\ttNum901012 & パレット\ttNum1 ジグ内側幅(機械座標系$B$0における$Z$方向)\\\hline
\ttNum901013 & パレット\ttNum1 ジグ幅(機械座標系$B$0における$X$方向)\\\hline
\ttNum901014 & (予備)\\\hline
\ttNum901015 & パレット\ttNum2 ジグ外側幅$2l$(機械座標系$B$0における$Z$方向)\\\hline
\ttNum901016 & パレット\ttNum2 ジグ内側幅(機械座標系$B$0における$Z$方向)\\\hline
\ttNum901017 & パレット\ttNum2 ジグ幅(機械座標系$B$0における$X$方向)\\\hline
& (以下予備)
\end{twoCtable}
%%%%%%%%%%%%%%%%%%%%%%%%%%%%%%%%%%%%%%%%%%%%%%%%%%%%%%%%%%
%% hosoku %%%%%%%%%%%%%%%%%%%%%%%%%%%%%%%%%%%%%%%%%%%%%%%%
%%%%%%%%%%%%%%%%%%%%%%%%%%%%%%%%%%%%%%%%%%%%%%%%%%%%%%%%%%
\begin{hosoku}
図面上の数値は、
\begin{align*}
  \ttNum901011 = 660~, \quad \ttNum901013 = 455
\end{align*}\noindent
その他、
\begin{enumerate}
\item テーブル中心 と C面側ジグ端 との水平距離:196.5
\item 受板の円の半径$\rho$:100
\item 受板の鉛直方向の幅$\sigma$:40
\item テーブル中心 と 受板の円の中心 との水平距離$\varDelta$:201.5
\item 受板の円の中心 と 受板の水平方向の底 との距離:70
\end{enumerate}
また、2023/09/26時点において、
\begin{align*}
  \ttNum901001 = -550.019~, \quad \ttNum901003 = -1149.974
\end{align*}
\end{hosoku}
%%%%%%%%%%%%%%%%%%%%%%%%%%%%%%%%%%%%%%%%%%%%%%%%%%%%%%%%%%
%%%%%%%%%%%%%%%%%%%%%%%%%%%%%%%%%%%%%%%%%%%%%%%%%%%%%%%%%%
%%%%%%%%%%%%%%%%%%%%%%%%%%%%%%%%%%%%%%%%%%%%%%%%%%%%%%%%%%
