%!TEX root = ../Mould_Analytical_Calculation_Note.tex

\makeatletter

%!TEX root = ../Mould_Analytical_Calculation_Note.tex

%%%%%%%%%%%%%%%%%%%%%%%%%%%%%%%%%%%%%%%%%%%%%%%%%%%%%%%%%%%%%%%%%%
\def\mouldCoordinate{%
\begin{tikzpicture}
% 値の計算
\pgfmathsetmacro{\Ax}{9.6} %A:T_iのx座標
\pgfmathsetmacro{\Ay}{3.5} %A:T_iのy座標
\pgfmathsetmacro{\Bx}{2.0+(\Ax)} %B:T_oのx座標
\pgfmathsetmacro{\Cy}{-4.2}                  %C:B_iのy座標
\pgfmathsetmacro{\Ri}{sqrt((\Ax)^2+(\Ay)^2)} %R_iの長さ
\pgfmathsetmacro{\Cx}{sqrt((\Ri)^2-(\Cy)^2)} %C:B_iのx座標
\pgfmathsetmacro{\Ro}{sqrt((\Bx)^2+(\Ay)^2)} %R_oの長さ
\pgfmathsetmacro{\Dx}{sqrt((\Ro)^2-(\Cy)^2)} %D:B_oのx座標
\pgfmathsetmacro{\Rc}{(\Ri+\Ro)/2}           %R_cの長さ
\pgfmathsetmacro{\Ex}{sqrt((\Rc)^2-(\Ay)^2)} %E:湾曲中心線トップ端のx座標
\pgfmathsetmacro{\Fx}{sqrt((\Rc)^2-(\Cy)^2)} %F:湾曲中心線ボトム端のx座標
\pgfmathsetmacro{\Ub}{2.4}                   %Ub:受板-モールド接点のy座標
\pgfmathsetmacro{\Ux}{sqrt((\Ri)^2-(\Ub)^2)} %Ux:受板-モールド接点のx座標
\pgfmathsetmacro{\Uxo}{sqrt((\Ro)^2-(\Ub)^2)} %Ux:受板-モールド接点のx座標
\pgfmathsetmacro{\Hx}{1+(\Ri)} %H:テーブルの中心
\pgfmathsetmacro{\Ix}{1.90} %I:テーブルx方向の長さの半分
% 座標系を描画
\draw[-latex, dotted] (-0.4, 0) -- (14.5, 0) node[below] {\textbf{Re}};
\draw[-latex, dotted] (0, -4) -- (0, 4) node[below right] {\textbf{Im}};
% 座標を定義
\coordinate (O) at (  0, 0); % 原点
\coordinate (A) at (\Ax, \Ay); %
\coordinate (B) at (\Bx, \Ay);
\coordinate (C) at (\Cx, \Cy);
\coordinate (D) at (\Dx, \Cy);
\coordinate (E) at (\Ex, \Ay);
\coordinate (F) at (\Fx, \Cy);
\coordinate (Rc) at (\Rc, 0);
\coordinate (Ri) at (\Ri, 0);
\coordinate (Ro) at (\Ro, 0);
\coordinate (Ut) at (\Ux, \Ub);
\coordinate (Ub) at (\Ux, -\Ub);
\coordinate (Uto) at (\Uxo, \Ub);
\coordinate (Ubo) at (\Uxo, -\Ub);
\coordinate (Tal) at (\Hx-\Ix, \Ub);
\coordinate (Tbl) at (\Hx-\Ix, -\Ub);
\coordinate (Tar) at (\Hx+\Ix, \Ub);
\coordinate (Tbr) at (\Hx+\Ix, -\Ub);
\coordinate (Lt) at (\Hx+\Ix+0.4, \Ub);
\coordinate (Lb) at (\Hx+\Ix+0.4, -\Ub);
\coordinate (Lc) at (\Hx+\Ix+0.4, 0);
\coordinate (Fc) at (\Hx+\Ix+0.9, 0);
\coordinate (Ft) at (\Hx+\Ix+0.9, \Ay);
\coordinate (Fb) at (\Hx+\Ix+0.9, \Cy);
% 点を描画
\fill (O) circle (2pt);
\fill (A) circle (2pt);
\fill (B) circle (2pt);
\fill (C) circle (2pt);
\fill (D) circle (2pt);
\fill (E) circle (2pt);
\fill (Rc) circle (2pt);
\fill (Ro) circle (2pt);
\fill (Ri) circle (2pt);
\fill (Ut) circle (2pt);
\fill (Ub) circle (2pt);
% 点にラベルを付ける
\node at (O) [below left] {O};
\node at (A) [above] {T$_\mathrm i$};
\node at (B) [above] {T$_\mathrm o$};
\node at (C) [below] {B$_\mathrm i$};
\node at (D) [below] {B$_\mathrm o$};
\node at (Rc) [above right] {$R_\mathrm c$};
\node at (Ro) [below right] {$R_\mathrm o$};
\node at (Ri) [below left] {$R_\mathrm i$};
\node at (Ut) [right] {U$_\mathrm T$};
\node at (Ub) [right] {U$_\mathrm B$};
% モールド外形
\draw[line width=1pt, fill=ffwwqq, fill opacity=0.1]
  let \p1=(A), \p2=(C), \p3=(B), \p4=(D), \n1={atan2(\y1,\x1)}, \n2={atan2(\y2,\x2)}, \n3={atan2(\y3,\x3)}, \n4={atan2(\y4,\x4)}
    in (A) -- (B) -- (\n3:\Ro) arc (\n3:\n4:\Ro) -- (C) -- (\n2:\Ri) arc (\n2:\n1:\Ri) -- cycle;
% モールド中心線
\draw[dotted, line width=1pt] let \p1=(E), \p2=(F), \n1={atan2(\y1,\x1)}, \n2={atan2(\y2,\x2)}
  in (\n1:\Rc) arc (\n1:\n2:\Rc);
% テーブル
\draw (Ut) -- (Tal) -- (Tbl) -- (Ub);
\draw (Uto) -- (Tar) -- (Tbr) -- (Ubo);
\draw[dotted] (Tar) -- (Lt);
\draw[dotted] (Tbr) -- (Lb);
\draw[dotted] (Tar) -- (Lt);
\draw[latex-latex, line width=1pt] (Lc) -- (Lt) node[midway, right] {$l$};
\draw[latex-latex, line width=1pt] (Lc) -- (Lb) node[midway, right] {$l$};
% 振分け
\draw[dotted] (B) -- (Ft);
\draw[dotted] (D) -- (Fb);
\draw[latex-latex, line width=1pt] (Fc) -- (Ft) node[midway, right] {$f_\mathrm T$};
\draw[latex-latex, line width=1pt] (Fc) -- (Fb) node[midway, right] {$f_\mathrm B$};
% 半径
\draw[dotted, line width=1pt] (O) -- (A) node[midway, above left] {$R_\mathrm i$} ;
\draw[dotted, line width=1pt] (O) -- (B) node[midway, below right] {$R_\mathrm o$} ;
\draw[dotted, line width=1pt] (O) -- (Ub) node[midway, above right] {$R_\mathrm i$} ;
% 角度
\draw[line width=1pt, fill=ffwwqq, fill opacity=0.1]
  let \p1=(Ri), \p2=(A), \n1={atan2(\y1,\x1)}, \n2={atan2(\y2,\x2)}
   in (\n1:2) arc (\n1:\n2:2) node[midway, right, opacity=1] {$\alpha_{\mathrm T_\mathrm i}$} -- (O);
\draw[line width=1pt, fill=qqzzqq, fill opacity=0.1]
  let \p1=(Ro), \p2=(B), \n1={atan2(\y1,\x1)}, \n2={atan2(\y2,\x2)}
  in (\n1:3.2) arc (\n1:\n2:3.2) node[midway, right, opacity=1] {$\alpha_{\mathrm T_\mathrm o}$} -- (O) -- cycle ;
\draw[line width=1pt, fill=wwqqcc, fill opacity=0.1]
  let \p1=(Ri), \p2=(Ub), \n1={atan2(\y1,\x1)}, \n2={atan2(\y2,\x2)}
  in (\n1:2.7) arc (\n1:\n2:2.7) node[midway, right, opacity=1] {$\alpha_{\mathrm U_\mathrm B}$} -- (O);
\end{tikzpicture}%
}
%%%%%%%%%%%%%%%%%%%%%%%%%%%%%%%%%%%%%%%%%%%%%%%%%%%%%%%%%%%%%%%%%%%%%%%%%%%%%

%%%%%% DATE %%%%%%%%%%%%%%%%%%%%%%%%%%%%%%%%%
\newcommand{\customdate}{%
  \the\year/\two@digits{\the\month}/\two@digits{\the\day}\
  \currenttime\ (\jadayofweek{\the\year}{\the\month}{\the\day})%
}
%%%%%% NEWIF %%%%%%%%%%%%%%%%%%%%%%%%%%%%%%%%%
\newif\if@backmatter%\@backmattertrue
\newif\if@frontmatter%\@frontmattertrue
\newif\if@appendix%\@appendixfalse
%%%%%% DEFINECOLOR %%%%%%%%%%%%%%%%%%%%%%%%%%%%%%%
\definecolor{ai}     {rgb}{0.2039, 0.3765, 0.4314}
\definecolor{kon}    {rgb}{0.0000, 0.2000, 0.4000}
\definecolor{konpeki}{rgb}{0.0902, 0.5098, 0.7333}
\definecolor{moegi}  {rgb}{0.3020, 0.5961, 0.1882}
\definecolor{sssec}  {rgb}{0.7333, 0.5, 0.7333}
\definecolor{sora}   {rgb}{0.1451, 0.7216, 0.8039}
\definecolor{sumire} {rgb}{0.3882, 0.2157, 0.5922}
\definecolor{wwqqcc}{rgb}{0.4, 0, 0.8}
\definecolor{qqzzqq}{rgb}{0, 0.6, 0}
\definecolor{ffwwqq}{rgb}{1, 0.4, 0}
%%%%% NEWCOLORBOX %%%%%%%%%%%%%%%%%%%%%%%%%%%%
%%%%% COLUMN %%%%%
\newcommand{\Columnname}{Column}
\newtcolorbox[auto counter, number within=chapter]{Column}[2][]{Columnbox, title={#2}, #1}
%%%%% HOSOKU BOX %%%%%
\newcommand{\hosokuname}{補}
\definecolor{hosoku}{cmyk}{0, 0, 0, .15}
\newtcolorbox[auto counter, number within=chapter]{hosokubox}[1][]{
  hosokubox, title={\termblue{\hosokuname~\thetcbcounter}~}, #1
}
%%%%% FIG BOX %%%%%
\newtcolorbox{Figbox}[1][]{Figurebox, #1}
%%%%% TIKZSET %%%%%%%%%%%%%%%%%%%%%%%%%%%%
\tikzset{
  %%%%% SECTIONFORMAT STYLE %%%%%
  sect/.style={signal, draw, text=white},
  section/.style={sect, fill=konpeki!100!, signal to=east, inner sep=3pt},
  subsection/.style={sect, fill=moegi!90!, signal to=nowhere, inner sep=3pt},
  subsubsection/.style={sect, fill=sssec!100!, signal to=nowhere, inner sep=3pt},
  %%%%% TERMINAL STYLE %%%%%
  terminal/.style={rectangle,
                   minimum size=10pt,
                   rounded corners=1.5mm,
                   thin, draw=black!75,
                   top color=white,
                   font=\fontfamily{pplx},
                   inner sep=3pt, inner xsep=3pt,
                   text height=1ex, text depth=0pt,},
  %%%%% BMATRIX STYLE %%%%%
  every left delimiter/.style={xshift=.5em},
  every right delimiter/.style={xshift=-.5em},
  bmatrix/.style={matrix of math nodes,
                  left delimiter=[, right delimiter=],},
}
%%%%% TCBSET %%%%%%%%%%%%%%%%%%%%%%%%%%%%
\tcbset{
  %%%%% HIGHLIGHT MATH STYLE %%%%%
  highlight math style={enhanced, arc=2pt, boxrule=\z@, frame hidden,
                        fuzzy halo=1pt with blue,
                        colback=yellow!40!white,
                        left=\z@, right=\z@, top=.4mm, bottom=.4mm},
  %%%%% COLUMNBOX STYLE %%%%%
  Columnbox/.style={enhanced jigsaw, breakable, left=.5ex, right=.5ex,
                    after title=\hfill\termblue{\Columnname~\thetcbcounter},
                    fonttitle=\gtfamily\bfseries,
                    bicolor, colbacklower=black!10!white,},
  %%%%% HOSOKUBOX STYLE %%%%%
  hosokubox/.style={breakable, enhanced jigsaw, attach title to upper,
                    colback=hosoku, colframe=hosoku,
                    size=fbox, arc=\z@, middle=1mm,
                    drop lifted shadow={blue!100!white!50!},
                    skin first is subskin of={enhanced jigsaw}{no shadow},
                    skin middle is subskin of={enhanced jigsaw}{no shadow},
                    skin last is subskin of={enhanced jigsaw}%
                                            {drop lifted shadow={blue!100!white!50!}},
                    segmentation style={draw=black!50!white},
                    after=\smallskip\noindent{\color{white}},},
  %%%%% FIGUREBOX STYLE %%%%%
  Figurebox/.style={notitle, center upper, center lower, arc=5pt, outer arc=2pt, boxrule=1pt,
                    colback=green!3!white, colframe=black!25!white,
                    boxsep=3mm, left=\z@, right=\z@, valign=center,
                    },
}
%%%%% OTHER TIKZ DEFINITION %%%%%%%%%%%%%%%%%%
\tikzfading[name=fade ball, inner color=transparent!60, outer color=transparent!30]
\def\sball#1{\tikz \shade [ball color=#1, path fading=fade ball] (0,0) circle (.7ex);}
\def\terminal#1#2{\tikz[baseline=(a.base)] \node (a) [terminal, bottom color=#2] {\small #1};}
%%%%%%%%%%%%%%%%%%%%%%%%%%%%%%%%%%%%%%%%%%%%%%
%%%%% LINK %%%%%%%%%%%%%%%%%%%%%%%%%%%%%%%%%%%
\newcommand\nextsectionlink[1]{\addtocounter{section}\@ne
                               \hyperlink{section.\thechapter.\the\c@section}{#1}%
                               \addtocounter{section}{-\@ne}}
\newcommand\previoussectionlink[1]{\addtocounter{section}{-\@ne}
                                   \hyperlink{section.\thechapter.\the\c@section}{#1}%
                                   \addtocounter{section}{\@ne}}
\newcommand\previouschapterlink[1]{\addtocounter{chapter}{-\@ne}
                                   \hyperlink{chapter.\the\c@chapter}{#1}%
                                   \addtocounter{chapter}{\@ne}}
\newcommand{\pageautoref}[1]{%
  \ifthenelse{\equal{\pageref{#1}}{\thepage}}%
    {\autoref{#1}}%
    {\autoref{#1}~[p.\pageref{#1}]}%
}
\newcommand{\pageeqref}[1]{%
  \ifthenelse{\equal{\pageref{#1}}{\thepage}}%
    {\eqref{#1}}%
    {\eqref{#1}~[p.\pageref{#1}]}%
}
%%%%% DIM %%%%%%%%%%%%%%%%%%%%%%%%%%%%%%%%%%%
\newcommand{\hk}{\hspace{\kanjiskip}}
\newcommand{\hx}{\hspace{\xkanjiskip}}
%%%%% OTHER DEFINITION %%%%%%%%%%%%%%%%%%
\def\termblue#1{\terminal{\color{blue}\fontsize{8pt}{\z@}\textbf{#1}}{gray!25}\hskip.75zw}
%%%%% DECLAREMATHOPERATOR %%%%%%%%%%%%%%%%%%%%
\DeclareRobustCommand{\bDiv}{\nonscript\mskip-\medmuskip\mkern5mu\mathbin
  {\operator@font div}\penalty900
  \mkern5mu\nonscript\mskip-\medmuskip}
\DeclareRobustCommand{\pod}[1]{\allowbreak
  \if@display\mkern18mu\else\mkern8mu\fi(#1)}
\DeclareRobustCommand{\pDiv}[1]{\pod{{\operator@font div}\mkern6mu#1}}
\DeclareRobustCommand{\Div}[1]{\allowbreak\if@display\mkern18mu
  \else\mkern12mu\fi{\operator@font div}\,\,#1}
%%%%% PART FOR APPENDIX %%%%%%%%%%%%%%%%%%%%%%%%%%%%%%%%%%%%%%%%%
\newcommand{\Apart}{
  \addpart*{第\hx\hx\hx\thepart\hx\hx\hx 部の補遺}
  \addcontentsline{toc}{part}{第\hx\hx\thepart\hx\hx 部の補遺}
}


