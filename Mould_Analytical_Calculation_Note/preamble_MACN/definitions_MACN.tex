%!TEX root = ../Mould_Analytical_Calculation_Note.tex

\makeatletter

\input{preamble_MACN/pictures_MACN}

%%%%%% DATE %%%%%%%%%%%%%%%%%%%%%%%%%%%%%%%%%
\newcommand{\customdate}{%
  \the\year/\two@digits{\the\month}/\two@digits{\the\day}\
  \currenttime\ (\jadayofweek{\the\year}{\the\month}{\the\day})%
}
%%%%%% NEWIF %%%%%%%%%%%%%%%%%%%%%%%%%%%%%%%%%
\newif\if@backmatter%\@backmattertrue
\newif\if@frontmatter%\@frontmattertrue
\newif\if@appendix%\@appendixfalse
%%%%%% DEFINECOLOR %%%%%%%%%%%%%%%%%%%%%%%%%%%%%%%
\definecolor{ai}     {rgb}{0.2039, 0.3765, 0.4314}
\definecolor{kon}    {rgb}{0.0000, 0.2000, 0.4000}
\definecolor{konpeki}{rgb}{0.0902, 0.5098, 0.7333}
\definecolor{moegi}  {rgb}{0.3020, 0.5961, 0.1882}
\definecolor{sssec}  {rgb}{0.7333, 0.5, 0.7333}
\definecolor{sora}   {rgb}{0.1451, 0.7216, 0.8039}
\definecolor{sumire} {rgb}{0.3882, 0.2157, 0.5922}
\definecolor{wwqqcc}{rgb}{0.4, 0, 0.8}
\definecolor{qqzzqq}{rgb}{0, 0.6, 0}
\definecolor{ffwwqq}{rgb}{1, 0.4, 0}
%%%%% NEWCOLORBOX %%%%%%%%%%%%%%%%%%%%%%%%%%%%
\newcounter{GlobalFootnote}% difine a counter Global Footnote
%%%%% COLUMN %%%%%
\newcommand{\Columnname}{Column}
\newtcolorbox[auto counter, number within=chapter]{Column}[2][]{Columnbox, title={#2}, #1}
%%%%% FIGBOX %%%%%
\newtcolorbox{Figbox}[1][]{Figurebox, #1}
%%%%% HOSOKU %%%%%
\newcommand{\hosokuname}{補}
\definecolor{hosoku}{cmyk}{0, 0, 0, .15}
\newtcolorbox[auto counter, number within=chapter]{hosoku}[1][]{hosokubox, #1}
%%%%% TWOCTABLE %%%%%
\newtcolorbox[auto counter, number within=chapter]{twoCtable}[2][]{twoCtablebox, title={#2}, #1}
%%%%% TCBSET %%%%%%%%%%%%%%%%%%%%%%%%%%%%
\tcbset{%
  %%%%% COLUMNBOX STYLE %%%%%
  Columnbox/.style={%
    after title=\hfill\termblue{\Columnname~\thetcbcounter},%
    fonttitle=\gtfamily\bfseries,%
    breakable,%
    enhanced jigsaw,%
    left=.5ex,%
    right=.5ex,%
    bicolor,%
    colbacklower=black!10!white,%
    before upper={%
      \setcounter{GlobalFootnote}{\value{footnote}}% GlobalFootnoteValue=footnoteValue
      \let\oldfootnote=\footnote% \oldfootnote=\footnote
      \def\footnote{\stepcounter{GlobalFootnote}\oldfootnote[\arabic{GlobalFootnote}]}% define \footnote to \footnote using counter GlobalFootnote
      \renewcommand\thempfootnote{\arabic{mpfootnote}}% arabic footnote
    },
    after upper={%
      \setcounter{footnote}{\value{GlobalFootnote}}% footnoteValue=GlobalFootnoteValue
      \let\footnote=\oldfootnote% \footnote=\oldfootnote
    },
  },%
  %%%%% FIGUREBOX STYLE %%%%%
  Figurebox/.style={%
    notitle,%
    height=\textwidth,
%    width=\textwidth,
    center upper,%
    center lower,%
    arc=5pt,%
    outer arc=2pt,%
    boxrule=1pt,%
    boxsep=3mm,%
    valign=center,%
    halign=center,%
    left=\z@,%
    right=\z@,%
    colback=green!3!white,%
    colframe=black!25!white,%
    before={\centering},
  },
  %%%%% HIGHLIGHT MATH STYLE %%%%%
%  highlight math/.style={%
%    enhanced,%
%    arc=2pt,%
%    boxrule=\z@,%
%    frame hidden,%
%    fuzzy halo=1pt with blue,%
%    left=\z@,%
%    right=\z@,%
%    top=.4mm,%
%    bottom=.4mm,%
%    colback=yellow!40!white,%
%  },%
  %%%%% HOSOKUBOX STYLE %%%%%
  hosokubox/.style={%
    title={\termblue{\hosokuname~\thetcbcounter}~},%
    attach title to upper,%
    breakable,%
    enhanced jigsaw,%
    size=fbox,%
    arc=\z@,%
    middle=1mm,%
    colback=hosoku,%
    colframe=hosoku,%
    drop lifted shadow={blue!100!white!50!},%
    skin first is subskin of={enhanced jigsaw}{no shadow},%
    skin middle is subskin of={enhanced jigsaw}{no shadow},%
    skin last is subskin of={enhanced jigsaw}{drop lifted shadow={blue!100!white!50!}},%
    segmentation style={draw=black!50!white},%
    after=\smallskip\noindent{\color{white}},%
  },
  %%%%% TWOCTABLEBOX STYLE %%%%%
  twoCtablebox/.style={%
    breakable,%
    enhanced,%
    fonttitle=\bfseries,%
    fontupper=\small\sffamily,%
    colframe=black!50!black,%
    colbacktitle=blue!10!white,%
    coltitle=black,%
    top=0pt,%
    bottom=0pt,%
    left=0pt,%
    right=0pt,%
    before upper={%
      \setcounter{GlobalFootnote}{\value{footnote}}% GlobalFootnoteValue=footnoteValue
      \let\oldfootnote=\footnote% \oldfootnote=\footnote
      \def\footnote{\stepcounter{GlobalFootnote}\oldfootnote[\arabic{GlobalFootnote}]}% define \footnote to \footnote using counter GlobalFootnote
      \renewcommand\thempfootnote{\arabic{mpfootnote}}% arabic footnote
      \renewcommand{\arraystretch}{1.2}% 行の高さを調整
      \setlength{\LTpre}{-3pt}%
      \setlength{\LTpost}{0pt}%
      \setlength{\LTleft}{0pt}%
      \setlength{\LTright}{0pt}%
      \begin{longtable}{@{}c|l@{\extracolsep{\fill}}c}%
    },%
    after upper={%
      \end{longtable}%
      \setcounter{footnote}{\value{GlobalFootnote}}% footnoteValue=GlobalFootnoteValue
      \let\footnote=\oldfootnote% \footnote=\oldfootnote
    },
  },
}
%%%%% TIKZSET %%%%%%%%%%%%%%%%%%%%%%%%%%%%
\tikzset{
  %%%%% SECTIONFORMAT STYLE %%%%%
%  sect/.style={signal, draw, text=white},
%  section/.style={sect, fill=konpeki!100!, signal to=east, inner sep=3pt},
%  subsection/.style={sect, fill=moegi!90!, signal to=nowhere, inner sep=3pt},
%  subsubsection/.style={sect, fill=sssec!100!, signal to=nowhere, inner sep=3pt},
  %%%%% TERMINAL STYLE %%%%%
  terminal/.style={%
    rectangle,%
    minimum size=10pt,%
    rounded corners=1.5mm,%
    thin,%
    draw=black!75,%
    top color=white,%
    font=\fontfamily{pplx},%
    inner sep=3pt,%
    inner xsep=3pt,%
    text height=1ex,%
    text depth=0pt,%
  },
  %%%%% BMATRIX STYLE %%%%%
%  every left delimiter/.style={xshift=.5em},
%  every right delimiter/.style={xshift=-.5em},
%  bmatrix/.style={matrix of math nodes, left delimiter=[, right delimiter=],},
}
%%%%% OTHER TIKZ DEFINITION %%%%%%%%%%%%%%%%%%
\tikzfading[name=fade ball, inner color=transparent!60, outer color=transparent!30]
\def\sball#1{\tikz \shade [ball color=#1, path fading=fade ball] (0,0) circle (.7ex);}
\def\terminal#1#2{\tikz[baseline=(a.base)] \node (a) [terminal, bottom color=#2] {\small #1};}
%%%%% LINK %%%%%%%%%%%%%%%%%%%%%%%%%%%%%%%%%%%
\newcommand\nextsectionlink[1]{\addtocounter{section}\@ne
                               \hyperlink{section.\thechapter.\the\c@section}{#1}%
                               \addtocounter{section}{-\@ne}}
\newcommand\previoussectionlink[1]{\addtocounter{section}{-\@ne}
                                   \hyperlink{section.\thechapter.\the\c@section}{#1}%
                                   \addtocounter{section}{\@ne}}
\newcommand\previouschapterlink[1]{\addtocounter{chapter}{-\@ne}
                                   \hyperlink{chapter.\the\c@chapter}{#1}%
                                   \addtocounter{chapter}{\@ne}}
%%%%% REF %%%%%%%%%%%%%%%%%%%%%%%%%%%%%%%%%%%
\newcommand{\pageautoref}[1]{%
  \ifthenelse{\equal{\pageref{#1}}{\thepage}}%
    {\autoref{#1}}%
    {\autoref{#1}~[p.\pageref{#1}]}%
}
\newcommand{\pageeqref}[1]{%
  \ifthenelse{\equal{\pageref{#1}}{\thepage}}%
    {\eqref{#1}}%
    {\eqref{#1}~[p.\pageref{#1}]}%
}
%%%%% DECLAREMATHOPERATOR %%%%%%%%%%%%%%%%%%%%
\DeclareMathOperator{\IP}{Im}
\DeclareMathOperator{\RP}{Re}
%%%%% DECLAREROBUSTCOMMAND %%%%%%%%%%%%%%%%%%%%
\DeclareRobustCommand{\bDiv}{\nonscript\mskip-\medmuskip\mkern5mu\mathbin
  {\operator@font div}\penalty900
  \mkern5mu\nonscript\mskip-\medmuskip}
\DeclareRobustCommand{\pod}[1]{\allowbreak
  \if@display\mkern18mu\else\mkern8mu\fi(#1)}
\DeclareRobustCommand{\pDiv}[1]{\pod{{\operator@font div}\mkern6mu#1}}
\DeclareRobustCommand{\Div}[1]{\allowbreak\if@display\mkern18mu
  \else\mkern12mu\fi{\operator@font div}\,\,#1}
%%%%% PART FOR APPENDIX %%%%%%%%%%%%%%%%%%%%%%%%%%%%%%%%%%%%%%%%%
\newcommand{\Apart}{
  \addpart*{第\hx\hx\hx\thepart\hx\hx\hx 部の補遺}
  \addcontentsline{toc}{part}{第\hx\hx\thepart\hx\hx 部の補遺}
}
%%%%% DIM %%%%%%%%%%%%%%%%%%%%%%%%%%%%%%%%%%%
\newcommand{\hk}{\hspace{\kanjiskip}}
\newcommand{\hx}{\hspace{\xkanjiskip}}
%%%%% PRG # %%%%%%%%%%%%%%%%%%
\newcommand\MXOThickness{O110001} % X外側中心計測
\newcommand\MYOThickness{O110002} % Y外側中心計測
\newcommand\MXIWidth{O110011} % X内側中心計測
\newcommand\MYIWidth{O110012} % Y内側中心計測
\newcommand\MXface{O110101} % X基準面計測
\newcommand\MYcenterline{O111002} % 通り芯Y
\newcommand\MXcenterline{O111003} % 通り芯X(Z測定)
\newcommand\DLone{O210000} % 内面溝用 レベル1
\newcommand\DLtwoAC{O220001} % 内面溝用 レベル2 AC
\newcommand\DLtwoBD{O220002} % 内面溝用 レベル2 BD
\newcommand\DMLthreeAC{O230001} % 内面溝 測定用 レベル3 AC
\newcommand\DMLthreeBD{O230002} % 内面溝 測定用 レベル3 BD
\newcommand\KRecRight{O515000} % 端面用 rectangle  左回り
\newcommand\KICRLeft{O515200} % 内面取用 コーナーR 左回り
\newcommand\KOCRLeft{O515600} % 内面取用 コーナーR 左回り
\newcommand\DKLthreeAC{O630001} % 内面溝 加工用 レベル3 AC
\newcommand\DKLthreeBD{O630002} % 内面溝 加工用 レベル3 BD
\newcommand\OsensorOn{O910001} % タッチセンサーON
\newcommand\OsensorOff{O910002} % タッチセンサーOFF
%%%%% MACHINING NAME %%%%%%%%%%%%%%%%%%
\newcommand{\DMname}{ディンプルマシニング}
\newcommand{\MMname}{三菱マシニング}
%%%%% OTHER DEFINITION %%%%%%%%%%%%%%%%%%
\def\termblue#1{\terminal{\color{blue}\fontsize{8pt}{\z@}\textbf{#1}}{gray!25}\hskip.75zw}
