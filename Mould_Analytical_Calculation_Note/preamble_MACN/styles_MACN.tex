%!TEX root = ../Mould_Analytical_Calculation_Note.tex

%%%%% DOCUMENTCLASS %%%%%%%%%%%%%%%%%%%%%%%%%%%%
\documentclass[12pt]{scrbook}
%%%%%%%%%%%%%%%%%%%%%%%%%%%%%%%%%%%%%%%%%%%%%%

%!TEX root = ../Mould_Analytical_Calculation_Note.tex

% Encoding and Language Settings
\usepackage[utf8]{inputenc} % Allows input of utf8 characters.
\usepackage{babel} % Multilingual support for LaTeX.
\usepackage[japanese]{pxbabel} % Japanese support for babel.
\usepackage[style=english]{csquotes} % Context sensitive quotation facilities.

% Font and Math Settings
\usepackage{amsmath, amssymb} % Enhanced math support in LaTeX.
\usepackage{eulervm} % Euler virtual math fonts.

% Layout and Table Settings
\usepackage{geometry} % Provides an easy and flexible user interface to customize page layout.
\usepackage{array} % Extends array and tabular environments.
\usepackage{longtable} % Allows tables to break across pages.
\usepackage{colortbl} % Adds color to LaTeX tables.

% Graphics and Color Settings
\usepackage[dvipsnames]{xcolor} % Provides driver-independent color extensions for LaTeX and pdfLaTeX.
  %%% tikz, pgf %%%
\def\pgfsysdriver{pgfsys-dvipdfmx.def} % Specifies the driver for PGF, a lower-level language for creating graphics.
\usepackage{pgfplots} % A tool to create 2D and 3D plots in LaTeX.
\pgfplotsset{compat=1.18} % Specifies the version of pgfplots to use for compatibility.

% Box Settings
\usepackage[most]{tcolorbox} % Provides an environment for colored and framed text boxes with a heading line.
\usepackage{tikzpagenodes}

% Bibliography Settings
\usepackage[backend=biber, style=numeric-comp]{biblatex} % references
\addbibresource{./preamble_MACN/reference_MACN.bib}

% List Settings
\usepackage{enumitem} % Control layout of itemize, enumerate, description.

% Hyperlink Settings
\usepackage[dvipdfmx]{hyperref} % Adds support for hyperlinks.
\usepackage{pxjahyper} % Adjusts hyperref for pLaTeX and upLaTeX.

% Header and Footer Settings
\usepackage{scrlayer-fancyhdr} % Combines the features of fancyhdr with KOMA-Script's scrlayer.

% Date and Time
\usepackage{datetime} % Date and time handling.
\usepackage{bxjaholiday} % Support for Japanese holidays.

% Page Layout
\usepackage{lastpage} % Reference last page for Page N of M type footers.
\usepackage{pdflscape} % Make landscape pages display as landscape.
\usepackage{afterpage} % Execute command after the next page break.

% Footnote
\usepackage{footnote} % Improve on LaTeX's footnote handling.

% Table of Contents and Headers
\usepackage{titletoc} % Alternative headings for toc/lof/lot.
\usepackage{appendix} % Extra control of appendices.

% Caption
\usepackage{caption} % Customizes captions in floating environments.
%\usepackage{subcaption} % Support for sub-captions.

% Line Spacing
\usepackage{setspace} % Set space between lines.

% Citation
%\usepackage{cite} % Improved citation handling.
%\usepackage{cleveref} % Intelligent cross-referencing.

% Conditional Processing
\usepackage{ifthen} % Conditional commands.

% Loop
\usepackage{pgffor} % Foreach loop structure.

% Arrow
\usepackage{extarrows} % Extra Arrows beyond those provided in AMSmath.

% Units
\usepackage{units} % Typeset units.

% Box Adjustment
\usepackage{adjustbox} % Graphics package-alike macros for "general" boxes.

% List Display
\usepackage{jvlisting} % For including code listings with Japanese comments.

% Other
\usepackage{scrhack} % Fix koma-script interaction with other packages.

%%%%% TIKZ LIBRARY etc %%%%%%%%%%%%%%%%%%%%%%%%%%%%
% Arrows
%\usetikzlibrary{arrows} % Arrow tip library.
%\usetikzlibrary{arrows.meta} % Advanced arrow tip library.
% Calculations
\usetikzlibrary{calc} % Coordinate calculations.
% Decorations
%\usetikzlibrary{decorations} % General decoration library.
%\usetikzlibrary{decorations.fractals} % Fractal decorations.
%\usetikzlibrary{decorations.markings} % Arbitrary markings on paths.
%\usetikzlibrary{decorations.pathmorphing} % "Morphing" decorations.
%\usetikzlibrary{decorations.shapes} % Shape decorations.
%\usetikzlibrary{decorations.text} % Text decorations.
%\usepgfmodule{decorations} % Decoration library module.
% Matrix
%\usetikzlibrary{matrix} % Matrix library.
% Plot Marks
%\usetikzlibrary{plotmarks} % Plot mark library.
% Positioning
%\usetikzlibrary{positioning} % Improved positioning of nodes.
% Shadows
%\usetikzlibrary{shadows} % Shadow library.
% Shapes
%\usetikzlibrary{shapes} % Shape library.
% Trees
%\usetikzlibrary{trees} % Tree library.
% tcolorbox Libraries
\tcbuselibrary{listings} % Enables the use of listings within tcolorboxes.
\tcbuselibrary{breakable} % Allows tcolorboxes to break across pages.
%\tcbuselibrary{skins} % Provides additional skins to customize the appearance of tcolorboxes.
%\tcbuselibrary{theorems} % Provides theorem environments within tcolorboxes.
%!TEX root = ../Mould_Analytical_Calculation_Note.tex

\makeatletter

\input{preamble_MACN/pictures_MACN}

%%%%%% DATE %%%%%%%%%%%%%%%%%%%%%%%%%%%%%%%%%
\newcommand{\customdate}{%
  \the\year/\two@digits{\the\month}/\two@digits{\the\day}\
  \currenttime\ (\jadayofweek{\the\year}{\the\month}{\the\day})%
}
%%%%%% NEWIF %%%%%%%%%%%%%%%%%%%%%%%%%%%%%%%%%
\newif\if@backmatter%\@backmattertrue
\newif\if@frontmatter%\@frontmattertrue
\newif\if@appendix%\@appendixfalse
%%%%%% DEFINECOLOR %%%%%%%%%%%%%%%%%%%%%%%%%%%%%%%
\definecolor{ai}     {rgb}{0.2039, 0.3765, 0.4314}
\definecolor{kon}    {rgb}{0.0000, 0.2000, 0.4000}
\definecolor{konpeki}{rgb}{0.0902, 0.5098, 0.7333}
\definecolor{moegi}  {rgb}{0.3020, 0.5961, 0.1882}
\definecolor{sssec}  {rgb}{0.7333, 0.5, 0.7333}
\definecolor{sora}   {rgb}{0.1451, 0.7216, 0.8039}
\definecolor{sumire} {rgb}{0.3882, 0.2157, 0.5922}
\definecolor{wwqqcc}{rgb}{0.4, 0, 0.8}
\definecolor{qqzzqq}{rgb}{0, 0.6, 0}
\definecolor{ffwwqq}{rgb}{1, 0.4, 0}
%%%%% NEWCOLORBOX %%%%%%%%%%%%%%%%%%%%%%%%%%%%
\newcounter{GlobalFootnote}% difine a counter Global Footnote
%%%%% COLUMN %%%%%
\newcommand{\Columnname}{Column}
\newtcolorbox[auto counter, number within=chapter]{Column}[2][]{Columnbox, title={#2}, #1}
%%%%% FIGBOX %%%%%
\newtcolorbox{Figbox}[1][]{Figurebox, #1}
%%%%% HOSOKU %%%%%
\newcommand{\hosokuname}{補}
\definecolor{hosoku}{cmyk}{0, 0, 0, .15}
\newtcolorbox[auto counter, number within=chapter]{hosoku}[1][]{hosokubox, #1}
%%%%% TWOCTABLE %%%%%
\newtcolorbox[auto counter, number within=chapter]{twoCtable}[2][]{twoCtablebox, title={#2}, #1}
%%%%% TCBSET %%%%%%%%%%%%%%%%%%%%%%%%%%%%
\tcbset{%
  %%%%% COLUMNBOX STYLE %%%%%
  Columnbox/.style={%
    after title=\hfill\termblue{\Columnname~\thetcbcounter},%
    fonttitle=\gtfamily\bfseries,%
    breakable,%
    enhanced jigsaw,%
    left=.5ex,%
    right=.5ex,%
    bicolor,%
    colbacklower=black!10!white,%
    before upper={%
      \setcounter{GlobalFootnote}{\value{footnote}}% GlobalFootnoteValue=footnoteValue
      \let\oldfootnote=\footnote% \oldfootnote=\footnote
      \def\footnote{\stepcounter{GlobalFootnote}\oldfootnote[\arabic{GlobalFootnote}]}% define \footnote to \footnote using counter GlobalFootnote
      \renewcommand\thempfootnote{\arabic{mpfootnote}}% arabic footnote
    },
    after upper={%
      \setcounter{footnote}{\value{GlobalFootnote}}% footnoteValue=GlobalFootnoteValue
      \let\footnote=\oldfootnote% \footnote=\oldfootnote
    },
  },%
  %%%%% FIGUREBOX STYLE %%%%%
  Figurebox/.style={%
    notitle,%
    height=\textwidth,
%    width=\textwidth,
    center upper,%
    center lower,%
    arc=5pt,%
    outer arc=2pt,%
    boxrule=1pt,%
    boxsep=3mm,%
    valign=center,%
    halign=center,%
    left=\z@,%
    right=\z@,%
    colback=green!3!white,%
    colframe=black!25!white,%
    before={\centering},
  },
  %%%%% HIGHLIGHT MATH STYLE %%%%%
%  highlight math/.style={%
%    enhanced,%
%    arc=2pt,%
%    boxrule=\z@,%
%    frame hidden,%
%    fuzzy halo=1pt with blue,%
%    left=\z@,%
%    right=\z@,%
%    top=.4mm,%
%    bottom=.4mm,%
%    colback=yellow!40!white,%
%  },%
  %%%%% HOSOKUBOX STYLE %%%%%
  hosokubox/.style={%
    title={\termblue{\hosokuname~\thetcbcounter}~},%
    attach title to upper,%
    breakable,%
    enhanced jigsaw,%
    size=fbox,%
    arc=\z@,%
    middle=1mm,%
    colback=hosoku,%
    colframe=hosoku,%
    drop lifted shadow={blue!100!white!50!},%
    skin first is subskin of={enhanced jigsaw}{no shadow},%
    skin middle is subskin of={enhanced jigsaw}{no shadow},%
    skin last is subskin of={enhanced jigsaw}{drop lifted shadow={blue!100!white!50!}},%
    segmentation style={draw=black!50!white},%
    after=\smallskip\noindent{\color{white}},%
  },
  %%%%% TWOCTABLEBOX STYLE %%%%%
  twoCtablebox/.style={%
    breakable,%
    enhanced,%
    fonttitle=\bfseries,%
    fontupper=\small\sffamily,%
    colframe=black!50!black,%
    colbacktitle=blue!10!white,%
    coltitle=black,%
    top=0pt,%
    bottom=0pt,%
    left=0pt,%
    right=0pt,%
    before upper={%
      \setcounter{GlobalFootnote}{\value{footnote}}% GlobalFootnoteValue=footnoteValue
      \let\oldfootnote=\footnote% \oldfootnote=\footnote
      \def\footnote{\stepcounter{GlobalFootnote}\oldfootnote[\arabic{GlobalFootnote}]}% define \footnote to \footnote using counter GlobalFootnote
      \renewcommand\thempfootnote{\arabic{mpfootnote}}% arabic footnote
      \renewcommand{\arraystretch}{1.2}% 行の高さを調整
      \setlength{\LTpre}{-3pt}%
      \setlength{\LTpost}{0pt}%
      \setlength{\LTleft}{0pt}%
      \setlength{\LTright}{0pt}%
      \begin{longtable}{@{}c|l@{\extracolsep{\fill}}c}%
    },%
    after upper={%
      \end{longtable}%
      \setcounter{footnote}{\value{GlobalFootnote}}% footnoteValue=GlobalFootnoteValue
      \let\footnote=\oldfootnote% \footnote=\oldfootnote
    },
  },
}
%%%%% TIKZSET %%%%%%%%%%%%%%%%%%%%%%%%%%%%
\tikzset{
  %%%%% SECTIONFORMAT STYLE %%%%%
%  sect/.style={signal, draw, text=white},
%  section/.style={sect, fill=konpeki!100!, signal to=east, inner sep=3pt},
%  subsection/.style={sect, fill=moegi!90!, signal to=nowhere, inner sep=3pt},
%  subsubsection/.style={sect, fill=sssec!100!, signal to=nowhere, inner sep=3pt},
  %%%%% TERMINAL STYLE %%%%%
  terminal/.style={%
    rectangle,%
    minimum size=10pt,%
    rounded corners=1.5mm,%
    thin,%
    draw=black!75,%
    top color=white,%
    font=\fontfamily{pplx},%
    inner sep=3pt,%
    inner xsep=3pt,%
    text height=1ex,%
    text depth=0pt,%
  },
  %%%%% BMATRIX STYLE %%%%%
%  every left delimiter/.style={xshift=.5em},
%  every right delimiter/.style={xshift=-.5em},
%  bmatrix/.style={matrix of math nodes, left delimiter=[, right delimiter=],},
}
%%%%% OTHER TIKZ DEFINITION %%%%%%%%%%%%%%%%%%
\tikzfading[name=fade ball, inner color=transparent!60, outer color=transparent!30]
\def\sball#1{\tikz \shade [ball color=#1, path fading=fade ball] (0,0) circle (.7ex);}
\def\terminal#1#2{\tikz[baseline=(a.base)] \node (a) [terminal, bottom color=#2] {\small #1};}
%%%%% LINK %%%%%%%%%%%%%%%%%%%%%%%%%%%%%%%%%%%
\newcommand\nextsectionlink[1]{\addtocounter{section}\@ne
                               \hyperlink{section.\thechapter.\the\c@section}{#1}%
                               \addtocounter{section}{-\@ne}}
\newcommand\previoussectionlink[1]{\addtocounter{section}{-\@ne}
                                   \hyperlink{section.\thechapter.\the\c@section}{#1}%
                                   \addtocounter{section}{\@ne}}
\newcommand\previouschapterlink[1]{\addtocounter{chapter}{-\@ne}
                                   \hyperlink{chapter.\the\c@chapter}{#1}%
                                   \addtocounter{chapter}{\@ne}}
%%%%% REF %%%%%%%%%%%%%%%%%%%%%%%%%%%%%%%%%%%
\newcommand{\pageautoref}[1]{%
  \ifthenelse{\equal{\pageref{#1}}{\thepage}}%
    {\autoref{#1}}%
    {\autoref{#1}~[p.\pageref{#1}]}%
}
\newcommand{\pageeqref}[1]{%
  \ifthenelse{\equal{\pageref{#1}}{\thepage}}%
    {\eqref{#1}}%
    {\eqref{#1}~[p.\pageref{#1}]}%
}
%%%%% DECLAREMATHOPERATOR %%%%%%%%%%%%%%%%%%%%
\DeclareMathOperator{\IP}{Im}
\DeclareMathOperator{\RP}{Re}
%%%%% DECLAREROBUSTCOMMAND %%%%%%%%%%%%%%%%%%%%
\DeclareRobustCommand{\bDiv}{\nonscript\mskip-\medmuskip\mkern5mu\mathbin
  {\operator@font div}\penalty900
  \mkern5mu\nonscript\mskip-\medmuskip}
\DeclareRobustCommand{\pod}[1]{\allowbreak
  \if@display\mkern18mu\else\mkern8mu\fi(#1)}
\DeclareRobustCommand{\pDiv}[1]{\pod{{\operator@font div}\mkern6mu#1}}
\DeclareRobustCommand{\Div}[1]{\allowbreak\if@display\mkern18mu
  \else\mkern12mu\fi{\operator@font div}\,\,#1}
%%%%% PART FOR APPENDIX %%%%%%%%%%%%%%%%%%%%%%%%%%%%%%%%%%%%%%%%%
\newcommand{\Apart}{
  \addpart*{第\hx\hx\hx\thepart\hx\hx\hx 部の補遺}
  \addcontentsline{toc}{part}{第\hx\hx\thepart\hx\hx 部の補遺}
}
%%%%% DIM %%%%%%%%%%%%%%%%%%%%%%%%%%%%%%%%%%%
\newcommand{\hk}{\hspace{\kanjiskip}}
\newcommand{\hx}{\hspace{\xkanjiskip}}
%%%%% PRG # %%%%%%%%%%%%%%%%%%
\newcommand\MXOThickness{O110001} % X外側中心計測
\newcommand\MYOThickness{O110002} % Y外側中心計測
\newcommand\MXIWidth{O110011} % X内側中心計測
\newcommand\MYIWidth{O110012} % Y内側中心計測
\newcommand\MXface{O110101} % X基準面計測
\newcommand\MYcenterline{O111002} % 通り芯Y
\newcommand\MXcenterline{O111003} % 通り芯X(Z測定)
\newcommand\DLone{O210000} % 内面溝用 レベル1
\newcommand\DLtwoAC{O220001} % 内面溝用 レベル2 AC
\newcommand\DLtwoBD{O220002} % 内面溝用 レベル2 BD
\newcommand\DMLthreeAC{O230001} % 内面溝 測定用 レベル3 AC
\newcommand\DMLthreeBD{O230002} % 内面溝 測定用 レベル3 BD
\newcommand\KRecRight{O515000} % 端面用 rectangle  左回り
\newcommand\KICRLeft{O515200} % 内面取用 コーナーR 左回り
\newcommand\KOCRLeft{O515600} % 内面取用 コーナーR 左回り
\newcommand\DKLthreeAC{O630001} % 内面溝 加工用 レベル3 AC
\newcommand\DKLthreeBD{O630002} % 内面溝 加工用 レベル3 BD
\newcommand\OsensorOn{O910001} % タッチセンサーON
\newcommand\OsensorOff{O910002} % タッチセンサーOFF
%%%%% MACHINING NAME %%%%%%%%%%%%%%%%%%
\newcommand{\DMname}{ディンプルマシニング}
\newcommand{\MMname}{三菱マシニング}
%%%%% OTHER DEFINITION %%%%%%%%%%%%%%%%%%
\def\termblue#1{\terminal{\color{blue}\fontsize{8pt}{\z@}\textbf{#1}}{gray!25}\hskip.75zw}


%%%%% GEOMETRY %%%%%%%%%%%%%%%%%%%%%%%%%%%%
\geometry{
  a4paper, % paper size
  twoside,
  centering,
  textwidth={6.5in},
  includehead,  % include the head of the page
%  headheight = 13.6pt,
  includefoot,  % include the foot of the page
  top=15truemm,
  bottom=0truemm,
}
%%%%% HEYPERSETUP %%%%%%%%%%%%%%%%%%%%%%%%%%%%
\hypersetup{
%  pdfcreationdate=date,
%  pdfcreator={upLaTeX with hyperref}, % creator for PDF subjct field
  pdftitle={モールド関連 -主に幾何学的性質-}, % title for PDF subjct field
  pdfsubject={Mould-Related - Mainly Geometric Properties}, % text for PDF subjct field
  pdfauthor={Kurahashi Nobuaki},  % text for PDF Author field
  pdfkeywords={mould, mold}, % keywords
%  pdfproducer=producer, % dvipdfmx
  linktocpage=false,   % (if it is true) make page number, not text, be link on TOC, LOF and LOT
  pdfcenterwindow=false, % position the document window center of the screen
  pdffitwindow=true,     % resize document window to fit document size
  bookmarksnumbered=true,
  bookmarksopen=true, %bookmarks open
  pdfstartview={FitH}, % Fit, FitV, FitH, FitB
  pdfpagemode=UseThumbs, % set default mode of PDF display
  unicode=true,
  pdfencoding=unicode,   % PDFDocEncoding or Unicode
  colorlinks=true,     % color links
  linkcolor=ai,      % color of links
  urlcolor=ai,         % color of urls
  citecolor=sora,      % color of citation links
}
%%%%% LINESPREAD %%%%%%%%%%%%%%%%%%%%%%%%%%%%
\linespread{1.15}\selectfont
%%%%% PARINDENT %%%%%%%%%%%%%%%%%%%%%%%%%%%%
\setlength\parindent{12pt}
%\def \globalscale {0.83}
%%%%% UNIT LENGTH %%%%%%%%%%%%%%%%%%%%%%%%%%%%
\setlength{\unitlength}{1pt}
%%%%% FOOTNOTE %%%%%%%%%%%%%%%%%%%%%%%%%%%%
\interfootnotelinepenalty=10000
\counterwithout{footnote}{chapter}
\def\@makefnmark{\hbox{}\hbox{\@textsuperscript{\normalfont\@thefnmark}}\hbox{}}
\deffootnote[1em]{1em}{1em}{\textsuperscript{\thefootnotemark}}
\renewcommand\footnoterule{%
  \kern3\p@
  \hrule\@width.75\columnwidth
  \kern2.6\p@
}
\makesavenoteenv{twoCtable}
\makesavenoteenv{longtable}
%%%%% DISPLAYBREAK %%%%%%%%%%%%%%%%%%%%%%%%%
\allowdisplaybreaks
%%%%% CAPTION WIDTH %%%%%%%%%%%%%%%%%%%%%%%%%%%%
\captionsetup{width=.8\textwidth}
%%%%% NAME, AUTOREFNAME %%%%%%%%%%%%%%%%%%%%%%%%%%%%
%\renewcommand{\partautorefname}{part}  % part --> part
\renewcommand{\chapterautorefname}{章}  % chapter --> 章
%\renewcommand{\sectionautorefname}{節\!} % section --> 節
\renewcommand{\subsectionautorefname}{\sectionautorefname} % subsection --> section
\renewcommand{\subsubsectionautorefname}{節} % subsubsection --> section
%\renewcommand{\appendixname}{補\hskip0.5em 遺} % appendix --> 補 遺
\renewcommand{\appendixautorefname}{補遺\!} % appendix --> 補遺
\renewcommand{\figurename}{図}
\renewcommand{\figureautorefname}{\figurename} % figure --> 図
\renewcommand{\footnoteautorefname}{脚注}
\newcommand{\tcb@cnt@hosokuboxautorefname}{補足}
%\newcommand{\tcb@cnt@Columnautorefname}{Column}
%\newcommand{\subfigureautorefname}{\figureautorefname} % subfigure --> figure
%\renewcommand{\tableautorefname}{表}
%\newcommand{\subtableautorefname}{\tableautorefname}
%%%%% HEADER AND FOOTER %%%%%
\pagestyle{fancy}
\renewcommand{\chaptermark}[1]{\markboth{#1}{}}
\renewcommand{\headrulewidth}{1.5pt}
\renewcommand{\footrulewidth}{0pt}
\newcommand{\commonfoot}{
  \fancyfoot{}
  \fancyfoot[LO]{\tiny\customdate} % footer left fields for main odd pages
  \fancyfoot[RE]{\tiny\customdate} % footer right fields for main even pages
}

\fancypagestyle{front}{
  \fancyhead{} % clear all header fields
  \fancyhead[RO]{\thepage}
  \chead{\leftmark}
  \fancyhead[LE]{\thepage}
  \fancyhead[RE]{}
  \commonfoot
}
\fancypagestyle{main}{
  \fancyhead[C]{} % header center fields for all main pages
  \fancyhead[LO]{\nouppercase\rightmark} % header left fields for all main odd pages
  \fancyhead[RO]{$\nicefrac{\thepage\,}{\pageref{LastPage}}$} % header right fields for all main odd pages
  \fancyhead[RE]{{\sffamily\bfseries\thechapter.\hskip0.75em\nouppercase\leftmark}} % header right fields for all main even pages
  \fancyhead[LE]{$\nicefrac{\thepage\,}{\pageref{LastPage}}$} % header left fields for all main even pages
  \commonfoot
}
\fancypagestyle{plainheadfront}{
  \fancyhead{}
  \fancyhead[RO]{\thepage}
  \fancyhead[LE]{\thepage}
  \commonfoot
}
\fancypagestyle{plainhead}{
  \fancyhead{}
  \fancyhead[RO]{$\nicefrac{\thepage\,}{\pageref{LastPage}}$}
  \fancyhead[LE]{$\nicefrac{\thepage\,}{\pageref{LastPage}}$}
  \commonfoot
}
\renewcommand*\frontmatter{%
  \if@twoside\cleardoubleoddpage\else\clearpage\fi
  \@frontmattertrue\@mainmatterfalse\@backmatterfalse\pagenumbering{roman}\pagestyle{front}%
}
\renewcommand*\mainmatter{%
  \if@twoside\cleardoubleoddpage\else\clearpage\fi
  \@frontmatterfalse\@mainmattertrue\@backmatterfalse\pagenumbering{arabic}\pagestyle{main}%
}
\renewcommand*\backmatter{%
  \if@openright\cleardoubleoddpage\else\clearpage\fi\@frontmatterfalse\@mainmatterfalse\@backmattertrue\pagestyle{front}
}
%%%%% WATERMARK %%%%%%%%%%%%%%%%%%%%%%%%%%%%%%%%
\AddLayersToPageStyle{@everystyle@}{WatermarkLayer}
%%%%% SETLIST %%%%%%%%%%%%%%%%%%%%%%%%%%%%%%%%
\setlist[enumerate]{listparindent=\parindent, parsep=0pt, partopsep=0pt, topsep=3pt, itemsep=3pt}
%%%%% APPENDICES %%%%%%%%%%%%%%%%%%%%%%%%%%%%
\renewcommand{\setthesection}{\Alph{section}}
%%%%% EQUATION %%%%%%%%%%%%%%%%%%%%%%%%%%%%
\renewcommand{\theequation}{\thesection.\arabic{equation}}
\@addtoreset{equation}{section}
%%%%% STYLE OF PART %%%%%
\renewcommand\partpagestyle{empty}
\renewcommand*{\partformat}{\begin{gtfamily}第\hx\hx\hx\thepart\hx\hx\hx 部\end{gtfamily}}
%%%%% STYLE OF CHAPTER %%%%%
\renewcommand\chapterpagestyle{\if@frontmatter plainheadfront\else plainhead\fi}
%%%%% STYLE OF SECTION %%%%%
\setcounter{secnumdepth}{3}
%%%%% STYLE OF PARAGRAPH %%%%%%%%%%%%%%%%%%%%%
%for scrbook.cls
\RedeclareSectionCommand[%
  style=section,%
  level=4,%
  indent=0pt,%
  beforeskip=3.25ex \@plus1ex \@minus.2ex,%
  afterskip=0.1ex \@plus.1ex \@minus.1ex,% -1em から変更
  tocindentfollows=subsubsection,%
  tocstyle=section,%
  tocindent=10em,%
  tocnumwidth=5em,%
  font=\raggedsection\normalfont\sectfont\gtfamily\nobreak\sball{blue}~
]{paragraph}
%for book.cls
%\renewcommand\paragraph[1]{%
%  \@startsection{paragraph}{\paragraphnumdepth}{0pt}%
%  {3.25ex \@plus1ex \@minus.2ex}% \@plus, \@minusは伸び縮みできるスペースの長さ
%  {0.1ex\@plus.1ex \@minus.1ex}% ここが正だと改行されて、値だけ垂直スペースが入る
%  {\raggedsection\normalfont\sectfont\gtfamily\nobreak\size@paragraph\sball{blue}~}{#1}\noindent
%}
%%%%% STYLE OF SUBPARAGRAPH %%%%%%%%%%%%%%%%%%%%%
\RedeclareSectionCommand[%
  style=section,%
  level=5,%
  indent=0pt,% \scr@parindent から変更
  beforeskip=0.5ex \@plus1ex \@minus .2ex,% 3.25ex \@plus1ex \@minus .2ex から変更
  afterskip=0.1ex \@plus.1ex \@minus.1ex,% -1em から変更
  tocstyle=section,%
  tocindent=12em,%
  tocnumwidth=6em%
]{subparagraph}
%%%%% STYLE OF TABLE OF CONTENTS %%%%%
\setcounter{tocdepth}{3}
\renewcommand\contentsname{目 次}
%%%%% STYLE OF LIST OF FIGURES %%%%%
\renewcommand\listfigurename{\texorpdfstring{\hbox to 3em{図目次}}{図目次}}
%\setcounter{lofdepth}{2}
\renewcommand*\l@figure{\@dottedtocline{1}{1.5em}{3.2em}}
%\renewcommand*{\l@subfigure}{\@dottedxxxline{\ext@subfigure}{2}{4.7em}{2.3em}}
%%%%%%%%%%%%%%%%%%%%%%%%%%%%%%%%%%%%%%%%%%%%%%
%%%%% STYLE OF LISTINGS %%%%%
\lstset{%
%  language=,
  numbers=left,%
  numbersep=7pt,%
  numberstyle=\ttfamily\footnotesize,%
  breaklines=true,%
  frame=single,%
  columns=fixed,%
  basewidth=0.54em,%
  breakindent=3pt,%
  postbreak=\mbox{\textcolor{blue}{$\hookrightarrow$}\,},%
  keywords={IF, SIN, COS, SQRT, FIX, FUP, ABS},
  otherkeywords={GOTO, EQ, GE, WHILE},
  basicstyle=\ttfamily,%
  keywordstyle=\fontfamily{pcr}\selectfont\bfseries,%
  commentstyle=\footnotesize\color{black!60!green!100}\slshape,
  comment=[l]{(},
  alsoletter={},
}
%%%%% STYLE OF BIBLATEX %%%%%
\renewcommand{\bibname}{\texorpdfstring{\hbox to 4.1em{参考文献}}{参考文献}}
\ExecuteBibliographyOptions{
 sorting=nyt,
 hyperref=true,
 block=nbpar,
 subentry=true,
 citecounter=true,
}
\appto\bibfont{\footnotesize\setstretch{1.1}}
%%%%% TCBSET %%%%%%%%%%%%%%%%%%%%%%%%%%%%
\definecolor{myheadercolor}{rgb}{0.68, 0.85, 0.90}
\tcbset{%
  %%%%% COLUMNBOX STYLE %%%%%
  Tabularbox/.style={%
  },%
  %%%%% COLUMNBOX STYLE %%%%%
  Columnbox/.style={%
    after title=\hfill\termblue{\Columnname~\thetcbcounter},%
    fonttitle=\gtfamily\bfseries,%
    breakable,%
    enhanced jigsaw,%
    left=.5ex,%
    right=.5ex,%
    bicolor,%
    colbacklower=black!10!white,%
    before upper={%
      \setcounter{GlobalFootnote}{\value{footnote}}% GlobalFootnoteValue=footnoteValue
      \let\oldfootnote=\footnote% \oldfootnote=\footnote
      \def\footnote{\stepcounter{GlobalFootnote}\oldfootnote[\arabic{GlobalFootnote}]}% define \footnote to \footnote using counter GlobalFootnote
      \renewcommand\thempfootnote{\arabic{mpfootnote}}% arabic footnote
    },
    after upper={%
      \setcounter{footnote}{\value{GlobalFootnote}}% footnoteValue=GlobalFootnoteValue
      \let\footnote=\oldfootnote% \footnote=\oldfootnote
    },
  },%
  %%%%% FIGUREBOX STYLE %%%%%
  Figurebox/.style={%
    notitle,%
    height=\textwidth,
%    width=\textwidth,
    center upper,%
    center lower,%
    arc=5pt,%
    outer arc=2pt,%
    boxrule=1pt,%
    boxsep=3mm,%
    valign=center,%
    halign=center,%
    left=0pt,%
    right=0pt,%
    colback=green!3!white,%
    colframe=black!25!white,%
    before={\centering},
  },
  %%%%% HIGHLIGHT MATH STYLE %%%%%
%  highlight math/.style={%
%    enhanced,%
%    arc=2pt,%
%    boxrule=0pt,%
%    frame hidden,%
%    fuzzy halo=1pt with blue,%
%    left=0pt,%
%    right=0pt,%
%    top=.4mm,%
%    bottom=.4mm,%
%    colback=yellow!40!white,%
%  },%
  %%%%% HOSOKUBOX STYLE %%%%%
  hosokubox/.style={%
    title={\termblue{\hosokuname~\thetcbcounter}~},%
    attach title to upper,%
    breakable,%
    enhanced jigsaw,%
    size=fbox,%
    arc=0pt,%
    middle=1mm,%
    colback=hosoku,%
    colframe=hosoku,%
    drop lifted shadow={blue!100!white!50!},%
    skin first is subskin of={enhanced jigsaw}{no shadow},%
    skin middle is subskin of={enhanced jigsaw}{no shadow},%
    skin last is subskin of={enhanced jigsaw}{drop lifted shadow={blue!100!white!50!}},%
    segmentation style={draw=black!50!white},%
    after=\smallskip\noindent{\color{white}},%
  },
  %%%%% TWOCTABLEBOX STYLE %%%%%
  twoCtablebox/.style={%
    breakable,%
    enhanced,%
    fonttitle=\bfseries,%
    fontupper=\small\sffamily,%
    colframe=black!50!black,%
    colbacktitle=blue!10!white,%
    coltitle=black,%
    top=0pt,%
    bottom=0pt,%
    left=0pt,%
    right=0pt,%
    before upper={%
      \setcounter{GlobalFootnote}{\value{footnote}}% GlobalFootnoteValue=footnoteValue
      \let\oldfootnote=\footnote% \oldfootnote=\footnote
      \def\footnote{\stepcounter{GlobalFootnote}\oldfootnote[\arabic{GlobalFootnote}]}% define \footnote to \footnote using counter GlobalFootnote
      \renewcommand\thempfootnote{\arabic{mpfootnote}}% arabic footnote
      \renewcommand{\arraystretch}{1.2}% 行の高さを調整
      \setlength{\LTpre}{-3pt}%
      \setlength{\LTpost}{0pt}%
      \setlength{\LTleft}{0pt}%
      \setlength{\LTright}{0pt}%
      \begin{longtable}{@{}c|l@{\extracolsep{\fill}}c}%
    },%
    after upper={%
      \end{longtable}%
      \setcounter{footnote}{\value{GlobalFootnote}}% footnoteValue=GlobalFootnoteValue
      \let\footnote=\oldfootnote% \footnote=\oldfootnote
    },
  },
}
%%%%% TIKZSET %%%%%%%%%%%%%%%%%%%%%%%%%%%%
\tikzset{
  %%%%% SECTIONFORMAT STYLE %%%%%
%  sect/.style={signal, draw, text=white},
%  section/.style={sect, fill=konpeki!100!, signal to=east, inner sep=3pt},
%  subsection/.style={sect, fill=moegi!90!, signal to=nowhere, inner sep=3pt},
%  subsubsection/.style={sect, fill=sssec!100!, signal to=nowhere, inner sep=3pt},
  %%%%% TERMINAL STYLE %%%%%
  terminal/.style={%
    rectangle,%
    minimum size=10pt,%
    rounded corners=1.5mm,%
    thin,%
    draw=black!75,%
    top color=white,%
    font=\fontfamily{pplx},%
    inner sep=3pt,%
    inner xsep=3pt,%
    text height=1ex,%
    text depth=0pt,%
  },
  %%%%% BMATRIX STYLE %%%%%
%  every left delimiter/.style={xshift=.5em},
%  every right delimiter/.style={xshift=-.5em},
%  bmatrix/.style={matrix of math nodes, left delimiter=[, right delimiter=],},
}

\makeatother