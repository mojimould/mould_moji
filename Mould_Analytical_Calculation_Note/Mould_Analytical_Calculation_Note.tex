%!TEX encoding = UTF-8 Unicode

%!TEX root = ../Mould_Analytical_Calculation_Note.tex

%%%%%%%%%%%%%%%%%%%%%%%%%%%%%%%%%%%%%%%%%%%%%%
%%%%% DOCUMENTCLASS %%%%%%%%%%%%%%%%%%%%%%%%%%
%%%%%%%%%%%%%%%%%%%%%%%%%%%%%%%%%%%%%%%%%%%%%%
\documentclass[11pt]{scrbook}

%!TEX root = ../Mould_Analytical_Calculation_Note.tex

% Encoding and Language Settings
\usepackage[utf8]{inputenc} % Allows input of utf8 characters.
\usepackage{babel} % Multilingual support for LaTeX.
\usepackage[japanese]{pxbabel} % Japanese support for babel.
\usepackage[style=english]{csquotes} % Context sensitive quotation facilities.

% Font and Math Settings
%\usepackage[T1]{fontenc} % for _ in \jobname
\usepackage{amsmath, amssymb} % Enhanced math support in LaTeX.
\usepackage{eulervm} % Euler virtual math fonts.

%%%%%%%%%%%%%%%%%%%%%%%%%%%%%%%%%%%%%%%%%%%%%%
%%%%% DOTFILL %%%%%%%%%%%%%%%%%%%%%%%%%%%%%%%%
%%%%%%%%%%%%%%%%%%%%%%%%%%%%%%%%%%%%%%%%%%%%%%
%\patchcmd{\@dottedtocline}{\hbox{.}}{\hbox{$\cdot$}}{}{}
% Layout and Table Settings
\usepackage{geometry} % Provides an easy and flexible user interface to customize page layout.
\usepackage{array} % Extends array and tabular environments.
\usepackage{longtable} % Allows tables to break across pages.
\usepackage{colortbl} % Adds color to LaTeX tables.

% Graphics and Color Settings
\usepackage[dvipsnames]{xcolor} % Provides driver-independent color extensions for LaTeX and pdfLaTeX.
  %%% tikz, pgf %%%
\def\pgfsysdriver{pgfsys-dvipdfmx.def} % Specifies the driver for PGF, a lower-level language for creating graphics.
\usepackage{pgfplots} % A tool to create 2D and 3D plots in LaTeX.
\pgfplotsset{compat=1.18} % Specifies the version of pgfplots to use for compatibility.

% Box Settings
\usepackage[most]{tcolorbox} % Provides an environment for colored and framed text boxes with a heading line.
\usepackage{tikzpagenodes}

% Bibliography Settings
\usepackage[backend=biber, style=numeric-comp]{biblatex} % references
\addbibresource{./preamble_MACN/reference_MACN.bib}

% List Settings
\usepackage{enumitem} % Control layout of itemize, enumerate, description.

% Footnote
\usepackage{footnotehyper} % Improve on LaTeX's footnote handling.

% Make index
\usepackage{makeidx} % expand makeidx
\makeindex

% Hyperlink Settings
\usepackage[dvipdfmx]{hyperref} % Adds support for hyperlinks.
\usepackage{pxjahyper} % Adjusts hyperref for pLaTeX and upLaTeX.

% Header and Footer Settings
\usepackage{scrlayer-fancyhdr} % Combines the features of fancyhdr with KOMA-Script's scrlayer.

% Date and Time
\usepackage{datetime} % Date and time handling.
\usepackage{bxjaholiday} % Support for Japanese holidays.

% Page Layout
\usepackage{lastpage} % Reference last page for Page N of M type footers.
\usepackage{pdflscape} % Make landscape pages display as landscape.
\usepackage{afterpage} % Execute command after the next page break.

% Table of Contents and Headers
\usepackage{titletoc} % Alternative headings for toc/lof/lot.
\usepackage{appendix} % Extra control of appendices.

% Caption
\usepackage{caption} % Customizes captions in floating environments.
%\usepackage{subcaption} % Support for sub-captions.

% Line Spacing
\usepackage{setspace} % Set space between lines.

% Citation
%\usepackage{cite} % Improved citation handling.
%\usepackage{cleveref} % Intelligent cross-referencing.

% Conditional Processing
\usepackage{ifthen} % Conditional commands.

% Loop
\usepackage{pgffor} % Foreach loop structure.
%\usepackage{datatool}
%\usepackage{calc}

% Arrow
\usepackage{extarrows} % Extra Arrows beyond those provided in AMSmath.

% Units
\usepackage{units} % Typeset units.

% Box Adjustment
\usepackage{adjustbox} % Graphics package-alike macros for "general" boxes.

% lines for table
\usepackage{booktabs}
\usepackage{cellspace}

% List Display
\usepackage{jvlisting} % For including code listings with Japanese comments.

% Other
\usepackage{scrhack} % Fix koma-script interaction with other packages.

%%%%% TIKZ LIBRARY etc %%%%%%%%%%%%%%%%%%%%%%%%%%%%
% Arrows
%\usetikzlibrary{arrows} % Arrow tip library.
%\usetikzlibrary{arrows.meta} % Advanced arrow tip library.
% Calculations
\usetikzlibrary{calc} % Coordinate calculations.
% Decorations
%\usetikzlibrary{decorations} % General decoration library.
%\usetikzlibrary{decorations.fractals} % Fractal decorations.
%\usetikzlibrary{decorations.markings} % Arbitrary markings on paths.
%\usetikzlibrary{decorations.pathmorphing} % "Morphing" decorations.
%\usetikzlibrary{decorations.shapes} % Shape decorations.
%\usetikzlibrary{decorations.text} % Text decorations.
%\usepgfmodule{decorations} % Decoration library module.
% Matrix
%\usetikzlibrary{matrix} % Matrix library.
% Plot Marks
%\usetikzlibrary{plotmarks} % Plot mark library.
% Positioning
%\usetikzlibrary{positioning} % Improved positioning of nodes.
% Shadows
%\usetikzlibrary{shadows} % Shadow library.
% Shapes
%\usetikzlibrary{shapes} % Shape library.
% Trees
%\usetikzlibrary{trees} % Tree library.
% tcolorbox Libraries
\tcbuselibrary{listings} % Enables the use of listings within tcolorboxes.
\tcbuselibrary{breakable} % Allows tcolorboxes to break across pages.
%\tcbuselibrary{skins} % Provides additional skins to customize the appearance of tcolorboxes.
%\tcbuselibrary{theorems} % Provides theorem environments within tcolorboxes.
%!TEX root = ../Mould_Analytical_Calculation_Note.tex

\makeatletter

%!TEX root = ../Mould_Analytical_Calculation_Note.tex

%%%%%%%%%%%%%%%%%%%%%%%%%%%%%%%%%%%%%%%%%%%%%%%%%%%%%%%%%%%%%%%%%%
\def\mouldCoordinate{%
\begin{tikzpicture}
% 値の計算
\pgfmathsetmacro{\Ax}{9.6} %A:T_iのx座標
\pgfmathsetmacro{\Ay}{3.5} %A:T_iのy座標
\pgfmathsetmacro{\Bx}{2.0+(\Ax)} %B:T_oのx座標
\pgfmathsetmacro{\Cy}{-4.2}                  %C:B_iのy座標
\pgfmathsetmacro{\Ri}{sqrt((\Ax)^2+(\Ay)^2)} %R_iの長さ
\pgfmathsetmacro{\Cx}{sqrt((\Ri)^2-(\Cy)^2)} %C:B_iのx座標
\pgfmathsetmacro{\Ro}{sqrt((\Bx)^2+(\Ay)^2)} %R_oの長さ
\pgfmathsetmacro{\Dx}{sqrt((\Ro)^2-(\Cy)^2)} %D:B_oのx座標
\pgfmathsetmacro{\Rc}{(\Ri+\Ro)/2}           %R_cの長さ
\pgfmathsetmacro{\Ex}{sqrt((\Rc)^2-(\Ay)^2)} %E:湾曲中心線トップ端のx座標
\pgfmathsetmacro{\Fx}{sqrt((\Rc)^2-(\Cy)^2)} %F:湾曲中心線ボトム端のx座標
\pgfmathsetmacro{\Ub}{2.4}                   %Ub:受板-モールド接点のy座標
\pgfmathsetmacro{\Ux}{sqrt((\Ri)^2-(\Ub)^2)} %Ux:受板-モールド接点のx座標
\pgfmathsetmacro{\Uxo}{sqrt((\Ro)^2-(\Ub)^2)} %Ux:受板-モールド接点のx座標
\pgfmathsetmacro{\Hx}{1+(\Ri)} %H:テーブルの中心
\pgfmathsetmacro{\Ix}{1.90} %I:テーブルx方向の長さの半分
% 座標系を描画
\draw[-latex, dotted] (-0.4, 0) -- (14.5, 0) node[below] {\textbf{Re}};
\draw[-latex, dotted] (0, -4) -- (0, 4) node[below right] {\textbf{Im}};
% 座標を定義
\coordinate (O) at (  0, 0); % 原点
\coordinate (A) at (\Ax, \Ay); %
\coordinate (B) at (\Bx, \Ay);
\coordinate (C) at (\Cx, \Cy);
\coordinate (D) at (\Dx, \Cy);
\coordinate (E) at (\Ex, \Ay);
\coordinate (F) at (\Fx, \Cy);
\coordinate (Rc) at (\Rc, 0);
\coordinate (Ri) at (\Ri, 0);
\coordinate (Ro) at (\Ro, 0);
\coordinate (Ut) at (\Ux, \Ub);
\coordinate (Ub) at (\Ux, -\Ub);
\coordinate (Uto) at (\Uxo, \Ub);
\coordinate (Ubo) at (\Uxo, -\Ub);
\coordinate (Tal) at (\Hx-\Ix, \Ub);
\coordinate (Tbl) at (\Hx-\Ix, -\Ub);
\coordinate (Tar) at (\Hx+\Ix, \Ub);
\coordinate (Tbr) at (\Hx+\Ix, -\Ub);
\coordinate (Lt) at (\Hx+\Ix+0.4, \Ub);
\coordinate (Lb) at (\Hx+\Ix+0.4, -\Ub);
\coordinate (Lc) at (\Hx+\Ix+0.4, 0);
\coordinate (Fc) at (\Hx+\Ix+0.9, 0);
\coordinate (Ft) at (\Hx+\Ix+0.9, \Ay);
\coordinate (Fb) at (\Hx+\Ix+0.9, \Cy);
% 点を描画
\fill (O) circle (2pt);
\fill (A) circle (2pt);
\fill (B) circle (2pt);
\fill (C) circle (2pt);
\fill (D) circle (2pt);
\fill (E) circle (2pt);
\fill (Rc) circle (2pt);
\fill (Ro) circle (2pt);
\fill (Ri) circle (2pt);
\fill (Ut) circle (2pt);
\fill (Ub) circle (2pt);
% 点にラベルを付ける
\node at (O) [below left] {O};
\node at (A) [above] {T$_\mathrm i$};
\node at (B) [above] {T$_\mathrm o$};
\node at (C) [below] {B$_\mathrm i$};
\node at (D) [below] {B$_\mathrm o$};
\node at (Rc) [above right] {$R_\mathrm c$};
\node at (Ro) [below right] {$R_\mathrm o$};
\node at (Ri) [below left] {$R_\mathrm i$};
\node at (Ut) [right] {U$_\mathrm T$};
\node at (Ub) [right] {U$_\mathrm B$};
% モールド外形
\draw[line width=1pt, fill=ffwwqq, fill opacity=0.1]
  let \p1=(A), \p2=(C), \p3=(B), \p4=(D), \n1={atan2(\y1,\x1)}, \n2={atan2(\y2,\x2)}, \n3={atan2(\y3,\x3)}, \n4={atan2(\y4,\x4)}
    in (A) -- (B) -- (\n3:\Ro) arc (\n3:\n4:\Ro) -- (C) -- (\n2:\Ri) arc (\n2:\n1:\Ri) -- cycle;
% モールド中心線
\draw[dotted, line width=1pt] let \p1=(E), \p2=(F), \n1={atan2(\y1,\x1)}, \n2={atan2(\y2,\x2)}
  in (\n1:\Rc) arc (\n1:\n2:\Rc);
% テーブル
\draw (Ut) -- (Tal) -- (Tbl) -- (Ub);
\draw (Uto) -- (Tar) -- (Tbr) -- (Ubo);
\draw[dotted] (Tar) -- (Lt);
\draw[dotted] (Tbr) -- (Lb);
\draw[dotted] (Tar) -- (Lt);
\draw[latex-latex, line width=1pt] (Lc) -- (Lt) node[midway, right] {$l$};
\draw[latex-latex, line width=1pt] (Lc) -- (Lb) node[midway, right] {$l$};
% 振分け
\draw[dotted] (B) -- (Ft);
\draw[dotted] (D) -- (Fb);
\draw[latex-latex, line width=1pt] (Fc) -- (Ft) node[midway, right] {$f_\mathrm T$};
\draw[latex-latex, line width=1pt] (Fc) -- (Fb) node[midway, right] {$f_\mathrm B$};
% 半径
\draw[dotted, line width=1pt] (O) -- (A) node[midway, above left] {$R_\mathrm i$} ;
\draw[dotted, line width=1pt] (O) -- (B) node[midway, below right] {$R_\mathrm o$} ;
\draw[dotted, line width=1pt] (O) -- (Ub) node[midway, above right] {$R_\mathrm i$} ;
% 角度
\draw[line width=1pt, fill=ffwwqq, fill opacity=0.1]
  let \p1=(Ri), \p2=(A), \n1={atan2(\y1,\x1)}, \n2={atan2(\y2,\x2)}
   in (\n1:2) arc (\n1:\n2:2) node[midway, right, opacity=1] {$\alpha_{\mathrm T_\mathrm i}$} -- (O);
\draw[line width=1pt, fill=qqzzqq, fill opacity=0.1]
  let \p1=(Ro), \p2=(B), \n1={atan2(\y1,\x1)}, \n2={atan2(\y2,\x2)}
  in (\n1:3.2) arc (\n1:\n2:3.2) node[midway, right, opacity=1] {$\alpha_{\mathrm T_\mathrm o}$} -- (O) -- cycle ;
\draw[line width=1pt, fill=wwqqcc, fill opacity=0.1]
  let \p1=(Ri), \p2=(Ub), \n1={atan2(\y1,\x1)}, \n2={atan2(\y2,\x2)}
  in (\n1:2.7) arc (\n1:\n2:2.7) node[midway, right, opacity=1] {$\alpha_{\mathrm U_\mathrm B}$} -- (O);
\end{tikzpicture}%
}
%%%%%%%%%%%%%%%%%%%%%%%%%%%%%%%%%%%%%%%%%%%%%%%%%%%%%%%%%%%%%%%%%%%%%%%%%%%%%

%%%%%%%%%%%%%%%%%%%%%%%%%%%%%%%%%%%%%%%%%%%%%%
%%%%% DATE %%%%%%%%%%%%%%%%%%%%%%%%%%%%%%%%%%%
%%%%%%%%%%%%%%%%%%%%%%%%%%%%%%%%%%%%%%%%%%%%%%
\newcommand{\customtoday}{\the\year/\two@digits{\the\month}/\two@digits{\the\day}}
\newcommand{\customdate}{\customtoday\ \currenttime\ (\jadayofweek{\the\year}{\the\month}{\the\day})}
\newcommand{\customtodayap}{\ifnum\currenthour<12 \customtoday\,a.m.\else\customtoday\,p.m.\fi}
%%%%%%%%%%%%%%%%%%%%%%%%%%%%%%%%%%%%%%%%%%%%%%
%%%%% NEWIF %%%%%%%%%%%%%%%%%%%%%%%%%%%%%%%%%%
%%%%%%%%%%%%%%%%%%%%%%%%%%%%%%%%%%%%%%%%%%%%%%
\newif\if@backmatter%\@backmattertrue
\newif\if@frontmatter%\@frontmattertrue
\newif\if@appendix%\@appendixfalse
%%%%%%%%%%%%%%%%%%%%%%%%%%%%%%%%%%%%%%%%%%%%%%
%%%%% DIMENSION %%%%%%%%%%%%%%%%%%%%%%%%%%%%%%
%%%%%%%%%%%%%%%%%%%%%%%%%%%%%%%%%%%%%%%%%%%%%%
\setlength{\kanjiskip}{0.0pt plus 0.4pt minus 0.5pt}
\setlength{\xkanjiskip}{2.40555pt plus 1.0pt minus 1.0pt}
\newcommand{\hk}{\hspace{\kanjiskip}} % \kanjiskip at 12pt scrbook.cls
\newcommand{\hx}{\hspace{\xkanjiskip}} % \xkanjiskip at 12pt scrbook.cls
%%%%%%%%%%%%%%%%%%%%%%%%%%%%%%%%%%%%%%%%%%%%%%
%%%%% CDOTFILL LIKE TOC %%%%%%%%%%%%%%%%%%%%%%
%%%%%%%%%%%%%%%%%%%%%%%%%%%%%%%%%%%%%%%%%%%%%%
\newcommand\cdotfill{%
  \leaders\hbox{$\m@th\mkern\@dotsep mu\hbox{$\cdot$}\mkern \@dotsep mu$}\hfill\kern\z@
}
%%%%%%%%%%%%%%%%%%%%%%%%%%%%%%%%%%%%%%%%%%%%%%
%%%%% COLOR %%%%%%%%%%%%%%%%%%%%%%%%%%%%%%%%%%
%%%%%%%%%%%%%%%%%%%%%%%%%%%%%%%%%%%%%%%%%%%%%%
\definecolor{ai}     {rgb}{0.2039, 0.3765, 0.4314}
\definecolor{kon}    {rgb}{0.0000, 0.2000, 0.4000}
\definecolor{konpeki}{rgb}{0.0902, 0.5098, 0.7333}
\definecolor{moegi}  {rgb}{0.3020, 0.5961, 0.1882}
\definecolor{sssec}  {rgb}{0.7333, 0.5, 0.7333}
\definecolor{sora}   {rgb}{0.1451, 0.7216, 0.8039}
\definecolor{sumire} {rgb}{0.3882, 0.2157, 0.5922}
\definecolor{wwqqcc} {rgb}{0.4, 0, 0.8}
\definecolor{qqzzqq} {rgb}{0, 0.6, 0}
\definecolor{ffwwqq} {rgb}{1, 0.4, 0}
%%%%%%%%%%%%%%%%%%%%%%%%%%%%%%%%%%%%%%%%%%%%%%
%%%%% REF %%%%%%%%%%%%%%%%%%%%%%%%%%%%%%%%%%%%
%%%%%%%%%%%%%%%%%%%%%%%%%%%%%%%%%%%%%%%%%%%%%%
%%%%% AUTOREFNAME %%%%%%%%%%%%%%%%%%%%%%%%%%%%
%\newcommand{\subfigureautorefname}{\figureautorefname} % subfigure --> figure
%\newcommand{\subtableautorefname}{\tableautorefname}
%%%%% PAGEREF %%%%%%%%%%%%%%%%%%%%%%%%%%%%%%%%%%%
\newcommand{\pageautoref}[1]{%
  \ifthenelse{\equal{\pageref{#1}}{\thepage}}%
    {\autoref{#1}}%
    {\autoref{#1}~[p.\pageref{#1}]}%
}
\newcommand{\pageeqref}[1]{%
  \ifthenelse{\equal{\pageref{#1}}{\thepage}}%
    {\eqref{#1}}%
    {\eqref{#1}~[p.\pageref{#1}]}%
}
%%%%%%%%%%%%%%%%%%%%%%%%%%%%%%%%%%%%%%%%%%%%%%
%%%%% FOR LISTINGS %%%%%%%%%%%%%%%%%%%%%%%%%%%
%%%%%%%%%%%%%%%%%%%%%%%%%%%%%%%%%%%%%%%%%%%%%%
%%%%% CAPTOINOF %%%%%%%%%%%%%%%%%%%%%%%%%%%%%%
\setlength{\abovecaptionskip}{0pt}
\newcommand{\modcaptionof}[2]{%
  \captionof{#1}{%
    \csname #1name\endcsname\thechapter.%
    \ifnum\value{#1}<10 0\fi
      \arabic{#1}. #2}}
%%%%%%%%%%%%%%%%%%%%%%%%%%%%%%%%%%%%%%%%%%%%%%
%%%%% FOR TOC %%%%%%%%%%%%%%%%%%%%%%%%%%%%%%%%
%%%%%%%%%%%%%%%%%%%%%%%%%%%%%%%%%%%%%%%%%%%%%%
%%%%% TOC LINE %%%%%%%%%%%%%%%%%%%%%%%%%%%%%%%
\newcommand\PartSeparateline[1]{\addtocontents{#1}{\protect\par\protect\hrulefill\protect\par\protect\vspace*{-10pt}}}%
\newcommand\tocAPartSeparateline[3]{\addtocontents{#1}{\protect\par\protect\vspace*{#2}\protect\hrule width 0.5\linewidth\protect\par\protect\vspace*{#3}}}%
%\newcommand\tableAPartSeparateline{\tocAPartSeparateline{lot}{2pt}{2pt}}
%%%%% PART FOR APPENDIX %%%%%%%%%%%%%%%%%%%%%%%%%%%%%%%%%%%%%%%%%
\newcommand{\Appendixpart}{
  \part*{\partname\ \thepart\hx の補遺}
  \addcontentsline{toc}{part}{\partname\ \thepart\hx の補遺}
}
%%%%%%%%%%%%%%%%%%%%%%%%%%%%%%%%%%%%%%%%%%%%%%
%%%%% AUTO LABELING %%%%%%%%%%%%%%%%%%%%%%%%%%
%%%%%%%%%%%%%%%%%%%%%%%%%%%%%%%%%%%%%%%%%%%%%%
%%%%% FOR CHAPTER OR APPENDIX %%%%%%%%%%%%%%%%%%%%%%%%%%%%%
\newcommand{\modHeadchapter}[2][]{%
  \ifx\relax#1\relax
    \chapter{#2}%
  \else%
    \chapter[#1]{#2}%
  \fi
  \ifx\@chapapp\appendixname
    \label{app:\thechapter}%
  \else
    \label{chap:\thechapter}%
  \fi
  \indentspace%
}
%%%%% FOR SECTION %%%%%%%%%%%%%%%%%%%%%%%%%%%%%
%%%%% TO LABEL SECTION  %%%%%%
\newcommand{\modHeadsection}[2][]{%
  \ifx\relax#1\relax
    \section{#2}%
  \else%
    \section[#1]{#2}%
  \fi
  \label{sec:\thesection}%
}
%%%%%%%%%%%%%%%%%%%%%%%%%%%%%%%%%%%%%%%%%%%%%%
%%%%% NEWCOLORBOX %%%%%%%%%%%%%%%%%%%%%%%%%%%%
%%%%%%%%%%%%%%%%%%%%%%%%%%%%%%%%%%%%%%%%%%%%%%
\newcounter{GlobalFootnote}% difine a counter Global Footnote
%%%%% PART %%%%%
\newcommand{\tablePartnname}{\thepart}
\newtcolorbox[auto counter, number within=part]{tablePart}[2][]{Columnbox, title={\termblue{\tablePartnname:#2}}, #1, after title={}}
%%%%% COLUMN %%%%%
\newcommand{\Columnname}{Column}
\newtcolorbox[auto counter, number within=chapter]{Column}[2][]{Columnbox, title={#2}, #1}
%\newcommand{\tcb@cnt@Columnautorefname}{Column}
%%%%% FIGBOX %%%%%
\newtcolorbox{Figbox}[1][]{Figurebox, #1}
%%%%% TABBOX %%%%%
\newtcolorbox{Tabbox}[1][]{Tabularbox, #1}
%\renewcommand{\tableautorefname}{表}
%%%%% HOSOKU %%%%%
\newcommand{\hosokuname}{補}
\definecolor{hosoku}{cmyk}{0, 0, 0, .15}
\newtcolorbox[auto counter, number within=chapter]{hosoku}[1][]{hosokubox, #1}
\newcommand{\tcb@cnt@hosokuautorefname}{補足}
%%%%% TWOCTABLE %%%%%
\newtcolorbox[number within=chapter]{twoCtable}[2][]{twoCtablebox, title={#2}, #1}
%%%%% OTHER TIKZ DEFINITION %%%%%%%%%%%%%%%%%%
\def\pgfname{\textsc{pgf}}
\def\tikzname{Ti\textit{k}Z}
\tikzfading[name=fade ball, inner color=transparent!60, outer color=transparent!30]
\def\sball#1{\tikz \shade [ball color=#1, path fading=fade ball] (0,0) circle (.7ex);}
\def\terminal#1#2{\tikz[baseline=(a.base)] \node (a) [terminal, bottom color=#2] {\small #1};}
\def\termblue#1{\terminal{\color{blue}\fontsize{8pt}{0pt}\textbf{#1}}{gray!25}\hskip6pt}
%%%%%%%%%%%%%%%%%%%%%%%%%%%%%%%%%%%%%%%%%%%%%%
%%%%% DECLARE %%%%%%%%%%%%%%%%%%%%%%%%%%%%%%%%
%%%%%%%%%%%%%%%%%%%%%%%%%%%%%%%%%%%%%%%%%%%%%%
%%%%% DECLAREMATHOPERATOR %%%%%%%%%%%%%%%%%%%%
\DeclareMathOperator{\IP}{Im}
\DeclareMathOperator{\RP}{Re}
%%%%% DECLAREROBUSTCOMMAND %%%%%%%%%%%%%%%%%%%%
\DeclareRobustCommand{\bDiv}{\nonscript\mskip-\medmuskip\mkern5mu\mathbin
  {\operator@font div}\penalty900
  \mkern5mu\nonscript\mskip-\medmuskip}
\DeclareRobustCommand{\pod}[1]{\allowbreak
  \if@display\mkern18mu\else\mkern8mu\fi(#1)}
\DeclareRobustCommand{\pDiv}[1]{\pod{{\operator@font div}\mkern6mu#1}}
\DeclareRobustCommand{\Div}[1]{\allowbreak\if@display\mkern18mu
  \else\mkern12mu\fi{\operator@font div}\,\,#1}
%%%%% DECLARENEWTOC %%%%%%%%%%%%%%%%%%%%
\DeclareNewTOC[owner=\jobname, name=Part]{lop}
%%%%% DECLARENEWLAYER %%%%%%%%%%%%%%%%%%%%
\newcommand{\setallpageWatermark}[3]{%
  \DeclareNewLayer[
    foreground,
    page,
    contents={%
      \begin{tikzpicture}[remember picture, overlay]
        \node[rotate=#2, scale=#3, anchor=center, opacity=0.05, text=lightgray] at (current page.center){\scshape#1};%
      \end{tikzpicture}%
    }%
  ]{WatermarkLayer}
}
%%%%%%%%%%%%%%%%%%%%%%%%%%%%%%%%%%%%%%%%%%%%%%
%%%%% LINK %%%%%%%%%%%%%%%%%%%%%%%%%%%%%%%%%%%
%%%%%%%%%%%%%%%%%%%%%%%%%%%%%%%%%%%%%%%%%%%%%%
%%%%% LINK NAME %%%%%%%%%%%%%%%%%%
\newcommand{\linkLaTeX}{\href{https://www.latex-project.org/}{\LaTeX}}
\newcommand{\linkLaTeXProject}{\href{https://www.latex-project.org/}{\LaTeX\ Project}}
\newcommand{\linkTeXLive}{\href{https://tug.org/texlive/}{\TeX\ Live}}
\newcommand{\linkTeXUsersGroup}{\href{http://www.tug.org/}{\TeX\ Users Group}}
\newcommand{\linkBibLaTeX}{\href{https://ctan.org/pkg/biblatex}{Bib\LaTeX}}
\newcommand{\linkBiber}{\href{https://ctan.org/pkg/biber}{Biber}}
\newcommand{\linkPhilippLehman}{\href{https://www.su.se/english/profiles/plehm-1.218839}{Philipp Lehman}}
\newcommand{\linkPGFTikZ}{\href{https://github.com/pgf-tikz/pgf}{\pgfname/\tikzname}}
\newcommand{\linkTillTantau}
           {\href{http://www.tcs.uni-luebeck.de/en/mitarbeiter/tantau/cv/index.html}{Till Tantau}}
\newcommand{\linkTeXStudio}{\href{https://texstudio.org/}{\TeX\ Studio}}
\newcommand{\linkBenitoVdZ}{\href{https://dev.to/benibela}{Benito van der Zander}}
\newcommand{\linkVSCode}{\href{https://code.visualstudio.com/}{VSCode}}
\newcommand{\linkExcel}{\href{https://www.microsoft.com/ja-jp/microsoft-365/excel}{Excel}}
\newcommand{\linkBingChat}{\href{https://www.bing.com/}{Bing Chat}}
\newcommand{\linkMicrosoftCorp}{\href{https://www.microsoft.com/}{Microsoft Corporation}}
\newcommand{\linkPython}{\href{https://www.python.org/}{Python}}
\newcommand{\linkPythonSF}{\href{https://www.python.org/psf-landing/}{Python Software Foundation}}
\newcommand{\linkGitHub}{\href{https://github.com/}{GitHub}}
\newcommand{\linkGitHubInc}{\href{https://github.com/}{GitHub, Inc}}
\newcommand{\linkDocker}{\href{https://www.docker.com/}{Docker}}
\newcommand{\linkDockerInc}{\href{https://www.docker.com/}{Docker, Inc}}
\newcommand{\linkUbuntu}{\href{https://ubuntu.com/}{Ubuntu}}
\newcommand{\linkCanonicalLtd}{\href{https://canonical.com/}{Canonical Ltd}}
\newcommand{\linkSQLite}{\href{https://www.sqlite.org/}{SQLite}}
\newcommand{\linkSQLiteConsortium}{\href{https://www.sqlite.org/consortium.html}{SQLite Consortium}}
\newcommand{\linkChatGPT}{\href{https://openai.com/chatgpt}{ChatGPT}}
\newcommand{\linkOpenAI}{\href{https://www.openai.com/}{OpenAI}}
%\newcommand\nextsectionlink[1]{\addtocounter{section}\@ne
%                               \hyperlink{section.\thechapter.\the\c@section}{#1}%
%                               \addtocounter{section}{-\@ne}}
%\newcommand\previoussectionlink[1]{\addtocounter{section}{-\@ne}
%                                   \hyperlink{section.\thechapter.\the\c@section}{#1}%
%                                   \addtocounter{section}{\@ne}}
%\newcommand\previouschapterlink[1]{\addtocounter{chapter}{-\@ne}
%                                   \hyperlink{chapter.\the\c@chapter}{#1}%
%                                   \addtocounter{chapter}{\@ne}}
%%%%%%%%%%%%%%%%%%%%%%%%%%%%%%%%%%%%%%%%%%%%%%
%%%%% OTHER DEFINITION %%%%%%%%%%%%%%%%%%%%%%%
%%%%%%%%%%%%%%%%%%%%%%%%%%%%%%%%%%%%%%%%%%%%%%
%%%%% TO GET CHAPTER TITLE %%%%%%
\newcommand\Chaptername{} % initialize \Chaptername
\let\old@chapter\@chapter
\def\@chapter[#1]#2{\gdef\Chaptername{#2}\old@chapter[#1]{#2}}
%%%%% TO GET SECTION TITLE %%%%%%
\newcommand\Sectionname{} % initialize \Sectionname
\let\Sectionmark\sectionmark
\def\sectionmark#1{\def\Sectionname{#1}\Sectionmark{#1}}
%%%%% TO PRG NAME %%%%%%
\newcommand\MainEx{O1916} % an example for main program
\newcommand\MXOThickness{O110001} % X外側中心計測
\newcommand\MYOThickness{O110002} % Y外側中心計測
\newcommand\MXIWidth{O130001} % X内側中心計測
\newcommand\MYIWidth{O130002} % Y内側中心計測
\newcommand\MXface{O140001} % 外削X基準面計測
\newcommand\MYcenterline{O150002} % 通り芯Y
\newcommand\MXcenterline{O150003} % 通り芯X(Z測定)
\newcommand\DLone{O210003} % 内面溝用 レベル1
\newcommand\DLtwoAC{O220001} % 内面溝用 レベル2 AC
\newcommand\DLtwoBD{O220002} % 内面溝用 レベル2 BD
\newcommand\DMLthreeAC{O230001} % 内面溝 測定用 レベル3 AC
\newcommand\DMLthreeBD{O230002} % 内面溝 測定用 レベル3 BD
\newcommand\KTanmenRight{O410000} % 端面用 右回り
\newcommand\KGaisakuRLeft{O420000} % 外側 左回り
\newcommand\KMizoConerLeft{O430000} % 溝用 左回り
\newcommand\KSotoMentoriRLeft{O440000} % 外側面取用 左回り
\newcommand\KUchiMentoriRLeft{O450000} % 内側面取用 左回り
\newcommand\KOLeft{O490005} % 外 左回り
\newcommand\DKLthreeAC{O530001} % 内面溝 加工用 レベル3 AC
\newcommand\DKLthreeBD{O530002} % 内面溝 加工用 レベル3 BD
\newcommand\OsensorOn{O910001} % タッチセンサーON
\newcommand\OsensorOff{O910002} % タッチセンサーOFF
%%%%% OTHER DEFINITION %%%%%%%%%%%%%%%%%%
\newcommand\ttNum{\ifmmode{\text{\texttt\#}}\else\texttt\#\fi}
\newcommand\cf{{\itshape cf.\,}}
%%%%% MACHINING NAME %%%%%%%%%%%%%%%%%%
\newcommand{\DMname}{Dマシニング}
\newcommand{\MMname}{Mマシニング}
%%%%% FOR PART WITH TABLE %%%%%%%%%%%%%%%%%%
\newcommand{\tPart}[4][]{
  \begingroup
  \@openrightfalse
  \part{#2}
  \addxcontentsline{lop}{part}{\protect\numberline{\thepart}#2}%
  \ifx#1\relax\else\addxcontentsline{#1}{part}{\protect\numberline{\thepart}#2}\fi
  \thispagestyle{empty}
  \vspace*{0.1\textheight}%
  \begin{tablePart}{#3}
  #4
  \end{tablePart}%
  \@openrighttrue
  \endgroup
}
%%%%% NOTATION TABLE %%%%%%%%%%%%%%%%%%
\newenvironment{Notation}[2]
{%
  \rowcolors{3}{gray!10}{white}
  \setlength\cellspacetoplimit{4pt}
  \setlength\cellspacebottomlimit{4pt}
  \if\relax\detokenize{#1}\relax
  \else
    \captionsetup{justification=centering}
    \setlength{\abovecaptionskip}{-7pt}
    \modcaptionof{table}{#1} % 追加したキャプション
    \addtocounter{table}{-1}
  \fi
  \begin{longtable}{|c|Sl|c|}
  \hline
  \rowcolor{orange!20}
  \textbf{記号} & \textbf{内容} & \textbf{#2}\\
  \hline
  \endfirsthead
  \hline
  \rowcolor{orange!20}
  \textbf{記号} & \textbf{内容} & \textbf{#2}\\
  \hline
  \endhead
  \hline
  \multicolumn{3}{|r|}{\scriptsize 次ページへ続く} \\
  \hline
  \endfoot
  \hline
  \endlastfoot
}
{%
  \end{longtable}
}

%%%%%%%%%%%%%%%%%%%%%%%%%%%%%%%%%%%%%%%%%%%%%%
%%%%% GEOMETRY %%%%%%%%%%%%%%%%%%%%%%%%%%%%%%%
%%%%%%%%%%%%%%%%%%%%%%%%%%%%%%%%%%%%%%%%%%%%%%
\geometry{
  a4paper, % paper size
  twoside,
  centering,
  textwidth={6.5in},
  includehead,  % include the head of the page
%  headheight = 13.6pt,
  includefoot,  % include the foot of the page
  top=15.0truemm,
  bottom=-0.5truemm,
}
%%%%%%%%%%%%%%%%%%%%%%%%%%%%%%%%%%%%%%%%%%%%%%
%%%%% HEYPERSETUP %%%%%%%%%%%%%%%%%%%%%%%%%%%%
%%%%%%%%%%%%%%%%%%%%%%%%%%%%%%%%%%%%%%%%%%%%%%
\hypersetup{
%  pdfcreationdate=date,
%  pdfcreator={upLaTeX with hyperref}, % creator for PDF subjct field
  pdftitle={モールド関連 -主に幾何学的性質-}, % title for PDF subjct field
  pdfsubject={Mould-Related - Mainly Geometric Properties}, % text for PDF subjct field
  pdfauthor={Kurahashi Nobuaki},  % text for PDF Author field
  pdfkeywords={mould, mold}, % keywords
%  pdfproducer=producer, % dvipdfmx
  linktoc=all,
%  linktocpage=false,   % (if it is true) make page number, not text, be link on TOC, LOF and LOT
  pdfcenterwindow=false, % position the document window center of the screen
  pdffitwindow=true,     % resize document window to fit document size
  bookmarksnumbered=true,
  bookmarksopen=true, %bookmarks open
  pdfstartview={FitH}, % Fit, FitV, FitH, FitB
  pdfpagemode=UseThumbs, % set default mode of PDF display
  unicode=true,
  pdfencoding=unicode,   % PDFDocEncoding or Unicode
  colorlinks=true,     % color links
  linkcolor=ai,      % color of links
  urlcolor=ai,         % color of urls
  citecolor=sora,      % color of citation links
}
%%%%%%%%%%%%%%%%%%%%%%%%%%%%%%%%%%%%%%%%%%%%%%
%%%%% DISPLAYBREAK %%%%%%%%%%%%%%%%%%%%%%%%%%%
%%%%%%%%%%%%%%%%%%%%%%%%%%%%%%%%%%%%%%%%%%%%%%
\allowdisplaybreaks
%%%%% TO NO PAGE BREAK IN TOC %%%
%\pretocmd{\modHeadchapter}{\addtocontents{lot}{\protect\nopagebreak}}{}{}
%\pretocmd{\modHeadsection}{\ifnum\value{section}=0\addtocontents{lot}{\protect\nopagebreak}\fi}{}{}
%\pretocmd{\subsection}{\ifnum\value{subsection}=0\addtocontents{toc}{\protect\nopagebreak[4]}\fi}{}{}
%\pretocmd{\subsubsection}{\ifnum\value{subsubsection}=0\addtocontents{toc}{\protect\nopagebreak}\fi}{}{}
%%%%%%%%%%%%%%%%%%%%%%%%%%%%%%%%%%%%%%%%%%%%%%
%%%%% UNIT LENGTH %%%%%%%%%%%%%%%%%%%%%%%%%%%%
%%%%%%%%%%%%%%%%%%%%%%%%%%%%%%%%%%%%%%%%%%%%%%
\setlength{\unitlength}{1pt}
%%%%%%%%%%%%%%%%%%%%%%%%%%%%%%%%%%%%%%%%%%%%%%
%%%%% LINESPREAD %%%%%%%%%%%%%%%%%%%%%%%%%%%%%
%%%%%%%%%%%%%%%%%%%%%%%%%%%%%%%%%%%%%%%%%%%%%%
\linespread{1.15}\selectfont
%%%%%%%%%%%%%%%%%%%%%%%%%%%%%%%%%%%%%%%%%%%%%%
%%%%% PARINDENT %%%%%%%%%%%%%%%%%%%%%%%%%%%%%%
%%%%%%%%%%%%%%%%%%%%%%%%%%%%%%%%%%%%%%%%%%%%%%
\newcommand{\indentspace}{\setlength\parindent{11pt}}
\indentspace
%\def \globalscale {0.83}
%%%%%%%%%%%%%%%%%%%%%%%%%%%%%%%%%%%%%%%%%%%%%%
%%%%% FOOTNOTE %%%%%%%%%%%%%%%%%%%%%%%%%%%%%%%
%%%%%%%%%%%%%%%%%%%%%%%%%%%%%%%%%%%%%%%%%%%%%%
\renewcommand*{\footnoteautorefname}{脚注}
\interfootnotelinepenalty=10000
\counterwithout{footnote}{chapter}
\def\@makefnmark{\hbox{}\hbox{\@textsuperscript{\normalfont\@thefnmark}}\hbox{}}
\deffootnote[1em]{1em}{1em}{\textsuperscript{\thefootnotemark}}
\renewcommand\footnoterule{%
  \kern3pt
  \hrule\@width.75\columnwidth
  \kern2.6pt
}
\makesavenoteenv{twoCtable}
\makesavenoteenv{longtable}
\makesavenoteenv{tablePart}
%%%%%%%%%%%%%%%%%%%%%%%%%%%%%%%%%%%%%%%%%%%%%%
%%%%% HEADER AND FOOTER %%%%%%%%%%%%%%%%%%%%%%
%%%%%%%%%%%%%%%%%%%%%%%%%%%%%%%%%%%%%%%%%%%%%%
%\pagestyle{fancy}
\renewcommand{\headrulewidth}{1.5pt}
\renewcommand{\footrulewidth}{0pt}
\newcommand{\commonheadfoot}{
  \fancyhead{}
  \fancyfoot{}
  \fancyfoot[LO]{\tiny\customdate} % footer left fields for main odd pages
  \fancyfoot[RE]{\tiny\customdate} % footer right fields for main even pages
}
\fancypagestyle{emptydate}{\renewcommand{\headrulewidth}{0pt}\commonheadfoot}
\fancypagestyle{front}{
  \commonheadfoot
  \fancyhead[RO]{\thepage}
  \fancyhead[LE]{\thepage}
}
\fancypagestyle{main}{
  \commonheadfoot
  \fancyhead[LO]{\hyperref[sec:\thesection]{{\small\nouppercase\rightmark}}}
  \fancyhead[RO]{{\small\partname.\,\thepart}~~~$\nicefrac{\thepage\,}{\pageref{LastPage}}$} % header right fields for all main odd pages
  \fancyhead[LE]{$\nicefrac{\thepage\,}{\pageref{LastPage}}$~~~{\small\partname.\,\thepart}} % header right fields for all main odd pages
  \fancyhead[RE]{\hyperref[\ifx\@chapapp\appendixname app:\else chap:\fi\thechapter]{{\sffamily\bfseries\thechapter.\hskip0.75em\nouppercase\Chaptername}}} % header right fields for all main even pages
}
\fancypagestyle{plainheadfront}{
  \commonheadfoot
  \fancyhead[RO]{\thepage}
  \fancyhead[LE]{\thepage}
}
\fancypagestyle{plainhead}{
  \commonheadfoot
  \fancyhead[RO]{{\small\partname.\,\thepart}~~~$\nicefrac{\thepage\,}{\pageref{LastPage}}$}
  \fancyhead[LE]{$\nicefrac{\thepage\,}{\pageref{LastPage}}$}
}
\fancypagestyle{plainheadback}{
  \commonheadfoot
  \fancyhead[RO]{$\nicefrac{\thepage\,}{\pageref{LastPage}}$}
  \fancyhead[LE]{$\nicefrac{\thepage\,}{\pageref{LastPage}}$}
}
\renewcommand*\frontmatter{%
  \if@twoside\cleardoubleoddpage\else\clearpage\fi
  \@frontmattertrue\@mainmatterfalse\@backmatterfalse\pagenumbering{roman}\pagestyle{plainheadfront}%
}
\renewcommand*\mainmatter{%
  \if@twoside\cleardoubleoddpage\else\clearpage\fi
  \@frontmatterfalse\@mainmattertrue\@backmatterfalse\pagenumbering{arabic}\pagestyle{main}%
}
\renewcommand*\backmatter{%
  \if@openright\cleardoubleoddpage\else\clearpage\fi
  \@frontmatterfalse\@mainmatterfalse\@backmattertrue\pagestyle{plainheadback}
}
%\AtBeginEnvironment{theindex}{%
%  \let\oldhyperpage\hyperpage
%  \renewcommand{\hyperpage}[1]{\cdotfill~\oldhyperpage{#1}~}
%}
\AtEndEnvironment{theindex}{%
  \thispagestyle{plainheadback}
  \pagestyle{plainheadback}
}
%%%%% WATERMARK %%%%%%%%%%%%%%%%%%%%%%%%%%%%%%%%
\AddLayersToPageStyle{@everystyle@}{WatermarkLayer}
%%%%% SETLIST %%%%%%%%%%%%%%%%%%%%%%%%%%%%%%%%
\setlist[enumerate]{listparindent=\parindent, parsep=0pt, partopsep=0pt, topsep=3pt, itemsep=3pt, leftmargin=*}
\setlist[enumerate, 1]{leftmargin=\leftmargini}
%%%%% EQUATION %%%%%%%%%%%%%%%%%%%%%%%%%%%%
\renewcommand{\theequation}{\thesection.\arabic{equation}}
\@addtoreset{equation}{section}
%%%%% CAPTION STYLE %%%%%%%%%%%%%%%%%%%%%%%%%%%%
\captionsetup[figure]{%
  width=.8\textwidth,
  format=hang,
  labelfont={bf, sf},
  labelsep={colon},
  labelformat=simple,
  font={small},
}
\captionsetup[lstlisting]{
  justification=raggedright,
  singlelinecheck=false,
  position=above,
  aboveskip=1.1pt,
  belowskip=0pt,
  labelformat={empty},
  labelfont={bf, sf},
  labelsep={space},
  font={bf, large, sf},
  hypcap=false,
}
\captionsetup[table]{
  justification=raggedright,
  singlelinecheck=false,
  position=above,
  aboveskip=4pt,
  belowskip=1pt,
  labelformat={empty},
  labelfont={bf, sf},
  labelsep={space},
  font={bf, large, sf},
  hypcap=false,
}
%%%%% STYLE OF TABLE OF CONTENTS %%%%%
%\RedeclareSectionCommand[
%  tocindent=0em,
%  tocnumwidth=4.25em
%]{part}
%\addtokomafont{partentry}{\def\autodot{}}
%\RedeclareSectionCommand[
%  tocindent=1.5em,
%  tocnumwidth=1.5em
%]{chapter}
%\RedeclareSectionCommand[
%  tocindent=3em,
%  tocnumwidth=2.5em,
%  tocentrynumberformat=\entrynumberwithdot
%]{section}
%\RedeclareSectionCommand[
%  tocindent=4.5em,
%  tocnumwidth=2.5em,
%  tocentrynumberformat=\entrynumberwithdot
%]{subsection}
%%%%% STYLE OF LIST OF TABLES %%%%%
\AtBeginEnvironment{appendices}{%
  \patchcmd{\part}{\newpage}{\relax}{}{}%
  \pretocmd{\part}{\tocAPartSeparateline{toc}{15pt}{-10pt}}{}{}{}% add page break before parts, except part 1
 % \patchcmd{\part}{\tableAPartSeparateline}{\relax}{}{}%
}
\AfterEndEnvironment{appendices}{%
  \patchcmd{\part}{\tocAPartSeparateline}{\relax}{}{}%
  \pretocmd{\part}{\addtocontents{toc}{\protect\newpage}}{}{}{}% add page break before parts, except the first one
}
\setcounter{tocdepth}{3}
\renewcommand\contentsname{\texorpdfstring{\hbox to 1.9em{目次}}{目次}}
%%%%% STYLE OF LIST OF PARTTABLES %%%%%
\renewcommand\listoflopname{\texorpdfstring{\hbox to 2.75em{大項目}}{大項目}}
%%%%% STYLE OF LIST OF FIGURES %%%%%
\renewcommand\listfigurename{\texorpdfstring{\hbox to 2.75em{図目次}}{図目次}}
\renewcommand{\figurename}{図}
\renewcommand{\figureautorefname}{\figurename} % figure --> 図
%\setcounter{lofdepth}{2}
%\renewcommand*\l@figure{\@dottedtocline{1}{1.5em}{3.2em}}
%\renewcommand*{\l@subfigure}{\@dottedxxxline{\ext@subfigure}{2}{4.7em}{2.3em}}
%%%%% STYLE OF LIST OF TABLES %%%%%
\renewcommand\listtablename{\texorpdfstring{\hbox to 2.75em{表目次}}{表目次}}
\renewcommand{\tablename}{表}
\AtBeginDocument{\renewcommand{\thetable}{}}
%%%%% STYLE OF LISTINGS %%%%%
\renewcommand{\lstlistlistingname}{プログラム 目次}
\renewcommand{\lstlistingname}{{\scshape Prg: }}
\AtBeginDocument{%
%  \counterwithout{lstlisting}{chapter}
  \renewcommand{\thelstlisting}{}%
}
\renewcommand*\l@lstlisting{\@dottedtocline{1}{0.6em}{3.2em}}
\lstdefinelanguage{GcodeBasic}{
  sensitive=true,%
  basicstyle=\small\ttfamily\linespread{1}\selectfont,%
  abovecaptionskip=0pt,
  belowcaptionskip=0pt,
%  aboveskip=0\baselineskip,
%  belowskip=0\baselineskip,
  numbers=left,%
  numbersep=6pt,%
  numberstyle=\ttfamily\scriptsize\color{black!75!},%
  breaklines=true,%
  breakindent=3pt,%
  postbreak=\mbox{\textcolor{blue}{$\hookrightarrow$}\,},%
  frame=single,%
  columns=fixed,%
  basewidth=0.53em,%
  lineskip=-1.2pt,
  comment=[l]{(},%
  commentstyle=\footnotesize\color{black!60!green!100}\slshape,%
  stringstyle={},%
  alsoletter={},%
  keywords={},%
  keywordstyle=\bfseries,%
}
\lstdefinestyle{Gcode-more}{
  language=GcodeBasic,
  nolol,
  morekeywords={[20]IF, GOTO, THEN, WHILE, DO1, DO2, END1, END2},%
  keywordstyle={[20]\fontfamily{pcr}\selectfont\bfseries},%
  emph={G90, G91},
  emphstyle={\bfseries\color{red}},
  emph={[2]G53, G54, G55, G56, G57},
  emphstyle={[2]\bfseries\color{blue}},
  emph={[3]G40, G41, G42, G43, G44},
  emphstyle={[3]\bfseries\color{orange}},
  emph={[4]G00, G01, G02, G03, G31},
  emphstyle={[4]\bfseries\color{magenta}},
  emph={[5]G28, G30},
  emphstyle={[5]\bfseries\color{violet}},
  emph={[6]G04, G10, G13, G17, G49, G58, G65, G80},
  emphstyle={[6]\bfseries\color{cyan}},
  emph={[11]
    T01, T02, T06, T11, T13, T16, T31, T50,
    M00, M01, M03, M04, M05, M06, M08, M09,
    M10, M11, M19,
    M30, M32, M33,
    M48, M49,
    M60, M61, M62, M63,
    M71, M72, M73, M74, M78, M79,
    M98, M99,
    M117, M214, M262},
  emphstyle={[11]\fontfamily{pcr}\bfseries\color{yellow!100!green!80!black!100!}},
  emph={[22]SIN, COS, SQRT, FIX, FUP, ROUND, ABS},%
  emphstyle={[22]\fontfamily{pcr}\selectfont\bfseries},%
  emph={[31]
    P140001, P110002, P130001, P130002, P150002, P150003,
    P210003, P220001, P220002, P230001, P230002, P530001, P530002,
    P410000, P420000, P430000, P440000, P450000, P490005,
    P910001, P910002},%
  emphstyle={[31]\bfseries\color{cyan}},%
}
\lstdefinestyle{Gcode-bundle}{
  language=GcodeBasic,
  nolol,
  morekeywords={[120]WHILE, THEN, DO1, DO2, END1, END2},%
  keywordstyle={[120]\fontfamily{pcr}\selectfont\bfseries},%
  emph={[102]},
  emphstyle={[102]\bfseries\color{blue}},
  emph={[111]M10, M11, M19, M30, M32, M33, M48, M49,  M61, M62, M74, M78, M262},
  emphstyle={[111]\fontfamily{pcr}\bfseries\color{yellow!100!green!80!black!100!}},
  emph={[112]M98P9001},
  emphstyle={[112]\bfseries\color{cyan}},
  emph={[122]SIN, COS, SQRT, FIX, FUP, ABS},%
  emphstyle={[122]\fontfamily{pcr}\selectfont\bfseries},%
}
%\lstdefinestyle{Gcode-ren}{
%  language=GcodeBasic,
%  nolol,
%  morekeywords={[220]IF, WHILE, THEN, DO1, DO2, END1, END2},%
%  keywordstyle={[220]\fontfamily{pcr}\selectfont\bfseries},%
%  emph={[202]},
%  emphstyle={[202]\bfseries\color{blue}},
%  emph={[211]M0, M5, M19, M32, M33, M99, M214},
%  emphstyle={[211]\fontfamily{pcr}\bfseries\color{yellow!100!green!80!black!100!}},
%  emph={[222]SIN, COS, TAN, ATAN, SQRT, FIX, FUP, ROUND, DPRNT},%
%  emphstyle={[222]\fontfamily{pcr}\selectfont\bfseries},%
%}
%%%%% STYLE OF INDEX %%%%%
\renewcommand{\indexname}{\texorpdfstring{\hbox to 1.9em{索引}}{索引}}
%%%%% STYLE OF BIBLATEX %%%%%
\renewcommand{\bibname}{\texorpdfstring{\hbox to 3.75em{参考文献}}{参考文献}}
\ExecuteBibliographyOptions{
  sorting=nyt,
  hyperref=true,
  block=nbpar,
  subentry=true,
  citecounter=true,
}
\appto\bibfont{\footnotesize\setstretch{1.1}}
\DeclareFieldFormat{labelnumberwidth}{\mkbibbrackets{#1}\hspace{-6pt}}
%%%%% STYLE OF PART %%%%%
\renewcommand{\partautorefname}{part}  % part --> part
\renewcommand\partpagestyle{emptydate}
%\renewcommand*{\partformat}{\begin{gtfamily}\thepart\end{gtfamily}}
%\renewcommand{\partname}{}
%\renewcommand{\thepart}{第\hspace{2truemm}\Roman{part}\hspace{2truemm}部} %
%\renewcommand*{\addparttocentry}[2]{\addtocentrydefault{part}{#1}{第#2部}}
%%%%% STYLE OF CHAPTER %%%%%
\renewcommand{\chapterautorefname}{章}  % chapter --> 章
\renewcommand\chapterpagestyle{\if@frontmatter plainheadfront\else plainhead\fi}
%%%%% STYLE OF APPENDICES %%%%%%%%%%%%%%%%%%%%%%%%%%%%
\renewcommand{\setthesection}{\Alph{section}}
%\renewcommand{\appendixname}{補\hskip0.5em 遺} % appendix --> 補 遺
\renewcommand{\appendixautorefname}{補遺\!} % appendix --> 補遺
%%%%% STYLE OF SECTION %%%%%
%\renewcommand{\sectionautorefname}{節\!} % section --> 節
\setcounter{secnumdepth}{4}
\renewcommand{\subsectionautorefname}{\sectionautorefname} % subsection --> section
\renewcommand{\subsubsectionautorefname}{\sectionautorefname} % subsubsection --> section
%%%%% STYLE OF PARAGRAPH %%%%%%%%%%%%%%%%%%%%%
%for scrbook.cls
\RedeclareSectionCommand[%
  style=section,%
  level=4,%
  indent=0pt,%
  beforeskip=3.25ex \@plus1ex \@minus.2ex,%
  afterskip=0.1ex \@plus.1ex \@minus.1ex,% -1em から変更
  tocindentfollows=subsubsection,%
  tocstyle=section,%
  tocindent=10em,%
  tocnumwidth=5em,% def: 5em
  font=\raggedsection\normalfont\sectfont\gtfamily\nobreak\sball{blue}~
]{paragraph}
%for book.cls
%\renewcommand\paragraph[1]{%
%  \@startsection{paragraph}{\paragraphnumdepth}{0pt}%
%  {3.25ex \@plus1ex \@minus.2ex}% \@plus, \@minusは伸び縮みできるスペースの長さ
%  {0.1ex\@plus.1ex \@minus.1ex}% ここが正だと改行されて、値だけ垂直スペースが入る
%  {\raggedsection\normalfont\sectfont\gtfamily\nobreak\size@paragraph\sball{blue}~}{#1}\noindent
%}
%%%%% STYLE OF SUBPARAGRAPH %%%%%%%%%%%%%%%%%%%%%
\RedeclareSectionCommand[%
  style=section,%
  level=5,%
  indent=0pt,% \scr@parindent から変更
  beforeskip=0.5ex \@plus1ex \@minus.2ex,% 3.25ex \@plus1ex \@minus .2ex から変更
  afterskip=0.1ex \@plus.1ex \@minus.1ex,% -1em から変更
  tocstyle=section,%
  tocindent=12em,%
  tocnumwidth=6em%
]{subparagraph}
%%%%% TCBSET %%%%%%%%%%%%%%%%%%%%%%%%%%%%
\definecolor{myheadercolor}{rgb}{0.68, 0.85, 0.90}
\tcbset{%
  %%%%% COLUMNBOX STYLE %%%%%
  Tabularbox/.style={%
  },%
  %%%%% COLUMNBOX STYLE %%%%%
  Columnbox/.style={%
    after title=\hfill\termblue{\Columnname~\thetcbcounter},%
    fonttitle=\gtfamily\bfseries,%
    breakable,%
    enhanced jigsaw,%
    left=.5ex,%
    right=.5ex,%
    bicolor,%
    colbacklower=black!10!white,%
    before upper={%
      \setcounter{GlobalFootnote}{\value{footnote}}% GlobalFootnoteValue=footnoteValue
      \let\oldfootnote=\footnote% \oldfootnote=\footnote
      \def\footnote{\stepcounter{GlobalFootnote}\oldfootnote[\arabic{GlobalFootnote}]}% define \footnote to \footnote using counter GlobalFootnote
      \renewcommand\thempfootnote{\arabic{mpfootnote}}% arabic footnote
    },
    after upper={%
      \setcounter{footnote}{\value{GlobalFootnote}}% footnoteValue=GlobalFootnoteValue
      \let\footnote=\oldfootnote% \footnote=\oldfootnote
    },
  },%
  %%%%% FIGUREBOX STYLE %%%%%
  Figurebox/.style={%
    notitle,%
    height=\textwidth,
%    width=\textwidth,
    center upper,%
    center lower,%
    arc=5pt,%
    outer arc=2pt,%
    boxrule=1pt,%
    boxsep=3mm,%
    valign=center,%
    halign=center,%
    left=0pt,%
    right=0pt,%
    colback=green!3!white,%
    colframe=black!25!white,%
    before={\centering},
  },
  %%%%% HIGHLIGHT MATH STYLE %%%%%
%  highlight math/.style={%
%    enhanced,%
%    arc=2pt,%
%    boxrule=0pt,%
%    frame hidden,%
%    fuzzy halo=1pt with blue,%
%    left=0pt,%
%    right=0pt,%
%    top=.4mm,%
%    bottom=.4mm,%
%    colback=yellow!40!white,%
%  },%
  %%%%% HOSOKUBOX STYLE %%%%%
  hosokubox/.style={%
    title={\termblue{\hosokuname~\thetcbcounter}~},%
    attach title to upper,%
    breakable,%
    enhanced jigsaw,%
    size=fbox,%
    arc=0pt,%
    middle=1mm,%
    colback=hosoku,%
    colframe=hosoku,%
    drop lifted shadow={blue!100!white!50!},%
    skin first is subskin of={enhanced jigsaw}{no shadow},%
    skin middle is subskin of={enhanced jigsaw}{no shadow},%
    skin last is subskin of={enhanced jigsaw}{drop lifted shadow={blue!100!white!50!}},%
    segmentation style={draw=black!50!white},%
    after=\smallskip\noindent{\color{white}},%
  },
  %%%%% TWOCTABLEBOX STYLE %%%%%
  twoCtablebox/.style={%
    breakable,%
    enhanced,%
    fonttitle=\bfseries,%
    fontupper=\small\sffamily,%
    colframe=black!50!black,%
    colbacktitle=blue!10!white,%
    coltitle=black,%
    top=0pt,%
    bottom=0pt,%
    left=0pt,%
    right=0pt,%
    enlarge top by=-5pt,
    enlarge bottom by=5pt,
    before upper={%
      \setcounter{GlobalFootnote}{\value{footnote}}% GlobalFootnoteValue=footnoteValue
      \let\oldfootnote=\footnote% \oldfootnote=\footnote
      \def\footnote{\stepcounter{GlobalFootnote}\oldfootnote[\arabic{GlobalFootnote}]}% define \footnote to \footnote using counter GlobalFootnote
      \renewcommand\thempfootnote{\arabic{mpfootnote}}% arabic footnote
      \renewcommand{\arraystretch}{1.2}% 行の高さを調整
      \setlength{\LTpre}{-3pt}%
      \setlength{\LTpost}{0pt}%
      \setlength{\LTleft}{0pt}%
      \setlength{\LTright}{0pt}%
      \begin{longtable}{@{}c|Sl@{\extracolsep{\fill}}c}%
      \addtocounter{table}{-1}%
    },%
    after upper={%
      \end{longtable}%
      \setcounter{footnote}{\value{GlobalFootnote}}% footnoteValue=GlobalFootnoteValue
      \let\footnote=\oldfootnote% \footnote=\oldfootnote
    },
  },
}
%%%%% TIKZSET %%%%%%%%%%%%%%%%%%%%%%%%%%%%
\tikzset{
  %%%%% SECTIONFORMAT STYLE %%%%%
%  sect/.style={signal, draw, text=white},
%  section/.style={sect, fill=konpeki!100!, signal to=east, inner sep=3pt},
%  subsection/.style={sect, fill=moegi!90!, signal to=nowhere, inner sep=3pt},
%  subsubsection/.style={sect, fill=sssec!100!, signal to=nowhere, inner sep=3pt},
  %%%%% TERMINAL STYLE %%%%%
  terminal/.style={%
    rectangle,%
    minimum size=10pt,%
    rounded corners=1.5mm,%
    thin,%
    draw=black!75,%
    top color=white,%
    font=\fontfamily{pplx},%
    inner sep=3pt,%
    inner xsep=3pt,%
    text height=1ex,%
    text depth=0pt,%
  },
  %%%%% BMATRIX STYLE %%%%%
%  every left delimiter/.style={xshift=.5em},
%  every right delimiter/.style={xshift=-.5em},
%  bmatrix/.style={matrix of math nodes, left delimiter=[, right delimiter=],},
}

\makeatother

\usepackage{refcheck}

\begin{document}
\setlength\baselineskip{18pt}
\setlength\normalbaselineskip{\baselineskip}


%%%%% TITLE %%%%%%%%%%%%%%%%%%%%%%%%%%%%%%%%%%%%%%%%%%%%%%%%
\titlehead{\hfill\small\customdate}
\subject{--- 計算メモ ---}
\title{\relax
  モールド関連\\[6pt]
  \vskip1\baselineskip
}
\subtitle{主に幾何学的性質}
\author{}
\date{}
\publishers{}

\maketitle
\thispagestyle{empty}~
\vfill
\noindent
{\scriptsize\relax
Copyright © 2023.\\
This document is owned by the individual writer, not any corporation. All rights reserved.}





%%%%%%%%%%%%%%%%%%%%%%%%%%%%%%%%%%%%%%%%%%%%%%%%%%%%%%%%%%%
%%             %%%%%%%%%%%%%%%%%%%%%%%%%%%%%%%%%%%%%%%%%%%%
%%             %%%%%%%%%%%%%%%%%%%%%%%%%%%%%%%%%%%%%%%%%%%%
%% FRONTMATTER %%%%%%%%%%%%%%%%%%%%%%%%%%%%%%%%%%%%%%%%%%%%
%%             %%%%%%%%%%%%%%%%%%%%%%%%%%%%%%%%%%%%%%%%%%%%
%%             %%%%%%%%%%%%%%%%%%%%%%%%%%%%%%%%%%%%%%%%%%%%
%%%%%%%%%%%%%%%%%%%%%%%%%%%%%%%%%%%%%%%%%%%%%%%%%%%%%%%%%%%
\frontmatter



%%%%%%%%%%%%%%%%%%%%%%%%%%%%%%%%%%%%%%%%%%%%%%%%%%%%%%%%%%%%%%%%%%%%
%%                   %%%%%%%%%%%%%%%%%%%%%%%%%%%%%%%%%%%%%%%%%%%%%%%
%% TABLE OF CONTENTS %%%%%%%%%%%%%%%%%%%%%%%%%%%%%%%%%%%%%%%%%%%%%%%
%%                   %%%%%%%%%%%%%%%%%%%%%%%%%%%%%%%%%%%%%%%%%%%%%%%
%%%%%%%%%%%%%%%%%%%%%%%%%%%%%%%%%%%%%%%%%%%%%%%%%%%%%%%%%%%%%%%%%%%%
\tableofcontents



%%%%%%%%%%%%%%%%%%%%%%%%%%%%%%%%%%%%%%%%%%%%%%%%%%%%%%%%%%
%%            %%%%%%%%%%%%%%%%%%%%%%%%%%%%%%%%%%%%%%%%%%%%
%%            %%%%%%%%%%%%%%%%%%%%%%%%%%%%%%%%%%%%%%%%%%%%
%% MAINMATTER %%%%%%%%%%%%%%%%%%%%%%%%%%%%%%%%%%%%%%%%%%%%
%%            %%%%%%%%%%%%%%%%%%%%%%%%%%%%%%%%%%%%%%%%%%%%
%%            %%%%%%%%%%%%%%%%%%%%%%%%%%%%%%%%%%%%%%%%%%%%
%%%%%%%%%%%%%%%%%%%%%%%%%%%%%%%%%%%%%%%%%%%%%%%%%%%%%%%%%%
\mainmatter





%%%%%%%%%%%%%%%%%%%%%%%%%%%%%%%%%%%%%%%%%%%%%%%%%%%%%%%%%
%%         %%%%%%%%%%%%%%%%%%%%%%%%%%%%%%%%%%%%%%%%%%%%%%
%%         %%%%%%%%%%%%%%%%%%%%%%%%%%%%%%%%%%%%%%%%%%%%%%
%% Part I  %%%%%%%%%%%%%%%%%%%%%%%%%%%%%%%%%%%%%%%%%%%%%%
%%         %%%%%%%%%%%%%%%%%%%%%%%%%%%%%%%%%%%%%%%%%%%%%%
%%         %%%%%%%%%%%%%%%%%%%%%%%%%%%%%%%%%%%%%%%%%%%%%%
%%%%%%%%%%%%%%%%%%%%%%%%%%%%%%%%%%%%%%%%%%%%%%%%%%%%%%%%%
\part{モールドの幾何}



%%%%%%%%%%%%%%%%%%%%%%%%%%%%%%%%%%%%%%%%%%%%%%%%%%%%%%%%%%
%%           %%%%%%%%%%%%%%%%%%%%%%%%%%%%%%%%%%%%%%%%%%%%%
%% chapter 1 %%%%%%%%%%%%%%%%%%%%%%%%%%%%%%%%%%%%%%%%%%%%%
%%           %%%%%%%%%%%%%%%%%%%%%%%%%%%%%%%%%%%%%%%%%%%%%
%%%%%%%%%%%%%%%%%%%%%%%%%%%%%%%%%%%%%%%%%%%%%%%%%%%%%%%%%%
\chapter{振分け}
%\label{chap:mouldHuriwake}
モールドの振分けの長さ(振分長)は、トップ側とボトム側では一般に異なる。
しかし、加工をする際には、ジグの中心に対して両者の長さの差が小さいほうが一般的には好都合である。
そうした場合の対処法として、ここでは以下のような2つの方法を考える。
\begin{enumerate}
\item
適当な厚さのスペーサをモールドとジグの接点に取り付けることで、双方の振分長を調節する。
\item
適当な角度にテーブルを回転することで、双方の振分長を調節する。
\end{enumerate}
このとき、モールドがどのように移動するかを考える。

基本的な考え方として、半径$R$の円の中心を原点として$\Omega$だけ回転し、次にモールドとの(スペーサを入れてない側の)接点を中心に$-\theta$だけ回転したと考えることができる。
なお、ここでは話の簡単化のため、もとの振分けではトップ側よりボトム側の振分長のほうが長いものとする。




%%%%%%%%%%%%%%%%%%%%%%%%%%%%%%%%%%%%%%%%%%%%%%%%%%%%%%%%%%
%% section 1.1 %%%%%%%%%%%%%%%%%%%%%%%%%%%%%%%%%%%%%%%%%%%
%%%%%%%%%%%%%%%%%%%%%%%%%%%%%%%%%%%%%%%%%%%%%%%%%%%%%%%%%%
\section{ジグの接点部が点の場合}
まずは簡単のため、ジグのモールドとの接点部(\pageautoref{fig:mouldOnComplexPlane1}のU$_\mathrm T$, U$_\mathrm B$の部分)は点であるとして考える。
%%%%%%%%%%%%%%%%%%%%%%%%%%%%%%%%%%%%%%%%%%%%%%%%%%%%%%%%%%
%% figure %%%%%%%%%%%%%%%%%%%%%%%%%%%%%%%%%%%%%%%%%%%%%%%%
%%%%%%%%%%%%%%%%%%%%%%%%%%%%%%%%%%%%%%%%%%%%%%%%%%%%%%%%%%
\newlength{\oldtextwidth}
\setlength{\oldtextwidth}{\textwidth}
\newlength{\oldtextheight}
\setlength{\oldtextheight}{\textheight}
\afterpage{%
\begin{landscape}
\setlength{\textwidth}{\oldtextwidth}
\setlength{\textheight}{\oldtextwidth}
\begin{figure}[p]
\centering
\begin{Figbox}
\resizebox{\linewidth}{!}{\adjustbox{max height=\textheight}{\mouldCoordinate}}%
\captionsetup{width=.75\textwidth}
\caption[湾曲中心Oを原点とした複素平面上のモールド]
  {湾曲中心Oを原点とした複素平面上のモールド\newline
   T$_\mathrm o$, T$_\mathrm i$, B$_\mathrm o$, B$_\mathrm i$, U$_\mathrm T$, U$_\mathrm B$は点、%
   $R_\mathrm c$, $R_\mathrm o$, $R_\mathrm i$, $f_\mathrm T$, $f_\mathrm B$, $l$は長さ、%
   $\alpha_{\mathrm T_\mathrm o}$, $\alpha_{\mathrm T_\mathrm i}$, $\alpha_{\mathrm U_\mathrm B}$は角度を示す。}
\label{fig:mouldOnComplexPlane1}
\end{Figbox}%
\end{figure}%
\end{landscape}%
\setlength{\textwidth}{\oldtextwidth}
\setlength{\textheight}{\oldtextheight}%
}
%%%%%%%%%%%%%%%%%%%%%%%%%%%%%%%%%%%%%%%%%%%%%%%%%%%%%%%%%%
%%%%%%%%%%%%%%%%%%%%%%%%%%%%%%%%%%%%%%%%%%%%%%%%%%%%%%%%%%
%%%%%%%%%%%%%%%%%%%%%%%%%%%%%%%%%%%%%%%%%%%%%%%%%%%%%%%%%%



%%%%%%%%%%%%%%%%%%%%%%%%%%%%%%%%%%%%%%%%%%%%%%%%%%%%%%%%%%
%% subsection 1.1.1 %%%%%%%%%%%%%%%%%%%%%%%%%%%%%%%%%%%%%%
%%%%%%%%%%%%%%%%%%%%%%%%%%%%%%%%%%%%%%%%%%%%%%%%%%%%%%%%%%
\subsection{スペーサを用いた再振分け}
モールドの湾曲における円の中心Oを原点とした複素数平面を考える
%% footnote %%%%%%%%%%%%%%%%%%%%%
\footnote{ここでは$0 < R_\mathrm c < \infty$ ($0 < \nicefrac1{R_\mathrm c} < \infty$)としている。
$R_\mathrm c \to \infty$ ($\nicefrac1{R_\mathrm c} \to 0$)の場合、すなわち湾曲のないまっすぐなモールドの場合は、別途考える必要がある。}。
%%%%%%%%%%%%%%%%%%%%%%%%%%%%%%%%%
このとき、\pageautoref{fig:mouldOnComplexPlane1}のように、$R_\mathrm c$, $R_\mathrm i$, $R_\mathrm o$, $f_\mathrm T$, $f_\mathrm B$, $l$, $\alpha_{\mathrm T_\mathrm i}$, $\alpha_{\mathrm T_\mathrm o}$, $\alpha_{\mathrm U_\mathrm B}$をとると、
\begin{subequations}
%% label{eq:constraintUpoint1}
%% label{eq:constraintUpoint2}
\begin{gather}
  \label{eq:constraintUpoint1}
  R_\mathrm o - R_\mathrm c = R_\mathrm c - R_\mathrm i = \frac{W_x}2~, \qquad
  \IP\!\left(R_\mathrm oe^{i\alpha_{\mathrm T_\mathrm o}} - R_\mathrm ie^{i\alpha_{\mathrm T_\mathrm i}}\right)
  = 0~,\\
  \label{eq:constraintUpoint2}
  \sin\alpha_{\mathrm T_\mathrm i} = \frac{f_\mathrm T}{R_\mathrm i}, \qquad
  \sin\alpha_{\mathrm U_\mathrm B} = \frac l{R_\mathrm i}, \qquad
  \tan\theta = \frac\delta{2l}~.
\end{gather}
\end{subequations}
ここで$W_x$はモールドの(AC)外径、$\delta$はスペーサの厚さである。
このときモールドを原点Oを中心に$\Omega$だけ回転し、さらに点U$_\mathrm B$($R_\mathrm i$, $-\alpha_{\mathrm U_\mathrm B}$)を中心に$-\theta$だけ回転すると、点T$_\mathrm i$($R_\mathrm i$, $\alpha_{\mathrm T_\mathrm i}$)は、
%% label{eq:afterftUpoint}
\begin{align}
  \notag
  & e^{-i\theta}\!\left\{R_\mathrm ie^{i(\alpha_{\mathrm T_\mathrm i} + \Omega)} - R_\mathrm ie^{-i\alpha_{\mathrm U_\mathrm B}}\right\}
    +R_\mathrm ie^{-i\alpha_{\mathrm U_\mathrm B}}\\
  &= R_\mathrm i
     \left\{
       e^{i(\alpha_{\mathrm T_\mathrm i} + \Omega - \theta)} - e^{-i(\alpha_{\mathrm U_\mathrm B} + \theta)} + e^{-i\alpha_{\mathrm U_\mathrm B}}
     \right\}
  \label{eq:afterftUpoint}
\end{align}
に移動する。
また同様に点T$_\mathrm o$($R_\mathrm o$, $\alpha_{\mathrm T_\mathrm o}$)は
\begin{align*}
  \notag
  R_\mathrm oe^{i(\alpha_{\mathrm T_\mathrm o} + \Omega - \theta)}
  -R_\mathrm i\left\{e^{-i(\alpha_{\mathrm U_\mathrm B} + \theta)}-e^{-i\alpha_{\mathrm U_\mathrm B}}\right\}
\end{align*}
に移動する。
したがって、これらの差
\begin{equation}
  \notag
  e^{i(\Omega - \theta)}\left(R_\mathrm oe^{i\alpha_{\mathrm T_\mathrm o}} - R_\mathrm ie^{i\alpha_{\mathrm T_\mathrm i}}\right)
\end{equation}
の虚部が$0$であればよい。
つまり、\pageeqref{eq:constraintUpoint1}より、$\Omega = \theta$である
%% footnote %%%%%%%%%%%%%%%%%%%%%
\footnote{ここでは$0 \leqq \Omega, \theta < \nicefrac \pi2$としている。}。
%%%%%%%%%%%%%%%%%%%%%%%%%%%%%%%%%

スペーサを入れた後の(トップ側の)振分長は、\pageeqref{eq:afterftUpoint}の虚部を見ればよい。
\begin{align*}
  R_\mathrm i\left\{\sin\alpha_{\mathrm T_\mathrm i} + \sin(\alpha_{\mathrm U_\mathrm B} + \theta) - \sin\alpha_{\mathrm U_\mathrm B}\right\}
  &= f_\mathrm T -l
     +R_\mathrm i\left(\sin\alpha_{\mathrm U_\mathrm B}\cos\theta + \cos\alpha_{\mathrm U_\mathrm B}\sin\theta\right)\\
  &= f_\mathrm T -l+l\cdot\frac{2l}{\sqrt{4l^2+\delta^2}}
     +\sqrt{R_\mathrm i^2-l^2}\cdot\frac{\delta}{\sqrt{4l^2+\delta^2}}\\
  &= f_\mathrm T -l+\frac{2l^2+\delta\sqrt{R_\mathrm i^2-l^2}}{\sqrt{4l^2+\delta^2}}~.
\end{align*}
まとめると、厚さ$\delta$のスペーサを入れた後のトップ側の振分長$f'_\mathrm T$は、
\begin{align*}
  f'_\mathrm T
  = f_\mathrm T -l
    +\frac{2l^2+\delta\sqrt{\left(R_\mathrm c-\nicefrac{W_x}2\right)^2-l^2}}{\sqrt{4l^2+\delta^2}}~.
\end{align*}



%%%%%%%%%%%%%%%%%%%%%%%%%%%%%%%%%%%%%%%%%%%%%%%%%%%%%%%%%%
%% subsection 1.1.2 %%%%%%%%%%%%%%%%%%%%%%%%%%%%%%%%%%%%%%
%%%%%%%%%%%%%%%%%%%%%%%%%%%%%%%%%%%%%%%%%%%%%%%%%%%%%%%%%%
\subsection{振分長が均等になるスペーサ厚}
%%%%%%%%%%%%%%%%%%%%%%%%%%%%%%%
トップ側とボトム側の振分長が同じになるとき、$\delta$は
\begin{align*}
  f'_\mathrm T - f_\mathrm T = \frac{f_\mathrm B - f_\mathrm T}2
\end{align*}
を満たす。
これより、
\begin{align*}
  \frac{2l^2+\delta\sqrt{R_\mathrm i^2-l^2}}{\sqrt{4l^2+\delta^2}} = l'\qquad
  \left(l' \equiv l + \frac{f_\mathrm B-f_\mathrm T}2\right)
\end{align*}
両辺を2乗すると、
\begin{gather*}
  4l^4+\delta^2\left(R_\mathrm i^2-l^2\right)+4l^2\delta\sqrt{R_\mathrm i^2-l^2}
  = l'^2\left(4l^2+\delta^2\right)\\
  \longrightarrow\quad
  \delta^2\left(R_\mathrm i^2-l^2-l'^2\right)
  +4l^2\delta\sqrt{R_\mathrm i^2-l^2} -4l^2\left(l'^2 - l^2\right)
  = 0.
\end{gather*}
$\delta > 0$より、
\begin{align*}
  \delta
  &= \frac{\sqrt{4l^4\left(R_\mathrm i^2-l^2\right)
                 +4l^2\left(R_\mathrm i^2-l^2-l'^2\right)\left(l'^2 - l^2\right)}
           -2l^2\sqrt{R_\mathrm i^2-l^2}}{R_\mathrm i^2-l^2-l'^2}\\
  &= 2l\cdot\frac{l'\sqrt{R_\mathrm i^2-l'^2}-l\sqrt{R_\mathrm i^2-l^2}}{R_\mathrm i^2-l^2-l'^2}
\end{align*}
まとめると、求めるスペーサの厚さ$\delta$は、
\begin{align*}
  \delta
  = 2l\cdot
    \frac{\displaystyle
          \left(l+\frac{f_\mathrm B-f_\mathrm T}2\right)\!
          \sqrt{\left(R_\mathrm c-\nicefrac{W_x}2\right)^2
                -\left(l+\frac{f_\mathrm B-f_\mathrm T}2\right)^{\!\!2}}
          -l\sqrt{\left(R_\mathrm c-\nicefrac{W_x}2\right)^2-l^2}}
         {\displaystyle
          \left(R_\mathrm c-\nicefrac{W_x}2\right)^2-l^2
          -\left(l+\frac{f_\mathrm B-f_\mathrm T}2\right)^{\!\!2}}~.
\end{align*}




\clearpage
%%%%%%%%%%%%%%%%%%%%%%%%%%%%%%%%%%%%%%%%%%%%%%%%%%%%%%%%%%
%% section 1.2 %%%%%%%%%%%%%%%%%%%%%%%%%%%%%%%%%%%%%%%%%%%
%%%%%%%%%%%%%%%%%%%%%%%%%%%%%%%%%%%%%%%%%%%%%%%%%%%%%%%%%%
\section{受板がある場合}
ジグのモールドと接する部品(受板)の大きさを考慮した場合を考える。
モールドに接する側の面が半径$\rho$の円弧、虚軸方向の厚みが$\sigma$とする。
また受板の虚軸負方向側の面は、ジグのそれと同じ平面上にあるものとする。

受板の径の中心を改めてU$_\mathrm B$とし、また原点に対する偏角を改めて$-\alpha_{\mathrm U_\mathrm B}$とすると、これはU$_\mathrm B$($R_\mathrm i-\rho$, $-\alpha_{\mathrm U_\mathrm B}$)表すことができる。
ただし、\pageeqref{eq:constraintUpoint2}は以下のようになる。
\begin{align*}
  \sin\alpha_{\mathrm U_\mathrm B} = \frac{\bar l}{R_\mathrm i-\rho}\quad, \quad
  \tan\psi = \frac\delta{2\bar l} \quad
  \left(~\bar l \equiv l-\frac\sigma2~\right).
\end{align*}

これを原点Oを中心に$\Omega$だけ回転し、さらに点U$_\mathrm B$($R_\mathrm i-\rho$, $-\alpha_{\mathrm U_\mathrm B}$)を中心に点T$_\mathrm i$($R_\mathrm i$, $\alpha_{\mathrm T_\mathrm i}$)を$-\theta$だけ回転すると、
%% label{eq:afterftUfinite}
\begin{align}
  \notag
  & e^{-i\theta}\!
    \left\{R_\mathrm ie^{i(\alpha_{\mathrm T_\mathrm i}+\Omega)}
           -R_\mathrm i'e^{-i\alpha_{\mathrm U_\mathrm B}}\right\}
    +R_\mathrm i'e^{-i\alpha_{\mathrm U_\mathrm B}}\\
  & = R_\mathrm ie^{i(\alpha_{\mathrm T_\mathrm i}+\Omega-\theta)}
      -R_\mathrm i'\!
       \left\{e^{-i(\alpha_{\mathrm U_\mathrm B}+\theta)}-e^{-i\alpha_{\mathrm U_\mathrm B}}\right\}\qquad
    \big(R_\mathrm i' \equiv R_\mathrm i-\rho\big)
    \label{eq:afterftUfinite}
\end{align}
に移動する。
同様に点T$_\mathrm o$($R_\mathrm o$, $\alpha_{\mathrm T_\mathrm o}$)は
\begin{align*}
  R_\mathrm oe^{i(\alpha_{\mathrm T_\mathrm o}+\Omega-\theta)}
  -R_\mathrm i'\!
   \left\{e^{-i(\alpha_{\mathrm U_\mathrm B} + \theta)} - e^{-i\alpha_{\mathrm U_\mathrm B}}\right\}
\end{align*}
に移動する。
したがって、これらの差
\begin{align*}
  e^{i(\Omega-\theta)}\!
  \left(R_\mathrm oe^{i\alpha_{\mathrm T_\mathrm o}} - R_\mathrm ie^{i\alpha_{\mathrm T_\mathrm i}}\right)
\end{align*}
の虚部が$0$であればよい。
つまり、\pageeqref{eq:constraintUpoint1}より、受板がある場合も$\Omega = \theta$である。



%%%%%%%%%%%%%%%%%%%%%%%%%%%%%%%%%%%%%%%%%%%%%%%%%%%%%%%%%%
%% subsection 1.2.1 %%%%%%%%%%%%%%%%%%%%%%%%%%%%%%%%%%%%%%
%%%%%%%%%%%%%%%%%%%%%%%%%%%%%%%%%%%%%%%%%%%%%%%%%%%%%%%%%%
\subsection{受板の接点}
受板とモールドとの(トップ側の)接点は、$R_\mathrm ie^{i\alpha_{\mathrm U_\mathrm B}}$で与えられる。
このとき厚さ$\delta$のスペーサを取付けると、U$_\mathrm B$を中心に回転するが、それに伴い受板における接点の位置も変化する。


%%%%%%%%%%%%%%%%%%%%%%%%%%%%%%%%%%%%%%%%%%%%%%%%%%%%%%%%%%
%% subsubsection 1.2.1.1 %%%%%%%%%%%%%%%%%%%%%%%%%%%%%%%%%
%%%%%%%%%%%%%%%%%%%%%%%%%%%%%%%%%%%%%%%%%%%%%%%%%%%%%%%%%%
\subsubsection{回転後のモールドの湾曲中心}
%%%%%%%%%%%%%%%%%%%%%%%%%
厚さ$\delta$のスペーサを挟むと、トップ側における受板の円の中心U$_\mathrm B$は実軸方向に$\delta$だけ移動するので、
\begin{align*}
  R_\mathrm i'e^{i\alpha_{\mathrm U_\mathrm B}}
  \quad\longrightarrow\quad
  \delta+R_\mathrm i'e^{i\alpha_{\mathrm U_\mathrm B}}\ .
\end{align*}
よって、それぞれの受板の中心U$_\mathrm B$, U$_\mathrm T$を結んだ線分U$_\mathrm B$U$_\mathrm T$は、U$_\mathrm B$を中心に$-\psi$だけ傾いた線分U$_\mathrm B'$U$_\mathrm T'$となる
%% footnote %%%%%%%%%%%%%%%%%%%%%
\footnote{%
U$_\mathrm B'$U$_\mathrm T'$の長さは$\bar l\sec\psi$であり、U$_\mathrm B$U$_\mathrm T$の長さ$\bar l$より長くなることに注意。}。
%%%%%%%%%%%%%%%%%%%%%%%%%%%%%%%%%
回転後のモールドの湾曲中心は、この線分の垂直二等分線上にあり、またそれぞれの受板の中心から$R_\mathrm i'$の距離の位置にある。
つまり、この傾いた線分U$_\mathrm B'$U$_\mathrm T'$の中点から、角度$\pi-\psi$, 大きさ$\sqrt{R_\mathrm i'^2-\frac{\delta^2+(2\bar l)^2}4}$の位置に移動する。
したがって、回転後における湾曲の円の中心O$'$は、
%% label{eq:afterOrgin}
\begin{align}
  \notag
  & \frac\delta2+\sqrt{R_\mathrm i'^2-\bar l^2}
    +\sqrt{R_\mathrm i'^2-\frac{\delta^2+(2\bar l)^2}4}e^{i(\pi-\psi)}\\
  & = \frac\delta2+\sqrt{R_\mathrm i'^2-\bar l^2}-\sqrt{R_\mathrm i'^2-\frac{\delta^2+(2\bar l)^2}4}\cos\psi
      +i\sqrt{R_\mathrm i'^2-\frac{\delta^2+(2\bar l)^2}4}\sin\psi\ .
    \label{eq:afterOrgin}
\end{align}


%%%%%%%%%%%%%%%%%%%%%%%%%%%%%%%%%%%%%%%%%%%%%%%%%%%%%%%%%%
%% subsubsection 1.2.1.2 %%%%%%%%%%%%%%%%%%%%%%%%%%%%%%%%%
%%%%%%%%%%%%%%%%%%%%%%%%%%%%%%%%%%%%%%%%%%%%%%%%%%%%%%%%%%
\subsubsection{回転後の接点(トップ側)}
回転後のトップ側における受板の中心U$_\mathrm T'$とモールドの湾曲中心O$'$との差をとると、
\begin{align*}
  \frac\delta2+\sqrt{R_\mathrm i'^2-\frac{\delta^2+(2\bar l)^2}4}\cos\psi
  +i\left\{\bar l-\sqrt{R_\mathrm i'^2-\frac{\delta^2+(2\bar l)^2}4}\sin\psi\right\}
  = R_\mathrm i'e^{i\alpha'_{\mathrm U_\mathrm T}}\ .
\end{align*}
ここで、
\begin{align*}
  \tan\alpha'_{\mathrm U_\mathrm T}
  = \frac{\displaystyle\bar l-\sqrt{R_\mathrm i'^2-\frac{\delta^2+(2\bar l)^2}4}\sin\psi}
         {\displaystyle\frac\delta2+\sqrt{R_\mathrm i'^2-\frac{\delta^2+(2\bar l)^2}4}\cos\psi}\ .
\end{align*}
%%%%%%%%%%%%%%%%%%%%%%%%%%%%%%%%%%%%%%%%%%%%%%%%%%%%%%%%%%
%% hosoku %%%%%%%%%%%%%%%%%%%%%%%%%%%%%%%%%%%%%%%%%%%%%%%%
%%%%%%%%%%%%%%%%%%%%%%%%%%%%%%%%%%%%%%%%%%%%%%%%%%%%%%%%%%
\begin{hosoku}
これの大きさは、$\delta\cos\psi-2\bar l\sin\psi = 0$より、
\begin{align*}
  \left\{\frac\delta2+\sqrt{R_\mathrm i'^2-\frac{\delta^2+(2\bar l)^2}4}\cos\psi\right\}^{\!\!2}
  +\left\{\bar l-\sqrt{R_\mathrm i'^2-\frac{\delta^2+(2\bar l)^2}4}\sin\psi\right\}^{\!\!2}
  = R_\mathrm i'^2\ .
\end{align*}
\end{hosoku}
%%%%%%%%%%%%%%%%%%%%%%%%%%%%%%%%%%%%%%%%%%%%%%%%%%%%%%%%%%
%%%%%%%%%%%%%%%%%%%%%%%%%%%%%%%%%%%%%%%%%%%%%%%%%%%%%%%%%%
%%%%%%%%%%%%%%%%%%%%%%%%%%%%%%%%%%%%%%%%%%%%%%%%%%%%%%%%%%
よって、回転後の接点は以下で与えられる。
\begin{align*}
  &  R_\mathrm ie^{i\alpha'_{\mathrm U_\mathrm T}}
     +\frac\delta2+\sqrt{R_\mathrm i'^2-\bar l^2}-\sqrt{R_\mathrm i'^2-\frac{\delta^2+(2\bar l)^2}4}\cos\psi
     +i\sqrt{R_\mathrm i'^2-\frac{\delta^2+(2\bar l)^2}4}\sin\psi\\
  &= \delta+R_\mathrm i'e^{i\alpha_{\mathrm U_\mathrm B}}+\rho e^{i\alpha'_{\mathrm U_\mathrm T}}\ .
\end{align*}


%%%%%%%%%%%%%%%%%%%%%%%%%%%%%%%%%%%%%%%%%%%%%%%%%%%%%%%%%%
%% subsubsection 1.2.1.3 %%%%%%%%%%%%%%%%%%%%%%%%%%%%%%%%%
%%%%%%%%%%%%%%%%%%%%%%%%%%%%%%%%%%%%%%%%%%%%%%%%%%%%%%%%%%
\subsubsection{回転後の接点(ボトム側)}
回転後のボトム側における受板の中心U$_\mathrm B$とモールドの湾曲中心O$'$との差をとると、
\begin{align*}
  -\frac\delta2+\sqrt{R_\mathrm i'^2-\frac{\delta^2+(2\bar l)^2}4}\cos\psi
  -i\left\{\bar l+\sqrt{R_\mathrm i'^2-\frac{\delta^2+(2\bar l)^2}4}\sin\psi\right\}
  = R_\mathrm i'e^{-i\alpha'_{\mathrm U_\mathrm B}}
\end{align*}
ここで、
\begin{align*}
  \tan\alpha'_{\mathrm U_\mathrm B}
  = \frac{\displaystyle\bar l+\sqrt{R_\mathrm i'^2-\frac{\delta^2+(2\bar l)^2}4}\sin\psi}
         {\displaystyle-\frac\delta2+\sqrt{R_\mathrm i'^2-\frac{\delta^2+(2\bar l)^2}4}\cos\psi}
\end{align*}
よって、回転後の接点は以下で与えられる。
%% label{eq:afterUBcontact}
\begin{align}
  \notag
  &  R_\mathrm ie^{-i\alpha'_{\mathrm U_\mathrm B}}
     +\frac\delta2+\sqrt{R_\mathrm i'^2-\bar l^2}-\sqrt{R_\mathrm i'^2-\frac{\delta^2+(2\bar l)^2}4}\cos\psi
     +i\sqrt{R_\mathrm i'^2-\frac{\delta^2+(2\bar l)^2}4}\sin\psi\\
  &= R_\mathrm i'e^{-i\alpha_{\mathrm U_\mathrm B}}+\rho e^{-i\alpha'_{\mathrm U_\mathrm B}}
   \label{eq:afterUBcontact}
\end{align}
%%%%%%%%%%%%%%%%%%%%%%%%%%%%%%%%%%%%%%%%%%%%%%%%%%%%%%%%%%
%% hosoku %%%%%%%%%%%%%%%%%%%%%%%%%%%%%%%%%%%%%%%%%%%%%%%%
%%%%%%%%%%%%%%%%%%%%%%%%%%%%%%%%%%%%%%%%%%%%%%%%%%%%%%%%%%
\begin{hosoku}
辺の長さが$R_i'$, $R_i'$, $2\bar l$の二等辺三角形$\triangle$OU$_\mathrm B$U$_\mathrm T$の部分が、回転後には辺の長さ$R_i'$, $R_i'$, $\sqrt{\delta^2+(2\bar l)^2}$の二等辺三角形$\triangle$O$'$U$_\mathrm B'$U$_\mathrm T'$となる。
実際、$\cos2a = 1-2\sin^2\!a$より、
\begin{align*}
  \sin^2\frac{\alpha'_{\mathrm U_\mathrm T}+\alpha'_{\mathrm U_\mathrm B}}2
  = \frac{\delta^2+(2\bar l)^2}{4R_\mathrm i'^2}\ .
\end{align*}
\end{hosoku}
%%%%%%%%%%%%%%%%%%%%%%%%%%%%%%%%%%%%%%%%%%%%%%%%%%%%%%%%%%
%%%%%%%%%%%%%%%%%%%%%%%%%%%%%%%%%%%%%%%%%%%%%%%%%%%%%%%%%%
%%%%%%%%%%%%%%%%%%%%%%%%%%%%%%%%%%%%%%%%%%%%%%%%%%%%%%%%%%



%%%%%%%%%%%%%%%%%%%%%%%%%%%%%%%%%%%%%%%%%%%%%%%%%%%%%%%%%%
%% subsection 1.2.2 %%%%%%%%%%%%%%%%%%%%%%%%%%%%%%%%%%%%%%
%%%%%%%%%%%%%%%%%%%%%%%%%%%%%%%%%%%%%%%%%%%%%%%%%%%%%%%%%%
\subsection{スペーサによるモールドの回転角}
厚さ$\delta$のスペーサを挿入すると、モールドの湾曲中心Oは、U$_\mathrm B$を中心に$-\left(\alpha'_{\mathrm U_\mathrm B}\!-\alpha_{\mathrm U_\mathrm B}\right)$だけ回転する。
実際、
\begin{align*}
  -R_\mathrm i'e^{-i\alpha'_{\mathrm U_\mathrm B}}+R_\mathrm i'e^{-i\alpha_{\mathrm U_\mathrm B}}
  &= R_\mathrm i'(\cos\alpha_{\mathrm U_\mathrm B}-\cos\alpha'_{\mathrm U_\mathrm B})
     +iR_\mathrm i'(\sin\alpha'_{\mathrm U_\mathrm B}-\sin\alpha_{\mathrm U_\mathrm B})
\end{align*}
であり、これは回転後の湾曲中心\pageeqref{eq:afterOrgin}に一致する。
つまり、$\alpha'_{\mathrm U_\mathrm B}\!-\alpha_{\mathrm U_\mathrm B}$が$\theta$に相当する。
%%%%%%%%%%%%%%%%%%%%%%%%%%%%%%%%%%%%%%%%%%%%%%%%%%%%%%%%%%
%% hosoku %%%%%%%%%%%%%%%%%%%%%%%%%%%%%%%%%%%%%%%%%%%%%%%%
%%%%%%%%%%%%%%%%%%%%%%%%%%%%%%%%%%%%%%%%%%%%%%%%%%%%%%%%%%
\begin{hosoku}
トップ側の接点U$_\mathrm T'$とボトム側の接点U$_\mathrm B'$の差をとると、
\begin{align*}
  R_\mathrm i\!\left(e^{i\alpha_{\mathrm U_\mathrm T}'}-e^{-i\alpha'_{\mathrm U_\mathrm B}}\right)
  &= \frac{R_\mathrm i'+\rho}{R_\mathrm i'}\left\{\delta+i(2\bar l)\right\}
   = \frac{R_\mathrm i}{R_\mathrm i'}\sqrt{\delta^2+(2\bar l)^2}e^{i(\nicefrac\pi2-\psi)}\ .
\end{align*}
したがって、厚さ$\delta$のスペーサを挿入すると、両接点を通る直線は$-\psi$だけ傾くことがわかる。
また、その長さは受板中心間U$_\mathrm T'$U$_\mathrm B'$の距離$\sqrt{\delta^2+(2\bar l)^2}$の$\nicefrac{R_i}{R_i'}$倍になっていることも確かめられる。
\end{hosoku}
%%%%%%%%%%%%%%%%%%%%%%%%%%%%%%%%%%%%%%%%%%%%%%%%%%%%%%%%%%
%%%%%%%%%%%%%%%%%%%%%%%%%%%%%%%%%%%%%%%%%%%%%%%%%%%%%%%%%%
%%%%%%%%%%%%%%%%%%%%%%%%%%%%%%%%%%%%%%%%%%%%%%%%%%%%%%%%%%


%%%%%%%%%%%%%%%%%%%%%%%%%%%%%%%%%%%%%%%%%%%%%%%%%%%%%%%%%%
%% subsection 1.2.3 %%%%%%%%%%%%%%%%%%%%%%%%%%%%%%%%%%%%%%
%%%%%%%%%%%%%%%%%%%%%%%%%%%%%%%%%%%%%%%%%%%%%%%%%%%%%%%%%%
\subsection{スペーサによる再振分け}


%%%%%%%%%%%%%%%%%%%%%%%%%%%%%%%%%%%%%%%%%%%%%%%%%%%%%%%%%%
%% subsubsection 1.2.3.1 %%%%%%%%%%%%%%%%%%%%%%%%%%%%%%%%%
%%%%%%%%%%%%%%%%%%%%%%%%%%%%%%%%%%%%%%%%%%%%%%%%%%%%%%%%%%
\subsubsection{振分長}
スペーサを入れた後の(トップ側の)振分長$f'_\mathrm T$は、\pageeqref{eq:afterftUfinite}の虚部を見ればよい。
回転角は$-(\alpha'_{\mathrm U_\mathrm B}-\alpha_{\mathrm U_\mathrm B})$なので、
\begin{align*}
  f'_\mathrm T
  = R_\mathrm i\sin\alpha_{\mathrm T_\mathrm i}
    +R_\mathrm i'\left(\sin\alpha'_{\mathrm U_\mathrm B}-\sin\alpha_{\mathrm U_\mathrm B}\right)
  &= f_\mathrm T+\sqrt{R_\mathrm i'^2-\frac{\delta^2+(2\bar l)^2}4}\sin\psi\\
  &= f_\mathrm T+\sqrt{R_\mathrm i'^2-\frac{\delta^2+(2\bar l)^2}4}\frac\delta{\sqrt{\delta^2+(2\bar l)^2}}\ .
\end{align*}


%%%%%%%%%%%%%%%%%%%%%%%%%%%%%%%%%%%%%%%%%%%%%%%%%%%%%%%%%%
%% subsubsection 1.2.3.2 %%%%%%%%%%%%%%%%%%%%%%%%%%%%%%%%%
%%%%%%%%%%%%%%%%%%%%%%%%%%%%%%%%%%%%%%%%%%%%%%%%%%%%%%%%%%
\subsubsection{モールドの移動距離}
\pageeqref{eq:afterftUfinite}の実部は、
\begin{align*}
  & R_\mathrm i\cos\alpha_{\mathrm T_\mathrm i}
    -R_\mathrm i'(\cos\alpha'_{\mathrm U_\mathrm B}-\cos\alpha_{\mathrm U_\mathrm B})\\
  & = \sqrt{R_\mathrm i^2-f_\mathrm T^2}+\frac\delta2+\sqrt{R_\mathrm i'^2-\bar l^2}
      -\sqrt{R_\mathrm i'^2-\frac{\delta^2+(2\bar l)^2}4}\cos\psi
\end{align*}
となる。
よって、スペーサを挿入することにより、モールドは水平・鉛直方向にそれぞれ、
\begin{subequations}
%% label{eq:spacerMoveHdistance}
\begin{alignat}{2}
  \label{eq:spacerMoveHdistance}
  \text{水平方向:}\quad
  & \frac\delta2+\sqrt{R_\mathrm i'^2-\bar l^2}-\sqrt{R_\mathrm i'^2-\frac{\delta^2+(2\bar l)^2}4}\frac{2\bar l}{\sqrt{\delta^2+(2\bar l)^2}}\\
  \text{鉛直方向:}\quad
  & \sqrt{R_\mathrm i'^2-\frac{\delta^2+(2\bar l)^2}4}\frac\delta{\sqrt{\delta^2+(2\bar l)^2}}
\end{alignat}
\end{subequations}
だけ移動することがわかる。



%%%%%%%%%%%%%%%%%%%%%%%%%%%%%%%%%%%%%%%%%%%%%%%%%%%%%%%%%%
%% subsection 1.2.4 %%%%%%%%%%%%%%%%%%%%%%%%%%%%%%%%%%%%%%
%%%%%%%%%%%%%%%%%%%%%%%%%%%%%%%%%%%%%%%%%%%%%%%%%%%%%%%%%%
\subsection{振分長が均等になるスペーサ厚}
トップ側とボトム側の振分長が同じになるとき、$\delta$は
\begin{align*}
  \sqrt{R_\mathrm i'^2-\frac{\delta^2+(2\bar l)^2}4}\frac\delta{\sqrt{\delta^2+(2\bar l)^2}} = f_d \qquad
  \left(f_d \equiv \frac{f_\mathrm B-f_\mathrm T}2\right)\ .
\end{align*}
を満たす。
両辺を2乗して$-4$倍すると、
\begin{align*}
  \delta^2\left\{\delta^2+(2\bar l)^2-4R_\mathrm i'^2\right\}+4f_d^2\left\{\delta^2+(2\bar l)^2\right\}
  & = \delta^4-2\left\{2R_\mathrm i'^2-2\bar l^2-2f_d^2\right\}\delta^2+4f_d^2(2\bar l)^2\\
  & = 0\ .
\end{align*}
したがって、$f_\mathrm T = f_\mathrm B$ ($f_d = 0$)のとき$\delta = 0$であることを考慮して、
\begin{align*}
  \delta^2
  &= 2\left\{
       R_\mathrm i'^2-\bar l^2-f_d^2-\sqrt{\left(R_\mathrm i'^2-\bar l^2-f_d^2\right)^2-4f_d^2\bar l^2}\,
     \right\}\\
  &= \left(\sqrt{R_\mathrm i'^2-(\bar l-f_d)^2}-\sqrt{R_\mathrm i'^2-(\bar l+f_d)^2}\,\right)^{\!\!2}.
\end{align*}
なお、これはモールドが水平・鉛直方向にそれぞれ、
\begin{align*}
  \text{水平方向:}~\frac\delta2+\sqrt{R_\mathrm i^2-\bar l^2}-\frac{2\bar l}{\delta}f_d\quad(\delta>0)\ , \qquad
  \text{鉛直方向:}~\frac{f_\mathrm B-f_\mathrm T}2
\end{align*}
だけ移動することを意味する。


%%%%%%%%%%%%%%%%%%%%%%%%%%%%%%%%%%%%%%%%%%%%%%%%%%%%%%%%%%
%% hosoku %%%%%%%%%%%%%%%%%%%%%%%%%%%%%%%%%%%%%%%%%%%%%%%%
%%%%%%%%%%%%%%%%%%%%%%%%%%%%%%%%%%%%%%%%%%%%%%%%%%%%%%%%%%
\begin{hosoku}
改めてまとめると、厚さ$\delta$のスペーサを(トップ側に)挿入した後のトップ側の振分長$f_\mathrm T'$は、
\begin{align*}
  f_\mathrm T'
  = f_\mathrm T+\sqrt{R_\mathrm i'^2-\frac{\delta^2+(2\bar l)^2}4}\frac\delta{\sqrt{\delta^2+(2\bar l)^2}}\ .
\end{align*}
トップ側とボトム側の振分長が均等になるときのスペーサ厚$\delta'$は、
\begin{align*}
  \delta' = \sqrt{R_\mathrm i'^2-(\bar l-f_d)^2}-\sqrt{R_\mathrm i'^2-(\bar l+f_d)^2}\ .
\end{align*}
ここで、
\begin{align*}
  R_\mathrm i' = R_\mathrm c-\frac{W_x}2-\rho\ ,\quad
  \bar l = l-\frac\sigma2\ ,\quad
  f_d = \frac{f_\mathrm B-f_\mathrm T}2\ .
\end{align*}
\end{hosoku}
%%%%%%%%%%%%%%%%%%%%%%%%%%%%%%%%%%%%%%%%%%%%%%%%%%%%%%%%%%
%%%%%%%%%%%%%%%%%%%%%%%%%%%%%%%%%%%%%%%%%%%%%%%%%%%%%%%%%%
%%%%%%%%%%%%%%%%%%%%%%%%%%%%%%%%%%%%%%%%%%%%%%%%%%%%%%%%%%




\clearpage
%%%%%%%%%%%%%%%%%%%%%%%%%%%%%%%%%%%%%%%%%%%%%%%%%%%%%%%%%%
%% section 1.3 %%%%%%%%%%%%%%%%%%%%%%%%%%%%%%%%%%%%%%%%%%%
%%%%%%%%%%%%%%%%%%%%%%%%%%%%%%%%%%%%%%%%%%%%%%%%%%%%%%%%%%
\section{テーブルの回転による振分長の調節}
これまでトップ・ボトム振分長の差を小さくするためにスペーサを用いる手法を考えてきた。
スペーサを取付けることは、本質的にはボトム側の受板の点U$_\mathrm B$を中心に回転しているということである。
このとき回転の中心はU$_\mathrm B$である必要はなく、他の点でも問題ない。
したがって、スペーサを用いて回転をしなくても、テーブルそのものを回転するという手法が考えられる。
これは回転の中心が、受板の点U$_\mathrm B$からテーブル中心Pに変わることに相当する。

受板の円の中心U$_\mathrm B$と、テーブル中心Pとの実軸方向の距離を$\varDelta$とすると、テーブル中心Pの$X$座標$\varDelta'$は次で与えられる。
%% label{eq:tableCenter}
\begin{align}
  \label{eq:tableCenter}
  \varDelta'
  = \varDelta+R_\mathrm i'\cos\alpha_{\mathrm U_\mathrm B} = \varDelta+\sqrt{R_\mathrm i'^2-\bar l^2}\ .
\end{align}
モールドのトップ側におけるC側端点T$_\mathrm i$($R_\mathrm i$, $\alpha_{\mathrm T_\mathrm i}$)を、原点Oを中心に$\Omega$だけ回転し、さらに点P\,($\varDelta'$, $0$)を中心に$-\theta$だけ回転すると、
\begin{align}
  \label{eq:afterfttable}
  e^{-i\theta}\left\{R_\mathrm i^{i(\alpha_{\mathrm T_\mathrm i}+\Omega)}-\varDelta'\right\}+\varDelta'
  = R_\mathrm ie^{i(\alpha_{\mathrm T_\mathrm i}+\Omega-\theta)}+\varDelta'\left(1-e^{-i\theta}\right)
\end{align}
に移動する。
同様に、トップ側におけるA側端点T$_\mathrm o$($R_\mathrm o$, $\alpha_{\mathrm T_\mathrm o}$)は、
\begin{align*}
  e^{-i\theta}\!\left\{R_\mathrm o^{i(\alpha_{\mathrm T_\mathrm o}+\Omega)}-\varDelta'\right\}+\varDelta'
  = R_\mathrm ie^{i(\alpha_{\mathrm T_\mathrm o}+\Omega-\theta)}+\varDelta'\!\left(1-e^{-i\theta}\right)
\end{align*}
に移動する。
したがって、これらの差
\begin{align*}
  e^{i(\Omega-\theta)}\!
  \left(R_\mathrm oe^{i\alpha_{\mathrm T_\mathrm o}}-R_\mathrm ie^{i\alpha_{\mathrm T_\mathrm i}}\right)
\end{align*}
の虚部が$0$であればよい。
よって\pageeqref{eq:constraintUpoint1}より、この場合も$\Omega = \theta$となる。



%%%%%%%%%%%%%%%%%%%%%%%%%%%%%%%%%%%%%%%%%%%%%%%%%%%%%%%%%%
%% subsection 1.3.1 %%%%%%%%%%%%%%%%%%%%%%%%%%%%%%%%%%%%%%
%%%%%%%%%%%%%%%%%%%%%%%%%%%%%%%%%%%%%%%%%%%%%%%%%%%%%%%%%%
\subsection{回転後のモールドの湾曲中心および受板との接点}


%%%%%%%%%%%%%%%%%%%%%%%%%%%%%%%%%%%%%%%%%%%%%%%%%%%%%%%%%%
%% subsubsection 1.3.1.1 %%%%%%%%%%%%%%%%%%%%%%%%%%%%%%%%%
%%%%%%%%%%%%%%%%%%%%%%%%%%%%%%%%%%%%%%%%%%%%%%%%%%%%%%%%%%
\subsubsection{回転後のモールドの湾曲中心}
回転後のモールドの湾曲中心O$'$は、点Pを中心に$-\theta$だけ回転するので、
\begin{align*}
  \varDelta'\!\left(1-e^{-i\theta}\right) = \varDelta'(1-\cos\theta)+i\varDelta'\sin\theta\ .
\end{align*}


%%%%%%%%%%%%%%%%%%%%%%%%%%%%%%%%%%%%%%%%%%%%%%%%%%%%%%%%%%
%% subsubsection 1.3.1.2 %%%%%%%%%%%%%%%%%%%%%%%%%%%%%%%%%
%%%%%%%%%%%%%%%%%%%%%%%%%%%%%%%%%%%%%%%%%%%%%%%%%%%%%%%%%%
\subsubsection{回転後の接点}
回転後におけるトップ側の受板との接点は、点Pを中心に$-\theta$だけ回転するので、
\begin{align*}
  &  e^{-i\theta}\!\left(R_\mathrm ie^{i\alpha_{\mathrm U_\mathrm B}}-\varDelta'\right)+\varDelta'\\
  &= R_\mathrm ie^{i(\alpha_{\mathrm U_\mathrm B}-\theta)}+\varDelta'\!\left(1-e^{-i\theta}\right)\\
  &= R_\mathrm i\cos(\alpha_{\mathrm U_\mathrm B}-\theta)+\varDelta'(1-\cos\theta)
     +i\left\{R_\mathrm i\sin(\alpha_{\mathrm U_\mathrm B}-\theta)+i\varDelta'\sin\theta\right\}\ .
\end{align*}
同様に、ボトム側の受板との接点は、
\begin{align*}
  &  e^{-i\theta}\!\left(R_\mathrm ie^{-i\alpha_{\mathrm U_\mathrm B}}-\varDelta'\right)+\varDelta'\\
  &= R_\mathrm ie^{-i(\alpha_{\mathrm U_\mathrm B}+\theta)}+\varDelta'\!\left(1-e^{-i\theta}\right)\\
  &= R_\mathrm i\cos(\alpha_{\mathrm U_\mathrm B}+\theta)+\varDelta'(1-\cos\theta)
     -i\left\{R_\mathrm i\sin(\alpha_{\mathrm U_\mathrm B}+\theta)-i\varDelta'\sin\theta\right\}\ .
\end{align*}
%%%%%%%%%%%%%%%%%%%%%%%%%%%%%%%%%%%%%%%%%%%%%%%%%%%%%%%%%%
%% hosoku %%%%%%%%%%%%%%%%%%%%%%%%%%%%%%%%%%%%%%%%%%%%%%%%
%%%%%%%%%%%%%%%%%%%%%%%%%%%%%%%%%%%%%%%%%%%%%%%%%%%%%%%%%%
\begin{hosoku}
両接点との差をとると、
\begin{align*}
  2R_\mathrm i\sin\alpha_{\mathrm U_\mathrm B}\sin\theta+2iR_\mathrm i\sin\alpha_{\mathrm U_\mathrm B}\cos\theta
  = \frac{R_\mathrm i}{R_\mathrm i'}(2\bar l)e^{i(\pi-\theta)}
\end{align*}
となり、受板の両中心間(長さ$2\bar l$)を結んだ線分を$\nicefrac{R_\mathrm i}{R_\mathrm i'}$倍し、虚軸から$-\theta$だけ傾けたものになっていることがわかる。
\end{hosoku}
%%%%%%%%%%%%%%%%%%%%%%%%%%%%%%%%%%%%%%%%%%%%%%%%%%%%%%%%%%
%%%%%%%%%%%%%%%%%%%%%%%%%%%%%%%%%%%%%%%%%%%%%%%%%%%%%%%%%%
%%%%%%%%%%%%%%%%%%%%%%%%%%%%%%%%%%%%%%%%%%%%%%%%%%%%%%%%%%




%%%%%%%%%%%%%%%%%%%%%%%%%%%%%%%%%%%%%%%%%%%%%%%%%%%%%%%%%%
%% subsection 1.3.2 %%%%%%%%%%%%%%%%%%%%%%%%%%%%%%%%%%%%%%
%%%%%%%%%%%%%%%%%%%%%%%%%%%%%%%%%%%%%%%%%%%%%%%%%%%%%%%%%%
\subsection{回転後の振分長}
回転後のトップ側の振分長$f_\mathrm T'$は、\pageeqref{eq:afterfttable}の虚部を見ればよいので、
\begin{align*}
  f_\mathrm T' = f_\mathrm T+\varDelta'\sin\theta = f_\mathrm T+\left(\varDelta+\sqrt{R_\mathrm i'-\bar l^2}\right)\sin\theta\ .
\end{align*}
同様に、ボトムの振分長$f_\mathrm B'$は、符号に注意して、
\begin{align*}
  f_\mathrm B' = f_\mathrm B-\varDelta'\sin\theta = (f_\mathrm T+f_\mathrm B)-f_\mathrm T'\ .
\end{align*}
%%%%%%%%%%%%%%%%%%%%%%%%%%%%%%%%%%%%%%%%%%%%%%%%%%%%%%%%%%
%% hosoku %%%%%%%%%%%%%%%%%%%%%%%%%%%%%%%%%%%%%%%%%%%%%%%%
%%%%%%%%%%%%%%%%%%%%%%%%%%%%%%%%%%%%%%%%%%%%%%%%%%%%%%%%%%
\begin{hosoku}
テーブル中心Pによる回転は振分長に影響しない(端面を水平に戻す)ので、湾曲中心Oによる回転だけが影響する。
よって、振分長は$\varDelta'\sin\theta$だけ変化する。
\end{hosoku}
%%%%%%%%%%%%%%%%%%%%%%%%%%%%%%%%%%%%%%%%%%%%%%%%%%%%%%%%%%
%%%%%%%%%%%%%%%%%%%%%%%%%%%%%%%%%%%%%%%%%%%%%%%%%%%%%%%%%%
%%%%%%%%%%%%%%%%%%%%%%%%%%%%%%%%%%%%%%%%%%%%%%%%%%%%%%%%%%
なお、\pageeqref{eq:afterfttable}の実部は、
\begin{align*}
  R_\mathrm i\cos\alpha_{\mathrm T_\mathrm i}+\varDelta'(1-\cos\theta)
  = \sqrt{R_\mathrm i^2-\bar l^2}+\left(\varDelta+\sqrt{R_\mathrm i'-\bar l^2}\right)\!(1-\cos\theta)\ .
\end{align*}
となるので、実軸の正方向に動くことがわかる。



%%%%%%%%%%%%%%%%%%%%%%%%%%%%%%%%%%%%%%%%%%%%%%%%%%%%%%%%%%
%% subsection 1.3.3 %%%%%%%%%%%%%%%%%%%%%%%%%%%%%%%%%%%%%%
%%%%%%%%%%%%%%%%%%%%%%%%%%%%%%%%%%%%%%%%%%%%%%%%%%%%%%%%%%
\subsection{振分長が均等になるときの回転角}
トップ側およびボトム側の振分長が同じになるとき、$\theta$は
\begin{align*}
  \varDelta'\sin\theta = f_d \qquad \left(f_d = \frac{f_\mathrm B-f_\mathrm T}2\right)
\end{align*}
であればいいので、
\begin{align*}
  \sin\theta = \frac{f_d}{\varDelta'}
  = \frac{f_\mathrm B-f_\mathrm T}{2\left(\varDelta+\sqrt{R_\mathrm i'-\bar l^2}\right)}~.
\end{align*}
%%%%%%%%%%%%%%%%%%%%%%%%%%%%%%%%%%%%%%%%%%%%%%%%%%%%%%%%%%
%% hosoku %%%%%%%%%%%%%%%%%%%%%%%%%%%%%%%%%%%%%%%%%%%%%%%%
%%%%%%%%%%%%%%%%%%%%%%%%%%%%%%%%%%%%%%%%%%%%%%%%%%%%%%%%%%
\begin{hosoku}
改めてまとめると、テーブルを$-\theta$だけ傾けた後のトップ側の振分長$f_\mathrm T'$は、
\begin{align*}
  f_\mathrm T' = f_\mathrm T+\left(\varDelta+\sqrt{R_\mathrm i'-\bar l^2}\right)\!\sin\theta\ .
\end{align*}
トップ側とボトム側の振分長が均等になるときの回転角$\theta'$は、
\begin{align*}
  \sin\theta' = \frac{f_\mathrm B-f_\mathrm T}{2\left(\varDelta+\sqrt{R_\mathrm i'-\bar l^2}\right)}\ .
\end{align*}
ここで、
\begin{align*}
  R_\mathrm i' = R_\mathrm c-\frac{W_x}2-\rho\ ,\quad
  \bar l = l-\frac\sigma2\ ,\quad
  f_d = \frac{f_\mathrm B-f_\mathrm T}2\ .
\end{align*}
\end{hosoku}
%%%%%%%%%%%%%%%%%%%%%%%%%%%%%%%%%%%%%%%%%%%%%%%%%%%%%%%%%%
%%%%%%%%%%%%%%%%%%%%%%%%%%%%%%%%%%%%%%%%%%%%%%%%%%%%%%%%%%
%%%%%%%%%%%%%%%%%%%%%%%%%%%%%%%%%%%%%%%%%%%%%%%%%%%%%%%%%%




%%%%%%%%%%%%%%%%%%%%%%%%%%%%%%%%%%%%%%%%%%%%%%%%%%%%%%%%%%
%%           %%%%%%%%%%%%%%%%%%%%%%%%%%%%%%%%%%%%%%%%%%%%%
%% chapter 2 %%%%%%%%%%%%%%%%%%%%%%%%%%%%%%%%%%%%%%%%%%%%%
%%           %%%%%%%%%%%%%%%%%%%%%%%%%%%%%%%%%%%%%%%%%%%%%
%%%%%%%%%%%%%%%%%%%%%%%%%%%%%%%%%%%%%%%%%%%%%%%%%%%%%%%%%%
\chapter{端面(外径)}
ここではモールドの端面における各々の位置を考える。
ただし機内のモールドは、考えている側の端面が工具側に向いているものとする。




%%%%%%%%%%%%%%%%%%%%%%%%%%%%%%%%%%%%%%%%%%%%%%%%%%%%%%%%%%
%% section 2.1 %%%%%%%%%%%%%%%%%%%%%%%%%%%%%%%%%%%%%%%%%%%
%%%%%%%%%%%%%%%%%%%%%%%%%%%%%%%%%%%%%%%%%%%%%%%%%%%%%%%%%%
\section{トップ側の端面}



%%%%%%%%%%%%%%%%%%%%%%%%%%%%%%%%%%%%%%%%%%%%%%%%%%%%%%%%%%
%% subsection 2.1.1 %%%%%%%%%%%%%%%%%%%%%%%%%%%%%%%%%%%%%%
%%%%%%%%%%%%%%%%%%%%%%%%%%%%%%%%%%%%%%%%%%%%%%%%%%%%%%%%%%
\subsection{トップ端面における湾曲中心の位置}


%%%%%%%%%%%%%%%%%%%%%%%%%%%%%%%%%%%%%%%%%%%%%%%%%%%%%%%%%%
%% subsubsection 2.1.1.1 %%%%%%%%%%%%%%%%%%%%%%%%%%%%%%%%%
%%%%%%%%%%%%%%%%%%%%%%%%%%%%%%%%%%%%%%%%%%%%%%%%%%%%%%%%%%
\subsubsection{スペーサを用いた場合のT\texorpdfstring{$_{R_\mathrm c}'$}{Rc'}}
スペーサを取付けた後のトップ端面における湾曲中心の位置T$_{R_\mathrm c}'$と、テーブル中心Pとの$X$方向の差は、
\begin{align*}
  \left(
    R_\mathrm ce^{i\alpha_\mathrm c}
    -R_\mathrm i'e^{-i\alpha'_{\mathrm U_\mathrm B}}
    +R_\mathrm i'e^{-i\alpha_{\mathrm U_\mathrm B}}
  \right)
  -\varDelta'
  = R_\mathrm ce^{i\alpha_\mathrm c}-R_\mathrm i'e^{-i\alpha'_{\mathrm U_\mathrm B}}-\varDelta \qquad
    \left(\sin\alpha_\mathrm c = \frac{f_\mathrm T}{R_\mathrm c}\right)
\end{align*}
の実部を見ればよい。
したがって
%% footnote %%%%%%%%%%%%%%%%%%%%%
\footnote{この場合、トップ側が工具側に向いている。}、
%%%%%%%%%%%%%%%%%%%%%%%%%%%%%%%%%
%% label{eq:spacerTRc}
\begin{align}
  \notag
  &  R_\mathrm c\cos\alpha_\mathrm c-R_\mathrm i'\cos\alpha'_{\mathrm U_\mathrm B}-\varDelta\\
  &= -\varDelta+\sqrt{R_\mathrm c^2-f_\mathrm T^2}+\frac\delta2
     -\sqrt{R_\mathrm i'^2-\frac{\delta^2+(2\bar l)^2}4}
      \frac{2\bar l}{\sqrt{\delta^2+\left(2\bar l\right)^2}}
     \label{eq:spacerTRc}
\end{align}
で与えられる
%% footnote %%%%%%%%%%%%%%%%%%%%%
\footnote{実際の作業では、この点を(端面の湾曲中心T$_{R_\mathrm c}\!$でなく)端面の外側中心T$_\mathrm c$とみなすことが多い。}。
%%%%%%%%%%%%%%%%%%%%%%%%%%%%%%%%%


%%%%%%%%%%%%%%%%%%%%%%%%%%%%%%%%%%%%%%%%%%%%%%%%%%%%%%%%%%
%% subsubsection 2.1.1.2 %%%%%%%%%%%%%%%%%%%%%%%%%%%%%%%%%
%%%%%%%%%%%%%%%%%%%%%%%%%%%%%%%%%%%%%%%%%%%%%%%%%%%%%%%%%%
\subsubsection{テーブルを傾けた場合のT\texorpdfstring{$_{R_\mathrm c}'$}{Rc'}}
テーブルを傾けた後のトップ端面における湾曲中心の位置T$_{R_\mathrm c}'$と、テーブル中心Pとの$X$方向の差は、
\begin{align*}
  \left(R_\mathrm ce^{i\alpha_\mathrm c}-\varDelta'e^{-i\theta}+\varDelta'\right)-\varDelta'
  = R_\mathrm ce^{i\alpha_\mathrm c}-\varDelta'e^{-i\theta}
\end{align*}
の実部を見ればよい。
すなわち、
%% label{eq:tableTRc}
\begin{align}
  \label{eq:tableTRc}
  R_\mathrm c\cos\alpha_\mathrm c-\varDelta'\cos\theta
  = \sqrt{R_\mathrm c^2-f_\mathrm T^2}-\left(\varDelta+\sqrt{R_i'^2-\bar l^2}\right)\cos\theta~.
\end{align}



%%%%%%%%%%%%%%%%%%%%%%%%%%%%%%%%%%%%%%%%%%%%%%%%%%%%%%%%%%
%% subsection 2.1.2 %%%%%%%%%%%%%%%%%%%%%%%%%%%%%%%%%%%%%%
%%%%%%%%%%%%%%%%%%%%%%%%%%%%%%%%%%%%%%%%%%%%%%%%%%%%%%%%%%
\subsection{トップ端面における外側中心の位置}
トップ端における湾曲中心T$_{R_\mathrm c}'$と外径中心T$_\mathrm c'$との差は、以下で与えられる。
%% label{eq:TRc-Tc}
\begin{align}
  \label{eq:TRc-Tc}
  \sqrt{R_\mathrm c^2-f_\mathrm T^2}
  -\frac{\sqrt{R_\mathrm o^2-f_\mathrm T^2}+\sqrt{R_\mathrm i^2-f_\mathrm T^2}}2\ .
\end{align}
よって、外径中心T$_\mathrm c'$の位置は、湾曲中心T$_{R_\mathrm c}'$から\pageeqref{eq:TRc-Tc}だけ加味すればよい。
以下では、外径中心T$_\mathrm c'$の位置を直接計算し、このことが整合していることを確かめる。



%%%%%%%%%%%%%%%%%%%%%%%%%%%%%%%%%%%%%%%%%%%%%%%%%%%%%%%%%%
%% subsubsection 2.1.2.1 %%%%%%%%%%%%%%%%%%%%%%%%%%%%%%%%%
%%%%%%%%%%%%%%%%%%%%%%%%%%%%%%%%%%%%%%%%%%%%%%%%%%%%%%%%%%
\subsubsection{スペーサを用いた場合のT\texorpdfstring{$_\mathrm c'$}{c'}}
同様にして、外面A・C側のトップ端点T$_\mathrm o'$, T$_\mathrm i'$の、テーブル中心Pを原点とした場合の$X$座標はそれぞれ、
\begin{align*}
  \text{C面側端点:}&
  -\varDelta+\sqrt{R_\mathrm i^2-f_\mathrm T^2}+\frac\delta2
  -\sqrt{R_\mathrm i'^2-\frac{\delta^2+(2\bar l)^2}4}\frac{2\bar l}{\sqrt{\delta^2+(2\bar l)^2}}\ ,\\
  \text{A面側端点:}&
  -\varDelta+\sqrt{R_\mathrm o^2-f_\mathrm T^2}+\frac\delta2
  -\sqrt{R_\mathrm i'^2-\frac{\delta^2+(2\bar l)^2}4}\frac{2\bar l}{\sqrt{\delta^2+(2\bar l)^2}}\ .
\end{align*}
したがって、トップ端における外径中心T$_\mathrm c'$の$X$座標は、
%% label{eq:spacerTc}
\begin{align}
  \label{eq:spacerTc}
  -\varDelta+\frac{\sqrt{R_\mathrm o^2-f_\mathrm T^2}+\sqrt{R_\mathrm i^2-f_\mathrm T^2}}2
  +\frac\delta2-\sqrt{R_\mathrm i'^2-\frac{\delta^2+(2\bar l)^2}4}\frac{2\bar l}{\sqrt{\delta^2+(2\bar l)^2}}\ .
\end{align}
これより、湾曲中心T$_{R_\mathrm c}'$と外径中心T$_\mathrm c'$との差は\pageeqref{eq:TRc-Tc}となることがわかる。


%%%%%%%%%%%%%%%%%%%%%%%%%%%%%%%%%%%%%%%%%%%%%%%%%%%%%%%%%%
%% subsubsection 2.1.2.2 %%%%%%%%%%%%%%%%%%%%%%%%%%%%%%%%%
%%%%%%%%%%%%%%%%%%%%%%%%%%%%%%%%%%%%%%%%%%%%%%%%%%%%%%%%%%
\subsubsection{テーブルを傾けた場合のT\texorpdfstring{$_\mathrm c'$}{c'}}
同様にして、外面A・C側のトップ端点T$_\mathrm o'$, T$_\mathrm i'$の、テーブル中心Pを原点とした場合の$X$座標はそれぞれ、
\begin{align*}
  \text{C側端点:}&~
  \sqrt{R_\mathrm i^2-f_\mathrm T^2}-\varDelta'\cos\theta\ ,\\
  \text{A側端点:}&~
  \sqrt{R_\mathrm o^2-f_\mathrm T^2}-\varDelta'\cos\theta\ .
\end{align*}
したがって、トップ端における(AC外径の)中点T$_\mathrm c'$の$X$座標は、
%% label{eq:tableTc}
\begin{align}
  \label{eq:tableTc}
  \frac{\sqrt{R_\mathrm o^2-f_\mathrm T^2}+\sqrt{R_\mathrm i^2-f_\mathrm T^2}}2-\varDelta'\cos\theta\ .
\end{align}
これより、湾曲中心T$_{R_\mathrm c}'$と外径中心T$_\mathrm c'$との差は\pageeqref{eq:TRc-Tc}となることがわかる。




\clearpage
%%%%%%%%%%%%%%%%%%%%%%%%%%%%%%%%%%%%%%%%%%%%%%%%%%%%%%%%%%
%% section 2.2 %%%%%%%%%%%%%%%%%%%%%%%%%%%%%%%%%%%%%%%%%%%
%%%%%%%%%%%%%%%%%%%%%%%%%%%%%%%%%%%%%%%%%%%%%%%%%%%%%%%%%%
\section{ボトム側の端面}



%%%%%%%%%%%%%%%%%%%%%%%%%%%%%%%%%%%%%%%%%%%%%%%%%%%%%%%%%%
%% subsection 2.2.1 %%%%%%%%%%%%%%%%%%%%%%%%%%%%%%%%%%%%%%
%%%%%%%%%%%%%%%%%%%%%%%%%%%%%%%%%%%%%%%%%%%%%%%%%%%%%%%%%%
\subsection{ボトム端面における湾曲中心の位置}


%%%%%%%%%%%%%%%%%%%%%%%%%%%%%%%%%%%%%%%%%%%%%%%%%%%%%%%%%%
%% subsubsection 2.2.1.1 %%%%%%%%%%%%%%%%%%%%%%%%%%%%%%%%%
%%%%%%%%%%%%%%%%%%%%%%%%%%%%%%%%%%%%%%%%%%%%%%%%%%%%%%%%%%
\subsubsection{スペーサを用いた場合のB\texorpdfstring{$_{R_\mathrm c}'$}{Rc'}}
スペーサ取付後のボトム端面における湾曲中心B$_{R_\mathrm c}'$と、テーブル中心Pとの$X$方向の差は、トップ側の場合と同様に考えて
%% footnote %%%%%%%%%%%%%%%%%%%%%
\footnote{この場合は、ボトム側が工具側に向いている。}、
%%%%%%%%%%%%%%%%%%%%%%%%%%%%%%%%%
\begin{align*}
%  \label{eq:spacerBRc}
  \varDelta-\sqrt{R_\mathrm c^2-f_\mathrm B^2}-\frac\delta2
  +\sqrt{R_\mathrm i'^2-\frac{\delta^2+(2\bar l)^2}4}\frac{2\bar l}{\sqrt{\delta^2+(2\bar l)^2}}\ .
\end{align*}


%%%%%%%%%%%%%%%%%%%%%%%%%%%%%%%%%%%%%%%%%%%%%%%%%%%%%%%%%%
%% subsubsection 2.2.1.2 %%%%%%%%%%%%%%%%%%%%%%%%%%%%%%%%%
%%%%%%%%%%%%%%%%%%%%%%%%%%%%%%%%%%%%%%%%%%%%%%%%%%%%%%%%%%
\subsubsection{テーブルを傾けた場合のB\texorpdfstring{$_{R_\mathrm c}'$}{Rc'}}
テーブルを傾けた後のボトム端面における湾曲中心の位置B$_{R_\mathrm c}'$と、テーブル中心Pとの$X$方向の差は、トップ側の場合と同様に考えて
\begin{align*}
%  \label{eq:tableBRc}
  \left(\varDelta+\sqrt{R_i'^2-\bar l^2}\right)\cos\theta-\sqrt{R_\mathrm c^2-f_\mathrm B^2}~.
\end{align*}



%%%%%%%%%%%%%%%%%%%%%%%%%%%%%%%%%%%%%%%%%%%%%%%%%%%%%%%%%%
%% subsection 2.2.2 %%%%%%%%%%%%%%%%%%%%%%%%%%%%%%%%%%%%%%
%%%%%%%%%%%%%%%%%%%%%%%%%%%%%%%%%%%%%%%%%%%%%%%%%%%%%%%%%%
\subsection{ボトム端面における外側中心の位置}
ボトム端における湾曲中心B$_{R_\mathrm c}'$と外径中心B$_\mathrm c'$との差は、以下で与えられる。
%% label{eq:BRc-Bc}
\begin{align}
  \label{eq:BRc-Bc}
  \sqrt{R_\mathrm c^2-f_\mathrm B^2}
  -\frac{\sqrt{R_\mathrm o^2-f_\mathrm B^2}+\sqrt{R_\mathrm i^2-f_\mathrm B^2}}2\ .
\end{align}
よって、外径中心B$_\mathrm c'$の位置は、湾曲中心B$_{R_\mathrm c}'$から\pageeqref{eq:BRc-Bc}だけ加味すればよい。
以下では、外径中心B$_\mathrm c'$の位置を直接計算し、このことが整合していることを確かめる。


%%%%%%%%%%%%%%%%%%%%%%%%%%%%%%%%%%%%%%%%%%%%%%%%%%%%%%%%%%
%% subsubsection 2.2.2.1 %%%%%%%%%%%%%%%%%%%%%%%%%%%%%%%%%
%%%%%%%%%%%%%%%%%%%%%%%%%%%%%%%%%%%%%%%%%%%%%%%%%%%%%%%%%%
\subsubsection{スペーサを用いた場合のB\texorpdfstring{$_\mathrm c'$}{c'}}
外面A・C面側のボトム端点B$_{R_\mathrm o}'$, B$_{R_\mathrm i}'$の、テーブル中心Pを原点とした場合の$X$座標はそれぞれ、
\begin{align*}
  \text{C面側端点:}&~~
  \varDelta-\sqrt{R_\mathrm i^2-f_\mathrm B^2}-\frac\delta2+\sqrt{R_\mathrm i'^2-\frac{\delta^2+(2\bar l)^2}4}\frac{2\bar l}{\sqrt{\delta^2+(2\bar l)^2}}\ ,\\
  \text{A面側端点:}&~~
  \varDelta-\sqrt{R_\mathrm o^2-f_\mathrm B^2}-\frac\delta2+\sqrt{R_\mathrm i'^2-\frac{\delta^2+(2\bar l)^2}4}\frac{2\bar l}{\sqrt{\delta^2+(2\bar l^2}}\ .
\end{align*}
したがって、ボトム端における(AC外径の)中点B$_\mathrm c'$の$X$座標は、
%% label{eq:spacerBc}
\begin{align}
  \label{eq:spacerBc}
  \varDelta-\frac{\sqrt{R_\mathrm o^2-f_\mathrm B^2}+\sqrt{R_\mathrm i^2-f_\mathrm B^2}}2
  -\frac\delta2+\sqrt{R_\mathrm i'^2-\frac{\delta^2+(2\bar l)^2}4}\frac{2\bar l}{\sqrt{\delta^2+(2\bar l)^2}}\ .
\end{align}
これより、湾曲中心B$_{R_\mathrm c}'$と外径中心B$_\mathrm c'$との差は\pageeqref{eq:BRc-Bc}となることがわかる。


%%%%%%%%%%%%%%%%%%%%%%%%%%%%%%%%%%%%%%%%%%%%%%%%%%%%%%%%%%
%% subsubsection 2.2.2.2 %%%%%%%%%%%%%%%%%%%%%%%%%%%%%%%%%
%%%%%%%%%%%%%%%%%%%%%%%%%%%%%%%%%%%%%%%%%%%%%%%%%%%%%%%%%%
\subsubsection{テーブルを傾けた場合のB\texorpdfstring{$_\mathrm c'$}{c'}}
外面A・C面側のボトム端点B$_{R_\mathrm o}'$, B$_{R_\mathrm i}'$の、テーブル中心Pを原点とした場合の$X$座標はそれぞれ、
\begin{align*}
  \text{C面側端点:}&~~
  \varDelta'\cos\theta-\sqrt{R_\mathrm i^2-f_\mathrm B^2}\ ,\\
  \text{A面側端点:}&~~
  \varDelta'\cos\theta-\sqrt{R_\mathrm o^2-f_\mathrm B^2}\ .
\end{align*}
したがって、ボトム端における(AC外径の)中点B$_\mathrm c'$の$X$座標は、
%% label{eq:tableBc}
\begin{align}
  \label{eq:tableBc}
  \varDelta'\cos\theta-\frac{\sqrt{R_\mathrm o^2-f_\mathrm B^2}+\sqrt{R_\mathrm i^2-f_\mathrm B^2}}2
\end{align}
これより、湾曲中心B$_{R_\mathrm c}'$と外径中心B$_\mathrm c'$との差は\pageeqref{eq:BRc-Bc}となることがわかる。




\clearpage
%%%%%%%%%%%%%%%%%%%%%%%%%%%%%%%%%%%%%%%%%%%%%%%%%%%%%%%%%%
%% section 2.3 %%%%%%%%%%%%%%%%%%%%%%%%%%%%%%%%%%%%%%%%%%%
%%%%%%%%%%%%%%%%%%%%%%%%%%%%%%%%%%%%%%%%%%%%%%%%%%%%%%%%%%
\section{端面加工の工具径補正}
端面の加工として、$X+$, $Y+$方向の角から始めて
%% footnote %%%%%%%%%%%%%%%%%%%%%
\footnote{\DMname の場合、工具交換位置に近いので、このほうが移動距離が短くなる。}、
%%%%%%%%%%%%%%%%%%%%%%%%%%%%%%%%%
(工具から見て)時計回りに加工する場合を考える。
このとき加工の経路として、単純に外径の値を指定すれば加工することは可能である。
しかしその場合、工具(フェイスミル)のほぼ中心に近い位置で切削する形になるので、工具に大きな負荷がかかることになる。
これを避けるために、理想的には、内径$w_{\mathrm T, \mathrm B}$の外側に沿う形で切削するのが望ましい。
つまり、内径$w_{\mathrm T, \mathrm B}$を基準として工具半径分だけ(進行方向に対して左側に)補正をする形にすればよい。
ここでは誤差等を考慮して、内径から$\delta_w$だけ縮めた輪郭(の外側)に沿う形を考える。



\paragraph{加工の開始可能範囲}\noindent
初めの位置は$X+$, $Y+$方向の角の上方($Y+$方向)に工具があるものとする。
工具の刃径(直径)を$\phi_\mathrm D$, 最大刃径(直径)$\phi'_\mathrm D$とすると
%% footnote %%%%%%%%%%%%%%%%%%%%%
\footnote{通常、刃径はDC、最大刃径はDCXと表記され、それぞれ直径として与えられることが多い。}、
%%%%%%%%%%%%%%%%%%%%%%%%%%%%%%%%%
$X$位置については、工具の中心が
\begin{align}
  \label{eq:tanmenKakouStartX}
  \frac{w_x}2-\delta_w+\frac{\phi_\mathrm D}2
\end{align}
にあればよい。
そのためここでは、まず$X$方向に絶対座標(G90)
\begin{align*}
  \frac{w_x}2-\delta_w
\end{align*}
まで移動し、その後に工具半径分の補正量として$\nicefrac{\phi_\mathrm D}2$だけ$X+$方向にずらす形で、下方向($Y-$方向)に移動する場合を考える。



%%%%%%%%%%%%%%%%%%%%%%%%%%%%%%%%%%%%%%%%%%%%%%%%%%%%%%%%%%
%% subsection 2.3.1 %%%%%%%%%%%%%%%%%%%%%%%%%%%%%%%%%%%%%%
%%%%%%%%%%%%%%%%%%%%%%%%%%%%%%%%%%%%%%%%%%%%%%%%%%%%%%%%%%
\subsection{工具径補正G41を用いる場合}
G41を用いて工具径補正を行う場合は、動き始めは$X$方向の補正分も加えて斜めに移動することになる。
ここで、
\begin{enumerate}
\item $X$, $Y$方向には同じ速さで動く
\item $X$方向の移動がなくなるまで工具はモールドに触れない
\end{enumerate}
とすると、加工(移動)の開始位置の$Y$座標は、
\begin{align*}
  \frac{W_y}2+\frac{\phi'_\mathrm D+\phi_\mathrm D}2
  = \frac{w_y}2+\tau_y+\frac{\phi'_\mathrm D+\phi_\mathrm D}2
\end{align*}
より上方向($Y+$方向)であればよい。
なお、$W_y$, $\tau_y$はBD方向の外径およびBD方向の肉厚である
%% footnote %%%%%%%%%%%%%%%%%%%%%
\footnote{ここでは話を単純化し、めっき膜厚や偏肉は無視している。}。
%%%%%%%%%%%%%%%%%%%%%%%%%%%%%%%%%



%%%%%%%%%%%%%%%%%%%%%%%%%%%%%%%%%%%%%%%%%%%%%%%%%%%%%%%%%%
%% subsection 2.3.2 %%%%%%%%%%%%%%%%%%%%%%%%%%%%%%%%%%%%%%
%%%%%%%%%%%%%%%%%%%%%%%%%%%%%%%%%%%%%%%%%%%%%%%%%%%%%%%%%%
\subsection{手動で補正を行う場合}
手動で補正する場合は、予め$X$位置を\pageeqref{eq:tanmenKakouStartX}に移動しておいて、そのまま下方向($Y-$方向)に移動すればよい。
よって、加工(移動)の開始位置の$Y$座標は、
\begin{align*}
  \frac{W_y+\phi_\mathrm D}2 = \frac{w_y+\phi_\mathrm D}2+\tau_y
\end{align*}
より上方向($Y+$方向)にあればよい%% footnote %%%%%%%%%%%%%%%%%%%%%
\footnote{実際のプログラムでは、安全を考慮して$\nicefrac{w_y}2+\phi'_\mathrm D$としている。
この場合、
\begin{align*}
  \phi'_\mathrm D > \frac{\phi_\mathrm D}2+\tau_y
\end{align*}
である限り、衝突は生じないことになる。
一般に、$\tau_y < \nicefrac{\phi_\mathrm D}2$であるので、これは常に満たされる。}。
%%%%%%%%%%%%%%%%%%%%%%%%%%%%%%%%%




%%%%%%%%%%%%%%%%%%%%%%%%%%%%%%%%%%%%%%%%%%%%%%%%%%%%%%%%%%
%%           %%%%%%%%%%%%%%%%%%%%%%%%%%%%%%%%%%%%%%%%%%%%%
%% chapter 3 %%%%%%%%%%%%%%%%%%%%%%%%%%%%%%%%%%%%%%%%%%%%%
%%           %%%%%%%%%%%%%%%%%%%%%%%%%%%%%%%%%%%%%%%%%%%%%
%%%%%%%%%%%%%%%%%%%%%%%%%%%%%%%%%%%%%%%%%%%%%%%%%%%%%%%%%%
\chapter{外削}
モールドに外削があるときは、たいていの場合、A側内面の端面における位置を基準として考えることが多い。

トップ・ボトム端における内径をそれぞれ$w_\mathrm T$, $w_\mathrm B$, 外削径をそれぞれ$\mathfrak W_\mathrm T$, $\mathfrak W_\mathrm B$, A側肉厚をそれぞれ$\tau_\mathrm T$, $\tau_\mathrm B$とする。
また、内面のめっき膜厚を$\mu$とし、通り心(トップ外削中心$\mathfrak T_\mathrm c$とボトム外削中心$\mathfrak B_\mathrm c$の差)の$X$, $Y$成分をそれぞれ$T_x$, $T_y$とする。
ただし、$T_x \geqq 0$として、トップ外削中心$\mathfrak T_\mathrm c$はボトム外削中心$\mathfrak B_\mathrm c$よりA面方向にあるものとする。
%%%%%%%%%%%%%%%%%%%%%%%%%%%%%%%%%%%%%%%%%%%%%%%%%%%%%%%%%%
%% hosoku %%%%%%%%%%%%%%%%%%%%%%%%%%%%%%%%%%%%%%%%%%%%%%%%
%%%%%%%%%%%%%%%%%%%%%%%%%%%%%%%%%%%%%%%%%%%%%%%%%%%%%%%%%%
\begin{hosoku}
内径$w_\mathrm T$, $w_\mathrm B$は、湾曲の中心O(またはO$'$)に向かった方向にあることに注意。
内径の中心がそれぞれの端に位置している。
\end{hosoku}
%%%%%%%%%%%%%%%%%%%%%%%%%%%%%%%%%%%%%%%%%%%%%%%%%%%%%%%%%%
%%%%%%%%%%%%%%%%%%%%%%%%%%%%%%%%%%%%%%%%%%%%%%%%%%%%%%%%%%
%%%%%%%%%%%%%%%%%%%%%%%%%%%%%%%%%%%%%%%%%%%%%%%%%%%%%%%%%%



%%%%%%%%%%%%%%%%%%%%%%%%%%%%%%%%%%%%%%%%%%%%%%%%%%%%%%%%%%
%% section 3.1 %%%%%%%%%%%%%%%%%%%%%%%%%%%%%%%%%%%%%%%%%%%
%%%%%%%%%%%%%%%%%%%%%%%%%%%%%%%%%%%%%%%%%%%%%%%%%%%%%%%%%%
\section{ボトム側外削径の中心(ボトム基準)}
肉厚を基準とする場合、ボトム端のA側肉厚を基準にすることが多い。
このとき、ボトム端における外削径中心B$_\mathrm c'$から、ボトム端内径$w_\mathrm B$の半分を引き、さらにA側肉厚$\tau_\mathrm B$とめっき膜厚$\mu$との差を引いたものが(おおよその)外削A側面の位置$\mathfrak B_\mathrm o'$に相当する
%% footnote %%%%%%%%%%%%%%%%%%%%%
\footnote{ボトム側が工具側にある場合は、A面は$X$の負方向にあることに注意。}。
%%%%%%%%%%%%%%%%%%%%%%%%%%%%%%%%%


%%%%%%%%%%%%%%%%%%%%%%%%%%%%%%%%%%%%%%%%%%%%%%%%%%%%%%%%%%
%% subsubsection 3.1.1 %%%%%%%%%%%%%%%%%%%%%%%%%%%%%%%%%%%
%%%%%%%%%%%%%%%%%%%%%%%%%%%%%%%%%%%%%%%%%%%%%%%%%%%%%%%%%%
\subsection[スペーサを用いた場合の\texorpdfstring{$\mathfrak B_\mathrm c'$}{Bc'}]
           {スペーサを用いた場合の$\boldsymbol{\mathfrak B_\mathrm c'}$}
厚さ$\delta$のスペーサを用いた場合、テーブル中心Pを原点とした
%% footnote %%%%%%%%%%%%%%%%%%%%%
\footnote{マシニングによって機械原点(の$X$座標)がテーブル中心Pと同じだったり異なったりする場合がある。}\relax
%%%%%%%%%%%%%%%%%%%%%%%%%%%%%%%%%
ボトム側外削径の中心$\mathfrak B_\mathrm c'$の(おおよその)$X$座標は、\pageeqref{eq:spacerBc}より、
\begin{align*}
  \varDelta-\frac{\sqrt{R_\mathrm o^2-f_\mathrm B^2}+\sqrt{R_\mathrm i^2-f_\mathrm B^2}}2-\frac\delta2
  +\sqrt{R_\mathrm i'^2-\frac{\delta^2+(2\bar l)^2}4}\frac{2\bar l}{\sqrt{\delta^2+(2\bar l)^2}}
  -\frac{w_\mathrm B}2-\tau_\mathrm B+\frac{\mathfrak W_\mathrm B}2\ .
\end{align*}
%%%%%%%%%%%%%%%%%%%%%%%%%%%%%%%%%%%%%%%%%%%%%%%%%%%%%%%%%%
%% hosoku %%%%%%%%%%%%%%%%%%%%%%%%%%%%%%%%%%%%%%%%%%%%%%%%
%%%%%%%%%%%%%%%%%%%%%%%%%%%%%%%%%%%%%%%%%%%%%%%%%%%%%%%%%%
\begin{hosoku}
正確には、ボトム端における(内径ではなく)A・C面側の内面中心b$_\mathrm c'$を見る必要がある。
しかし実際の作業においては、これはタッチセンサーの測定開始点として用いるものであるため、おおよその値($\pm10$mm以内程度)で十分である。
そのため、ここでは単純に中心b$_\mathrm c'$の代わりにボトム外削径中心B$_\mathrm c'$とし、またボトム端における内径$w_\mathrm B$を用いている。
さらにいうと、外削径中心B$_\mathrm c'$はボトム端の湾曲中心B$_{\mathrm R_\mathrm c}'$で代用してもまず問題はない。
\end{hosoku}
%%%%%%%%%%%%%%%%%%%%%%%%%%%%%%%%%%%%%%%%%%%%%%%%%%%%%%%%%%
%%%%%%%%%%%%%%%%%%%%%%%%%%%%%%%%%%%%%%%%%%%%%%%%%%%%%%%%%%
%%%%%%%%%%%%%%%%%%%%%%%%%%%%%%%%%%%%%%%%%%%%%%%%%%%%%%%%%%
これは飽くまで(図面の数字をもとにした)計算値であり、タッチセンサーでの測定開始点として用いる。
そして、現物のボトム端に相当する箇所のA面(負方向)側内面b$_\mathrm o'$の位置を直接計測し、その位置を基準として(ワーク座標系の)原点$\mathfrak B_\mathrm c'$を定める。
計測で定めた原点$\mathfrak B_\mathrm c'$と、ボトム端A側内面b$_\mathrm o'$との差の$X$座標は、
\begin{align*}
  -\left(\frac{\mathfrak W_\mathrm B}2-\tau_\mathrm B+\mu\right).
\end{align*}


%%%%%%%%%%%%%%%%%%%%%%%%%%%%%%%%%%%%%%%%%%%%%%%%%%%%%%%%%%
%% subsubsection 3.1.2 %%%%%%%%%%%%%%%%%%%%%%%%%%%%%%%%%%%
%%%%%%%%%%%%%%%%%%%%%%%%%%%%%%%%%%%%%%%%%%%%%%%%%%%%%%%%%%
\subsection[テーブルを傾けた場合の\texorpdfstring{$\mathfrak B_\mathrm c'$}{Bc'}]
           {テーブルを傾けた場合の$\boldsymbol{\mathfrak B_\mathrm c'}$}
テーブルを$-\theta$傾けた場合、テーブル中心Pを原点としたボトム側外削径の中心$\mathfrak B_\mathrm c'$の(おおよその)$X$座標は、\pageeqref{eq:tableBc}より、
\begin{align*}
  \varDelta'\cos\theta-\frac{\sqrt{R_\mathrm o^2-f_\mathrm B^2}+\sqrt{R_\mathrm i^2-f_\mathrm B^2}}2
  -\frac{w_\mathrm B}2-\tau_\mathrm B+\frac{\mathfrak W_\mathrm B}2\ .
\end{align*}
計測して定めた原点$\mathfrak B_\mathrm c'$と、ボトム端A側内面b$_\mathrm o'$との差の$X$座標は、
\begin{align*}
  -\left(\frac{\mathfrak W_\mathrm B}2-\tau_\mathrm B+\mu\right).
\end{align*}


%%%%%%%%%%%%%%%%%%%%%%%%%%%%%%%%%%%%%%%%%%%%%%%%%%%%%%%%%%
%% subsection 3.1.3 %%%%%%%%%%%%%%%%%%%%%%%%%%%%%%%%%%%%%%
%%%%%%%%%%%%%%%%%%%%%%%%%%%%%%%%%%%%%%%%%%%%%%%%%%%%%%%%%%
\subsection{トップ側外削径中心(ボトム基準)}
トップ側にも外削がある場合、ボトム側外削から通り芯を指定する形でトップ外削の位置を決めるのが通常である。
このとき、テーブル中心Pを原点としたトップ側外削径中心$\mathfrak T_\mathrm c'$の$X$座標は、計測で定めた$\mathfrak B_\mathrm c'$の$X$座標$\mathcal G_{\mathrm Bx}$の符号を反転し
%% footnote %%%%%%%%%%%%%%%%%%%%%
\footnote{トップ側が工具側にある場合は、A面は$X$の正方向にある。
ボトム側と比べてテーブルを$B$軸($Y$軸まわり)に$180^\circ$回転する必要があるため、$X$座標の符号が反転する形になる。}、
%%%%%%%%%%%%%%%%%%%%%%%%%%%%%%%%%
通り芯$T_x$の分を加味すればよい。
したがって、
\begin{align*}
  -\mathcal G_{Bx}+T_x
\end{align*}
で与えられる
%% footnote %%%%%%%%%%%%%%%%%%%%%
\footnote{$Y$座標については、$B$軸の回転に影響しないので、$\mathcal G_{\mathrm By}+T_y$となる。
なお、実際の作業においては、$T_y = 0$であることが通常である。}。
%%%%%%%%%%%%%%%%%%%%%%%%%%%%%%%%%
ただし実際の作業では、テーブル中心Pの回転中心からのずれも考慮する必要がある
%% footnote %%%%%%%%%%%%%%%%%%%%%
\footnote{回転中心とテーブル中心は通常一致しているものとして考えるが、実際にはわずかにずれている。
特に$X$方向のずれは、$B$軸回転を伴う場合に効いてくる。}。
%%%%%%%%%%%%%%%%%%%%%%%%%%%%%%%%%



\clearpage
%%%%%%%%%%%%%%%%%%%%%%%%%%%%%%%%%%%%%%%%%%%%%%%%%%%%%%%%%%
%% section 3.2 %%%%%%%%%%%%%%%%%%%%%%%%%%%%%%%%%%%%%%%%%%%
%%%%%%%%%%%%%%%%%%%%%%%%%%%%%%%%%%%%%%%%%%%%%%%%%%%%%%%%%%
\section{トップ側外削径の中心}


%%%%%%%%%%%%%%%%%%%%%%%%%%%%%%%%%%%%%%%%%%%%%%%%%%%%%%%%%%
%% subsection 3.2.1 %%%%%%%%%%%%%%%%%%%%%%%%%%%%%%%%%%%%%%
%%%%%%%%%%%%%%%%%%%%%%%%%%%%%%%%%%%%%%%%%%%%%%%%%%%%%%%%%%
\subsection[スペーサを用いた場合の\texorpdfstring{$\mathfrak T_\mathrm c'$}{Tc'}]
           {スペーサを用いた場合の$\boldsymbol{\mathfrak T_\mathrm c'}$}
トップ端外削A側面が基準となる場合も考慮しておく。
この場合も考えかたはボトム基準のそれと同様である。
テーブル中心Pを原点とした場合の、トップ側外削径の中心$\mathfrak T_\mathrm c'$のおおよその$X$座標は、\pageeqref{eq:spacerTc}より、
\begin{align*}
  -\varDelta+\frac{\sqrt{R_\mathrm o^2-f_\mathrm T^2}+\sqrt{R_\mathrm i^2-f_\mathrm T^2}}2+\frac\delta2
  -\sqrt{R_\mathrm i'^2-\frac{\delta^2+(2\bar l)^2}4}\frac{2\bar l}{\sqrt{\delta^2+(2\bar l)^2}}
  +\frac{w_\mathrm T}2+\tau_\mathrm T-\frac{\mathfrak W_\mathrm T}2\ .
\end{align*}
これをタッチセンサーでの測定開始点とし、計測した原点の$X$座標(実測値)を$\mathcal G_{tx}$とすると、トップ端におけるA面側内面と$\mathcal G_{tx}$との差の$X$座標は、
\begin{align*}
  \frac{\mathfrak W_\mathrm T}2-\tau_\mathrm T+\mu~.
\end{align*}


%%%%%%%%%%%%%%%%%%%%%%%%%%%%%%%%%%%%%%%%%%%%%%%%%%%%%%%%%%
%% subsection 3.2.2 %%%%%%%%%%%%%%%%%%%%%%%%%%%%%%%%%%%%%%
%%%%%%%%%%%%%%%%%%%%%%%%%%%%%%%%%%%%%%%%%%%%%%%%%%%%%%%%%%
\subsection[テーブルを傾けた場合の\texorpdfstring{$\mathfrak T_\mathrm c'$}{Tc'}]
           {テーブルを傾けた場合の$\boldsymbol{\mathfrak T_\mathrm c'}$}
テーブルを$-\theta$傾けた場合、テーブル中心Pを原点としたボトム側外削径の中心$\mathfrak T_\mathrm c'$の(おおよその)$X$座標は、\pageeqref{eq:tableTc}より、
\begin{align*}
  \frac{\sqrt{R_\mathrm o^2-f_\mathrm T^2}+\sqrt{R_\mathrm i^2-f_\mathrm T^2}}2-\varDelta'\cos\theta
  +\frac{w_\mathrm T}2+\tau_\mathrm T-\frac{\mathfrak W_\mathrm T}2\ .
\end{align*}
計測して定めた原点$\mathfrak T_\mathrm c'$と、トップ端A側内面t$_\mathrm o'$との差の$X$座標は、
\begin{align*}
  \frac{\mathfrak W_\mathrm T}2-\tau_\mathrm T+\mu~.
\end{align*}




%%%%%%%%%%%%%%%%%%%%%%%%%%%%%%%%%%%%%%%%%%%%%%%%%%%%%%%%%%
%% subsection 3.2.3 %%%%%%%%%%%%%%%%%%%%%%%%%%%%%%%%%%%%%%
%%%%%%%%%%%%%%%%%%%%%%%%%%%%%%%%%%%%%%%%%%%%%%%%%%%%%%%%%%
\subsection{ボトム側外削径中心(トップ基準)}
ボトム側にも外削がある場合、トップ側外削から通り芯を指定する形でボトム外削の位置を決めることが多い。
このとき、テーブル中心Pを原点としたボトム側外削径中心$\mathfrak B_\mathrm c'$の$X$座標は、計測で定めた$\mathfrak T_\mathrm c'$の$X$座標$\mathcal G_{\mathrm Tx}$の符号を反転し、通り芯$T_x$の分を加味すればよい。
したがって、
\begin{align*}
  -\mathcal G_{Tx}+T_x
\end{align*}
で与えられる。





%%%%%%%%%%%%%%%%%%%%%%%%%%%%%%%%%%%%%%%%%%%%%%%%%%%%%%%%%%
%%           %%%%%%%%%%%%%%%%%%%%%%%%%%%%%%%%%%%%%%%%%%%%%
%% chapter 4 %%%%%%%%%%%%%%%%%%%%%%%%%%%%%%%%%%%%%%%%%%%%%
%%           %%%%%%%%%%%%%%%%%%%%%%%%%%%%%%%%%%%%%%%%%%%%%
%%%%%%%%%%%%%%%%%%%%%%%%%%%%%%%%%%%%%%%%%%%%%%%%%%%%%%%%%%
\chapter{溝}
モールドの溝について考える。
溝に関しては、その基準が以下のように与えられる場合が考えられる。
\begin{enumerate}
\item 溝径の中心Mが、モールドの湾曲中心線上にある場合
\item 溝径の中心Mが、(トップ側)外削径の中心線上にある場合
\item A面側の溝深さに指定がある場合
\end{enumerate}
なお、溝径を$W_\mathrm M$, 溝位置(端面から溝までの長さ)・溝幅・A側溝深さをそれぞれ$\kappa_p$, $\kappa_w$, $\kappa_d$とする。
このときいずれの場合も、$y$方向(機内における$Z$方向
%% footnote %%%%%%%%%%%%%%%%%%%%%
\footnote{計算上の$xy$座標($x$:実軸, $y$:虚軸)と、機内における$XZ$座標とが混在する形で話を進めているので注意されたし。})
%%%%%%%%%%%%%%%%%%%%%%%%%%%%%%%%%
の切削範囲は、テーブル中心Pを原点として、
\begin{align*}
  \big[f_\mathrm T'-(\kappa_p+\kappa_w)\ ,\ f_\mathrm T'-\kappa_p\big]
\end{align*}
であり、また溝中心M$'$の$y$座標はこの切削範囲の中央
\begin{align*}
  f_\mathrm T'-\left(\kappa_p+\frac{\kappa_w}2\right)
\end{align*}
で与えられる。



%%%%%%%%%%%%%%%%%%%%%%%%%%%%%%%%%%%%%%%%%%%%%%%%%%%%%%%%%%
%% section 4.1 %%%%%%%%%%%%%%%%%%%%%%%%%%%%%%%%%%%%%%%%%%%
%%%%%%%%%%%%%%%%%%%%%%%%%%%%%%%%%%%%%%%%%%%%%%%%%%%%%%%%%%
\section{湾曲中心が基準の場合}
トップ端における湾曲中心T$_{R_\mathrm c}'$と溝中心M$'$との$x$方向の差は、
\begin{align*}
  \sqrt{R_\mathrm c^2-\left(f_\mathrm T-\kappa_p-\frac{\kappa_w}2\right)^{\!2}}
  -\sqrt{R_\mathrm c^2-f_\mathrm T^2}
\end{align*}
で与えられる。


%%%%%%%%%%%%%%%%%%%%%%%%%%%%%%%%%%%%%%%%%%%%%%%%%%%%%%%%%%
%% subsubsection 4.1.1 %%%%%%%%%%%%%%%%%%%%%%%%%%%%%%%%%%%
%%%%%%%%%%%%%%%%%%%%%%%%%%%%%%%%%%%%%%%%%%%%%%%%%%%%%%%%%%
\subsection{スペーサを用いた場合の溝中心(湾曲中心基準)}
溝中心M$'$がモールドの湾曲中心線上にある場合、テーブル中心Pを原点とした$x$座標は、\pageeqref{eq:spacerTRc}より、
\begin{align*}
  -\varDelta+\sqrt{R_\mathrm c^2-\left(f_\mathrm T-\kappa_p-\frac{\kappa_w}2\right)^{\!2}}+\frac\delta2
  -\sqrt{R_\mathrm i'^2-\frac{\delta^2+(2\bar l)^2}4}\frac{2\bar l}{\sqrt{\delta^2+\left(2\bar l\right)^2}}
\end{align*}
となる。
なお実際の作業では、簡単のため、トップ端面の外側中心T$_\mathrm c'$を測定し、それをトップ端面における湾曲中心T$_{R_\mathrm c}'$とみなして溝中心M$'$の位置を計算することが多い。
実測した外側中心の$X$・$Y$座標$G_{\mathrm Tx}$, $G_{\mathrm Ty}$を湾曲中心のそれとみなすと、機内における溝中心M$'$の位置は、テーブル中心Pを原点として、
%% label{eq:Mreal}
\begin{subequations}
  \label{eq:Mreal}
\begin{align}
  \left(
    G_{\mathrm Tx}
    +\sqrt{R_\mathrm c^2-\left(f_\mathrm T-\kappa_p-\frac{\kappa_w}2\right)^{\!2}}
    -\sqrt{R_\mathrm c^2-f_\mathrm T^2}\ ,\
    G_{\mathrm Ty}~,~
    f_\mathrm T'-\kappa_p-\frac{\kappa_w}2
  \right).
\end{align}
湾曲中心とみなさずに正確に求めるなら、これに\pageeqref{eq:TRc-Tc}を引けばよい。
その場合の$X$座標は、
\begin{align}
  G_{\mathrm Tx}
  +\sqrt{R_\mathrm c^2-\left(f_\mathrm T-\kappa_p-\frac{\kappa_w}2\right)^{\!2}}
  -\frac{\sqrt{R_\mathrm o^2-f_\mathrm T^2}+\sqrt{R_\mathrm i^2-f_\mathrm T^2}}2\ .
\end{align}
\end{subequations}


%%%%%%%%%%%%%%%%%%%%%%%%%%%%%%%%%%%%%%%%%%%%%%%%%%%%%%%%%%
%% subsubsection 4.1.2 %%%%%%%%%%%%%%%%%%%%%%%%%%%%%%%%%%%
%%%%%%%%%%%%%%%%%%%%%%%%%%%%%%%%%%%%%%%%%%%%%%%%%%%%%%%%%%
\subsection{テーブルを傾けた場合の溝中心(湾曲中心基準)}
溝中心M$'$がモールドの湾曲中心線上にある場合、テーブル中心Pを原点とした$x$座標は、\pageeqref{eq:tableTRc}より、
\begin{align*}
  \sqrt{R_\mathrm c^2-\left(f_\mathrm T-\kappa_p-\frac{\kappa_w}2\right)^{\!2}}
  -\varDelta'\cos\theta\ .
\end{align*}
実測した外側中心の$X$・$Y$座標$G_{\mathrm Tx}$, $G_{\mathrm Ty}$を湾曲中心のそれとみなした場合とそうでない場合は、\pageeqref{eq:Mreal}で与えられる。




\clearpage
%%%%%%%%%%%%%%%%%%%%%%%%%%%%%%%%%%%%%%%%%%%%%%%%%%%%%%%%%%
%% section 4.2 %%%%%%%%%%%%%%%%%%%%%%%%%%%%%%%%%%%%%%%%%%%
%%%%%%%%%%%%%%%%%%%%%%%%%%%%%%%%%%%%%%%%%%%%%%%%%%%%%%%%%%
\section{外削径の中心が基準の場合}
溝中心M$'$がトップ外削径の中心線上にある場合、機内におけるその位置座標は、
\begin{align*}
  \left(
    -\mathcal G_{Bx}+T_x\ ,\
    \mathcal G_{By}\ ,\
    f_\mathrm T'-\kappa_p-\frac{\kappa_w}2
  \right) \qquad
  \text{または}\qquad
  \left(
    \mathcal G_{Bx}\ ,\
    \mathcal G_{By}\ ,\
    f_\mathrm T'-\kappa_p-\frac{\kappa_w}2
  \right).
\end{align*}
ただし、前者はボトムの外削を基準にした(ボトム基準の通り芯がある)場合であり、後者はトップの外削を基準にした場合である。




%%%%%%%%%%%%%%%%%%%%%%%%%%%%%%%%%%%%%%%%%%%%%%%%%%%%%%%%%%
%% section 4.3 %%%%%%%%%%%%%%%%%%%%%%%%%%%%%%%%%%%%%%%%%%%
%%%%%%%%%%%%%%%%%%%%%%%%%%%%%%%%%%%%%%%%%%%%%%%%%%%%%%%%%%
\section{A面側の溝深さが基準の場合}
モールドA側面からの溝深さが指定されている場合を考える。
このとき、溝中心の位置の$X$座標は、テーブル中心Pを原点として、
\begin{align*}
  \sqrt{R_\mathrm o^2-\left(f_\mathrm T-\kappa_p-\frac{\kappa_w}2\right)^{\!2}}-\kappa_d-\frac{W_\mathrm M}2
  -\varDelta'
\end{align*}
で与えられる。
ここで、$W_{mx}$は溝のAC方向の径を表す。
なお実際の作業では、モールドのA側外面の溝幅中央に相当する箇所を直接計測し、その位置を基準として原点を割り出す。
その原点の$X$座標(実測値)を$G_{mx}$とすると、溝幅中央に対するモールドのA側外面と溝中心$G_{mx}$との差は、
\begin{align*}
  \frac{W_m}2+\kappa_d
\end{align*}
となる。




%%%%%%%%%%%%%%%%%%%%%%%%%%%%%%%%%%%%%%%%%%%%%%%%%%%%%%%%%%
%%           %%%%%%%%%%%%%%%%%%%%%%%%%%%%%%%%%%%%%%%%%%%%%
%% chapter 5 %%%%%%%%%%%%%%%%%%%%%%%%%%%%%%%%%%%%%%%%%%%%%
%%           %%%%%%%%%%%%%%%%%%%%%%%%%%%%%%%%%%%%%%%%%%%%%
%%%%%%%%%%%%%%%%%%%%%%%%%%%%%%%%%%%%%%%%%%%%%%%%%%%%%%%%%%
\chapter{内面溝}
ここでは主に内面溝に関する計測・加工に必要な、モールドの幾何学的性質を考える。

なお、内面溝の加工は\MMname で行うことはできず、\DMname のみで行う。
また\DMname では、振分長の調整についてスペーサを用いた方法は行わず、テーブルの回転を用いた方法のみで行う方針である。
したがって、スペーサを用いた方法の場合は考慮する必要がない。
そのため以降では、(内面溝に関する計測・加工については)テーブルを$-\theta$だけ回転した場合についてのみを考えることにする。




%%%%%%%%%%%%%%%%%%%%%%%%%%%%%%%%%%%%%%%%%%%%%%%%%%%%%%%%%%
%% section 5.1 %%%%%%%%%%%%%%%%%%%%%%%%%%%%%%%%%%%%%%%%%%%
%%%%%%%%%%%%%%%%%%%%%%%%%%%%%%%%%%%%%%%%%%%%%%%%%%%%%%%%%%
\section{ノーテーション}
初めに、内面溝に関するノーテーションを簡単にまとめておく。
なお内面溝は振分けのトップ側にあるため、モールドはトップ側が工具側に向いているものとして話を進める。
%%%%%%%%%%%%%%%%%%%%%%%%%%%%%%%%%%%%%%%%%%%%%%%%%%%%%%%%%%
%% tcolorbox %%%%%%%%%%%%%%%%%%%%%%%%%%%%%%%%%%%%%%%%%%%%%
%%%%%%%%%%%%%%%%%%%%%%%%%%%%%%%%%%%%%%%%%%%%%%%%%%%%%%%%%%
\begin{tcolorbox}[title={内面溝に関するノーテーション}, fonttitle=\gtfamily\bfseries, breakable, enhanced jigsaw]
\begin{enumerate}
\item
\subparagraph{列の数えかた}
内面溝は$m$列あるものとし、トップ側から順に1列目, 2列目, …,$m$列目のように数える。

\item
\subparagraph{列内の個数の数えかた}
各々の列の内面溝は、AC面側については工具側からみて下から順に、BD面については工具側からみて右から順に1つ目,2つ目,…のように数える。

\item
\subparagraph{内面溝の寸法}
トップ端面から1列目までの距離を$q$, 鉛直・水平方向のピッチをそれぞれ$p_z$, $p_x$とし、$i$列目の長さをそれぞれ$d_i$とする。

特に、奇数列目の長さが全て同じ場合はその長さを$d_\mathrm o$, 偶数列目の長さが全て同じ場合はその長さを$d_\mathrm e$とも表記する。
(\pageautoref{fn:generallyDimpleN}および\pageautoref{hosoku:generallyDimpleN}参照)

\item
\subparagraph{内径テーパ表の寸法}
内径テーパ表におけるトップ端からの距離を$\lambda_i$ ($i = 0$, $1$, $2$, $\cdots$), それに対するAC・BD側内径をそれぞれ$w_{\mathrm Ai}$, $w_{\mathrm Bi}$とする。
(\pageautoref{hosoku:example4taper}参照)

\item
\subparagraph{内径の(近似)寸法}
トップ端から$\lambda$の位置のAC内径を$w_{\mathrm A\lambda}$と表す。
このとき$w_{\mathrm A\lambda}$は、$\lambda_j \leqq \lambda < \lambda_{j+1}$に対する$w_{\mathrm Aj}$, $w_{\mathrm Aj+1}$の加重算術平均(ウェイト算術平均)
\begin{align*}
  w_{\mathrm A\lambda}
  = \frac{(\lambda-\lambda_j)w_{\mathrm Aj+1}+(\lambda_{j+1}-\lambda)w_{\mathrm Aj}}{\lambda_{j+1}-\lambda_j}
  \qquad
  \Big(\lambda_j \leqq \lambda < \lambda_{j+1}\Big)
\end{align*}
とみなすことにする。($w_{\mathrm B\lambda}$についても同様)

\item
\subparagraph{めっき厚を含めた内径の(近似)寸法}
めっき膜厚$\mu$を考慮したAC・BD内径$w'_{\mathrm A\lambda}$, $w'_{\mathrm B\lambda}$をそれぞれ以下のように表す。
\begin{align*}
  w'_{\mathrm A\lambda} \equiv w_{\mathrm A\lambda}+2\mu~, \quad
  w'_{\mathrm B\lambda} \equiv w_{\mathrm B\lambda}+2\mu~.
\end{align*}
\end{enumerate}
\end{tcolorbox}\noindent
%%%%%%%%%%%%%%%%%%%%%%%%%%%%%%%%%%%%%%%%%%%%%%%%%%%%%%%%%%
%%%%%%%%%%%%%%%%%%%%%%%%%%%%%%%%%%%%%%%%%%%%%%%%%%%%%%%%%%
%%%%%%%%%%%%%%%%%%%%%%%%%%%%%%%%%%%%%%%%%%%%%%%%%%%%%%%%%%
このとき$m$列目の内面溝の個数$n_m$は、$n_m = \nicefrac{d_m}{p_x}+1$となる
%% footnote %%%%%%%%%%%%%%%%%%%%%
\footnote{\label{fn:generallyDimpleN}
たいていの場合、奇数列の個数は全て同じ数$n_\mathrm o$であり、偶数列の個数も全て同じ$n_\mathrm e$である。
また$|n_\mathrm o-n_\mathrm d| = 1$である。}。
%%%%%%%%%%%%%%%%%%%%%%%%%%%%%%%%%
%%%%%%%%%%%%%%%%%%%%%%%%%%%%%%%%%%%%%%%%%%%%%%%%%%%%%%%%%%
%% hosoku %%%%%%%%%%%%%%%%%%%%%%%%%%%%%%%%%%%%%%%%%%%%%%%%
%%%%%%%%%%%%%%%%%%%%%%%%%%%%%%%%%%%%%%%%%%%%%%%%%%%%%%%%%%
\begin{hosoku}[label=hosoku:example4taper]
たとえば内径テーパ表の値が25mmピッチの場合、$\lambda_0=0$, $\lambda_1=25$, $\lambda_2=50$, $\cdots$とし、それぞれのACおよびBD側内径を$w_{\mathrm A0}$, $w_{\mathrm A1}$, $w_{\mathrm A2}$, $\cdots$および$w_{\mathrm B0}$, $w_{\mathrm B1}$, $w_{\mathrm B2}$, $\cdots$とする、という意味である。
ここでは離散値である$\lambda_i$を、連続値$\lambda$に(近似的に)置きかえている。
実際、たとえば$\lambda = \lambda_j$のとき$w_{\mathrm Aj} = w_{\mathrm A\lambda}$となることがわかる。
\end{hosoku}\relax
%%%%%%%%%%%%%%%%%%%%%%%%%%%%%%%%%%%%%%%%%%%%%%%%%%%%%%%%%%
%%%%%%%%%%%%%%%%%%%%%%%%%%%%%%%%%%%%%%%%%%%%%%%%%%%%%%%%%%
%%%%%%%%%%%%%%%%%%%%%%%%%%%%%%%%%%%%%%%%%%%%%%%%%%%%%%%%%%
%%%%%%%%%%%%%%%%%%%%%%%%%%%%%%%%%%%%%%%%%%%%%%%%%%%%%%%%%%
%% hosoku %%%%%%%%%%%%%%%%%%%%%%%%%%%%%%%%%%%%%%%%%%%%%%%%
%%%%%%%%%%%%%%%%%%%%%%%%%%%%%%%%%%%%%%%%%%%%%%%%%%%%%%%%%%
\begin{hosoku}
内径テーパの$Z$方向のピッチ$\lambda_{i+1}-\lambda_i$は常に一定の場合が多い。
$\lambda_{i+1}-\lambda_i$が$i$について常に一定であれば、$\lambda_j \leqq z < \lambda_{j+1}$となる$j$は、
\begin{align*}
  j = z \bDiv (\lambda_{i+1}-\lambda_i) = \left\lfloor\frac z{\lambda_{i+1}-\lambda_i}\right\rfloor
\end{align*}
のように表すことができる。
\end{hosoku}
%%%%%%%%%%%%%%%%%%%%%%%%%%%%%%%%%%%%%%%%%%%%%%%%%%%%%%%%%%
%%%%%%%%%%%%%%%%%%%%%%%%%%%%%%%%%%%%%%%%%%%%%%%%%%%%%%%%%%
%%%%%%%%%%%%%%%%%%%%%%%%%%%%%%%%%%%%%%%%%%%%%%%%%%%%%%%%%%
%%%%%%%%%%%%%%%%%%%%%%%%%%%%%%%%%%%%%%%%%%%%%%%%%%%%%%%%%%
%% Column %%%%%%%%%%%%%%%%%%%%%%%%%%%%%%%%%%%%%%%%%%%%%%%%
%%%%%%%%%%%%%%%%%%%%%%%%%%%%%%%%%%%%%%%%%%%%%%%%%%%%%%%%%%
\begin{Column}{商$\boldsymbol{\bDiv}$と余り$\boldsymbol{\bmod}$とガウス括弧$\boldsymbol{\lfloor\,\rfloor}$}
\renewcommand\theequation{c\thechapter.\arabic{equation}}
\setcounter{equation}{0}
\paragraph{$\boldsymbol\bDiv$と$\boldsymbol\bmod$}
割り算の余りを表す記号としては$\bmod$が広く使われる。
商を表す記号は一般的な数学の教科書等ではあまり用いられないが、プログラミング言語等では$\bDiv$を用いられることがある。
これに倣って、ここでは商には$\bDiv$, 余りには$\bmod$を用いている。

 一般に、実数$a$, $b$ ($b\neq0$)に対して$a = bq+r$ ($0 \leqq r < |b|$)を満たす整数$q$を商、$r$を余りと呼び、このとき$a \bDiv b = q$および$a \bmod b = r$のように表される。
なお、ここでは簡単のため、$q \geqq 0$として考えることにする。
\tcbline*
\paragraph{ガウス括弧}
$\lfloor x\rfloor$は、$x \in R$ に対して$x$を超えない最大の整数。
簡単にいうと、($x > 0$の場合は)小数点以下を切り捨てた整数部分を表す。
ガウス記号, 床関数(floor function)などとも呼ばれる。
\end{Column}
%%%%%%%%%%%%%%%%%%%%%%%%%%%%%%%%%%%%%%%%%%%%%%%%%%%%%%%%%%
%%%%%%%%%%%%%%%%%%%%%%%%%%%%%%%%%%%%%%%%%%%%%%%%%%%%%%%%%%
%%%%%%%%%%%%%%%%%%%%%%%%%%%%%%%%%%%%%%%%%%%%%%%%%%%%%%%%%%
\begin{tcolorbox}[title={内面溝加工に関する工具(T31, T50)の情報}, fonttitle=\gtfamily\bfseries]
・タッチセンサープローブの半径:5 \quad ・タッチセンサープローブの軸の半径:3.75\\
・切削用工具径:40 \quad ・切削用工具シャンク径:25
\end{tcolorbox}




\clearpage
%%%%%%%%%%%%%%%%%%%%%%%%%%%%%%%%%%%%%%%%%%%%%%%%%%%%%%%%%%
%% section 5.2 %%%%%%%%%%%%%%%%%%%%%%%%%%%%%%%%%%%%%%%%%%%
%%%%%%%%%%%%%%%%%%%%%%%%%%%%%%%%%%%%%%%%%%%%%%%%%%%%%%%%%%
\section{基本方針}
内面溝の加工における留意事項の1つに、モールドの内面(特にトップ端)と工具が接触してしまうアンダーカットというものがある
%% footnote %%%%%%%%%%%%%%%%%%%%%
\footnote{内面溝の測定・加工ではとりわけアンダーカットが生じやすい、という意味である。
その他の計測・加工についても当然アンダーカットは十分に生じうる。}。
%%%%%%%%%%%%%%%%%%%%%%%%%%%%%%%%%
特にモールドA面には工具へ向かう方向に湾曲があるため、アンダーカットが生じやすい。
そこで、アンダーカットを避けつつ加工ができるようにするため、モールドをいくらか(湾曲と反対側に)傾けて加工を行う。
その傾き角$\phi$ ($0 \leqq \phi < \nicefrac\pi2$)について、ここでは次の2点を基準に考えることにする。
\begin{tcolorbox}[title=A面の内面溝, fonttitle=\gtfamily\bfseries]
\begin{enumerate}
\item[a)]
A側内面のトップ端点
\item[b)]
A側内面の内面溝1列目(トップ端から$q$)の位置
\end{enumerate}
\end{tcolorbox}\noindent
この2点を通る直線と鉛直方向との角度を、傾き角$-\phi$とする
%% footnote %%%%%%%%%%%%%%%%%%%%%
\footnote{振分長の調整に用いたテーブルの傾き角$\theta$と混同しないように注意。}。
%%%%%%%%%%%%%%%%%%%%%%%%%%%%%%%%%
なお、トップ端のAC内径は$w'_{\mathrm A0}$で代用してもよいものとする。
このとき$\phi > 0$となる(C面側に傾く)場合は$\phi$だけ傾けて加工を行う。
一方、$\phi \leqq 0$となる(A面側に傾く)場合は、そもそもアンダーカットが生じないので、傾けずにそのまま加工を行うものとする。
%%%%%%%%%%%%%%%%%%%%%%%%%%%%%%%%%%%%%%%%%%%%%%%%%%%%%%%%%%
%% hosoku %%%%%%%%%%%%%%%%%%%%%%%%%%%%%%%%%%%%%%%%%%%%%%%%
%%%%%%%%%%%%%%%%%%%%%%%%%%%%%%%%%%%%%%%%%%%%%%%%%%%%%%%%%%
\begin{hosoku}
ここでは内面溝の工具として、Tスロットカッターを考えている。
しかし、当然ながら工具径は有限であるため、いくら適切に傾けたところで限界はある。
ここではその限界として、A側内面のトップ端の$X$座標と、それと最も$X$座標が近い内面溝との($X$方向の)距離を算出する。
そしてそれを工具径と比べることで、どこまでの範囲を加工するかを決定する。
加工できない部分に内面溝がある場合は、別の工具(アングルヘッド)を使用して加工を行う。
\end{hosoku}
%%%%%%%%%%%%%%%%%%%%%%%%%%%%%%%%%%%%%%%%%%%%%%%%%%%%%%%%%%
%%%%%%%%%%%%%%%%%%%%%%%%%%%%%%%%%%%%%%%%%%%%%%%%%%%%%%%%%%
%%%%%%%%%%%%%%%%%%%%%%%%%%%%%%%%%%%%%%%%%%%%%%%%%%%%%%%%%%
%%%%%%%%%%%%%%%%%%%%%%%%%%%%%%%%%%%%%%%%%%%%%%%%%%%%%%%%%%
%% Column %%%%%%%%%%%%%%%%%%%%%%%%%%%%%%%%%%%%%%%%%%%%%%%%
%%%%%%%%%%%%%%%%%%%%%%%%%%%%%%%%%%%%%%%%%%%%%%%%%%%%%%%%%%
\begin{Column}{曲率と傾き}
内面A側・C側の湾曲をそれぞれ$\mathcal R_\mathrm o$, $\mathcal R_\mathrm i$とすると、曲率はそれぞれ$\mathcal R_\mathrm o^{-1} < R_\mathrm c^{-1} < \mathcal R_\mathrm i^{-1}$である。
そのため、(トップ側の)A側の$\mathcal R_\mathrm o$を基準にするとより緩やかに、C側の$\mathcal R_\mathrm i$を基準にするとよりきつく傾くことになる。
また、トップ端から($Z$方向に)遠い点を基準にするとより緩やかに、近い点を基準にするとよりきつく傾くことになる。
\end{Column}
%%%%%%%%%%%%%%%%%%%%%%%%%%%%%%%%%%%%%%%%%%%%%%%%%%%%%%%%%%
%%%%%%%%%%%%%%%%%%%%%%%%%%%%%%%%%%%%%%%%%%%%%%%%%%%%%%%%%%
%%%%%%%%%%%%%%%%%%%%%%%%%%%%%%%%%%%%%%%%%%%%%%%%%%%%%%%%%%

以下ではこの傾き角$\phi$と、回転後の内面溝や内面の位置を定量的に与えることを試みる。




\clearpage
%%%%%%%%%%%%%%%%%%%%%%%%%%%%%%%%%%%%%%%%%%%%%%%%%%%%%%%%%%
%% section 5.3 %%%%%%%%%%%%%%%%%%%%%%%%%%%%%%%%%%%%%%%%%%%
%%%%%%%%%%%%%%%%%%%%%%%%%%%%%%%%%%%%%%%%%%%%%%%%%%%%%%%%%%
\section{内面溝の位置と傾き角(傾き前)}
\pageeqref{eq:tableTRc}より、テーブルを$-\theta$傾けて振分長の調整を行った場合、テーブル中心Pを原点としたモールド中心湾曲線のトップ端における$X$座標は、
\begin{align*}
  R_\mathrm c\cos\alpha_\mathrm c-\varDelta'\cos\theta = \sqrt{R_\mathrm c^2-f_\mathrm T^2}-\varDelta'\cos\theta
\end{align*}
で与えられる。
これはタッチセンサーによる測定の開始点として用いることができる。
一方で、それ以外の作業では、トップ端における内径の中心座標$g_t$を直接測定するので、それを用いることにする
%% footnote %%%%%%%%%%%%%%%%%%%%%
\footnote{これは中心湾曲線上にない点であるが、公差の範囲内であるものとして、ここではこれで代用する。}。
%%%%%%%%%%%%%%%%%%%%%%%%%%%%%%%%%
よって、テーブル中心Pを原点とした場合における、内面溝1列目中央の(だいたいの)位置
%% footnote %%%%%%%%%%%%%%%%%%%%%
\footnote{$w_{\mathrm Aq}$, $w_{\mathrm Bq}$はモールド湾曲の中心(0, 0)方向への長さであるため正確ではないことに注意。}\relax
%%%%%%%%%%%%%%%%%%%%%%%%%%%%%%%%%
は、次で与えられる。
\begin{align*}
\begin{array}{rl}
  \text{A面($+X$方向):}
  & \displaystyle
    \left(
      g_{tx}+\mathcal L_0+\frac{w'_{\mathrm Aq}}2~,~
      g_{ty}~,~
      f_t'-q
    \right),\\[12pt]
  \text{C面($-X$方向):}
  & \displaystyle
    \left(
      g_{tx}+\mathcal L_0-\frac{w'_{\mathrm Aq}}2~,~
      g_{ty}~,~
      f_t'-q
    \right),\\[12pt]
  \text{B面($+Y$方向):}
  & \displaystyle
    \left(
      g_{tx}+\mathcal L_0~,~
      g_{ty}+\frac{w'_{\mathrm Bq}}2~,~
      f_t'-q
    \right),\\[12pt]
  \text{D面($-Y$方向):}
  & \displaystyle
    \left(
      g_{tx}+\mathcal L_0~,~
      g_{ty}-\frac{w'_{\mathrm Bq}}2~,~
      f_t'-q
    \right).
\end{array}
\end{align*}
ここで、$i$列目の湾曲中心とトップ端の湾曲中心との$X$座標の差を、
%% label{eq:}
\begin{align}
  \label{eq:dimpleCenterDistance}
  \mathcal L_i
  \equiv \sqrt{R_\mathrm c^2-\left\{f_\mathrm T-q-(i-1)p_z\right\}^2}-\sqrt{R_\mathrm c^2-f_\mathrm T^2}
\end{align}
と表した。
なお、$i$列目の湾曲中心と$j$列目の湾曲中心との$X$座標の差を
\begin{align*}
  \mathcal L_{i,j}
  \equiv \mathcal L_i-\mathcal L_j
  = \sqrt{R_\mathrm c^2-\left(f_\mathrm T-q-(i-1)p_z\right)^2}
    -\sqrt{R_\mathrm c^2-\left\{f_\mathrm T-q-(j-1)p_z\right\}^2}
\end{align*}
と表すことにする。



%%%%%%%%%%%%%%%%%%%%%%%%%%%%%%%%%%%%%%%%%%%%%%%%%%%%%%%%%%
%% subsection 5.3.1 %%%%%%%%%%%%%%%%%%%%%%%%%%%%%%%%%%%%%%
%%%%%%%%%%%%%%%%%%%%%%%%%%%%%%%%%%%%%%%%%%%%%%%%%%%%%%%%%%
\subsection{内面溝の\texorpdfstring{$X$}{X}座標(傾き前)}
テーブル中心Pを原点としたとき、傾き前の$i$列目$j$番目の内面溝の$X$座標は、
%% label{eq:dPosXBefore}
\begin{align}
  \notag
  \text{A面:}\quad
  \mathcal D_{xi,\mathrm A}
  &= g_{tx}+\mathcal L_i+\frac{w'_{\mathrm Aq+(i-1)p_z}}2\\
  \label{eq:dPosXBefore}
  \text{C面:}\quad
  \mathcal D_{xi,\mathrm C}
  &= g_{tx}+\mathcal L_i-\frac{w'_{\mathrm Aq+(i-1)p_z}}2\\
  \notag
  \text{B, D面:}\quad
  \mathcal D_{xij,\mathrm B}
  &= g_{tx}+\mathcal L_i+\frac{d_i}2-(j-1)p_x
\end{align}
なお、A・C面については$j$に依らないことがわかる。
そのため、たとえば$\mathcal D_{xij,\mathrm A}$ではなく、$\mathcal D_{xi,\mathrm A}$のように表記している。



%%%%%%%%%%%%%%%%%%%%%%%%%%%%%%%%%%%%%%%%%%%%%%%%%%%%%%%%%%
%% subsection 5.3.2 %%%%%%%%%%%%%%%%%%%%%%%%%%%%%%%%%%%%%%
%%%%%%%%%%%%%%%%%%%%%%%%%%%%%%%%%%%%%%%%%%%%%%%%%%%%%%%%%%
\subsection{内面溝の\texorpdfstring{$Y$}{Y}座標(傾き前)}
テーブル中心Pを原点としたとき、傾き前の$i$列目$j$番目の内面溝の$Y$座標は、
%% label{eq:dPosYBefore}
\begin{alignat}{3}
  \notag
  \text{A, C面:}\quad
  && \mathcal D_{yij,\mathrm A} &= g_{ty}-\frac{d_i}2+(j-1)p_x\\
  \label{eq:dPosYBefore}
  \text{B面:}\quad
  && \mathcal D_{yi,\mathrm B} &= g_{ty}+\frac{w'_{\mathrm Bq+(i-1)p_z}}2\\
  \notag
  \text{D面:}\quad
  && \mathcal D_{yi,\mathrm D} &= g_{ty}-\frac{w'_{\mathrm Bq+(i-1)p_z}}2
\end{alignat}
B・D面については$j$に依らないことがわかる。



%%%%%%%%%%%%%%%%%%%%%%%%%%%%%%%%%%%%%%%%%%%%%%%%%%%%%%%%%%
%% subsection 5.3.3 %%%%%%%%%%%%%%%%%%%%%%%%%%%%%%%%%%%%%%
%%%%%%%%%%%%%%%%%%%%%%%%%%%%%%%%%%%%%%%%%%%%%%%%%%%%%%%%%%
\subsection{内面溝の\texorpdfstring{$Z$}{Z}座標(傾き前)}
テーブル中心Pを原点としたとき、傾き前の$i$列目$j$番目の内面溝の$Z$座標は、
%% label{eq:dPosZBefore}
\begin{align}
  \label{eq:dPosZBefore}
  \text{A, B, C, D面:}\quad
  \mathcal D_{zi} = f_t'-q-(i-1)p_z
\end{align}
$Z$座標についてはどの面も$j$に依らないことがわかる。



%%%%%%%%%%%%%%%%%%%%%%%%%%%%%%%%%%%%%%%%%%%%%%%%%%%%%%%%%%
%% subsection 5.3.4 %%%%%%%%%%%%%%%%%%%%%%%%%%%%%%%%%%%%%%
%%%%%%%%%%%%%%%%%%%%%%%%%%%%%%%%%%%%%%%%%%%%%%%%%%%%%%%%%%
\subsection{傾き角}
A側内面トップ端と、A側内面のトップ端から$q$の位置との$x$方向の差は、
\begin{align*}
  \sqrt{\left(R_\mathrm c+\frac{w'_{\mathrm Aq}}2\right)^{\!\!2}-(f_\mathrm T-q)^2}
  -\sqrt{\left(R_\mathrm c+\frac{w'_{\mathrm A0}}2\right)^{\!\!2}-f_\mathrm T^2}
\end{align*}
で与えられる。
このとき、これが負になる場合は傾ける必要はなく、正となる場合のみ傾ける。
したがってその傾き角$\phi$は、
%% label{eq:dKatamuki}
\begin{subequations}
\label{eq:dKatamuki}
\begin{alignat}{2}
  \text{正の場合:}&&\quad
  \tan\phi
  &= \frac{\displaystyle
           \sqrt{\left(R_\mathrm c+\frac{w'_{\mathrm Aq}}2\right)^{\!\!2}-(f_\mathrm T-q)^2}
           -\sqrt{\left(R_\mathrm c+\frac{w'_{\mathrm A0}}2\right)^{\!\!2}-f_\mathrm T^2}}q\\[8pt]
  \text{負の場合:}&&
  \phi
  &= 0
\end{alignat}
\end{subequations}
で与えられる。
%%%%%%%%%%%%%%%%%%%%%%%%%%%%%%%%%%%%%%%%%%%%%%%%%%%%%%%%%%
%% hosoku %%%%%%%%%%%%%%%%%%%%%%%%%%%%%%%%%%%%%%%%%%%%%%%%
%%%%%%%%%%%%%%%%%%%%%%%%%%%%%%%%%%%%%%%%%%%%%%%%%%%%%%%%%%
\begin{hosoku}
なお、これが負になるのは、
\begin{align*}
  & \left(R_\mathrm c+\frac{w'_{\mathrm Aq}}2\right)^{\!\!2}-(f_\mathrm T-q)^2
    < \left(R_\mathrm c+\frac{w'_{\mathrm A0}}2\right)^{\!\!2}-f_\mathrm T^2\\
  \longrightarrow~~
  & \frac{w_{\mathrm A0}-w_{\mathrm Aq}}2
    \left(2R_\mathrm c+\frac{w_{\mathrm A0}'+w_{\mathrm Aq}'}2\right)
    > q(2f_\mathrm T-q)
\end{align*}
である。
したがって、以下のような場合に生じる傾向があることがわかる。
\begin{enumerate}
\item 曲率が小さい(湾曲$R_\mathrm c$が大きい)
\item テーパがきつい($w_{\mathrm A0}-w_{\mathrm Aq}$が大きい)
\item 内径・めっき膜厚が大きい
\end{enumerate}
たとえば、曲率0 ($R = +\infty$)のモールド、つまり(外形が)まっすぐのモールドなどが、これに該当する。
\end{hosoku}
%%%%%%%%%%%%%%%%%%%%%%%%%%%%%%%%%%%%%%%%%%%%%%%%%%%%%%%%%%
%%%%%%%%%%%%%%%%%%%%%%%%%%%%%%%%%%%%%%%%%%%%%%%%%%%%%%%%%%
%%%%%%%%%%%%%%%%%%%%%%%%%%%%%%%%%%%%%%%%%%%%%%%%%%%%%%%%%%
%%%%%%%%%%%%%%%%%%%%%%%%%%%%%%%%%%%%%%%%%%%%%%%%%%%%%%%%%%
%% Column %%%%%%%%%%%%%%%%%%%%%%%%%%%%%%%%%%%%%%%%%%%%%%%%
%%%%%%%%%%%%%%%%%%%%%%%%%%%%%%%%%%%%%%%%%%%%%%%%%%%%%%%%%%
\begin{Column}{C側内面溝の傾き角}
C側内面溝については傾斜が外側に向いているため、傾けなくともアンダーカットの心配はない。
しかし、傾けたまま加工をすると形状が歪になってしまうため、内面溝の形状をよりよくするためには傾いていないほうが望ましい。
また一方で、面によって傾ける傾けないを分けると、プログラムが複雑になる(条件分岐が増える)要因にもなる。
そのためここでは、どの面の内面溝に対しても同じ角度$\phi$を用いて加工を行うことにする。
\tcbline*
なお、C面に対する内面溝の形状をできるだけよいものにするには、C面のテーパに基づいた角度を用いるほうが望ましい。
そのため、C側内面溝に対する傾き角$\phi_\mathrm C$についても(1つの例として)与えておく。
具体的には、以下の2点を基準として角度$\phi_\mathrm C$を取ることとする。
\begin{enumerate}
\item[a)]
C側内面の内面溝1列目(トップ端から$q$)の位置
\item[b)]
C側内面の内面溝$m$列目(トップ端から$q+(m-1)p_z$)の位置
\end{enumerate}
C側内面のトップ端から$q$の位置と、C側内面のトップ端から$q+(m-1)p_z$の位置との$x$方向の差は、
\begin{align*}
  \sqrt{\bigg(R_\mathrm c-\frac{w'_{\mathrm Aq+(m-1)p_z}}2\bigg)^{\!\!2}-\left\{f_\mathrm T-q-(m-1)p_z\right\}^2}
  -\sqrt{\left(R_\mathrm c-\frac{w'_{\mathrm Aq}}2\right)^{\!\!2}-(f_\mathrm T-q)^2}
\end{align*}
これより、C側内面溝に対する傾き角$\phi_\mathrm C$ ($\phi_\mathrm C > 0$)は、
\begin{align*}
  \tan\phi_\mathrm C
  = \frac{\sqrt{\left(R_\mathrm c-\frac{w'_{\mathrm Aq+(m-1)p_z}}2\right)^{\!2}
                -\left\{f_\mathrm T-q-(m-1)p_z\right\}^2}
          -\sqrt{\left(R_\mathrm c-\frac{w'_{\mathrm Aq}}2\right)^{\!2}-(f_\mathrm T-q)^2}}
         {(m-1)p_z}
\end{align*}
で与えられる。
なお、前述の通り$w_{\mathrm Aq+(m-1)p_z}$は$\lambda_j \leqq q+(m-1)p_z < \lambda_{j+1}$に対する$w_{\mathrm Aj}$, $w_{\mathrm Aj+1}$の加重算術平均
\begin{align*}
  w_{\mathrm Aq+(m-1)p_z}
  = \frac{\{q+(m-1)p_z-\lambda_j\}w_{\mathrm Aj+1}+\{\lambda_{j+1}-q-(m-1)p_z\}w_{\mathrm Aj}}
         {\lambda_{j+1}-\lambda_j}
\end{align*}
であり、内径として代用している。($w_{\mathrm Bq+(m-1)p_z}$についても同様)
\end{Column}
%%%%%%%%%%%%%%%%%%%%%%%%%%%%%%%%%%%%%%%%%%%%%%%%%%%%%%%%%%
%%%%%%%%%%%%%%%%%%%%%%%%%%%%%%%%%%%%%%%%%%%%%%%%%%%%%%%%%%
%%%%%%%%%%%%%%%%%%%%%%%%%%%%%%%%%%%%%%%%%%%%%%%%%%%%%%%%%%




%%%%%%%%%%%%%%%%%%%%%%%%%%%%%%%%%%%%%%%%%%%%%%%%%%%%%%%%%%
%% subsection 5.3.5 %%%%%%%%%%%%%%%%%%%%%%%%%%%%%%%%%%%%%%
%%%%%%%%%%%%%%%%%%%%%%%%%%%%%%%%%%%%%%%%%%%%%%%%%%%%%%%%%%
\subsection{B, D面の内面溝の位置(傾き前)}
B, D側内面溝において、その$X$座標がA側内面に最も近いものは、$m-1$列目または$m$列目の1番目の内面溝である。
これらの$X$座標は\pageeqref{eq:dPosXBefore}よりそれぞれ、
\begin{align*}
  m-1\text{列目:}&\quad
  g_{tx}+\mathcal L_{m-1}+\frac{d_{m-1}}2\\
  m\text{列目:}&\quad
  g_{tx}+\mathcal L_m+\frac{d_m}2
\end{align*}
%%%%%%%%%%%%%%%%%%%%%%%%%%%%%%%%%%%%%%%%%%%%%%%%%%%%%%%%%%
%% hosoku %%%%%%%%%%%%%%%%%%%%%%%%%%%%%%%%%%%%%%%%%%%%%%%%
%%%%%%%%%%%%%%%%%%%%%%%%%%%%%%%%%%%%%%%%%%%%%%%%%%%%%%%%%%
\begin{hosoku}
$d_{m-1} > d_m$のときは$m-1$列目, $d_m > d_{m-1}$のときは$m$列目をみればよい。
\end{hosoku}
%%%%%%%%%%%%%%%%%%%%%%%%%%%%%%%%%%%%%%%%%%%%%%%%%%%%%%%%%%
%%%%%%%%%%%%%%%%%%%%%%%%%%%%%%%%%%%%%%%%%%%%%%%%%%%%%%%%%%
%%%%%%%%%%%%%%%%%%%%%%%%%%%%%%%%%%%%%%%%%%%%%%%%%%%%%%%%%%
A側内面のトップ端からの($X$方向の)距離は、トップ端のAC側内径として$w'_{\mathrm A0}$を代用すると、それぞれ
\begin{align*}
  m-1\text{列目:}&\quad
  \frac{w'_{\mathrm A0}}2-\mathcal L_{m-1}-\frac{d_{m-1}}2\\
  m\text{列目:}&\quad
  \frac{w'_{\mathrm A0}}2-\mathcal L_m-\frac{d_m}2
\end{align*}
これらのいずれか小さいほうが工具径(半径)よりも小さければ、モールドを傾けて加工をする必要があると判断できる
%% footnote %%%%%%%%%%%%%%%%%%%%%
\footnote{もちろん、いくらか余裕代をとる必要がある。}。
%%%%%%%%%%%%%%%%%%%%%%%%%%%%%%%%%
%%%%%%%%%%%%%%%%%%%%%%%%%%%%%%%%%%%%%%%%%%%%%%%%%%%%%%%%%%
%% Column %%%%%%%%%%%%%%%%%%%%%%%%%%%%%%%%%%%%%%%%%%%%%%%%
%%%%%%%%%%%%%%%%%%%%%%%%%%%%%%%%%%%%%%%%%%%%%%%%%%%%%%%%%%
\begin{Column}{B, D側内面溝加工で考慮すべき点}
\paragraph{工具径とシャンク径}
アンダーカットが生じるのは主に(A側内面の)トップ端なので、実際には工具径(工具の切削する部分)ではなくシャンク径等(工具のトップ端に相当する箇所)でよい。
そのため工具径よりシャンク径のほうが小さい場合は、より広い範囲の(B, D面の)内面溝をモールドを傾けずに切削することが可能となる。
\tcbline*
\paragraph{端面の削り代}
内面溝の測定・加工は、モールドの端面を切削する前に行う。
そのため測定・加工の際は、モールドは端面の削り代の分だけ大きい(長い)ことに注意する必要がある。
削り代の分だけ湾曲も加味する必要があり、特にA側内面と工具とのアンダーカットに留意しなければならない。
\tcbline*
\paragraph{その他のずれ}
モールドの形状は当然ながら図面のものとは一致はしない。
特に湾曲や肉厚などの図面とのずれは、アンダーカットに大きく寄与するのでこれも注意する必要がある。
\end{Column}
%%%%%%%%%%%%%%%%%%%%%%%%%%%%%%%%%%%%%%%%%%%%%%%%%%%%%%%%%%
%%%%%%%%%%%%%%%%%%%%%%%%%%%%%%%%%%%%%%%%%%%%%%%%%%%%%%%%%%
%%%%%%%%%%%%%%%%%%%%%%%%%%%%%%%%%%%%%%%%%%%%%%%%%%%%%%%%%%





\clearpage
%%%%%%%%%%%%%%%%%%%%%%%%%%%%%%%%%%%%%%%%%%%%%%%%%%%%%%%%%%
%% section 5.4 %%%%%%%%%%%%%%%%%%%%%%%%%%%%%%%%%%%%%%%%%%%
%%%%%%%%%%%%%%%%%%%%%%%%%%%%%%%%%%%%%%%%%%%%%%%%%%%%%%%%%%
\section{傾き後の内面溝}
機内での回転はテーブル中心Pを原点として行われる。
また内面溝の加工はトップ端における内径中心を基準にして切削を行う。
傾ける前のトップ端内径中心$g_t$の座標は実測により(Pを中心とした$XYZ$直交座標でいうところの)[$g_{tx}$, $g_{ty}$, $f_t'$]で与えられる
%% footnote %%%%%%%%%%%%%%%%%%%%%
\footnote{ここではこれをテーブル中心Pを原点とした座標値として取り扱っている。
しかし、計測では機械座標系の値として$g_t$が与えられる。
たとえば\DMname の場合、$g_t$は通常(今の場合はテーブル中心Pより負側に湾曲中心があることが多いので)負の値として得られることに注意。
(ここでの計測では$XY$成分のみであり、$Z$については計測しないことにも注意。)}。
%%%%%%%%%%%%%%%%%%%%%%%%%%%%%%%%%
このとき、テーブルを角度$-\phi$だけ傾けた後のトップ端内面中心の座標$g'_t$は
%% footnote %%%%%%%%%%%%%%%%%%%%%
\footnote{これらをワーク座標原点としてもよいし、ワーク座標原点$g_t$はそのままで各面ごとに傾けてもよい。
ここでは後者の方法で加工を行うものとする。}、
%%%%%%%%%%%%%%%%%%%%%%%%%%%%%%%%%
%% label{eq:afterPhiTCenterFromO}
\begin{align}
  \label{eq:afterPhiTCenterFromO}
  \left[
  \begin{array}{c}
    g_{tx}'\\
    g_{ty}'\\
    g_{tz}'
  \end{array}
  \right]
  =\left[
   \begin{array}{c}
     g_{tx}\cos\phi+f_t'\sin\phi\\
     g_{ty}\\
     -g_{tx}\sin\phi+f_t'\cos\phi
   \end{array}
   \right].
   \end{align}
同様に、$i$列目における(傾ける前の)湾曲中心の位置は、[$g_{tx}+\mathcal L_i$, $g_{ty}$, $f_t'-q-(i-1)p_z$]で与えられる
%% footnote %%%%%%%%%%%%%%%%%%%%%
\footnote{ここではトップ端における湾曲中心を、トップ端における内面中心と同一視している。}
%%%%%%%%%%%%%%%%%%%%%%%%%%%%%%%%%
ので、テーブルを角度$-\phi$だけ傾けた後の$i$列目における湾曲中心の位置は、
\begin{align*}
  \left[
  \begin{array}{c}
    (g_{tx}+\mathcal L_i)\cos\phi+\{f_t'-q-(i-1)p_z\}\sin\phi\\
    g_{ty}\\
    -(g_{tx}+\mathcal L_i)\sin\phi+\{f_t'-q-(i-1)p_z\}\cos\phi
  \end{array}
  \right].
\end{align*}
したがって、傾けた後のトップ端の湾曲中心と$i$列目に対する湾曲中心との差分は、
%% label{eq:afterPhidimpleCenterDistance}
\begin{align}
  \label{eq:afterPhidimpleCenterDistance}
  \left[
  \begin{array}{c}
    \mathcal L_i\cos\phi-\{q+(i-1)p_z\}\sin\phi\\
    0\\
    -\mathcal L_i\sin\phi-\{q+(i-1)p_z\}\cos\phi
  \end{array}
  \right].
\end{align}
%%%%%%%%%%%%%%%%%%%%%%%%%%%%%%%%%%%%%%%%%%%%%%%%%%%%%%%%%%
%% hosoku %%%%%%%%%%%%%%%%%%%%%%%%%%%%%%%%%%%%%%%%%%%%%%%%
%%%%%%%%%%%%%%%%%%%%%%%%%%%%%%%%%%%%%%%%%%%%%%%%%%%%%%%%%%
\begin{hosoku}
傾けた後の$i$列目に対する湾曲中心と$j$列目に対する湾曲中心との差分は、
\begin{align*}
  \left[
  \begin{array}{c}
    \mathcal L_{j,i}\cos\phi-(j-i)p_z\sin\phi\\
    0\\
    -\mathcal L_{j,i}\sin\phi-(j-i)p_z\cos\phi
  \end{array}
  \right].
\end{align*}
特に、$j = i+1$の場合は、
\begin{align*}
  \left[
  \begin{array}{c}
    \mathcal L_{i+1,i}\cos\phi-p_z\sin\phi\\
    0\\
    -\mathcal L_{i+1,i}\sin\phi-p_z\cos\phi
  \end{array}
  \right].
\end{align*}
\end{hosoku}
%%%%%%%%%%%%%%%%%%%%%%%%%%%%%%%%%%%%%%%%%%%%%%%%%%%%%%%%%%
%%%%%%%%%%%%%%%%%%%%%%%%%%%%%%%%%%%%%%%%%%%%%%%%%%%%%%%%%%
%%%%%%%%%%%%%%%%%%%%%%%%%%%%%%%%%%%%%%%%%%%%%%%%%%%%%%%%%%
%%%%%%%%%%%%%%%%%%%%%%%%%%%%%%%%%%%%%%%%%%%%%%%%%%%%%%%%%%
%% Column %%%%%%%%%%%%%%%%%%%%%%%%%%%%%%%%%%%%%%%%%%%%%%%%
%%%%%%%%%%%%%%%%%%%%%%%%%%%%%%%%%%%%%%%%%%%%%%%%%%%%%%%%%%
\begin{Column}{プローブ径の考慮:$XY$と$Z$方向の非対称性}
マシニング内の計測ではタッチセンサーを用いる。
そのため、プローブ径の大きさに対して考慮・補正しなければならない。
プローブの位置の基準については、以下のようにとるのが通常である。
\begin{enumerate}
\item $X$方向:基準はプローブの($X$方向の)中心
\item $Y$方向:基準はプローブの($Y$方向の)中心
\item $Z$方向:基準はプローブの($Z$方向の)先端
\end{enumerate}
したがって、$XY$方向と$Z$方向とでは基準点が異なり非対称となっている。
今の場合、基準が非対称な$X$と$Z$が混合する移動(回転)であるが、あくまでもプローブの先端(上記の基準点)が回転後の位置にある、ということである。
そのため補正については(傾きに関係なく)$Z$方向に対してのみ径の半分だけ補正すればよい。
\end{Column}
%%%%%%%%%%%%%%%%%%%%%%%%%%%%%%%%%%%%%%%%%%%%%%%%%%%%%%%%%%
%%%%%%%%%%%%%%%%%%%%%%%%%%%%%%%%%%%%%%%%%%%%%%%%%%%%%%%%%%
%%%%%%%%%%%%%%%%%%%%%%%%%%%%%%%%%%%%%%%%%%%%%%%%%%%%%%%%%%




%%%%%%%%%%%%%%%%%%%%%%%%%%%%%%%%%%%%%%%%%%%%%%%%%%%%%%%%%%
%% subsection 5.4.1 %%%%%%%%%%%%%%%%%%%%%%%%%%%%%%%%%%%%%%
%%%%%%%%%%%%%%%%%%%%%%%%%%%%%%%%%%%%%%%%%%%%%%%%%%%%%%%%%%
\subsection{傾き後の内面溝(A, C面側)}
傾ける角度$\phi$は\pageeqref{eq:dKatamuki}で与えられる。
このとき、傾けた後のAおよびC面側に対する$i$列目$j$番目の内面溝の位置は、\pageeqref{eq:dPosXBefore}, \eqref{eq:dPosYBefore}, \pageeqref{eq:dPosZBefore}より、
\begin{alignat*}{3}
  \text{A面:}&~~&
  \left[
  \begin{array}{c}
    \mathcal D_{xij,\mathrm A}'\\
    \mathcal D_{yij,\mathrm A}'\\
    \mathcal D_{zij,\mathrm A}'
  \end{array}
  \right]
 &= \left[
    \begin{array}{c}
      \mathcal D_{xi,\mathrm A}\cos\phi+\mathcal D_{zi}\sin\phi\\
      \mathcal D_{yij,\mathrm A}\\
      -\mathcal D_{xi,\mathrm A}\sin\phi+\mathcal D_{zi}\cos\phi
    \end{array}
    \right],\\[2pt]
  \text{C面:}&~~&
  \left[
  \begin{array}{c}
    \mathcal D_{xij,\mathrm C}'\\
    \mathcal D_{yij,\mathrm C}'\\
    \mathcal D_{zij,\mathrm C}'
  \end{array}
  \right]
 &= \left[
    \begin{array}{c}
      \mathcal D_{xi,\mathrm C}\cos\phi+\mathcal D_{zi}\sin\phi\\
      \mathcal D_{yij,\mathrm A}\\
      -\mathcal D_{xi,\mathrm C}\sin\phi+\mathcal D_{zi}\cos\phi
    \end{array}
    \right].
\end{alignat*}
特に、各列の中央(各列の湾曲中心)$[g_{tx}+\mathcal L_i, g_{ty}, f_t'-q-(i-1)p_z]$を原点としてみた場合の位置は、
\begin{align*}
  \left[
  \begin{array}{c}
    \displaystyle \pm\frac{w_{Aq+(i-1)p_z}'}2\cos\phi\\[6pt]
    \displaystyle -\frac{d_i}2+(j-1)p_x\\[6pt]
    \displaystyle \mp\frac{w_{Aq+(i-1)p_z}'}2\sin\phi
  \end{array}
  \right]\qquad
  %%%%%%%%
  \left(
  \text{複号}
  \left\{
  \begin{array}{rl}
    \!\text{上}\!\!\!& \text{: A面}\\
    \!\text{下}\!\!\!& \text{: C面}\\
  \end{array}
  \right.
  \right).
\end{align*}





\paragraph{$j$方向の差分}\noindent
$Y$方向の隣同士の差分、すなわち$i$を固定したときの$j$番目と$j+1$番目の位置の差分は、
\begin{align*}
  \left[
  \begin{array}{c}
    0\\
    \mathcal D_{yi(j+1),\mathrm A}-\mathcal D_{yij,\mathrm A}\\
    0
  \end{array}
  \right]
  = \left[
    \begin{array}{c}
      0\\
      p_x\\
      0
    \end{array}
    \right]\ .
\end{align*}


\paragraph{$i$方向の差分}\noindent
$Z$方向の隣同士の差分、すなわち$j$を固定したときの$i$番目と$i+1$番目の位置の差分については、
\begin{align*}
 &\left[
  \begin{array}{c}
    (\mathcal D_{x(i+1),\mathrm A}-\mathcal D_{xi,\mathrm A})\cos\phi
    +(\mathcal D_{z(i+1)}-\mathcal D_{zi})\sin\phi\\
    (\mathcal D_{y(i+1)j,\mathrm A}-\mathcal D_{yij,\mathrm A})\\
    (\mathcal D_{xi,\mathrm A}-\mathcal D_{x(i+1),\mathrm A})\sin\phi
    +(\mathcal D_{z(i+1)}-\mathcal D_{zi})\cos\phi
  \end{array}
  \right]\\
 &= \left[
    \begin{array}{c}
      \displaystyle
      \left(\mathcal L_{i+1, i}+\frac{w'_{\mathrm Aq+ip_z}-w'_{\mathrm Aq+(i-1)p_z}}2\right)\!\cos\phi
      -p_z\sin\phi\\[6pt]
      \displaystyle-\frac{d_{i+1}-d_i}2\\[6pt]
      \displaystyle
      -\left(\mathcal L_{i+1, i}+\frac{w'_{\mathrm Aq+ip_z}-w'_{\mathrm Aq+(i-1)p_z}}2\right)\!\sin\phi
      -p_z\cos\phi
    \end{array}
    \right]\ .
\end{align*}
C面に対しては、これの各々の内径$w_\mathrm A'$の符号を入れ換えたものとなる。
%%%%%%%%%%%%%%%%%%%%%%%%%%%%%%%%%%%%%%%%%%%%%%%%%%%%%%%%%%
%% hosoku %%%%%%%%%%%%%%%%%%%%%%%%%%%%%%%%%%%%%%%%%%%%%%%%
%%%%%%%%%%%%%%%%%%%%%%%%%%%%%%%%%%%%%%%%%%%%%%%%%%%%%%%%%%
\begin{hosoku}
$X$成分の差分の大きさが($\mathcal L_{i+1, i}$からみて)A面(の$\cos\phi$成分)のそれと同じであることがわかる。
これは(水平方向の)内径を$w_{\mathrm A\lambda}$等で代用したからであり、実際の長さは異なる(振分中心を除いて対称ではなく、C側のほうが長い)ことに注意。
\end{hosoku}
%%%%%%%%%%%%%%%%%%%%%%%%%%%%%%%%%%%%%%%%%%%%%%%%%%%%%%%%%%
%%%%%%%%%%%%%%%%%%%%%%%%%%%%%%%%%%%%%%%%%%%%%%%%%%%%%%%%%%
%%%%%%%%%%%%%%%%%%%%%%%%%%%%%%%%%%%%%%%%%%%%%%%%%%%%%%%%%%
%%%%%%%%%%%%%%%%%%%%%%%%%%%%%%%%%%%%%%%%%%%%%%%%%%%%%%%%%%
%% hosoku %%%%%%%%%%%%%%%%%%%%%%%%%%%%%%%%%%%%%%%%%%%%%%%%
%%%%%%%%%%%%%%%%%%%%%%%%%%%%%%%%%%%%%%%%%%%%%%%%%%%%%%%%%%
\begin{hosoku}[label=hosoku:generallyDimpleN]
\pageautoref{fn:generallyDimpleN}でも述べたように、たいていの場合は$|d_{i+1}-d_i|=p_x$であり、また$d_{i+2} = d_i$である。
\end{hosoku}
%%%%%%%%%%%%%%%%%%%%%%%%%%%%%%%%%%%%%%%%%%%%%%%%%%%%%%%%%%
%%%%%%%%%%%%%%%%%%%%%%%%%%%%%%%%%%%%%%%%%%%%%%%%%%%%%%%%%%
%%%%%%%%%%%%%%%%%%%%%%%%%%%%%%%%%%%%%%%%%%%%%%%%%%%%%%%%%%




%%%%%%%%%%%%%%%%%%%%%%%%%%%%%%%%%%%%%%%%%%%%%%%%%%%%%%%%%%
%% subsection 5.4.2 %%%%%%%%%%%%%%%%%%%%%%%%%%%%%%%%%%%%%%
%%%%%%%%%%%%%%%%%%%%%%%%%%%%%%%%%%%%%%%%%%%%%%%%%%%%%%%%%%
\subsection{傾き後の内面溝(B, D面側)}
傾けた後のBおよびD面側に対する$i$列目$j$番目の内面溝の位置は、A面側のときと同様に、
\begin{alignat*}{3}
  \text{B面:}&~~&
  \left[
    \begin{array}{c}
      \mathcal D_{xij,\mathrm B}'\\
      \mathcal D_{yij,\mathrm B}'\\
      \mathcal D_{zij,\mathrm B}'
    \end{array}
  \right]
 &= \left[
    \begin{array}{c}
      \mathcal D_{xij,\mathrm B}\cos\phi+\mathcal D_{zi}\sin\phi\\
      \mathcal D_{yi,\mathrm B}\\
      -\mathcal D_{xij,\mathrm B}\sin\phi+\mathcal D_{zi}\cos\phi
    \end{array}
    \right],\\[2pt]
  \text{D面:}&~~&
  \left[
    \begin{array}{c}
      \mathcal D_{xij,\mathrm D}'\\
      \mathcal D_{yij,\mathrm D}'\\
      \mathcal D_{zij,\mathrm D}'
    \end{array}
  \right]
 &= \left[
    \begin{array}{c}
      \mathcal D_{xij,\mathrm B}\cos\phi+\mathcal D_{zi}\sin\phi\\
      \mathcal D_{yi,\mathrm D}\\
      -\mathcal D_{xij,\mathrm B}\sin\phi+\mathcal D_{zi}\cos\phi
    \end{array}
    \right].
\end{alignat*}
特に、各列の中央(各列の湾曲中心)$[g_{tx}+\mathcal L_i, g_{ty}, f_t'-q-(i-1)p_z]$を原点としてみた場合の位置は、
\begin{align*}
  \left[
  \begin{array}{c}
    \displaystyle \left\{\frac{d_i}2-(j-1)p_z\right\}\cos\phi\\
    \displaystyle \pm\frac{w_{Bq+(i-1)p_z}'}2\\
    \displaystyle -\left\{\frac{d_i}2-(j-1)p_z\right\}\sin\phi
  \end{array}
  \right]\qquad
  %%%%%%%%
  \left(
  \text{複号}
  \left\{
  \begin{array}{rl}
    \!+\!\!\!& \text{: B面}\\
    \!-\!\!\!& \text{: D面}\\
  \end{array}
  \right.
  \right).
\end{align*}



\paragraph{$j$方向の差分}\noindent
$Y$方向の隣同士の差分、すなわち$i$を固定したときの$j$番目と$j+1$番目の位置の差分は、
\begin{align*}
  \left[
  \begin{array}{c}
    \left(\mathcal D_{xi(j+1),\mathrm B}-\mathcal D_{xij,\mathrm B}\right)\cos\phi\\
    0\\
    -\left(\mathcal D_{xi(j+1),\mathrm B}-\mathcal D_{xij,\mathrm B}\right)\sin\phi
  \end{array}
  \right]
  = \left[
    \begin{array}{c}
      -p_x\cos\phi\\[6pt]
      0\\
      p_x\sin\phi
    \end{array}
    \right]\ .
\end{align*}


\paragraph{$i$方向の差分}\noindent
B面に対する$Z$方向の隣同士の差分、すなわち$j$を固定したときの$i$番目と$i+1$番目の位置の差分については、
\begin{align*}
 &\left[
  \begin{array}{c}
    \left(\mathcal D_{x(i+1)j,\mathrm B}-\mathcal D_{xij,\mathrm B}\right)\cos\phi
    +\left(\mathcal D_{z(i+1)}-\mathcal D_{zi}\right)\sin\phi\\[3pt]
    \mathcal D_{yi+1,\mathrm B}-\mathcal D_{yi,\mathrm B}\\[3pt]
    -\left(\mathcal D_{x(i+1)j,\mathrm B}-\mathcal D_{xij,\mathrm B}\right)\sin\phi
    +\left(\mathcal D_{z(i+1)}-\mathcal D_{zi}\right)\cos\phi
  \end{array}
  \right]\\
 &= \left[
    \begin{array}{c}
      \displaystyle\left(\mathcal L_{i+1, i}+\frac{d_{i+1}-d_i}2\right)\!\cos\phi-p_z\sin\phi\\[10pt]
      \displaystyle\frac{w'_{\mathrm Bq+ip_z}-w'_{\mathrm Bq+(i-1)p_z}}2\\[8pt]
      \displaystyle-\left(\mathcal L_{i+1, i}+\frac{d_{i+1}-d_i}2\right)\!\cos\phi-p_z\cos\phi
    \end{array}
    \right]\ .
\end{align*}
D面に対しては、これの各々の内径$w_\mathrm B'$の符号(この場合$Y$成分の符号)を入れ換えたものとなる。






%%%%%%%%%%%%%%%%%%%%%%%%%%%%%%%%%%%%%%%%%%%%%%%%%%%%%%%%%%
%%           %%%%%%%%%%%%%%%%%%%%%%%%%%%%%%%%%%%%%%%%%%%%%
%% chapter 6 %%%%%%%%%%%%%%%%%%%%%%%%%%%%%%%%%%%%%%%%%%%%%
%%           %%%%%%%%%%%%%%%%%%%%%%%%%%%%%%%%%%%%%%%%%%%%%
%%%%%%%%%%%%%%%%%%%%%%%%%%%%%%%%%%%%%%%%%%%%%%%%%%%%%%%%%%
\chapter[その他の\texorpdfstring{$B$}{B}軸回転に伴う幾何]
         {その他の$\textbf B$軸回転に伴う幾何}
マシニング機内では、$XYZ$直交座標軸の他に、それぞれの軸とした$ABC$回転軸がある。
たとえば、\MMname ではテーブルが$Y$軸まわりに回転する$B$回転軸座標がある。
さらに\DMname では、$B$回転軸に加えて、工具が$Z$軸まわりに回転する$C$回転軸座標も存在する。
これまでに扱った再振分けのためのテーブル回転や、内面溝のためのテーブル回転は、どちらも$B$軸の回転($Y$軸まわりの回転)である。
ここではさらに、その他の$B$回転軸を伴う事象を取り扱うことにする。



%%%%%%%%%%%%%%%%%%%%%%%%%%%%%%%%%%%%%%%%%%%%%%%%%%%%%%%%%%
%% section 6.1 %%%%%%%%%%%%%%%%%%%%%%%%%%%%%%%%%%%%%%%%%%%
%%%%%%%%%%%%%%%%%%%%%%%%%%%%%%%%%%%%%%%%%%%%%%%%%%%%%%%%%%
\section{ジグと回転中心との補正}
ここでは回転の中心は原点にあるものとし、テーブルに乗っているジグの中心が原点から($\delta x$, $\delta z$)だけずれているものとする。
このとき、ある任意の点($x$, $z$)を$\theta$だけ回転すると、
\begin{align*}
  \left[
    \begin{array}{c}
      x'\\
      z'
    \end{array}
  \right]
  = \left[
    \begin{array}{cc}
      \cos\theta & -\sin\theta\\
      \sin\theta & \cos\theta
    \end{array}
  \right]\!\!
  \left[
    \begin{array}{c}
      x+\delta x\\
      z+\delta z
    \end{array}
  \right]
  = \left[
    \begin{array}{c}
      (x+\delta x)\cos\theta-(z+\delta z)\sin\theta\\
      (x+\delta x)\sin\theta+(z+\delta z)\cos\theta
    \end{array}
  \right].
\end{align*}
特に、$\theta = \nicefrac\pi2$の場合は、
\begin{align*}
  \left[
    \begin{array}{c}
      x'\\
      z'
    \end{array}
  \right]
  = \left[
    \begin{array}{c}
      -z-\delta z\\
      x+\delta x
    \end{array}
  \right].
\end{align*}
よって、ずれによる差分は、
\begin{align*}
  \left[
    \begin{array}{c}
      \delta x\cos\theta-\delta z\sin\theta\\
      \delta x\sin\theta+\delta z\cos\theta
    \end{array}
  \right]
\end{align*}
となる。




\clearpage
%%%%%%%%%%%%%%%%%%%%%%%%%%%%%%%%%%%%%%%%%%%%%%%%%%%%%%%%%%
%% section 6.2 %%%%%%%%%%%%%%%%%%%%%%%%%%%%%%%%%%%%%%%%%%%
%%%%%%%%%%%%%%%%%%%%%%%%%%%%%%%%%%%%%%%%%%%%%%%%%%%%%%%%%%
\section{通り芯}
トップ・ボトムの両方に外削がある場合を考える。
通常、それぞれの外削の中心は個別に決められはせず、片方の中心の位置を基準として、もう片方の中心が定められる。
これらの中心の位置の差(通り芯
%% footnote %%%%%%%%%%%%%%%%%%%%%
\footnote{通常、通り芯(centerline)というのはその名の通り中心線を表すことが多い。
しかし、ここではトップ側外削中心とボトム側外削中心との位置の差を表す用語として「通り芯」と呼んでいる。})
%%%%%%%%%%%%%%%%%%%%%%%%%%%%%%%%%
$T_x$, $T_y$ ($T_x \geqq 0$)を機内で測定する際は、C面が工具側に向くようにテーブルを$\pm90^\circ$回転($B$軸回転)し、タッチセンサーを用いてそれぞれの外削部の$Z$座標および$Y$座標を見ることで測定する。

ここでは、この通り芯の測定に必要な位置等について定量的に求める。
なお、テーブルの中心Pを原点として考えることにする。
またモールドのC面が工具側に向くように$B$軸を(G91にて)$\pm90^\circ$回転した状態であるとする
%% footnote %%%%%%%%%%%%%%%%%%%%%
\footnote{G90(絶対座標)の場合、テーブルを傾けて振分長を調整した場合はその回転角$-\theta$を忘れないよう注意。}。
%%%%%%%%%%%%%%%%%%%%%%%%%%%%%%%%%



%%%%%%%%%%%%%%%%%%%%%%%%%%%%%%%%%%%%%%%%%%%%%%%%%%%%%%%%%%
%% subsection 6.2.1 %%%%%%%%%%%%%%%%%%%%%%%%%%%%%%%%%%%%%%
%%%%%%%%%%%%%%%%%%%%%%%%%%%%%%%%%%%%%%%%%%%%%%%%%%%%%%%%%%
\subsection{ボトムの外削が基準の場合}
通常、トップの外削中心は、ボトムの外削中心よりA面側($-Z$側)にある。
このとき、溝位置$\kappa_p$およびボトム側の外削の長さ$h_\mathrm B$ ($h_\mathrm B > 0$)を用いると、ボトム側($-X$側)およびトップ側($+X$側)の外削C面の中心
%% footnote %%%%%%%%%%%%%%%%%%%%%
\footnote{トップ側には溝があるので、トップ側の外削の長さは溝位置$\kappa_p$とみなしている。}
%%%%%%%%%%%%%%%%%%%%%%%%%%%%%%%%%
は、それぞれ
%% footnote %%%%%%%%%%%%%%%%%%%%%
\footnote{通常、$Y$方向の通り芯は$T_y=0$である。}
%%%%%%%%%%%%%%%%%%%%%%%%%%%%%%%%%
\begin{align*}
  \text{ボトム側:}\quad
  \left[
    \begin{array}{c}
      \displaystyle -f_\mathrm B'+\frac{h_\mathrm B}2\\[5pt]
      \mathcal G_{\mathrm By}\\[3pt]
      \displaystyle \mathcal G_{\mathrm Bx}+\frac{\mathfrak W_\mathrm B}2
    \end{array}
    \right]~, \qquad
  \text{トップ側:}\quad
  \left[
    \begin{array}{c}
      \displaystyle f_\mathrm T'-\frac{\kappa_p}2\\[5pt]
      \mathcal G_{\mathrm By}-T_y\\[3pt]
      \displaystyle \mathcal G_{\mathrm Bx}-T_x+\frac{\mathfrak W_\mathrm T}2
    \end{array}
  \right].
\end{align*}



%%%%%%%%%%%%%%%%%%%%%%%%%%%%%%%%%%%%%%%%%%%%%%%%%%%%%%%%%%
%% subsection 6.2.2 %%%%%%%%%%%%%%%%%%%%%%%%%%%%%%%%%%%%%%
%%%%%%%%%%%%%%%%%%%%%%%%%%%%%%%%%%%%%%%%%%%%%%%%%%%%%%%%%%
\subsection{トップの外削が基準の場合}
通常、ボトムの外削中心は、トップの外削中心よりC面側($+Z$側)にある。
このとき、トップ側($+X$側)およびボトム側($-X$側)の外削C面の中心は、それぞれ
\begin{align*}
  \text{トップ側:}\quad
  \left[
    \begin{array}{c}
      \displaystyle f_\mathrm T'-\frac{\kappa_p}2\\[5pt]
      \mathcal G_{\mathrm Ty}\\[3pt]
      \displaystyle \mathcal G_{\mathrm Tx}+\frac{\mathfrak W_\mathrm B}2
    \end{array}
    \right]~, \qquad
  \text{ボトム側:}\quad
  \left[
    \begin{array}{c}
      \displaystyle -f_\mathrm B'+\frac{h_\mathrm B}2\\[5pt]
      \mathcal G_{\mathrm Ty}+T_y\\[3pt]
      \displaystyle \mathcal G_{\mathrm Tx}+T_x+\frac{\mathfrak W_\mathrm B}2
    \end{array}
  \right].
\end{align*}




\clearpage
%%%%%%%%%%%%%%%%%%%%%%%%%%%%%%%%%%%%%%%%%%%%%%%%%%%%%%%%%%
%% section 6.3 %%%%%%%%%%%%%%%%%%%%%%%%%%%%%%%%%%%%%%%%%%%
%%%%%%%%%%%%%%%%%%%%%%%%%%%%%%%%%%%%%%%%%%%%%%%%%%%%%%%%%%
\section{角度をつけて外削する場合の位置}
テーブルの中心Pを原点として考える。
また、ボトム端面側が工具側に向いているとする。
端面のA面側・C面側・中心の$X$座標をそれぞれ$x_A$, $x_C$, $x_m$とする。
このとき端面のそれぞれの位置は
%% footnote %%%%%%%%%%%%%%%%%%%%%
\footnote{ここでは$Z$の正方向を実軸、$X$の正方向を虚軸として考えている。}、
%%%%%%%%%%%%%%%%%%%%%%%%%%%%%%%%%
\begin{subequations}
\begin{align*}
  \text{中点:}&\quad \sqrt{z_b^2+x_m^2}e^{i\theta_m}, \quad \tan\theta_m = \frac{x_m}{z_b}\\
  \text{C面側端点:}&\quad \sqrt{z_b^2+x_C^2}e^{i\theta_C}, \quad \tan\theta_C = \frac{x_C}{z_b}\\
  \text{A面側端点:}&\quad \sqrt{z_b^2+x_A^2}e^{i\theta_A}, \quad \tan\theta_A = \frac{x_A}{z_b}.
\end{align*}
\end{subequations}
外削部分の高さを$h_\mathrm B$とすると、外柵部分の傾きはボトム端から距離$\nicefrac{h_\mathrm B}2$の断面に平行になる形にとるので、必要な回転角$\theta_b$は、
\begin{align*}
  \sin\theta_b = \frac{z_b-\nicefrac{h_\mathrm B}2}{R_\mathrm c}
\end{align*}
を満たす。
回転後のボトム端面の中心の位置は、
\begin{align*}
  \sqrt{z_b^2+x_m^2}e^{i(\theta_m-\theta_b)}
\end{align*}
となるので、これの虚部が$X$座標、実部が$Z$座標となる
%% footnote %%%%%%%%%%%%%%%%%%%%%
\footnote{ここでは複素数平面を考えているが、通常の直行座標系で単純に回転行列をかけているのと同義である。
\begin{align*}
  \left[
    \begin{array}{c}
      x'_m\\
      z'_b
    \end{array}
  \right]
  = \left[
    \begin{array}{cc}
      \cos\theta_b & -\sin\theta_b\\
      \sin\theta_b & \cos\theta_b
    \end{array}
  \right]\!\!
  \left[
    \begin{array}{c}
      x_m\\
      z_b
    \end{array}
  \right]
  = \left[
    \begin{array}{c}
      x_m\cos\theta_b-z_b\sin\theta_b\\
      x_m\sin\theta_b+z_b\cos\theta_b
    \end{array}
  \right].
\end{align*}%
}。
%%%%%%%%%%%%%%%%%%%%%%%%%%%%%%%%%
\begin{align*}
  \sqrt{z_b^2+x_m^2}\sin(\theta_m-\theta_b)
  &= \sqrt{z_b^2+x_m^2}(\sin\theta_m\cos\theta_b-\cos\theta_m\sin\theta_b)\\
  &= x_m\sqrt{1-\left(\frac{z_b-\nicefrac{h_\mathrm B}2}{R_\mathrm c}\right)^{\!2}}
     -z_b\cdot\frac{z_b-\nicefrac{h_\mathrm B}2}{R_\mathrm c}~,\\
  \sqrt{z_b^2+x_m^2}\cos(\theta_m-\theta_b)
  &= \sqrt{z_b^2+x_m^2}(\cos\theta_m\cos\theta_b+\sin\theta_m\sin\theta_b)\\
  &= z_b\sqrt{1-\left(\frac{z_b-\nicefrac{h_\mathrm B}2}{R_\mathrm c}\right)^{\!2}}
     +x_m\cdot\frac{z_b-\nicefrac{h_\mathrm B}2}{R_\mathrm c}~.
\end{align*}
端面のA面側・C面側の位置についても同様である。
まとめると、
\begin{subequations}
\begin{align*}
  \text{中点:}&\quad
    \left[
      \begin{array}{c}
        x'_m\\
        z'_b
      \end{array}
    \right]
    = \left[
      \begin{array}{c}
        \displaystyle
        x_m\sqrt{1-\left(\frac{z_b-\nicefrac{h_\mathrm B}2}{R_\mathrm c}\right)^{\!2}}
        -z_b\cdot\frac{z_b-\nicefrac{h_\mathrm B}2}{R_\mathrm c}\\[15pt]
        \displaystyle
        z_b\sqrt{1-\left(\frac{z_b-\nicefrac{h_\mathrm B}2}{R_\mathrm c}\right)^{\!2}}
        +x_m\cdot\frac{z_b-\nicefrac{h_\mathrm B}2}{R_\mathrm c}
      \end{array}
    \right],\\
  \text{C面側端点:}&\quad
    \left[
      \begin{array}{c}
        x'_C\\
        z'_b
      \end{array}
    \right]
    = \left[
      \begin{array}{c}
        \displaystyle
        x_C\sqrt{1-\left(\frac{z_b-\nicefrac{h_\mathrm B}2}{R_\mathrm c}\right)^{\!2}}
        -z_b\cdot\frac{z_b-\nicefrac{h_\mathrm B}2}{R_\mathrm c}\\[15pt]
        \displaystyle
        z_b\sqrt{1-\left(\frac{z_b-\nicefrac{h_\mathrm B}2}{R_\mathrm c}\right)^{\!2}}
        +x_C\cdot\frac{z_b-\nicefrac{h_\mathrm B}2}{R_\mathrm c}
      \end{array}
    \right],\\
  \text{A面側端点:}&\quad
    \left[
      \begin{array}{c}
        x'_A\\
        z'_b
      \end{array}
    \right]
    = \left[
      \begin{array}{c}
        \displaystyle
        x_A\sqrt{1-\left(\frac{z_b-\nicefrac{h_\mathrm B}2}{R_\mathrm c}\right)^{\!2}}
        -z_b\cdot\frac{z_b-\nicefrac{h_\mathrm B}2}{R_\mathrm c}\\[15pt]
        \displaystyle
        z_b\sqrt{1-\left(\frac{z_b-\nicefrac{h_\mathrm B}2}{R_\mathrm c}\right)^{\!2}}
        +x_A\cdot\frac{z_b-\nicefrac{h_\mathrm B}2}{R_\mathrm c}
      \end{array}
    \right].
\end{align*}
\end{subequations}




%%%%%%%%%%%%%%%%%%%%%%%%%%%%%%%%%%%%%%%%%%%%%%%%%%%%%%%%%%
%%            %%%%%%%%%%%%%%%%%%%%%%%%%%%%%%%%%%%%%%%%%%%%
%% Chapter 7  %%%%%%%%%%%%%%%%%%%%%%%%%%%%%%%%%%%%%%%%%%%%
%%            %%%%%%%%%%%%%%%%%%%%%%%%%%%%%%%%%%%%%%%%%%%%
%%%%%%%%%%%%%%%%%%%%%%%%%%%%%%%%%%%%%%%%%%%%%%%%%%%%%%%%%%
\chapter{内径}
%!TEX root = ../Mould_Analytical_Calculation_Note.tex



%%%%%%%%%%%%%%%%%%%%%%%%%%%%%%%%%%%%%%%%%%%%%%%%%%%%%%%%%%
%% section 7.1 %%%%%%%%%%%%%%%%%%%%%%%%%%%%%%%%%%%%%%%%%%%
%%%%%%%%%%%%%%%%%%%%%%%%%%%%%%%%%%%%%%%%%%%%%%%%%%%%%%%%%%
\section{eテーパの算定}
使用する鋼材の種類として、C, Si, Mn, P, Sが含まれている場合を考える。
JIS規格に基づいた鋼種を用いるものとすれば、その規格によってそれぞれの化学組成の含有量も決定される。
それぞれの化学組成の含有量($\mathrm{wt}\%$)を$X_\mathrm C$, $X_\mathrm{Si}$, $X_\mathrm{Mn}$, $X_\mathrm P$, $X_\mathrm S$とし、またその影響係数を$k_\mathrm C$, $k_\mathrm{Si}$, $k_\mathrm{Mn}$, $k_\mathrm P$, $k_\mathrm S$とする。
また、その鋼材の液相線温度を$T_\mathrm l$[$^\circ\mathrm C$], 固相線温度を$T_\mathrm s$[$^\circ\mathrm C$]とする。
一般にこれらの温度は、ある基準となる温度$T_0$に対して、
\begin{align*}
  T = T_0-\sum_i k_iX_i
\end{align*}
として与えられる。
今の場合だと、
\begin{align*}
  T_l
  &= 1536-78X_\mathrm C-7.6X_\mathrm{Si}-4.9X_\mathrm{Mn}-34.4X_\mathrm P-38X_\mathrm S~,\\
  T_s
  &= 1536-415.5X_\mathrm C-12.3X_\mathrm{Si}-6.8X_\mathrm{Mn}-124.5X_\mathrm P-183.9X_\mathrm S
\end{align*}
となることが知られている\cite{1986KO}。




\begin{appendices}
%%%%%%%%%%%%%%%%%%%%%%%%%%%%%%%%%%%%%%%%%%%%%%%%%%%%%%%%%
%%                 %%%%%%%%%%%%%%%%%%%%%%%%%%%%%%%%%%%%%%
%%                 %%%%%%%%%%%%%%%%%%%%%%%%%%%%%%%%%%%%%%
%% Part I Appendix %%%%%%%%%%%%%%%%%%%%%%%%%%%%%%%%%%%%%%
%%                 %%%%%%%%%%%%%%%%%%%%%%%%%%%%%%%%%%%%%%
%%                 %%%%%%%%%%%%%%%%%%%%%%%%%%%%%%%%%%%%%%
%%%%%%%%%%%%%%%%%%%%%%%%%%%%%%%%%%%%%%%%%%%%%%%%%%%%%%%%%
\Appendixpart





%%%%%%%%%%%%%%%%%%%%%%%%%%%%%%%%%%%%%%%%%%%%%%%%%%%%%%%%%%
%%            %%%%%%%%%%%%%%%%%%%%%%%%%%%%%%%%%%%%%%%%%%%%
%% Appendix A %%%%%%%%%%%%%%%%%%%%%%%%%%%%%%%%%%%%%%%%%%%%
%%            %%%%%%%%%%%%%%%%%%%%%%%%%%%%%%%%%%%%%%%%%%%%
%%%%%%%%%%%%%%%%%%%%%%%%%%%%%%%%%%%%%%%%%%%%%%%%%%%%%%%%%%
\chapter{モールドとテーブルとの位置}
%!TEX root = ../Mould_Analytical_Calculation_Note.tex

\begin{tcolorbox}[title={2023/07/28時点の\MMname 実測値}, fonttitle=\gtfamily\bfseries]
\begin{align*}
  \text{Bot ($B=0$)}
  \left\{
  \begin{array}{rl}
    X: & 97.790 \sim 99.930\\
    Y: & -823.850\\
    Z: & -634.620
  \end{array}
  \right.\quad
  \text{Top ($B=180.$)}
  \left\{
  \begin{array}{rl}
    X: & -97.980 \sim -99.570\\
    Y: & -823.780\\
    Z: & -634.720
  \end{array}
  \right.
\end{align*}\\
・$X$については、ジグの当たる点の凸部と端部($Z$方向は目分量)\\
・$Y$については、モールドの底が当たる面\\
・$Z$については、$X0$ $Y-850.$における、ジグとの接点\\
※これらの値に、センサー先端球の半径を加減する必要がある
\end{tcolorbox}
%%%%%%%%%%%%%%%%%%
\begin{tcolorbox}[title={\DMname における各々の値(図面上の値)}, fonttitle=\gtfamily\bfseries]
・ジグ長さ$2l$:660 \quad ・ジグ幅:455\\
・テーブル中心 と C面側ジグ端 との水平距離:196.5\\
・受板の円の半径$\rho$:100 \quad ・受板の鉛直方向の幅$\sigma$:40\\
・テーブル中心 と 受板の円の中心 との水平距離$\varDelta$:201.5\\
・受板の円の中心 と 受板の水平方向の底 との距離:70\\[12pt]
2023/09/26時点の\DMname における各々の実測値\\
・テーブル回転中心の$X$座標:$-550.019$ \quad ・テーブル回転中心の$Z$座標:$-1149.974$
\end{tcolorbox}




%%%%%%%%%%%%%%%%%%%%%%%%%%%%%%%%%%%%%%%%%%%%%%%%%%%%%%%%%%
%% section 2.1 %%%%%%%%%%%%%%%%%%%%%%%%%%%%%%%%%%%%%%%%%%%
%%%%%%%%%%%%%%%%%%%%%%%%%%%%%%%%%%%%%%%%%%%%%%%%%%%%%%%%%%
\section{テーブル中心にあるモールド}
CADによる描画において、テーブルの回転中心が原点(ワールド原点)に置かれているとする。
ここでモールドを描画する際、モールドの中心
%% footnote %%%%%%%%%%%%%%%%%%%%%
\footnote{$R_\mathrm c$に相当する点。}\relax
%%%%%%%%%%%%%%%%%%%%%%%%%%%%%%%%%
をCAD上の原点(ワールド原点)にして描くほうが都合のいいことがある。
このとき、モールドと受板が接するように移動する必要がある。
鉛直方向(トップ-ボトム方向)においては$f_d$だけ動かせばよいが、水平方向の移動距離はあまり自明とはいいがたい。
C側モールド面と受板面との寸法を単純に測ると、(水平方向でなく)最短距離が測定されてしまう。
工夫により水平方向の距離を出すことも可能ではあるが、ここではその距離を定量的に求めておく。



%%%%%%%%%%%%%%%%%%%%%%%%%%%%%%%%%%%%%%%%%%%%%%%%%%%%%%%%%%
%% subsection 2.1.1 %%%%%%%%%%%%%%%%%%%%%%%%%%%%%%%%%%%%%%
%%%%%%%%%%%%%%%%%%%%%%%%%%%%%%%%%%%%%%%%%%%%%%%%%%%%%%%%%%
\subsection{スペーサ取付前}
(スペーサを取付る前の)モールドの中心がテーブル中心Pに置かれている場合を考える。
ボトム側の受板に接するモールドの点と、テーブル中心Pとは、実軸方向に
\begin{align*}
  R_\mathrm c-R_\mathrm i\cos\alpha_{\mathrm U_\mathrm B}
\end{align*}
だけ差がある。
したがって、モールドの受板と接する点の位置は実軸方向に、
\begin{align*}
  \varDelta+\sqrt{R_\mathrm i'^2-\bar l^2}-R_\mathrm c+R_\mathrm i\cos\alpha_{\mathrm U_\mathrm B}\ .
\end{align*}
そのため\pageeqref{eq:afterUBcontact} ($\delta = 0$)より、モールドと(ボトム側の)受板は
\begin{align*}
  \varDelta+\sqrt{R_\mathrm i'^2-\bar l^2}-R_\mathrm c
\end{align*}
だけ実軸方向に離れていることがわかる。
\pageeqref{eq:tableCenter}より、これはテーブル中心Pとモールドの中心湾曲$R_\mathrm c$との差であることがわかる。



%%%%%%%%%%%%%%%%%%%%%%%%%%%%%%%%%%%%%%%%%%%%%%%%%%%%%%%%%%
%% subsection 2.1.2 %%%%%%%%%%%%%%%%%%%%%%%%%%%%%%%%%%%%%%
%%%%%%%%%%%%%%%%%%%%%%%%%%%%%%%%%%%%%%%%%%%%%%%%%%%%%%%%%%
\subsection{スペーサ取付後}
厚さ$\delta\,(>0)$のスペーサを取付けた場合、モールドの受板と接する点とテーブル中心Pとは、実軸方向に
\begin{align*}
  R_\mathrm c-R_\mathrm i\cos\alpha'_{\mathrm U_\mathrm B}
\end{align*}
だけ差があるので、その実軸方向の位置は、
\begin{align*}
  \varDelta+\sqrt{R_\mathrm i'^2-\bar l^2}-R_\mathrm c+R_\mathrm i\cos\alpha'_{\mathrm U_\mathrm B}\ .
\end{align*}
そのため\pageeqref{eq:afterUBcontact}より、モールドと(ボトム側の)受板は
\begin{align*}
  &  \varDelta+\sqrt{R_\mathrm i'^2-\bar l^2}-R_\mathrm c+R_\mathrm i\cos\alpha'_{\mathrm U_\mathrm B}
     -\left(R_\mathrm i'\cos\alpha_{\mathrm U_\mathrm B}+\rho\cos\alpha'_{\mathrm U_\mathrm B}\right)\\
  &= \varDelta-R_\mathrm c+R_\mathrm i'\cos\alpha'_{\mathrm U_\mathrm B}\\
  &= \varDelta-R_\mathrm c
     -\frac\delta2+\sqrt{R_\mathrm i'^2-\frac{\delta^2+(2\bar l)^2}4}\frac{2\bar l}{\sqrt{\delta^2+(2\bar l)^2}}
\end{align*}
だけ実軸方向に離れていることがわかる。
\pageeqref{eq:tableCenter}および\pageeqref{eq:spacerMoveHdistance}より、これはスペーサ取付け後のモールド中心とモールドの中心湾曲$R_\mathrm c$との差であることがわかる。





%%%%%%%%%%%%%%%%%%%%%%%%%%%%%%%%%%%%%%%%%%%%%%%%%%%%%%%%%%
%%            %%%%%%%%%%%%%%%%%%%%%%%%%%%%%%%%%%%%%%%%%%%%
%% Appendix B %%%%%%%%%%%%%%%%%%%%%%%%%%%%%%%%%%%%%%%%%%%%
%%            %%%%%%%%%%%%%%%%%%%%%%%%%%%%%%%%%%%%%%%%%%%%
%%%%%%%%%%%%%%%%%%%%%%%%%%%%%%%%%%%%%%%%%%%%%%%%%%%%%%%%%%
\chapter{寸法関連の決めごと}
%!TEX root = ../Mould_Analytical_Calculation_Note.tex

ここではマシニング用プログラムを記述する際やCADで描画をする際に必要となる、図面の数値等の読み取りかたについて触れる。

なお、前提として、特別な指定やその他特記事項がある場合は、それを優先するものとする。
以下では主にそうした特別な記述のない、いわゆる一般的な場合について記載する。




%%%%%%%%%%%%%%%%%%%%%%%%%%%%%%%%%%%%%%%%%%%%%%%%%%%%%%%%%%
%% section A.1 %%%%%%%%%%%%%%%%%%%%%%%%%%%%%%%%%%%%%%%%%%%
%%%%%%%%%%%%%%%%%%%%%%%%%%%%%%%%%%%%%%%%%%%%%%%%%%%%%%%%%%
\section{基本事項}
%% paragraph %%%%%%%%%%%%%%%%%%%%%
\paragraph{寸法公差の取扱い}\noindent
全体的に、寸法公差がある場合、$+$公差と$-$公差の中央(平均)を見るものとする。
たとえば、$100^{+0.5}_{\phantom -0}$であれば、100.25とみなす
%% footnote %%%%%%%%%%%%%%%%%%%%%
\footnote{内面のテーパ表を見る際はこの限りではないことに注意。}。
%%%%%%%%%%%%%%%%%%%%%%%%%%%%%%%%%

%% paragraph %%%%%%%%%%%%%%%%%%%%%
\paragraph{寸法の優先度}\noindent
公差のある寸法と公差のない寸法(括弧寸法含む)とが共存して記載されている場合、公差のある寸法を優先する。
たとえば、2つの線の寸法がそれぞれ$12^{+0.1}_{\phantom -0}$, $4.05$と記述されていて、かつその和に相当する部分の寸法が16と記述されている場合は、16.10とみなす。




%%%%%%%%%%%%%%%%%%%%%%%%%%%%%%%%%%%%%%%%%%%%%%%%%%%%%%%%%%
%% section A.2 %%%%%%%%%%%%%%%%%%%%%%%%%%%%%%%%%%%%%%%%%%%
%%%%%%%%%%%%%%%%%%%%%%%%%%%%%%%%%%%%%%%%%%%%%%%%%%%%%%%%%%
\section{全長・振分長}
振分長の公差については、全長の公差をトップ振分長とボトム振分長との比率で分配する。
たとえば、全長が$1000^{\phantom +0}_{-1.0}$でトップ振分長が200であれば、全長の公差分$-0.5$を振分長の比$200:800$に分配し、それぞれ$-0.1$, $-0.4$とする。
つまり、トップ振分長は199.9, ボトム振分長は799.6とみなす。

ただし、簡単のため、単純に2等分してもよいものとする。
すなわち、上記の例でいうと、トップ振分長を199.75, ボトム振分長を799.75とみなしてもよいものとする
%% footnote %%%%%%%%%%%%%%%%%%%%%
\footnote{もう少し正確には、公差の分配は両者の間に収まっていればよいものとする。
すなわち、上記の例でいうと、トップ振分長は199.75~199.90に収まっていればよいものとする。}。
%%%%%%%%%%%%%%%%%%%%%%%%%%%%%%%%%

%% paragraph %%%%%%%%%%%%%%%%%%%%%
\paragraph{括弧寸法の場合}\noindent
片方の振分長が括弧寸法の場合は、全長の公差をそのまま括弧寸法に割り当てる。
たとえば、全長が$1000^{\phantom +0}_{-1.0}$でトップ振分長が200, ボトム振分長が(800)であれば、トップ振分長は200, ボトム振分長は799.5とする。




%%%%%%%%%%%%%%%%%%%%%%%%%%%%%%%%%%%%%%%%%%%%%%%%%%%%%%%%%%
%% section A.3 %%%%%%%%%%%%%%%%%%%%%%%%%%%%%%%%%%%%%%%%%%%
%%%%%%%%%%%%%%%%%%%%%%%%%%%%%%%%%%%%%%%%%%%%%%%%%%%%%%%%%%
\section{外径\label{app:gaikei}}
プログラムを記述する際は、簡単のため、端面部の水平方向の長さは、モールドの外径(中心湾曲と水平な方向)とみなしてもよいものとする。

実際には、中心湾曲を$R$, トップ振分長を$f_\mathrm T$, 外径を$W_x$とすると、トップ端面部の水平方向の長さ$W_\mathrm T$は以下で与えられる。(ボトム端面部も同様)
\begin{align*}
  W_\mathrm T
  = \sqrt{\left(R+\frac{W_x}2\right)^{\!2}-f_\mathrm T^2}
    -\sqrt{\left(R-\frac{W_x}2\right)^{\!2}-f_\mathrm T^2}\ .
\end{align*}
\begin{Column}{近似計算}
テイラー展開(マクローリン展開)より、
\begin{align*}
  (1+x)^\frac12 = 1+\frac x2-\frac{x^2}8+\frac{x^3}{16}-\frac{5x^4}{128}+o\!\left(x^5\right)
\end{align*}
なので、
\begin{align*}
  & (1+x)^\frac12(1+y)^\frac12-(1-x)^\frac12(1-y)^\frac12\\
  &= x+y+\frac{(x+y)(x-y)^2}8-\frac{xy(x+y)\big\{5(x-y)^2+7xy\big\}}{128}+\cdots\ .
\end{align*}
したがって、
\begin{align*}
  x = \frac{\nicefrac{W_x}2+f_\mathrm T}R\ ,\quad y = \frac{\nicefrac{W_x}2-f_\mathrm T}R\quad
  \longrightarrow \quad
  x+y = \frac{W_x}R\ , \quad x-y = \frac{2f_\mathrm T}R
\end{align*}
であるので、
\begin{align*}
  W_\mathrm T
  = R\left\{(1+x)^\frac12(1+y)^\frac12-(1-x)^\frac12(1-y)^\frac12\right\}
  = W_x\!\left(1+\frac{f_\mathrm T^2}{2R^2}+\cdots\right).
\end{align*}
\end{Column}




%%%%%%%%%%%%%%%%%%%%%%%%%%%%%%%%%%%%%%%%%%%%%%%%%%%%%%%%%%
%% section A.4 %%%%%%%%%%%%%%%%%%%%%%%%%%%%%%%%%%%%%%%%%%%
%%%%%%%%%%%%%%%%%%%%%%%%%%%%%%%%%%%%%%%%%%%%%%%%%%%%%%%%%%
\section{内径}
プログラムを記述する際は、簡単のため、端面部の水平方向の長さは、モールドの外径(中心湾曲と水平な方向)とみなしてもよいものとする。
これは外径(\pageautoref{app:gaikei})と同様である。

%% paragraph %%%%%%%%%%%%%%%%%%%%%
\paragraph{内面テーパおよびテーパ表}\noindent
テーパ表を参照する際は、全長の公差は考慮しないものとする。
また、トップ端からの距離のピッチも、同様に公差は考慮しないものとする。

たとえば、全長が$800^{+0.5}_{\phantom -0}$, トップ振分長が400, ピッチが25である場合を考える。
このとき、トップ端は振分中心から400の位置にあり、ピッチは25であるものとし、両端についてはそれを適宜延長して調整する。




%%%%%%%%%%%%%%%%%%%%%%%%%%%%%%%%%%%%%%%%%%%%%%%%%%%%%%%%%%
%%            %%%%%%%%%%%%%%%%%%%%%%%%%%%%%%%%%%%%%%%%%%%%
%% Appendix C %%%%%%%%%%%%%%%%%%%%%%%%%%%%%%%%%%%%%%%%%%%%
%%            %%%%%%%%%%%%%%%%%%%%%%%%%%%%%%%%%%%%%%%%%%%%
%%%%%%%%%%%%%%%%%%%%%%%%%%%%%%%%%%%%%%%%%%%%%%%%%%%%%%%%%%
\chapter{プログラム関連の決めごと}
%!TEX root = ../Mould_Analytical_Calculation_Note.tex

ここでは\DMname における製品の加工・測定に関する、プログラムの構成や番号付けの規則を記載する
%% footnote %%%%%%%%%%%%%%%%%%%%%
\footnote{工具長の測定やジグの測定など、製品の加工とは直接関係しないプログラムについてはこの限りではない。}。
%%%%%%%%%%%%%%%%%%%%%%%%%%%%%%%%%
なお、ここに記載しているものは正式なルールではなく、だいたいの目安・方針である。


%%%%%%%%%%%%%%%%%%%%%%%%%%%%%%%%%%%%%%%%%%%%%%%%%%%%%%%%%%
%% section C.1 %%%%%%%%%%%%%%%%%%%%%%%%%%%%%%%%%%%%%%%%%%%
%%%%%%%%%%%%%%%%%%%%%%%%%%%%%%%%%%%%%%%%%%%%%%%%%%%%%%%%%%
\section{プログラムの構成}
\begin{enumerate}
\item 個々の製品の計測・加工に対するプログラムはメインプログラムとする
\item 製品の各々の部分の計測・加工に対するプログラムはサブプログラムとして、メインプログラムに挿入する
\item サブプログラムの種類(番号)は、以下のように内容で分ける
  \begin{enumerate}
  \item 内面溝・逃し溝以外の計測に対するプログラム
  \item 内面溝の計測に対するプログラム
  \item 逃し溝の計測に対するプログラム
  \item 内面溝・逃し溝以外の加工に対するプログラム
  \item 内面溝の加工に対するプログラム
  \item 逃し溝の加工に対するプログラム
  \item その他、間接的な用途に使用するプログラム
  \end{enumerate}
\end{enumerate}



%%%%%%%%%%%%%%%%%%%%%%%%%%%%%%%%%%%%%%%%%%%%%%%%%%%%%%%%%%
%% section C.2 %%%%%%%%%%%%%%%%%%%%%%%%%%%%%%%%%%%%%%%%%%%
%%%%%%%%%%%%%%%%%%%%%%%%%%%%%%%%%%%%%%%%%%%%%%%%%%%%%%%%%%
\section{プログラムの番号付け}
\begin{enumerate}
\item プログラム番号には半角数字のみを用いる
\item プログラム番号には8桁の数字を用いる(ただし、左側0埋めの有無は問わない)
\item プログラム番号は右詰めとする(左側0埋めの有無は問わない)
\end{enumerate}
これをふまえ、メインプログラムとサブプログラムではそれぞれの以下のように番号付けを行う。



%%%%%%%%%%%%%%%%%%%%%%%%%%%%%%%%%%%%%%%%%%%%%%%%%%%%%%%%%%
%% subsection C.2.1 %%%%%%%%%%%%%%%%%%%%%%%%%%%%%%%%%%%%%%
%%%%%%%%%%%%%%%%%%%%%%%%%%%%%%%%%%%%%%%%%%%%%%%%%%%%%%%%%%
\subsection{メインプログラム}
\begin{enumerate}
\item 製品の図面番号(番号部分)とメインプログラム番号は同じものとする
%% footnote %%%%%%%%%%%%%%%%%%%%%
\footnote{この規則だと、バンドルのプログラム(O7xxx, O8xxx, O9xxx)と重複する恐れがある。
これについてはそうした問題に直面したときにその都度に対応するものとする。
基本的には、バンドルのプログラムを(可能であれば)変更する方針とする。}
%%%%%%%%%%%%%%%%%%%%%%%%%%%%%%%%%
\end{enumerate}



%%%%%%%%%%%%%%%%%%%%%%%%%%%%%%%%%%%%%%%%%%%%%%%%%%%%%%%%%%
%% subsection C.2.2 %%%%%%%%%%%%%%%%%%%%%%%%%%%%%%%%%%%%%%
%%%%%%%%%%%%%%%%%%%%%%%%%%%%%%%%%%%%%%%%%%%%%%%%%%%%%%%%%%
\subsection{サブプログラム}
\begin{enumerate}
\item 測定(内面溝・逃し溝を除く)に関するものは6桁目を1とする
\item 測定(内面溝)に関するものは6桁目を2とする
\item 測定(逃し溝)の測定に関するものは6桁目を3とする
\item 加工(内面溝・逃し溝を除く)に関するものは6桁目を5とする
\item 加工(内面溝)に関するものは6桁目を6とする
\item 加工(逃し溝)に関するものは6桁目を7とする
\item その他、計測・加工に直接関しないものは6桁目を9とする
\item 計測・加工の両方に同じもの用いるものは番号の若いほうに合わせる
\end{enumerate}



%%%%%%%%%%%%%%%%%%%%%%%%%%%%%%%%%%%%%%%%%%%%%%%%%%%%%%%%%%
%% section C.3 %%%%%%%%%%%%%%%%%%%%%%%%%%%%%%%%%%%%%%%%%%%
%%%%%%%%%%%%%%%%%%%%%%%%%%%%%%%%%%%%%%%%%%%%%%%%%%%%%%%%%%
\section{工具の速さ(Fコード値)}
\DMname では、全長の長いタッチセンサーを用いる。
したがって、速さを大きくして移動をすると、その慣性によってセンサーが反応してしまったり、タッチセンサーそのものに大きな負担がかかる。
そのため、タッチセンサーの速さに関しては他の工具よりF値を低めに設定するものとする。



%%%%%%%%%%%%%%%%%%%%%%%%%%%%%%%%%%%%%%%%%%%%%%%%%%%%%%%%%%
%% subsection C.3.1 %%%%%%%%%%%%%%%%%%%%%%%%%%%%%%%%%%%%%%
%%%%%%%%%%%%%%%%%%%%%%%%%%%%%%%%%%%%%%%%%%%%%%%%%%%%%%%%%%
\subsection{タッチセンサー}
\begin{enumerate}
\item 原則として、G00は使用しない
\item G01を位置決め(早送り)として用いるものとし、速さはF5400以下とする
\item ワークへのアプローチの際は、G31を用いるものとし、速さはF1500以下とする
\item 計測の際のスキップ(G31)の速さは、計測の仕方に応じて以下のものとする
  \begin{enumerate}
  \item 信号遅れ補正を考慮する必要があるような場合は、速さはF50とする
  \item 信号遅れ補正を考慮する必要がない場合は、速さはF50以上300以下とする
  \end{enumerate}
\item 測定直後、ワークから離れる際は、G01を用いて、速さはF3600以下とする
\end{enumerate}




%%%%%%%%%%%%%%%%%%%%%%%%%%%%%%%%%%%%%%%%%%%%%%%%%%%%%%%%%%
%% subsection C.3.2 %%%%%%%%%%%%%%%%%%%%%%%%%%%%%%%%%%%%%%
%%%%%%%%%%%%%%%%%%%%%%%%%%%%%%%%%%%%%%%%%%%%%%%%%%%%%%%%%%
\subsection{タッチセンサー以外の工具}
\begin{enumerate}
\item G00(位置決め・早送り)は、速さはF10800以下とする
\item ワークへのアプローチの際は、G01を用いるものとし、速さはF5400以下とする
\item 加工の際は、それぞれの加工に応じた適切な速さ値を設定する
\end{enumerate}





%%%%%%%%%%%%%%%%%%%%%%%%%%%%%%%%%%%%%%%%%%%%%%%%%%%%%%%%%%
%%            %%%%%%%%%%%%%%%%%%%%%%%%%%%%%%%%%%%%%%%%%%%%
%% Appendix D %%%%%%%%%%%%%%%%%%%%%%%%%%%%%%%%%%%%%%%%%%%%
%%            %%%%%%%%%%%%%%%%%%%%%%%%%%%%%%%%%%%%%%%%%%%%
%%%%%%%%%%%%%%%%%%%%%%%%%%%%%%%%%%%%%%%%%%%%%%%%%%%%%%%%%%
\chapter{コモン変数}
%!TEX root = ../Mould_Analytical_Calculation_Note.tex

\DMname で取り決めているコモン変数について、以下に挙げる。



%%%%%%%%%%%%%%%%%%%%%%%%%%%%%%%%%%%%%%%%%%%%%%%%%%%%%%%%%%
%% section C.1 %%%%%%%%%%%%%%%%%%%%%%%%%%%%%%%%%%%%%%%%%%%
%%%%%%%%%%%%%%%%%%%%%%%%%%%%%%%%%%%%%%%%%%%%%%%%%%%%%%%%%%
\section{\DMname}



%%%%%%%%%%%%%%%%%%%%%%%%%%%%%%%%%%%%%%%%%%%%%%%%%%%%%%%%%%
%% subsection C.1.1 %%%%%%%%%%%%%%%%%%%%%%%%%%%%%%%%%%%%%%
%%%%%%%%%%%%%%%%%%%%%%%%%%%%%%%%%%%%%%%%%%%%%%%%%%%%%%%%%%
\subsection{コモン変数 (\#100 - \#199)}
\#100 - \#199については、(機械設置時の)バンドルのプログラムで既に使用されているものが多いため、基本的には使用しないものとする。
使用する場合は、一時的なものとして扱うものとする。



%%%%%%%%%%%%%%%%%%%%%%%%%%%%%%%%%%%%%%%%%%%%%%%%%%%%%%%%%%
%% subsection C.1.2 %%%%%%%%%%%%%%%%%%%%%%%%%%%%%%%%%%%%%%
%%%%%%%%%%%%%%%%%%%%%%%%%%%%%%%%%%%%%%%%%%%%%%%%%%%%%%%%%%
\subsection{コモン変数 (\#400 - \#499)}
\#400 - \#499については、主に管理者が入力・変更することが想定されるものとする。
\begin{twoCtable}{}
\#400 & (予備)\\\hline
\hline
\#101 & 端面加工 1回あたりの$Z$方向削り代\\\hline
\#102 & (予備)\\\hline
\hline
\#103 & 工具T31(Tスロット)A側内面溝 深さ補正値(深さに$+$補正)\\\hline
\#104 & 工具T31(Tスロット)C側内面溝 深さ補正値(深さに$+$補正)\\\hline
\#105 & 工具T31(Tスロット)B側内面溝 深さ補正値(深さに$+$補正)\\\hline
\#106 & 工具T31(Tスロット)D側内面溝 深さ補正値(深さに$+$補正)\\\hline
\#107 & (予備)\\\hline
\hline
\#108 & 工具T32(Tスロット)A側内面溝 深さ補正値(深さに$+$補正)\\\hline
\#109 & 工具T32(Tスロット)C側内面溝 深さ補正値(深さに$+$補正)\\\hline
\#110 & 工具T32(Tスロット)B側内面溝 深さ補正値(深さに$+$補正)\\\hline
\#111 & 工具T32(Tスロット)D側内面溝 深さ補正値(深さに$+$補正)\\\hline
\#112 & (予備)\\\hline
\hline
\#113 & 工具T33(Tスロット)A側内面溝 深さ補正値(深さに$+$補正)\\\hline
\#114 & 工具T33(Tスロット)C側内面溝 深さ補正値(深さに$+$補正)\\\hline
\#115 & 工具T33(Tスロット)B側内面溝 深さ補正値(深さに$+$補正)\\\hline
\#116 & 工具T33(Tスロット)D側内面溝 深さ補正値(深さに$+$補正)\\\hline
\#117 & (予備)\\
\end{twoCtable}



\clearpage
%%%%%%%%%%%%%%%%%%%%%%%%%%%%%%%%%%%%%%%%%%%%%%%%%%%%%%%%%%
%% subsection C.1.3 %%%%%%%%%%%%%%%%%%%%%%%%%%%%%%%%%%%%%%
%%%%%%%%%%%%%%%%%%%%%%%%%%%%%%%%%%%%%%%%%%%%%%%%%%%%%%%%%%
\subsection{コモン変数 (\#400 - \#499)}
\#450 - \#499については、主に管理者が入力・変更することが想定されるものとする。
\begin{twoCtable}{}
\#450 & (予備)\\\hline
\#451 & パレット\#1 ジグ中心機械座標$X$\\\hline
\#452 & パレット\#1 ジグ中心機械座標$Y$\\\hline
\#453 & パレット\#1 ジグ中心機械座標$Z$\\\hline
\#454 & パレット\#1 ジグ中心機械座標$B$\\\hline
\#455 & パレット\#2 ジグ中心機械座標$X$\\\hline
\#456 & パレット\#2 ジグ中心機械座標$Y$\\\hline
\#457 & パレット\#2 ジグ中心機械座標$Z$\\\hline
\#458 & パレット\#2 ジグ中心機械座標$B$\\\hline
\#459 & 工具中心機械座標$C$\\\hline
\#460 & (予備)\\\hline
\hline
\#461 & パレット\#1ジグ外側幅(機械座標系$B$0における$Z$方向)\\\hline
\#462 & パレット\#1ジグ内側幅(機械座標系$B$0における$Z$方向)\\\hline
\#463 & パレット\#1ジグ幅(機械座標系$B$0における$X$方向)\\\hline
\#464 & (予備)\\\hline
\#465 & パレット\#2ジグ外側幅(機械座標系$B$0における$Z$方向)\\\hline
\#466 & パレット\#2ジグ内側幅(機械座標系$B$0における$Z$方向)\\\hline
\#467 & パレット\#2ジグ幅(機械座標系$B$0における$X$方向)\\\hline
\#468 & (予備)\\
\end{twoCtable}



\clearpage
%%%%%%%%%%%%%%%%%%%%%%%%%%%%%%%%%%%%%%%%%%%%%%%%%%%%%%%%%%
%% subsection C.1.3 %%%%%%%%%%%%%%%%%%%%%%%%%%%%%%%%%%%%%%
%%%%%%%%%%%%%%%%%%%%%%%%%%%%%%%%%%%%%%%%%%%%%%%%%%%%%%%%%%
\subsection{コモン変数 (\#500-\#599)}
\#500 - \#599については、主にO910xおよびO93xxで使用されるものとする。
\begin{twoCtable}{}
\#500 & 芯ずれ許容差 (O93xx)\\\hline
\#501 & タッチセンサー信号遅れ補正 (O93xx)\\\hline
\#502 & タッチセンサープローブ中心$X$補正 (O93xx)\\\hline
\#503 & タッチセンサープローブ中心$Y$補正 (O93xx)\\\hline
\#504 & 測定距離 (O910x)\\\hline
\#505 & プローブ表面からプログラムの加工原点($Z$0)までの距離 (O910x)\\\hline
\#506 & 工具長の変化の許容差 (O910x)\\\hline
\#507 & 工具破損検出の許容差 (O910x)\\\hline
\#509 & Z座標系設定 (O93xx)\\\hline
\#511 & インチ/ミリ切替 (O910x)\\\hline
\#512 & タッチセンサープローブ半径$\mathrm{mm}$値 (O93xx)\\\hline
\#513 & 移動時用の送り速さ値 (O910x)\\\hline
\#514 & スキップ(G31)測定時用 送り速さ値 (O910x, O93xx)\\\hline
\#516 & センサーの位置$X$座標 (O910x)\\\hline
\#517 & センサーの位置$Y$座標 (O910x)\\\hline
\#518 & センサーの位置$Z$座標 (O910x)\\\hline
\#520 & 拡張ワーク座標系 (O910x)\\\hline
\#523 & アプローチ時用の送り速さ値 (O910x)\\\hline
\#524 & 測定時用の送り速さ値 (O910x)\\
\end{twoCtable}



\clearpage
%%%%%%%%%%%%%%%%%%%%%%%%%%%%%%%%%%%%%%%%%%%%%%%%%%%%%%%%%%
%% subsection C.1.4 %%%%%%%%%%%%%%%%%%%%%%%%%%%%%%%%%%%%%%
%%%%%%%%%%%%%%%%%%%%%%%%%%%%%%%%%%%%%%%%%%%%%%%%%%%%%%%%%%
\subsection{コモン変数 (\#600-\#699)}
\#600 - \#699については、主に管理者が入力・変更することが想定されるものとする。
\begin{twoCtable}{}
\#600 & 振分調整用角度$-\theta[\deg]$\\\hline
\hline
\#601 & 工具T02(フェイスミル)最大刃径(直径)DCX公称値$\phi'_\mathrm D$\\\hline
\#602 & (工具T02用予備)\\\hline
\#603 & (工具T03フェイスミル用予備)\\\hline
\#604 & (工具T03フェイスミル用予備)\\\hline
\hline
\#605 & 工具T06(サイドカッター)厚さ$t$\\\hline
\#606 & (工具T06用予備)\\\hline
\#607 & (工具T07サイドカッター用予備)\\\hline
\#608 & (工具T07サイドカッター用予備)\\\hline
\#609 & 工具T08(サイドカッター)厚さ$t$\\\hline
\#610 & (工具T08用予備)\\\hline
\#611 & (工具T09サイドカッター用予備)\\\hline
\#612 & (工具T09サイドカッター用予備)\\\hline
\hline
\#613 & (工具T11テーパエンドミル用予備)\\\hline
\#614 & (工具T11テーパエンドミル用予備)\\\hline
\#615 & (工具T12テーパエンドミル用予備)\\\hline
\#616 & (工具T12テーパエンドミル用予備)\\\hline
\#617 & (工具T13テーパエンドミル用予備)\\\hline
\#618 & (工具T13テーパエンドミル用予備)\\\hline
\#619 & (工具T14テーパエンドミル用予備)\\\hline
\#620 & (工具T14テーパエンドミル用予備)\\\hline
\hline
\#621 & (工具T16スクエアエンドミル用予備)\\\hline
\#622 & (工具T16スクエアエンドミル用予備)\\\hline
\#623 & (工具T17スクエアエンドミル用予備)\\\hline
\#624 & (工具T17スクエアエンドミル用予備)\\\hline
\#625 & (工具T18スクエアエンドミル用予備)\\\hline
\#626 & (工具T18スクエアエンドミル用予備)\\\hline
\hline
\#627 & (工具T31 Tスロットカッター用予備)\\\hline
\#628 & (工具T31 Tスロットカッター用予備)\\\hline
\#629 & (工具T32 Tスロットカッター用予備)\\\hline
\#630 & (工具T32 Tスロットカッター用予備)\\\hline
\#631 & (工具T33 Tスロットカッター用予備)\\\hline
\#632 & (工具T33 Tスロットカッター用予備)\\\hline
\#633 & (工具T34 Tスロットカッター用予備)\\\hline
\#634 & (工具T34 Tスロットカッター用予備)\\
\end{twoCtable}



%\clearpage
%%%%%%%%%%%%%%%%%%%%%%%%%%%%%%%%%%%%%%%%%%%%%%%%%%%%%%%%%%
%% subsection C.1.5 %%%%%%%%%%%%%%%%%%%%%%%%%%%%%%%%%%%%%%
%%%%%%%%%%%%%%%%%%%%%%%%%%%%%%%%%%%%%%%%%%%%%%%%%%%%%%%%%%
\subsection{コモン変数 (\#701-\#799)}
\#700 - \#799については、主に内面溝用サブプログラム(O2x000x, O6x000x)で使用されるものとする


%%%%%%%%%%%%%%%%%%%%%%%%%%%%%%%%%%%%%%%%%%%%%%%%%%%%%%%%%%
%% subsection C.1.5.1 %%%%%%%%%%%%%%%%%%%%%%%%%%%%%%%%%%%%
%%%%%%%%%%%%%%%%%%%%%%%%%%%%%%%%%%%%%%%%%%%%%%%%%%%%%%%%%%
\subsubsection{コモン変数 (\#701-733)}
\#700 - \#733については、主に\DLone で使用されるものとする。
\begin{twoCtable}{}
\#701 & プログラム読込み時の座標系(\#4012)\\\hline
\#702 & 工具別$Z$補正(T50:\#512, T3x:0)\\\hline
\#703 & 工具別$XY$補正(T50:\#512, T3x:\#[2400+\#4111]+\#[2600+\#4111])\\\hline
\#704 & 工具別移動G\# (T50:31, T3x:1)\\\hline
\#705 & テーブル中心からワーク座標(\#701)原点までの$X$距離\\\hline
\#706 & 傾き後のトップ端面中心(機械座標)$X$ (cf. \pageeqref{eq:afterPhiTCenterFromO})\\\hline
\#707 & テーブル中心から傾き後のトップ端面中心までの$Z$距離 (cf. \pageeqref{eq:afterPhiTCenterFromO})\\\hline
\#708 & 傾き後トップ端中心(ブロックエンド)$X$座標(\#5001)\\\hline
\#709 & 傾き後トップ端中心(ブロックエンド)$Z$座標(\#5003)\\\hline
\#710 & テーブル中心から内面溝1列目までの$Z$距離$Z-q$\\\hline
\#711 & トップ端中心から内面溝1列目中心までの$X$距離(cf. \pageeqref{eq:dimpleCenterDistance})\\\hline
\#712 & 傾き後内面溝1列目中心$X$移動距離(cf. \pageeqref{eq:afterPhidimpleCenterDistance})\\\hline
\#713 & 傾き後内面溝1列目中心$Z$移動距離(cf. \pageeqref{eq:afterPhidimpleCenterDistance})\\\hline
\#714 & 傾き後内面溝1列目中心(ブロックエンド)$X$座標 (\#5001)\\\hline
\#715 & 傾き後内面溝1列目中心(ブロックエンド)$Y$座標 (\#5002)\\\hline
\#716 & 傾き後内面溝1列目中心(ブロックエンド)$Z$座標 (\#5003)\\\hline
\#717 & 各面用ループ番号(1:A, 2:C, 3:B, 4:D)\\\hline
\#718 & BD内半径$-\#703-10$\\\hline
\#719 & (AC内半径$-\#703-10)\cos\phi$\\\hline
& (以下予備)
\end{twoCtable}



\clearpage
%%%%%%%%%%%%%%%%%%%%%%%%%%%%%%%%%%%%%%%%%%%%%%%%%%%%%%%%%%
%% subsection C.1.5.2 %%%%%%%%%%%%%%%%%%%%%%%%%%%%%%%%%%%%
%%%%%%%%%%%%%%%%%%%%%%%%%%%%%%%%%%%%%%%%%%%%%%%%%%%%%%%%%%
\subsubsection{コモン変数 (\#734-766)}
\#734 - \#766については、主に\DLtwoAC, \DLtwoBD で使用されるものとする。
\begin{twoCtable}{}
\#734 & プログラム読込時ブロックエンド$Y$ or $X$ (\#5002, \#5001)\\\hline
\#735 & プログラム読込時ブロックエンド$Z$ (\#5003)\\\hline
\#736 & 内面溝 偶数列の列数\\\hline
\#737 & 内面溝 偶数列(一列)の内面溝数\\\hline
\#738 & 内面溝 奇数列(一列)の内面溝数\\\hline
\#739 & 内面溝 現在の列の内面溝数\\\hline
& (以下予備)
\end{twoCtable}



%%%%%%%%%%%%%%%%%%%%%%%%%%%%%%%%%%%%%%%%%%%%%%%%%%%%%%%%%%
%% subsection C.1.5.3 %%%%%%%%%%%%%%%%%%%%%%%%%%%%%%%%%%%%
%%%%%%%%%%%%%%%%%%%%%%%%%%%%%%%%%%%%%%%%%%%%%%%%%%%%%%%%%%
\subsubsection{コモン変数 (\#767-799)}
\#767 - \#799については、主に\DMLthreeAC, \DMLthreeBD, \DKLthreeAC, \DKLthreeBD で使用されるものとする。
\begin{twoCtable}{}
\#767 & プログラム読込時ブロックエンド$X$ or $Y$ (\#5001, \#5002)\\\hline
\#768 & 内面溝 表面位置$X$ or $Y$測定値\\\hline
& (以下予備)
\end{twoCtable}



\clearpage
%%%%%%%%%%%%%%%%%%%%%%%%%%%%%%%%%%%%%%%%%%%%%%%%%%%%%%%%%%
%% subsection C.1.6 %%%%%%%%%%%%%%%%%%%%%%%%%%%%%%%%%%%%%%
%%%%%%%%%%%%%%%%%%%%%%%%%%%%%%%%%%%%%%%%%%%%%%%%%%%%%%%%%%
\subsection{コモン変数 (\#900001-\#900031, \#900101-\#900500)}
\#900000 - \#900999については、主に実測値を格納する。



%%%%%%%%%%%%%%%%%%%%%%%%%%%%%%%%%%%%%%%%%%%%%%%%%%%%%%%%%%
%% subsection C.1.6.1  %%%%%%%%%%%%%%%%%%%%%%%%%%%%%%%%%%%
%%%%%%%%%%%%%%%%%%%%%%%%%%%%%%%%%%%%%%%%%%%%%%%%%%%%%%%%%%
\subsubsection{\#900001-900005}
\#900001 - \#900005については、主に\MXOThickness で使用されるものとする。
\begin{twoCtable}{}
\#900001 & $X$外中心測定 $-X$側測定値\\\hline
\#900002 & $X$外中心測定 $+X$側測定値\\\hline
\#900003 & $X$外中心測定値\\\hline
\#900004 & $X$外中心測定 厚さ測定値\\\hline
\#900005 & (予備)\\
\end{twoCtable}



%%%%%%%%%%%%%%%%%%%%%%%%%%%%%%%%%%%%%%%%%%%%%%%%%%%%%%%%%%
%% subsection C.1.6.2  %%%%%%%%%%%%%%%%%%%%%%%%%%%%%%%%%%%
%%%%%%%%%%%%%%%%%%%%%%%%%%%%%%%%%%%%%%%%%%%%%%%%%%%%%%%%%%
\subsubsection{\#900006-900010}
\#900006 - \#900010については、主に\MYOThickness で使用されるものとする。
\begin{twoCtable}{}
\#900006 & $Y$外中心測定 $-Y$側測定値\\\hline
\#900007 & $Y$外中心測定 $+Y$側測定値\\\hline
\#900008 & $Y$外中心測定値\\\hline
\#900009 & $Y$外中心測定 厚さ測定値\\\hline
\#900010 & (予備)\\
\end{twoCtable}



%%%%%%%%%%%%%%%%%%%%%%%%%%%%%%%%%%%%%%%%%%%%%%%%%%%%%%%%%%
%% subsection C.1.6.3  %%%%%%%%%%%%%%%%%%%%%%%%%%%%%%%%%%%
%%%%%%%%%%%%%%%%%%%%%%%%%%%%%%%%%%%%%%%%%%%%%%%%%%%%%%%%%%
\subsubsection{\#900011-900015}
\#900011 - \#900015については、主に\MXIWidth で使用されるものとする。
\begin{twoCtable}{}
\#900011 & $X$内中心測定 $-X$側測定値\\\hline
\#900012 & $X$内中心測定 $+X$側測定値\\\hline
\#900013 & $X$内中心測定値\\\hline
\#900014 & $X$内中心測定 厚さ測定値\\\hline
\#900015 & (予備)\\
\end{twoCtable}



%%%%%%%%%%%%%%%%%%%%%%%%%%%%%%%%%%%%%%%%%%%%%%%%%%%%%%%%%%
%% subsection C.1.6.4  %%%%%%%%%%%%%%%%%%%%%%%%%%%%%%%%%%%
%%%%%%%%%%%%%%%%%%%%%%%%%%%%%%%%%%%%%%%%%%%%%%%%%%%%%%%%%%
\subsubsection{\#900016-900020}
\#900016 - \#900020については、主に\MYIWidth で使用されるものとする。
\begin{twoCtable}{}
\#900016 & $Y$内中心測定 $-Y$側測定値\\\hline
\#900017 & $Y$内中心測定 $+Y$側測定値\\\hline
\#900018 & $Y$内中心測定値\\\hline
\#900019 & $Y$内中心測定 厚さ測定値\\\hline
\#900020 & (予備)\\
\end{twoCtable}



%%%%%%%%%%%%%%%%%%%%%%%%%%%%%%%%%%%%%%%%%%%%%%%%%%%%%%%%%%
%% subsection C.1.6.5  %%%%%%%%%%%%%%%%%%%%%%%%%%%%%%%%%%%
%%%%%%%%%%%%%%%%%%%%%%%%%%%%%%%%%%%%%%%%%%%%%%%%%%%%%%%%%%
\subsubsection{\#900021-900023}
\#900021 - \#900023については、主に\MXface で使用されるものとする。
\begin{twoCtable}{}
\#900021 & $X$外削中心測定 内面測定値\\\hline
\#900022 & (予備)\\\hline
\#900023 & (予備)\\
\end{twoCtable}



%%%%%%%%%%%%%%%%%%%%%%%%%%%%%%%%%%%%%%%%%%%%%%%%%%%%%%%%%%
%% subsection C.1.6.6  %%%%%%%%%%%%%%%%%%%%%%%%%%%%%%%%%%%
%%%%%%%%%%%%%%%%%%%%%%%%%%%%%%%%%%%%%%%%%%%%%%%%%%%%%%%%%%
\subsubsection{\#900024-900027}
\#900024 - \#900027については、主に\MYcenterline で使用されるものとする。
\begin{twoCtable}{}
\#900024 & $Y$通り芯 ボトム側測定値\\\hline
\#900025 & $Y$通り芯 トップ側測定値\\\hline
\#900026 & $Y$通り芯 測定値\\\hline
\#900027 & (予備)\\
\end{twoCtable}



%%%%%%%%%%%%%%%%%%%%%%%%%%%%%%%%%%%%%%%%%%%%%%%%%%%%%%%%%%
%% subsection C.1.6.7  %%%%%%%%%%%%%%%%%%%%%%%%%%%%%%%%%%%
%%%%%%%%%%%%%%%%%%%%%%%%%%%%%%%%%%%%%%%%%%%%%%%%%%%%%%%%%%
\subsubsection{\#900028-900031}
\#900028 - \#900031については、主に\MXcenterline で使用されるものとする。
\begin{twoCtable}{}
\#900028 & $X$通り芯 トップ側測定値\\\hline
\#900029 & $X$通り芯 ボトム側測定値\\\hline
\#900030 & $X$通り芯 測定値\\\hline
\#900031 & (予備)\\
\end{twoCtable}



%%%%%%%%%%%%%%%%%%%%%%%%%%%%%%%%%%%%%%%%%%%%%%%%%%%%%%%%%%
%% subsection C.1.6.8  %%%%%%%%%%%%%%%%%%%%%%%%%%%%%%%%%%%
%%%%%%%%%%%%%%%%%%%%%%%%%%%%%%%%%%%%%%%%%%%%%%%%%%%%%%%%%%
\subsubsection{\#900101-900500}
\#900101 - \#900500については、主に\DMLthreeAC, \DMLthreeBD で使用されるものとする。
\begin{twoCtable}{}
\#900101-\#900200 & A側内面溝 深さ測定値(Tスロット)\\\hline
\#900201-\#900300 & C側内面溝 深さ測定値(Tスロット)\\\hline
\#900301-\#900400 & B側内面溝 深さ測定値(Tスロット)\\\hline
\#900401-\#900500 & D側内面溝 深さ測定値(Tスロット)
\end{twoCtable}




\clearpage
%%%%%%%%%%%%%%%%%%%%%%%%%%%%%%%%%%%%%%%%%%%%%%%%%%%%%%%%%%
%% subsection C.1.3 %%%%%%%%%%%%%%%%%%%%%%%%%%%%%%%%%%%%%%
%%%%%%%%%%%%%%%%%%%%%%%%%%%%%%%%%%%%%%%%%%%%%%%%%%%%%%%%%%
\subsection{システム変数}

\begin{twoCtable}{\paragraph{システム変数:\DMname}}
\#1000 & パレット\#~~0:\#1, 1:\#2\\\hline
\#1004 & タッチセンサー電源~~0: off, 1: on\\\hline
\#1005 & タッチセンサー電池残量~~0: ok, 1: low\\\hline
\#2000+xx & 工具長補正 \#xx補正量(摩耗0とした値, xx=1-200)\\\hline
\#2200+xx & 工具長補正 \#xx摩耗 (xx=1-200)\\\hline
\#2400+xx & 工具径補正 \#xx補正量(摩耗0とした値, xx=1-200)\\\hline
\#2600+xx & 工具径補正 \#xx摩耗 (xx=1-200)\\\hline
\#3000 & アラーム\\\hline
\#3011 & 現在の年月日(yyyymmdd)\\\hline
\#4012 & 現在のワーク座標系\# (G\#)\\\hline
\#4107 & 直前の工具径補正コード\# (D\#)\\\hline
\#4111 & 直前の工具長補正コード\# (H\#)\\\hline
\#4113 & 直前のブロック指令 Mコード\# (M\#)\\\hline
\#4114 & 直前のブロック指令 シーケンス\# (N\#)\\\hline
\#4115 & 直前のブロック指令 プログラム\# (O\#)\\\hline
\#4120 & 直前のブロック指令 工具コード\# (T\#)\\\hline
\#500x & ブロック終点位置 1:X, 2:Y, 3:Z, 4:B, 5:C(ワーク座標系)
\footnote{\#500x:工具補正値を引いた値。途中でスキップがオンになったときはそのときの値。}\\\hline
\#502x & 現在の機械座標系の座標 1:X, 2:Y, 3:Z, 4:B, 5:C\\\hline
\#504x & 現在のワーク座標系の座標 1:X, 2:Y, 3:Z, 4:B, 5:C\\\hline
\#506x & スキップ座標 1:X, 2:Y, 3:Z, 4:B, 5:C(工具補正0とした値)\\\hline
\#522x & ワーク座標系G54原点の機械座標 1:X, 2:Y, 3:Z, 4:B, 5:C\\\hline
\#524x & ワーク座標系G55原点の機械座標 1:X, 2:Y, 3:Z, 4:B, 5:C\\\hline
\#526x & ワーク座標系G56原点の機械座標 1:X, 2:Y, 3:Z, 4:B, 5:C\\\hline
\#528x & ワーク座標系G57原点の機械座標 1:X, 2:Y, 3:Z, 4:B, 5:C\\\hline
\#530x & ワーク座標系G58原点の機械座標 1:X, 2:Y, 3:Z, 4:B, 5:C\\\hline
\#532x & ワーク座標系G59原点の機械座標 1:X, 2:Y, 3:Z, 4:B, 5:C\\\hline
\#10000+xx & 工具長補正 \#xx補正量(摩耗0とした値, xx=1-200)
\end{twoCtable}




\clearpage
%%%%%%%%%%%%%%%%%%%%%%%%%%%%%%%%%%%%%%%%%%%%%%%%%%%%%%%%%%
%% section C.2 %%%%%%%%%%%%%%%%%%%%%%%%%%%%%%%%%%%%%%%%%%%
%%%%%%%%%%%%%%%%%%%%%%%%%%%%%%%%%%%%%%%%%%%%%%%%%%%%%%%%%%
\section{\MMname}



%%%%%%%%%%%%%%%%%%%%%%%%%%%%%%%%%%%%%%%%%%%%%%%%%%%%%%%%%%
%% subsection C.2.1 %%%%%%%%%%%%%%%%%%%%%%%%%%%%%%%%%%%%%%
%%%%%%%%%%%%%%%%%%%%%%%%%%%%%%%%%%%%%%%%%%%%%%%%%%%%%%%%%%
\subsection{コモン変数}

\begin{twoCtable}{コモン変数:\DMname}
\#145 &  スキップ(G31)測定時用 送り速さ値\\
\end{twoCtable}




%%%%%%%%%%%%%%%%%%%%%%%%%%%%%%%%%%%%%%%%%%%%%%%%%%%%%%%%%%
%% subsection B.1.2 %%%%%%%%%%%%%%%%%%%%%%%%%%%%%%%%%%%%%%
%%%%%%%%%%%%%%%%%%%%%%%%%%%%%%%%%%%%%%%%%%%%%%%%%%%%%%%%%%
\subsection{システム変数}

\begin{twoCtable}{システム変数:\DMname}
\#4111 & 現在の工具長補正 Hコード\#\\\hline
\#4120 & 現在の工具 Tコード\#\\\hline
\#502x & 現在の機械座標系の座標 1:X, 2:Y, 3:Z, 4:B\\\hline
\#504x & 現在のワーク座標系の座標 1:X, 2:Y, 3:Z, 4:B\\\hline
\#506x & スキップ座標 1:X, 2:Y, 3:Z, 4:B(工具補正0とした値)\\\hline
\#522x & ワーク座標系G54原点の機械座標 1:X, 2:Y, 3:Z, 4:B\\\hline
\#524x & ワーク座標系G55原点の機械座標 1:X, 2:Y, 3:Z, 4:B\\\hline
\#526x & ワーク座標系G56原点の機械座標 1:X, 2:Y, 3:Z, 4:B\\\hline
\#528x & ワーク座標系G57原点の機械座標 1:X, 2:Y, 3:Z, 4:B\\\hline
\#530x & ワーク座標系G58原点の機械座標 1:X, 2:Y, 3:Z, 4:B\\\hline
\#532x & ワーク座標系G59原点の機械座標 1:X, 2:Y, 3:Z, 4:B\\
\end{twoCtable}







%%%%%%%%%%%%%%%%%%%%%%%%%%%%%%%%%%%%%%%%%%%%%%%%%%%%%%%%%%
%%            %%%%%%%%%%%%%%%%%%%%%%%%%%%%%%%%%%%%%%%%%%%%
%% Appendix D %%%%%%%%%%%%%%%%%%%%%%%%%%%%%%%%%%%%%%%%%%%%
%%            %%%%%%%%%%%%%%%%%%%%%%%%%%%%%%%%%%%%%%%%%%%%
%%%%%%%%%%%%%%%%%%%%%%%%%%%%%%%%%%%%%%%%%%%%%%%%%%%%%%%%%%
\chapter{システム変数}
%!TEX root = ../Mould_Analytical_Calculation_Note.tex


%%%%%%%%%%%%%%%%%%%%%%%%%%%%%%%%%%%%%%%%%%%%%%%%%%%%%%%%%%
%% section E.1 %%%%%%%%%%%%%%%%%%%%%%%%%%%%%%%%%%%%%%%%%%%
%%%%%%%%%%%%%%%%%%%%%%%%%%%%%%%%%%%%%%%%%%%%%%%%%%%%%%%%%%
\section{\DMname}
\begin{twoCtable}{}
\#1000 & パレット\#~~0:\#1, 1:\#2\\\hline
\#1004 & タッチセンサー電源~~0: off, 1: on\\\hline
\#1005 & タッチセンサー電池残量~~0: ok, 1: low\\\hline
\#2000+xxx & 工具長補正 \#xxx補正量(摩耗0とした値, xxx=1-200)\\\hline
\#2200+xxx & 工具長補正 \#xxx摩耗 (xxx=1-200)\\\hline
\#2400+xxx & 工具径補正 \#xxx補正量(摩耗0とした値, xxx=1-200)\\\hline
\#2600+xxx & 工具径補正 \#xxx摩耗 (xxx=1-200)\\\hline
\#3000 & アラーム\\\hline
\#3011 & 現在の年月日(yyyymmdd)\\\hline
\#4012 & 現在のワーク座標系\# (G\#)\\\hline
\#4107 & 直前の工具径補正コード\# (D\#)\\\hline
\#4111 & 直前の工具長補正コード\# (H\#)\\\hline
\#4113 & 直前のブロック指令 Mコード\# (M\#)\\\hline
\#4114 & 直前のブロック指令 シーケンス\# (N\#)\\\hline
\#4115 & 直前のブロック指令 プログラム\# (O\#)\\\hline
\#4120 & 直前のブロック指令 工具コード\# (T\#)\\\hline
\#500x\footnote{\#500x:工具補正値が加味された値(ワーク座標系に表示される数値)。途中でスキップがオンになったときはそのときの値。}
       & ブロック終点位置(現在のワーク座標系)1:X, 2:Y, 3:Z, 4:B, 5:C\\\hline
\#502x & 現在の機械座標系の座標 1:X, 2:Y, 3:Z, 4:B, 5:C\\\hline
\#504x & 現在のワーク座標系の座標 1:X, 2:Y, 3:Z, 4:B, 5:C\\\hline
\#506x & スキップ座標 1:X, 2:Y, 3:Z, 4:B, 5:C(工具補正0とした値)\\\hline
\#522x & ワーク座標系G54原点の機械座標 1:X, 2:Y, 3:Z, 4:B, 5:C\\\hline
\#524x & ワーク座標系G55原点の機械座標 1:X, 2:Y, 3:Z, 4:B, 5:C\\\hline
\#526x & ワーク座標系G56原点の機械座標 1:X, 2:Y, 3:Z, 4:B, 5:C\\\hline
\#528x & ワーク座標系G57原点の機械座標 1:X, 2:Y, 3:Z, 4:B, 5:C\\\hline
\#530x & ワーク座標系G58原点の機械座標 1:X, 2:Y, 3:Z, 4:B, 5:C\\\hline
\#532x & ワーク座標系G59原点の機械座標 1:X, 2:Y, 3:Z, 4:B, 5:C\\\hline
\#10000+xxx & 工具長補正 \#xxx補正量(摩耗0とした値, xxx=1-200)
\end{twoCtable}



%\clearpage
%%%%%%%%%%%%%%%%%%%%%%%%%%%%%%%%%%%%%%%%%%%%%%%%%%%%%%%%%%
%% section C.2 %%%%%%%%%%%%%%%%%%%%%%%%%%%%%%%%%%%%%%%%%%%
%%%%%%%%%%%%%%%%%%%%%%%%%%%%%%%%%%%%%%%%%%%%%%%%%%%%%%%%%%
\section{\MMname}
\begin{twoCtable}{}
\#4111 & 直前の工具長補正 Hコード\#\\\hline
\#4120 & 直前の工具 Tコード\#\\\hline
\#502x & 現在の機械座標系の座標 1:X, 2:Y, 3:Z, 4:B\\\hline
\#504x & 現在のワーク座標系の座標 1:X, 2:Y, 3:Z, 4:B\\\hline
\#506x & スキップ座標 1:X, 2:Y, 3:Z, 4:B(工具補正0とした値)\\\hline
\#522x & ワーク座標系G54原点の機械座標 1:X, 2:Y, 3:Z, 4:B\\\hline
\#524x & ワーク座標系G55原点の機械座標 1:X, 2:Y, 3:Z, 4:B\\\hline
\#526x & ワーク座標系G56原点の機械座標 1:X, 2:Y, 3:Z, 4:B\\\hline
\#528x & ワーク座標系G57原点の機械座標 1:X, 2:Y, 3:Z, 4:B\\\hline
\#530x & ワーク座標系G58原点の機械座標 1:X, 2:Y, 3:Z, 4:B\\\hline
\#532x & ワーク座標系G59原点の機械座標 1:X, 2:Y, 3:Z, 4:B\\
\end{twoCtable}







%%%%%%%%%%%%%%%%%%%%%%%%%%%%%%%%%%%%%%%%%%%%%%%%%%%%%%%%%%
%%            %%%%%%%%%%%%%%%%%%%%%%%%%%%%%%%%%%%%%%%%%%%%
%% Appendix E %%%%%%%%%%%%%%%%%%%%%%%%%%%%%%%%%%%%%%%%%%%%
%%            %%%%%%%%%%%%%%%%%%%%%%%%%%%%%%%%%%%%%%%%%%%%
%%%%%%%%%%%%%%%%%%%%%%%%%%%%%%%%%%%%%%%%%%%%%%%%%%%%%%%%%%
%\chapter{諸公式}
%%!TEX root = ../Mould_Analytical_Calculation_Note.tex





%%%%%%%%%%%%%%%%%%%%%%%%%%%%%%%%%%%%%%%%%%%%%%%%%%%%%%%%%%
%% section C.1 %%%%%%%%%%%%%%%%%%%%%%%%%%%%%%%%%%%%%%%%%%%
%%%%%%%%%%%%%%%%%%%%%%%%%%%%%%%%%%%%%%%%%%%%%%%%%%%%%%%%%%
\section{2点間の距離}
\begin{Column}{}
点($p$, $q$)と直線$ax+by+c=0$との距離$d$は、以下で与えられる。
\begin{align*}
  d = \frac{|ap+bq+c|}{\sqrt{a^2+b^2}}.
\end{align*}
\end{Column}
\begin{Column}{}
点$\boldsymbol p$を通り方向ベクトルが$\boldsymbol m$の直線L上の点と、点$\boldsymbol q$を通り方向ベクトルが$\boldsymbol m'$の直線$\mathrm L'$上の点は、それぞれパラメータ$t$, $t'$を用いて、
\begin{align*}
  \mathrm L: \boldsymbol p+t\boldsymbol m\ , \qquad
  \mathrm L': \boldsymbol q+t'\boldsymbol m'
\end{align*}
で表される。
このとき、L上の点の中で$\mathrm L'$に最も近づく点の位置$\boldsymbol k$は、以下で与えられる
%% footnote %%%%%%%%%%%%%%%%%%%%%
\footnote{2点間の距離の2乗$|\boldsymbol p-\boldsymbol q+t\boldsymbol m-t'\boldsymbol m'|^2$に対し、それぞれのパラメータ$t$, $t'$に関する微分が0となる。
それらを連立して解けば$\boldsymbol k$, $\boldsymbol k'$が求まる。}。
%%%%%%%%%%%%%%%%%%%%%%%%%%%%%%%%%
$\mathrm L'$上の点の中でLに最も近づく点の位置$\boldsymbol k'$についても同様である。
\begin{align*}
  \boldsymbol k
  = \boldsymbol p
    +\frac{(\boldsymbol m-(\boldsymbol m, \boldsymbol m')\boldsymbol m', \boldsymbol p-\boldsymbol q)}
          {1+(\boldsymbol m, \boldsymbol m')^2}\boldsymbol m
\end{align*}
また、これらの差の大きさ$|\boldsymbol k-\boldsymbol k'|$から、2直線間の距離$d$が求まる。
\end{Column}





%%%%%%%%%%%%%%%%%%%%%%%%%%%%%%%%%%%%%%%%%%%%%%%%%%%%%%%%%%
%%            %%%%%%%%%%%%%%%%%%%%%%%%%%%%%%%%%%%%%%%%%%%%
%% Appendix F %%%%%%%%%%%%%%%%%%%%%%%%%%%%%%%%%%%%%%%%%%%%
%%            %%%%%%%%%%%%%%%%%%%%%%%%%%%%%%%%%%%%%%%%%%%%
%%%%%%%%%%%%%%%%%%%%%%%%%%%%%%%%%%%%%%%%%%%%%%%%%%%%%%%%%%
\chapter{バンドルのNCプログラム}
%!TEX root = ../Mould_Analytical_Calculation_Note.tex

ここでは機械設置時にバンドルで付属していたNCプログラムを記載する。


%%%%%%%%%%%%%%%%%%%%%%%%%%%%%%%%%%%%%%%%%%%%%%%%%%%%%%%%%%
%% section E.1 %%%%%%%%%%%%%%%%%%%%%%%%%%%%%%%%%%%%%%%%%%%
%%%%%%%%%%%%%%%%%%%%%%%%%%%%%%%%%%%%%%%%%%%%%%%%%%%%%%%%%%
\section{O7000-O9393}

%%%%%%%%%%%%%%%%%%%%%%%%%%%%%%%%%%%%%%%%%%%%%%%%%%%%%%%%%%
%% subsection E.1.1 %%%%%%%%%%%%%%%%%%%%%%%%%%%%%%%%%%%%%%
%%%%%%%%%%%%%%%%%%%%%%%%%%%%%%%%%%%%%%%%%%%%%%%%%%%%%%%%%%
\subsection{O7000:parameter set}
\lstinputlisting{../Mould_Machining_Programs/bundle_programs_k/O7000.nc}


\clearpage
%%%%%%%%%%%%%%%%%%%%%%%%%%%%%%%%%%%%%%%%%%%%%%%%%%%%%%%%%%
%% subsection E.1.1 %%%%%%%%%%%%%%%%%%%%%%%%%%%%%%%%%%%%%%
%%%%%%%%%%%%%%%%%%%%%%%%%%%%%%%%%%%%%%%%%%%%%%%%%%%%%%%%%%
\subsection{O7100:}
\lstinputlisting{../Mould_Machining_Programs/bundle_programs_k/O7100.nc}


%\clearpage
%%%%%%%%%%%%%%%%%%%%%%%%%%%%%%%%%%%%%%%%%%%%%%%%%%%%%%%%%%
%% subsection E.1.1 %%%%%%%%%%%%%%%%%%%%%%%%%%%%%%%%%%%%%%
%%%%%%%%%%%%%%%%%%%%%%%%%%%%%%%%%%%%%%%%%%%%%%%%%%%%%%%%%%
\subsection{O7101:}
\lstinputlisting{../Mould_Machining_Programs/bundle_programs_k/O7101.nc}


\clearpage
%%%%%%%%%%%%%%%%%%%%%%%%%%%%%%%%%%%%%%%%%%%%%%%%%%%%%%%%%%
%% subsection E.1.1 %%%%%%%%%%%%%%%%%%%%%%%%%%%%%%%%%%%%%%
%%%%%%%%%%%%%%%%%%%%%%%%%%%%%%%%%%%%%%%%%%%%%%%%%%%%%%%%%%
\subsection{O7102:}
\lstinputlisting{../Mould_Machining_Programs/bundle_programs_k/O7102.nc}


%\clearpage
%%%%%%%%%%%%%%%%%%%%%%%%%%%%%%%%%%%%%%%%%%%%%%%%%%%%%%%%%%
%% subsection E.1.1 %%%%%%%%%%%%%%%%%%%%%%%%%%%%%%%%%%%%%%
%%%%%%%%%%%%%%%%%%%%%%%%%%%%%%%%%%%%%%%%%%%%%%%%%%%%%%%%%%
\subsection{O7103:}
\lstinputlisting{../Mould_Machining_Programs/bundle_programs_k/O7103.nc}


\clearpage
%%%%%%%%%%%%%%%%%%%%%%%%%%%%%%%%%%%%%%%%%%%%%%%%%%%%%%%%%%
%% subsection E.1.1 %%%%%%%%%%%%%%%%%%%%%%%%%%%%%%%%%%%%%%
%%%%%%%%%%%%%%%%%%%%%%%%%%%%%%%%%%%%%%%%%%%%%%%%%%%%%%%%%%
\subsection{O7300:}
\lstinputlisting{../Mould_Machining_Programs/bundle_programs_k/O7300.nc}


%\clearpage
%%%%%%%%%%%%%%%%%%%%%%%%%%%%%%%%%%%%%%%%%%%%%%%%%%%%%%%%%%
%% subsection E.1.1 %%%%%%%%%%%%%%%%%%%%%%%%%%%%%%%%%%%%%%
%%%%%%%%%%%%%%%%%%%%%%%%%%%%%%%%%%%%%%%%%%%%%%%%%%%%%%%%%%
\subsection{O7301:}
\lstinputlisting{../Mould_Machining_Programs/bundle_programs_k/O7301.nc}


\clearpage
%%%%%%%%%%%%%%%%%%%%%%%%%%%%%%%%%%%%%%%%%%%%%%%%%%%%%%%%%%
%% subsection E.1.1 %%%%%%%%%%%%%%%%%%%%%%%%%%%%%%%%%%%%%%
%%%%%%%%%%%%%%%%%%%%%%%%%%%%%%%%%%%%%%%%%%%%%%%%%%%%%%%%%%
\subsection{O7302:}
\lstinputlisting{../Mould_Machining_Programs/bundle_programs_k/O7302.nc}


%\clearpage
%%%%%%%%%%%%%%%%%%%%%%%%%%%%%%%%%%%%%%%%%%%%%%%%%%%%%%%%%%
%% subsection E.1.1 %%%%%%%%%%%%%%%%%%%%%%%%%%%%%%%%%%%%%%
%%%%%%%%%%%%%%%%%%%%%%%%%%%%%%%%%%%%%%%%%%%%%%%%%%%%%%%%%%
\subsection{O7303:}
\lstinputlisting{../Mould_Machining_Programs/bundle_programs_k/O7303.nc}


%\clearpage
%%%%%%%%%%%%%%%%%%%%%%%%%%%%%%%%%%%%%%%%%%%%%%%%%%%%%%%%%%
%% subsection E.1.1 %%%%%%%%%%%%%%%%%%%%%%%%%%%%%%%%%%%%%%
%%%%%%%%%%%%%%%%%%%%%%%%%%%%%%%%%%%%%%%%%%%%%%%%%%%%%%%%%%
\subsection{O7310:}
\lstinputlisting{../Mould_Machining_Programs/bundle_programs_k/O7310.nc}


\clearpage
%%%%%%%%%%%%%%%%%%%%%%%%%%%%%%%%%%%%%%%%%%%%%%%%%%%%%%%%%%
%% subsection E.1.1 %%%%%%%%%%%%%%%%%%%%%%%%%%%%%%%%%%%%%%
%%%%%%%%%%%%%%%%%%%%%%%%%%%%%%%%%%%%%%%%%%%%%%%%%%%%%%%%%%
\subsection{O7311:}
\lstinputlisting{../Mould_Machining_Programs/bundle_programs_k/O7311.nc}


%\clearpage
%%%%%%%%%%%%%%%%%%%%%%%%%%%%%%%%%%%%%%%%%%%%%%%%%%%%%%%%%%
%% subsection E.1.1 %%%%%%%%%%%%%%%%%%%%%%%%%%%%%%%%%%%%%%
%%%%%%%%%%%%%%%%%%%%%%%%%%%%%%%%%%%%%%%%%%%%%%%%%%%%%%%%%%
\subsection{O7312:}
\lstinputlisting{../Mould_Machining_Programs/bundle_programs_k/O7312.nc}


%\clearpage
%%%%%%%%%%%%%%%%%%%%%%%%%%%%%%%%%%%%%%%%%%%%%%%%%%%%%%%%%%
%% subsection E.1.1 %%%%%%%%%%%%%%%%%%%%%%%%%%%%%%%%%%%%%%
%%%%%%%%%%%%%%%%%%%%%%%%%%%%%%%%%%%%%%%%%%%%%%%%%%%%%%%%%%
\subsection[O7313:\texorpdfstring{$Z$} axis coordinate set]{O7313:$\textbf Z$ axis coordinate set}
\lstinputlisting{../Mould_Machining_Programs/bundle_programs_k/O7313.nc}



\clearpage
%%%%%%%%%%%%%%%%%%%%%%%%%%%%%%%%%%%%%%%%%%%%%%%%%%%%%%%%%%
%% section E.2 %%%%%%%%%%%%%%%%%%%%%%%%%%%%%%%%%%%%%%%%%%%
%%%%%%%%%%%%%%%%%%%%%%%%%%%%%%%%%%%%%%%%%%%%%%%%%%%%%%%%%%
\section{O8123-8999}


%%%%%%%%%%%%%%%%%%%%%%%%%%%%%%%%%%%%%%%%%%%%%%%%%%%%%%%%%%
%% subsection E.1.1 %%%%%%%%%%%%%%%%%%%%%%%%%%%%%%%%%%%%%%
%%%%%%%%%%%%%%%%%%%%%%%%%%%%%%%%%%%%%%%%%%%%%%%%%%%%%%%%%%
\subsection{O8123:}
\lstinputlisting{../Mould_Machining_Programs/bundle_programs_k/O8123.nc}


\clearpage
%%%%%%%%%%%%%%%%%%%%%%%%%%%%%%%%%%%%%%%%%%%%%%%%%%%%%%%%%%
%% subsection E.1.1 %%%%%%%%%%%%%%%%%%%%%%%%%%%%%%%%%%%%%%
%%%%%%%%%%%%%%%%%%%%%%%%%%%%%%%%%%%%%%%%%%%%%%%%%%%%%%%%%%
\subsection{O8898:}
\lstinputlisting{../Mould_Machining_Programs/bundle_programs_k/O8898.nc}


%\clearpage
%%%%%%%%%%%%%%%%%%%%%%%%%%%%%%%%%%%%%%%%%%%%%%%%%%%%%%%%%%
%% subsection E.1.1 %%%%%%%%%%%%%%%%%%%%%%%%%%%%%%%%%%%%%%
%%%%%%%%%%%%%%%%%%%%%%%%%%%%%%%%%%%%%%%%%%%%%%%%%%%%%%%%%%
\subsection{O8899:}
\lstinputlisting{../Mould_Machining_Programs/bundle_programs_k/O8899.nc}


\clearpage
%%%%%%%%%%%%%%%%%%%%%%%%%%%%%%%%%%%%%%%%%%%%%%%%%%%%%%%%%%
%% subsection E.1.1 %%%%%%%%%%%%%%%%%%%%%%%%%%%%%%%%%%%%%%
%%%%%%%%%%%%%%%%%%%%%%%%%%%%%%%%%%%%%%%%%%%%%%%%%%%%%%%%%%
\subsection{O8998:}
\lstinputlisting{../Mould_Machining_Programs/bundle_programs_k/O8998.nc}


%\clearpage
%%%%%%%%%%%%%%%%%%%%%%%%%%%%%%%%%%%%%%%%%%%%%%%%%%%%%%%%%%
%% subsection E.1.1 %%%%%%%%%%%%%%%%%%%%%%%%%%%%%%%%%%%%%%
%%%%%%%%%%%%%%%%%%%%%%%%%%%%%%%%%%%%%%%%%%%%%%%%%%%%%%%%%%
\subsection{O8999:}
\lstinputlisting{../Mould_Machining_Programs/bundle_programs_k/O8999.nc}



\clearpage
%%%%%%%%%%%%%%%%%%%%%%%%%%%%%%%%%%%%%%%%%%%%%%%%%%%%%%%%%%
%% section E.2 %%%%%%%%%%%%%%%%%%%%%%%%%%%%%%%%%%%%%%%%%%%
%%%%%%%%%%%%%%%%%%%%%%%%%%%%%%%%%%%%%%%%%%%%%%%%%%%%%%%%%%
\section{O9001-9921}


%%%%%%%%%%%%%%%%%%%%%%%%%%%%%%%%%%%%%%%%%%%%%%%%%%%%%%%%%%
%% subsection E.1.1 %%%%%%%%%%%%%%%%%%%%%%%%%%%%%%%%%%%%%%
%%%%%%%%%%%%%%%%%%%%%%%%%%%%%%%%%%%%%%%%%%%%%%%%%%%%%%%%%%
\subsection{O9001:}
\lstinputlisting{../Mould_Machining_Programs/bundle_programs_k/O9001.nc}


%\clearpage
%%%%%%%%%%%%%%%%%%%%%%%%%%%%%%%%%%%%%%%%%%%%%%%%%%%%%%%%%%
%% subsection E.1.1 %%%%%%%%%%%%%%%%%%%%%%%%%%%%%%%%%%%%%%
%%%%%%%%%%%%%%%%%%%%%%%%%%%%%%%%%%%%%%%%%%%%%%%%%%%%%%%%%%
\subsection{O9002:}
\lstinputlisting{../Mould_Machining_Programs/bundle_programs_k/O9002.nc}


\clearpage
%%%%%%%%%%%%%%%%%%%%%%%%%%%%%%%%%%%%%%%%%%%%%%%%%%%%%%%%%%
%% subsection E.1.1 %%%%%%%%%%%%%%%%%%%%%%%%%%%%%%%%%%%%%%
%%%%%%%%%%%%%%%%%%%%%%%%%%%%%%%%%%%%%%%%%%%%%%%%%%%%%%%%%%
\subsection{O9003:}
\lstinputlisting{../Mould_Machining_Programs/bundle_programs_k/O9003.nc}


%\clearpage
%%%%%%%%%%%%%%%%%%%%%%%%%%%%%%%%%%%%%%%%%%%%%%%%%%%%%%%%%%
%% subsection E.1.1 %%%%%%%%%%%%%%%%%%%%%%%%%%%%%%%%%%%%%%
%%%%%%%%%%%%%%%%%%%%%%%%%%%%%%%%%%%%%%%%%%%%%%%%%%%%%%%%%%
\subsection{O9006:}
\lstinputlisting{../Mould_Machining_Programs/bundle_programs_k/O9006.nc}


\clearpage
%%%%%%%%%%%%%%%%%%%%%%%%%%%%%%%%%%%%%%%%%%%%%%%%%%%%%%%%%%
%% subsection E.1.1 %%%%%%%%%%%%%%%%%%%%%%%%%%%%%%%%%%%%%%
%%%%%%%%%%%%%%%%%%%%%%%%%%%%%%%%%%%%%%%%%%%%%%%%%%%%%%%%%%
\subsection{O9020:}
\lstinputlisting{../Mould_Machining_Programs/bundle_programs_k/O9020.nc}


%\clearpage
%%%%%%%%%%%%%%%%%%%%%%%%%%%%%%%%%%%%%%%%%%%%%%%%%%%%%%%%%%
%% subsection E.1.1 %%%%%%%%%%%%%%%%%%%%%%%%%%%%%%%%%%%%%%
%%%%%%%%%%%%%%%%%%%%%%%%%%%%%%%%%%%%%%%%%%%%%%%%%%%%%%%%%%
\subsection{O9021:}
\lstinputlisting{../Mould_Machining_Programs/bundle_programs_k/O9021.nc}


\clearpage
%%%%%%%%%%%%%%%%%%%%%%%%%%%%%%%%%%%%%%%%%%%%%%%%%%%%%%%%%%
%% subsection E.1.1 %%%%%%%%%%%%%%%%%%%%%%%%%%%%%%%%%%%%%%
%%%%%%%%%%%%%%%%%%%%%%%%%%%%%%%%%%%%%%%%%%%%%%%%%%%%%%%%%%
\subsection{O9022:}
\lstinputlisting{../Mould_Machining_Programs/bundle_programs_k/O9022.nc}


\clearpage
%%%%%%%%%%%%%%%%%%%%%%%%%%%%%%%%%%%%%%%%%%%%%%%%%%%%%%%%%%
%% subsection E.1.1 %%%%%%%%%%%%%%%%%%%%%%%%%%%%%%%%%%%%%%
%%%%%%%%%%%%%%%%%%%%%%%%%%%%%%%%%%%%%%%%%%%%%%%%%%%%%%%%%%
\subsection{O9100:}
\lstinputlisting{../Mould_Machining_Programs/bundle_programs_k/O9100.nc}


\clearpage
%%%%%%%%%%%%%%%%%%%%%%%%%%%%%%%%%%%%%%%%%%%%%%%%%%%%%%%%%%
%% subsection E.1.1 %%%%%%%%%%%%%%%%%%%%%%%%%%%%%%%%%%%%%%
%%%%%%%%%%%%%%%%%%%%%%%%%%%%%%%%%%%%%%%%%%%%%%%%%%%%%%%%%%
\subsection{O9101:}
\lstinputlisting{../Mould_Machining_Programs/bundle_programs_k/O9101.nc}


\clearpage
%%%%%%%%%%%%%%%%%%%%%%%%%%%%%%%%%%%%%%%%%%%%%%%%%%%%%%%%%%
%% subsection E.1.1 %%%%%%%%%%%%%%%%%%%%%%%%%%%%%%%%%%%%%%
%%%%%%%%%%%%%%%%%%%%%%%%%%%%%%%%%%%%%%%%%%%%%%%%%%%%%%%%%%
\subsection{O9102:}
\lstinputlisting{../Mould_Machining_Programs/bundle_programs_k/O9102.nc}


\clearpage
%%%%%%%%%%%%%%%%%%%%%%%%%%%%%%%%%%%%%%%%%%%%%%%%%%%%%%%%%%
%% subsection E.1.1 %%%%%%%%%%%%%%%%%%%%%%%%%%%%%%%%%%%%%%
%%%%%%%%%%%%%%%%%%%%%%%%%%%%%%%%%%%%%%%%%%%%%%%%%%%%%%%%%%
\subsection{O9103:}
\lstinputlisting{../Mould_Machining_Programs/bundle_programs_k/O9103.nc}


\clearpage
%%%%%%%%%%%%%%%%%%%%%%%%%%%%%%%%%%%%%%%%%%%%%%%%%%%%%%%%%%
%% subsection E.1.1 %%%%%%%%%%%%%%%%%%%%%%%%%%%%%%%%%%%%%%
%%%%%%%%%%%%%%%%%%%%%%%%%%%%%%%%%%%%%%%%%%%%%%%%%%%%%%%%%%
\subsection{O9200:}
\lstinputlisting{../Mould_Machining_Programs/bundle_programs_k/O9200.nc}


\clearpage
%%%%%%%%%%%%%%%%%%%%%%%%%%%%%%%%%%%%%%%%%%%%%%%%%%%%%%%%%%
%% subsection E.1.1 %%%%%%%%%%%%%%%%%%%%%%%%%%%%%%%%%%%%%%
%%%%%%%%%%%%%%%%%%%%%%%%%%%%%%%%%%%%%%%%%%%%%%%%%%%%%%%%%%
\subsection{O9300:}
\lstinputlisting{../Mould_Machining_Programs/bundle_programs_k/O9300.nc}


\clearpage
%%%%%%%%%%%%%%%%%%%%%%%%%%%%%%%%%%%%%%%%%%%%%%%%%%%%%%%%%%
%% subsection E.1.1 %%%%%%%%%%%%%%%%%%%%%%%%%%%%%%%%%%%%%%
%%%%%%%%%%%%%%%%%%%%%%%%%%%%%%%%%%%%%%%%%%%%%%%%%%%%%%%%%%
\subsection{O9301:}
\lstinputlisting{../Mould_Machining_Programs/bundle_programs_k/O9301.nc}


\clearpage
%%%%%%%%%%%%%%%%%%%%%%%%%%%%%%%%%%%%%%%%%%%%%%%%%%%%%%%%%%
%% subsection E.1.1 %%%%%%%%%%%%%%%%%%%%%%%%%%%%%%%%%%%%%%
%%%%%%%%%%%%%%%%%%%%%%%%%%%%%%%%%%%%%%%%%%%%%%%%%%%%%%%%%%
\subsection{O9302:}
\lstinputlisting{../Mould_Machining_Programs/bundle_programs_k/O9302.nc}


\clearpage
%%%%%%%%%%%%%%%%%%%%%%%%%%%%%%%%%%%%%%%%%%%%%%%%%%%%%%%%%%
%% subsection E.1.1 %%%%%%%%%%%%%%%%%%%%%%%%%%%%%%%%%%%%%%
%%%%%%%%%%%%%%%%%%%%%%%%%%%%%%%%%%%%%%%%%%%%%%%%%%%%%%%%%%
\subsection{O9303:}
\lstinputlisting{../Mould_Machining_Programs/bundle_programs_k/O9303.nc}


\clearpage
%%%%%%%%%%%%%%%%%%%%%%%%%%%%%%%%%%%%%%%%%%%%%%%%%%%%%%%%%%
%% subsection E.1.1 %%%%%%%%%%%%%%%%%%%%%%%%%%%%%%%%%%%%%%
%%%%%%%%%%%%%%%%%%%%%%%%%%%%%%%%%%%%%%%%%%%%%%%%%%%%%%%%%%
\subsection{O9310:}
\lstinputlisting{../Mould_Machining_Programs/bundle_programs_k/O9310.nc}


\clearpage
%%%%%%%%%%%%%%%%%%%%%%%%%%%%%%%%%%%%%%%%%%%%%%%%%%%%%%%%%%
%% subsection E.1.1 %%%%%%%%%%%%%%%%%%%%%%%%%%%%%%%%%%%%%%
%%%%%%%%%%%%%%%%%%%%%%%%%%%%%%%%%%%%%%%%%%%%%%%%%%%%%%%%%%
\subsection{O9311:}
\lstinputlisting{../Mould_Machining_Programs/bundle_programs_k/O9311.nc}


\clearpage
%%%%%%%%%%%%%%%%%%%%%%%%%%%%%%%%%%%%%%%%%%%%%%%%%%%%%%%%%%
%% subsection E.1.1 %%%%%%%%%%%%%%%%%%%%%%%%%%%%%%%%%%%%%%
%%%%%%%%%%%%%%%%%%%%%%%%%%%%%%%%%%%%%%%%%%%%%%%%%%%%%%%%%%
\subsection{O9312:}
\lstinputlisting{../Mould_Machining_Programs/bundle_programs_k/O9312.nc}


\clearpage
%%%%%%%%%%%%%%%%%%%%%%%%%%%%%%%%%%%%%%%%%%%%%%%%%%%%%%%%%%
%% subsection E.1.1 %%%%%%%%%%%%%%%%%%%%%%%%%%%%%%%%%%%%%%
%%%%%%%%%%%%%%%%%%%%%%%%%%%%%%%%%%%%%%%%%%%%%%%%%%%%%%%%%%
\subsection{O9313:}
\lstinputlisting{../Mould_Machining_Programs/bundle_programs_k/O9313.nc}


\clearpage
%%%%%%%%%%%%%%%%%%%%%%%%%%%%%%%%%%%%%%%%%%%%%%%%%%%%%%%%%%
%% subsection E.1.1 %%%%%%%%%%%%%%%%%%%%%%%%%%%%%%%%%%%%%%
%%%%%%%%%%%%%%%%%%%%%%%%%%%%%%%%%%%%%%%%%%%%%%%%%%%%%%%%%%
\subsection{O9390:}
\lstinputlisting{../Mould_Machining_Programs/bundle_programs_k/O9390.nc}


\clearpage
%%%%%%%%%%%%%%%%%%%%%%%%%%%%%%%%%%%%%%%%%%%%%%%%%%%%%%%%%%
%% subsection E.1.1 %%%%%%%%%%%%%%%%%%%%%%%%%%%%%%%%%%%%%%
%%%%%%%%%%%%%%%%%%%%%%%%%%%%%%%%%%%%%%%%%%%%%%%%%%%%%%%%%%
\subsection{O9391:}
\lstinputlisting{../Mould_Machining_Programs/bundle_programs_k/O9391.nc}


%\clearpage
%%%%%%%%%%%%%%%%%%%%%%%%%%%%%%%%%%%%%%%%%%%%%%%%%%%%%%%%%%
%% subsection E.1.1 %%%%%%%%%%%%%%%%%%%%%%%%%%%%%%%%%%%%%%
%%%%%%%%%%%%%%%%%%%%%%%%%%%%%%%%%%%%%%%%%%%%%%%%%%%%%%%%%%
\subsection{O9392:}
\lstinputlisting{../Mould_Machining_Programs/bundle_programs_k/O9392.nc}


\clearpage
%%%%%%%%%%%%%%%%%%%%%%%%%%%%%%%%%%%%%%%%%%%%%%%%%%%%%%%%%%
%% subsection E.1.1 %%%%%%%%%%%%%%%%%%%%%%%%%%%%%%%%%%%%%%
%%%%%%%%%%%%%%%%%%%%%%%%%%%%%%%%%%%%%%%%%%%%%%%%%%%%%%%%%%
\subsection{O9393:}
\lstinputlisting{../Mould_Machining_Programs/bundle_programs_k/O9393.nc}





%%%%%%%%%%%%%%%%%%%%%%%%%%%%%%%%%%%%%%%%%%%%%%%%%%%%%%%%%%
%%            %%%%%%%%%%%%%%%%%%%%%%%%%%%%%%%%%%%%%%%%%%%%
%% Appendix G %%%%%%%%%%%%%%%%%%%%%%%%%%%%%%%%%%%%%%%%%%%%
%%            %%%%%%%%%%%%%%%%%%%%%%%%%%%%%%%%%%%%%%%%%%%%
%%%%%%%%%%%%%%%%%%%%%%%%%%%%%%%%%%%%%%%%%%%%%%%%%%%%%%%%%%
%\chapter{バンドルのNCプログラム}
%%!TEX root = ../Mould_Analytical_Calculation_Note.tex

ここでは機械設置時にバンドルで付属していたNCプログラムを記載する。


%%%%%%%%%%%%%%%%%%%%%%%%%%%%%%%%%%%%%%%%%%%%%%%%%%%%%%%%%%
%% section E.1 %%%%%%%%%%%%%%%%%%%%%%%%%%%%%%%%%%%%%%%%%%%
%%%%%%%%%%%%%%%%%%%%%%%%%%%%%%%%%%%%%%%%%%%%%%%%%%%%%%%%%%
\section{O9700-O9921}

%%%%%%%%%%%%%%%%%%%%%%%%%%%%%%%%%%%%%%%%%%%%%%%%%%%%%%%%%%
%% subsection E.1.1 %%%%%%%%%%%%%%%%%%%%%%%%%%%%%%%%%%%%%%
%%%%%%%%%%%%%%%%%%%%%%%%%%%%%%%%%%%%%%%%%%%%%%%%%%%%%%%%%%
\subsection{O9700:}
\lstinputlisting{../Mould_Machining_Programs/bundle_programs_r/O9700.nc}


\clearpage
%%%%%%%%%%%%%%%%%%%%%%%%%%%%%%%%%%%%%%%%%%%%%%%%%%%%%%%%%%
%% subsection E.1.1 %%%%%%%%%%%%%%%%%%%%%%%%%%%%%%%%%%%%%%
%%%%%%%%%%%%%%%%%%%%%%%%%%%%%%%%%%%%%%%%%%%%%%%%%%%%%%%%%%
\subsection{O9701:}
\lstinputlisting{../Mould_Machining_Programs/bundle_programs_r/O9701.nc}


\clearpage
%%%%%%%%%%%%%%%%%%%%%%%%%%%%%%%%%%%%%%%%%%%%%%%%%%%%%%%%%%
%% subsection E.1.1 %%%%%%%%%%%%%%%%%%%%%%%%%%%%%%%%%%%%%%
%%%%%%%%%%%%%%%%%%%%%%%%%%%%%%%%%%%%%%%%%%%%%%%%%%%%%%%%%%
\subsection{O9721:}
\lstinputlisting{../Mould_Machining_Programs/bundle_programs_r/O9721.nc}


\clearpage
%%%%%%%%%%%%%%%%%%%%%%%%%%%%%%%%%%%%%%%%%%%%%%%%%%%%%%%%%%
%% subsection E.1.1 %%%%%%%%%%%%%%%%%%%%%%%%%%%%%%%%%%%%%%
%%%%%%%%%%%%%%%%%%%%%%%%%%%%%%%%%%%%%%%%%%%%%%%%%%%%%%%%%%
\subsection{O9722:}
\lstinputlisting{../Mould_Machining_Programs/bundle_programs_r/O9722.nc}


%\clearpage
%%%%%%%%%%%%%%%%%%%%%%%%%%%%%%%%%%%%%%%%%%%%%%%%%%%%%%%%%%
%% subsection E.1.1 %%%%%%%%%%%%%%%%%%%%%%%%%%%%%%%%%%%%%%
%%%%%%%%%%%%%%%%%%%%%%%%%%%%%%%%%%%%%%%%%%%%%%%%%%%%%%%%%%
\subsection{O9723:}
\lstinputlisting{../Mould_Machining_Programs/bundle_programs_r/O9723.nc}


\clearpage
%%%%%%%%%%%%%%%%%%%%%%%%%%%%%%%%%%%%%%%%%%%%%%%%%%%%%%%%%%
%% subsection E.1.1 %%%%%%%%%%%%%%%%%%%%%%%%%%%%%%%%%%%%%%
%%%%%%%%%%%%%%%%%%%%%%%%%%%%%%%%%%%%%%%%%%%%%%%%%%%%%%%%%%
\subsection{O9724:}
\lstinputlisting{../Mould_Machining_Programs/bundle_programs_r/O9724.nc}


\clearpage
%%%%%%%%%%%%%%%%%%%%%%%%%%%%%%%%%%%%%%%%%%%%%%%%%%%%%%%%%%
%% subsection E.1.1 %%%%%%%%%%%%%%%%%%%%%%%%%%%%%%%%%%%%%%
%%%%%%%%%%%%%%%%%%%%%%%%%%%%%%%%%%%%%%%%%%%%%%%%%%%%%%%%%%
\subsection{O9725:}
\lstinputlisting{../Mould_Machining_Programs/bundle_programs_r/O9725.nc}


\clearpage
%%%%%%%%%%%%%%%%%%%%%%%%%%%%%%%%%%%%%%%%%%%%%%%%%%%%%%%%%%
%% subsection E.1.1 %%%%%%%%%%%%%%%%%%%%%%%%%%%%%%%%%%%%%%
%%%%%%%%%%%%%%%%%%%%%%%%%%%%%%%%%%%%%%%%%%%%%%%%%%%%%%%%%%
\subsection{O9726:}
\lstinputlisting{../Mould_Machining_Programs/bundle_programs_r/O9726.nc}


\clearpage
%%%%%%%%%%%%%%%%%%%%%%%%%%%%%%%%%%%%%%%%%%%%%%%%%%%%%%%%%%
%% subsection E.1.1 %%%%%%%%%%%%%%%%%%%%%%%%%%%%%%%%%%%%%%
%%%%%%%%%%%%%%%%%%%%%%%%%%%%%%%%%%%%%%%%%%%%%%%%%%%%%%%%%%
\subsection{O9727:}
\lstinputlisting{../Mould_Machining_Programs/bundle_programs_r/O9727.nc}


\clearpage
%%%%%%%%%%%%%%%%%%%%%%%%%%%%%%%%%%%%%%%%%%%%%%%%%%%%%%%%%%
%% subsection E.1.1 %%%%%%%%%%%%%%%%%%%%%%%%%%%%%%%%%%%%%%
%%%%%%%%%%%%%%%%%%%%%%%%%%%%%%%%%%%%%%%%%%%%%%%%%%%%%%%%%%
\subsection{O9729:}
\lstinputlisting{../Mould_Machining_Programs/bundle_programs_r/O9729.nc}


\clearpage
%%%%%%%%%%%%%%%%%%%%%%%%%%%%%%%%%%%%%%%%%%%%%%%%%%%%%%%%%%
%% subsection E.1.1 %%%%%%%%%%%%%%%%%%%%%%%%%%%%%%%%%%%%%%
%%%%%%%%%%%%%%%%%%%%%%%%%%%%%%%%%%%%%%%%%%%%%%%%%%%%%%%%%%
\subsection{O9730:}
\lstinputlisting{../Mould_Machining_Programs/bundle_programs_r/O9730.nc}


\clearpage
%%%%%%%%%%%%%%%%%%%%%%%%%%%%%%%%%%%%%%%%%%%%%%%%%%%%%%%%%%
%% subsection E.1.1 %%%%%%%%%%%%%%%%%%%%%%%%%%%%%%%%%%%%%%
%%%%%%%%%%%%%%%%%%%%%%%%%%%%%%%%%%%%%%%%%%%%%%%%%%%%%%%%%%
\subsection{O9731:}
\lstinputlisting{../Mould_Machining_Programs/bundle_programs_r/O9731.nc}


\clearpage
%%%%%%%%%%%%%%%%%%%%%%%%%%%%%%%%%%%%%%%%%%%%%%%%%%%%%%%%%%
%% subsection E.1.1 %%%%%%%%%%%%%%%%%%%%%%%%%%%%%%%%%%%%%%
%%%%%%%%%%%%%%%%%%%%%%%%%%%%%%%%%%%%%%%%%%%%%%%%%%%%%%%%%%
\subsection{O9732:}
\lstinputlisting{../Mould_Machining_Programs/bundle_programs_r/O9732.nc}


\clearpage
%%%%%%%%%%%%%%%%%%%%%%%%%%%%%%%%%%%%%%%%%%%%%%%%%%%%%%%%%%
%% subsection E.1.1 %%%%%%%%%%%%%%%%%%%%%%%%%%%%%%%%%%%%%%
%%%%%%%%%%%%%%%%%%%%%%%%%%%%%%%%%%%%%%%%%%%%%%%%%%%%%%%%%%
\subsection{O9735:}
\lstinputlisting{../Mould_Machining_Programs/bundle_programs_r/O9735.nc}


\clearpage
%%%%%%%%%%%%%%%%%%%%%%%%%%%%%%%%%%%%%%%%%%%%%%%%%%%%%%%%%%
%% subsection E.1.1 %%%%%%%%%%%%%%%%%%%%%%%%%%%%%%%%%%%%%%
%%%%%%%%%%%%%%%%%%%%%%%%%%%%%%%%%%%%%%%%%%%%%%%%%%%%%%%%%%
\subsection{O9750:}
\lstinputlisting{../Mould_Machining_Programs/bundle_programs_r/O9750.nc}


\clearpage
%%%%%%%%%%%%%%%%%%%%%%%%%%%%%%%%%%%%%%%%%%%%%%%%%%%%%%%%%%
%% subsection E.1.1 %%%%%%%%%%%%%%%%%%%%%%%%%%%%%%%%%%%%%%
%%%%%%%%%%%%%%%%%%%%%%%%%%%%%%%%%%%%%%%%%%%%%%%%%%%%%%%%%%
\subsection{O9751:}
\lstinputlisting{../Mould_Machining_Programs/bundle_programs_r/O9751.nc}


\clearpage
%%%%%%%%%%%%%%%%%%%%%%%%%%%%%%%%%%%%%%%%%%%%%%%%%%%%%%%%%%
%% subsection E.1.1 %%%%%%%%%%%%%%%%%%%%%%%%%%%%%%%%%%%%%%
%%%%%%%%%%%%%%%%%%%%%%%%%%%%%%%%%%%%%%%%%%%%%%%%%%%%%%%%%%
\subsection{O9752:}
\lstinputlisting{../Mould_Machining_Programs/bundle_programs_r/O9752.nc}


\clearpage
%%%%%%%%%%%%%%%%%%%%%%%%%%%%%%%%%%%%%%%%%%%%%%%%%%%%%%%%%%
%% subsection E.1.1 %%%%%%%%%%%%%%%%%%%%%%%%%%%%%%%%%%%%%%
%%%%%%%%%%%%%%%%%%%%%%%%%%%%%%%%%%%%%%%%%%%%%%%%%%%%%%%%%%
\subsection{O9753:}
\lstinputlisting{../Mould_Machining_Programs/bundle_programs_r/O9753.nc}


\clearpage
%%%%%%%%%%%%%%%%%%%%%%%%%%%%%%%%%%%%%%%%%%%%%%%%%%%%%%%%%%
%% subsection E.1.1 %%%%%%%%%%%%%%%%%%%%%%%%%%%%%%%%%%%%%%
%%%%%%%%%%%%%%%%%%%%%%%%%%%%%%%%%%%%%%%%%%%%%%%%%%%%%%%%%%
\subsection{O9754:}
\lstinputlisting{../Mould_Machining_Programs/bundle_programs_r/O9754.nc}


\clearpage
%%%%%%%%%%%%%%%%%%%%%%%%%%%%%%%%%%%%%%%%%%%%%%%%%%%%%%%%%%
%% subsection E.1.1 %%%%%%%%%%%%%%%%%%%%%%%%%%%%%%%%%%%%%%
%%%%%%%%%%%%%%%%%%%%%%%%%%%%%%%%%%%%%%%%%%%%%%%%%%%%%%%%%%
\subsection{O9755:}
\lstinputlisting{../Mould_Machining_Programs/bundle_programs_r/O9755.nc}


\clearpage
%%%%%%%%%%%%%%%%%%%%%%%%%%%%%%%%%%%%%%%%%%%%%%%%%%%%%%%%%%
%% subsection E.1.1 %%%%%%%%%%%%%%%%%%%%%%%%%%%%%%%%%%%%%%
%%%%%%%%%%%%%%%%%%%%%%%%%%%%%%%%%%%%%%%%%%%%%%%%%%%%%%%%%%
\subsection{O9759:}
\lstinputlisting{../Mould_Machining_Programs/bundle_programs_r/O9759.nc}


\clearpage
%%%%%%%%%%%%%%%%%%%%%%%%%%%%%%%%%%%%%%%%%%%%%%%%%%%%%%%%%%
%% subsection E.1.1 %%%%%%%%%%%%%%%%%%%%%%%%%%%%%%%%%%%%%%
%%%%%%%%%%%%%%%%%%%%%%%%%%%%%%%%%%%%%%%%%%%%%%%%%%%%%%%%%%
\subsection{O9773:}
\lstinputlisting{../Mould_Machining_Programs/bundle_programs_r/O9773.nc}


\clearpage
%%%%%%%%%%%%%%%%%%%%%%%%%%%%%%%%%%%%%%%%%%%%%%%%%%%%%%%%%%
%% subsection E.1.1 %%%%%%%%%%%%%%%%%%%%%%%%%%%%%%%%%%%%%%
%%%%%%%%%%%%%%%%%%%%%%%%%%%%%%%%%%%%%%%%%%%%%%%%%%%%%%%%%%
\subsection{O9800:}
\lstinputlisting{../Mould_Machining_Programs/bundle_programs_r/O9800.nc}


\clearpage
%%%%%%%%%%%%%%%%%%%%%%%%%%%%%%%%%%%%%%%%%%%%%%%%%%%%%%%%%%
%% subsection E.1.1 %%%%%%%%%%%%%%%%%%%%%%%%%%%%%%%%%%%%%%
%%%%%%%%%%%%%%%%%%%%%%%%%%%%%%%%%%%%%%%%%%%%%%%%%%%%%%%%%%
\subsection{O9801:}
\lstinputlisting{../Mould_Machining_Programs/bundle_programs_r/O9801.nc}


\clearpage
%%%%%%%%%%%%%%%%%%%%%%%%%%%%%%%%%%%%%%%%%%%%%%%%%%%%%%%%%%
%% subsection E.1.1 %%%%%%%%%%%%%%%%%%%%%%%%%%%%%%%%%%%%%%
%%%%%%%%%%%%%%%%%%%%%%%%%%%%%%%%%%%%%%%%%%%%%%%%%%%%%%%%%%
\subsection{O9810:}
\lstinputlisting{../Mould_Machining_Programs/bundle_programs_r/O9810.nc}


\clearpage
%%%%%%%%%%%%%%%%%%%%%%%%%%%%%%%%%%%%%%%%%%%%%%%%%%%%%%%%%%
%% subsection E.1.1 %%%%%%%%%%%%%%%%%%%%%%%%%%%%%%%%%%%%%%
%%%%%%%%%%%%%%%%%%%%%%%%%%%%%%%%%%%%%%%%%%%%%%%%%%%%%%%%%%
\subsection{O9811:}
\lstinputlisting{../Mould_Machining_Programs/bundle_programs_r/O9811.nc}


\clearpage
%%%%%%%%%%%%%%%%%%%%%%%%%%%%%%%%%%%%%%%%%%%%%%%%%%%%%%%%%%
%% subsection E.1.1 %%%%%%%%%%%%%%%%%%%%%%%%%%%%%%%%%%%%%%
%%%%%%%%%%%%%%%%%%%%%%%%%%%%%%%%%%%%%%%%%%%%%%%%%%%%%%%%%%
\subsection{O9812:}
\lstinputlisting{../Mould_Machining_Programs/bundle_programs_r/O9812.nc}


\clearpage
%%%%%%%%%%%%%%%%%%%%%%%%%%%%%%%%%%%%%%%%%%%%%%%%%%%%%%%%%%
%% subsection E.1.1 %%%%%%%%%%%%%%%%%%%%%%%%%%%%%%%%%%%%%%
%%%%%%%%%%%%%%%%%%%%%%%%%%%%%%%%%%%%%%%%%%%%%%%%%%%%%%%%%%
\subsection{O9814:}
\lstinputlisting{../Mould_Machining_Programs/bundle_programs_r/O9814.nc}


\clearpage
%%%%%%%%%%%%%%%%%%%%%%%%%%%%%%%%%%%%%%%%%%%%%%%%%%%%%%%%%%
%% subsection E.1.1 %%%%%%%%%%%%%%%%%%%%%%%%%%%%%%%%%%%%%%
%%%%%%%%%%%%%%%%%%%%%%%%%%%%%%%%%%%%%%%%%%%%%%%%%%%%%%%%%%
\subsection{O9815:}
\lstinputlisting{../Mould_Machining_Programs/bundle_programs_r/O9815.nc}


\clearpage
%%%%%%%%%%%%%%%%%%%%%%%%%%%%%%%%%%%%%%%%%%%%%%%%%%%%%%%%%%
%% subsection E.1.1 %%%%%%%%%%%%%%%%%%%%%%%%%%%%%%%%%%%%%%
%%%%%%%%%%%%%%%%%%%%%%%%%%%%%%%%%%%%%%%%%%%%%%%%%%%%%%%%%%
\subsection{O9816:}
\lstinputlisting{../Mould_Machining_Programs/bundle_programs_r/O9816.nc}


\clearpage
%%%%%%%%%%%%%%%%%%%%%%%%%%%%%%%%%%%%%%%%%%%%%%%%%%%%%%%%%%
%% subsection E.1.1 %%%%%%%%%%%%%%%%%%%%%%%%%%%%%%%%%%%%%%
%%%%%%%%%%%%%%%%%%%%%%%%%%%%%%%%%%%%%%%%%%%%%%%%%%%%%%%%%%
\subsection{O9817:}
\lstinputlisting{../Mould_Machining_Programs/bundle_programs_r/O9817.nc}


\clearpage
%%%%%%%%%%%%%%%%%%%%%%%%%%%%%%%%%%%%%%%%%%%%%%%%%%%%%%%%%%
%% subsection E.1.1 %%%%%%%%%%%%%%%%%%%%%%%%%%%%%%%%%%%%%%
%%%%%%%%%%%%%%%%%%%%%%%%%%%%%%%%%%%%%%%%%%%%%%%%%%%%%%%%%%
\subsection{O9818:}
\lstinputlisting{../Mould_Machining_Programs/bundle_programs_r/O9818.nc}


\clearpage
%%%%%%%%%%%%%%%%%%%%%%%%%%%%%%%%%%%%%%%%%%%%%%%%%%%%%%%%%%
%% subsection E.1.1 %%%%%%%%%%%%%%%%%%%%%%%%%%%%%%%%%%%%%%
%%%%%%%%%%%%%%%%%%%%%%%%%%%%%%%%%%%%%%%%%%%%%%%%%%%%%%%%%%
\subsection{O9819:}
\lstinputlisting{../Mould_Machining_Programs/bundle_programs_r/O9819.nc}


\clearpage
%%%%%%%%%%%%%%%%%%%%%%%%%%%%%%%%%%%%%%%%%%%%%%%%%%%%%%%%%%
%% subsection E.1.1 %%%%%%%%%%%%%%%%%%%%%%%%%%%%%%%%%%%%%%
%%%%%%%%%%%%%%%%%%%%%%%%%%%%%%%%%%%%%%%%%%%%%%%%%%%%%%%%%%
\subsection{O9820:}
\lstinputlisting{../Mould_Machining_Programs/bundle_programs_r/O9820.nc}


\clearpage
%%%%%%%%%%%%%%%%%%%%%%%%%%%%%%%%%%%%%%%%%%%%%%%%%%%%%%%%%%
%% subsection E.1.1 %%%%%%%%%%%%%%%%%%%%%%%%%%%%%%%%%%%%%%
%%%%%%%%%%%%%%%%%%%%%%%%%%%%%%%%%%%%%%%%%%%%%%%%%%%%%%%%%%
\subsection{O9821:}
\lstinputlisting{../Mould_Machining_Programs/bundle_programs_r/O9821.nc}


\clearpage
%%%%%%%%%%%%%%%%%%%%%%%%%%%%%%%%%%%%%%%%%%%%%%%%%%%%%%%%%%
%% subsection E.1.1 %%%%%%%%%%%%%%%%%%%%%%%%%%%%%%%%%%%%%%
%%%%%%%%%%%%%%%%%%%%%%%%%%%%%%%%%%%%%%%%%%%%%%%%%%%%%%%%%%
\subsection{O9822:}
\lstinputlisting{../Mould_Machining_Programs/bundle_programs_r/O9822.nc}


\clearpage
%%%%%%%%%%%%%%%%%%%%%%%%%%%%%%%%%%%%%%%%%%%%%%%%%%%%%%%%%%
%% subsection E.1.1 %%%%%%%%%%%%%%%%%%%%%%%%%%%%%%%%%%%%%%
%%%%%%%%%%%%%%%%%%%%%%%%%%%%%%%%%%%%%%%%%%%%%%%%%%%%%%%%%%
\subsection{O9823:}
\lstinputlisting{../Mould_Machining_Programs/bundle_programs_r/O9823.nc}


\clearpage
%%%%%%%%%%%%%%%%%%%%%%%%%%%%%%%%%%%%%%%%%%%%%%%%%%%%%%%%%%
%% subsection E.1.1 %%%%%%%%%%%%%%%%%%%%%%%%%%%%%%%%%%%%%%
%%%%%%%%%%%%%%%%%%%%%%%%%%%%%%%%%%%%%%%%%%%%%%%%%%%%%%%%%%
\subsection{O9832:}
\lstinputlisting{../Mould_Machining_Programs/bundle_programs_r/O9832.nc}


\clearpage
%%%%%%%%%%%%%%%%%%%%%%%%%%%%%%%%%%%%%%%%%%%%%%%%%%%%%%%%%%
%% subsection E.1.1 %%%%%%%%%%%%%%%%%%%%%%%%%%%%%%%%%%%%%%
%%%%%%%%%%%%%%%%%%%%%%%%%%%%%%%%%%%%%%%%%%%%%%%%%%%%%%%%%%
\subsection{O9833:}
\lstinputlisting{../Mould_Machining_Programs/bundle_programs_r/O9833.nc}


\clearpage
%%%%%%%%%%%%%%%%%%%%%%%%%%%%%%%%%%%%%%%%%%%%%%%%%%%%%%%%%%
%% subsection E.1.1 %%%%%%%%%%%%%%%%%%%%%%%%%%%%%%%%%%%%%%
%%%%%%%%%%%%%%%%%%%%%%%%%%%%%%%%%%%%%%%%%%%%%%%%%%%%%%%%%%
\subsection{O9834:}
\lstinputlisting{../Mould_Machining_Programs/bundle_programs_r/O9834.nc}


\clearpage
%%%%%%%%%%%%%%%%%%%%%%%%%%%%%%%%%%%%%%%%%%%%%%%%%%%%%%%%%%
%% subsection E.1.1 %%%%%%%%%%%%%%%%%%%%%%%%%%%%%%%%%%%%%%
%%%%%%%%%%%%%%%%%%%%%%%%%%%%%%%%%%%%%%%%%%%%%%%%%%%%%%%%%%
\subsection{O9835:}
\lstinputlisting{../Mould_Machining_Programs/bundle_programs_r/O9835.nc}


\clearpage
%%%%%%%%%%%%%%%%%%%%%%%%%%%%%%%%%%%%%%%%%%%%%%%%%%%%%%%%%%
%% subsection E.1.1 %%%%%%%%%%%%%%%%%%%%%%%%%%%%%%%%%%%%%%
%%%%%%%%%%%%%%%%%%%%%%%%%%%%%%%%%%%%%%%%%%%%%%%%%%%%%%%%%%
\subsection{O9843:}
\lstinputlisting{../Mould_Machining_Programs/bundle_programs_r/O9843.nc}


\clearpage
%%%%%%%%%%%%%%%%%%%%%%%%%%%%%%%%%%%%%%%%%%%%%%%%%%%%%%%%%%
%% subsection E.1.1 %%%%%%%%%%%%%%%%%%%%%%%%%%%%%%%%%%%%%%
%%%%%%%%%%%%%%%%%%%%%%%%%%%%%%%%%%%%%%%%%%%%%%%%%%%%%%%%%%
\subsection{O9855:}
\lstinputlisting{../Mould_Machining_Programs/bundle_programs_r/O9855.nc}


\clearpage
%%%%%%%%%%%%%%%%%%%%%%%%%%%%%%%%%%%%%%%%%%%%%%%%%%%%%%%%%%
%% subsection E.1.1 %%%%%%%%%%%%%%%%%%%%%%%%%%%%%%%%%%%%%%
%%%%%%%%%%%%%%%%%%%%%%%%%%%%%%%%%%%%%%%%%%%%%%%%%%%%%%%%%%
\subsection{O9856:}
\lstinputlisting{../Mould_Machining_Programs/bundle_programs_r/O9857.nc}


\clearpage
%%%%%%%%%%%%%%%%%%%%%%%%%%%%%%%%%%%%%%%%%%%%%%%%%%%%%%%%%%
%% subsection E.1.1 %%%%%%%%%%%%%%%%%%%%%%%%%%%%%%%%%%%%%%
%%%%%%%%%%%%%%%%%%%%%%%%%%%%%%%%%%%%%%%%%%%%%%%%%%%%%%%%%%
\subsection{O9857:}
\lstinputlisting{../Mould_Machining_Programs/bundle_programs_r/O9857.nc}


\clearpage
%%%%%%%%%%%%%%%%%%%%%%%%%%%%%%%%%%%%%%%%%%%%%%%%%%%%%%%%%%
%% subsection E.1.1 %%%%%%%%%%%%%%%%%%%%%%%%%%%%%%%%%%%%%%
%%%%%%%%%%%%%%%%%%%%%%%%%%%%%%%%%%%%%%%%%%%%%%%%%%%%%%%%%%
\subsection{O9858:}
\lstinputlisting{../Mould_Machining_Programs/bundle_programs_r/O9858.nc}


\clearpage
%%%%%%%%%%%%%%%%%%%%%%%%%%%%%%%%%%%%%%%%%%%%%%%%%%%%%%%%%%
%% subsection E.1.1 %%%%%%%%%%%%%%%%%%%%%%%%%%%%%%%%%%%%%%
%%%%%%%%%%%%%%%%%%%%%%%%%%%%%%%%%%%%%%%%%%%%%%%%%%%%%%%%%%
\subsection{O9859:}
\lstinputlisting{../Mould_Machining_Programs/bundle_programs_r/O9859.nc}


\clearpage
%%%%%%%%%%%%%%%%%%%%%%%%%%%%%%%%%%%%%%%%%%%%%%%%%%%%%%%%%%
%% subsection E.1.1 %%%%%%%%%%%%%%%%%%%%%%%%%%%%%%%%%%%%%%
%%%%%%%%%%%%%%%%%%%%%%%%%%%%%%%%%%%%%%%%%%%%%%%%%%%%%%%%%%
\subsection{O9890:}
\lstinputlisting{../Mould_Machining_Programs/bundle_programs_r/O9890.nc}


%\clearpage
%%%%%%%%%%%%%%%%%%%%%%%%%%%%%%%%%%%%%%%%%%%%%%%%%%%%%%%%%%
%% subsection E.1.1 %%%%%%%%%%%%%%%%%%%%%%%%%%%%%%%%%%%%%%
%%%%%%%%%%%%%%%%%%%%%%%%%%%%%%%%%%%%%%%%%%%%%%%%%%%%%%%%%%
\subsection{O9891:}
\lstinputlisting{../Mould_Machining_Programs/bundle_programs_r/O9891.nc}


%\clearpage
%%%%%%%%%%%%%%%%%%%%%%%%%%%%%%%%%%%%%%%%%%%%%%%%%%%%%%%%%%
%% subsection E.1.1 %%%%%%%%%%%%%%%%%%%%%%%%%%%%%%%%%%%%%%
%%%%%%%%%%%%%%%%%%%%%%%%%%%%%%%%%%%%%%%%%%%%%%%%%%%%%%%%%%
\subsection{O9892:}
\lstinputlisting{../Mould_Machining_Programs/bundle_programs_r/O9892.nc}


\clearpage
%%%%%%%%%%%%%%%%%%%%%%%%%%%%%%%%%%%%%%%%%%%%%%%%%%%%%%%%%%
%% subsection E.1.1 %%%%%%%%%%%%%%%%%%%%%%%%%%%%%%%%%%%%%%
%%%%%%%%%%%%%%%%%%%%%%%%%%%%%%%%%%%%%%%%%%%%%%%%%%%%%%%%%%
\subsection{O9893:}
\lstinputlisting{../Mould_Machining_Programs/bundle_programs_r/O9893.nc}


\clearpage
%%%%%%%%%%%%%%%%%%%%%%%%%%%%%%%%%%%%%%%%%%%%%%%%%%%%%%%%%%
%% subsection E.1.1 %%%%%%%%%%%%%%%%%%%%%%%%%%%%%%%%%%%%%%
%%%%%%%%%%%%%%%%%%%%%%%%%%%%%%%%%%%%%%%%%%%%%%%%%%%%%%%%%%
\subsection{O9901:}
\lstinputlisting{../Mould_Machining_Programs/bundle_programs_r/O9901.nc}


\clearpage
%%%%%%%%%%%%%%%%%%%%%%%%%%%%%%%%%%%%%%%%%%%%%%%%%%%%%%%%%%
%% subsection E.1.1 %%%%%%%%%%%%%%%%%%%%%%%%%%%%%%%%%%%%%%
%%%%%%%%%%%%%%%%%%%%%%%%%%%%%%%%%%%%%%%%%%%%%%%%%%%%%%%%%%
\subsection{O9921:}
\lstinputlisting{../Mould_Machining_Programs/bundle_programs_r/O9821.nc}





%%%%%%%%%%%%%%%%%%%%%%%%%%%%%%%%%%%%%%%%%%%%%%%%%%%%%%%%%%
%%            %%%%%%%%%%%%%%%%%%%%%%%%%%%%%%%%%%%%%%%%%%%%
%% Appendix H %%%%%%%%%%%%%%%%%%%%%%%%%%%%%%%%%%%%%%%%%%%%
%%            %%%%%%%%%%%%%%%%%%%%%%%%%%%%%%%%%%%%%%%%%%%%
%%%%%%%%%%%%%%%%%%%%%%%%%%%%%%%%%%%%%%%%%%%%%%%%%%%%%%%%%%
%\chapter{作成したNCプログラム}
%%!TEX root = ../Mould_Analytical_Calculation_Note.tex

ここでは具体的に作成した(サブ)プログラムを記載しておく。



%%%%%%%%%%%%%%%%%%%%%%%%%%%%%%%%%%%%%%%%%%%%%%%%%%%%%%%%%%
%% section E.1 %%%%%%%%%%%%%%%%%%%%%%%%%%%%%%%%%%%%%%%%%%%
%%%%%%%%%%%%%%%%%%%%%%%%%%%%%%%%%%%%%%%%%%%%%%%%%%%%%%%%%%
\section{測定用サブプログラム}
\addcontentsline{lol}{section}{\numberline{\thesection}測定用サブプログラム}

\captionof{lstlisting}{\MXOThickness:測定 外側中心\texorpdfstring{$X$}{X}}
\lstinputlisting[style=Gcode-more]{../Mould_Machining_Programs/sub_programs/\MXOThickness}

\clearpage
%%%%%%%%%%%%%%%%%%%%%%%%%%%%%%%%%%%%%%%%%%%%%%%%%%%%%%%%%%
%% subsection E.1.2 %%%%%%%%%%%%%%%%%%%%%%%%%%%%%%%%%%%%%%
%%%%%%%%%%%%%%%%%%%%%%%%%%%%%%%%%%%%%%%%%%%%%%%%%%%%%%%%%%
\captionof{lstlisting}{\MYOThickness:測定 外側中心\texorpdfstring{$Y$}{Y}}
\lstinputlisting[style=Gcode-more]{../Mould_Machining_Programs/sub_programs/\MYOThickness}


\clearpage
%%%%%%%%%%%%%%%%%%%%%%%%%%%%%%%%%%%%%%%%%%%%%%%%%%%%%%%%%%
%% subsection E.1.3 %%%%%%%%%%%%%%%%%%%%%%%%%%%%%%%%%%%%%%
%%%%%%%%%%%%%%%%%%%%%%%%%%%%%%%%%%%%%%%%%%%%%%%%%%%%%%%%%%
\captionof{lstlisting}{\MXIWidth:測定 内側中心\texorpdfstring{$X$}{X}}
\lstinputlisting[style=Gcode-more]{../Mould_Machining_Programs/sub_programs/\MXIWidth}


\clearpage
%%%%%%%%%%%%%%%%%%%%%%%%%%%%%%%%%%%%%%%%%%%%%%%%%%%%%%%%%%
%% subsection E.1.4 %%%%%%%%%%%%%%%%%%%%%%%%%%%%%%%%%%%%%%
%%%%%%%%%%%%%%%%%%%%%%%%%%%%%%%%%%%%%%%%%%%%%%%%%%%%%%%%%%
\captionof{lstlisting}{\MYIWidth:測定 内側中心\texorpdfstring{$Y$}{Y}}
\lstinputlisting[style=Gcode-more]{../Mould_Machining_Programs/sub_programs/\MYIWidth}


\clearpage
%%%%%%%%%%%%%%%%%%%%%%%%%%%%%%%%%%%%%%%%%%%%%%%%%%%%%%%%%%
%% subsection E.1.5 %%%%%%%%%%%%%%%%%%%%%%%%%%%%%%%%%%%%%%
%%%%%%%%%%%%%%%%%%%%%%%%%%%%%%%%%%%%%%%%%%%%%%%%%%%%%%%%%%
\captionof{lstlisting}{\MXface:測定 \texorpdfstring{$-X$}{-X}方向基準面}
\lstinputlisting[style=Gcode-more]{../Mould_Machining_Programs/sub_programs/\MXface}


\clearpage
%%%%%%%%%%%%%%%%%%%%%%%%%%%%%%%%%%%%%%%%%%%%%%%%%%%%%%%%%%
%% subsection E.1.6 %%%%%%%%%%%%%%%%%%%%%%%%%%%%%%%%%%%%%%
%%%%%%%%%%%%%%%%%%%%%%%%%%%%%%%%%%%%%%%%%%%%%%%%%%%%%%%%%%
\captionof{lstlisting}{\MYcenterline:測定 通り芯\texorpdfstring{$Y$}{Y}}
\lstinputlisting[style=Gcode-more]{../Mould_Machining_Programs/sub_programs/\MYcenterline}


\clearpage
%%%%%%%%%%%%%%%%%%%%%%%%%%%%%%%%%%%%%%%%%%%%%%%%%%%%%%%%%%
%% subsection E.1.7 %%%%%%%%%%%%%%%%%%%%%%%%%%%%%%%%%%%%%%
%%%%%%%%%%%%%%%%%%%%%%%%%%%%%%%%%%%%%%%%%%%%%%%%%%%%%%%%%%
\captionof{lstlisting}{\MXcenterline:測定 通り芯\texorpdfstring{$X$}{X}}
\lstinputlisting[style=Gcode-more]{../Mould_Machining_Programs/sub_programs/\MXcenterline}



\clearpage
%%%%%%%%%%%%%%%%%%%%%%%%%%%%%%%%%%%%%%%%%%%%%%%%%%%%%%%%%%
%% section E.2 %%%%%%%%%%%%%%%%%%%%%%%%%%%%%%%%%%%%%%%%%%%
%%%%%%%%%%%%%%%%%%%%%%%%%%%%%%%%%%%%%%%%%%%%%%%%%%%%%%%%%%
\section{加工用サブプログラム}


%%%%%%%%%%%%%%%%%%%%%%%%%%%%%%%%%%%%%%%%%%%%%%%%%%%%%%%%%%
%% subsection E.2.1 %%%%%%%%%%%%%%%%%%%%%%%%%%%%%%%%%%%%%%
%%%%%%%%%%%%%%%%%%%%%%%%%%%%%%%%%%%%%%%%%%%%%%%%%%%%%%%%%%
\captionof{lstlisting}{\KRecRight:加工 端面}
\lstinputlisting[style=Gcode-more]{../Mould_Machining_Programs/sub_programs/\KRecRight}


\clearpage
%%%%%%%%%%%%%%%%%%%%%%%%%%%%%%%%%%%%%%%%%%%%%%%%%%%%%%%%%%
%% subsection E.2.2 %%%%%%%%%%%%%%%%%%%%%%%%%%%%%%%%%%%%%%
%%%%%%%%%%%%%%%%%%%%%%%%%%%%%%%%%%%%%%%%%%%%%%%%%%%%%%%%%%
\captionof{lstlisting}{\KICRLeft:加工 内側1周 コーナーR 左回り}
\lstinputlisting[style=Gcode-more]{../Mould_Machining_Programs/sub_programs/\KICRLeft}


\clearpage
%%%%%%%%%%%%%%%%%%%%%%%%%%%%%%%%%%%%%%%%%%%%%%%%%%%%%%%%%%
%% subsection E.2.3 %%%%%%%%%%%%%%%%%%%%%%%%%%%%%%%%%%%%%%
%%%%%%%%%%%%%%%%%%%%%%%%%%%%%%%%%%%%%%%%%%%%%%%%%%%%%%%%%%
\captionof{lstlisting}{\KOCRLeft:加工 外側1周 コーナーR 左回り}
\lstinputlisting[style=Gcode-more]{../Mould_Machining_Programs/sub_programs/\KOCRLeft}

\texttt{}

\clearpage
%%%%%%%%%%%%%%%%%%%%%%%%%%%%%%%%%%%%%%%%%%%%%%%%%%%%%%%%%%
%% section E.3 %%%%%%%%%%%%%%%%%%%%%%%%%%%%%%%%%%%%%%%%%%%
%%%%%%%%%%%%%%%%%%%%%%%%%%%%%%%%%%%%%%%%%%%%%%%%%%%%%%%%%%
\section{内面溝用サブプログラム}


%%%%%%%%%%%%%%%%%%%%%%%%%%%%%%%%%%%%%%%%%%%%%%%%%%%%%%%%%%
%% subsection E.3.1 %%%%%%%%%%%%%%%%%%%%%%%%%%%%%%%%%%%%%%
%%%%%%%%%%%%%%%%%%%%%%%%%%%%%%%%%%%%%%%%%%%%%%%%%%%%%%%%%%
\captionof{lstlisting}{\DLone:移動 各列の中心上}
\lstinputlisting[style=Gcode-more]{../Mould_Machining_Programs/sub_programs/\DLone}


\clearpage
%%%%%%%%%%%%%%%%%%%%%%%%%%%%%%%%%%%%%%%%%%%%%%%%%%%%%%%%%%
%% subsection E.3.2 %%%%%%%%%%%%%%%%%%%%%%%%%%%%%%%%%%%%%%
%%%%%%%%%%%%%%%%%%%%%%%%%%%%%%%%%%%%%%%%%%%%%%%%%%%%%%%%%%
\captionof{lstlisting}{\DLtwoAC:移動 AC面 列内の各内面溝上}
\lstinputlisting[style=Gcode-more]{../Mould_Machining_Programs/sub_programs/\DLtwoAC}


\clearpage
%%%%%%%%%%%%%%%%%%%%%%%%%%%%%%%%%%%%%%%%%%%%%%%%%%%%%%%%%%
%% subsection E.3.3 %%%%%%%%%%%%%%%%%%%%%%%%%%%%%%%%%%%%%%
%%%%%%%%%%%%%%%%%%%%%%%%%%%%%%%%%%%%%%%%%%%%%%%%%%%%%%%%%%
\captionof{lstlisting}{\DLtwoBD:移動 BC面 列内の各内面溝上}
\lstinputlisting[style=Gcode-more]{../Mould_Machining_Programs/sub_programs/\DLtwoBD}


\clearpage
%%%%%%%%%%%%%%%%%%%%%%%%%%%%%%%%%%%%%%%%%%%%%%%%%%%%%%%%%%
%% subsection E.3.4 %%%%%%%%%%%%%%%%%%%%%%%%%%%%%%%%%%%%%%
%%%%%%%%%%%%%%%%%%%%%%%%%%%%%%%%%%%%%%%%%%%%%%%%%%%%%%%%%%
\captionof{lstlisting}{\DMLthreeAC:測定 AC内表面\texorpdfstring{$X$}{X}}
\lstinputlisting[style=Gcode-more]{../Mould_Machining_Programs/sub_programs/\DMLthreeAC}


\clearpage
%%%%%%%%%%%%%%%%%%%%%%%%%%%%%%%%%%%%%%%%%%%%%%%%%%%%%%%%%%
%% subsection E.3.5 %%%%%%%%%%%%%%%%%%%%%%%%%%%%%%%%%%%%%%
%%%%%%%%%%%%%%%%%%%%%%%%%%%%%%%%%%%%%%%%%%%%%%%%%%%%%%%%%%
\captionof{lstlisting}{\DMLthreeBD:測定 BD内表面\texorpdfstring{$Y$}{Y}}
\lstinputlisting[style=Gcode-more]{../Mould_Machining_Programs/sub_programs/\DMLthreeBD}


\clearpage
%%%%%%%%%%%%%%%%%%%%%%%%%%%%%%%%%%%%%%%%%%%%%%%%%%%%%%%%%%
%% subsection E.3.6 %%%%%%%%%%%%%%%%%%%%%%%%%%%%%%%%%%%%%%
%%%%%%%%%%%%%%%%%%%%%%%%%%%%%%%%%%%%%%%%%%%%%%%%%%%%%%%%%%
\captionof{lstlisting}{\DKLthreeAC:加工 AC内表面\texorpdfstring{$X$}{X}}
\lstinputlisting[style=Gcode-more]{../Mould_Machining_Programs/sub_programs/\DKLthreeAC}


\clearpage
%%%%%%%%%%%%%%%%%%%%%%%%%%%%%%%%%%%%%%%%%%%%%%%%%%%%%%%%%%
%% subsection E.3.7 %%%%%%%%%%%%%%%%%%%%%%%%%%%%%%%%%%%%%%
%%%%%%%%%%%%%%%%%%%%%%%%%%%%%%%%%%%%%%%%%%%%%%%%%%%%%%%%%%
\captionof{lstlisting}{\DKLthreeBD:加工 BD内表面\texorpdfstring{$Y$}{Y}}
\lstinputlisting[style=Gcode-more]{../Mould_Machining_Programs/sub_programs/\DKLthreeBD}



\clearpage
%%%%%%%%%%%%%%%%%%%%%%%%%%%%%%%%%%%%%%%%%%%%%%%%%%%%%%%%%%
%% section E.4 %%%%%%%%%%%%%%%%%%%%%%%%%%%%%%%%%%%%%%%%%%%
%%%%%%%%%%%%%%%%%%%%%%%%%%%%%%%%%%%%%%%%%%%%%%%%%%%%%%%%%%
\section{その他のサブプログラム}


%%%%%%%%%%%%%%%%%%%%%%%%%%%%%%%%%%%%%%%%%%%%%%%%%%%%%%%%%%
%% subsection E.4.1 %%%%%%%%%%%%%%%%%%%%%%%%%%%%%%%%%%%%%%
%%%%%%%%%%%%%%%%%%%%%%%%%%%%%%%%%%%%%%%%%%%%%%%%%%%%%%%%%%
\captionof{lstlisting}{\OsensorOn:タッチセンサー電源ON}
\lstinputlisting[style=Gcode-more]{../Mould_Machining_Programs/sub_programs/\OsensorOn}


\clearpage
%%%%%%%%%%%%%%%%%%%%%%%%%%%%%%%%%%%%%%%%%%%%%%%%%%%%%%%%%%
%% subsection E.4.2 %%%%%%%%%%%%%%%%%%%%%%%%%%%%%%%%%%%%%%
%%%%%%%%%%%%%%%%%%%%%%%%%%%%%%%%%%%%%%%%%%%%%%%%%%%%%%%%%%
\captionof{lstlisting}{\OsensorOff:タッチセンサー電源OFF}
\lstinputlisting[style=Gcode-more]{../Mould_Machining_Programs/sub_programs/\OsensorOff}



\end{appendices}




%%%%%%%%%%%%%%%%%%%%%%%%%%%%%%%%%%%%%%%%%%%%%%%%%%%%%%%%%
%%               %%%%%%%%%%%%%%%%%%%%%%%%%%%%%%%%%%%%%%%%
%%               %%%%%%%%%%%%%%%%%%%%%%%%%%%%%%%%%%%%%%%%
%% Part II       %%%%%%%%%%%%%%%%%%%%%%%%%%%%%%%%%%%%%%%%
%%               %%%%%%%%%%%%%%%%%%%%%%%%%%%%%%%%%%%%%%%%
%%               %%%%%%%%%%%%%%%%%%%%%%%%%%%%%%%%%%%%%%%%
%%%%%%%%%%%%%%%%%%%%%%%%%%%%%%%%%%%%%%%%%%%%%%%%%%%%%%%%%
%\part{モールドのRDBの作成}
%%!TEX root = Mould_Analytical_Calculation_Note.tex





%%%%%%%%%%%%%%%%%%%%%%%%%%%%%%%%%%%%%%%%%%%%%%%%%%%%%%%%%%
%%            %%%%%%%%%%%%%%%%%%%%%%%%%%%%%%%%%%%%%%%%%%%%
%% chapter    %%%%%%%%%%%%%%%%%%%%%%%%%%%%%%%%%%%%%%%%%%%%
%%            %%%%%%%%%%%%%%%%%%%%%%%%%%%%%%%%%%%%%%%%%%%%
%%%%%%%%%%%%%%%%%%%%%%%%%%%%%%%%%%%%%%%%%%%%%%%%%%%%%%%%%%
\chapter{データベースの構造}



%%%%%%%%%%%%%%%%%%%%%%%%%%%%%%%%%%%%%%%%%%%%%%%%%%%%%%%%%%
%% section D.1 %%%%%%%%%%%%%%%%%%%%%%%%%%%%%%%%%%%%%%%%%%%
%%%%%%%%%%%%%%%%%%%%%%%%%%%%%%%%%%%%%%%%%%%%%%%%%%%%%%%%%%
\section{列}



%%%%%%%%%%%%%%%%%%%%%%%%%%%%%%%%%%%%%%%%%%%%%%%%%%%%%%%%%%
%% section D.2 %%%%%%%%%%%%%%%%%%%%%%%%%%%%%%%%%%%%%%%%%%%
%%%%%%%%%%%%%%%%%%%%%%%%%%%%%%%%%%%%%%%%%%%%%%%%%%%%%%%%%%
\section{}


%%%%%%%%%%%%%%%%%%%%%%%%%%%%%%%%%%%%%%%%%%%%%%%%%%%%%%%%%%
%%           %%%%%%%%%%%%%%%%%%%%%%%%%%%%%%%%%%%%%%%%%%%%%
%% chapter 2 %%%%%%%%%%%%%%%%%%%%%%%%%%%%%%%%%%%%%%%%%%%%%
%%           %%%%%%%%%%%%%%%%%%%%%%%%%%%%%%%%%%%%%%%%%%%%%
%%%%%%%%%%%%%%%%%%%%%%%%%%%%%%%%%%%%%%%%%%%%%%%%%%%%%%%%%%
\chapter{}



%%%%%%%%%%%%%%%%%%%%%%%%%%%%%%%%%%%%%%%%%%%%%%%%%%%%%%%%%%
%% section 1.1 %%%%%%%%%%%%%%%%%%%%%%%%%%%%%%%%%%%%%%%%%%%
%%%%%%%%%%%%%%%%%%%%%%%%%%%%%%%%%%%%%%%%%%%%%%%%%%%%%%%%%%
\section{}



%%%%%%%%%%%%%%%%%%%%%%%%%%%%%%%%%%%%%%%%%%%%%%%%%%%%%%%%%%
%% section 1.2 %%%%%%%%%%%%%%%%%%%%%%%%%%%%%%%%%%%%%%%%%%%
%%%%%%%%%%%%%%%%%%%%%%%%%%%%%%%%%%%%%%%%%%%%%%%%%%%%%%%%%%
\section{}








%%%%%%%%%%%%%%%%%%%%%%%%%%%%%%%%%%%%%%%%%%%%%%%%%%%%%%%%%
%%               %%%%%%%%%%%%%%%%%%%%%%%%%%%%%%%%%%%%%%%%
%%               %%%%%%%%%%%%%%%%%%%%%%%%%%%%%%%%%%%%%%%%
%% Part III      %%%%%%%%%%%%%%%%%%%%%%%%%%%%%%%%%%%%%%%%
%%               %%%%%%%%%%%%%%%%%%%%%%%%%%%%%%%%%%%%%%%%
%%               %%%%%%%%%%%%%%%%%%%%%%%%%%%%%%%%%%%%%%%%
%%%%%%%%%%%%%%%%%%%%%%%%%%%%%%%%%%%%%%%%%%%%%%%%%%%%%%%%%
%\part{システム設計の流れ}
%%!TEX root = Mould_Analytical_Calculation_Note.tex





%%%%%%%%%%%%%%%%%%%%%%%%%%%%%%%%%%%%%%%%%%%%%%%%%%%%%%%%%%
%%            %%%%%%%%%%%%%%%%%%%%%%%%%%%%%%%%%%%%%%%%%%%%
%% Appendix D %%%%%%%%%%%%%%%%%%%%%%%%%%%%%%%%%%%%%%%%%%%%
%%            %%%%%%%%%%%%%%%%%%%%%%%%%%%%%%%%%%%%%%%%%%%%
%%%%%%%%%%%%%%%%%%%%%%%%%%%%%%%%%%%%%%%%%%%%%%%%%%%%%%%%%%
\chapter{現状および業務フロー分析}



%%%%%%%%%%%%%%%%%%%%%%%%%%%%%%%%%%%%%%%%%%%%%%%%%%%%%%%%%%
%% section D.1 %%%%%%%%%%%%%%%%%%%%%%%%%%%%%%%%%%%%%%%%%%%
%%%%%%%%%%%%%%%%%%%%%%%%%%%%%%%%%%%%%%%%%%%%%%%%%%%%%%%%%%
\section{業務フロー}
現在、社内のマシニング(以下、三菱マシニング)を用いてモールドの加工を行っている。
しかし、三菱マシニング(および社内のマシニング)ではモールドの内面溝加工を行うことできる能力を持っていない。
そのため、内面溝加工に関しては全て外注により行われている。
具体的なフローは以下のようになっている。

\paragraph{加工の依頼}\noindent
必要な加工内容と数量を社内で決定し、それを外注先に依頼する。

\paragraph{加工}\noindent
指定された内容に基づき、外注先が内面溝の加工を行う。

\paragraph{検品}\noindent
加工された製品が社内に戻った後、内面溝部の品質検査を行う。



%%%%%%%%%%%%%%%%%%%%%%%%%%%%%%%%%%%%%%%%%%%%%%%%%%%%%%%%%%
%% section D.2 %%%%%%%%%%%%%%%%%%%%%%%%%%%%%%%%%%%%%%%%%%%
%%%%%%%%%%%%%%%%%%%%%%%%%%%%%%%%%%%%%%%%%%%%%%%%%%%%%%%%%%
\section{}


%%%%%%%%%%%%%%%%%%%%%%%%%%%%%%%%%%%%%%%%%%%%%%%%%%%%%%%%%%
%%           %%%%%%%%%%%%%%%%%%%%%%%%%%%%%%%%%%%%%%%%%%%%%
%% chapter 2 %%%%%%%%%%%%%%%%%%%%%%%%%%%%%%%%%%%%%%%%%%%%%
%%           %%%%%%%%%%%%%%%%%%%%%%%%%%%%%%%%%%%%%%%%%%%%%
%%%%%%%%%%%%%%%%%%%%%%%%%%%%%%%%%%%%%%%%%%%%%%%%%%%%%%%%%%
\chapter{要件定義}



%%%%%%%%%%%%%%%%%%%%%%%%%%%%%%%%%%%%%%%%%%%%%%%%%%%%%%%%%%
%% section 1.1 %%%%%%%%%%%%%%%%%%%%%%%%%%%%%%%%%%%%%%%%%%%
%%%%%%%%%%%%%%%%%%%%%%%%%%%%%%%%%%%%%%%%%%%%%%%%%%%%%%%%%%
\section{目標}
新しいマシニングを導入することで、これまで外注していた管状の金属の内部表面の加工を社内で可能にし、生産効率を向上させる。具体的には、1時間あたりの生産量を現状の2倍にし、加工コストを20\%削減する。



%%%%%%%%%%%%%%%%%%%%%%%%%%%%%%%%%%%%%%%%%%%%%%%%%%%%%%%%%%
%% section 1.2 %%%%%%%%%%%%%%%%%%%%%%%%%%%%%%%%%%%%%%%%%%%
%%%%%%%%%%%%%%%%%%%%%%%%%%%%%%%%%%%%%%%%%%%%%%%%%%%%%%%%%%
\section{機能}
\paragraph{特殊な加工を行うためのプログラムを実行する機能}
この機能により、特殊な加工が自動化されます。具体的には、材料の形状や大きさに応じて最適な加工パラメータが自動的に設定されます。

\paragraph{加工結果をモニターに表示する機能}
この機能により、ユーザーは加工結果をリアルタイムで確認することができます。具体的には、加工後の材料の形状や大きさ、加工時間、加工精度などが表示されます。

\paragraph{加工パラメータをユーザーが設定できる機能}
この機能により、ユーザーは加工パラメータを自由に設定することができます。具体的には、切削速度や送り速度、切削深さなどのパラメータを設定できます。

\paragraph{エラーが発生した場合に警告を表示する機能}
この機能により、ユーザーはシステムの異常をすぐに把握することができます。具体的には、システムが停止した場合や異常な振動が発生した場合などに警告が表示されます。



%%%%%%%%%%%%%%%%%%%%%%%%%%%%%%%%%%%%%%%%%%%%%%%%%%%%%%%%%%
%% section 1.3 %%%%%%%%%%%%%%%%%%%%%%%%%%%%%%%%%%%%%%%%%%%
%%%%%%%%%%%%%%%%%%%%%%%%%%%%%%%%%%%%%%%%%%%%%%%%%%%%%%%%%%
\section{性能要件}
\begin{enumerate}
\item 加工時間は1ピースあたり30秒以内であること
\item 故障率は5%以下であること
\item 連続稼働時間は24時間以上であること
\end{enumerate}



%%%%%%%%%%%%%%%%%%%%%%%%%%%%%%%%%%%%%%%%%%%%%%%%%%%%%%%%%%
%% section 1.3 %%%%%%%%%%%%%%%%%%%%%%%%%%%%%%%%%%%%%%%%%%%
%%%%%%%%%%%%%%%%%%%%%%%%%%%%%%%%%%%%%%%%%%%%%%%%%%%%%%%%%%
\section{制約条件}
\begin{enumerate}
\item 開発期間は6ヶ月以内であること
\item 開発予算は500万円以内であること
\item 機械のサイズは設置スペース(幅2m×奥行き2m×高さ2m)に収まる範囲であること
\end{enumerate}




%%%%%%%%%%%%%%%%%%%%%%%%%%%%%%%%%%%%%%%%%%%%%%%%%%%%%%%%%%
%%           %%%%%%%%%%%%%%%%%%%%%%%%%%%%%%%%%%%%%%%%%%%%%
%% chapter 2 %%%%%%%%%%%%%%%%%%%%%%%%%%%%%%%%%%%%%%%%%%%%%
%%           %%%%%%%%%%%%%%%%%%%%%%%%%%%%%%%%%%%%%%%%%%%%%
%%%%%%%%%%%%%%%%%%%%%%%%%%%%%%%%%%%%%%%%%%%%%%%%%%%%%%%%%%
\chapter{設計}
要件定義に基づいてシステムの設計を行います。具体的には以下のような活動が行われます:




%%%%%%%%%%%%%%%%%%%%%%%%%%%%%%%%%%%%%%%%%%%%%%%%%%%%%%%%%%
%% section 1.1 %%%%%%%%%%%%%%%%%%%%%%%%%%%%%%%%%%%%%%%%%%%
%%%%%%%%%%%%%%%%%%%%%%%%%%%%%%%%%%%%%%%%%%%%%%%%%%%%%%%%%%
\section{システム設計}
システム全体のアーキテクチャを設計します。これには、システムの主要なコンポーネントやそれらがどのように相互作用するか、データがどのように流れるかなどが含まれます。




%%%%%%%%%%%%%%%%%%%%%%%%%%%%%%%%%%%%%%%%%%%%%%%%%%%%%%%%%%
%% section 1.1 %%%%%%%%%%%%%%%%%%%%%%%%%%%%%%%%%%%%%%%%%%%
%%%%%%%%%%%%%%%%%%%%%%%%%%%%%%%%%%%%%%%%%%%%%%%%%%%%%%%%%%
\section{詳細設計}
各コンポーネントの内部構造や動作を詳細に設計します。これには、データ構造、アルゴリズム、インターフェースなどが含まれます。



%%%%%%%%%%%%%%%%%%%%%%%%%%%%%%%%%%%%%%%%%%%%%%%%%%%%%%%%%%
%%           %%%%%%%%%%%%%%%%%%%%%%%%%%%%%%%%%%%%%%%%%%%%%
%% chapter 3 %%%%%%%%%%%%%%%%%%%%%%%%%%%%%%%%%%%%%%%%%%%%%
%%           %%%%%%%%%%%%%%%%%%%%%%%%%%%%%%%%%%%%%%%%%%%%%
%%%%%%%%%%%%%%%%%%%%%%%%%%%%%%%%%%%%%%%%%%%%%%%%%%%%%%%%%%
\chapter{実装}
設計に基づいてプログラムを書きます。この段階では、選択したプログラミング言語を使用してコードを書きます。



%%%%%%%%%%%%%%%%%%%%%%%%%%%%%%%%%%%%%%%%%%%%%%%%%%%%%%%%%%
%%           %%%%%%%%%%%%%%%%%%%%%%%%%%%%%%%%%%%%%%%%%%%%%
%% chapter 4 %%%%%%%%%%%%%%%%%%%%%%%%%%%%%%%%%%%%%%%%%%%%%
%%           %%%%%%%%%%%%%%%%%%%%%%%%%%%%%%%%%%%%%%%%%%%%%
%%%%%%%%%%%%%%%%%%%%%%%%%%%%%%%%%%%%%%%%%%%%%%%%%%%%%%%%%%
\chapter{テスト}
システムが正しく動作することを確認します。具体的には以下のような活動が行われます:




%%%%%%%%%%%%%%%%%%%%%%%%%%%%%%%%%%%%%%%%%%%%%%%%%%%%%%%%%%
%% section 1.1 %%%%%%%%%%%%%%%%%%%%%%%%%%%%%%%%%%%%%%%%%%%
%%%%%%%%%%%%%%%%%%%%%%%%%%%%%%%%%%%%%%%%%%%%%%%%%%%%%%%%%%
\section{ユニットテスト}
個々のコンポーネントが正しく動作するかどうかを確認します。




%%%%%%%%%%%%%%%%%%%%%%%%%%%%%%%%%%%%%%%%%%%%%%%%%%%%%%%%%%
%% section 1.1 %%%%%%%%%%%%%%%%%%%%%%%%%%%%%%%%%%%%%%%%%%%
%%%%%%%%%%%%%%%%%%%%%%%%%%%%%%%%%%%%%%%%%%%%%%%%%%%%%%%%%%
\section{統合テスト}
全体としてシステムが正しく動作するかどうかを確認します。



%%%%%%%%%%%%%%%%%%%%%%%%%%%%%%%%%%%%%%%%%%%%%%%%%%%%%%%%%%
%%           %%%%%%%%%%%%%%%%%%%%%%%%%%%%%%%%%%%%%%%%%%%%%
%% chapter 5 %%%%%%%%%%%%%%%%%%%%%%%%%%%%%%%%%%%%%%%%%%%%%
%%           %%%%%%%%%%%%%%%%%%%%%%%%%%%%%%%%%%%%%%%%%%%%%
%%%%%%%%%%%%%%%%%%%%%%%%%%%%%%%%%%%%%%%%%%%%%%%%%%%%%%%%%%
\chapter{保守}
システムが稼働した後も、新たな要件の追加やバグ修正など、継続的な更新が必要となります。具体的には以下のような活動が行われます




%%%%%%%%%%%%%%%%%%%%%%%%%%%%%%%%%%%%%%%%%%%%%%%%%%%%%%%%%%
%% section 1.1 %%%%%%%%%%%%%%%%%%%%%%%%%%%%%%%%%%%%%%%%%%%
%%%%%%%%%%%%%%%%%%%%%%%%%%%%%%%%%%%%%%%%%%%%%%%%%%%%%%%%%%
\section{修正保守}
バグ修正や性能改善など、既存の機能に対する修正を行います。




%%%%%%%%%%%%%%%%%%%%%%%%%%%%%%%%%%%%%%%%%%%%%%%%%%%%%%%%%%
%% section 1.1 %%%%%%%%%%%%%%%%%%%%%%%%%%%%%%%%%%%%%%%%%%%
%%%%%%%%%%%%%%%%%%%%%%%%%%%%%%%%%%%%%%%%%%%%%%%%%%%%%%%%%%
\section{適応保守}
環境変化(例えば、OSやハードウェアのアップデート)に対応するための更新を行います。




%%%%%%%%%%%%%%%%%%%%%%%%%%%%%%%%%%%%%%%%%%%%%%%%%%%%%%%%%%
%% section 1.1 %%%%%%%%%%%%%%%%%%%%%%%%%%%%%%%%%%%%%%%%%%%
%%%%%%%%%%%%%%%%%%%%%%%%%%%%%%%%%%%%%%%%%%%%%%%%%%%%%%%%%%
\section{機能追加保守}
新たな機能を追加するための更新を行います。




%%%%%%%%%%%%%%%%%%%%%%%%%%%%%%%%%%%%%%%%%%%%%%%%%%%%%%%%%%
%%            %%%%%%%%%%%%%%%%%%%%%%%%%%%%%%%%%%%%%%%%%%%%
%%            %%%%%%%%%%%%%%%%%%%%%%%%%%%%%%%%%%%%%%%%%%%%
%% BACKMATTER %%%%%%%%%%%%%%%%%%%%%%%%%%%%%%%%%%%%%%%%%%%%
%%            %%%%%%%%%%%%%%%%%%%%%%%%%%%%%%%%%%%%%%%%%%%%
%%            %%%%%%%%%%%%%%%%%%%%%%%%%%%%%%%%%%%%%%%%%%%%
%%%%%%%%%%%%%%%%%%%%%%%%%%%%%%%%%%%%%%%%%%%%%%%%%%%%%%%%%%
\backmatter



%%%%%%%%%%%%%%%%%%%%%%%%%%%%%%%%%%%%%%%%%%%%%%%%%%%%%%%%%%%%%%%%%%%%
%%                 %%%%%%%%%%%%%%%%%%%%%%%%%%%%%%%%%%%%%%%%%%%%%%%%%
%% LIST OF FIGURES %%%%%%%%%%%%%%%%%%%%%%%%%%%%%%%%%%%%%%%%%%%%%%%%%
%%                 %%%%%%%%%%%%%%%%%%%%%%%%%%%%%%%%%%%%%%%%%%%%%%%%%
%%%%%%%%%%%%%%%%%%%%%%%%%%%%%%%%%%%%%%%%%%%%%%%%%%%%%%%%%%%%%%%%%%%%
\addchaptertocentry{}{\listfigurename}
\listoffigures



\clearpage
%%%%%%%%%%%%%%%%%%%%%%%%%%%%%%%%%%%%%%%%%%%%%%%%%%%%%%%%%%%%%%%%%%%%
%%              %%%%%%%%%%%%%%%%%%%%%%%%%%%%%%%%%%%%%%%%%%%%%%%%%%%%
%% BIBLIOGRAPHY %%%%%%%%%%%%%%%%%%%%%%%%%%%%%%%%%%%%%%%%%%%%%%%%%%%%
%%              %%%%%%%%%%%%%%%%%%%%%%%%%%%%%%%%%%%%%%%%%%%%%%%%%%%%
%%%%%%%%%%%%%%%%%%%%%%%%%%%%%%%%%%%%%%%%%%%%%%%%%%%%%%%%%%%%%%%%%%%%
\addchaptertocentry{}{\bibname}
{\footnotesize%
%\setlength{\baselineskip}{12pt}
\printbibliography}




\end{document}