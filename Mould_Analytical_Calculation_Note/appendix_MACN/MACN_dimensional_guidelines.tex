%!TEX root = ../Mould_Analytical_Calculation_Note.tex

ここではマシニング用プログラムを記述する際やCADで描画をする際に必要となる、図面の数値等の読み取りかたについて触れる。

なお、前提として、特別な指定やその他特記事項がある場合は、それを優先するものとする。
以下では主にそうした特別な記述のない、いわゆる一般的な場合について記載する。




%%%%%%%%%%%%%%%%%%%%%%%%%%%%%%%%%%%%%%%%%%%%%%%%%%%%%%%%%%
%% section A.1 %%%%%%%%%%%%%%%%%%%%%%%%%%%%%%%%%%%%%%%%%%%
%%%%%%%%%%%%%%%%%%%%%%%%%%%%%%%%%%%%%%%%%%%%%%%%%%%%%%%%%%
\section{基本事項}
%% paragraph %%%%%%%%%%%%%%%%%%%%%
\paragraph{寸法公差の取扱い}\noindent
全体的に、寸法公差がある場合、$+$公差と$-$公差の中央(平均)を見るものとする。
たとえば、$100^{+0.5}_{\phantom -0}$であれば、100.25とみなす
%% footnote %%%%%%%%%%%%%%%%%%%%%
\footnote{内面のテーパ表を見る際はこの限りではないことに注意。}。
%%%%%%%%%%%%%%%%%%%%%%%%%%%%%%%%%

%% paragraph %%%%%%%%%%%%%%%%%%%%%
\paragraph{寸法の優先度}\noindent
公差のある寸法と公差のない寸法(括弧寸法含む)とが共存して記載されている場合、公差のある寸法を優先する。
たとえば、2つの線の寸法がそれぞれ$12^{+0.1}_{\phantom -0}$, $4.05$と記述されていて、かつその和に相当する部分の寸法が16と記述されている場合は、16.10とみなす。




%%%%%%%%%%%%%%%%%%%%%%%%%%%%%%%%%%%%%%%%%%%%%%%%%%%%%%%%%%
%% section A.2 %%%%%%%%%%%%%%%%%%%%%%%%%%%%%%%%%%%%%%%%%%%
%%%%%%%%%%%%%%%%%%%%%%%%%%%%%%%%%%%%%%%%%%%%%%%%%%%%%%%%%%
\section{振分け}
振分長の公差については、全長の公差をトップ振分長とボトム振分長との比率で分配する。
たとえば、全長が$1000^{\phantom +0}_{-1.0}$でトップ振分長が200であれば、全長の公差分$-0.5$を振分長の比$200:800$に分配し、それぞれ$-0.1$, $-0.4$とする。
つまり、トップ振分長は199.9, ボトム振分長は799.6とみなす。

ただし、簡単のため、単純に2等分してもよいものとする。
すなわち、上記の例でいうと、トップ振分長を199.75, ボトム振分長を799.75とみなしてもよいものとする
%% footnote %%%%%%%%%%%%%%%%%%%%%
\footnote{もう少し正確には、公差の分配は両者の間に収まっていればよいものとする。
すなわち、上記の例でいうと、トップ振分長は199.75~199.90に収まっていればよいものとする。}。
%%%%%%%%%%%%%%%%%%%%%%%%%%%%%%%%%

%% paragraph %%%%%%%%%%%%%%%%%%%%%
\paragraph{括弧寸法の場合}\noindent
片方の振分長が括弧寸法の場合は、全長の公差をそのまま括弧寸法に割り当てる。
たとえば、全長が$1000^{\phantom +0}_{-1.0}$でトップ振分長が200, ボトム振分長が(800)であれば、トップ振分長は200, ボトム振分長は799.5とする。




%%%%%%%%%%%%%%%%%%%%%%%%%%%%%%%%%%%%%%%%%%%%%%%%%%%%%%%%%%
%% section A.3 %%%%%%%%%%%%%%%%%%%%%%%%%%%%%%%%%%%%%%%%%%%
%%%%%%%%%%%%%%%%%%%%%%%%%%%%%%%%%%%%%%%%%%%%%%%%%%%%%%%%%%
\section{外径\label{app:gaikei}}
プログラムを記述する際は、簡単のため、端面部の水平方向の長さは、モールドの外径(中心湾曲と水平な方向)とみなしてもよいものとする。

実際には、中心湾曲を$R$, トップ振分長を$f_\mathrm T$, 外径を$W_x$とすると、トップ端面部の水平方向の長さ$W_\mathrm T$は以下で与えられる。(ボトム端面部も同様)
\begin{align*}
  W_\mathrm T
  = \sqrt{\left(R+\frac{W_x}2\right)^{\!2}-f_\mathrm T^2}
    -\sqrt{\left(R-\frac{W_x}2\right)^{\!2}-f_\mathrm T^2}\ .
\end{align*}
\begin{Column}{近似計算}
テイラー展開(マクローリン展開)より、
\begin{align*}
  (1+x)^\frac12 = 1+\frac x2-\frac{x^2}8+\frac{x^3}{16}-\frac{5x^4}{128}+o\!\left(x^5\right)
\end{align*}
なので、
\begin{align*}
  & (1+x)^\frac12(1+y)^\frac12-(1-x)^\frac12(1-y)^\frac12\\
  &= x+y+\frac{(x+y)(x-y)^2}8-\frac{xy(x+y)\big\{5(x-y)^2+7xy\big\}}{128}+\cdots\ .
\end{align*}
したがって、
\begin{align*}
  x = \frac{\nicefrac{W_x}2+f_\mathrm T}R\ ,\quad y = \frac{\nicefrac{W_x}2-f_\mathrm T}R\quad
  \longrightarrow \quad
  x+y = \frac{W_x}R\ , \quad x-y = \frac{2f_\mathrm T}R
\end{align*}
であるので、
\begin{align*}
  W_\mathrm T
  = R\left\{(1+x)^\frac12(1+y)^\frac12-(1-x)^\frac12(1-y)^\frac12\right\}
  = W_x\!\left(1+\frac{f_\mathrm T^2}{2R^2}+\cdots\right).
\end{align*}
\end{Column}




%%%%%%%%%%%%%%%%%%%%%%%%%%%%%%%%%%%%%%%%%%%%%%%%%%%%%%%%%%
%% section A.4 %%%%%%%%%%%%%%%%%%%%%%%%%%%%%%%%%%%%%%%%%%%
%%%%%%%%%%%%%%%%%%%%%%%%%%%%%%%%%%%%%%%%%%%%%%%%%%%%%%%%%%
\section{内径}
プログラムを記述する際は、簡単のため、端面部の水平方向の長さは、モールドの外径(中心湾曲と水平な方向)とみなしてもよいものとする。
これは外径(\pageautoref{app:gaikei})と同様である。

%% paragraph %%%%%%%%%%%%%%%%%%%%%
\paragraph{内面テーパおよびテーパ表}\noindent
テーパ表を参照する際は、全長の公差は考慮しないものとする。
また、トップ端からの距離のピッチも、同様に公差は考慮しないものとする。

たとえば、全長が$800^{+0.5}_{\phantom -0}$, トップ振分長が400, ピッチが25である場合を考える。
このとき、トップ端は振分中心から400の位置にあり、ピッチは25であるものとし、両端についてはそれを適宜延長して調整する。