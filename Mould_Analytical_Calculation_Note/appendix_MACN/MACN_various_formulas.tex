%!TEX root = ../Mould_Analytical_Calculation_Note.tex





%%%%%%%%%%%%%%%%%%%%%%%%%%%%%%%%%%%%%%%%%%%%%%%%%%%%%%%%%%
%% section C.1 %%%%%%%%%%%%%%%%%%%%%%%%%%%%%%%%%%%%%%%%%%%
%%%%%%%%%%%%%%%%%%%%%%%%%%%%%%%%%%%%%%%%%%%%%%%%%%%%%%%%%%
\section{2点間の距離}
\begin{Column}{}
点($p$, $q$)と直線$ax+by+c=0$との距離$d$は、以下で与えられる。
\begin{align*}
  d = \frac{|ap+bq+c|}{\sqrt{a^2+b^2}}.
\end{align*}
\end{Column}
\begin{Column}{}
点$\boldsymbol p$を通り方向ベクトルが$\boldsymbol m$の直線L上の点と、点$\boldsymbol q$を通り方向ベクトルが$\boldsymbol m'$の直線$\mathrm L'$上の点は、それぞれパラメータ$t$, $t'$を用いて、
\begin{align*}
  \mathrm L: \boldsymbol p+t\boldsymbol m\ , \qquad
  \mathrm L': \boldsymbol q+t'\boldsymbol m'
\end{align*}
で表される。
このとき、L上の点の中で$\mathrm L'$に最も近づく点の位置$\boldsymbol k$は、以下で与えられる
%% footnote %%%%%%%%%%%%%%%%%%%%%
\footnote{2点間の距離の2乗$|\boldsymbol p-\boldsymbol q+t\boldsymbol m-t'\boldsymbol m'|^2$に対し、それぞれのパラメータ$t$, $t'$に関する微分が0となる。
それらを連立して解けば$\boldsymbol k$, $\boldsymbol k'$が求まる。}。
%%%%%%%%%%%%%%%%%%%%%%%%%%%%%%%%%
$\mathrm L'$上の点の中でLに最も近づく点の位置$\boldsymbol k'$についても同様である。
\begin{align*}
  \boldsymbol k
  = \boldsymbol p
    +\frac{(\boldsymbol m-(\boldsymbol m, \boldsymbol m')\boldsymbol m', \boldsymbol p-\boldsymbol q)}
          {1+(\boldsymbol m, \boldsymbol m')^2}\boldsymbol m
\end{align*}
また、これらの差の大きさ$|\boldsymbol k-\boldsymbol k'|$から、2直線間の距離$d$が求まる。
\end{Column}
