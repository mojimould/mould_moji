%!TEX root = ../Mould_Analytical_Calculation_Note.tex

\begin{tcolorbox}[title={2023/07/28時点の\MMname 実測値}, fonttitle=\gtfamily\bfseries]
\begin{align*}
  \text{Bot ($B=0$)}
  \left\{
  \begin{array}{rl}
    X: & 97.790 \sim 99.930\\
    Y: & -823.850\\
    Z: & -634.620
  \end{array}
  \right.\quad
  \text{Top ($B=180.$)}
  \left\{
  \begin{array}{rl}
    X: & -97.980 \sim -99.570\\
    Y: & -823.780\\
    Z: & -634.720
  \end{array}
  \right.
\end{align*}\\
・$X$については、ジグの当たる点の凸部と端部($Z$方向は目分量)\\
・$Y$については、モールドの底が当たる面\\
・$Z$については、$X0$ $Y-850.$における、ジグとの接点\\
※これらの値に、センサー先端球の半径を加減する必要がある
\end{tcolorbox}
%%%%%%%%%%%%%%%%%%
\begin{tcolorbox}[title={\DMname における各々の値(図面上の値)}, fonttitle=\gtfamily\bfseries]
・ジグ長さ$2l$:660 \quad ・ジグ幅:455\\
・テーブル中心 と C面側ジグ端 との水平距離:196.5\\
・受板の円の半径$\rho$:100 \quad ・受板の鉛直方向の幅$\sigma$:40\\
・テーブル中心 と 受板の円の中心 との水平距離$\varDelta$:201.5\\
・受板の円の中心 と 受板の水平方向の底 との距離:70\\[12pt]
2023/09/26時点の\DMname における各々の実測値\\
・テーブル回転中心の$X$座標:$-550.019$ \quad ・テーブル回転中心の$Z$座標:$-1149.974$
\end{tcolorbox}




%%%%%%%%%%%%%%%%%%%%%%%%%%%%%%%%%%%%%%%%%%%%%%%%%%%%%%%%%%
%% section 2.1 %%%%%%%%%%%%%%%%%%%%%%%%%%%%%%%%%%%%%%%%%%%
%%%%%%%%%%%%%%%%%%%%%%%%%%%%%%%%%%%%%%%%%%%%%%%%%%%%%%%%%%
\section{テーブル中心にあるモールド}
CADによる描画において、テーブルの回転中心が原点(ワールド原点)に置かれているとする。
ここでモールドを描画する際、モールドの中心
%% footnote %%%%%%%%%%%%%%%%%%%%%
\footnote{$R_\mathrm c$に相当する点。}\relax
%%%%%%%%%%%%%%%%%%%%%%%%%%%%%%%%%
をCAD上の原点(ワールド原点)にして描くほうが都合のいいことがある。
このとき、モールドと受板が接するように移動する必要がある。
鉛直方向(トップ-ボトム方向)においては$f_d$だけ動かせばよいが、水平方向の移動距離はあまり自明とはいいがたい。
C側モールド面と受板面との寸法を単純に測ると、(水平方向でなく)最短距離が測定されてしまう。
工夫により水平方向の距離を出すことも可能ではあるが、ここではその距離を定量的に求めておく。



%%%%%%%%%%%%%%%%%%%%%%%%%%%%%%%%%%%%%%%%%%%%%%%%%%%%%%%%%%
%% subsection 2.1.1 %%%%%%%%%%%%%%%%%%%%%%%%%%%%%%%%%%%%%%
%%%%%%%%%%%%%%%%%%%%%%%%%%%%%%%%%%%%%%%%%%%%%%%%%%%%%%%%%%
\subsection{スペーサ取付前}
(スペーサを取付る前の)モールドの中心がテーブル中心Pに置かれている場合を考える。
ボトム側の受板に接するモールドの点と、テーブル中心Pとは、実軸方向に
\begin{align*}
  R_\mathrm c-R_\mathrm i\cos\alpha_{\mathrm U_\mathrm B}
\end{align*}
だけ差がある。
したがって、モールドの受板と接する点の位置は実軸方向に、
\begin{align*}
  \varDelta+\sqrt{R_\mathrm i'^2-\bar l^2}-R_\mathrm c+R_\mathrm i\cos\alpha_{\mathrm U_\mathrm B}\ .
\end{align*}
そのため\pageeqref{eq:afterUBcontact} ($\delta = 0$)より、モールドと(ボトム側の)受板は
\begin{align*}
  \varDelta+\sqrt{R_\mathrm i'^2-\bar l^2}-R_\mathrm c
\end{align*}
だけ実軸方向に離れていることがわかる。
\pageeqref{eq:tableCenter}より、これはテーブル中心Pとモールドの中心湾曲$R_\mathrm c$との差であることがわかる。



%%%%%%%%%%%%%%%%%%%%%%%%%%%%%%%%%%%%%%%%%%%%%%%%%%%%%%%%%%
%% subsection 2.1.2 %%%%%%%%%%%%%%%%%%%%%%%%%%%%%%%%%%%%%%
%%%%%%%%%%%%%%%%%%%%%%%%%%%%%%%%%%%%%%%%%%%%%%%%%%%%%%%%%%
\subsection{スペーサ取付後}
厚さ$\delta\,(>0)$のスペーサを取付けた場合、モールドの受板と接する点とテーブル中心Pとは、実軸方向に
\begin{align*}
  R_\mathrm c-R_\mathrm i\cos\alpha'_{\mathrm U_\mathrm B}
\end{align*}
だけ差があるので、その実軸方向の位置は、
\begin{align*}
  \varDelta+\sqrt{R_\mathrm i'^2-\bar l^2}-R_\mathrm c+R_\mathrm i\cos\alpha'_{\mathrm U_\mathrm B}\ .
\end{align*}
そのため\pageeqref{eq:afterUBcontact}より、モールドと(ボトム側の)受板は
\begin{align*}
  &  \varDelta+\sqrt{R_\mathrm i'^2-\bar l^2}-R_\mathrm c+R_\mathrm i\cos\alpha'_{\mathrm U_\mathrm B}
     -\left(R_\mathrm i'\cos\alpha_{\mathrm U_\mathrm B}+\rho\cos\alpha'_{\mathrm U_\mathrm B}\right)\\
  &= \varDelta-R_\mathrm c+R_\mathrm i'\cos\alpha'_{\mathrm U_\mathrm B}\\
  &= \varDelta-R_\mathrm c
     -\frac\delta2+\sqrt{R_\mathrm i'^2-\frac{\delta^2+(2\bar l)^2}4}\frac{2\bar l}{\sqrt{\delta^2+(2\bar l)^2}}
\end{align*}
だけ実軸方向に離れていることがわかる。
\pageeqref{eq:tableCenter}および\pageeqref{eq:spacerMoveHdistance}より、これはスペーサ取付け後のモールド中心とモールドの中心湾曲$R_\mathrm c$との差であることがわかる。
