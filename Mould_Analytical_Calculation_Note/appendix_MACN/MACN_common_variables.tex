%!TEX root = ../Mould_Analytical_Calculation_Note.tex

%%%%%%%%%%%%%%%%%%%%%%%%%%%%%%%%%%%%%%%%%%%%%%%%%%%%%%%%%%
%% section B.1 %%%%%%%%%%%%%%%%%%%%%%%%%%%%%%%%%%%%%%%%%%%
%%%%%%%%%%%%%%%%%%%%%%%%%%%%%%%%%%%%%%%%%%%%%%%%%%%%%%%%%%
\section{北村マシニング}
北村(横型)マシニングで取り決めているコモン変数について、以下に挙げておく。



%%%%%%%%%%%%%%%%%%%%%%%%%%%%%%%%%%%%%%%%%%%%%%%%%%%%%%%%%%
%% subsection B.1.1 %%%%%%%%%%%%%%%%%%%%%%%%%%%%%%%%%%%%%%
%%%%%%%%%%%%%%%%%%%%%%%%%%%%%%%%%%%%%%%%%%%%%%%%%%%%%%%%%%
\subsection{コモン変数 (\#400-\#999)}
\begin{twoCtable}{\paragraph{コモン変数:北村マシニング}}
\#401 & パレット\#1 テーブル(ジグ)中心機械座標$X$\\\hline
\#402 & パレット\#1 テーブル(ジグ)中心機械座標$Y$\\\hline
\#403 & パレット\#1 テーブル(ジグ)中心機械座標$Z$\\\hline
\#404 & パレット\#1 テーブル(ジグ)中心機械座標$B$\\\hline
\#405 & パレット\#2 テーブル(ジグ)中心機械座標$X$\\\hline
\#406 & パレット\#2 テーブル(ジグ)中心機械座標$Y$\\\hline
\#407 & パレット\#2 テーブル(ジグ)中心機械座標$Z$\\\hline
\#408 & パレット\#2 テーブル(ジグ)中心機械座標$B$\\\hline
\#409 & 工具中心機械座標$C$\\\hline
\#410 & (予備)\\\hline
\#411 & ジグ外側幅(機械座標系$B$0における$Z$方向)\\\hline
\#412 & ジグ内側幅(機械座標系$B$0における$Z$方向)\\\hline
\#413 & ジグ幅(機械座標系$B$0における$X$方向)\\\hline
\#501 & タッチセンサー信号遅れ補正\\\hline
\#502 & タッチセンサープローブ中心$X$補正\\\hline
\#503 & タッチセンサープローブ中心$Y$補正\\\hline
\#512 & タッチセンサープローブ半径\\\hline
\#514 & スキップ(G31)速さ(未指定時)\\\hline
\#600 & 振分調整用角度$-\theta[\deg]$\\\hline
\#601 & 工具T31(Tスロット)A側内面溝用補正\\\hline
\#602 & 工具T31(Tスロット)C側内面溝用補正\\\hline
\#603 & 工具T31(Tスロット)B側内面溝用補正\\\hline
\#604 & 工具T31(Tスロット)D側内面溝用補正\\\hline
\#605 & 工具T32(Tスロット)A側内面溝用補正\\\hline
\#606 & 工具T32(Tスロット)C側内面溝用補正\\\hline
\#607 & 工具T32(Tスロット)B側内面溝用補正\\\hline
\#608 & 工具T32(Tスロット)D側内面溝用補正\\\hline
\#609 & 工具T33(Tスロット)A側内面溝用補正\\\hline
\#610 & 工具T33(Tスロット)C側内面溝用補正\\\hline
\#611 & 工具T33(Tスロット)B側内面溝用補正\\\hline
\#612 & 工具T33(Tスロット)D側内面溝用補正\\\hline
\#613 & 工具T06(サイドカッター)厚さ$t$\\\hline
\#614 & (予備)\\\hline
\#615 & 工具T08(サイドカッター)厚さ$t$\\\hline
\hline
\#701-733 & 内面溝(Tスロット)レベル1サブプログラム用\\\hline
\#734-766 & 内面溝(Tスロット)レベル2サブプログラム用\\\hline
\#767-799 & 内面溝(Tスロット)レベル3サブプログラム用\\\hline
\#801-833 & 内面溝(アングルヘッド)レベル1サブプログラム用\\\hline
\#834-866 & 内面溝(アングルヘッド)レベル2サブプログラム用\\\hline
\#867-899 & 内面溝(アングルヘッド)レベル3サブプログラム用
\end{twoCtable}



%%%%%%%%%%%%%%%%%%%%%%%%%%%%%%%%%%%%%%%%%%%%%%%%%%%%%%%%%%
%% subsection B.1.2 %%%%%%%%%%%%%%%%%%%%%%%%%%%%%%%%%%%%%%
%%%%%%%%%%%%%%%%%%%%%%%%%%%%%%%%%%%%%%%%%%%%%%%%%%%%%%%%%%
\subsection{コモン変数 (\#900000-\#901000)}
\begin{twoCtable}{\paragraph{コモン変数:北村マシニング}}
\#900001 & (O100001) $X$外中心測定 $-X$側測定値\\\hline
\#900002 & (O100001) $X$外中心測定 $+X$側測定値\\\hline
\#900003 & (O100001) $X$外中心測定値\\\hline
\#900004 & (O100001) $X$外中心測定 厚さ測定値\\\hline
\#900005 & (O100002) $Y$外中心測定 $-Y$側測定値\\\hline
\#900006 & (O100002) $Y$外中心測定 $+Y$側測定値\\\hline
\#900007 & (O100002) $Y$外中心測定値\\\hline
\#900008 & (O100002) $Y$外中心測定 厚さ測定値\\\hline
\#900009 & (O100011) $X$内中心測定 $-X$側測定値\\\hline
\#900010 & (O100011) $X$内中心測定 $+X$側測定値\\\hline
\#900011 & (O100011) $X$内中心測定値\\\hline
\#900012 & (O100011) $X$内中心測定 厚さ測定値\\\hline
\#900013 & (O100012) $Y$内中心測定 $-Y$側測定値\\\hline
\#900014 & (O100012) $Y$内中心測定 $+Y$側測定値\\\hline
\#900015 & (O100012) $Y$内中心測定値\\\hline
\#900016 & (O100012) $Y$内中心測定 厚さ測定値\\\hline
\#900017 & (O100101) $X$外削中心測定 内面測定値\\\hline
\#900018 & (予備)\\\hline
\#900019 & (O101002) $Y$通り芯 ボトム側測定値\\\hline
\#900020 & (O101002) $Y$通り芯 トップ側測定値\\\hline
\#900021 & (O101002) $Y$通り芯 測定値\\\hline
\#900022 & (O101003) $X$通り芯 ボトム側測定値\\\hline
\#900023 & (O101003) $X$通り芯 トップ側測定値\\\hline
\#900024 & (O101003) $X$通り芯 測定値\\\hline
\#900101-\#900200 & (O230001) A側内面溝 深さ測定値(Tスロット)\\\hline
\#900201-\#900300 & (O230001) C側内面溝 深さ測定値(Tスロット)\\\hline
\#900301-\#900400 & (O230002) B側内面溝 深さ測定値(Tスロット)\\\hline
\#900401-\#900500 & (O230002) D側内面溝 深さ測定値(Tスロット)
\end{twoCtable}




%%%%%%%%%%%%%%%%%%%%%%%%%%%%%%%%%%%%%%%%%%%%%%%%%%%%%%%%%%
%% subsection B.1.2 %%%%%%%%%%%%%%%%%%%%%%%%%%%%%%%%%%%%%%
%%%%%%%%%%%%%%%%%%%%%%%%%%%%%%%%%%%%%%%%%%%%%%%%%%%%%%%%%%
\subsection{システム変数}
北村(横型)マシニングのシステム変数について、主なものを以下に挙げておく。

\begin{twoCtable}{\paragraph{システム変数:北村マシニング}}
\#1000 & パレット\#~~0:\#1, 1:\#2\\\hline
\#1004 & タッチセンサー電源~~0: off, 1: on\\\hline
\#1005 & タッチセンサー電池残量~~0: ok, 1: low\\\hline
\#2000+xx & 工具長補正 \#xx補正量(摩耗0とした値, xx=1-200)\\\hline
\#2200+xx & 工具長補正 \#xx摩耗 (xx=1-200)\\\hline
\#2400+xx & 工具径補正 \#xx補正量(摩耗0とした値, xx=1-200)\\\hline
\#2600+xx & 工具径補正 \#xx摩耗 (xx=1-200)\\\hline
\#3000 & アラーム\\\hline
\#3011 & 現在の年月日(yyyymmdd)\\\hline
\#4012 & 現在のワーク座標系\# (G\#)\\\hline
\#4111 & 直前の工具長補正コード\# (H\#)\\\hline
\#4113 & 直前のブロック指令 Mコード\# (M\#)\\\hline
\#4114 & 直前のブロック指令 シーケンス\# (N\#)\\\hline
\#4115 & 直前のブロック指令 プログラム\# (O\#)\\\hline
\#4120 & 直前のブロック指令 工具コード\# (T\#)\\\hline
\#500x & ブロック終点位置 1:X, 2:Y, 3:Z, 4:B, 5:C(ワーク座標系)
\footnote{\#500x:工具補正値を引いた値。途中でスキップがオンになったときはそのときの値。}\\\hline
\#502x & 現在の機械座標系の座標 1:X, 2:Y, 3:Z, 4:B, 5:C\\\hline
\#504x & 現在のワーク座標系の座標 1:X, 2:Y, 3:Z, 4:B, 5:C\\\hline
\#506x & スキップ座標 1:X, 2:Y, 3:Z, 4:B, 5:C(工具補正0とした値)\\\hline
\#522x & ワーク座標系G54原点の機械座標 1:X, 2:Y, 3:Z, 4:B, 5:C\\\hline
\#524x & ワーク座標系G55原点の機械座標 1:X, 2:Y, 3:Z, 4:B, 5:C\\\hline
\#526x & ワーク座標系G56原点の機械座標 1:X, 2:Y, 3:Z, 4:B, 5:C\\\hline
\#528x & ワーク座標系G57原点の機械座標 1:X, 2:Y, 3:Z, 4:B, 5:C\\\hline
\#530x & ワーク座標系G58原点の機械座標 1:X, 2:Y, 3:Z, 4:B, 5:C\\\hline
\#532x & ワーク座標系G59原点の機械座標 1:X, 2:Y, 3:Z, 4:B, 5:C\\\hline
\#10000+xx & 工具長補正 \#xx補正量(摩耗0とした値, xx=1-200)
\end{twoCtable}




\clearpage
%%%%%%%%%%%%%%%%%%%%%%%%%%%%%%%%%%%%%%%%%%%%%%%%%%%%%%%%%%
%% section B.2 %%%%%%%%%%%%%%%%%%%%%%%%%%%%%%%%%%%%%%%%%%%
%%%%%%%%%%%%%%%%%%%%%%%%%%%%%%%%%%%%%%%%%%%%%%%%%%%%%%%%%%
\section{三菱マシニング}



%%%%%%%%%%%%%%%%%%%%%%%%%%%%%%%%%%%%%%%%%%%%%%%%%%%%%%%%%%
%% subsection B.2.1 %%%%%%%%%%%%%%%%%%%%%%%%%%%%%%%%%%%%%%
%%%%%%%%%%%%%%%%%%%%%%%%%%%%%%%%%%%%%%%%%%%%%%%%%%%%%%%%%%
\subsection{コモン変数}
三菱マシニングで取り決めているコモン変数について、以下に挙げておく。

\begin{twoCtable}{コモン変数:北村マシニング}
\#145 & スキップ(G31)速さ\\
\end{twoCtable}




%%%%%%%%%%%%%%%%%%%%%%%%%%%%%%%%%%%%%%%%%%%%%%%%%%%%%%%%%%
%% subsection B.1.2 %%%%%%%%%%%%%%%%%%%%%%%%%%%%%%%%%%%%%%
%%%%%%%%%%%%%%%%%%%%%%%%%%%%%%%%%%%%%%%%%%%%%%%%%%%%%%%%%%
\subsection{システム変数}
三菱マシニングのシステム変数について、主なものを以下に挙げておく。

\begin{twoCtable}{システム変数:北村マシニング}
\#4111 & 現在の工具長補正 Hコード\#\\\hline
\#4120 & 現在の工具 Tコード\#\\\hline
\#502x & 現在の機械座標系の座標 1:X, 2:Y, 3:Z, 4:B\\\hline
\#504x & 現在のワーク座標系の座標 1:X, 2:Y, 3:Z, 4:B\\\hline
\#506x & スキップ座標 1:X, 2:Y, 3:Z, 4:B(工具補正0とした値)\\\hline
\#522x & ワーク座標系G54原点の機械座標 1:X, 2:Y, 3:Z, 4:B\\\hline
\#524x & ワーク座標系G55原点の機械座標 1:X, 2:Y, 3:Z, 4:B\\\hline
\#526x & ワーク座標系G56原点の機械座標 1:X, 2:Y, 3:Z, 4:B\\\hline
\#528x & ワーク座標系G57原点の機械座標 1:X, 2:Y, 3:Z, 4:B\\\hline
\#530x & ワーク座標系G58原点の機械座標 1:X, 2:Y, 3:Z, 4:B\\\hline
\#532x & ワーク座標系G59原点の機械座標 1:X, 2:Y, 3:Z, 4:B\\
\end{twoCtable}


