%!TEX root = ../Mould_Analytical_Calculation_Note.tex

%%%%% DOCUMENTCLASS %%%%%%%%%%%%%%%%%%%%%%%%%%%%
\documentclass[12pt]{scrbook}
%%%%%%%%%%%%%%%%%%%%%%%%%%%%%%%%%%%%%%%%%%%%%%

\input{preamble_Mould_Analytical_Calculation_Note/package_Mould_Analytical_Calculation_Note}
\input{preamble_Mould_Analytical_Calculation_Note/definition_Mould_Analytical_Calculation_Note}

%%%%% GEOMETRY %%%%%%%%%%%%%%%%%%%%%%%%%%%%
\geometry{
  a4paper,  % paper size
  twoside,
  text = {6.4in, 9.6in},
  centering,
  includehead,  % include the head of the page
%  headheight = 13.6pt,
  includefoot,  % include the foot of the page
}
%%%%% LINESPREAD %%%%%%%%%%%%%%%%%%%%%%%%%%%%
\linespread{1.15}\selectfont
%%%%% PARINDENT %%%%%%%%%%%%%%%%%%%%%%%%%%%%
\setlength\parindent{12pt}
%\def \globalscale {0.83}
%%%%%%%%%%%%%%%%%%%%%%%%%%%%%%%%%%%%%%%%%%%%%%
%%%%% UNIT LENGTH %%%%%%%%%%%%%%%%%%%%%%%%%%%%
\setlength{\unitlength}{1pt}
%%%%%%%%%%%%%%%%%%%%%%%%%%%%%%%%%%%%%%%%%%%%%%
%%%%% FOOTNOTE %%%%%%%%%%%%%%%%%%%%%%%%%%%%
\counterwithout{footnote}{chapter}
\def\@makefnmark{\hbox{}\hbox{\@textsuperscript{\normalfont\@thefnmark}}\hbox{}}
\deffootnote[1em]{1em}{1em}{\textsuperscript{\thefootnotemark}}
\renewcommand\footnoterule{%
  \kern3\p@
  \hrule\@width.75\columnwidth
  \kern2.6\p@
}
%%%%%%%%%%%%%%%%%%%%%%%%%%%%%%%%%%%%%%%%%%%%%%
%%%%% DISPLAYBREAK %%%%%%%%%%%%%%%%%%%%%%%%%
\allowdisplaybreaks
%%%%%%%%%%%%%%%%%%%%%%%%%%%%%%%%%%%%%%%%%%%%%%
%%%%% CAPTION WIDTH %%%%%%%%%%%%%%%%%%%%%%%%%%%%
\captionsetup{width=.8\textwidth}
%%%%%%%%%%%%%%%%%%%%%%%%%%%%%%%%%%%%%%%%%%%%%%
%%%%% NAME, AUTOREFNAME %%%%%%%%%%%%%%%%%%%%%%%%%%%%
%\renewcommand{\partautorefname}{part}  % part --> part
\renewcommand{\chapterautorefname}{章}  % chapter --> 章
\renewcommand{\sectionautorefname}{節\!} % section --> 節
\renewcommand{\subsectionautorefname}{\sectionautorefname} % subsection --> section
\renewcommand{\subsubsectionautorefname}{節} % subsubsection --> section
\renewcommand{\appendixname}{補\hskip.5zw 遺} % appendix --> 補 遺
\renewcommand{\appendixautorefname}{補遺\!} % appendix --> 補遺
\renewcommand{\figurename}{図}
\renewcommand{\figureautorefname}{\figurename} % figure --> 図
\renewcommand{\footnoteautorefname}{脚注}
\newcommand{\tcb@cnt@hosokuboxautorefname}{補足}
%\newcommand{\tcb@cnt@Columnautorefname}{Column}
%\newcommand{\subfigureautorefname}{\figureautorefname} % subfigure --> figure
%\renewcommand{\tableautorefname}{表}
%\newcommand{\subtableautorefname}{\tableautorefname}
\renewcommand\bibname{\hbox to 5zw{参考文献}}
%%%%%%%%%%%%%%%%%%%%%%%%%%%%%%%%%%%%%%%%%%%%%%
%%%%% HEADER AND FOOTER %%%%%
\pagestyle{fancy}
\renewcommand{\chaptermark}[1]{\markboth{#1}{}}
\renewcommand{\headrulewidth}{1.5pt}
%\renewcommand{\footrulewidth}{1pt}

\fancypagestyle{front}{
  \fancyhead{} % clear all header fields
  \fancyhead[RO]{\thepage}
  \chead{\leftmark}
  \fancyhead[LE]{\thepage}
  \fancyhead[RE]{}
  \fancyfoot{}
}
\fancypagestyle{main}{
  \fancyhead{} % clear all header fields
  \fancyhead[LO]{\nouppercase\rightmark}
  \fancyhead[RO]{$\nicefrac{\thepage\,}{\pageref{LastPage}}$}
  \fancyhead[RE]{\thechapter\hskip1zw\nouppercase\leftmark}
  \fancyhead[LE]{$\nicefrac{\thepage\,}{\pageref{LastPage}}$}
  \fancyfoot{} % clear all footer fields
}
\fancypagestyle{plainheadfront}{
  \fancyhead{}
  \fancyhead[RO]{\thepage}
  \fancyhead[LE]{\thepage}
  \fancyfoot{}
}
\fancypagestyle{plainhead}{
  \fancyhead{}
  \fancyhead[RO]{$\nicefrac{\thepage\,}{\pageref{LastPage}}$}
  \fancyhead[LE]{$\nicefrac{\thepage\,}{\pageref{LastPage}}$}
  \fancyfoot{}
}
%%%%% SETLIST %%%%%%%%%%%%%%%%%%%%%%%%%%%%%%%%
\setlist[enumerate]{listparindent=\parindent, parsep=\z@, partopsep=\z@, topsep=3pt, itemsep=3pt}
%%%%%%%%%%%%%%%%%%%%%%%%%%%%%%%%%%%%%%%%%%%%%%
%%%%% APPENDICES %%%%%%%%%%%%%%%%%%%%%%%%%%%%
\renewcommand{\setthesection}{\Alph{section}}
%%%%%%%%%%%%%%%%%%%%%%%%%%%%%%%%%%%%%%%%%%%%%%
%%%%% EQUATION %%%%%%%%%%%%%%%%%%%%%%%%%%%%
\renewcommand{\theequation}{\thesection.\arabic{equation}}
\@addtoreset{equation}{section}
%%%%%%%%%%%%%%%%%%%%%%%%%%%%%%%%%%%%%%%%%%%%%%
%%%%% STYLE OF PARAGRAPH %%%%%%%%%%%%%%%%%%%%%
%for scrbook.cls
\RedeclareSectionCommand[%
  style=section,%
  level=4,%
  indent=\z@,%
  beforeskip=3.25ex \@plus1ex \@minus.2ex,%
  afterskip=0.1ex \@plus.1ex \@minus.1ex,% -1em から変更
  tocindentfollows=subsubsection,%
  tocstyle=section,%
  tocindent=10em,%
  tocnumwidth=5em,%
  font=\raggedsection\normalfont\sectfont\gtfamily\nobreak\sball{blue}~
]{paragraph}
%for book.cls
%\renewcommand\paragraph[1]{%
%  \@startsection{paragraph}{\paragraphnumdepth}{\z@}%
%  {3.25ex \@plus1ex \@minus.2ex}% \@plus, \@minusは伸び縮みできるスペースの長さ
%  {0.1ex\@plus.1ex \@minus.1ex}% ここが正だと改行されて、値だけ垂直スペースが入る
%  {\raggedsection\normalfont\sectfont\gtfamily\nobreak\size@paragraph\sball{blue}~}{#1}\noindent
%}
%%%%%%%%%%%%%%%%%%%%%%%%%%%%%%%%%%%%%%%%%%%%%%
\RedeclareSectionCommand[%
  style=section,%
  level=5,%
  indent=\z@,% \scr@parindent から変更
  beforeskip=0.5ex \@plus1ex \@minus .2ex,% 3.25ex \@plus1ex \@minus .2ex から変更
  afterskip=0.1ex \@plus.1ex \@minus.1ex,% -1em から変更
  tocstyle=section,%
  tocindent=12em,%
  tocnumwidth=6em%
]{subparagraph}
%%%%% STYLE OF TABLE OF CONTENTS %%%%%
\setcounter{secnumdepth}{3}
\setcounter{tocdepth}{3}
\renewcommand\contentsname{目 次}

\makeatother