%!TEX root = Mould_Analytical_Calculation_Note.tex



%%%%%%%%%%%%%%%%%%%%%%%%%%%%%%%%%%%%%%%%%%%%%%%%%%%%%%%%%%
%%           %%%%%%%%%%%%%%%%%%%%%%%%%%%%%%%%%%%%%%%%%%%%%
%% chapter 1 %%%%%%%%%%%%%%%%%%%%%%%%%%%%%%%%%%%%%%%%%%%%%
%%           %%%%%%%%%%%%%%%%%%%%%%%%%%%%%%%%%%%%%%%%%%%%%
%%%%%%%%%%%%%%%%%%%%%%%%%%%%%%%%%%%%%%%%%%%%%%%%%%%%%%%%%%
\modHeadchapter{振分け}
%\label{chap:Huriwake}
モールドの振分けの長さ(振分長)は、トップ側とボトム側では一般に異なる。
しかし、加工をする際には、ジグの中心に対して両者の長さの差が小さいほうが一般的には好都合である。
そうした場合の対処法として、ここでは以下のような2つの方法を考える。
\begin{enumerate}
\item
適当な厚さのスペーサをモールドとジグの接点に取り付けることで、双方の振分長を調節する。
\item
適当な角度にテーブルを回転することで、双方の振分長を調節する。
\end{enumerate}
このとき、モールドがどのように移動するかを考える。

基本的な考え方として、半径$R$の円の中心を原点として$\Omega$だけ回転し、次にモールドとの(スペーサを入れてない側の)接点を中心に$-\theta$だけ回転したと考えることができる。
なお、ここでは話の簡単化のため、もとの振分けではトップ側よりボトム側の振分長のほうが長いものとする。




%%%%%%%%%%%%%%%%%%%%%%%%%%%%%%%%%%%%%%%%%%%%%%%%%%%%%%%%%%
%% section 1.1 %%%%%%%%%%%%%%%%%%%%%%%%%%%%%%%%%%%%%%%%%%%
%%%%%%%%%%%%%%%%%%%%%%%%%%%%%%%%%%%%%%%%%%%%%%%%%%%%%%%%%%
\modHeadsection{ジグの接点部が点の場合}
まずは簡単のため、ジグのモールドとの接点部(\pageautoref{fig:mouldOnComplexPlane1}のU$_\mathrm T$, U$_\mathrm B$の部分)は点であるとして考える。
%%%%%%%%%%%%%%%%%%%%%%%%%%%%%%%%%%%%%%%%%%%%%%%%%%%%%%%%%%
%% figure %%%%%%%%%%%%%%%%%%%%%%%%%%%%%%%%%%%%%%%%%%%%%%%%
%%%%%%%%%%%%%%%%%%%%%%%%%%%%%%%%%%%%%%%%%%%%%%%%%%%%%%%%%%
\addcontentsline{lof}{chapter}{\thechapter. \Chaptername}
\newlength{\oldtextwidth}
\setlength{\oldtextwidth}{\textwidth}
\newlength{\oldtextheight}
\setlength{\oldtextheight}{\textheight}
\afterpage{%
\begin{landscape}
\setlength{\textwidth}{\oldtextwidth}
\setlength{\textheight}{\oldtextwidth}
\begin{figure}[p]
\centering
\begin{Figbox}
\resizebox{\linewidth}{!}{\adjustbox{max height=\textheight}{\mouldCoordinate}}%
\captionsetup{width=.75\textwidth}
\caption[湾曲中心Oを原点とした複素平面上のモールド]
  {湾曲中心Oを原点とした複素平面上のモールド\newline
   T$_\mathrm o$, T$_\mathrm i$, B$_\mathrm o$, B$_\mathrm i$, U$_\mathrm T$, U$_\mathrm B$は点、%
   $R_\mathrm c$, $R_\mathrm o$, $R_\mathrm i$, $f_\mathrm T$, $f_\mathrm B$, $l$は長さ、%
   $\alpha_{\mathrm T_\mathrm o}$, $\alpha_{\mathrm T_\mathrm i}$, $\alpha_{\mathrm U_\mathrm B}$は角度を示す。}
\label{fig:mouldOnComplexPlane1}
\end{Figbox}%
\end{figure}%
\end{landscape}%
\setlength{\textwidth}{\oldtextwidth}
\setlength{\textheight}{\oldtextheight}%
}
%%%%%%%%%%%%%%%%%%%%%%%%%%%%%%%%%%%%%%%%%%%%%%%%%%%%%%%%%%
%%%%%%%%%%%%%%%%%%%%%%%%%%%%%%%%%%%%%%%%%%%%%%%%%%%%%%%%%%
%%%%%%%%%%%%%%%%%%%%%%%%%%%%%%%%%%%%%%%%%%%%%%%%%%%%%%%%%%



%%%%%%%%%%%%%%%%%%%%%%%%%%%%%%%%%%%%%%%%%%%%%%%%%%%%%%%%%%
%% subsection 1.1.1 %%%%%%%%%%%%%%%%%%%%%%%%%%%%%%%%%%%%%%
%%%%%%%%%%%%%%%%%%%%%%%%%%%%%%%%%%%%%%%%%%%%%%%%%%%%%%%%%%
\subsection{スペーサを用いた再振分け}
モールドの湾曲における円の中心Oを原点とした複素数平面を考える
%% footnote %%%%%%%%%%%%%%%%%%%%%
\footnote{ここでは$0 < R_\mathrm c < \infty$ ($0 < \nicefrac1{R_\mathrm c} < \infty$)としている。
$R_\mathrm c \to \infty$ ($\nicefrac1{R_\mathrm c} \to 0$)の場合、すなわち湾曲のないまっすぐなモールドの場合は、別途考える必要がある。}。
%%%%%%%%%%%%%%%%%%%%%%%%%%%%%%%%%
このとき、\pageautoref{fig:mouldOnComplexPlane1}のように、$R_\mathrm c$, $R_\mathrm i$, $R_\mathrm o$, $f_\mathrm T$, $f_\mathrm B$, $l$, $\alpha_{\mathrm T_\mathrm i}$, $\alpha_{\mathrm T_\mathrm o}$, $\alpha_{\mathrm U_\mathrm B}$をとると、
\begin{subequations}
%% label{eq:constraintUpoint1}
%% label{eq:constraintUpoint2}
\begin{gather}
  \label{eq:constraintUpoint1}
  R_\mathrm o - R_\mathrm c = R_\mathrm c - R_\mathrm i = \frac{W_x}2~, \qquad
  \IP\!\left(R_\mathrm oe^{i\alpha_{\mathrm T_\mathrm o}} - R_\mathrm ie^{i\alpha_{\mathrm T_\mathrm i}}\right)
  = 0~,\\
  \label{eq:constraintUpoint2}
  \sin\alpha_{\mathrm T_\mathrm i} = \frac{f_\mathrm T}{R_\mathrm i}, \qquad
  \sin\alpha_{\mathrm U_\mathrm B} = \frac l{R_\mathrm i}, \qquad
  \tan\theta = \frac\delta{2l}~.
\end{gather}
\end{subequations}
ここで$W_x$はモールドの(AC)外径、$\delta$はスペーサの厚さである。
このときモールドを原点Oを中心に$\Omega$だけ回転し、さらに点U$_\mathrm B$($R_\mathrm i$, $-\alpha_{\mathrm U_\mathrm B}$)を中心に$-\theta$だけ回転すると、点T$_\mathrm i$($R_\mathrm i$, $\alpha_{\mathrm T_\mathrm i}$)は、
%% label{eq:afterftUpoint}
\begin{align}
  \notag
  & e^{-i\theta}\!\left\{R_\mathrm ie^{i(\alpha_{\mathrm T_\mathrm i} + \Omega)} - R_\mathrm ie^{-i\alpha_{\mathrm U_\mathrm B}}\right\}
    +R_\mathrm ie^{-i\alpha_{\mathrm U_\mathrm B}}\\
  &= R_\mathrm i
     \left\{
       e^{i(\alpha_{\mathrm T_\mathrm i} + \Omega - \theta)} - e^{-i(\alpha_{\mathrm U_\mathrm B} + \theta)} + e^{-i\alpha_{\mathrm U_\mathrm B}}
     \right\}
  \label{eq:afterftUpoint}
\end{align}
に移動する。
また同様に点T$_\mathrm o$($R_\mathrm o$, $\alpha_{\mathrm T_\mathrm o}$)は
\begin{align*}
  \notag
  R_\mathrm oe^{i(\alpha_{\mathrm T_\mathrm o} + \Omega - \theta)}
  -R_\mathrm i\left\{e^{-i(\alpha_{\mathrm U_\mathrm B} + \theta)}-e^{-i\alpha_{\mathrm U_\mathrm B}}\right\}
\end{align*}
に移動する。
したがって、これらの差
\begin{align*}
  \notag
  e^{i(\Omega - \theta)}\left(R_\mathrm oe^{i\alpha_{\mathrm T_\mathrm o}} - R_\mathrm ie^{i\alpha_{\mathrm T_\mathrm i}}\right)
\end{align*}
の虚部が$0$であればよい。
つまり、\pageeqref{eq:constraintUpoint1}より、$\Omega = \theta$である
%% footnote %%%%%%%%%%%%%%%%%%%%%
\footnote{ここでは$0 \leqq \Omega, \theta < \nicefrac \pi2$としている。}。
%%%%%%%%%%%%%%%%%%%%%%%%%%%%%%%%%

スペーサを入れた後の(トップ側の)振分長は、\pageeqref{eq:afterftUpoint}の虚部を見ればよい。
\begin{align*}
  R_\mathrm i\left\{\sin\alpha_{\mathrm T_\mathrm i} + \sin(\alpha_{\mathrm U_\mathrm B} + \theta) - \sin\alpha_{\mathrm U_\mathrm B}\right\}
  &= f_\mathrm T -l
     +R_\mathrm i\left(\sin\alpha_{\mathrm U_\mathrm B}\cos\theta + \cos\alpha_{\mathrm U_\mathrm B}\sin\theta\right)\\
  &= f_\mathrm T -l+l\cdot\frac{2l}{\sqrt{4l^2+\delta^2}}
     +\sqrt{R_\mathrm i^2-l^2}\cdot\frac{\delta}{\sqrt{4l^2+\delta^2}}\\
  &= f_\mathrm T -l+\frac{2l^2+\delta\sqrt{R_\mathrm i^2-l^2}}{\sqrt{4l^2+\delta^2}}~.
\end{align*}
まとめると、厚さ$\delta$のスペーサを入れた後のトップ側の振分長$f'_\mathrm T$は、
\begin{align*}
  f'_\mathrm T
  = f_\mathrm T -l
    +\frac{2l^2+\delta\sqrt{\left(R_\mathrm c-\nicefrac{W_x}2\right)^2-l^2}}{\sqrt{4l^2+\delta^2}}~.
\end{align*}



%%%%%%%%%%%%%%%%%%%%%%%%%%%%%%%%%%%%%%%%%%%%%%%%%%%%%%%%%%
%% subsection 1.1.2 %%%%%%%%%%%%%%%%%%%%%%%%%%%%%%%%%%%%%%
%%%%%%%%%%%%%%%%%%%%%%%%%%%%%%%%%%%%%%%%%%%%%%%%%%%%%%%%%%
\subsection{振分長が均等になるスペーサ厚}
%%%%%%%%%%%%%%%%%%%%%%%%%%%%%%%
トップ側とボトム側の振分長が同じになるとき、$\delta$は
\begin{align*}
  f'_\mathrm T - f_\mathrm T = \frac{f_\mathrm B - f_\mathrm T}2
\end{align*}
を満たす。
これより、
\begin{align*}
  \frac{2l^2+\delta\sqrt{R_\mathrm i^2-l^2}}{\sqrt{4l^2+\delta^2}} = l'\qquad
  \left(l' \equiv l + \frac{f_\mathrm B-f_\mathrm T}2\right)
\end{align*}
両辺を2乗すると、
\begin{gather*}
  4l^4+\delta^2\left(R_\mathrm i^2-l^2\right)+4l^2\delta\sqrt{R_\mathrm i^2-l^2}
  = l'^2\left(4l^2+\delta^2\right)\\
  \longrightarrow\quad
  \delta^2\left(R_\mathrm i^2-l^2-l'^2\right)
  +4l^2\delta\sqrt{R_\mathrm i^2-l^2} -4l^2\left(l'^2 - l^2\right)
  = 0.
\end{gather*}
$\delta > 0$より、
\begin{align*}
  \delta
  &= \frac{\sqrt{4l^4\left(R_\mathrm i^2-l^2\right)
                 +4l^2\left(R_\mathrm i^2-l^2-l'^2\right)\left(l'^2 - l^2\right)}
           -2l^2\sqrt{R_\mathrm i^2-l^2}}{R_\mathrm i^2-l^2-l'^2}\\
  &= 2l\cdot\frac{l'\sqrt{R_\mathrm i^2-l'^2}-l\sqrt{R_\mathrm i^2-l^2}}{R_\mathrm i^2-l^2-l'^2}
\end{align*}
まとめると、求めるスペーサの厚さ$\delta$は、
\begin{align*}
  \delta
  = 2l\cdot
    \frac{\displaystyle
          \left(l+\frac{f_\mathrm B-f_\mathrm T}2\right)\!
          \sqrt{\left(R_\mathrm c-\frac{W_x}2\right)^{\!\!2}
                -\left(l+\frac{f_\mathrm B-f_\mathrm T}2\right)^{\!\!2}}
          -l\sqrt{\left(R_\mathrm c-\frac{W_x}2\right)^{\!\!2}-l^2}}
         {\displaystyle
          \left(R_\mathrm c-\frac{W_x}2\right)^{\!\!2}-l^2
          -\left(l+\frac{f_\mathrm B-f_\mathrm T}2\right)^{\!\!2}}~.
\end{align*}




\clearpage
%%%%%%%%%%%%%%%%%%%%%%%%%%%%%%%%%%%%%%%%%%%%%%%%%%%%%%%%%%
%% section 1.2 %%%%%%%%%%%%%%%%%%%%%%%%%%%%%%%%%%%%%%%%%%%
%%%%%%%%%%%%%%%%%%%%%%%%%%%%%%%%%%%%%%%%%%%%%%%%%%%%%%%%%%
\modHeadsection{受板がある場合}
ジグのモールドと接する部品(受板)の大きさを考慮した場合を考える。
モールドに接する側の面が半径$\rho$の円弧、虚軸方向の厚みが$\sigma$とする。
また受板の虚軸負方向側の面は、ジグのそれと同じ平面上にあるものとする。

受板の径の中心を改めてU$_\mathrm B$とし、また原点に対する偏角を改めて$-\alpha_{\mathrm U_\mathrm B}$とすると、これはU$_\mathrm B$($R_\mathrm i-\rho$, $-\alpha_{\mathrm U_\mathrm B}$)表すことができる。
ただし、\pageeqref{eq:constraintUpoint2}は以下のようになる。
\begin{align*}
  \sin\alpha_{\mathrm U_\mathrm B} = \frac{\bar l}{R_\mathrm i-\rho}\quad, \quad
  \tan\psi = \frac\delta{2\bar l} \quad
  \left(~\bar l \equiv l-\frac\sigma2~\right).
\end{align*}

これを原点Oを中心に$\Omega$だけ回転し、さらに点U$_\mathrm B$($R_\mathrm i-\rho$, $-\alpha_{\mathrm U_\mathrm B}$)を中心に点T$_\mathrm i$($R_\mathrm i$, $\alpha_{\mathrm T_\mathrm i}$)を$-\theta$だけ回転すると、
%% label{eq:afterftUfinite}
\begin{align}
  \notag
  & e^{-i\theta}\!
    \left\{R_\mathrm ie^{i(\alpha_{\mathrm T_\mathrm i}+\Omega)}
           -R_\mathrm i'e^{-i\alpha_{\mathrm U_\mathrm B}}\right\}
    +R_\mathrm i'e^{-i\alpha_{\mathrm U_\mathrm B}}\\
  & = R_\mathrm ie^{i(\alpha_{\mathrm T_\mathrm i}+\Omega-\theta)}
      -R_\mathrm i'\!
       \left\{e^{-i(\alpha_{\mathrm U_\mathrm B}+\theta)}-e^{-i\alpha_{\mathrm U_\mathrm B}}\right\}\qquad
    \big(R_\mathrm i' \equiv R_\mathrm i-\rho\big)
    \label{eq:afterftUfinite}
\end{align}
に移動する。
同様に点T$_\mathrm o$($R_\mathrm o$, $\alpha_{\mathrm T_\mathrm o}$)は
\begin{align*}
  R_\mathrm oe^{i(\alpha_{\mathrm T_\mathrm o}+\Omega-\theta)}
  -R_\mathrm i'\!
   \left\{e^{-i(\alpha_{\mathrm U_\mathrm B} + \theta)} - e^{-i\alpha_{\mathrm U_\mathrm B}}\right\}
\end{align*}
に移動する。
したがって、これらの差
\begin{align*}
  e^{i(\Omega-\theta)}\!
  \left(R_\mathrm oe^{i\alpha_{\mathrm T_\mathrm o}} - R_\mathrm ie^{i\alpha_{\mathrm T_\mathrm i}}\right)
\end{align*}
の虚部が$0$であればよい。
つまり、\pageeqref{eq:constraintUpoint1}より、受板がある場合も$\Omega = \theta$である。



%%%%%%%%%%%%%%%%%%%%%%%%%%%%%%%%%%%%%%%%%%%%%%%%%%%%%%%%%%
%% subsection 1.2.1 %%%%%%%%%%%%%%%%%%%%%%%%%%%%%%%%%%%%%%
%%%%%%%%%%%%%%%%%%%%%%%%%%%%%%%%%%%%%%%%%%%%%%%%%%%%%%%%%%
\subsection{受板の接点}
受板とモールドとの(トップ側の)接点は、$R_\mathrm ie^{i\alpha_{\mathrm U_\mathrm B}}$で与えられる。
このとき厚さ$\delta$のスペーサを取付けると、U$_\mathrm B$を中心に回転するが、それに伴い受板における接点の位置も変化する。


%%%%%%%%%%%%%%%%%%%%%%%%%%%%%%%%%%%%%%%%%%%%%%%%%%%%%%%%%%
%% subsubsection 1.2.1.1 %%%%%%%%%%%%%%%%%%%%%%%%%%%%%%%%%
%%%%%%%%%%%%%%%%%%%%%%%%%%%%%%%%%%%%%%%%%%%%%%%%%%%%%%%%%%
\subsubsection{回転後のモールドの湾曲中心}
%%%%%%%%%%%%%%%%%%%%%%%%%
厚さ$\delta$のスペーサを挟むと、トップ側における受板の円の中心U$_\mathrm B$は実軸方向に$\delta$だけ移動するので、
\begin{align*}
  R_\mathrm i'e^{i\alpha_{\mathrm U_\mathrm B}}
  \quad\longrightarrow\quad
  \delta+R_\mathrm i'e^{i\alpha_{\mathrm U_\mathrm B}}\ .
\end{align*}
よって、それぞれの受板の中心U$_\mathrm B$, U$_\mathrm T$を結んだ線分U$_\mathrm B$U$_\mathrm T$は、U$_\mathrm B$を中心に$-\psi$だけ傾いた線分U$_\mathrm B'$U$_\mathrm T'$となる
%% footnote %%%%%%%%%%%%%%%%%%%%%
\footnote{%
U$_\mathrm B'$U$_\mathrm T'$の長さは$\bar l\sec\psi$であり、U$_\mathrm B$U$_\mathrm T$の長さ$\bar l$より長くなることに注意。}。
%%%%%%%%%%%%%%%%%%%%%%%%%%%%%%%%%
回転後のモールドの湾曲中心は、この線分の垂直二等分線上にあり、またそれぞれの受板の中心から$R_\mathrm i'$の距離の位置にある。
つまり、この傾いた線分U$_\mathrm B'$U$_\mathrm T'$の中点から、角度$\pi-\psi$, 大きさ$\sqrt{R_\mathrm i'^2-\frac{\delta^2+(2\bar l)^2}4}$の位置に移動する。
したがって、回転後における湾曲の円の中心O$'$は、
%% label{eq:afterOrgin}
\begin{align}
  \notag
  & \frac\delta2+\sqrt{R_\mathrm i'^2-\bar l^2}
    +\sqrt{R_\mathrm i'^2-\frac{\delta^2+(2\bar l)^2}4}e^{i(\pi-\psi)}\\
  & = \frac\delta2+\sqrt{R_\mathrm i'^2-\bar l^2}-\sqrt{R_\mathrm i'^2-\frac{\delta^2+(2\bar l)^2}4}\cos\psi
      +i\sqrt{R_\mathrm i'^2-\frac{\delta^2+(2\bar l)^2}4}\sin\psi\ .
    \label{eq:afterOrgin}
\end{align}


%%%%%%%%%%%%%%%%%%%%%%%%%%%%%%%%%%%%%%%%%%%%%%%%%%%%%%%%%%
%% subsubsection 1.2.1.2 %%%%%%%%%%%%%%%%%%%%%%%%%%%%%%%%%
%%%%%%%%%%%%%%%%%%%%%%%%%%%%%%%%%%%%%%%%%%%%%%%%%%%%%%%%%%
\subsubsection{回転後の接点(トップ側)}
回転後のトップ側における受板の中心U$_\mathrm T'$とモールドの湾曲中心O$'$との差をとると、
\begin{align*}
  \frac\delta2+\sqrt{R_\mathrm i'^2-\frac{\delta^2+(2\bar l)^2}4}\cos\psi
  +i\left\{\bar l-\sqrt{R_\mathrm i'^2-\frac{\delta^2+(2\bar l)^2}4}\sin\psi\right\}
  = R_\mathrm i'e^{i\alpha'_{\mathrm U_\mathrm T}}\ .
\end{align*}
ここで、
\begin{align*}
  \tan\alpha'_{\mathrm U_\mathrm T}
  = \frac{\displaystyle\bar l-\sqrt{R_\mathrm i'^2-\frac{\delta^2+(2\bar l)^2}4}\sin\psi}
         {\displaystyle\frac\delta2+\sqrt{R_\mathrm i'^2-\frac{\delta^2+(2\bar l)^2}4}\cos\psi}\ .
\end{align*}
%%%%%%%%%%%%%%%%%%%%%%%%%%%%%%%%%%%%%%%%%%%%%%%%%%%%%%%%%%
%% hosoku %%%%%%%%%%%%%%%%%%%%%%%%%%%%%%%%%%%%%%%%%%%%%%%%
%%%%%%%%%%%%%%%%%%%%%%%%%%%%%%%%%%%%%%%%%%%%%%%%%%%%%%%%%%
\begin{hosoku}
これの大きさは、$\delta\cos\psi-2\bar l\sin\psi = 0$より、
\begin{align*}
  \left\{\frac\delta2+\sqrt{R_\mathrm i'^2-\frac{\delta^2+(2\bar l)^2}4}\cos\psi\right\}^{\!\!2}
  +\left\{\bar l-\sqrt{R_\mathrm i'^2-\frac{\delta^2+(2\bar l)^2}4}\sin\psi\right\}^{\!\!2}
  = R_\mathrm i'^2\ .
\end{align*}
\end{hosoku}
%%%%%%%%%%%%%%%%%%%%%%%%%%%%%%%%%%%%%%%%%%%%%%%%%%%%%%%%%%
%%%%%%%%%%%%%%%%%%%%%%%%%%%%%%%%%%%%%%%%%%%%%%%%%%%%%%%%%%
%%%%%%%%%%%%%%%%%%%%%%%%%%%%%%%%%%%%%%%%%%%%%%%%%%%%%%%%%%
よって、回転後の接点は以下で与えられる。
\begin{align*}
  &  R_\mathrm ie^{i\alpha'_{\mathrm U_\mathrm T}}
     +\frac\delta2+\sqrt{R_\mathrm i'^2-\bar l^2}-\sqrt{R_\mathrm i'^2-\frac{\delta^2+(2\bar l)^2}4}\cos\psi
     +i\sqrt{R_\mathrm i'^2-\frac{\delta^2+(2\bar l)^2}4}\sin\psi\\
  &= \delta+R_\mathrm i'e^{i\alpha_{\mathrm U_\mathrm B}}+\rho e^{i\alpha'_{\mathrm U_\mathrm T}}\ .
\end{align*}


%%%%%%%%%%%%%%%%%%%%%%%%%%%%%%%%%%%%%%%%%%%%%%%%%%%%%%%%%%
%% subsubsection 1.2.1.3 %%%%%%%%%%%%%%%%%%%%%%%%%%%%%%%%%
%%%%%%%%%%%%%%%%%%%%%%%%%%%%%%%%%%%%%%%%%%%%%%%%%%%%%%%%%%
\subsubsection{回転後の接点(ボトム側)}
回転後のボトム側における受板の中心U$_\mathrm B$とモールドの湾曲中心O$'$との差をとると、
\begin{align*}
  -\frac\delta2+\sqrt{R_\mathrm i'^2-\frac{\delta^2+(2\bar l)^2}4}\cos\psi
  -i\left\{\bar l+\sqrt{R_\mathrm i'^2-\frac{\delta^2+(2\bar l)^2}4}\sin\psi\right\}
  = R_\mathrm i'e^{-i\alpha'_{\mathrm U_\mathrm B}}
\end{align*}
ここで、
\begin{align*}
  \tan\alpha'_{\mathrm U_\mathrm B}
  = \frac{\displaystyle\bar l+\sqrt{R_\mathrm i'^2-\frac{\delta^2+(2\bar l)^2}4}\sin\psi}
         {\displaystyle-\frac\delta2+\sqrt{R_\mathrm i'^2-\frac{\delta^2+(2\bar l)^2}4}\cos\psi}
\end{align*}
よって、回転後の接点は以下で与えられる。
%% label{eq:afterUBcontact}
\begin{align}
  \notag
  &  R_\mathrm ie^{-i\alpha'_{\mathrm U_\mathrm B}}
     +\frac\delta2+\sqrt{R_\mathrm i'^2-\bar l^2}-\sqrt{R_\mathrm i'^2-\frac{\delta^2+(2\bar l)^2}4}\cos\psi
     +i\sqrt{R_\mathrm i'^2-\frac{\delta^2+(2\bar l)^2}4}\sin\psi\\
  &= R_\mathrm i'e^{-i\alpha_{\mathrm U_\mathrm B}}+\rho e^{-i\alpha'_{\mathrm U_\mathrm B}}
   \label{eq:afterUBcontact}
\end{align}
%%%%%%%%%%%%%%%%%%%%%%%%%%%%%%%%%%%%%%%%%%%%%%%%%%%%%%%%%%
%% hosoku %%%%%%%%%%%%%%%%%%%%%%%%%%%%%%%%%%%%%%%%%%%%%%%%
%%%%%%%%%%%%%%%%%%%%%%%%%%%%%%%%%%%%%%%%%%%%%%%%%%%%%%%%%%
\begin{hosoku}
辺の長さが$R_i'$, $R_i'$, $2\bar l$の二等辺三角形$\triangle$OU$_\mathrm B$U$_\mathrm T$の部分が、回転後には辺の長さ$R_i'$, $R_i'$, $\sqrt{\delta^2+(2\bar l)^2}$の二等辺三角形$\triangle$O$'$U$_\mathrm B'$U$_\mathrm T'$となる。
実際、$\cos2a = 1-2\sin^2\!a$より、
\begin{align*}
  \sin^2\frac{\alpha'_{\mathrm U_\mathrm T}+\alpha'_{\mathrm U_\mathrm B}}2
  = \frac{\delta^2+(2\bar l)^2}{4R_\mathrm i'^2}\ .
\end{align*}
\end{hosoku}
%%%%%%%%%%%%%%%%%%%%%%%%%%%%%%%%%%%%%%%%%%%%%%%%%%%%%%%%%%
%%%%%%%%%%%%%%%%%%%%%%%%%%%%%%%%%%%%%%%%%%%%%%%%%%%%%%%%%%
%%%%%%%%%%%%%%%%%%%%%%%%%%%%%%%%%%%%%%%%%%%%%%%%%%%%%%%%%%



%%%%%%%%%%%%%%%%%%%%%%%%%%%%%%%%%%%%%%%%%%%%%%%%%%%%%%%%%%
%% subsection 1.2.2 %%%%%%%%%%%%%%%%%%%%%%%%%%%%%%%%%%%%%%
%%%%%%%%%%%%%%%%%%%%%%%%%%%%%%%%%%%%%%%%%%%%%%%%%%%%%%%%%%
\subsection{スペーサによるモールドの回転角}
厚さ$\delta$のスペーサを挿入すると、モールドの湾曲中心Oは、U$_\mathrm B$を中心に$-\left(\alpha'_{\mathrm U_\mathrm B}\!-\alpha_{\mathrm U_\mathrm B}\right)$だけ回転する。
実際、
\begin{align*}
  -R_\mathrm i'e^{-i\alpha'_{\mathrm U_\mathrm B}}+R_\mathrm i'e^{-i\alpha_{\mathrm U_\mathrm B}}
  &= R_\mathrm i'(\cos\alpha_{\mathrm U_\mathrm B}-\cos\alpha'_{\mathrm U_\mathrm B})
     +iR_\mathrm i'(\sin\alpha'_{\mathrm U_\mathrm B}-\sin\alpha_{\mathrm U_\mathrm B})
\end{align*}
であり、これは回転後の湾曲中心\pageeqref{eq:afterOrgin}に一致する。
つまり、$\alpha'_{\mathrm U_\mathrm B}\!-\alpha_{\mathrm U_\mathrm B}$が$\theta$に相当する。
%%%%%%%%%%%%%%%%%%%%%%%%%%%%%%%%%%%%%%%%%%%%%%%%%%%%%%%%%%
%% hosoku %%%%%%%%%%%%%%%%%%%%%%%%%%%%%%%%%%%%%%%%%%%%%%%%
%%%%%%%%%%%%%%%%%%%%%%%%%%%%%%%%%%%%%%%%%%%%%%%%%%%%%%%%%%
\begin{hosoku}
トップ側の接点U$_\mathrm T'$とボトム側の接点U$_\mathrm B'$の差をとると、
\begin{align*}
  R_\mathrm i\!\left(e^{i\alpha_{\mathrm U_\mathrm T}'}-e^{-i\alpha'_{\mathrm U_\mathrm B}}\right)
  &= \frac{R_\mathrm i'+\rho}{R_\mathrm i'}\left\{\delta+i(2\bar l)\right\}
   = \frac{R_\mathrm i}{R_\mathrm i'}\sqrt{\delta^2+(2\bar l)^2}e^{i(\nicefrac\pi2-\psi)}\ .
\end{align*}
したがって、厚さ$\delta$のスペーサを挿入すると、両接点を通る直線は$-\psi$だけ傾くことがわかる。
また、その長さは受板中心間U$_\mathrm T'$U$_\mathrm B'$の距離$\sqrt{\delta^2+(2\bar l)^2}$の$\nicefrac{R_i}{R_i'}$倍になっていることも確かめられる。
\end{hosoku}
%%%%%%%%%%%%%%%%%%%%%%%%%%%%%%%%%%%%%%%%%%%%%%%%%%%%%%%%%%
%%%%%%%%%%%%%%%%%%%%%%%%%%%%%%%%%%%%%%%%%%%%%%%%%%%%%%%%%%
%%%%%%%%%%%%%%%%%%%%%%%%%%%%%%%%%%%%%%%%%%%%%%%%%%%%%%%%%%


%%%%%%%%%%%%%%%%%%%%%%%%%%%%%%%%%%%%%%%%%%%%%%%%%%%%%%%%%%
%% subsection 1.2.3 %%%%%%%%%%%%%%%%%%%%%%%%%%%%%%%%%%%%%%
%%%%%%%%%%%%%%%%%%%%%%%%%%%%%%%%%%%%%%%%%%%%%%%%%%%%%%%%%%
\subsection{スペーサによる再振分け}


%%%%%%%%%%%%%%%%%%%%%%%%%%%%%%%%%%%%%%%%%%%%%%%%%%%%%%%%%%
%% subsubsection 1.2.3.1 %%%%%%%%%%%%%%%%%%%%%%%%%%%%%%%%%
%%%%%%%%%%%%%%%%%%%%%%%%%%%%%%%%%%%%%%%%%%%%%%%%%%%%%%%%%%
\subsubsection{再振分長}
スペーサを入れた後の(トップ側の)振分長$f'_\mathrm T$は、\pageeqref{eq:afterftUfinite}の虚部を見ればよい。
回転角は$-(\alpha'_{\mathrm U_\mathrm B}-\alpha_{\mathrm U_\mathrm B})$なので、
\begin{align*}
  f'_\mathrm T
  = R_\mathrm i\sin\alpha_{\mathrm T_\mathrm i}
    +R_\mathrm i'\left(\sin\alpha'_{\mathrm U_\mathrm B}-\sin\alpha_{\mathrm U_\mathrm B}\right)
  &= f_\mathrm T+\sqrt{R_\mathrm i'^2-\frac{\delta^2+(2\bar l)^2}4}\sin\psi\\
  &= f_\mathrm T+\sqrt{R_\mathrm i'^2-\frac{\delta^2+(2\bar l)^2}4}\frac\delta{\sqrt{\delta^2+(2\bar l)^2}}\ .
\end{align*}


%%%%%%%%%%%%%%%%%%%%%%%%%%%%%%%%%%%%%%%%%%%%%%%%%%%%%%%%%%
%% subsubsection 1.2.3.2 %%%%%%%%%%%%%%%%%%%%%%%%%%%%%%%%%
%%%%%%%%%%%%%%%%%%%%%%%%%%%%%%%%%%%%%%%%%%%%%%%%%%%%%%%%%%
\subsubsection{モールドの移動距離}
\pageeqref{eq:afterftUfinite}の実部は、
\begin{align*}
  & R_\mathrm i\cos\alpha_{\mathrm T_\mathrm i}
    -R_\mathrm i'(\cos\alpha'_{\mathrm U_\mathrm B}-\cos\alpha_{\mathrm U_\mathrm B})\\
  & = \sqrt{R_\mathrm i^2-f_\mathrm T^2}+\frac\delta2+\sqrt{R_\mathrm i'^2-\bar l^2}
      -\sqrt{R_\mathrm i'^2-\frac{\delta^2+(2\bar l)^2}4}\cos\psi
\end{align*}
となる。
よって、スペーサを挿入することにより、モールドは水平・鉛直方向にそれぞれ、
\begin{subequations}
%% label{eq:spacerMoveHdistance}
\begin{alignat}{2}
  \label{eq:spacerMoveHdistance}
  \text{水平方向:}\quad
  & \frac\delta2+\sqrt{R_\mathrm i'^2-\bar l^2}-\sqrt{R_\mathrm i'^2-\frac{\delta^2+(2\bar l)^2}4}\frac{2\bar l}{\sqrt{\delta^2+(2\bar l)^2}}\\
  \text{鉛直方向:}\quad
  & \sqrt{R_\mathrm i'^2-\frac{\delta^2+(2\bar l)^2}4}\frac\delta{\sqrt{\delta^2+(2\bar l)^2}}
\end{alignat}
\end{subequations}
だけ移動することがわかる。



%%%%%%%%%%%%%%%%%%%%%%%%%%%%%%%%%%%%%%%%%%%%%%%%%%%%%%%%%%
%% subsection 1.2.4 %%%%%%%%%%%%%%%%%%%%%%%%%%%%%%%%%%%%%%
%%%%%%%%%%%%%%%%%%%%%%%%%%%%%%%%%%%%%%%%%%%%%%%%%%%%%%%%%%
\subsection{再振分長が均等になるスペーサ厚}
トップ側とボトム側の振分長が同じになるとき、$\delta$は
\begin{align*}
  \sqrt{R_\mathrm i'^2-\frac{\delta^2+(2\bar l)^2}4}\frac\delta{\sqrt{\delta^2+(2\bar l)^2}} = f_d \qquad
  \left(f_d \equiv \frac{f_\mathrm B-f_\mathrm T}2\right)\ .
\end{align*}
を満たす。
両辺を2乗して$-4$倍すると、
\begin{align*}
  \delta^2\left\{\delta^2+(2\bar l)^2-4R_\mathrm i'^2\right\}+4f_d^2\left\{\delta^2+(2\bar l)^2\right\}
  & = \delta^4-2\left\{2R_\mathrm i'^2-2\bar l^2-2f_d^2\right\}\delta^2+4f_d^2(2\bar l)^2\\
  & = 0\ .
\end{align*}
したがって、$f_\mathrm T = f_\mathrm B$ ($f_d = 0$)のとき$\delta = 0$であることを考慮して、
\begin{align*}
  \delta^2
  &= 2\left\{
       R_\mathrm i'^2-\bar l^2-f_d^2-\sqrt{\left(R_\mathrm i'^2-\bar l^2-f_d^2\right)^2-4f_d^2\bar l^2}\,
     \right\}\\
  &= \left(\sqrt{R_\mathrm i'^2-(\bar l-f_d)^2}-\sqrt{R_\mathrm i'^2-(\bar l+f_d)^2}\,\right)^{\!\!2}.
\end{align*}
なお、これはモールドが水平・鉛直方向にそれぞれ、
\begin{align*}
  \text{水平方向:}~\frac\delta2+\sqrt{R_\mathrm i^2-\bar l^2}-\frac{2\bar l}{\delta}f_d\quad(\delta>0)\ , \qquad
  \text{鉛直方向:}~\frac{f_\mathrm B-f_\mathrm T}2
\end{align*}
だけ移動することを意味する。


%%%%%%%%%%%%%%%%%%%%%%%%%%%%%%%%%%%%%%%%%%%%%%%%%%%%%%%%%%
%% hosoku %%%%%%%%%%%%%%%%%%%%%%%%%%%%%%%%%%%%%%%%%%%%%%%%
%%%%%%%%%%%%%%%%%%%%%%%%%%%%%%%%%%%%%%%%%%%%%%%%%%%%%%%%%%
\begin{hosoku}
改めてまとめると、厚さ$\delta$のスペーサを(トップ側に)挿入した後のトップ側の振分長$f_\mathrm T'$は、
\begin{align*}
  f_\mathrm T'
  = f_\mathrm T+\sqrt{R_\mathrm i'^2-\frac{\delta^2+(2\bar l)^2}4}\frac\delta{\sqrt{\delta^2+(2\bar l)^2}}\ .
\end{align*}
トップ側とボトム側の振分長が均等になるときのスペーサ厚$\delta'$は、
\begin{align*}
  \delta' = \sqrt{R_\mathrm i'^2-(\bar l-f_d)^2}-\sqrt{R_\mathrm i'^2-(\bar l+f_d)^2}\ .
\end{align*}
ここで、
\begin{align*}
  R_\mathrm i' = R_\mathrm c-\frac{W_x}2-\rho\ ,\quad
  \bar l = l-\frac\sigma2\ ,\quad
  f_d = \frac{f_\mathrm B-f_\mathrm T}2\ .
\end{align*}
\end{hosoku}
%%%%%%%%%%%%%%%%%%%%%%%%%%%%%%%%%%%%%%%%%%%%%%%%%%%%%%%%%%
%%%%%%%%%%%%%%%%%%%%%%%%%%%%%%%%%%%%%%%%%%%%%%%%%%%%%%%%%%
%%%%%%%%%%%%%%%%%%%%%%%%%%%%%%%%%%%%%%%%%%%%%%%%%%%%%%%%%%




\clearpage
%%%%%%%%%%%%%%%%%%%%%%%%%%%%%%%%%%%%%%%%%%%%%%%%%%%%%%%%%%
%% section 1.3 %%%%%%%%%%%%%%%%%%%%%%%%%%%%%%%%%%%%%%%%%%%
%%%%%%%%%%%%%%%%%%%%%%%%%%%%%%%%%%%%%%%%%%%%%%%%%%%%%%%%%%
\modHeadsection{テーブルの回転による振分長の調節}
これまでトップ・ボトム振分長の差を小さくするためにスペーサを用いる手法を考えてきた。
スペーサを取付けることは、本質的にはボトム側の受板の点U$_\mathrm B$を中心に回転しているということである。
このとき回転の中心はU$_\mathrm B$である必要はなく、他の点でも問題ない。
したがって、スペーサを用いて回転をしなくても、テーブルそのものを回転するという手法が考えられる。
これは回転の中心が、受板の点U$_\mathrm B$からテーブル中心Pに変わることに相当する。

受板の円の中心U$_\mathrm B$と、テーブル中心Pとの実軸方向の距離を$\varDelta$とすると、テーブル中心Pの$X$座標$\varDelta'$は次で与えられる。
%% label{eq:tableCenter}
\begin{align}
  \label{eq:tableCenter}
  \varDelta'
  = \varDelta+R_\mathrm i'\cos\alpha_{\mathrm U_\mathrm B} = \varDelta+\sqrt{R_\mathrm i'^2-\bar l^2}\ .
\end{align}
モールドのトップ側におけるC側端点T$_\mathrm i$($R_\mathrm i$, $\alpha_{\mathrm T_\mathrm i}$)を、原点Oを中心に$\Omega$だけ回転し、さらに点P\,($\varDelta'$, $0$)を中心に$-\theta$だけ回転すると、
\begin{align}
  \label{eq:afterfttable}
  e^{-i\theta}\left\{R_\mathrm i^{i(\alpha_{\mathrm T_\mathrm i}+\Omega)}-\varDelta'\right\}+\varDelta'
  = R_\mathrm ie^{i(\alpha_{\mathrm T_\mathrm i}+\Omega-\theta)}+\varDelta'\left(1-e^{-i\theta}\right)
\end{align}
に移動する。
同様に、トップ側におけるA側端点T$_\mathrm o$($R_\mathrm o$, $\alpha_{\mathrm T_\mathrm o}$)は、
\begin{align*}
  e^{-i\theta}\!\left\{R_\mathrm o^{i(\alpha_{\mathrm T_\mathrm o}+\Omega)}-\varDelta'\right\}+\varDelta'
  = R_\mathrm ie^{i(\alpha_{\mathrm T_\mathrm o}+\Omega-\theta)}+\varDelta'\!\left(1-e^{-i\theta}\right)
\end{align*}
に移動する。
したがって、これらの差
\begin{align*}
  e^{i(\Omega-\theta)}\!
  \left(R_\mathrm oe^{i\alpha_{\mathrm T_\mathrm o}}-R_\mathrm ie^{i\alpha_{\mathrm T_\mathrm i}}\right)
\end{align*}
の虚部が$0$であればよい。
よって\pageeqref{eq:constraintUpoint1}より、この場合も$\Omega = \theta$となる。



%%%%%%%%%%%%%%%%%%%%%%%%%%%%%%%%%%%%%%%%%%%%%%%%%%%%%%%%%%
%% subsection 1.3.1 %%%%%%%%%%%%%%%%%%%%%%%%%%%%%%%%%%%%%%
%%%%%%%%%%%%%%%%%%%%%%%%%%%%%%%%%%%%%%%%%%%%%%%%%%%%%%%%%%
\subsection{回転後のモールドの湾曲中心および受板との接点}


%%%%%%%%%%%%%%%%%%%%%%%%%%%%%%%%%%%%%%%%%%%%%%%%%%%%%%%%%%
%% subsubsection 1.3.1.1 %%%%%%%%%%%%%%%%%%%%%%%%%%%%%%%%%
%%%%%%%%%%%%%%%%%%%%%%%%%%%%%%%%%%%%%%%%%%%%%%%%%%%%%%%%%%
\subsubsection{回転後のモールドの湾曲中心}
回転後のモールドの湾曲中心O$'$は、点Pを中心に$-\theta$だけ回転するので、
\begin{align*}
  \varDelta'\!\left(1-e^{-i\theta}\right) = \varDelta'(1-\cos\theta)+i\varDelta'\sin\theta\ .
\end{align*}


%%%%%%%%%%%%%%%%%%%%%%%%%%%%%%%%%%%%%%%%%%%%%%%%%%%%%%%%%%
%% subsubsection 1.3.1.2 %%%%%%%%%%%%%%%%%%%%%%%%%%%%%%%%%
%%%%%%%%%%%%%%%%%%%%%%%%%%%%%%%%%%%%%%%%%%%%%%%%%%%%%%%%%%
\subsubsection{回転後の接点}
回転後におけるトップ側の受板との接点は、点Pを中心に$-\theta$だけ回転するので、
\begin{align*}
  &  e^{-i\theta}\!\left(R_\mathrm ie^{i\alpha_{\mathrm U_\mathrm B}}-\varDelta'\right)+\varDelta'\\
  &= R_\mathrm ie^{i(\alpha_{\mathrm U_\mathrm B}-\theta)}+\varDelta'\!\left(1-e^{-i\theta}\right)\\
  &= R_\mathrm i\cos(\alpha_{\mathrm U_\mathrm B}-\theta)+\varDelta'(1-\cos\theta)
     +i\left\{R_\mathrm i\sin(\alpha_{\mathrm U_\mathrm B}-\theta)+i\varDelta'\sin\theta\right\}\ .
\end{align*}
同様に、ボトム側の受板との接点は、
\begin{align*}
  &  e^{-i\theta}\!\left(R_\mathrm ie^{-i\alpha_{\mathrm U_\mathrm B}}-\varDelta'\right)+\varDelta'\\
  &= R_\mathrm ie^{-i(\alpha_{\mathrm U_\mathrm B}+\theta)}+\varDelta'\!\left(1-e^{-i\theta}\right)\\
  &= R_\mathrm i\cos(\alpha_{\mathrm U_\mathrm B}+\theta)+\varDelta'(1-\cos\theta)
     -i\left\{R_\mathrm i\sin(\alpha_{\mathrm U_\mathrm B}+\theta)-i\varDelta'\sin\theta\right\}\ .
\end{align*}
%%%%%%%%%%%%%%%%%%%%%%%%%%%%%%%%%%%%%%%%%%%%%%%%%%%%%%%%%%
%% hosoku %%%%%%%%%%%%%%%%%%%%%%%%%%%%%%%%%%%%%%%%%%%%%%%%
%%%%%%%%%%%%%%%%%%%%%%%%%%%%%%%%%%%%%%%%%%%%%%%%%%%%%%%%%%
\begin{hosoku}
両接点との差をとると、
\begin{align*}
  2R_\mathrm i\sin\alpha_{\mathrm U_\mathrm B}\sin\theta+2iR_\mathrm i\sin\alpha_{\mathrm U_\mathrm B}\cos\theta
  = \frac{R_\mathrm i}{R_\mathrm i'}(2\bar l)e^{i(\pi-\theta)}
\end{align*}
となり、受板の両中心間(長さ$2\bar l$)を結んだ線分を$\nicefrac{R_\mathrm i}{R_\mathrm i'}$倍し、虚軸から$-\theta$だけ傾けたものになっていることがわかる。
\end{hosoku}
%%%%%%%%%%%%%%%%%%%%%%%%%%%%%%%%%%%%%%%%%%%%%%%%%%%%%%%%%%
%%%%%%%%%%%%%%%%%%%%%%%%%%%%%%%%%%%%%%%%%%%%%%%%%%%%%%%%%%
%%%%%%%%%%%%%%%%%%%%%%%%%%%%%%%%%%%%%%%%%%%%%%%%%%%%%%%%%%




%%%%%%%%%%%%%%%%%%%%%%%%%%%%%%%%%%%%%%%%%%%%%%%%%%%%%%%%%%
%% subsection 1.3.2 %%%%%%%%%%%%%%%%%%%%%%%%%%%%%%%%%%%%%%
%%%%%%%%%%%%%%%%%%%%%%%%%%%%%%%%%%%%%%%%%%%%%%%%%%%%%%%%%%
\subsection{回転後の振分長}
回転後のトップ側の振分長$f_\mathrm T'$は、\pageeqref{eq:afterfttable}の虚部を見ればよいので、
\begin{align*}
  f_\mathrm T' = f_\mathrm T+\varDelta'\sin\theta = f_\mathrm T+\left(\varDelta+\sqrt{R_\mathrm i'-\bar l^2}\right)\sin\theta\ .
\end{align*}
同様に、ボトムの振分長$f_\mathrm B'$は、符号に注意して、
\begin{align*}
  f_\mathrm B' = f_\mathrm B-\varDelta'\sin\theta = (f_\mathrm T+f_\mathrm B)-f_\mathrm T'\ .
\end{align*}
%%%%%%%%%%%%%%%%%%%%%%%%%%%%%%%%%%%%%%%%%%%%%%%%%%%%%%%%%%
%% hosoku %%%%%%%%%%%%%%%%%%%%%%%%%%%%%%%%%%%%%%%%%%%%%%%%
%%%%%%%%%%%%%%%%%%%%%%%%%%%%%%%%%%%%%%%%%%%%%%%%%%%%%%%%%%
\begin{hosoku}
テーブル中心Pによる回転は振分長に影響しない(端面を水平に戻す)ので、湾曲中心Oによる回転だけが影響する。
よって、振分長は$\varDelta'\sin\theta$だけ変化する。
\end{hosoku}
%%%%%%%%%%%%%%%%%%%%%%%%%%%%%%%%%%%%%%%%%%%%%%%%%%%%%%%%%%
%%%%%%%%%%%%%%%%%%%%%%%%%%%%%%%%%%%%%%%%%%%%%%%%%%%%%%%%%%
%%%%%%%%%%%%%%%%%%%%%%%%%%%%%%%%%%%%%%%%%%%%%%%%%%%%%%%%%%
なお、\pageeqref{eq:afterfttable}の実部は、
\begin{align*}
  R_\mathrm i\cos\alpha_{\mathrm T_\mathrm i}+\varDelta'(1-\cos\theta)
  = \sqrt{R_\mathrm i^2-\bar l^2}+\left(\varDelta+\sqrt{R_\mathrm i'-\bar l^2}\right)\!(1-\cos\theta)\ .
\end{align*}
となるので、実軸の正方向に動くことがわかる。



%%%%%%%%%%%%%%%%%%%%%%%%%%%%%%%%%%%%%%%%%%%%%%%%%%%%%%%%%%
%% subsection 1.3.3 %%%%%%%%%%%%%%%%%%%%%%%%%%%%%%%%%%%%%%
%%%%%%%%%%%%%%%%%%%%%%%%%%%%%%%%%%%%%%%%%%%%%%%%%%%%%%%%%%
\subsection{振分長が均等になるときの回転角}
トップ側およびボトム側の振分長が同じになるとき、$\theta$は
\begin{align*}
  \varDelta'\sin\theta = f_d \qquad \left(f_d = \frac{f_\mathrm B-f_\mathrm T}2\right)
\end{align*}
であればいいので、
\begin{align*}
  \sin\theta = \frac{f_d}{\varDelta'}
  = \frac{f_\mathrm B-f_\mathrm T}{2\left(\varDelta+\sqrt{R_\mathrm i'-\bar l^2}\right)}~.
\end{align*}
%%%%%%%%%%%%%%%%%%%%%%%%%%%%%%%%%%%%%%%%%%%%%%%%%%%%%%%%%%
%% hosoku %%%%%%%%%%%%%%%%%%%%%%%%%%%%%%%%%%%%%%%%%%%%%%%%
%%%%%%%%%%%%%%%%%%%%%%%%%%%%%%%%%%%%%%%%%%%%%%%%%%%%%%%%%%
\begin{hosoku}
改めてまとめると、テーブルを$-\theta$だけ傾けた後のトップ側の振分長$f_\mathrm T'$は、
\begin{align*}
  f_\mathrm T' = f_\mathrm T+\left(\varDelta+\sqrt{R_\mathrm i'-\bar l^2}\right)\!\sin\theta\ .
\end{align*}
トップ側とボトム側の振分長が均等になるときの回転角$\theta'$は、
\begin{align*}
  \sin\theta' = \frac{f_\mathrm B-f_\mathrm T}{2\left(\varDelta+\sqrt{R_\mathrm i'-\bar l^2}\right)}\ .
\end{align*}
ここで、
\begin{align*}
  R_\mathrm i' = R_\mathrm c-\frac{W_x}2-\rho\ ,\quad
  \bar l = l-\frac\sigma2\ ,\quad
  f_d = \frac{f_\mathrm B-f_\mathrm T}2\ .
\end{align*}
\end{hosoku}
%%%%%%%%%%%%%%%%%%%%%%%%%%%%%%%%%%%%%%%%%%%%%%%%%%%%%%%%%%
%%%%%%%%%%%%%%%%%%%%%%%%%%%%%%%%%%%%%%%%%%%%%%%%%%%%%%%%%%
%%%%%%%%%%%%%%%%%%%%%%%%%%%%%%%%%%%%%%%%%%%%%%%%%%%%%%%%%%




%%%%%%%%%%%%%%%%%%%%%%%%%%%%%%%%%%%%%%%%%%%%%%%%%%%%%%%%%%
%%           %%%%%%%%%%%%%%%%%%%%%%%%%%%%%%%%%%%%%%%%%%%%%
%% chapter 2 %%%%%%%%%%%%%%%%%%%%%%%%%%%%%%%%%%%%%%%%%%%%%
%%           %%%%%%%%%%%%%%%%%%%%%%%%%%%%%%%%%%%%%%%%%%%%%
%%%%%%%%%%%%%%%%%%%%%%%%%%%%%%%%%%%%%%%%%%%%%%%%%%%%%%%%%%
\modHeadchapter{端面(外径)}
ここではモールドの端面における各々の位置を考える。
ただし機内のモールドは、考えている側の端面が工具側に向いているものとする。




%%%%%%%%%%%%%%%%%%%%%%%%%%%%%%%%%%%%%%%%%%%%%%%%%%%%%%%%%%
%% section 2.1 %%%%%%%%%%%%%%%%%%%%%%%%%%%%%%%%%%%%%%%%%%%
%%%%%%%%%%%%%%%%%%%%%%%%%%%%%%%%%%%%%%%%%%%%%%%%%%%%%%%%%%
\modHeadsection{トップ側の端面}



%%%%%%%%%%%%%%%%%%%%%%%%%%%%%%%%%%%%%%%%%%%%%%%%%%%%%%%%%%
%% subsection 2.1.1 %%%%%%%%%%%%%%%%%%%%%%%%%%%%%%%%%%%%%%
%%%%%%%%%%%%%%%%%%%%%%%%%%%%%%%%%%%%%%%%%%%%%%%%%%%%%%%%%%
\subsection{トップ端面における湾曲中心の位置}


%%%%%%%%%%%%%%%%%%%%%%%%%%%%%%%%%%%%%%%%%%%%%%%%%%%%%%%%%%
%% subsubsection 2.1.1.1 %%%%%%%%%%%%%%%%%%%%%%%%%%%%%%%%%
%%%%%%%%%%%%%%%%%%%%%%%%%%%%%%%%%%%%%%%%%%%%%%%%%%%%%%%%%%
\subsubsection{スペーサを用いた場合のT\texorpdfstring{$_{R_\mathrm c}'$}{Rc'}}
スペーサを取付けた後のトップ端面における湾曲中心の位置T$_{R_\mathrm c}'$と、テーブル中心Pとの$X$方向の差は、
\begin{align*}
  \left(
    R_\mathrm ce^{i\alpha_\mathrm c}
    -R_\mathrm i'e^{-i\alpha'_{\mathrm U_\mathrm B}}
    +R_\mathrm i'e^{-i\alpha_{\mathrm U_\mathrm B}}
  \right)
  -\varDelta'
  = R_\mathrm ce^{i\alpha_\mathrm c}-R_\mathrm i'e^{-i\alpha'_{\mathrm U_\mathrm B}}-\varDelta \qquad
    \left(\sin\alpha_\mathrm c = \frac{f_\mathrm T}{R_\mathrm c}\right)
\end{align*}
の実部を見ればよい。
したがって
%% footnote %%%%%%%%%%%%%%%%%%%%%
\footnote{この場合、トップ側が工具側に向いている。}、
%%%%%%%%%%%%%%%%%%%%%%%%%%%%%%%%%
%% label{eq:spacerTRc}
\begin{align}
  \notag
  &  R_\mathrm c\cos\alpha_\mathrm c-R_\mathrm i'\cos\alpha'_{\mathrm U_\mathrm B}-\varDelta\\
  &= -\varDelta+\sqrt{R_\mathrm c^2-f_\mathrm T^2}+\frac\delta2
     -\sqrt{R_\mathrm i'^2-\frac{\delta^2+(2\bar l)^2}4}
      \frac{2\bar l}{\sqrt{\delta^2+\left(2\bar l\right)^2}}
     \label{eq:spacerTRc}
\end{align}
で与えられる
%% footnote %%%%%%%%%%%%%%%%%%%%%
\footnote{実際の作業では、この点を(端面の湾曲中心T$_{R_\mathrm c}\!$でなく)端面の外側中心T$_\mathrm c$とみなすことが多い。}。
%%%%%%%%%%%%%%%%%%%%%%%%%%%%%%%%%


%%%%%%%%%%%%%%%%%%%%%%%%%%%%%%%%%%%%%%%%%%%%%%%%%%%%%%%%%%
%% subsubsection 2.1.1.2 %%%%%%%%%%%%%%%%%%%%%%%%%%%%%%%%%
%%%%%%%%%%%%%%%%%%%%%%%%%%%%%%%%%%%%%%%%%%%%%%%%%%%%%%%%%%
\subsubsection{テーブルを傾けた場合のT\texorpdfstring{$_{R_\mathrm c}'$}{Rc'}}
テーブルを傾けた後のトップ端面における湾曲中心の位置T$_{R_\mathrm c}'$と、テーブル中心Pとの$X$方向の差は、
\begin{align*}
  \left(R_\mathrm ce^{i\alpha_\mathrm c}-\varDelta'e^{-i\theta}+\varDelta'\right)-\varDelta'
  = R_\mathrm ce^{i\alpha_\mathrm c}-\varDelta'e^{-i\theta}
\end{align*}
の実部を見ればよい。
すなわち、
%% label{eq:tableTRc}
\begin{align}
  \label{eq:tableTRc}
  R_\mathrm c\cos\alpha_\mathrm c-\varDelta'\cos\theta
  = \sqrt{R_\mathrm c^2-f_\mathrm T^2}-\left(\varDelta+\sqrt{R_i'^2-\bar l^2}\right)\!\cos\theta~.
\end{align}



%%%%%%%%%%%%%%%%%%%%%%%%%%%%%%%%%%%%%%%%%%%%%%%%%%%%%%%%%%
%% subsection 2.1.2 %%%%%%%%%%%%%%%%%%%%%%%%%%%%%%%%%%%%%%
%%%%%%%%%%%%%%%%%%%%%%%%%%%%%%%%%%%%%%%%%%%%%%%%%%%%%%%%%%
\subsection{トップ端面における外側中心の位置}
トップ端における湾曲中心T$_{R_\mathrm c}'$と外径中心T$_\mathrm c'$との差は、以下で与えられる。
%% label{eq:TRc-Tc}
\begin{align}
  \label{eq:TRc-Tc}
  \sqrt{R_\mathrm c^2-f_\mathrm T^2}
  -\frac{\sqrt{R_\mathrm o^2-f_\mathrm T^2}+\sqrt{R_\mathrm i^2-f_\mathrm T^2}}2\ .
\end{align}
よって、外径中心T$_\mathrm c'$の位置は、湾曲中心T$_{R_\mathrm c}'$から\pageeqref{eq:TRc-Tc}だけ加味すればよい。
以下では、外径中心T$_\mathrm c'$の位置を直接計算し、このことが整合していることを確かめる。



%%%%%%%%%%%%%%%%%%%%%%%%%%%%%%%%%%%%%%%%%%%%%%%%%%%%%%%%%%
%% subsubsection 2.1.2.1 %%%%%%%%%%%%%%%%%%%%%%%%%%%%%%%%%
%%%%%%%%%%%%%%%%%%%%%%%%%%%%%%%%%%%%%%%%%%%%%%%%%%%%%%%%%%
\subsubsection{スペーサを用いた場合のT\texorpdfstring{$_\mathrm c'$}{c'}}
同様にして、外面A・C側のトップ端点T$_\mathrm o'$, T$_\mathrm i'$の、テーブル中心Pを原点とした場合の$X$座標はそれぞれ、
\begin{align*}
  \text{C側端点:}&
  -\varDelta+\sqrt{R_\mathrm i^2-f_\mathrm T^2}+\frac\delta2
  -\sqrt{R_\mathrm i'^2-\frac{\delta^2+(2\bar l)^2}4}\frac{2\bar l}{\sqrt{\delta^2+(2\bar l)^2}}\ ,\\
  \text{A側端点:}&
  -\varDelta+\sqrt{R_\mathrm o^2-f_\mathrm T^2}+\frac\delta2
  -\sqrt{R_\mathrm i'^2-\frac{\delta^2+(2\bar l)^2}4}\frac{2\bar l}{\sqrt{\delta^2+(2\bar l)^2}}\ .
\end{align*}
したがって、トップ端における外径中心T$_\mathrm c'$の$X$座標は、
%% label{eq:spacerTc}
\begin{align}
  \label{eq:spacerTc}
  -\varDelta+\frac{\sqrt{R_\mathrm o^2-f_\mathrm T^2}+\sqrt{R_\mathrm i^2-f_\mathrm T^2}}2
  +\frac\delta2-\sqrt{R_\mathrm i'^2-\frac{\delta^2+(2\bar l)^2}4}\frac{2\bar l}{\sqrt{\delta^2+(2\bar l)^2}}\ .
\end{align}
これより、湾曲中心T$_{R_\mathrm c}'$と外径中心T$_\mathrm c'$との差は\pageeqref{eq:TRc-Tc}となることがわかる。


%%%%%%%%%%%%%%%%%%%%%%%%%%%%%%%%%%%%%%%%%%%%%%%%%%%%%%%%%%
%% subsubsection 2.1.2.2 %%%%%%%%%%%%%%%%%%%%%%%%%%%%%%%%%
%%%%%%%%%%%%%%%%%%%%%%%%%%%%%%%%%%%%%%%%%%%%%%%%%%%%%%%%%%
\subsubsection{テーブルを傾けた場合のT\texorpdfstring{$_\mathrm c'$}{c'}}
同様にして、外面A・C側のトップ端点T$_\mathrm o'$, T$_\mathrm i'$の、テーブル中心Pを原点とした場合の$X$座標はそれぞれ、
\begin{subequations}
\begin{align}
%% label{eq:tableTi}
  \label{eq:tableTi}
  \text{C側端点:}&~
  \sqrt{R_\mathrm i^2-f_\mathrm T^2}-\varDelta'\cos\theta\ ,\\
  \text{A側端点:}&~
  \sqrt{R_\mathrm o^2-f_\mathrm T^2}-\varDelta'\cos\theta\ .
\end{align}
\end{subequations}
したがって、トップ端における(AC外径の)中点T$_\mathrm c'$の$X$座標は、
%% label{eq:tableTc}
\begin{align}
  \label{eq:tableTc}
  \frac{\sqrt{R_\mathrm o^2-f_\mathrm T^2}+\sqrt{R_\mathrm i^2-f_\mathrm T^2}}2-\varDelta'\cos\theta\ .
\end{align}
これより、湾曲中心T$_{R_\mathrm c}'$と外径中心T$_\mathrm c'$との差は\pageeqref{eq:TRc-Tc}となることがわかる。




\clearpage
%%%%%%%%%%%%%%%%%%%%%%%%%%%%%%%%%%%%%%%%%%%%%%%%%%%%%%%%%%
%% section 2.2 %%%%%%%%%%%%%%%%%%%%%%%%%%%%%%%%%%%%%%%%%%%
%%%%%%%%%%%%%%%%%%%%%%%%%%%%%%%%%%%%%%%%%%%%%%%%%%%%%%%%%%
\modHeadsection{ボトム側の端面}



%%%%%%%%%%%%%%%%%%%%%%%%%%%%%%%%%%%%%%%%%%%%%%%%%%%%%%%%%%
%% subsection 2.2.1 %%%%%%%%%%%%%%%%%%%%%%%%%%%%%%%%%%%%%%
%%%%%%%%%%%%%%%%%%%%%%%%%%%%%%%%%%%%%%%%%%%%%%%%%%%%%%%%%%
\subsection{ボトム端面における湾曲中心の位置}


%%%%%%%%%%%%%%%%%%%%%%%%%%%%%%%%%%%%%%%%%%%%%%%%%%%%%%%%%%
%% subsubsection 2.2.1.1 %%%%%%%%%%%%%%%%%%%%%%%%%%%%%%%%%
%%%%%%%%%%%%%%%%%%%%%%%%%%%%%%%%%%%%%%%%%%%%%%%%%%%%%%%%%%
\subsubsection{スペーサを用いた場合のB\texorpdfstring{$_{R_\mathrm c}'$}{Rc'}}
スペーサ取付後のボトム端面における湾曲中心B$_{R_\mathrm c}'$と、テーブル中心Pとの$X$方向の差は、トップ側の場合と同様に考えて
%% footnote %%%%%%%%%%%%%%%%%%%%%
\footnote{この場合は、ボトム側が工具側に向いている。}、
%%%%%%%%%%%%%%%%%%%%%%%%%%%%%%%%%
\begin{align*}
%  \label{eq:spacerBRc}
  \varDelta-\sqrt{R_\mathrm c^2-f_\mathrm B^2}-\frac\delta2
  +\sqrt{R_\mathrm i'^2-\frac{\delta^2+(2\bar l)^2}4}\frac{2\bar l}{\sqrt{\delta^2+(2\bar l)^2}}\ .
\end{align*}


%%%%%%%%%%%%%%%%%%%%%%%%%%%%%%%%%%%%%%%%%%%%%%%%%%%%%%%%%%
%% subsubsection 2.2.1.2 %%%%%%%%%%%%%%%%%%%%%%%%%%%%%%%%%
%%%%%%%%%%%%%%%%%%%%%%%%%%%%%%%%%%%%%%%%%%%%%%%%%%%%%%%%%%
\subsubsection{テーブルを傾けた場合のB\texorpdfstring{$_{R_\mathrm c}'$}{Rc'}}
テーブルを傾けた後のボトム端面における湾曲中心の位置B$_{R_\mathrm c}'$と、テーブル中心Pとの$X$方向の差は、トップ側の場合と同様に考えて
\begin{align*}
%  \label{eq:tableBRc}
  \left(\varDelta+\sqrt{R_i'^2-\bar l^2}\right)\!\cos\theta-\sqrt{R_\mathrm c^2-f_\mathrm B^2}~.
\end{align*}



%%%%%%%%%%%%%%%%%%%%%%%%%%%%%%%%%%%%%%%%%%%%%%%%%%%%%%%%%%
%% subsection 2.2.2 %%%%%%%%%%%%%%%%%%%%%%%%%%%%%%%%%%%%%%
%%%%%%%%%%%%%%%%%%%%%%%%%%%%%%%%%%%%%%%%%%%%%%%%%%%%%%%%%%
\subsection{ボトム端面における外側中心の位置}
ボトム端における湾曲中心B$_{R_\mathrm c}'$と外径中心B$_\mathrm c'$との差は、以下で与えられる。
%% label{eq:BRc-Bc}
\begin{align}
  \label{eq:BRc-Bc}
  \sqrt{R_\mathrm c^2-f_\mathrm B^2}
  -\frac{\sqrt{R_\mathrm o^2-f_\mathrm B^2}+\sqrt{R_\mathrm i^2-f_\mathrm B^2}}2\ .
\end{align}
よって、外径中心B$_\mathrm c'$の位置は、湾曲中心B$_{R_\mathrm c}'$から\pageeqref{eq:BRc-Bc}だけ加味すればよい。
以下では、外径中心B$_\mathrm c'$の位置を直接計算し、このことが整合していることを確かめる。


%%%%%%%%%%%%%%%%%%%%%%%%%%%%%%%%%%%%%%%%%%%%%%%%%%%%%%%%%%
%% subsubsection 2.2.2.1 %%%%%%%%%%%%%%%%%%%%%%%%%%%%%%%%%
%%%%%%%%%%%%%%%%%%%%%%%%%%%%%%%%%%%%%%%%%%%%%%%%%%%%%%%%%%
\subsubsection{スペーサを用いた場合のB\texorpdfstring{$_\mathrm c'$}{c'}}
外面A・C面側のボトム端点B$_{R_\mathrm o}'$, B$_{R_\mathrm i}'$の、テーブル中心Pを原点とした場合の$X$座標はそれぞれ、
\begin{align*}
  \text{C側端点:}&~~
  \varDelta-\sqrt{R_\mathrm i^2-f_\mathrm B^2}-\frac\delta2+\sqrt{R_\mathrm i'^2-\frac{\delta^2+(2\bar l)^2}4}\frac{2\bar l}{\sqrt{\delta^2+(2\bar l)^2}}\ ,\\
  \text{A側端点:}&~~
  \varDelta-\sqrt{R_\mathrm o^2-f_\mathrm B^2}-\frac\delta2+\sqrt{R_\mathrm i'^2-\frac{\delta^2+(2\bar l)^2}4}\frac{2\bar l}{\sqrt{\delta^2+(2\bar l^2}}\ .
\end{align*}
したがって、ボトム端における(AC外径の)中点B$_\mathrm c'$の$X$座標は、
%% label{eq:spacerBc}
\begin{align}
  \label{eq:spacerBc}
  \varDelta-\frac{\sqrt{R_\mathrm o^2-f_\mathrm B^2}+\sqrt{R_\mathrm i^2-f_\mathrm B^2}}2
  -\frac\delta2+\sqrt{R_\mathrm i'^2-\frac{\delta^2+(2\bar l)^2}4}\frac{2\bar l}{\sqrt{\delta^2+(2\bar l)^2}}\ .
\end{align}
これより、湾曲中心B$_{R_\mathrm c}'$と外径中心B$_\mathrm c'$との差は\pageeqref{eq:BRc-Bc}となることがわかる。


%%%%%%%%%%%%%%%%%%%%%%%%%%%%%%%%%%%%%%%%%%%%%%%%%%%%%%%%%%
%% subsubsection 2.2.2.2 %%%%%%%%%%%%%%%%%%%%%%%%%%%%%%%%%
%%%%%%%%%%%%%%%%%%%%%%%%%%%%%%%%%%%%%%%%%%%%%%%%%%%%%%%%%%
\subsubsection{テーブルを傾けた場合のB\texorpdfstring{$_\mathrm c'$}{c'}}
外面A・C面側のボトム端点B$_{R_\mathrm o}'$, B$_{R_\mathrm i}'$の、テーブル中心Pを原点とした場合の$X$座標はそれぞれ、
\begin{subequations}
\begin{align}
%% label{eq:tableBRi}
  \label{eq:tableBRi}
  \text{C側端点:}&~~
  \varDelta'\cos\theta-\sqrt{R_\mathrm i^2-f_\mathrm B^2}\ ,\\
  \text{A側端点:}&~~
  \varDelta'\cos\theta-\sqrt{R_\mathrm o^2-f_\mathrm B^2}\ .
\end{align}
\end{subequations}
したがって、ボトム端における(AC外径の)中点B$_\mathrm c'$の$X$座標は、
%% label{eq:tableBc}
\begin{align}
  \label{eq:tableBc}
  \varDelta'\cos\theta-\frac{\sqrt{R_\mathrm o^2-f_\mathrm B^2}+\sqrt{R_\mathrm i^2-f_\mathrm B^2}}2
\end{align}
これより、湾曲中心B$_{R_\mathrm c}'$と外径中心B$_\mathrm c'$との差は\pageeqref{eq:BRc-Bc}となることがわかる。




\clearpage
%%%%%%%%%%%%%%%%%%%%%%%%%%%%%%%%%%%%%%%%%%%%%%%%%%%%%%%%%%
%% section 2.3 %%%%%%%%%%%%%%%%%%%%%%%%%%%%%%%%%%%%%%%%%%%
%%%%%%%%%%%%%%%%%%%%%%%%%%%%%%%%%%%%%%%%%%%%%%%%%%%%%%%%%%
\modHeadsection{端面加工の工具径補正}
端面の加工として、$X+$, $Y+$方向の角から始めて
%% footnote %%%%%%%%%%%%%%%%%%%%%
\footnote{\DMname の場合、工具交換位置に近いので、このほうが移動距離が短くなる。}、
%%%%%%%%%%%%%%%%%%%%%%%%%%%%%%%%%
(工具から見て)時計回りに加工する場合を考える。
このとき加工の経路として、単純に外径の値を指定すれば加工することは可能である。
しかしその場合、工具(フェイスミル)のほぼ中心に近い位置で切削する形になるので、工具に大きな負荷がかかることになる。
これを避けるために、理想的には、内径$w_{\mathrm T, \mathrm B}$の外側に沿う形で切削するのが望ましい。
つまり、内径$w_{\mathrm T, \mathrm B}$を基準として工具半径分だけ(進行方向に対して左側に)補正をする形にすればよい。
ここでは誤差等を考慮して、内径から$\delta_w$だけ縮めた輪郭(の外側)に沿う形を考える。



\paragraph*{加工の開始可能範囲}\noindent
初めの位置は$X+$, $Y+$方向の角の上方($Y+$方向)に工具があるものとする。
工具の刃径(直径)を$\phi_\mathrm D$, 最大刃径(直径)$\phi'_\mathrm D$とすると
%% footnote %%%%%%%%%%%%%%%%%%%%%
\footnote{通常、刃径はDC、最大刃径はDCXと表記され、それぞれ直径として与えられることが多い。}、
%%%%%%%%%%%%%%%%%%%%%%%%%%%%%%%%%
$X$位置については、工具の中心が
\begin{align}
  \label{eq:tanmenKakouStartX}
  \frac{w_x}2-\delta_w+\frac{\phi_\mathrm D}2
\end{align}
にあればよい。
そのためここでは、まず$X$方向に絶対座標(G90)
\begin{align*}
  \frac{w_x}2-\delta_w
\end{align*}
まで移動し、その後に工具半径分の補正量として$\nicefrac{\phi_\mathrm D}2$だけ$X+$方向にずらす形で、下方向($Y-$方向)に移動する場合を考える。



%%%%%%%%%%%%%%%%%%%%%%%%%%%%%%%%%%%%%%%%%%%%%%%%%%%%%%%%%%
%% subsection 2.3.1 %%%%%%%%%%%%%%%%%%%%%%%%%%%%%%%%%%%%%%
%%%%%%%%%%%%%%%%%%%%%%%%%%%%%%%%%%%%%%%%%%%%%%%%%%%%%%%%%%
\subsection{工具径補正を用いる場合}
\verb|G41|を用いて工具径補正を行う場合を考える。
このとき、動き始めは$X$方向の補正分も加えて斜めに移動することになる。
ここで、
\begin{enumerate}
\item $X$, $Y$方向には同じ速さで動く
\item $X$方向の移動がなくなるまで工具はモールドに触れない
\end{enumerate}
とすると、加工(移動)の開始位置の$Y$座標は、
\begin{align*}
  \frac{W_y}2+\frac{\phi'_\mathrm D+\phi_\mathrm D}2
  = \frac{w_y}2+\tau_y+\frac{\phi'_\mathrm D+\phi_\mathrm D}2
\end{align*}
より上方向($Y+$方向)であればよい。
なお、$W_y$, $\tau_y$はBD方向の外径およびBD方向の肉厚である
%% footnote %%%%%%%%%%%%%%%%%%%%%
\footnote{ここでは話を単純化し、めっき膜厚や偏肉は無視している。}。
%%%%%%%%%%%%%%%%%%%%%%%%%%%%%%%%%



%%%%%%%%%%%%%%%%%%%%%%%%%%%%%%%%%%%%%%%%%%%%%%%%%%%%%%%%%%
%% subsection 2.3.2 %%%%%%%%%%%%%%%%%%%%%%%%%%%%%%%%%%%%%%
%%%%%%%%%%%%%%%%%%%%%%%%%%%%%%%%%%%%%%%%%%%%%%%%%%%%%%%%%%
\subsection{手動で補正を行う場合}
手動で補正する場合は、予め$X$位置を\pageeqref{eq:tanmenKakouStartX}に移動しておいて、そのまま下方向($Y-$方向)に移動すればよい。
よって、加工(移動)の開始位置の$Y$座標は、
\begin{align*}
  \frac{W_y+\phi_\mathrm D}2 = \frac{w_y+\phi_\mathrm D}2+\tau_y
\end{align*}
より上方向($Y+$方向)にあればよい%% footnote %%%%%%%%%%%%%%%%%%%%%
\footnote{実際のプログラムでは、安全を考慮して$\nicefrac{w_y}2+\phi'_\mathrm D$としている。
この場合、
\begin{align*}
  \phi'_\mathrm D > \frac{\phi_\mathrm D}2+\tau_y
\end{align*}
である限り、衝突は生じないことになる。
一般に、$\tau_y < \nicefrac{\phi_\mathrm D}2$であるので、これは常に満たされる。}。
%%%%%%%%%%%%%%%%%%%%%%%%%%%%%%%%%




%%%%%%%%%%%%%%%%%%%%%%%%%%%%%%%%%%%%%%%%%%%%%%%%%%%%%%%%%%
%%           %%%%%%%%%%%%%%%%%%%%%%%%%%%%%%%%%%%%%%%%%%%%%
%% chapter 3 %%%%%%%%%%%%%%%%%%%%%%%%%%%%%%%%%%%%%%%%%%%%%
%%           %%%%%%%%%%%%%%%%%%%%%%%%%%%%%%%%%%%%%%%%%%%%%
%%%%%%%%%%%%%%%%%%%%%%%%%%%%%%%%%%%%%%%%%%%%%%%%%%%%%%%%%%
\modHeadchapter{外削}
モールドに外削があるときは、たいていの場合、A側内面の端面における位置を基準として考えることが多い。

トップ・ボトム端における内径をそれぞれ$w_\mathrm T$, $w_\mathrm B$, 外削径をそれぞれ$\mathfrak W_\mathrm T$, $\mathfrak W_\mathrm B$, A側肉厚をそれぞれ$\tau_\mathrm T$, $\tau_\mathrm B$とする。
また、内面のめっき膜厚を$\mu$とし、通り心(トップ外削中心$\mathfrak T_\mathrm c$とボトム外削中心$\mathfrak B_\mathrm c$の差)の$X$, $Y$成分をそれぞれ$T_x$, $T_y$とする。
ただし、$T_x \geqq 0$として、トップ外削中心$\mathfrak T_\mathrm c$はボトム外削中心$\mathfrak B_\mathrm c$よりA面方向にあるものとする。
%%%%%%%%%%%%%%%%%%%%%%%%%%%%%%%%%%%%%%%%%%%%%%%%%%%%%%%%%%
%% hosoku %%%%%%%%%%%%%%%%%%%%%%%%%%%%%%%%%%%%%%%%%%%%%%%%
%%%%%%%%%%%%%%%%%%%%%%%%%%%%%%%%%%%%%%%%%%%%%%%%%%%%%%%%%%
\begin{hosoku}
内径$w_\mathrm T$, $w_\mathrm B$は、湾曲の中心O(またはO$'$)に向かった方向にあることに注意。
内径の中心がそれぞれの端に位置している。
\end{hosoku}
%%%%%%%%%%%%%%%%%%%%%%%%%%%%%%%%%%%%%%%%%%%%%%%%%%%%%%%%%%
%%%%%%%%%%%%%%%%%%%%%%%%%%%%%%%%%%%%%%%%%%%%%%%%%%%%%%%%%%
%%%%%%%%%%%%%%%%%%%%%%%%%%%%%%%%%%%%%%%%%%%%%%%%%%%%%%%%%%



%%%%%%%%%%%%%%%%%%%%%%%%%%%%%%%%%%%%%%%%%%%%%%%%%%%%%%%%%%
%% section 3.1 %%%%%%%%%%%%%%%%%%%%%%%%%%%%%%%%%%%%%%%%%%%
%%%%%%%%%%%%%%%%%%%%%%%%%%%%%%%%%%%%%%%%%%%%%%%%%%%%%%%%%%
\modHeadsection{ボトム側外削径の中心(ボトム基準)}
肉厚を基準とする場合、ボトム端のA側肉厚を基準にすることが多い。
このとき、ボトム端における外削径中心B$_\mathrm c'$から、ボトム端内径$w_\mathrm B$の半分を引き、さらにA側肉厚$\tau_\mathrm B$とめっき膜厚$\mu$との差を引いたものが(おおよその)外削A側面の位置$\mathfrak B_\mathrm o'$に相当する
%% footnote %%%%%%%%%%%%%%%%%%%%%
\footnote{ボトム側が工具側にある場合は、A面は$X$の負方向にあることに注意。}。
%%%%%%%%%%%%%%%%%%%%%%%%%%%%%%%%%


%%%%%%%%%%%%%%%%%%%%%%%%%%%%%%%%%%%%%%%%%%%%%%%%%%%%%%%%%%
%% subsubsection 3.1.1 %%%%%%%%%%%%%%%%%%%%%%%%%%%%%%%%%%%
%%%%%%%%%%%%%%%%%%%%%%%%%%%%%%%%%%%%%%%%%%%%%%%%%%%%%%%%%%
\subsection[スペーサを用いた場合の\texorpdfstring{$\mathfrak B_\mathrm c'$}{Bc'}]
           {スペーサを用いた場合の$\boldsymbol{\mathfrak B_\mathrm c'}$}
厚さ$\delta$のスペーサを用いた場合、テーブル中心Pを原点とした
%% footnote %%%%%%%%%%%%%%%%%%%%%
\footnote{マシニングによって機械原点(の$X$座標)がテーブル中心Pと同じだったり異なったりする場合がある。}\relax
%%%%%%%%%%%%%%%%%%%%%%%%%%%%%%%%%
ボトム側外削径の中心$\mathfrak B_\mathrm c'$の(おおよその)$X$座標は、\pageeqref{eq:spacerBc}より、
\begin{align*}
  \varDelta-\frac{\sqrt{R_\mathrm o^2-f_\mathrm B^2}+\sqrt{R_\mathrm i^2-f_\mathrm B^2}}2-\frac\delta2
  +\sqrt{R_\mathrm i'^2-\frac{\delta^2+(2\bar l)^2}4}\frac{2\bar l}{\sqrt{\delta^2+(2\bar l)^2}}
  -\frac{w_\mathrm B}2-\tau_\mathrm B+\frac{\mathfrak W_\mathrm B}2\ .
\end{align*}
%%%%%%%%%%%%%%%%%%%%%%%%%%%%%%%%%%%%%%%%%%%%%%%%%%%%%%%%%%
%% hosoku %%%%%%%%%%%%%%%%%%%%%%%%%%%%%%%%%%%%%%%%%%%%%%%%
%%%%%%%%%%%%%%%%%%%%%%%%%%%%%%%%%%%%%%%%%%%%%%%%%%%%%%%%%%
\begin{hosoku}
正確には、ボトム端における(内径ではなく)A・C面側の内面中心b$_\mathrm c'$を見る必要がある。
しかし実際の作業においては、これはタッチセンサーの測定開始点として用いるものであるため、おおよその値($\pm10$mm以内程度)で十分である。
そのため、ここでは単純に中心b$_\mathrm c'$の代わりにボトム外削径中心B$_\mathrm c'$とし、またボトム端における内径$w_\mathrm B$を用いている。
さらにいうと、外削径中心B$_\mathrm c'$はボトム端の湾曲中心B$_{\mathrm R_\mathrm c}'$で代用してもまず問題はない。
\end{hosoku}
%%%%%%%%%%%%%%%%%%%%%%%%%%%%%%%%%%%%%%%%%%%%%%%%%%%%%%%%%%
%%%%%%%%%%%%%%%%%%%%%%%%%%%%%%%%%%%%%%%%%%%%%%%%%%%%%%%%%%
%%%%%%%%%%%%%%%%%%%%%%%%%%%%%%%%%%%%%%%%%%%%%%%%%%%%%%%%%%
これは飽くまで(図面の数字をもとにした)計算値であり、タッチセンサーでの測定開始点として用いる。
そして、現物のボトム端に相当する箇所のA面(負方向)側内面b$_\mathrm o'$の位置を直接計測し、その位置を基準として(ワーク座標系の)原点$\mathfrak B_\mathrm c'$を定める。
計測で定めた原点$\mathfrak B_\mathrm c'$と、ボトム端A側内面b$_\mathrm o'$との差の$X$座標は、
\begin{align*}
  -\left(\frac{\mathfrak W_\mathrm B}2-\tau_\mathrm B+\mu\right).
\end{align*}


%%%%%%%%%%%%%%%%%%%%%%%%%%%%%%%%%%%%%%%%%%%%%%%%%%%%%%%%%%
%% subsubsection 3.1.2 %%%%%%%%%%%%%%%%%%%%%%%%%%%%%%%%%%%
%%%%%%%%%%%%%%%%%%%%%%%%%%%%%%%%%%%%%%%%%%%%%%%%%%%%%%%%%%
\subsection[テーブルを傾けた場合の\texorpdfstring{$\mathfrak B_\mathrm c'$}{Bc'}]
           {テーブルを傾けた場合の$\boldsymbol{\mathfrak B_\mathrm c'}$}
テーブルを$-\theta$傾けた場合、テーブル中心Pを原点としたボトム側外削径の中心$\mathfrak B_\mathrm c'$の(おおよその)$X$座標は、\pageeqref{eq:tableBc}より、
\begin{align*}
  \varDelta'\cos\theta-\frac{\sqrt{R_\mathrm o^2-f_\mathrm B^2}+\sqrt{R_\mathrm i^2-f_\mathrm B^2}}2
  -\frac{w_\mathrm B}2-\tau_\mathrm B+\frac{\mathfrak W_\mathrm B}2\ .
\end{align*}
計測して定めた原点$\mathfrak B_\mathrm c'$と、ボトム端A側内面b$_\mathrm o'$との差の$X$座標は、
\begin{align*}
  -\left(\frac{\mathfrak W_\mathrm B}2-\tau_\mathrm B+\mu\right).
\end{align*}


%%%%%%%%%%%%%%%%%%%%%%%%%%%%%%%%%%%%%%%%%%%%%%%%%%%%%%%%%%
%% subsection 3.1.3 %%%%%%%%%%%%%%%%%%%%%%%%%%%%%%%%%%%%%%
%%%%%%%%%%%%%%%%%%%%%%%%%%%%%%%%%%%%%%%%%%%%%%%%%%%%%%%%%%
\subsection{トップ側外削径中心(ボトム基準)}
トップ側にも外削がある場合、ボトム側外削から通り芯を指定する形でトップ外削の位置を決めるのが通常である。
このとき、テーブル中心Pを原点としたトップ側外削径中心$\mathfrak T_\mathrm c'$の$X$座標は、計測で定めた$\mathfrak B_\mathrm c'$の$X$座標$\mathcal G_{\mathrm Bx}$の符号を反転し
%% footnote %%%%%%%%%%%%%%%%%%%%%
\footnote{トップ側が工具側にある場合は、A面は$X$の正方向にある。
ボトム側と比べてテーブルを$B$軸($Y$軸まわり)に$180^\circ$回転する必要があるため、$X$座標の符号が反転する形になる。}、
%%%%%%%%%%%%%%%%%%%%%%%%%%%%%%%%%
通り芯$T_x$の分を加味すればよい。
したがって、
\begin{align*}
  -\mathcal G_{Bx}+T_x
\end{align*}
で与えられる
%% footnote %%%%%%%%%%%%%%%%%%%%%
\footnote{$Y$座標については、$B$軸の回転に影響しないので、$\mathcal G_{\mathrm By}+T_y$となる。
なお、実際の作業においては、$T_y = 0$であることが通常である。}。
%%%%%%%%%%%%%%%%%%%%%%%%%%%%%%%%%
ただし実際の作業では、テーブル中心Pの回転中心からのずれも考慮する必要がある
%% footnote %%%%%%%%%%%%%%%%%%%%%
\footnote{回転中心とテーブル中心は通常一致しているものとして考えるが、実際にはわずかにずれている。
特に$X$方向のずれは、$B$軸回転を伴う場合に効いてくる。}。
%%%%%%%%%%%%%%%%%%%%%%%%%%%%%%%%%



\clearpage
%%%%%%%%%%%%%%%%%%%%%%%%%%%%%%%%%%%%%%%%%%%%%%%%%%%%%%%%%%
%% section 3.2 %%%%%%%%%%%%%%%%%%%%%%%%%%%%%%%%%%%%%%%%%%%
%%%%%%%%%%%%%%%%%%%%%%%%%%%%%%%%%%%%%%%%%%%%%%%%%%%%%%%%%%
\modHeadsection{トップ側外削径の中心}


%%%%%%%%%%%%%%%%%%%%%%%%%%%%%%%%%%%%%%%%%%%%%%%%%%%%%%%%%%
%% subsection 3.2.1 %%%%%%%%%%%%%%%%%%%%%%%%%%%%%%%%%%%%%%
%%%%%%%%%%%%%%%%%%%%%%%%%%%%%%%%%%%%%%%%%%%%%%%%%%%%%%%%%%
\subsection[スペーサを用いた場合の\texorpdfstring{$\mathfrak T_\mathrm c'$}{Tc'}]
           {スペーサを用いた場合の$\boldsymbol{\mathfrak T_\mathrm c'}$}
トップ端外削A側面が基準となる場合も考慮しておく。
この場合も考えかたはボトム基準のそれと同様である。
テーブル中心Pを原点とした場合の、トップ側外削径の中心$\mathfrak T_\mathrm c'$のおおよその$X$座標は、\pageeqref{eq:spacerTc}より、
\begin{align*}
  -\varDelta+\frac{\sqrt{R_\mathrm o^2-f_\mathrm T^2}+\sqrt{R_\mathrm i^2-f_\mathrm T^2}}2+\frac\delta2
  -\sqrt{R_\mathrm i'^2-\frac{\delta^2+(2\bar l)^2}4}\frac{2\bar l}{\sqrt{\delta^2+(2\bar l)^2}}
  +\frac{w_\mathrm T}2+\tau_\mathrm T-\frac{\mathfrak W_\mathrm T}2\ .
\end{align*}
これをタッチセンサーでの測定開始点とし、計測した原点の$X$座標(実測値)を$\mathcal G_{tx}$とすると、トップ端におけるA面側内面と$\mathcal G_{tx}$との差の$X$座標は、
\begin{align*}
  \frac{\mathfrak W_\mathrm T}2-\tau_\mathrm T+\mu~.
\end{align*}


%%%%%%%%%%%%%%%%%%%%%%%%%%%%%%%%%%%%%%%%%%%%%%%%%%%%%%%%%%
%% subsection 3.2.2 %%%%%%%%%%%%%%%%%%%%%%%%%%%%%%%%%%%%%%
%%%%%%%%%%%%%%%%%%%%%%%%%%%%%%%%%%%%%%%%%%%%%%%%%%%%%%%%%%
\subsection[テーブルを傾けた場合の\texorpdfstring{$\mathfrak T_\mathrm c'$}{Tc'}]
           {テーブルを傾けた場合の$\boldsymbol{\mathfrak T_\mathrm c'}$}
テーブルを$-\theta$傾けた場合、テーブル中心Pを原点としたボトム側外削径の中心$\mathfrak T_\mathrm c'$の(おおよその)$X$座標は、\pageeqref{eq:tableTc}より、
\begin{align*}
  \frac{\sqrt{R_\mathrm o^2-f_\mathrm T^2}+\sqrt{R_\mathrm i^2-f_\mathrm T^2}}2-\varDelta'\cos\theta
  +\frac{w_\mathrm T}2+\tau_\mathrm T-\frac{\mathfrak W_\mathrm T}2\ .
\end{align*}
計測して定めた原点$\mathfrak T_\mathrm c'$と、トップ端A側内面t$_\mathrm o'$との差の$X$座標は、
\begin{align*}
  \frac{\mathfrak W_\mathrm T}2-\tau_\mathrm T+\mu~.
\end{align*}




%%%%%%%%%%%%%%%%%%%%%%%%%%%%%%%%%%%%%%%%%%%%%%%%%%%%%%%%%%
%% subsection 3.2.3 %%%%%%%%%%%%%%%%%%%%%%%%%%%%%%%%%%%%%%
%%%%%%%%%%%%%%%%%%%%%%%%%%%%%%%%%%%%%%%%%%%%%%%%%%%%%%%%%%
\subsection{ボトム側外削径中心(トップ基準)}
ボトム側にも外削がある場合、トップ側外削から通り芯を指定する形でボトム外削の位置を決めることが多い。
このとき、テーブル中心Pを原点としたボトム側外削径中心$\mathfrak B_\mathrm c'$の$X$座標は、計測で定めた$\mathfrak T_\mathrm c'$の$X$座標$\mathcal G_{\mathrm Tx}$の符号を反転し、通り芯$T_x$の分を加味すればよい。
したがって、
\begin{align*}
  -\mathcal G_{Tx}+T_x
\end{align*}
で与えられる。





%%%%%%%%%%%%%%%%%%%%%%%%%%%%%%%%%%%%%%%%%%%%%%%%%%%%%%%%%%
%%           %%%%%%%%%%%%%%%%%%%%%%%%%%%%%%%%%%%%%%%%%%%%%
%% chapter 4 %%%%%%%%%%%%%%%%%%%%%%%%%%%%%%%%%%%%%%%%%%%%%
%%           %%%%%%%%%%%%%%%%%%%%%%%%%%%%%%%%%%%%%%%%%%%%%
%%%%%%%%%%%%%%%%%%%%%%%%%%%%%%%%%%%%%%%%%%%%%%%%%%%%%%%%%%
\modHeadchapter{溝}
溝に関しては、その基準が以下のように与えられる場合が考えられる。
\begin{enumerate}
\item 溝径の中心Mが、モールドの湾曲中心線上にある場合
\item 溝径の中心Mが、(トップ側)外削径の中心線上にある場合
\item A面側の溝深さに指定がある場合
\end{enumerate}
なお、溝径を$W_\mathrm M$, 溝位置(端面から溝までの長さ)・溝幅・A側溝深さをそれぞれ$\kappa_p$, $\kappa_w$, $\kappa_d$とする。
このときいずれの場合も、$y$方向(機内における$Z$方向
%% footnote %%%%%%%%%%%%%%%%%%%%%
\footnote{計算上の$xy$座標($x$:実軸, $y$:虚軸)と、機内における$XZ$座標とが混在する形で話を進めているので注意されたし。})
%%%%%%%%%%%%%%%%%%%%%%%%%%%%%%%%%
の切削範囲は、テーブル中心Pを原点として、
\begin{align*}
  \big[f_\mathrm T'-(\kappa_p+\kappa_w)\ ,\ f_\mathrm T'-\kappa_p\big]
\end{align*}
であり、また溝中心M$'$の$y$座標はこの切削範囲の中央
\begin{align*}
  f_\mathrm T'-\left(\kappa_p+\frac{\kappa_w}2\right)
\end{align*}
で与えられる。



%%%%%%%%%%%%%%%%%%%%%%%%%%%%%%%%%%%%%%%%%%%%%%%%%%%%%%%%%%
%% section 4.1 %%%%%%%%%%%%%%%%%%%%%%%%%%%%%%%%%%%%%%%%%%%
%%%%%%%%%%%%%%%%%%%%%%%%%%%%%%%%%%%%%%%%%%%%%%%%%%%%%%%%%%
\modHeadsection{湾曲中心が基準の場合}
トップ端における湾曲中心T$_{R_\mathrm c}'$と溝中心M$'$との$x$方向の差は、
\begin{align*}
  \sqrt{R_\mathrm c^2-\left(f_\mathrm T-\kappa_p-\frac{\kappa_w}2\right)^{\!2}}
  -\sqrt{R_\mathrm c^2-f_\mathrm T^2}
\end{align*}
で与えられる。


%%%%%%%%%%%%%%%%%%%%%%%%%%%%%%%%%%%%%%%%%%%%%%%%%%%%%%%%%%
%% subsubsection 4.1.1 %%%%%%%%%%%%%%%%%%%%%%%%%%%%%%%%%%%
%%%%%%%%%%%%%%%%%%%%%%%%%%%%%%%%%%%%%%%%%%%%%%%%%%%%%%%%%%
\subsection{スペーサを用いた場合の溝中心(湾曲中心基準)}
溝中心M$'$がモールドの湾曲中心線上にある場合、テーブル中心Pを原点とした$x$座標は、\pageeqref{eq:spacerTRc}より、
\begin{align*}
  -\varDelta+\sqrt{R_\mathrm c^2-\left(f_\mathrm T-\kappa_p-\frac{\kappa_w}2\right)^{\!2}}+\frac\delta2
  -\sqrt{R_\mathrm i'^2-\frac{\delta^2+(2\bar l)^2}4}\frac{2\bar l}{\sqrt{\delta^2+\left(2\bar l\right)^2}}
\end{align*}
となる。
なお実際の作業では、簡単のため、トップ端面の外側中心T$_\mathrm c'$を測定し、それをトップ端面における湾曲中心T$_{R_\mathrm c}'$とみなして溝中心M$'$の位置を計算することが多い。
実測した外側中心の$X$・$Y$座標$G_{\mathrm Tx}$, $G_{\mathrm Ty}$を湾曲中心のそれとみなすと、機内における溝中心M$'$の位置は、テーブル中心Pを原点として、
%% label{eq:Mreal}
\begin{subequations}
  \label{eq:Mreal}
\begin{align}
  \left(
    G_{\mathrm Tx}
    +\sqrt{R_\mathrm c^2-\left(f_\mathrm T-\kappa_p-\frac{\kappa_w}2\right)^{\!2}}
    -\sqrt{R_\mathrm c^2-f_\mathrm T^2}\ ,\
    G_{\mathrm Ty}~,~
    f_\mathrm T'-\kappa_p-\frac{\kappa_w}2
  \right).
\end{align}
湾曲中心とみなさずに正確に求めるなら、これに\pageeqref{eq:TRc-Tc}を引けばよい。
その場合の$X$座標は、
\begin{align}
  G_{\mathrm Tx}
  +\sqrt{R_\mathrm c^2-\left(f_\mathrm T-\kappa_p-\frac{\kappa_w}2\right)^{\!2}}
  -\frac{\sqrt{R_\mathrm o^2-f_\mathrm T^2}+\sqrt{R_\mathrm i^2-f_\mathrm T^2}}2\ .
\end{align}
\end{subequations}


%%%%%%%%%%%%%%%%%%%%%%%%%%%%%%%%%%%%%%%%%%%%%%%%%%%%%%%%%%
%% subsubsection 4.1.2 %%%%%%%%%%%%%%%%%%%%%%%%%%%%%%%%%%%
%%%%%%%%%%%%%%%%%%%%%%%%%%%%%%%%%%%%%%%%%%%%%%%%%%%%%%%%%%
\subsection{テーブルを傾けた場合の溝中心(湾曲中心基準)}
溝中心M$'$がモールドの湾曲中心線上にある場合、テーブル中心Pを原点とした$x$座標は、\pageeqref{eq:tableTRc}より、
\begin{align*}
  \sqrt{R_\mathrm c^2-\left(f_\mathrm T-\kappa_p-\frac{\kappa_w}2\right)^{\!2}}
  -\varDelta'\cos\theta\ .
\end{align*}
実測した外側中心の$X$・$Y$座標$G_{\mathrm Tx}$, $G_{\mathrm Ty}$を湾曲中心のそれとみなした場合とそうでない場合は、\pageeqref{eq:Mreal}で与えられる。




\clearpage
%%%%%%%%%%%%%%%%%%%%%%%%%%%%%%%%%%%%%%%%%%%%%%%%%%%%%%%%%%
%% section 4.2 %%%%%%%%%%%%%%%%%%%%%%%%%%%%%%%%%%%%%%%%%%%
%%%%%%%%%%%%%%%%%%%%%%%%%%%%%%%%%%%%%%%%%%%%%%%%%%%%%%%%%%
\modHeadsection{外削径の中心が基準の場合}
溝中心M$'$がトップ外削径の中心線上にある場合、機内におけるその位置座標は、
\begin{align*}
  \left(
    -\mathcal G_{Bx}+T_x\ ,\
    \mathcal G_{By}\ ,\
    f_\mathrm T'-\kappa_p-\frac{\kappa_w}2
  \right) \qquad
  \text{または}\qquad
  \left(
    \mathcal G_{Bx}\ ,\
    \mathcal G_{By}\ ,\
    f_\mathrm T'-\kappa_p-\frac{\kappa_w}2
  \right).
\end{align*}
ただし、前者はボトムの外削を基準にした(ボトム基準の通り芯がある)場合であり、後者はトップの外削を基準にした場合である。




%%%%%%%%%%%%%%%%%%%%%%%%%%%%%%%%%%%%%%%%%%%%%%%%%%%%%%%%%%
%% section 4.3 %%%%%%%%%%%%%%%%%%%%%%%%%%%%%%%%%%%%%%%%%%%
%%%%%%%%%%%%%%%%%%%%%%%%%%%%%%%%%%%%%%%%%%%%%%%%%%%%%%%%%%
\modHeadsection{A面側の溝深さが基準の場合}
モールドA側面からの溝深さが指定されている場合を考える。
このとき、溝中心の位置の$X$座標は、テーブル中心Pを原点として、
\begin{align*}
  \sqrt{R_\mathrm o^2-\left(f_\mathrm T-\kappa_p-\frac{\kappa_w}2\right)^{\!2}}-\kappa_d-\frac{W_\mathrm M}2
  -\varDelta'
\end{align*}
で与えられる。
ここで、$W_{mx}$は溝のAC方向の径を表す。
なお実際の作業では、モールドのA側外面の溝幅中央に相当する箇所を直接計測し、その位置を基準として原点を割り出す。
その原点の$X$座標(実測値)を$G_{mx}$とすると、溝幅中央に対するモールドのA側外面と溝中心$G_{mx}$との差は、
\begin{align*}
  \frac{W_m}2+\kappa_d
\end{align*}
となる。





%%%%%%%%%%%%%%%%%%%%%%%%%%%%%%%%%%%%%%%%%%%%%%%%%%%%%%%%%%
%%           %%%%%%%%%%%%%%%%%%%%%%%%%%%%%%%%%%%%%%%%%%%%%
%% chapter 5 %%%%%%%%%%%%%%%%%%%%%%%%%%%%%%%%%%%%%%%%%%%%%
%%           %%%%%%%%%%%%%%%%%%%%%%%%%%%%%%%%%%%%%%%%%%%%%
%%%%%%%%%%%%%%%%%%%%%%%%%%%%%%%%%%%%%%%%%%%%%%%%%%%%%%%%%%
\modHeadchapter{面取}
ここでは主に、テーパーエンドミルを用いたC面取について考える。
テーパーエンドミルは、その名の通りテーパーの付いた工具であり、先端が平坦になっているものも多い。
しかし、先端部分を工具長として設定すると、その部分が段差となりテーパー加工が適切に行うことができない。
そのため、先端部分から一定の距離$d_\mathrm e$だけずらした箇所を工具長として設定し、またその箇所の直径(参照直径)$D_\mathrm r$を工具直径として補正を行うことが推奨される。

ここで、先端径(直径)
%% footnote %%%%%%%%%%%%%%%%%%%%%
\footnote{先端が平坦でなく尖っている場合は$D_\mathrm e = 0$とする。}
%%%%%%%%%%%%%%%%%%%%%%%%%%%%%%%%%
およびテーパーの角度(片角)を$D_\mathrm e$, $\xi_\mathrm e$とすると、参照直径$D_\mathrm r$は
\begin{align*}
  D_\mathrm r
  = (D_\mathrm e\cot\xi_\mathrm e+2d_\mathrm e)\tan\xi_\mathrm e
  = D_\mathrm e+2d_\mathrm e\tan\xi_\mathrm e
\end{align*}
で与えられる。
通常、工具径補正は工具の半径を用いて行うので、工具径を
\begin{align*}
  \frac{D_\mathrm r}2 = \frac{D_\mathrm e}2+d_\mathrm e\tan\xi_\mathrm e
\end{align*}
として設定すればよい。
あるいは、先端径を基準としてそこから補正を行う形にする場合は、その差
\begin{align*}
  \frac{D_\mathrm r}2-\frac{D_\mathrm e}2 = d_\mathrm e\tan\xi_\mathrm e
\end{align*}
だけ補正すればよい。
%%%%%%%%%%%%%%%%%%%%%%%%%%%%%%%%%%%%%%%%%%%%%%%%%%%%%%%%%%
%% hosoku %%%%%%%%%%%%%%%%%%%%%%%%%%%%%%%%%%%%%%%%%%%%%%%%
%%%%%%%%%%%%%%%%%%%%%%%%%%%%%%%%%%%%%%%%%%%%%%%%%%%%%%%%%%
\begin{hosoku}
なお、$\xi_\mathrm e = \nicefrac\pi{12}$\,($15^\circ$), $\nicefrac\pi6$\,($30^\circ$), $\nicefrac\pi4$\,($45^\circ$)のとき、それぞれ
\begin{align*}
  \tan\frac\pi{12} = 2-\sqrt3\ , \quad
  \tan\frac\pi6 = \frac1{\sqrt3}\ , \quad
  \tan\frac\pi4 = 1\ .
\end{align*}
\end{hosoku}\relax
%%%%%%%%%%%%%%%%%%%%%%%%%%%%%%%%%%%%%%%%%%%%%%%%%%%%%%%%%%
%%%%%%%%%%%%%%%%%%%%%%%%%%%%%%%%%%%%%%%%%%%%%%%%%%%%%%%%%%
%%%%%%%%%%%%%%%%%%%%%%%%%%%%%%%%%%%%%%%%%%%%%%%%%%%%%%%%%%




%%%%%%%%%%%%%%%%%%%%%%%%%%%%%%%%%%%%%%%%%%%%%%%%%%%%%%%%%%
%%           %%%%%%%%%%%%%%%%%%%%%%%%%%%%%%%%%%%%%%%%%%%%%
%% chapter 6 %%%%%%%%%%%%%%%%%%%%%%%%%%%%%%%%%%%%%%%%%%%%%
%%           %%%%%%%%%%%%%%%%%%%%%%%%%%%%%%%%%%%%%%%%%%%%%
%%%%%%%%%%%%%%%%%%%%%%%%%%%%%%%%%%%%%%%%%%%%%%%%%%%%%%%%%%
\modHeadchapter{内面溝}
ここでは主に内面溝に関する計測・加工に必要な、モールドの幾何学的性質を考える。

なお、内面溝の加工は\MMname で行うことはできず、\DMname のみで行う。
また\DMname では、振分長の調整についてスペーサを用いた方法は行わず、テーブルの回転を用いた方法のみで行う方針である。
したがって、スペーサを用いた方法の場合は考慮する必要がない。
そのため以降では、(内面溝に関する計測・加工については)テーブルを$-\theta$だけ回転した場合についてのみを考えることにする。




%%%%%%%%%%%%%%%%%%%%%%%%%%%%%%%%%%%%%%%%%%%%%%%%%%%%%%%%%%
%% section 6.1 %%%%%%%%%%%%%%%%%%%%%%%%%%%%%%%%%%%%%%%%%%%
%%%%%%%%%%%%%%%%%%%%%%%%%%%%%%%%%%%%%%%%%%%%%%%%%%%%%%%%%%
\modHeadsection{内面溝のノーテーション}
初めに、内面溝に関するノーテーションを簡単にまとめておく。
なお内面溝は振分けのトップ側にあるため、モールドはトップ側が工具側に向いているものとして話を進める。
%%%%%%%%%%%%%%%%%%%%%%%%%%%%%%%%%%%%%%%%%%%%%%%%%%%%%%%%%%
%% tcolorbox %%%%%%%%%%%%%%%%%%%%%%%%%%%%%%%%%%%%%%%%%%%%%
%%%%%%%%%%%%%%%%%%%%%%%%%%%%%%%%%%%%%%%%%%%%%%%%%%%%%%%%%%
\begin{tcolorbox}[title={内面溝に関するノーテーション}, fonttitle=\gtfamily\bfseries, breakable, enhanced jigsaw]
\begin{enumerate}
\item
\subparagraph*{列の数えかた}
内面溝は$m$列あるものとし、トップ側から順に1列目, 2列目, …,$m$列目のように数える。

\item
\subparagraph*{列内の個数の数えかた}
各々の列の内面溝は、AC面側については工具側からみて下から順に、BD面については工具側からみて右から順に1つ目,2つ目,…のように数える。

\item
\subparagraph*{内面溝の寸法}
トップ端面から1列目までの距離を$q$, 鉛直・水平方向のピッチをそれぞれ$p_z$, $p_x$とし、$i$列目の長さをそれぞれ$d_i$とする。

特に、奇数列目の長さが全て同じ場合はその長さを$d_\mathrm o$, 偶数列目の長さが全て同じ場合はその長さを$d_\mathrm e$とも表記する。
(\pageautoref{fn:generallyDimpleN}および\pageautoref{hosoku:generallyDimpleN}参照)

\item
\subparagraph*{内径テーパ表の寸法}
内径テーパ表におけるトップ端からの距離を$\lambda_i$ ($i = 0$, $1$, $2$, $\cdots$), それに対するAC・BD側内径をそれぞれ$w_{\mathrm Ai}$, $w_{\mathrm Bi}$とする。
(\pageautoref{hosoku:example4taper}参照)

\item
\subparagraph*{内径の(近似)寸法}
トップ端から$\lambda$の位置のAC内径を$w_{\mathrm A\lambda}$と表す。
このとき$w_{\mathrm A\lambda}$は、$\lambda_j \leqq \lambda < \lambda_{j+1}$に対する$w_{\mathrm Aj}$, $w_{\mathrm Aj+1}$の加重算術平均(ウェイト算術平均)
\begin{align*}
  w_{\mathrm A\lambda}
  = \frac{(\lambda-\lambda_j)w_{\mathrm Aj+1}+(\lambda_{j+1}-\lambda)w_{\mathrm Aj}}{\lambda_{j+1}-\lambda_j}
  \qquad
  \Big(\lambda_j \leqq \lambda < \lambda_{j+1}\Big)
\end{align*}
とみなすことにする。($w_{\mathrm B\lambda}$についても同様)

\item
\subparagraph*{めっき厚を含めた内径の(近似)寸法}
めっき膜厚$\mu$を考慮したAC・BD内径$w'_{\mathrm A\lambda}$, $w'_{\mathrm B\lambda}$をそれぞれ以下のように表す。
\begin{align*}
  w'_{\mathrm A\lambda} \equiv w_{\mathrm A\lambda}+2\mu~, \quad
  w'_{\mathrm B\lambda} \equiv w_{\mathrm B\lambda}+2\mu~.
\end{align*}
\end{enumerate}
\end{tcolorbox}\noindent
%%%%%%%%%%%%%%%%%%%%%%%%%%%%%%%%%%%%%%%%%%%%%%%%%%%%%%%%%%
%%%%%%%%%%%%%%%%%%%%%%%%%%%%%%%%%%%%%%%%%%%%%%%%%%%%%%%%%%
%%%%%%%%%%%%%%%%%%%%%%%%%%%%%%%%%%%%%%%%%%%%%%%%%%%%%%%%%%
このとき$m$列目の内面溝の個数$n_m$は、$n_m = \nicefrac{d_m}{p_x}+1$となる
%% footnote %%%%%%%%%%%%%%%%%%%%%
\footnote{\label{fn:generallyDimpleN}%
たいていの場合、奇数列の個数は全て同じ数$n_\mathrm o$であり、偶数列の個数も全て同じ$n_\mathrm e$である。
また$|n_\mathrm o-n_\mathrm d| = 1$である。}。
%%%%%%%%%%%%%%%%%%%%%%%%%%%%%%%%%
%%%%%%%%%%%%%%%%%%%%%%%%%%%%%%%%%%%%%%%%%%%%%%%%%%%%%%%%%%
%% hosoku %%%%%%%%%%%%%%%%%%%%%%%%%%%%%%%%%%%%%%%%%%%%%%%%
%%%%%%%%%%%%%%%%%%%%%%%%%%%%%%%%%%%%%%%%%%%%%%%%%%%%%%%%%%
\begin{hosoku}[label=hosoku:example4taper]
たとえば内径テーパ表の値が25mmピッチの場合、$\lambda_0=0$, $\lambda_1=25$, $\lambda_2=50$, $\cdots$とし、それぞれのACおよびBD側内径を$w_{\mathrm A0}$, $w_{\mathrm A1}$, $w_{\mathrm A2}$, $\cdots$および$w_{\mathrm B0}$, $w_{\mathrm B1}$, $w_{\mathrm B2}$, $\cdots$とする、という意味である。
ここでは離散値である$\lambda_i$を、連続値$\lambda$に(近似的に)置きかえている。
実際、たとえば$\lambda = \lambda_j$のとき$w_{\mathrm Aj} = w_{\mathrm A\lambda}$となることがわかる。
\end{hosoku}\relax
%%%%%%%%%%%%%%%%%%%%%%%%%%%%%%%%%%%%%%%%%%%%%%%%%%%%%%%%%%
%%%%%%%%%%%%%%%%%%%%%%%%%%%%%%%%%%%%%%%%%%%%%%%%%%%%%%%%%%
%%%%%%%%%%%%%%%%%%%%%%%%%%%%%%%%%%%%%%%%%%%%%%%%%%%%%%%%%%
%%%%%%%%%%%%%%%%%%%%%%%%%%%%%%%%%%%%%%%%%%%%%%%%%%%%%%%%%%
%% hosoku %%%%%%%%%%%%%%%%%%%%%%%%%%%%%%%%%%%%%%%%%%%%%%%%
%%%%%%%%%%%%%%%%%%%%%%%%%%%%%%%%%%%%%%%%%%%%%%%%%%%%%%%%%%
\begin{hosoku}
内径テーパの$Z$方向のピッチ$\lambda_{i+1}-\lambda_i$は常に一定の場合が多い。
$\lambda_{i+1}-\lambda_i$が$i$について常に一定であれば、$\lambda_j \leqq z < \lambda_{j+1}$となる$j$は、
\begin{align*}
  j = z \bDiv (\lambda_{i+1}-\lambda_i) = \left\lfloor\frac z{\lambda_{i+1}-\lambda_i}\right\rfloor
\end{align*}
のように表すことができる。
\end{hosoku}
%%%%%%%%%%%%%%%%%%%%%%%%%%%%%%%%%%%%%%%%%%%%%%%%%%%%%%%%%%
%%%%%%%%%%%%%%%%%%%%%%%%%%%%%%%%%%%%%%%%%%%%%%%%%%%%%%%%%%
%%%%%%%%%%%%%%%%%%%%%%%%%%%%%%%%%%%%%%%%%%%%%%%%%%%%%%%%%%
%%%%%%%%%%%%%%%%%%%%%%%%%%%%%%%%%%%%%%%%%%%%%%%%%%%%%%%%%%
%% Column %%%%%%%%%%%%%%%%%%%%%%%%%%%%%%%%%%%%%%%%%%%%%%%%
%%%%%%%%%%%%%%%%%%%%%%%%%%%%%%%%%%%%%%%%%%%%%%%%%%%%%%%%%%
\begin{Column}{商$\boldsymbol{\bDiv}$と余り$\boldsymbol{\bmod}$とガウス括弧$\boldsymbol{\lfloor\,\rfloor}$}
\renewcommand\theequation{c\thechapter.\arabic{equation}}
\setcounter{equation}{0}
\paragraph*{$\boldsymbol\bDiv$と$\boldsymbol\bmod$}
割り算の余りを表す記号としては$\bmod$が広く使われる。
商を表す記号は一般的な数学の教科書等ではあまり用いられないが、プログラミング言語等では$\bDiv$を用いられることがある。
これに倣って、ここでは商には$\bDiv$, 余りには$\bmod$を用いている。

 一般に、実数$a$, $b$ ($b\neq0$)に対して$a = bq+r$ ($0 \leqq r < |b|$)を満たす整数$q$を商、$r$を余りと呼び、このとき$a \bDiv b = q$および$a \bmod b = r$のように表される。
なお、ここでは簡単のため、$q \geqq 0$として考えることにする。
\tcbline*
\paragraph*{ガウス括弧}
$\lfloor x\rfloor$は、$x \in R$ に対して$x$を超えない最大の整数。
簡単にいうと、($x > 0$の場合は)小数点以下を切り捨てた整数部分を表す。
ガウス記号, 床関数(floor function)などとも呼ばれる。
\end{Column}
%%%%%%%%%%%%%%%%%%%%%%%%%%%%%%%%%%%%%%%%%%%%%%%%%%%%%%%%%%
%%%%%%%%%%%%%%%%%%%%%%%%%%%%%%%%%%%%%%%%%%%%%%%%%%%%%%%%%%
%%%%%%%%%%%%%%%%%%%%%%%%%%%%%%%%%%%%%%%%%%%%%%%%%%%%%%%%%%
\begin{tcolorbox}[title={内面溝加工に関する工具(T31, T50)の情報}, fonttitle=\gtfamily\bfseries]
・タッチセンサープローブの半径:5 \quad ・タッチセンサープローブの軸の半径:3.75\\
・切削用工具径:40 \quad ・切削用工具シャンク径:25
\end{tcolorbox}




\clearpage
%%%%%%%%%%%%%%%%%%%%%%%%%%%%%%%%%%%%%%%%%%%%%%%%%%%%%%%%%%
%% section 6.2 %%%%%%%%%%%%%%%%%%%%%%%%%%%%%%%%%%%%%%%%%%%
%%%%%%%%%%%%%%%%%%%%%%%%%%%%%%%%%%%%%%%%%%%%%%%%%%%%%%%%%%
\modHeadsection{基本方針}
内面溝の加工における留意事項の1つに、モールドの内面(特にトップ端)と工具が接触してしまうアンダーカットというものがある
%% footnote %%%%%%%%%%%%%%%%%%%%%
\footnote{内面溝の測定・加工ではとりわけアンダーカットが生じやすい、という意味である。
その他の計測・加工についても当然アンダーカットは十分に生じうる。}。
%%%%%%%%%%%%%%%%%%%%%%%%%%%%%%%%%
特にモールドA面には工具へ向かう方向に湾曲があるため、アンダーカットが生じやすい。
そこで、アンダーカットを避けつつ加工ができるようにするため、モールドをいくらか(湾曲と反対側に)傾けて加工を行う。
その傾き角$\phi$ ($0 \leqq \phi < \nicefrac\pi2$)について、ここでは次の2点を基準に考えることにする。
\begin{tcolorbox}[title=A面の内面溝, fonttitle=\gtfamily\bfseries]
\begin{enumerate}
\item[a)]
A側内面のトップ端点
\item[b)]
A側内面の内面溝1列目(トップ端から$q$)の位置
\end{enumerate}
\end{tcolorbox}\noindent
この2点を通る直線と鉛直方向との角度を、傾き角$-\phi$とする
%% footnote %%%%%%%%%%%%%%%%%%%%%
\footnote{振分長の調整に用いたテーブルの傾き角$\theta$と混同しないように注意。}。
%%%%%%%%%%%%%%%%%%%%%%%%%%%%%%%%%
なお、トップ端のAC内径は$w'_{\mathrm A0}$で代用してもよいものとする。
このとき$\phi > 0$となる(C面側に傾く)場合は$\phi$だけ傾けて加工を行う。
一方、$\phi \leqq 0$となる(A面側に傾く)場合は、そもそもアンダーカットが生じないので、傾けずにそのまま加工を行うものとする。
%%%%%%%%%%%%%%%%%%%%%%%%%%%%%%%%%%%%%%%%%%%%%%%%%%%%%%%%%%
%% hosoku %%%%%%%%%%%%%%%%%%%%%%%%%%%%%%%%%%%%%%%%%%%%%%%%
%%%%%%%%%%%%%%%%%%%%%%%%%%%%%%%%%%%%%%%%%%%%%%%%%%%%%%%%%%
\begin{hosoku}
ここでは内面溝の工具として、Tスロットカッターを考えている。
しかし、当然ながら工具径は有限であるため、いくら適切に傾けたところで限界はある。
ここではその限界として、A側内面のトップ端の$X$座標と、それと最も$X$座標が近い内面溝との($X$方向の)距離を算出する。
そしてそれを工具径と比べることで、どこまでの範囲を加工するかを決定する。
加工できない部分に内面溝がある場合は、別の工具(アングルヘッド)を使用して加工を行う。
\end{hosoku}
%%%%%%%%%%%%%%%%%%%%%%%%%%%%%%%%%%%%%%%%%%%%%%%%%%%%%%%%%%
%%%%%%%%%%%%%%%%%%%%%%%%%%%%%%%%%%%%%%%%%%%%%%%%%%%%%%%%%%
%%%%%%%%%%%%%%%%%%%%%%%%%%%%%%%%%%%%%%%%%%%%%%%%%%%%%%%%%%
%%%%%%%%%%%%%%%%%%%%%%%%%%%%%%%%%%%%%%%%%%%%%%%%%%%%%%%%%%
%% Column %%%%%%%%%%%%%%%%%%%%%%%%%%%%%%%%%%%%%%%%%%%%%%%%
%%%%%%%%%%%%%%%%%%%%%%%%%%%%%%%%%%%%%%%%%%%%%%%%%%%%%%%%%%
\begin{Column}{曲率と傾き}
内面A側・C側の湾曲をそれぞれ$\mathcal R_\mathrm o$, $\mathcal R_\mathrm i$とすると、曲率はそれぞれ$\mathcal R_\mathrm o^{-1} < R_\mathrm c^{-1} < \mathcal R_\mathrm i^{-1}$である。
そのため、(トップ側の)A側の$\mathcal R_\mathrm o$を基準にするとより緩やかに、C側の$\mathcal R_\mathrm i$を基準にするとよりきつく傾くことになる。
また、トップ端から($Z$方向に)遠い点を基準にするとより緩やかに、近い点を基準にするとよりきつく傾くことになる。
\end{Column}
%%%%%%%%%%%%%%%%%%%%%%%%%%%%%%%%%%%%%%%%%%%%%%%%%%%%%%%%%%
%%%%%%%%%%%%%%%%%%%%%%%%%%%%%%%%%%%%%%%%%%%%%%%%%%%%%%%%%%
%%%%%%%%%%%%%%%%%%%%%%%%%%%%%%%%%%%%%%%%%%%%%%%%%%%%%%%%%%

以下ではこの傾き角$\phi$と、回転後の内面溝や内面の位置を定量的に与えることを試みる。




\clearpage
%%%%%%%%%%%%%%%%%%%%%%%%%%%%%%%%%%%%%%%%%%%%%%%%%%%%%%%%%%
%% section 5.3 %%%%%%%%%%%%%%%%%%%%%%%%%%%%%%%%%%%%%%%%%%%
%%%%%%%%%%%%%%%%%%%%%%%%%%%%%%%%%%%%%%%%%%%%%%%%%%%%%%%%%%
\modHeadsection{内面溝の位置と傾き角(傾き前)}
\pageeqref{eq:tableTRc}より、テーブルを$-\theta$傾けて振分長の調整を行った場合、テーブル中心Pを原点としたモールド中心湾曲線のトップ端における$X$座標は、
\begin{align*}
  R_\mathrm c\cos\alpha_\mathrm c-\varDelta'\cos\theta = \sqrt{R_\mathrm c^2-f_\mathrm T^2}-\varDelta'\cos\theta
\end{align*}
で与えられる。
これはタッチセンサーによる測定の開始点として用いることができる。
一方で、それ以外の作業では、トップ端における内径の中心座標$g_t$を直接測定するので、それを用いることにする
%% footnote %%%%%%%%%%%%%%%%%%%%%
\footnote{これは中心湾曲線上にない点であるが、公差の範囲内であるものとして、ここではこれで代用する。}。
%%%%%%%%%%%%%%%%%%%%%%%%%%%%%%%%%
よって、テーブル中心Pを原点とした場合における、内面溝1列目中央の(だいたいの)位置
%% footnote %%%%%%%%%%%%%%%%%%%%%
\footnote{$w_{\mathrm Aq}$, $w_{\mathrm Bq}$はモールド湾曲の中心(0, 0)方向への長さであるため正確ではないことに注意。}\relax
%%%%%%%%%%%%%%%%%%%%%%%%%%%%%%%%%
は、次で与えられる。
\begin{align*}
\begin{array}{rl}
  \text{A面($+X$方向):}
  & \displaystyle
    \left(
      g_{tx}+\mathcal L_0+\frac{w'_{\mathrm Aq}}2~,~
      g_{ty}~,~
      f_t'-q
    \right),\\[12pt]
  \text{C面($-X$方向):}
  & \displaystyle
    \left(
      g_{tx}+\mathcal L_0-\frac{w'_{\mathrm Aq}}2~,~
      g_{ty}~,~
      f_t'-q
    \right),\\[12pt]
  \text{B面($+Y$方向):}
  & \displaystyle
    \left(
      g_{tx}+\mathcal L_0~,~
      g_{ty}+\frac{w'_{\mathrm Bq}}2~,~
      f_t'-q
    \right),\\[12pt]
  \text{D面($-Y$方向):}
  & \displaystyle
    \left(
      g_{tx}+\mathcal L_0~,~
      g_{ty}-\frac{w'_{\mathrm Bq}}2~,~
      f_t'-q
    \right).
\end{array}
\end{align*}
ここで、$i$列目の湾曲中心とトップ端の湾曲中心との$X$座標の差を、
%% label{eq:}
\begin{align}
  \label{eq:dimpleCenterDistance}
  \mathcal L_i
  \equiv \sqrt{R_\mathrm c^2-\left\{f_\mathrm T-q-(i-1)p_z\right\}^2}-\sqrt{R_\mathrm c^2-f_\mathrm T^2}
\end{align}
と表した。
なお、$i$列目の湾曲中心と$j$列目の湾曲中心との$X$座標の差を
\begin{align*}
  \mathcal L_{i,j}
  \equiv \mathcal L_i-\mathcal L_j
  = \sqrt{R_\mathrm c^2-\left(f_\mathrm T-q-(i-1)p_z\right)^2}
    -\sqrt{R_\mathrm c^2-\left\{f_\mathrm T-q-(j-1)p_z\right\}^2}
\end{align*}
と表すことにする。



%%%%%%%%%%%%%%%%%%%%%%%%%%%%%%%%%%%%%%%%%%%%%%%%%%%%%%%%%%
%% subsection 5.3.1 %%%%%%%%%%%%%%%%%%%%%%%%%%%%%%%%%%%%%%
%%%%%%%%%%%%%%%%%%%%%%%%%%%%%%%%%%%%%%%%%%%%%%%%%%%%%%%%%%
\subsection{内面溝の\texorpdfstring{$X$}{X}座標(傾き前)}
テーブル中心Pを原点としたとき、傾き前の$i$列目$j$番目の内面溝の$X$座標は、
%% label{eq:dPosXBefore}
\begin{align}
  \notag
  \text{A面:}\quad
  \mathcal D_{xi,\mathrm A}
  &= g_{tx}+\mathcal L_i+\frac{w'_{\mathrm Aq+(i-1)p_z}}2\\
  \label{eq:dPosXBefore}
  \text{C面:}\quad
  \mathcal D_{xi,\mathrm C}
  &= g_{tx}+\mathcal L_i-\frac{w'_{\mathrm Aq+(i-1)p_z}}2\\
  \notag
  \text{B, D面:}\quad
  \mathcal D_{xij,\mathrm B}
  &= g_{tx}+\mathcal L_i+\frac{d_i}2-(j-1)p_x
\end{align}
なお、A・C面については$j$に依らないことがわかる。
そのため、たとえば$\mathcal D_{xij,\mathrm A}$ではなく、$\mathcal D_{xi,\mathrm A}$のように表記している。



%%%%%%%%%%%%%%%%%%%%%%%%%%%%%%%%%%%%%%%%%%%%%%%%%%%%%%%%%%
%% subsection 5.3.2 %%%%%%%%%%%%%%%%%%%%%%%%%%%%%%%%%%%%%%
%%%%%%%%%%%%%%%%%%%%%%%%%%%%%%%%%%%%%%%%%%%%%%%%%%%%%%%%%%
\subsection{内面溝の\texorpdfstring{$Y$}{Y}座標(傾き前)}
テーブル中心Pを原点としたとき、傾き前の$i$列目$j$番目の内面溝の$Y$座標は、
%% label{eq:dPosYBefore}
\begin{alignat}{3}
  \notag
  \text{A, C面:}\quad
  && \mathcal D_{yij,\mathrm A} &= g_{ty}-\frac{d_i}2+(j-1)p_x\\
  \label{eq:dPosYBefore}
  \text{B面:}\quad
  && \mathcal D_{yi,\mathrm B} &= g_{ty}+\frac{w'_{\mathrm Bq+(i-1)p_z}}2\\
  \notag
  \text{D面:}\quad
  && \mathcal D_{yi,\mathrm D} &= g_{ty}-\frac{w'_{\mathrm Bq+(i-1)p_z}}2
\end{alignat}
B・D面については$j$に依らないことがわかる。



%%%%%%%%%%%%%%%%%%%%%%%%%%%%%%%%%%%%%%%%%%%%%%%%%%%%%%%%%%
%% subsection 5.3.3 %%%%%%%%%%%%%%%%%%%%%%%%%%%%%%%%%%%%%%
%%%%%%%%%%%%%%%%%%%%%%%%%%%%%%%%%%%%%%%%%%%%%%%%%%%%%%%%%%
\subsection{内面溝の\texorpdfstring{$Z$}{Z}座標(傾き前)}
テーブル中心Pを原点としたとき、傾き前の$i$列目$j$番目の内面溝の$Z$座標は、
%% label{eq:dPosZBefore}
\begin{align}
  \label{eq:dPosZBefore}
  \text{A, B, C, D面:}\quad
  \mathcal D_{zi} = f_t'-q-(i-1)p_z
\end{align}
$Z$座標についてはどの面も$j$に依らないことがわかる。



%%%%%%%%%%%%%%%%%%%%%%%%%%%%%%%%%%%%%%%%%%%%%%%%%%%%%%%%%%
%% subsection 5.3.4 %%%%%%%%%%%%%%%%%%%%%%%%%%%%%%%%%%%%%%
%%%%%%%%%%%%%%%%%%%%%%%%%%%%%%%%%%%%%%%%%%%%%%%%%%%%%%%%%%
\subsection{傾き角}
A側内面トップ端と、A側内面のトップ端から$q$の位置との$x$方向の差は、
\begin{align*}
  \sqrt{\left(R_\mathrm c+\frac{w'_{\mathrm Aq}}2\right)^{\!\!2}-(f_\mathrm T-q)^2}
  -\sqrt{\left(R_\mathrm c+\frac{w'_{\mathrm A0}}2\right)^{\!\!2}-f_\mathrm T^2}
\end{align*}
で与えられる。
このとき、これが負になる場合は傾ける必要はなく、正となる場合のみ傾ける。
したがってその傾き角$\phi$は、
%% label{eq:dKatamuki}
\begin{subequations}
\label{eq:dKatamuki}
\begin{alignat}{2}
  \text{正の場合:}&&\quad
  \tan\phi
  &= \frac{\displaystyle
           \sqrt{\left(R_\mathrm c+\frac{w'_{\mathrm Aq}}2\right)^{\!\!2}-(f_\mathrm T-q)^2}
           -\sqrt{\left(R_\mathrm c+\frac{w'_{\mathrm A0}}2\right)^{\!\!2}-f_\mathrm T^2}}q\\[8pt]
  \text{負の場合:}&&
  \phi
  &= 0
\end{alignat}
\end{subequations}
で与えられる。
%%%%%%%%%%%%%%%%%%%%%%%%%%%%%%%%%%%%%%%%%%%%%%%%%%%%%%%%%%
%% hosoku %%%%%%%%%%%%%%%%%%%%%%%%%%%%%%%%%%%%%%%%%%%%%%%%
%%%%%%%%%%%%%%%%%%%%%%%%%%%%%%%%%%%%%%%%%%%%%%%%%%%%%%%%%%
\begin{hosoku}
なお、これが負になるのは、
\begin{align*}
  & \left(R_\mathrm c+\frac{w'_{\mathrm Aq}}2\right)^{\!\!2}-(f_\mathrm T-q)^2
    < \left(R_\mathrm c+\frac{w'_{\mathrm A0}}2\right)^{\!\!2}-f_\mathrm T^2\\
  \longrightarrow~~
  & \frac{w_{\mathrm A0}-w_{\mathrm Aq}}2
    \left(2R_\mathrm c+\frac{w_{\mathrm A0}'+w_{\mathrm Aq}'}2\right)
    > q(2f_\mathrm T-q)
\end{align*}
である。
したがって、以下のような場合に生じる傾向があることがわかる。
\begin{enumerate}
\item 曲率が小さい(湾曲$R_\mathrm c$が大きい)
\item テーパがきつい($w_{\mathrm A0}-w_{\mathrm Aq}$が大きい)
\item 内径・めっき膜厚が大きい
\end{enumerate}
たとえば、曲率0 ($R = +\infty$)のモールド、つまり(外形が)まっすぐのモールドなどが、これに該当する。
\end{hosoku}
%%%%%%%%%%%%%%%%%%%%%%%%%%%%%%%%%%%%%%%%%%%%%%%%%%%%%%%%%%
%%%%%%%%%%%%%%%%%%%%%%%%%%%%%%%%%%%%%%%%%%%%%%%%%%%%%%%%%%
%%%%%%%%%%%%%%%%%%%%%%%%%%%%%%%%%%%%%%%%%%%%%%%%%%%%%%%%%%
%%%%%%%%%%%%%%%%%%%%%%%%%%%%%%%%%%%%%%%%%%%%%%%%%%%%%%%%%%
%% Column %%%%%%%%%%%%%%%%%%%%%%%%%%%%%%%%%%%%%%%%%%%%%%%%
%%%%%%%%%%%%%%%%%%%%%%%%%%%%%%%%%%%%%%%%%%%%%%%%%%%%%%%%%%
\begin{Column}{C側内面溝の傾き角}
C側内面溝については傾斜が外側に向いているため、傾けなくともアンダーカットの心配はない。
しかし、傾けたまま加工をすると形状が歪になってしまうため、内面溝の形状をよりよくするためには傾いていないほうが望ましい。
また一方で、面によって傾ける傾けないを分けると、プログラムが複雑になる(条件分岐が増える)要因にもなる。
そのためここでは、どの面の内面溝に対しても同じ角度$\phi$を用いて加工を行うことにする。
\tcbline*
なお、C面に対する内面溝の形状をできるだけよいものにするには、C面のテーパに基づいた角度を用いるほうが望ましい。
そのため、C側内面溝に対する傾き角$\phi_\mathrm C$についても(1つの例として)与えておく。
具体的には、以下の2点を基準として角度$\phi_\mathrm C$を取ることとする。
\begin{enumerate}
\item[a)]
C側内面の内面溝1列目(トップ端から$q$)の位置
\item[b)]
C側内面の内面溝$m$列目(トップ端から$q+(m-1)p_z$)の位置
\end{enumerate}
C側内面のトップ端から$q$の位置と、C側内面のトップ端から$q+(m-1)p_z$の位置との$x$方向の差は、
\begin{align*}
  \sqrt{\bigg(R_\mathrm c-\frac{w'_{\mathrm Aq+(m-1)p_z}}2\bigg)^{\!\!2}-\left\{f_\mathrm T-q-(m-1)p_z\right\}^2}
  -\sqrt{\left(R_\mathrm c-\frac{w'_{\mathrm Aq}}2\right)^{\!\!2}-(f_\mathrm T-q)^2}
\end{align*}
これより、C側内面溝に対する傾き角$\phi_\mathrm C$ ($\phi_\mathrm C > 0$)は、
\begin{align*}
  \tan\phi_\mathrm C
  = \frac{\sqrt{\left(R_\mathrm c-\frac{w'_{\mathrm Aq+(m-1)p_z}}2\right)^{\!2}
                -\left\{f_\mathrm T-q-(m-1)p_z\right\}^2}
          -\sqrt{\left(R_\mathrm c-\frac{w'_{\mathrm Aq}}2\right)^{\!2}-(f_\mathrm T-q)^2}}
         {(m-1)p_z}
\end{align*}
で与えられる。
なお、前述の通り$w_{\mathrm Aq+(m-1)p_z}$は$\lambda_j \leqq q+(m-1)p_z < \lambda_{j+1}$に対する$w_{\mathrm Aj}$, $w_{\mathrm Aj+1}$の加重算術平均
\begin{align*}
  w_{\mathrm Aq+(m-1)p_z}
  = \frac{\{q+(m-1)p_z-\lambda_j\}w_{\mathrm Aj+1}+\{\lambda_{j+1}-q-(m-1)p_z\}w_{\mathrm Aj}}
         {\lambda_{j+1}-\lambda_j}
\end{align*}
であり、内径として代用している。($w_{\mathrm Bq+(m-1)p_z}$についても同様)
\end{Column}
%%%%%%%%%%%%%%%%%%%%%%%%%%%%%%%%%%%%%%%%%%%%%%%%%%%%%%%%%%
%%%%%%%%%%%%%%%%%%%%%%%%%%%%%%%%%%%%%%%%%%%%%%%%%%%%%%%%%%
%%%%%%%%%%%%%%%%%%%%%%%%%%%%%%%%%%%%%%%%%%%%%%%%%%%%%%%%%%




%%%%%%%%%%%%%%%%%%%%%%%%%%%%%%%%%%%%%%%%%%%%%%%%%%%%%%%%%%
%% subsection 5.3.5 %%%%%%%%%%%%%%%%%%%%%%%%%%%%%%%%%%%%%%
%%%%%%%%%%%%%%%%%%%%%%%%%%%%%%%%%%%%%%%%%%%%%%%%%%%%%%%%%%
\subsection{B, D面の内面溝の位置(傾き前)}
B, D側内面溝において、その$X$座標がA側内面に最も近いものは、$m-1$列目または$m$列目の1番目の内面溝である。
これらの$X$座標は\pageeqref{eq:dPosXBefore}よりそれぞれ、
\begin{align*}
  m-1\text{列目:}&\quad
  g_{tx}+\mathcal L_{m-1}+\frac{d_{m-1}}2\\
  m\text{列目:}&\quad
  g_{tx}+\mathcal L_m+\frac{d_m}2
\end{align*}
%%%%%%%%%%%%%%%%%%%%%%%%%%%%%%%%%%%%%%%%%%%%%%%%%%%%%%%%%%
%% hosoku %%%%%%%%%%%%%%%%%%%%%%%%%%%%%%%%%%%%%%%%%%%%%%%%
%%%%%%%%%%%%%%%%%%%%%%%%%%%%%%%%%%%%%%%%%%%%%%%%%%%%%%%%%%
\begin{hosoku}
$d_{m-1} > d_m$のときは$m-1$列目, $d_m > d_{m-1}$のときは$m$列目をみればよい。
\end{hosoku}
%%%%%%%%%%%%%%%%%%%%%%%%%%%%%%%%%%%%%%%%%%%%%%%%%%%%%%%%%%
%%%%%%%%%%%%%%%%%%%%%%%%%%%%%%%%%%%%%%%%%%%%%%%%%%%%%%%%%%
%%%%%%%%%%%%%%%%%%%%%%%%%%%%%%%%%%%%%%%%%%%%%%%%%%%%%%%%%%
A側内面のトップ端からの($X$方向の)距離は、トップ端のAC側内径として$w'_{\mathrm A0}$を代用すると、それぞれ
\begin{align*}
  m-1\text{列目:}&\quad
  \frac{w'_{\mathrm A0}}2-\mathcal L_{m-1}-\frac{d_{m-1}}2\\
  m\text{列目:}&\quad
  \frac{w'_{\mathrm A0}}2-\mathcal L_m-\frac{d_m}2
\end{align*}
これらのいずれか小さいほうが工具径(半径)よりも小さければ、モールドを傾けて加工をする必要があると判断できる
%% footnote %%%%%%%%%%%%%%%%%%%%%
\footnote{もちろん、いくらか余裕代をとる必要がある。}。
%%%%%%%%%%%%%%%%%%%%%%%%%%%%%%%%%
%%%%%%%%%%%%%%%%%%%%%%%%%%%%%%%%%%%%%%%%%%%%%%%%%%%%%%%%%%
%% Column %%%%%%%%%%%%%%%%%%%%%%%%%%%%%%%%%%%%%%%%%%%%%%%%
%%%%%%%%%%%%%%%%%%%%%%%%%%%%%%%%%%%%%%%%%%%%%%%%%%%%%%%%%%
\begin{Column}{B, D側内面溝加工で考慮すべき点}
\paragraph*{工具径とシャンク径}
アンダーカットが生じるのは主に(A側内面の)トップ端なので、実際には工具径(工具の切削する部分)ではなくシャンク径等(工具のトップ端に相当する箇所)でよい。
そのため工具径よりシャンク径のほうが小さい場合は、より広い範囲の(B, D面の)内面溝をモールドを傾けずに切削することが可能となる。
\tcbline*
\paragraph*{端面の削り代}
内面溝の測定・加工は、モールドの端面を切削する前に行う。
そのため測定・加工の際は、モールドは端面の削り代の分だけ大きい(長い)ことに注意する必要がある。
削り代の分だけ湾曲も加味する必要があり、特にA側内面と工具とのアンダーカットに留意しなければならない。
\tcbline*
\paragraph*{その他のずれ}
モールドの形状は当然ながら図面のものとは一致はしない。
特に湾曲や肉厚などの図面とのずれは、アンダーカットに大きく寄与するのでこれも注意する必要がある。
\end{Column}
%%%%%%%%%%%%%%%%%%%%%%%%%%%%%%%%%%%%%%%%%%%%%%%%%%%%%%%%%%
%%%%%%%%%%%%%%%%%%%%%%%%%%%%%%%%%%%%%%%%%%%%%%%%%%%%%%%%%%
%%%%%%%%%%%%%%%%%%%%%%%%%%%%%%%%%%%%%%%%%%%%%%%%%%%%%%%%%%





\clearpage
%%%%%%%%%%%%%%%%%%%%%%%%%%%%%%%%%%%%%%%%%%%%%%%%%%%%%%%%%%
%% section 5.4 %%%%%%%%%%%%%%%%%%%%%%%%%%%%%%%%%%%%%%%%%%%
%%%%%%%%%%%%%%%%%%%%%%%%%%%%%%%%%%%%%%%%%%%%%%%%%%%%%%%%%%
\modHeadsection{傾き後の内面溝}
機内での回転はテーブル中心Pを原点として行われる。
また内面溝の加工はトップ端における内径中心を基準にして切削を行う。
傾ける前のトップ端内径中心$g_t$の座標は実測により(Pを中心とした$XYZ$直交座標でいうところの)[$g_{tx}$, $g_{ty}$, $f_t'$]で与えられる
%% footnote %%%%%%%%%%%%%%%%%%%%%
\footnote{ここではこれをテーブル中心Pを原点とした座標値として取り扱っている。
しかし、計測では機械座標系の値として$g_t$が与えられる。
たとえば\DMname の場合、$g_t$は通常(今の場合はテーブル中心Pより負側に湾曲中心があることが多いので)負の値として得られることに注意。
(ここでの計測では$XY$成分のみであり、$Z$については計測しないことにも注意。)}。
%%%%%%%%%%%%%%%%%%%%%%%%%%%%%%%%%
このとき、テーブルを角度$-\phi$だけ傾けた後のトップ端内面中心の座標$g'_t$は
%% footnote %%%%%%%%%%%%%%%%%%%%%
\footnote{これらをワーク座標原点としてもよいし、ワーク座標原点$g_t$はそのままで各面ごとに傾けてもよい。
ここでは後者の方法で加工を行うものとする。}、
%%%%%%%%%%%%%%%%%%%%%%%%%%%%%%%%%
%% label{eq:afterPhiTCenterFromO}
\begin{align}
  \label{eq:afterPhiTCenterFromO}
  \left[
  \begin{array}{c}
    g_{tx}'\\
    g_{ty}'\\
    g_{tz}'
  \end{array}
  \right]
  =\left[
   \begin{array}{c}
     g_{tx}\cos\phi+f_t'\sin\phi\\
     g_{ty}\\
     -g_{tx}\sin\phi+f_t'\cos\phi
   \end{array}
   \right].
   \end{align}
同様に、$i$列目における(傾ける前の)湾曲中心の位置は、[$g_{tx}+\mathcal L_i$, $g_{ty}$, $f_t'-q-(i-1)p_z$]で与えられる
%% footnote %%%%%%%%%%%%%%%%%%%%%
\footnote{ここではトップ端における湾曲中心を、トップ端における内面中心と同一視している。}
%%%%%%%%%%%%%%%%%%%%%%%%%%%%%%%%%
ので、テーブルを角度$-\phi$だけ傾けた後の$i$列目における湾曲中心の位置は、
\begin{align*}
  \left[
  \begin{array}{c}
    (g_{tx}+\mathcal L_i)\cos\phi+\{f_t'-q-(i-1)p_z\}\sin\phi\\
    g_{ty}\\
    -(g_{tx}+\mathcal L_i)\sin\phi+\{f_t'-q-(i-1)p_z\}\cos\phi
  \end{array}
  \right].
\end{align*}
したがって、傾けた後のトップ端の湾曲中心と$i$列目に対する湾曲中心との差分は、
%% label{eq:afterPhidimpleCenterDistance}
\begin{align}
  \label{eq:afterPhidimpleCenterDistance}
  \left[
  \begin{array}{c}
    \mathcal L_i\cos\phi-\{q+(i-1)p_z\}\sin\phi\\
    0\\
    -\mathcal L_i\sin\phi-\{q+(i-1)p_z\}\cos\phi
  \end{array}
  \right].
\end{align}
%%%%%%%%%%%%%%%%%%%%%%%%%%%%%%%%%%%%%%%%%%%%%%%%%%%%%%%%%%
%% hosoku %%%%%%%%%%%%%%%%%%%%%%%%%%%%%%%%%%%%%%%%%%%%%%%%
%%%%%%%%%%%%%%%%%%%%%%%%%%%%%%%%%%%%%%%%%%%%%%%%%%%%%%%%%%
\begin{hosoku}
傾けた後の$i$列目に対する湾曲中心と$j$列目に対する湾曲中心との差分は、
\begin{align*}
  \left[
  \begin{array}{c}
    \mathcal L_{j,i}\cos\phi-(j-i)p_z\sin\phi\\
    0\\
    -\mathcal L_{j,i}\sin\phi-(j-i)p_z\cos\phi
  \end{array}
  \right].
\end{align*}
特に、$j = i+1$の場合は、
\begin{align*}
  \left[
  \begin{array}{c}
    \mathcal L_{i+1,i}\cos\phi-p_z\sin\phi\\
    0\\
    -\mathcal L_{i+1,i}\sin\phi-p_z\cos\phi
  \end{array}
  \right].
\end{align*}
\end{hosoku}
%%%%%%%%%%%%%%%%%%%%%%%%%%%%%%%%%%%%%%%%%%%%%%%%%%%%%%%%%%
%%%%%%%%%%%%%%%%%%%%%%%%%%%%%%%%%%%%%%%%%%%%%%%%%%%%%%%%%%
%%%%%%%%%%%%%%%%%%%%%%%%%%%%%%%%%%%%%%%%%%%%%%%%%%%%%%%%%%
%%%%%%%%%%%%%%%%%%%%%%%%%%%%%%%%%%%%%%%%%%%%%%%%%%%%%%%%%%
%% Column %%%%%%%%%%%%%%%%%%%%%%%%%%%%%%%%%%%%%%%%%%%%%%%%
%%%%%%%%%%%%%%%%%%%%%%%%%%%%%%%%%%%%%%%%%%%%%%%%%%%%%%%%%%
\begin{Column}{プローブ径の考慮:$XY$と$Z$方向の非対称性}
マシニング内の計測ではタッチセンサーを用いる。
そのため、プローブ径の大きさに対して考慮・補正しなければならない。
プローブの位置の基準については、以下のようにとるのが通常である。
\begin{enumerate}
\item $X$方向:基準はプローブの($X$方向の)中心
\item $Y$方向:基準はプローブの($Y$方向の)中心
\item $Z$方向:基準はプローブの($Z$方向の)先端
\end{enumerate}
したがって、$XY$方向と$Z$方向とでは基準点が異なり非対称となっている。
今の場合、基準が非対称な$X$と$Z$が混合する移動(回転)であるが、あくまでもプローブの先端(上記の基準点)が回転後の位置にある、ということである。
そのため補正については(傾きに関係なく)$Z$方向に対してのみ径の半分だけ補正すればよい。
\end{Column}
%%%%%%%%%%%%%%%%%%%%%%%%%%%%%%%%%%%%%%%%%%%%%%%%%%%%%%%%%%
%%%%%%%%%%%%%%%%%%%%%%%%%%%%%%%%%%%%%%%%%%%%%%%%%%%%%%%%%%
%%%%%%%%%%%%%%%%%%%%%%%%%%%%%%%%%%%%%%%%%%%%%%%%%%%%%%%%%%




%%%%%%%%%%%%%%%%%%%%%%%%%%%%%%%%%%%%%%%%%%%%%%%%%%%%%%%%%%
%% subsection 5.4.1 %%%%%%%%%%%%%%%%%%%%%%%%%%%%%%%%%%%%%%
%%%%%%%%%%%%%%%%%%%%%%%%%%%%%%%%%%%%%%%%%%%%%%%%%%%%%%%%%%
\subsection{傾き後の内面溝(A, C面側)}
傾ける角度$\phi$は\pageeqref{eq:dKatamuki}で与えられる。
このとき、傾けた後のAおよびC面側に対する$i$列目$j$番目の内面溝の位置は、\pageeqref{eq:dPosXBefore}, \eqref{eq:dPosYBefore}, \pageeqref{eq:dPosZBefore}より、
\begin{alignat*}{3}
  \text{A面:}&~~&
  \left[
  \begin{array}{c}
    \mathcal D_{xij,\mathrm A}'\\
    \mathcal D_{yij,\mathrm A}'\\
    \mathcal D_{zij,\mathrm A}'
  \end{array}
  \right]
 &= \left[
    \begin{array}{c}
      \mathcal D_{xi,\mathrm A}\cos\phi+\mathcal D_{zi}\sin\phi\\
      \mathcal D_{yij,\mathrm A}\\
      -\mathcal D_{xi,\mathrm A}\sin\phi+\mathcal D_{zi}\cos\phi
    \end{array}
    \right],\\[2pt]
  \text{C面:}&~~&
  \left[
  \begin{array}{c}
    \mathcal D_{xij,\mathrm C}'\\
    \mathcal D_{yij,\mathrm C}'\\
    \mathcal D_{zij,\mathrm C}'
  \end{array}
  \right]
 &= \left[
    \begin{array}{c}
      \mathcal D_{xi,\mathrm C}\cos\phi+\mathcal D_{zi}\sin\phi\\
      \mathcal D_{yij,\mathrm A}\\
      -\mathcal D_{xi,\mathrm C}\sin\phi+\mathcal D_{zi}\cos\phi
    \end{array}
    \right].
\end{alignat*}
特に、各列の中央(各列の湾曲中心)$[g_{tx}+\mathcal L_i, g_{ty}, f_t'-q-(i-1)p_z]$を原点としてみた場合の位置は、
\begin{align*}
  \left[
  \begin{array}{c}
    \displaystyle \pm\frac{w_{Aq+(i-1)p_z}'}2\cos\phi\\[6pt]
    \displaystyle -\frac{d_i}2+(j-1)p_x\\[6pt]
    \displaystyle \mp\frac{w_{Aq+(i-1)p_z}'}2\sin\phi
  \end{array}
  \right]\qquad
  %%%%%%%%
  \left(
  \text{複号}
  \left\{
  \begin{array}{rl}
    \!\text{上}\!\!\!& \text{: A面}\\
    \!\text{下}\!\!\!& \text{: C面}\\
  \end{array}
  \right.
  \right).
\end{align*}





\paragraph*{$j$方向の差分}\noindent
$Y$方向の隣同士の差分、すなわち$i$を固定したときの$j$番目と$j+1$番目の位置の差分は、
\begin{align*}
  \left[
  \begin{array}{c}
    0\\
    \mathcal D_{yi(j+1),\mathrm A}-\mathcal D_{yij,\mathrm A}\\
    0
  \end{array}
  \right]
  = \left[
    \begin{array}{c}
      0\\
      p_x\\
      0
    \end{array}
    \right]\ .
\end{align*}


\paragraph*{$i$方向の差分}\noindent
$Z$方向の隣同士の差分、すなわち$j$を固定したときの$i$番目と$i+1$番目の位置の差分については、
\begin{align*}
 &\left[
  \begin{array}{c}
    (\mathcal D_{x(i+1),\mathrm A}-\mathcal D_{xi,\mathrm A})\cos\phi
    +(\mathcal D_{z(i+1)}-\mathcal D_{zi})\sin\phi\\
    (\mathcal D_{y(i+1)j,\mathrm A}-\mathcal D_{yij,\mathrm A})\\
    (\mathcal D_{xi,\mathrm A}-\mathcal D_{x(i+1),\mathrm A})\sin\phi
    +(\mathcal D_{z(i+1)}-\mathcal D_{zi})\cos\phi
  \end{array}
  \right]\\
 &= \left[
    \begin{array}{c}
      \displaystyle
      \left(\mathcal L_{i+1, i}+\frac{w'_{\mathrm Aq+ip_z}-w'_{\mathrm Aq+(i-1)p_z}}2\right)\!\cos\phi
      -p_z\sin\phi\\[6pt]
      \displaystyle-\frac{d_{i+1}-d_i}2\\[6pt]
      \displaystyle
      -\left(\mathcal L_{i+1, i}+\frac{w'_{\mathrm Aq+ip_z}-w'_{\mathrm Aq+(i-1)p_z}}2\right)\!\sin\phi
      -p_z\cos\phi
    \end{array}
    \right]\ .
\end{align*}
C面に対しては、これの各々の内径$w_\mathrm A'$の符号を入れ換えたものとなる。
%%%%%%%%%%%%%%%%%%%%%%%%%%%%%%%%%%%%%%%%%%%%%%%%%%%%%%%%%%
%% hosoku %%%%%%%%%%%%%%%%%%%%%%%%%%%%%%%%%%%%%%%%%%%%%%%%
%%%%%%%%%%%%%%%%%%%%%%%%%%%%%%%%%%%%%%%%%%%%%%%%%%%%%%%%%%
\begin{hosoku}
$X$成分の差分の大きさが($\mathcal L_{i+1, i}$からみて)A面(の$\cos\phi$成分)のそれと同じであることがわかる。
これは(水平方向の)内径を$w_{\mathrm A\lambda}$等で代用したからであり、実際の長さは異なる(振分中心を除いて対称ではなく、C側のほうが長い)ことに注意。
\end{hosoku}
%%%%%%%%%%%%%%%%%%%%%%%%%%%%%%%%%%%%%%%%%%%%%%%%%%%%%%%%%%
%%%%%%%%%%%%%%%%%%%%%%%%%%%%%%%%%%%%%%%%%%%%%%%%%%%%%%%%%%
%%%%%%%%%%%%%%%%%%%%%%%%%%%%%%%%%%%%%%%%%%%%%%%%%%%%%%%%%%
%%%%%%%%%%%%%%%%%%%%%%%%%%%%%%%%%%%%%%%%%%%%%%%%%%%%%%%%%%
%% hosoku %%%%%%%%%%%%%%%%%%%%%%%%%%%%%%%%%%%%%%%%%%%%%%%%
%%%%%%%%%%%%%%%%%%%%%%%%%%%%%%%%%%%%%%%%%%%%%%%%%%%%%%%%%%
\begin{hosoku}[label=hosoku:generallyDimpleN]
\pageautoref{fn:generallyDimpleN}でも述べたように、たいていの場合は$|d_{i+1}-d_i|=p_x$であり、また$d_{i+2} = d_i$である。
\end{hosoku}
%%%%%%%%%%%%%%%%%%%%%%%%%%%%%%%%%%%%%%%%%%%%%%%%%%%%%%%%%%
%%%%%%%%%%%%%%%%%%%%%%%%%%%%%%%%%%%%%%%%%%%%%%%%%%%%%%%%%%
%%%%%%%%%%%%%%%%%%%%%%%%%%%%%%%%%%%%%%%%%%%%%%%%%%%%%%%%%%




%%%%%%%%%%%%%%%%%%%%%%%%%%%%%%%%%%%%%%%%%%%%%%%%%%%%%%%%%%
%% subsection 5.4.2 %%%%%%%%%%%%%%%%%%%%%%%%%%%%%%%%%%%%%%
%%%%%%%%%%%%%%%%%%%%%%%%%%%%%%%%%%%%%%%%%%%%%%%%%%%%%%%%%%
\subsection{傾き後の内面溝(B, D面側)}
傾けた後のBおよびD面側に対する$i$列目$j$番目の内面溝の位置は、A面側のときと同様に、
\begin{alignat*}{3}
  \text{B面:}&~~&
  \left[
    \begin{array}{c}
      \mathcal D_{xij,\mathrm B}'\\
      \mathcal D_{yij,\mathrm B}'\\
      \mathcal D_{zij,\mathrm B}'
    \end{array}
  \right]
 &= \left[
    \begin{array}{c}
      \mathcal D_{xij,\mathrm B}\cos\phi+\mathcal D_{zi}\sin\phi\\
      \mathcal D_{yi,\mathrm B}\\
      -\mathcal D_{xij,\mathrm B}\sin\phi+\mathcal D_{zi}\cos\phi
    \end{array}
    \right],\\[2pt]
  \text{D面:}&~~&
  \left[
    \begin{array}{c}
      \mathcal D_{xij,\mathrm D}'\\
      \mathcal D_{yij,\mathrm D}'\\
      \mathcal D_{zij,\mathrm D}'
    \end{array}
  \right]
 &= \left[
    \begin{array}{c}
      \mathcal D_{xij,\mathrm B}\cos\phi+\mathcal D_{zi}\sin\phi\\
      \mathcal D_{yi,\mathrm D}\\
      -\mathcal D_{xij,\mathrm B}\sin\phi+\mathcal D_{zi}\cos\phi
    \end{array}
    \right].
\end{alignat*}
特に、各列の中央(各列の湾曲中心)$[g_{tx}+\mathcal L_i, g_{ty}, f_t'-q-(i-1)p_z]$を原点としてみた場合の位置は、
\begin{align*}
  \left[
  \begin{array}{c}
    \displaystyle \left\{\frac{d_i}2-(j-1)p_z\right\}\cos\phi\\
    \displaystyle \pm\frac{w_{Bq+(i-1)p_z}'}2\\
    \displaystyle -\left\{\frac{d_i}2-(j-1)p_z\right\}\sin\phi
  \end{array}
  \right]\qquad
  %%%%%%%%
  \left(
  \text{複号}
  \left\{
  \begin{array}{rl}
    \!+\!\!\!& \text{: B面}\\
    \!-\!\!\!& \text{: D面}\\
  \end{array}
  \right.
  \right).
\end{align*}



\paragraph*{$j$方向の差分}\noindent
$Y$方向の隣同士の差分、すなわち$i$を固定したときの$j$番目と$j+1$番目の位置の差分は、
\begin{align*}
  \left[
  \begin{array}{c}
    \left(\mathcal D_{xi(j+1),\mathrm B}-\mathcal D_{xij,\mathrm B}\right)\cos\phi\\
    0\\
    -\left(\mathcal D_{xi(j+1),\mathrm B}-\mathcal D_{xij,\mathrm B}\right)\sin\phi
  \end{array}
  \right]
  = \left[
    \begin{array}{c}
      -p_x\cos\phi\\[6pt]
      0\\
      p_x\sin\phi
    \end{array}
    \right]\ .
\end{align*}


\paragraph*{$i$方向の差分}\noindent
B面に対する$Z$方向の隣同士の差分、すなわち$j$を固定したときの$i$番目と$i+1$番目の位置の差分については、
\begin{align*}
 &\left[
  \begin{array}{c}
    \left(\mathcal D_{x(i+1)j,\mathrm B}-\mathcal D_{xij,\mathrm B}\right)\cos\phi
    +\left(\mathcal D_{z(i+1)}-\mathcal D_{zi}\right)\sin\phi\\[3pt]
    \mathcal D_{yi+1,\mathrm B}-\mathcal D_{yi,\mathrm B}\\[3pt]
    -\left(\mathcal D_{x(i+1)j,\mathrm B}-\mathcal D_{xij,\mathrm B}\right)\sin\phi
    +\left(\mathcal D_{z(i+1)}-\mathcal D_{zi}\right)\cos\phi
  \end{array}
  \right]\\
 &= \left[
    \begin{array}{c}
      \displaystyle\left(\mathcal L_{i+1, i}+\frac{d_{i+1}-d_i}2\right)\!\cos\phi-p_z\sin\phi\\[10pt]
      \displaystyle\frac{w'_{\mathrm Bq+ip_z}-w'_{\mathrm Bq+(i-1)p_z}}2\\[8pt]
      \displaystyle-\left(\mathcal L_{i+1, i}+\frac{d_{i+1}-d_i}2\right)\!\cos\phi-p_z\cos\phi
    \end{array}
    \right]\ .
\end{align*}
D面に対しては、これの各々の内径$w_\mathrm B'$の符号(この場合$Y$成分の符号)を入れ換えたものとなる。






%%%%%%%%%%%%%%%%%%%%%%%%%%%%%%%%%%%%%%%%%%%%%%%%%%%%%%%%%%
%%           %%%%%%%%%%%%%%%%%%%%%%%%%%%%%%%%%%%%%%%%%%%%%
%% chapter 7 %%%%%%%%%%%%%%%%%%%%%%%%%%%%%%%%%%%%%%%%%%%%%
%%           %%%%%%%%%%%%%%%%%%%%%%%%%%%%%%%%%%%%%%%%%%%%%
%%%%%%%%%%%%%%%%%%%%%%%%%%%%%%%%%%%%%%%%%%%%%%%%%%%%%%%%%%
\modHeadchapter[その他の\texorpdfstring{$B$}{B}軸回転に伴う幾何]
         {その他の$\textbf B$軸回転に伴う幾何}
マシニング機内では、$XYZ$直交座標軸の他に、それぞれの軸とした$ABC$回転軸がある。
たとえば、\MMname ではテーブルが$Y$軸まわりに回転する$B$回転軸座標がある。
さらに\DMname では、$B$回転軸に加えて、工具が$Z$軸まわりに回転する$C$回転軸座標も存在する。
これまでに扱った再振分けのためのテーブル回転や、内面溝のためのテーブル回転は、どちらも$B$軸の回転($Y$軸まわりの回転)である。
ここではさらに、その他の$B$回転軸を伴う事象を取り扱うことにする。



%%%%%%%%%%%%%%%%%%%%%%%%%%%%%%%%%%%%%%%%%%%%%%%%%%%%%%%%%%
%% section 7.1 %%%%%%%%%%%%%%%%%%%%%%%%%%%%%%%%%%%%%%%%%%%
%%%%%%%%%%%%%%%%%%%%%%%%%%%%%%%%%%%%%%%%%%%%%%%%%%%%%%%%%%
\modHeadsection{ジグと回転中心との補正}
ここでは回転の中心は原点にあるものとし、テーブルに乗っているジグの中心が原点から($\delta x$, $\delta z$)だけずれているものとする。
このとき、ある任意の点($x$, $z$)を$\theta$だけ回転すると、
\begin{align*}
  \left[
    \begin{array}{c}
      x'\\
      z'
    \end{array}
  \right]
  = \left[
    \begin{array}{cc}
      \cos\theta & -\sin\theta\\
      \sin\theta & \cos\theta
    \end{array}
  \right]\!\!
  \left[
    \begin{array}{c}
      x+\delta x\\
      z+\delta z
    \end{array}
  \right]
  = \left[
    \begin{array}{c}
      (x+\delta x)\cos\theta-(z+\delta z)\sin\theta\\
      (x+\delta x)\sin\theta+(z+\delta z)\cos\theta
    \end{array}
  \right].
\end{align*}
特に、$\theta = \nicefrac\pi2$の場合は、
\begin{align*}
  \left[
    \begin{array}{c}
      x'\\
      z'
    \end{array}
  \right]
  = \left[
    \begin{array}{c}
      -z-\delta z\\
      x+\delta x
    \end{array}
  \right].
\end{align*}
よって、ずれによる差分は、
\begin{align*}
  \left[
    \begin{array}{c}
      \delta x\cos\theta-\delta z\sin\theta\\
      \delta x\sin\theta+\delta z\cos\theta
    \end{array}
  \right]
\end{align*}
となる。




\clearpage
%%%%%%%%%%%%%%%%%%%%%%%%%%%%%%%%%%%%%%%%%%%%%%%%%%%%%%%%%%
%% section 7.2 %%%%%%%%%%%%%%%%%%%%%%%%%%%%%%%%%%%%%%%%%%%
%%%%%%%%%%%%%%%%%%%%%%%%%%%%%%%%%%%%%%%%%%%%%%%%%%%%%%%%%%
\modHeadsection{通り芯}
トップ・ボトムの両方に外削がある場合を考える。
通常、それぞれの外削の中心は個別に決められはせず、片方の中心の位置を基準として、もう片方の中心が定められる。
これらの中心の位置の差(通り芯
%% footnote %%%%%%%%%%%%%%%%%%%%%
\footnote{通常、通り芯(centerline)というのはその名の通り中心線を表すことが多い。
しかし、ここではトップ側外削中心とボトム側外削中心との位置の差を表す用語として「通り芯」と呼んでいる。})
%%%%%%%%%%%%%%%%%%%%%%%%%%%%%%%%%
$T_x$, $T_y$ ($T_x \geqq 0$)を機内で測定する際は、C面が工具側に向くようにテーブルを$\pm90^\circ$回転($B$軸回転)し、タッチセンサーを用いてそれぞれの外削部の$Z$座標および$Y$座標を見ることで測定する。

ここでは、この通り芯の測定に必要な位置等について定量的に求める。
なお、テーブルの中心Pを原点として考えることにする。
またモールドのC面が工具側に向くように$B$軸を(\verb|G91|にて)$\pm90^\circ$回転した状態であるとする
%% footnote %%%%%%%%%%%%%%%%%%%%%
\footnote{G90(絶対座標)の場合、テーブルを傾けて振分長を調整した場合はその回転角$-\theta$を忘れないよう注意。}。
%%%%%%%%%%%%%%%%%%%%%%%%%%%%%%%%%



%%%%%%%%%%%%%%%%%%%%%%%%%%%%%%%%%%%%%%%%%%%%%%%%%%%%%%%%%%
%% subsection 7.2.1 %%%%%%%%%%%%%%%%%%%%%%%%%%%%%%%%%%%%%%
%%%%%%%%%%%%%%%%%%%%%%%%%%%%%%%%%%%%%%%%%%%%%%%%%%%%%%%%%%
\subsection{ボトムの外削が基準の場合}
通常、トップの外削中心は、ボトムの外削中心よりA面側($-Z$側)にある。
このとき、溝位置$\kappa_p$およびボトム側の外削の長さ$h_\mathrm B$ ($h_\mathrm B > 0$)を用いると、ボトム側($-X$側)およびトップ側($+X$側)の外削C面の中心
%% footnote %%%%%%%%%%%%%%%%%%%%%
\footnote{トップ側には溝があるので、トップ側の外削の長さは溝位置$\kappa_p$とみなしている。}
%%%%%%%%%%%%%%%%%%%%%%%%%%%%%%%%%
は、それぞれ
%% footnote %%%%%%%%%%%%%%%%%%%%%
\footnote{通常、$Y$方向の通り芯は$T_y=0$である。}
%%%%%%%%%%%%%%%%%%%%%%%%%%%%%%%%%
\begin{align*}
  \text{ボトム側:}\quad
  \left[
    \begin{array}{c}
      \displaystyle -f_\mathrm B'+\frac{h_\mathrm B}2\\[5pt]
      \mathcal G_{\mathrm By}\\[3pt]
      \displaystyle \mathcal G_{\mathrm Bx}+\frac{\mathfrak W_\mathrm B}2
    \end{array}
    \right]~, \qquad
  \text{トップ側:}\quad
  \left[
    \begin{array}{c}
      \displaystyle f_\mathrm T'-\frac{\kappa_p}2\\[5pt]
      \mathcal G_{\mathrm By}-T_y\\[3pt]
      \displaystyle \mathcal G_{\mathrm Bx}-T_x+\frac{\mathfrak W_\mathrm T}2
    \end{array}
  \right].
\end{align*}



%%%%%%%%%%%%%%%%%%%%%%%%%%%%%%%%%%%%%%%%%%%%%%%%%%%%%%%%%%
%% subsection 7.2.2 %%%%%%%%%%%%%%%%%%%%%%%%%%%%%%%%%%%%%%
%%%%%%%%%%%%%%%%%%%%%%%%%%%%%%%%%%%%%%%%%%%%%%%%%%%%%%%%%%
\subsection{トップの外削が基準の場合}
通常、ボトムの外削中心は、トップの外削中心よりC面側($+Z$側)にある。
このとき、トップ側($+X$側)およびボトム側($-X$側)の外削C面の中心は、それぞれ
\begin{align*}
  \text{トップ側:}\quad
  \left[
    \begin{array}{c}
      \displaystyle f_\mathrm T'-\frac{\kappa_p}2\\[5pt]
      \mathcal G_{\mathrm Ty}\\[3pt]
      \displaystyle \mathcal G_{\mathrm Tx}+\frac{\mathfrak W_\mathrm B}2
    \end{array}
    \right]~, \qquad
  \text{ボトム側:}\quad
  \left[
    \begin{array}{c}
      \displaystyle -f_\mathrm B'+\frac{h_\mathrm B}2\\[5pt]
      \mathcal G_{\mathrm Ty}+T_y\\[3pt]
      \displaystyle \mathcal G_{\mathrm Tx}+T_x+\frac{\mathfrak W_\mathrm B}2
    \end{array}
  \right].
\end{align*}



%%%%%%%%%%%%%%%%%%%%%%%%%%%%%%%%%%%%%%%%%%%%%%%%%%%%%%%%%%
%% subsection 7.2.3 %%%%%%%%%%%%%%%%%%%%%%%%%%%%%%%%%%%%%%
%%%%%%%%%%%%%%%%%%%%%%%%%%%%%%%%%%%%%%%%%%%%%%%%%%%%%%%%%%
\subsection{外削前に測定をする場合}
外削をせずに通り芯($X$)の測定を行うことはない。
ただ、プログラムの試運転などで動きをみるといった可能性はありうるので、外削を行っていない状態で測定する場合についても述べておく。
なお、ここでは振分けの調整にスペーサ用いた場合、かつボトムの外削が基準の場合を考える。
このとき、測定するC側外面の位置は、\pageeqref{eq:tableTi}, \pageeqref{eq:tableBRi}より、
\begin{align*}
  \text{トップ側:}
  \left[
    \begin{array}{c}
      \displaystyle f_\mathrm T'-\frac{\kappa_p}2\\[5pt]
      \mathcal G_{\mathrm By}-T_y\\[3pt]
      \displaystyle \varDelta'\!\cos\theta-\sqrt{R_\mathrm i^2-\left(f_\mathrm T-\frac{\kappa_p}2\right)^{\!\!2}}
    \end{array}
  \right]~, \quad
  \text{ボトム側:}
  \left[
    \begin{array}{c}
      \displaystyle -f_\mathrm B'+\frac{h_\mathrm B}2\\[5pt]
      \mathcal G_{\mathrm By}\\[3pt]
      \displaystyle \varDelta'\!\cos\theta-\sqrt{R_\mathrm i^2-\left(f_\mathrm B-\frac{h_\mathrm B}2\right)^{\!\!2}}
    \end{array}
    \right].
\end{align*}
なお、これらの$Z$座標の差は以下で与えられる。
\begin{align*}
  \sqrt{R_\mathrm i^2-\left(f_\mathrm T-\frac{\kappa_p}2\right)^{\!\!2}}
  -\sqrt{R_\mathrm i^2-\left(f_\mathrm B-\frac{h_\mathrm B}2\right)^{\!\!2}}~.
\end{align*}




\clearpage
%%%%%%%%%%%%%%%%%%%%%%%%%%%%%%%%%%%%%%%%%%%%%%%%%%%%%%%%%%
%% section 7.3 %%%%%%%%%%%%%%%%%%%%%%%%%%%%%%%%%%%%%%%%%%%
%%%%%%%%%%%%%%%%%%%%%%%%%%%%%%%%%%%%%%%%%%%%%%%%%%%%%%%%%%
\modHeadsection{角度をつけて外削する場合の位置}
テーブルの中心Pを原点として考える。
また、ボトム端面側が工具側に向いているとする。
端面のA面側・C面側・中心の$X$座標をそれぞれ$x_A$, $x_C$, $x_m$とする。
このとき端面のそれぞれの位置は
%% footnote %%%%%%%%%%%%%%%%%%%%%
\footnote{ここでは$Z$の正方向を実軸、$X$の正方向を虚軸として考えている。}、
%%%%%%%%%%%%%%%%%%%%%%%%%%%%%%%%%
\begin{subequations}
\begin{align*}
  \text{中点:}&\quad \sqrt{z_b^2+x_m^2}e^{i\theta_m}, \quad \tan\theta_m = \frac{x_m}{z_b}\\
  \text{C面側端点:}&\quad \sqrt{z_b^2+x_C^2}e^{i\theta_C}, \quad \tan\theta_C = \frac{x_C}{z_b}\\
  \text{A面側端点:}&\quad \sqrt{z_b^2+x_A^2}e^{i\theta_A}, \quad \tan\theta_A = \frac{x_A}{z_b}.
\end{align*}
\end{subequations}
外削部分の高さを$h_\mathrm B$とすると、外柵部分の傾きはボトム端から距離$\nicefrac{h_\mathrm B}2$の断面に平行になる形にとるので、必要な回転角$\theta_b$は、
\begin{align*}
  \sin\theta_b = \frac{z_b-\nicefrac{h_\mathrm B}2}{R_\mathrm c}
\end{align*}
を満たす。
回転後のボトム端面の中心の位置は、
\begin{align*}
  \sqrt{z_b^2+x_m^2}e^{i(\theta_m-\theta_b)}
\end{align*}
となるので、これの虚部が$X$座標、実部が$Z$座標となる
%% footnote %%%%%%%%%%%%%%%%%%%%%
\footnote{ここでは複素数平面を考えているが、通常の直行座標系で単純に回転行列をかけているのと同義である。
\begin{align*}
  \left[
    \begin{array}{c}
      x'_m\\
      z'_b
    \end{array}
  \right]
  = \left[
    \begin{array}{cc}
      \cos\theta_b & -\sin\theta_b\\
      \sin\theta_b & \cos\theta_b
    \end{array}
  \right]\!\!
  \left[
    \begin{array}{c}
      x_m\\
      z_b
    \end{array}
  \right]
  = \left[
    \begin{array}{c}
      x_m\cos\theta_b-z_b\sin\theta_b\\
      x_m\sin\theta_b+z_b\cos\theta_b
    \end{array}
  \right].
\end{align*}%
}。
%%%%%%%%%%%%%%%%%%%%%%%%%%%%%%%%%
\begin{align*}
  \sqrt{z_b^2+x_m^2}\sin(\theta_m-\theta_b)
  &= \sqrt{z_b^2+x_m^2}(\sin\theta_m\cos\theta_b-\cos\theta_m\sin\theta_b)\\
  &= x_m\sqrt{1-\left(\frac{z_b-\nicefrac{h_\mathrm B}2}{R_\mathrm c}\right)^{\!2}}
     -z_b\cdot\frac{z_b-\nicefrac{h_\mathrm B}2}{R_\mathrm c}~,\\
  \sqrt{z_b^2+x_m^2}\cos(\theta_m-\theta_b)
  &= \sqrt{z_b^2+x_m^2}(\cos\theta_m\cos\theta_b+\sin\theta_m\sin\theta_b)\\
  &= z_b\sqrt{1-\left(\frac{z_b-\nicefrac{h_\mathrm B}2}{R_\mathrm c}\right)^{\!2}}
     +x_m\cdot\frac{z_b-\nicefrac{h_\mathrm B}2}{R_\mathrm c}~.
\end{align*}
端面のA面側・C面側の位置についても同様である。
まとめると、
\begin{subequations}
\begin{align*}
  \text{中点:}&\quad
    \left[
      \begin{array}{c}
        x'_m\\
        z'_b
      \end{array}
    \right]
    = \left[
      \begin{array}{c}
        \displaystyle
        x_m\sqrt{1-\left(\frac{z_b-\nicefrac{h_\mathrm B}2}{R_\mathrm c}\right)^{\!2}}
        -z_b\cdot\frac{z_b-\nicefrac{h_\mathrm B}2}{R_\mathrm c}\\[15pt]
        \displaystyle
        z_b\sqrt{1-\left(\frac{z_b-\nicefrac{h_\mathrm B}2}{R_\mathrm c}\right)^{\!2}}
        +x_m\cdot\frac{z_b-\nicefrac{h_\mathrm B}2}{R_\mathrm c}
      \end{array}
    \right],\\
  \text{C面側端点:}&\quad
    \left[
      \begin{array}{c}
        x'_C\\
        z'_b
      \end{array}
    \right]
    = \left[
      \begin{array}{c}
        \displaystyle
        x_C\sqrt{1-\left(\frac{z_b-\nicefrac{h_\mathrm B}2}{R_\mathrm c}\right)^{\!2}}
        -z_b\cdot\frac{z_b-\nicefrac{h_\mathrm B}2}{R_\mathrm c}\\[15pt]
        \displaystyle
        z_b\sqrt{1-\left(\frac{z_b-\nicefrac{h_\mathrm B}2}{R_\mathrm c}\right)^{\!2}}
        +x_C\cdot\frac{z_b-\nicefrac{h_\mathrm B}2}{R_\mathrm c}
      \end{array}
    \right],\\
  \text{A面側端点:}&\quad
    \left[
      \begin{array}{c}
        x'_A\\
        z'_b
      \end{array}
    \right]
    = \left[
      \begin{array}{c}
        \displaystyle
        x_A\sqrt{1-\left(\frac{z_b-\nicefrac{h_\mathrm B}2}{R_\mathrm c}\right)^{\!2}}
        -z_b\cdot\frac{z_b-\nicefrac{h_\mathrm B}2}{R_\mathrm c}\\[15pt]
        \displaystyle
        z_b\sqrt{1-\left(\frac{z_b-\nicefrac{h_\mathrm B}2}{R_\mathrm c}\right)^{\!2}}
        +x_A\cdot\frac{z_b-\nicefrac{h_\mathrm B}2}{R_\mathrm c}
      \end{array}
    \right].
\end{align*}
\end{subequations}




%%%%%%%%%%%%%%%%%%%%%%%%%%%%%%%%%%%%%%%%%%%%%%%%%%%%%%%%%%
%%            %%%%%%%%%%%%%%%%%%%%%%%%%%%%%%%%%%%%%%%%%%%%
%% Chapter 7  %%%%%%%%%%%%%%%%%%%%%%%%%%%%%%%%%%%%%%%%%%%%
%%            %%%%%%%%%%%%%%%%%%%%%%%%%%%%%%%%%%%%%%%%%%%%
%%%%%%%%%%%%%%%%%%%%%%%%%%%%%%%%%%%%%%%%%%%%%%%%%%%%%%%%%%
\modHeadchapter{内径}
\addcontentsline{lot}{chapter}{\thechapter. \Chaptername}
%!TEX root = ../Mould_Analytical_Calculation_Note.tex



%%%%%%%%%%%%%%%%%%%%%%%%%%%%%%%%%%%%%%%%%%%%%%%%%%%%%%%%%%
%% section 7.1 %%%%%%%%%%%%%%%%%%%%%%%%%%%%%%%%%%%%%%%%%%%
%%%%%%%%%%%%%%%%%%%%%%%%%%%%%%%%%%%%%%%%%%%%%%%%%%%%%%%%%%
\modHeadsection{eテーパの算定\TBW}
使用する鋼材の種類として、\index{C@C(炭素)}C, \index{Si@Si(ケイ素)}Si, \index{Mn@Mn(マンガン)}Mn, \index{P@P(リン)}P, \index{S@S(硫黄)}Sが含まれている場合を考える。
\index{JISきかく@JIS規格}JIS規格に基づいた鋼種を用いるものとすれば、その規格によってそれぞれの化学組成の含有量も決定される。
それぞれの化学組成の含有量($\mathrm{wt}\%$)を$X_\mathrm C$, $X_\mathrm{Si}$, $X_\mathrm{Mn}$, $X_\mathrm P$, $X_\mathrm S$とし、またその影響係数を$k_\mathrm C$, $k_\mathrm{Si}$, $k_\mathrm{Mn}$, $k_\mathrm P$, $k_\mathrm S$とする。
また、その鋼材の\index{えきそうせんおんど@液相線温度}液相線温度を$T_\mathrm l$[$^\circ\mathrm C$], \index{こそうせんおんど@固相線温度}固相線温度を$T_\mathrm s$[$^\circ\mathrm C$]とする。
一般にこれらの温度は、ある基準となる温度$T_0$に対して、
\begin{align*}
  T = T_0-\sum_i k_iX_i
\end{align*}
として与えられる。
今の場合だと、
\begin{align*}
  T_l
  &= 1536-78X_\mathrm C-7.6X_\mathrm{Si}-4.9X_\mathrm{Mn}-34.4X_\mathrm P-38X_\mathrm S~,\\
  T_s
  &= 1536-415.5X_\mathrm C-12.3X_\mathrm{Si}-6.8X_\mathrm{Mn}-124.5X_\mathrm P-183.9X_\mathrm S
\end{align*}
となることが知られている\cite{1986KO}。\\
\addtocontents{lot}{\protect\addvspace{3pt}}{}{}
\addcontentsline{lot}{section}{\numberline{\thesection}\Sectionname}
\modcaptionof{table}{化学組成の含有量}
\begin{tabular}[t]{|c|c|c|c|c|c|c|}
  \hline
  鋼材(wt\%) & C & Si & Mn & P & S
  \\\hline
  \index{JISコード}JISコード & 0.1 & 0.5 & 0.6 & 0.7 & 0.8
  \\\hline
\end{tabular}


\clearpage
%%%%%%%%%%%%%%%%%%%%%%%%%%%%%%%%%%%%%%%%%%%%%%%%%%%%%%%%%%
%% section 9.1 %%%%%%%%%%%%%%%%%%%%%%%%%%%%%%%%%%%%%%%%%%%
%%%%%%%%%%%%%%%%%%%%%%%%%%%%%%%%%%%%%%%%%%%%%%%%%%%%%%%%%%
\modHeadsection{内側湾曲の近似曲線\TBW}
(to be written...)



\begin{appendices}
%%%%%%%%%%%%%%%%%%%%%%%%%%%%%%%%%%%%%%%%%%%%%%%%%%%%%%%%%
%%           %%%%%%%%%%%%%%%%%%%%%%%%%%%%%%%%%%%%%%%%%%%%
%% Appendix  %%%%%%%%%%%%%%%%%%%%%%%%%%%%%%%%%%%%%%%%%%%%
%% Part MACN %%%%%%%%%%%%%%%%%%%%%%%%%%%%%%%%%%%%%%%%%%%%
%% Start     %%%%%%%%%%%%%%%%%%%%%%%%%%%%%%%%%%%%%%%%%%%%
%%           %%%%%%%%%%%%%%%%%%%%%%%%%%%%%%%%%%%%%%%%%%%%
%%%%%%%%%%%%%%%%%%%%%%%%%%%%%%%%%%%%%%%%%%%%%%%%%%%%%%%%%
\Appendixpart




%%%%%%%%%%%%%%%%%%%%%%%%%%%%%%%%%%%%%%%%%%%%%%%%%%%%%%%%%%
%%            %%%%%%%%%%%%%%%%%%%%%%%%%%%%%%%%%%%%%%%%%%%%
%% Appendix A %%%%%%%%%%%%%%%%%%%%%%%%%%%%%%%%%%%%%%%%%%%%
%%            %%%%%%%%%%%%%%%%%%%%%%%%%%%%%%%%%%%%%%%%%%%%
%%%%%%%%%%%%%%%%%%%%%%%%%%%%%%%%%%%%%%%%%%%%%%%%%%%%%%%%%%
\modHeadchapter{モールドとテーブルとの位置}
%!TEX root = ../Mould_Analytical_Calculation_Note.tex

\begin{tcolorbox}[title={2023/07/28時点の\MMname 実測値}, fonttitle=\gtfamily\bfseries]
\begin{align*}
  \text{Bot ($B=0$)}
  \left\{
  \begin{array}{rl}
    X: & 97.790 \sim 99.930\\
    Y: & -823.850\\
    Z: & -634.620
  \end{array}
  \right.\quad
  \text{Top ($B=180.$)}
  \left\{
  \begin{array}{rl}
    X: & -97.980 \sim -99.570\\
    Y: & -823.780\\
    Z: & -634.720
  \end{array}
  \right.
\end{align*}\\
・$X$については、ジグの当たる点の凸部と端部($Z$方向は目分量)\\
・$Y$については、モールドの底が当たる面\\
・$Z$については、$X0$ $Y-850.$における、ジグとの接点\\
※これらの値に、センサー先端球の半径を加減する必要がある
\end{tcolorbox}
%%%%%%%%%%%%%%%%%%




%%%%%%%%%%%%%%%%%%%%%%%%%%%%%%%%%%%%%%%%%%%%%%%%%%%%%%%%%%
%% section 2.1 %%%%%%%%%%%%%%%%%%%%%%%%%%%%%%%%%%%%%%%%%%%
%%%%%%%%%%%%%%%%%%%%%%%%%%%%%%%%%%%%%%%%%%%%%%%%%%%%%%%%%%
\modHeadsection{テーブル中心にあるモールド}
CADによる描画において、テーブルの回転中心が原点(ワールド原点)に置かれているとする。
ここでモールドを描画する際、モールドの中心
%% footnote %%%%%%%%%%%%%%%%%%%%%
\footnote{$R_\mathrm c$に相当する点。}\relax
%%%%%%%%%%%%%%%%%%%%%%%%%%%%%%%%%
をCAD上の原点(ワールド原点)にして描くほうが都合のいいことがある。
このとき、モールドと受板が接するように移動する必要がある。
鉛直方向(トップ-ボトム方向)においては$f_d$だけ動かせばよいが、水平方向の移動距離はあまり自明とはいいがたい。
C側モールド面と受板面との寸法を単純に測ると、(水平方向でなく)最短距離が測定されてしまう。
工夫により水平方向の距離を出すことも可能ではあるが、ここではその距離を定量的に求めておく。



%%%%%%%%%%%%%%%%%%%%%%%%%%%%%%%%%%%%%%%%%%%%%%%%%%%%%%%%%%
%% subsection 2.1.1 %%%%%%%%%%%%%%%%%%%%%%%%%%%%%%%%%%%%%%
%%%%%%%%%%%%%%%%%%%%%%%%%%%%%%%%%%%%%%%%%%%%%%%%%%%%%%%%%%
\subsection{スペーサ取付前}
(スペーサを取付る前の)モールドの中心がテーブル中心Pに置かれている場合を考える。
ボトム側の受板に接するモールドの点と、テーブル中心Pとは、実軸方向に
\begin{align*}
  R_\mathrm c-R_\mathrm i\cos\alpha_{\mathrm U_\mathrm B}
\end{align*}
だけ差がある。
したがって、モールドの受板と接する点の位置は実軸方向に、
\begin{align*}
  \varDelta+\sqrt{R_\mathrm i'^2-\bar l^2}-R_\mathrm c+R_\mathrm i\cos\alpha_{\mathrm U_\mathrm B}\ .
\end{align*}
そのため\pageeqref{eq:afterUBcontact} ($\delta = 0$)より、モールドと(ボトム側の)受板は
\begin{align*}
  \varDelta+\sqrt{R_\mathrm i'^2-\bar l^2}-R_\mathrm c
\end{align*}
だけ実軸方向に離れていることがわかる。
\pageeqref{eq:tableCenter}より、これはテーブル中心Pとモールドの中心湾曲$R_\mathrm c$との差であることがわかる。



%%%%%%%%%%%%%%%%%%%%%%%%%%%%%%%%%%%%%%%%%%%%%%%%%%%%%%%%%%
%% subsection 2.1.2 %%%%%%%%%%%%%%%%%%%%%%%%%%%%%%%%%%%%%%
%%%%%%%%%%%%%%%%%%%%%%%%%%%%%%%%%%%%%%%%%%%%%%%%%%%%%%%%%%
\subsection{スペーサ取付後}
厚さ$\delta\,(>0)$のスペーサを取付けた場合、モールドの受板と接する点とテーブル中心Pとは、実軸方向に
\begin{align*}
  R_\mathrm c-R_\mathrm i\cos\alpha'_{\mathrm U_\mathrm B}
\end{align*}
だけ差があるので、その実軸方向の位置は、
\begin{align*}
  \varDelta+\sqrt{R_\mathrm i'^2-\bar l^2}-R_\mathrm c+R_\mathrm i\cos\alpha'_{\mathrm U_\mathrm B}\ .
\end{align*}
そのため\pageeqref{eq:afterUBcontact}より、モールドと(ボトム側の)受板は
\begin{align*}
  &  \varDelta+\sqrt{R_\mathrm i'^2-\bar l^2}-R_\mathrm c+R_\mathrm i\cos\alpha'_{\mathrm U_\mathrm B}
     -\left(R_\mathrm i'\cos\alpha_{\mathrm U_\mathrm B}+\rho\cos\alpha'_{\mathrm U_\mathrm B}\right)\\
  &= \varDelta-R_\mathrm c+R_\mathrm i'\cos\alpha'_{\mathrm U_\mathrm B}\\
  &= \varDelta-R_\mathrm c
     -\frac\delta2+\sqrt{R_\mathrm i'^2-\frac{\delta^2+(2\bar l)^2}4}\frac{2\bar l}{\sqrt{\delta^2+(2\bar l)^2}}
\end{align*}
だけ実軸方向に離れていることがわかる。
\pageeqref{eq:tableCenter}および\pageeqref{eq:spacerMoveHdistance}より、これはスペーサ取付け後のモールド中心とモールドの中心湾曲$R_\mathrm c$との差であることがわかる。





%%%%%%%%%%%%%%%%%%%%%%%%%%%%%%%%%%%%%%%%%%%%%%%%%%%%%%%%%%
%%            %%%%%%%%%%%%%%%%%%%%%%%%%%%%%%%%%%%%%%%%%%%%
%% Appendix B %%%%%%%%%%%%%%%%%%%%%%%%%%%%%%%%%%%%%%%%%%%%
%%            %%%%%%%%%%%%%%%%%%%%%%%%%%%%%%%%%%%%%%%%%%%%
%%%%%%%%%%%%%%%%%%%%%%%%%%%%%%%%%%%%%%%%%%%%%%%%%%%%%%%%%%
%\modHeadchapter{諸公式}
%%!TEX root = ../Mould_Analytical_Calculation_Note.tex





%%%%%%%%%%%%%%%%%%%%%%%%%%%%%%%%%%%%%%%%%%%%%%%%%%%%%%%%%%
%% section C.1 %%%%%%%%%%%%%%%%%%%%%%%%%%%%%%%%%%%%%%%%%%%
%%%%%%%%%%%%%%%%%%%%%%%%%%%%%%%%%%%%%%%%%%%%%%%%%%%%%%%%%%
\modHeadsection{2点間の距離}
\begin{Column}{}
点($p$, $q$)と直線$ax+by+c=0$との距離$d$は、以下で与えられる。
\begin{align*}
  d = \frac{|ap+bq+c|}{\sqrt{a^2+b^2}}.
\end{align*}
\end{Column}
\begin{Column}{}
点$\boldsymbol p$を通り方向ベクトルが$\boldsymbol m$の直線L上の点と、点$\boldsymbol q$を通り方向ベクトルが$\boldsymbol m'$の直線$\mathrm L'$上の点は、それぞれパラメータ$t$, $t'$を用いて、
\begin{align*}
  \mathrm L: \boldsymbol p+t\boldsymbol m\ , \qquad
  \mathrm L': \boldsymbol q+t'\boldsymbol m'
\end{align*}
で表される。
このとき、L上の点の中で$\mathrm L'$に最も近づく点の位置$\boldsymbol k$は、以下で与えられる
%% footnote %%%%%%%%%%%%%%%%%%%%%
\footnote{2点間の距離の2乗$|\boldsymbol p-\boldsymbol q+t\boldsymbol m-t'\boldsymbol m'|^2$に対し、それぞれのパラメータ$t$, $t'$に関する微分が0となる。
それらを連立して解けば$\boldsymbol k$, $\boldsymbol k'$が求まる。}。
%%%%%%%%%%%%%%%%%%%%%%%%%%%%%%%%%
$\mathrm L'$上の点の中でLに最も近づく点の位置$\boldsymbol k'$についても同様である。
\begin{align*}
  \boldsymbol k
  = \boldsymbol p
    +\frac{(\boldsymbol m-(\boldsymbol m, \boldsymbol m')\boldsymbol m', \boldsymbol p-\boldsymbol q)}
          {1+(\boldsymbol m, \boldsymbol m')^2}\boldsymbol m
\end{align*}
また、これらの差の大きさ$|\boldsymbol k-\boldsymbol k'|$から、2直線間の距離$d$が求まる。
\end{Column}



\end{appendices}
%%%%%%%%%%%%%%%%%%%%%%%%%%%%%%%%%%%%%%%%%%%%%%%%%%%%%%%%%
%%          %%%%%%%%%%%%%%%%%%%%%%%%%%%%%%%%%%%%%%%%%%%%%
%% Appendix %%%%%%%%%%%%%%%%%%%%%%%%%%%%%%%%%%%%%%%%%%%%%
%% Part ACN %%%%%%%%%%%%%%%%%%%%%%%%%%%%%%%%%%%%%%%%%%%%%
%% End      %%%%%%%%%%%%%%%%%%%%%%%%%%%%%%%%%%%%%%%%%%%%%
%%          %%%%%%%%%%%%%%%%%%%%%%%%%%%%%%%%%%%%%%%%%%%%%
%%%%%%%%%%%%%%%%%%%%%%%%%%%%%%%%%%%%%%%%%%%%%%%%%%%%%%%%%
