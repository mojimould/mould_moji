%!TEX root = Mould_Analytical_Calculation_Note.tex

ここでは北村マシニングに関して、プログラムの構成や番号付けの規則を記載しておく。
(なお、ここに記載しているものは正式なルールではない。)


%%%%%%%%%%%%%%%%%%%%%%%%%%%%%%%%%%%%%%%%%%%%%%%%%%%%%%%%%%
%% section C.1 %%%%%%%%%%%%%%%%%%%%%%%%%%%%%%%%%%%%%%%%%%%
%%%%%%%%%%%%%%%%%%%%%%%%%%%%%%%%%%%%%%%%%%%%%%%%%%%%%%%%%%
\section{プログラムの構成}
\begin{enumerate}
\item 各々の製品の計測・加工に対するプログラムをメインプログラムとする
\item 製品の各々の部分の計測・加工に対するプログラムをサブプログラムとして、メインプログラムに挿入する
\item プログラムの種類は、以下のように内容で分ける
  \begin{enumerate}
  \item 製品全体の計測・加工に対するプログラム(メインプログラム)
  \item 製品の部分の計測に対するプログラム(サブプログラム)
  \item 製品の部分の加工に対するプログラム(サブプログラム)
  \item その他、計測・加工には直接的には関係ないプログラム
  \end{enumerate}
\end{enumerate}



%%%%%%%%%%%%%%%%%%%%%%%%%%%%%%%%%%%%%%%%%%%%%%%%%%%%%%%%%%
%% section C.2 %%%%%%%%%%%%%%%%%%%%%%%%%%%%%%%%%%%%%%%%%%%
%%%%%%%%%%%%%%%%%%%%%%%%%%%%%%%%%%%%%%%%%%%%%%%%%%%%%%%%%%
\section{プログラムの番号付け}
\begin{enumerate}
\item プログラム番号には半角数字のみを用いる。
\item 製品の図面番号(番号部分)とメインプログラム番号は同じものとする。
\item
\end{enumerate}

