%!TEX root = Mould_Analytical_Calculation_Note.tex

ここでは北村マシニングに関して、プログラムの構成や番号付けの規則を記載しておく。
(なお、ここに記載しているものは正式なルールではない。)


%%%%%%%%%%%%%%%%%%%%%%%%%%%%%%%%%%%%%%%%%%%%%%%%%%%%%%%%%%
%% section C.1 %%%%%%%%%%%%%%%%%%%%%%%%%%%%%%%%%%%%%%%%%%%
%%%%%%%%%%%%%%%%%%%%%%%%%%%%%%%%%%%%%%%%%%%%%%%%%%%%%%%%%%
\section{プログラムの構成}
\begin{enumerate}
\item 各々の製品の計測・加工に対するプログラムをメインプログラムとする
\item 製品の各々の部分の計測・加工に対するプログラムをサブプログラムとして、メインプログラムに挿入する
\item サブプログラムの種類(番号)は、以下のように内容で分ける
  \begin{enumerate}
  \item 内面溝・逃し溝以外の計測に対するプログラム
  \item 内面溝の計測に対するプログラム
  \item 逃し溝の計測に対するプログラム
  \item 内面溝・逃し溝以外の加工に対するプログラム
  \item 内面溝の加工に対するプログラム
  \item 逃し溝の加工に対するプログラム
  \item その他、製品の加工には直接的には関係ないプログラム
  \end{enumerate}
\end{enumerate}



%%%%%%%%%%%%%%%%%%%%%%%%%%%%%%%%%%%%%%%%%%%%%%%%%%%%%%%%%%
%% section C.2 %%%%%%%%%%%%%%%%%%%%%%%%%%%%%%%%%%%%%%%%%%%
%%%%%%%%%%%%%%%%%%%%%%%%%%%%%%%%%%%%%%%%%%%%%%%%%%%%%%%%%%
\section{プログラムの番号付け}
\begin{enumerate}
\item プログラム番号には半角数字のみを用いる
\item プログラム番号には8桁の数字を用いる(ただし、左側0埋めの有無は問わない)
\item プログラム番号は右詰めとする(左側0埋めの有無は問わない)
\item 製品の図面番号(番号部分)とメインプログラム番号は同じものとする
\item 内面溝・逃し溝以外の測定に関するものは6桁目を1とする
\item 内面溝の測定に関するものは6桁目を2とする
\item 逃し溝の測定に関するものは6桁目を3とする
\item 内面溝・逃し溝以外の加工に関するものは6桁目を5とする
\item 内面溝の加工に関するものは6桁目を6とする
\item 逃し溝の加工に関するものは6桁目を7とする
\item その他に関するものは6桁目を9とする
\item 計測・加工の両方に用いるものは番号の若いほうに合わせる
\end{enumerate}
