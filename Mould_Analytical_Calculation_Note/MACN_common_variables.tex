%!TEX root = Mould_Analytical_Calculation_Note.tex

%%%%%%%%%%%%%%%%%%%%%%%%%%%%%%%%%%%%%%%%%%%%%%%%%%%%%%%%%%
%% section B.1 %%%%%%%%%%%%%%%%%%%%%%%%%%%%%%%%%%%%%%%%%%%
%%%%%%%%%%%%%%%%%%%%%%%%%%%%%%%%%%%%%%%%%%%%%%%%%%%%%%%%%%
\section{北村マシニング}



%%%%%%%%%%%%%%%%%%%%%%%%%%%%%%%%%%%%%%%%%%%%%%%%%%%%%%%%%%
%% subsection B.1.1 %%%%%%%%%%%%%%%%%%%%%%%%%%%%%%%%%%%%%%
%%%%%%%%%%%%%%%%%%%%%%%%%%%%%%%%%%%%%%%%%%%%%%%%%%%%%%%%%%
\subsection{コモン変数}
内面溝用マシニングで取り決めているコモン変数について、以下に挙げておく。

\begin{tcolorbox}[twoctable={コモン変数:北村マシニング}{}]
\#401 & パレット\#1 テーブル中心機械座標$X$\\\hline
\#402 & パレット\#1 テーブル中心機械座標$Y$\\\hline
\#403 & パレット\#1 テーブル中心機械座標$Z$\\\hline
\#404 & パレット\#1 テーブル中心機械座標$B$\\\hline
\#405 & パレット\#2 テーブル中心機械座標$X$\\\hline
\#406 & パレット\#2 テーブル中心機械座標$Y$\\\hline
\#407 & パレット\#2 テーブル中心機械座標$Z$\\\hline
\#408 & パレット\#2 テーブル中心機械座標$B$\\\hline
\#409 & 工具中心機械座標$C$\\\hline
\#512 & タッチセンサープローブ半径\\\hline
\#701-733 & 内面溝レベル1サブプログラム用\\\hline
\#734-766 & 内面溝レベル2サブプログラム用\\\hline
\#767-799 & 内面溝レベル3サブプログラム用\\
\end{tcolorbox}




%%%%%%%%%%%%%%%%%%%%%%%%%%%%%%%%%%%%%%%%%%%%%%%%%%%%%%%%%%
%% subsection B.1.2 %%%%%%%%%%%%%%%%%%%%%%%%%%%%%%%%%%%%%%
%%%%%%%%%%%%%%%%%%%%%%%%%%%%%%%%%%%%%%%%%%%%%%%%%%%%%%%%%%
\subsection{システム変数}
内面溝用マシニングのシステム変数について、主なものを以下に挙げておく。

\begin{tcolorbox}[twoctable={システム変数:北村マシニング}{}]
\#1000 & パレット\#~~1:\#1, 2:\#2\\\hline
\#1004 & タッチセンサー電源~~0: off, 1: on\\\hline
\#1005 & タッチセンサー電池残量~~0: ok, 1: low\\\hline
\#3000 & アラーム\\\hline
\#4012 & 現在のワーク座標系G\#\\\hline
\#4111 & 現在の工具長補正H\#\\\hline
\#504x & 現在のワーク座標系の座標 1:X, 2:Y, 3:Z, 4:B\\\hline
\#506x & スキップ座標 1:X, 2:Y, 3:Z, 4:B ※工具補正含まず\\\hline
\#522x & ワーク座標系G54原点の機械座標 1:X, 2:Y, 3:Z, 4:B\\\hline
\#524x & ワーク座標系G55原点の機械座標 1:X, 2:Y, 3:Z, 4:B\\\hline
\#526x & ワーク座標系G56原点の機械座標 1:X, 2:Y, 3:Z, 4:B\\\hline
\#528x & ワーク座標系G57原点の機械座標 1:X, 2:Y, 3:Z, 4:B\\\hline
\#530x & ワーク座標系G58原点の機械座標 1:X, 2:Y, 3:Z, 4:B\\\hline
\#532x & ワーク座標系G59原点の機械座標 1:X, 2:Y, 3:Z, 4:B\\\hline
\end{tcolorbox}


