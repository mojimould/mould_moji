%!TEX root = Mould_Analytical_Calculation_Note.tex

%%%%%%%%%%%%%%%%%%%%%%%%%%%%%%%%%%%%%%%%%%%%%%%%%%%%%%%%%%
%% section B.1 %%%%%%%%%%%%%%%%%%%%%%%%%%%%%%%%%%%%%%%%%%%
%%%%%%%%%%%%%%%%%%%%%%%%%%%%%%%%%%%%%%%%%%%%%%%%%%%%%%%%%%
\section{内面溝用マシニング}
内面溝用マシニングで取り決めているコモン変数について、以下に挙げておく。

\begin{tcolorbox}[twoctable={北村マシニング}{}]
コモン変数 & 内容 \\
\hline
\#401 & パレット\#1 テーブル中心機械座標$X$\\
\#402 & パレット\#1 テーブル中心機械座標$Y$\\
\#403 & パレット\#1 テーブル中心機械座標$Z$\\
\#404 & パレット\#1 テーブル中心機械座標$B$\\
\#405 & パレット\#2 テーブル中心機械座標$X$\\
\#406 & パレット\#2 テーブル中心機械座標$Y$\\
\#407 & パレット\#2 テーブル中心機械座標$Z$\\
\#408 & パレット\#2 テーブル中心機械座標$B$\\
\#409 & 工具中心機械座標$C$\\
\#512 & タッチセンサープローブ半径\\
\#701-733 & 内面溝レベル1サブプログラム用 \\
\#734-766 & 内面溝レベル2サブプログラム用 \\
\#767-799 & 内面溝レベル3サブプログラム用 \\
\end{tcolorbox}