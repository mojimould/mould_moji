%!TEX root = Mould_Analytical_Calculation_Note.tex


ここでは\pageautoref{part:ACN}で記述した変数等の表記を一覧にして示しておく。

%%%%%%%%%%%%%%%%%%%%%%%%%%%%%%%%%%%%%%%%%%%%%%%%%%%%%%%%%%
%% captionof %%%%%%%%%%%%%%%%%%%%%%%%%%%%%%%%%%%%%%%%%%%%%
%%%%%%%%%%%%%%%%%%%%%%%%%%%%%%%%%%%%%%%%%%%%%%%%%%%%%%%%%%
\begin{Notation}{長さ・角度}{図面}
$R_\mathrm c$ & 中心湾曲半径 & ○\\\hline
$R_\mathrm i$ & 内側(C面側)湾曲半径 & \\\hline
$R_\mathrm o$ & 外側(A面側)湾曲半径 & \\\hline
$f_\mathrm T$ & トップ振分長(調整前) & ○\\\hline
$f_\mathrm B$ & ボトム振分長(調整前) &○\\\hline
$f_\mathrm T'$ & トップ振分長(調整後) &\\\hline
$f_\mathrm B'$ & ボトム振分長(調整後) &\\\hline
$f_\mathrm d$ & $\displaystyle \frac{f_\mathrm B-f_\mathrm T}2$ &\\\hline
$\theta$ & 振分長調整用角度 &\\\hline
$\psi$ & スペーサによる$\mathrm U_\mathrm B$を中心とした傾き角 &\\\hline
$W_x$ & AC外径 & ○\\\hline
$W_y$ & BD外径 & ○\\\hline
$w_\mathrm T$ & トップ端内径(略称) & ○\\\hline
$w_\mathrm B$ & ボトム端内径(略称) & ○\\\hline
$w_x$ & 端面AC内径(略称) & ○\\\hline
$w_y$ & 端面BD内径(略称) & ○\\\hline
$\mathfrak W_\mathrm T$ & トップ側AC外削径 & ○\\\hline
$\mathfrak W_\mathrm B$ & トップ側BD外削径 & ○\\\hline
$h_\mathrm T$ & トップ側外削長 & ○\\\hline
$h_\mathrm B$ & ボトム側外削長 & ○\\\hline
$\tau_y$ & BD方向肉厚(略称) &\\\hline
$\tau_\mathrm T$ & トップ端A側肉厚(指定時) & ○\\\hline
$\tau_\mathrm B$ & ボトム端A側肉厚(指定時) & ○\\\hline
$\mu$ & めっき膜厚 & ○\\\hline
$T_x$ & $X$方向通り芯 & ○\\\hline
$T_y$ & $Y$方向通り芯 &\\\hline
$\delta x$ & 回転中心とジグ中心とのずれ$X$ &\\\hline
$\delta z$ & 回転中心とジグ中心とのずれ$Z$ &\\\hline
\hline
$\alpha_{\mathrm c}$ & 湾曲中心$O$と$\mathrm T_\mathrm c$の角度(鋭角) &\\\hline
$\alpha_{\mathrm T_\mathrm i}$ & 湾曲中心$O$と$\mathrm T_\mathrm i$の角度(鋭角) &\\\hline
$\alpha_{\mathrm T_\mathrm o}$ & 湾曲中心$O$と$\mathrm T_\mathrm o$の角度(鋭角) &\\\hline
$\alpha_{\mathrm U_\mathrm B}$ & 湾曲中心$O$と$\mathrm U_\mathrm B$の角度(鋭角) &\\\hline
$\alpha_{\mathrm U_\mathrm T}'$ & 湾曲中心$O'$と$\mathrm U_\mathrm T'$の角度(鋭角) &\\\hline
$\alpha_{\mathrm U_\mathrm B}'$ & 湾曲中心$O'$と$\mathrm U_\mathrm B'$の角度(鋭角) &\\\hline
\end{Notation}


\begin{Notation}{溝}{図面}
$W_\mathrm M$ & 溝径(略称) & ○\\\hline
$W_{mx}$ & 溝AC径 & ○\\\hline
$\kappa_p$ & 溝位置 & ○\\\hline
$\kappa_w$ & 溝幅 & ○\\\hline
$\kappa_d$ & A側溝深さ(指定時) & ○\\\hline
\end{Notation}


\begin{Notation}{内面溝}{図面}
$m$ & 内面溝列数 & ○\\\hline
$q$ & トップ端から内面溝1列目までの距離 & ○\\\hline
$p_x$ & 内面溝水平方向ピッチ & ○\\\hline
$p_z$ & 内面溝鉛直方向ピッチ & ○\\\hline
$d_i$ & 内面溝$i$列目の長さ & ○\\\hline
$d_\mathrm o$ & 内面溝奇数列目の長さ(全て同じ場合) & ○\\\hline
$d_\mathrm e$ & 内面溝偶数列目の長さ(全て同じ場合) & ○\\\hline
$n_m$ & $m$列目の内面溝の個数 & ○\\\hline
$n_\mathrm o$ & 奇数列目の内面溝の個数(全て同じ場合) & ○\\\hline
$n_\mathrm e$ & 偶数列目の内面溝の個数(全て同じ場合) & ○\\\hline
$\lambda_i$ & 内面テーパ表におけるトップ端からの$i$番目の距離 & ○\\\hline
$w_{\mathrm Ai}$ & $\lambda_i$に対するAC内径 & ○\\\hline
$w_{\mathrm Bi}$ & $\lambda_i$に対するBD内径 & ○\\\hline
$w_{\mathrm A\lambda}$ & トップ端からの距離$\lambda$に対する近似AC内径 &\\\hline
$w_{\mathrm B\lambda}$ & トップ端からの距離$\lambda$に対する近似BD内径 &\\\hline
$w_{\mathrm A\lambda}'$ & $w_{\mathrm A\lambda}+2\mu$ &\\\hline
$w_{\mathrm B\lambda}'$ & $w_{\mathrm B\lambda}+2\mu$ &\\\hline
$w_{\mathrm A}'$ & $w_{\mathrm A\lambda}'$の略記 &\\\hline
$w_{\mathrm B}'$ & $w_{\mathrm B\lambda}'$の略記 &\\\hline
$\phi$ & 内面溝用傾け角(鋭角) &\\\hline
$\phi_\mathrm C$ & C面用内面溝用傾け角(鋭角) &\\\hline
$g_t'$ & 傾け後のトップ端内面中心 &\\\hline
$g_{tx}'$ & 傾け後のトップ端内面中心$X$ &\\\hline
$g_{ty}'$ & 傾け後のトップ端内面中心$Y$ &\\\hline
$g_{tZ}'$ & 傾け後のトップ端内面中心$Z$ &\\\hline
$\mathcal R_\mathrm o$ & A側内面湾曲 &\\\hline
$\mathcal R_\mathrm i$ & C側内面湾曲 &\\\hline
$\mathcal L_i$ & $i$列目の湾曲中心とトップ端の湾曲中心との差$X$ &\\\hline
$\mathcal L_{i,j}$ & $i$列目の湾曲中心と$j$列目の湾曲中心との差$X$ &\\\hline
$\mathcal D_{xi,\mathrm A}$ & A面$i$列目$j$番目の内面溝$X$(P原点・傾け前) &\\\hline
$\mathcal D_{xi,\mathrm C}$ & C面$i$列目$j$番目の内面溝$X$(P原点・傾け前) &\\\hline
$\mathcal D_{xij,\mathrm B}$ & B, D面$i$列目$j$番目の内面溝$X$(P原点・傾け前) &\\\hline
$\mathcal D_{yij,\mathrm A}$ & A, C面$i$列目$j$番目の内面溝$Y$(P原点・傾け前) &\\\hline
$\mathcal D_{yi,\mathrm B}$ & B面$i$列目$j$番目の内面溝$Y$(P原点・傾け前) &\\\hline
$\mathcal D_{yi,\mathrm D}$ & D面$i$列目$j$番目の内面溝$Y$(P原点・傾け前) &\\\hline
$\mathcal D_{zi,\mathrm D}$ & $i$列目$j$番目の内面溝$Z$(P原点・傾け前) &\\\hline
$\mathcal D_{xij,\mathrm A}'$ & A面$i$列目$j$番目の内面溝$X$(P原点・傾け後) &\\\hline
$\mathcal D_{yij,\mathrm A}'$ & A面$i$列目$j$番目の内面溝$Y$(P原点・傾け後) &\\\hline
$\mathcal D_{zij,\mathrm A}'$ & A面$i$列目$j$番目の内面溝$Z$(P原点・傾け後) &\\\hline
$\mathcal D_{xij,\mathrm C}'$ & C面$i$列目$j$番目の内面溝$X$(P原点・傾け後) &\\\hline
$\mathcal D_{yij,\mathrm C}'$ & C面$i$列目$j$番目の内面溝$Y$(P原点・傾け後) &\\\hline
$\mathcal D_{zij,\mathrm C}'$ & C面$i$列目$j$番目の内面溝$Z$(P原点・傾け後) &\\\hline
$\mathcal D_{xij,\mathrm B}'$ & B面$i$列目$j$番目の内面溝$X$(P原点・傾け後) &\\\hline
$\mathcal D_{yij,\mathrm B}'$ & B面$i$列目$j$番目の内面溝$Y$(P原点・傾け後) &\\\hline
$\mathcal D_{zij,\mathrm B}'$ & B面$i$列目$j$番目の内面溝$Z$(P原点・傾け後) &\\\hline
$\mathcal D_{xij,\mathrm D}'$ & D面$i$列目$j$番目の内面溝$X$(P原点・傾け後) &\\\hline
$\mathcal D_{yij,\mathrm D}'$ & D面$i$列目$j$番目の内面溝$Y$(P原点・傾け後) &\\\hline
$\mathcal D_{zij,\mathrm D}'$ & D面$i$列目$j$番目の内面溝$Z$(P原点・傾け後) &\\\hline
\end{Notation}


\clearpage
\begin{Notation}{計測値}{参照}
$G_{Tx}$ & 計測した外側中心$X$ &\\\hline
$G_{Ty}$ & 計測した外側中心$Y$ &\\\hline
$\mathcal G_{Bx}$ & 計測した$\mathfrak B_\mathrm c'$の$X$座標 &\\\hline
$\mathcal G_{By}$ & 計測した$\mathfrak B_\mathrm c'$の$Y$座標 &\\\hline
$\mathcal G_{Tx}$ & 計測した$\mathfrak T_\mathrm c'$の$X$座標 &\\\hline
$\mathcal G_{mx}$ & 計測した(A側溝深さ基準)溝中心 &\\\hline
$g_t$ & 計測したトップ端における内径中心 &\\\hline
$g_{tx}$ & 計測したトップ端における内径中心$X$ &\\\hline
$g_{ty}$ & 計測したトップ端における内径中心$Y$ &\\\hline
\end{Notation}


\begin{Notation}{ジグ}{図面}
$l$ & ジグ長さの半分 & ○\\\hline
$\rho$ & 受板の半径 & ○\\\hline
$\sigma$ & 受板の幅 & ○\\\hline
$R_\mathrm i'$ & $R_i-\rho$ &\\\hline
$\bar l$ & $\displaystyle l-\frac\sigma2$ &\\\hline
$l'$ & $l+f_\mathrm d$ &\\\hline
$\delta$ & スペーサの厚さ &\\\hline
$\varDelta$ & $\mathrm U_\mathrm B$とテーブル中心$P$との水平距離 &\\\hline
$\varDelta'$ & Oを原点としたテーブル中心$P$の水平距離 &\\\hline
\end{Notation}


\begin{Notation}{工具}{図面}
$\delta_w$ & 端面加工 輪郭調整用パラメータ &\\\hline
$\phi_\mathrm D$ & 端面加工用フェイスミル刃径(直径) & ○\\\hline
$\phi_\mathrm D'$ & 端面加工用フェイスミル最大刃径(直径) & ○\\\hline
$d_\mathrm e$ & テーパーエンドミル先端からの距離(参照直径用) &\\\hline
$D_\mathrm r$ & テーパーエンドミル参照直径 &\\\hline
$D_\mathrm e$ & テーパーエンドミル先端径(直径) & ○\\\hline
$\xi_\mathrm e$ & テーパーエンドミル テーパー角度(片角) & ○\\\hline
\end{Notation}


\begin{Notation}{eテーパ}{参照}
$T_\mathrm l$ & 液相線温度 &\\\hline
$T_\mathrm s$ & 固相線温度 &\\\hline
$T_0$ & 基準温度 &\\\hline
$X_\mathrm i$ & 化学組成の含有量 &\\\hline
$k_\mathrm i$ & 化学組成における影響係数 &\\\hline
\end{Notation}


\clearpage
\begin{Notation}{位置}{参照}
O & 湾曲中心点 &\\\hline
P & テーブル中心点 &\\\hline
T$_{R_\mathrm c}$ & トップ端における湾曲中心 &\\\hline
$\mathrm T_\mathrm i$ & トップ内側(C側)端点 &\\\hline
$\mathrm T_\mathrm o$ & トップ外側(A側)端点 &\\\hline
$\mathrm B_\mathrm i$ & ボトム内側(C側)端点 &\\\hline
$\mathrm B_\mathrm o$ & ボトム外側(A側)端点 &\\\hline
$\mathrm U_\mathrm T$ & トップ側受板中心 &\\\hline
$\mathrm U_\mathrm B$ & ボトム側受板中心 &\\\hline
$O'$ & $-\psi$傾き後の湾曲中心点 &\\\hline
$\mathrm U_\mathrm T'$ & $-\psi$傾き後のトップ側受板中心 &\\\hline
$\mathrm U_\mathrm B'$ & $-\psi$傾き後のボトム側受板中心 &\\\hline
T$_{R_\mathrm c}'$ & 傾き後のトップ端における湾曲中心 &\\\hline
T$_\mathrm c'$ & 傾き後のトップ端における外径中心 &\\\hline
$\mathrm T_\mathrm i'$ & 傾き後のトップ内側(C側)端点 &\\\hline
$\mathrm T_\mathrm o'$ & 傾き後のトップ外側(A側)端点 &\\\hline
B$_{R_\mathrm c}'$ & 傾き後のボトム端における湾曲中心 &\\\hline
B$_\mathrm c'$ & 傾き後のボトム端における外径中心 &\\\hline
B$_{R_\mathrm i}'$ & 傾き後のボトム内側(C側)端点 &\\\hline
B$_{R_\mathrm o}'$ & 傾き後のボトム外側(A側)端点 &\\\hline
$\mathfrak T_\mathrm c$ & トップ外削中心 &\\\hline
$\mathfrak B_\mathrm c$ & ボトム外削中心 &\\\hline
$\mathfrak T_\mathrm c'$ & トップ外削中心(P原点) &\\\hline
$\mathfrak B_\mathrm c'$ & ボトム外削中心(P原点) &\\\hline
$\mathfrak B_\mathrm o'$ & ボトム外削A面 &\\\hline
b$_\mathrm c'$ & ボトム端内面中心 &\\\hline
b$_\mathrm o'$ & ボトム端A側内面 &\\\hline
t$_\mathrm o'$ & トップ端A側内面 &\\\hline
M & 溝中心 &\\\hline
M$'$ & 傾き後の溝中心 &\\\hline
\end{Notation}