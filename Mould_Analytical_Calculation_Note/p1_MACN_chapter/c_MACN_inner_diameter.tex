%!TEX root = ../Mould_Analytical_Calculation_Note.tex



%%%%%%%%%%%%%%%%%%%%%%%%%%%%%%%%%%%%%%%%%%%%%%%%%%%%%%%%%%
%% section 7.1 %%%%%%%%%%%%%%%%%%%%%%%%%%%%%%%%%%%%%%%%%%%
%%%%%%%%%%%%%%%%%%%%%%%%%%%%%%%%%%%%%%%%%%%%%%%%%%%%%%%%%%
\section{eテーパの算定}
使用する鋼材の種類として、C, Si, Mn, P, Sが含まれている場合を考える。
JIS規格に基づいた鋼種を用いるものとすれば、その規格によってそれぞれの化学組成の含有量も決定される。
それぞれの化学組成の含有量($\mathrm{wt}\%$)を$X_\mathrm C$, $X_\mathrm{Si}$, $X_\mathrm{Mn}$, $X_\mathrm P$, $X_\mathrm S$とし、またその影響係数を$k_\mathrm C$, $k_\mathrm{Si}$, $k_\mathrm{Mn}$, $k_\mathrm P$, $k_\mathrm S$とする。
また、その鋼材の液相線温度を$T_\mathrm l$[$^\circ\mathrm C$], 固相線温度を$T_\mathrm s$[$^\circ\mathrm C$]とする。
一般にこれらの温度は、ある基準となる温度$T_0$に対して、
\begin{align*}
  T = T_0-\sum_i k_iX_i
\end{align*}
として与えられる。
今の場合だと、
\begin{align*}
  T_l
  &= 1536-78X_\mathrm C-7.6X_\mathrm{Si}-4.9X_\mathrm{Mn}-34.4X_\mathrm P-38X_\mathrm S~,\\
  T_s
  &= 1536-415.5X_\mathrm C-12.3X_\mathrm{Si}-6.8X_\mathrm{Mn}-124.5X_\mathrm P-183.9X_\mathrm S
\end{align*}
となることが知られている\cite{1986KO}。

\begin{tabular}[t]{|c|c|c|c|c|c|c|}
  \hline
  鋼材(wt\%) & C & Si & Mn & P & S
  \\\hline
  JISコード & 0.1 & 0.5 & 0.6 & 0.7 & 0.8
  \\\hline
\end{tabular}
