%!TEX root = Mould_Analytical_Calculation_Note.tex





%%%%%%%%%%%%%%%%%%%%%%%%%%%%%%%%%%%%%%%%%%%%%%%%%%%%%%%%%%
%%            %%%%%%%%%%%%%%%%%%%%%%%%%%%%%%%%%%%%%%%%%%%%
%% chapter    %%%%%%%%%%%%%%%%%%%%%%%%%%%%%%%%%%%%%%%%%%%%
%%            %%%%%%%%%%%%%%%%%%%%%%%%%%%%%%%%%%%%%%%%%%%%
%%%%%%%%%%%%%%%%%%%%%%%%%%%%%%%%%%%%%%%%%%%%%%%%%%%%%%%%%%
\modHeadchapter{数値解析}
基本的に、\index{すうちかいせき@数値解析}数値解析については数値計算用の言語を用いて行うため、その詳細は別ドキュメントに譲る。
ここでは数値解析に必要な部分をピックアップしておく。
なお、ここでは主に\DMname について述べるため、スペーサに関するものは省略する。


%%%%%%%%%%%%%%%%%%%%%%%%%%%%%%%%%%%%%%%%%%%%%%%%%%%%%%%%%%
%% section 9.1 %%%%%%%%%%%%%%%%%%%%%%%%%%%%%%%%%%%%%%%%%%%
%%%%%%%%%%%%%%%%%%%%%%%%%%%%%%%%%%%%%%%%%%%%%%%%%%%%%%%%%%
\modHeadsection{振分け}
テーブルを$-\theta$だけ回転させたときのトップ・ボトム側の\index{ふりわけちょう@振分長}振分長$f'_\mathrm T$, $f'_\mathrm B$は、\eqref{eq:saifuriwake}より、
\begin{align*}
  f_\mathrm T' = f_\mathrm T+\varDelta'\!\sin\theta~~, \quad
  f_\mathrm B' = (f_\mathrm T+f_\mathrm B)-f_\mathrm T'
\end{align*}
トップ側とボトム側の振分長が均等になるときの回転角$\theta'$は、\pageeqref{eq:saifuriwakeangle}より、
\begin{align*}
  \sin\theta' = \frac{f_d}{\varDelta'}\ .
\end{align*}
ここで、
\begin{align*}
  \varDelta' = \varDelta+\sqrt{R_\mathrm i'-\bar l^2}\ , \quad
  R_\mathrm i' = R_\mathrm c-\frac{W_x}2-\rho\ ,\quad
  \bar l = l-\frac\sigma2\ ,\quad
  f_d = \frac{f_\mathrm B-f_\mathrm T}2\ .
\end{align*}



%%%%%%%%%%%%%%%%%%%%%%%%%%%%%%%%%%%%%%%%%%%%%%%%%%%%%%%%%%
%% section 9.2 %%%%%%%%%%%%%%%%%%%%%%%%%%%%%%%%%%%%%%%%%%%
%%%%%%%%%%%%%%%%%%%%%%%%%%%%%%%%%%%%%%%%%%%%%%%%%%%%%%%%%%
\modHeadsection{湾曲中心}
トップ端およびボトム端における($-\theta$回転後の)\index{わんきょくちゅうしん@湾曲中心}湾曲中心の$X$値は、\pageeqref{eq:tableTRc}, \pageeqref{eq:tableBRc}よりそれぞれ、
\begin{align*}
  \sqrt{R_\mathrm c^2-f_\mathrm T^2}-\varDelta'\!\cos\theta~, \quad
  \varDelta'\!\cos\theta-\sqrt{R_\mathrm c^2-f_\mathrm B^2}~.
\end{align*}
また、(計算上の)\index{うちがわちゅうしん@内側中心}内側中心はこの湾曲中心を持って代用してもよいものとする。


%%%%%%%%%%%%%%%%%%%%%%%%%%%%%%%%%%%%%%%%%%%%%%%%%%%%%%%%%%
%% section 9.2 %%%%%%%%%%%%%%%%%%%%%%%%%%%%%%%%%%%%%%%%%%%
%%%%%%%%%%%%%%%%%%%%%%%%%%%%%%%%%%%%%%%%%%%%%%%%%%%%%%%%%%
\modHeadsection{外側中心}
トップ端およびボトム端における\index{そとがわちゅうしん@外側中心}外側中心の$X$位置は、\pageeqref{eq:tableTc}, \pageeqref{eq:tableBc}よりそれぞれ、
\begin{align*}
  \frac{\sqrt{R_\mathrm o^2-f_\mathrm T^2}+\sqrt{R_\mathrm i^2-f_\mathrm T^2}}2-\varDelta'\cos\theta~, \quad
  \varDelta'\cos\theta-\frac{\sqrt{R_\mathrm o^2-f_\mathrm B^2}+\sqrt{R_\mathrm i^2-f_\mathrm B^2}}2\ .
\end{align*}



%%%%%%%%%%%%%%%%%%%%%%%%%%%%%%%%%%%%%%%%%%%%%%%%%%%%%%%%%%
%% section 9.4 %%%%%%%%%%%%%%%%%%%%%%%%%%%%%%%%%%%%%%%%%%%
%%%%%%%%%%%%%%%%%%%%%%%%%%%%%%%%%%%%%%%%%%%%%%%%%%%%%%%%%%
\modHeadsection{外削}


%%%%%%%%%%%%%%%%%%%%%%%%%%%%%%%%%%%%%%%%%%%%%%%%%%%%%%%%%%
%% subsection 9.4.1 %%%%%%%%%%%%%%%%%%%%%%%%%%%%%%%%%%%%%%
%%%%%%%%%%%%%%%%%%%%%%%%%%%%%%%%%%%%%%%%%%%%%%%%%%%%%%%%%%
\subsection{外削中心:ボトムA面肉厚基準の場合}
\index{テーブルちゅうしん@テーブル中心}テーブル中心\index{P(てーぶるちゅうしん)@P(テーブル中心)}Pを原点としたボトム側外削径の中心$\mathfrak B_\mathrm c'$の(おおよその)$X$座標は、\pageeqref{eq:gaisakucenterBt}より、
\begin{align*}
  \varDelta'\cos\theta-\frac{\sqrt{R_\mathrm o^2-f_\mathrm B^2}+\sqrt{R_\mathrm i^2-f_\mathrm B^2}}2
  -\frac{w_\mathrm B}2-\tau_\mathrm B+\frac{\mathfrak W_\mathrm B}2\ .
\end{align*}
このとき、計測したA側内面b$_\mathrm o'$の$X$座標が\pageeqref{eq:gaisakucenterBr}となるように、原点$\mathfrak B_\mathrm c'$を定める。
\begin{align*}
  -\left(\frac{\mathfrak W_\mathrm B}2-\tau_\mathrm B+\mu\right).
\end{align*}
トップ側にも外削がある場合、計測で定めた$\mathfrak B_\mathrm c'$の$X$座標$\mathcal G_{\mathrm Bx}$および\index{とおりしん@通り芯}通り芯$T_x$を用いて\pageeqref{eq:BbasedTx}で与えられる。
\begin{align*}
  -\mathcal G_{Bx}+T_x\ .
\end{align*}


%%%%%%%%%%%%%%%%%%%%%%%%%%%%%%%%%%%%%%%%%%%%%%%%%%%%%%%%%%
%% subsection 9.4.2 %%%%%%%%%%%%%%%%%%%%%%%%%%%%%%%%%%%%%%
%%%%%%%%%%%%%%%%%%%%%%%%%%%%%%%%%%%%%%%%%%%%%%%%%%%%%%%%%%
\subsection{外削中心:トップA面肉厚基準の場合}
\index{テーブルちゅうしん@テーブル中心}テーブル中心\index{P(てーぶるちゅうしん)@P(テーブル中心)}Pを原点としたトップ側外削径の中心$\mathfrak T_\mathrm c'$の(おおよその)$X$座標は、\pageeqref{eq:gaisakucenterTt}より、
\begin{align*}
  \frac{\sqrt{R_\mathrm o^2-f_\mathrm T^2}+\sqrt{R_\mathrm i^2-f_\mathrm T^2}}2-\varDelta'\cos\theta
  +\frac{w_\mathrm T}2+\tau_\mathrm T-\frac{\mathfrak W_\mathrm T}2\ .
\end{align*}
このとき、計測したA側内面t$_\mathrm o'$の$X$座標が\pageeqref{eq:gaisakucenterTr}となるように、原点$\mathfrak T_\mathrm c'$を定める。
\begin{align*}
  \frac{\mathfrak W_\mathrm T}2-\tau_\mathrm T+\mu~.
\end{align*}
ボトム側にも外削がある場合、計測で定めた$\mathfrak T_\mathrm c'$の$X$座標$\mathcal G_{\mathrm Tx}$および\index{とおりしん@通り芯}通り芯$T_x$を用いて\pageeqref{eq:TbasedTx}で与えられる。
\begin{align*}
  -\mathcal G_{Tx}+T_x
\end{align*}


%%%%%%%%%%%%%%%%%%%%%%%%%%%%%%%%%%%%%%%%%%%%%%%%%%%%%%%%%%
%% subsection 9.4.2 %%%%%%%%%%%%%%%%%%%%%%%%%%%%%%%%%%%%%%
%%%%%%%%%%%%%%%%%%%%%%%%%%%%%%%%%%%%%%%%%%%%%%%%%%%%%%%%%%
\subsection{外削長}
\index{がいさくちょう@外削長}外削長に関しては、基本的には\index{ふりわけちょう@振分長}振分長から外削長を引いた位置に$Z$座標を合わせればよい。
ただし、トップ側において、外削長$h_\mathrm T$が\index{みぞいち@溝位置}溝位置$\kappa_p$と\index{みぞはば@溝幅}溝幅$\kappa_w$の和に一致する場合は、外削長を$\kappa_p+1$mmとして切削する。
つまり、
\begin{align*}
  h_\mathrm T = \kappa_p+\kappa_w \quad \longrightarrow \quad h_\mathrm T+1[\mathrm{mm}]
\end{align*}



%%%%%%%%%%%%%%%%%%%%%%%%%%%%%%%%%%%%%%%%%%%%%%%%%%%%%%%%%%
%% section 9.2 %%%%%%%%%%%%%%%%%%%%%%%%%%%%%%%%%%%%%%%%%%%
%%%%%%%%%%%%%%%%%%%%%%%%%%%%%%%%%%%%%%%%%%%%%%%%%%%%%%%%%%
\modHeadsection{溝中心}



%%%%%%%%%%%%%%%%%%%%%%%%%%%%%%%%%%%%%%%%%%%%%%%%%%%%%%%%%%
%% section 9.2 %%%%%%%%%%%%%%%%%%%%%%%%%%%%%%%%%%%%%%%%%%%
%%%%%%%%%%%%%%%%%%%%%%%%%%%%%%%%%%%%%%%%%%%%%%%%%%%%%%%%%%
\modHeadsection{面取中心}





%%%%%%%%%%%%%%%%%%%%%%%%%%%%%%%%%%%%%%%%%%%%%%%%%%%%%%%%%%
%%%%%%%%%%%%%%%%%%%%%%%%%%%%%%%%%%%%%%%%%%%%%%%%%%%%%%%%%%
%%%%%%%%%%%%%%%%%%%%%%%%%%%%%%%%%%%%%%%%%%%%%%%%%%%%%%%%%%
\begin{tcolorbox}[title={内面溝加工に関する工具(T31, T50)の情報}, fonttitle=\gtfamily\bfseries]
・タッチセンサープローブの半径:5 \quad ・タッチセンサープローブの軸の半径:3.75\\
・切削用工具径:40 \quad ・切削用工具シャンク径:25
\end{tcolorbox}
%%%%%%%%%%%%%%%%%%%%%%%%%%%%%%%%%%%%%%%%%%%%%%%%%%%%%%%%%%
%%%%%%%%%%%%%%%%%%%%%%%%%%%%%%%%%%%%%%%%%%%%%%%%%%%%%%%%%%
%%%%%%%%%%%%%%%%%%%%%%%%%%%%%%%%%%%%%%%%%%%%%%%%%%%%%%%%%%
\begin{tcolorbox}[title={2023/07/28時点の\MMname 実測値}, fonttitle=\gtfamily\bfseries]
\begin{align*}
  \text{Bot ($B=0$)}
  \left\{
  \begin{array}{rl}
    X: & 97.790 \sim 99.930\\
    Y: & -823.850\\
    Z: & -634.620
  \end{array}
  \right.\quad
  \text{Top ($B=180.$)}
  \left\{
  \begin{array}{rl}
    X: & -97.980 \sim -99.570\\
    Y: & -823.780\\
    Z: & -634.720
  \end{array}
  \right.
\end{align*}\\
・$X$については、ジグの当たる点の凸部と端部($Z$方向は目分量)\\
・$Y$については、モールドの底が当たる面\\
・$Z$については、$X0$ $Y-850.$における、ジグとの接点\\
※これらの値に、センサー先端球の半径を加減する必要がある
\end{tcolorbox}
%%%%%%%%%%%%%%%%%%%%%%%%%%%%%%%%%%%%%%%%%%%%%%%%%%%%%%%%%%
%%%%%%%%%%%%%%%%%%%%%%%%%%%%%%%%%%%%%%%%%%%%%%%%%%%%%%%%%%
%%%%%%%%%%%%%%%%%%%%%%%%%%%%%%%%%%%%%%%%%%%%%%%%%%%%%%%%%%




%%%%%%%%%%%%%%%%%%%%%%%%%%%%%%%%%%%%%%%%%%%%%%%%%%%%%%%%%%
%%           %%%%%%%%%%%%%%%%%%%%%%%%%%%%%%%%%%%%%%%%%%%%%
%% chapter 2 %%%%%%%%%%%%%%%%%%%%%%%%%%%%%%%%%%%%%%%%%%%%%
%%           %%%%%%%%%%%%%%%%%%%%%%%%%%%%%%%%%%%%%%%%%%%%%
%%%%%%%%%%%%%%%%%%%%%%%%%%%%%%%%%%%%%%%%%%%%%%%%%%%%%%%%%%
%\modHeadchapter{}
%(to be written...)



%%%%%%%%%%%%%%%%%%%%%%%%%%%%%%%%%%%%%%%%%%%%%%%%%%%%%%%%%%
%% section 1.1 %%%%%%%%%%%%%%%%%%%%%%%%%%%%%%%%%%%%%%%%%%%
%%%%%%%%%%%%%%%%%%%%%%%%%%%%%%%%%%%%%%%%%%%%%%%%%%%%%%%%%%
%\modHeadsection{}
%(to be written...)



